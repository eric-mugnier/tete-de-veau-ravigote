\documentclass[tikz, border=0pt]{standalone}

% ─── PACKAGES ─────────────────────────────────────────────────────────────────
\usepackage[utf8]{inputenc}
\usepackage[T1]{fontenc}
\usepackage{ebgaramond}
\usepackage[french]{babel}
\usepackage{xcolor}
\usepackage{tikz}
\usetikzlibrary{calc}
\usepackage{graphicx}
\usepackage{microtype}

% ─── COULEURS (relevées sur la couverture Fayard originale) ───────────────────
\definecolor{creme}{RGB}{246,237,210}      % fond crème chaud
\definecolor{brun}{RGB}{106,60,22}         % brun titre/auteur

% ─── DOCUMENT ─────────────────────────────────────────────────────────────────
\begin{document}

% ─── CANVAS TIKZ (unité = 1mm, origine bas-gauche) ───────────────────────────
% Zones :
%   Fond perdu gauche   :  0 –   3 mm
%   4e de couverture    :  3 – 143 mm  (140 mm)
%   Dos                 : 143 – 161 mm ( 18 mm)
%   1ère de couverture  : 161 – 301 mm (140 mm)
%   Fond perdu droit    : 301 – 304 mm
%   Fond perdu bas/haut :  0 –   3 mm / 208 – 211 mm
%
% Photo (1ère de couv) : bas de la couv, y = 3 mm → 92 mm (89 mm de haut)
% Crème (1ère de couv) : y = 92 mm → 208 mm (116 mm de haut)

\begin{tikzpicture}[x=1mm, y=1mm, font=\normalfont]

  % ── FOND CRÈME GLOBAL ──────────────────────────────────────────────────────
  \fill[creme] (0,0) rectangle (304,211);

  % ─────────────────────────────────────────────────────────────────────────
  % ══ 1ÈRE DE COUVERTURE (161 – 301 mm) ═══════════════════════════════════
  % ─────────────────────────────────────────────────────────────────────────

  % ── Photo (partie basse, 89 mm de hauteur) ─────────────────────────────
  \begin{scope}
    \clip (161,3) rectangle (301,92);
    \node[anchor=south west, inner sep=0] at (161,3)
      {\includegraphics[width=140mm, height=89mm]
        {images/tc3aate-de-veau-roulc3a9e-via-produits-tripiers.jpg}};
  \end{scope}

  % ── Titre (grand, brun, centré) ────────────────────────────────────────
  \node[anchor=north, text=brun] at (231, 175)
    {\fontsize{24}{29}\selectfont Tête de veau ravigote};

  % ── Auteur (brun, plus petit, centré sous le titre) ───────────────────
  \node[anchor=north, text=brun] at (231, 154)
    {\fontsize{15}{18}\selectfont Éric Mugnier};

  % ── Logo CT (entre auteur et photo) ──────────────────────────────────────
  \node[anchor=center] at (231, 112)
    {\includegraphics[width=14mm]{images/ct_watermark.png}};


  % ─────────────────────────────────────────────────────────────────────────
  % ══ DOS (143 – 161 mm) ═══════════════════════════════════════════════════
  % ─────────────────────────────────────────────────────────────────────────

  % Titre vertical (bas → haut)
  \node[rotate=90, anchor=center, text=brun] at (152, 112)
    {\fontsize{8}{10}\selectfont\itshape Tête de veau ravigote};

  % Auteur en haut
  \node[anchor=north, text=brun] at (152, 194)
    {\fontsize{6.5}{8}\selectfont Éric Mugnier};

  % ── Logo CT (dos, bas, discret) ───────────────────────────────────────────
  \node[anchor=south] at (152, 6)
    {\includegraphics[width=6mm]{images/ct_watermark.png}};


  % ─────────────────────────────────────────────────────────────────────────
  % ══ 4ÈME DE COUVERTURE (3 – 143 mm) ═════════════════════════════════════
  % ─────────────────────────────────────────────────────────────────────────

  % ── Résumé
  \node[
    anchor=north west,
    text width=113mm,
    align=justify,
    text=brun,
    inner sep=0pt
  ] at (15, 191)
  {%
    \fontsize{9}{13.5}\selectfont
    Quand le père Vidal disparaît dans des circonstances troubles et que des
    colis macabres commencent à circuler dans la ville, le commandant Beauvais
    se lance dans une enquête labyrinthique. Mais le polar n'est ici qu'un fil
    rouge dérisoire : ce qui compte, ce sont les digressions – sur les chats,
    les dinosaures, l'Église, les pavillons de banlieue, la décadence de
    l'humanité.

    \medskip
    Roman-fleuve picaresque dans la lignée de Cervantes et Rabelais,
    \textit{Tête de veau ravigote} mêle la rage sadienne au pessimisme
    schopenhauerien. Éric Mugnier y déploie une prose incandescente, truculente
    et désespérée, qui rappelle autant \textit{Là-bas} de Huysmans que les
    pamphlets du XVIIIe~siècle. Inachevable, et délibérément inachevé, ce
    premier roman ignore la résolution et préfère la dérive : une logorrhée
    vengeresse et jubilatoire.%
  };

  % Filet séparateur
  \draw[brun!40, line width=0.3pt] (15, 41) -- (131, 41);

  % ── Bio
  \node[
    anchor=north west,
    text width=113mm,
    align=justify,
    text=brun,
    inner sep=0pt
  ] at (15, 38)
  {%
    \fontsize{8}{11}\selectfont
    Éric Mugnier est né en 1960. Il vit à Chaumont-en-Bassigny. Peintre,
    compositeur et parolier, il se consacre depuis quarante ans à la
    création. \textit{Tête de veau ravigote} est son premier roman.%
  };

\end{tikzpicture}

\end{document}
