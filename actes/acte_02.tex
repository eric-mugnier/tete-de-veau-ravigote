
\noindent Ça a officiellement commencé le jour où Rose Barbet, née Lortie, 75 ans, a reçu un putain de colis. Un colis assez lourd et volumineux. Quelques instants après la livraison, alors qu’elle avait encore le colis dans les mains, Thierry, son mari, est rentré avec une baguette sous le bras. Chaque jour, plus ou moins à la même heure, il rentrait avec une baguette sous le bras, la baguette du pain quotidien que sa femme et lui partageait religieusement depuis bientôt quarante ans (même si c’est lui qui en mangeait les trois quarts). Un record de longévité qui aurait pu leur valoir une citation à l’Ordre national du Mérite conjugal, ou, pour Thierry, à comparaître devant la justice pour entreprise de destruction à long terme d’une personne psychologiquement vulnérable, Rose ayant le profil type de l’épouse soumise et endurante, corvéable à merci, prête à tous les sacrifices pour assurer la survie de son ménage.

Donc, comme je le disais, Thierry est rentré avec sa baguette sous le bras, et, après avoir jeté un coup d’œil déjà légèrement embué par l’alcool (il avait profité de sa sortie pour s’enfiler quelques verres avec ses potes, rituel auquel il ne dérogeait que dans les cas de force majeure, et potes qu’il retrouvait en fin d’après-midi pour s’enfiler à nouveau quelques verres, nettement plus nombreux que ceux qu’il avait ingurgités dans la matinée, sachant qu’il avait entretemps avalé une bonne demi-bouteille de vin pendant le repas de midi, bouteille qu’il finirait le soir avant de s’endormir devant la télé) sur sa femme et surtout le colis, il a posé la question suivante : Qu’est-ce que c’est que ça ?

Rose a répondu : Je sais pas.

\textsc{Lui} : T’as commandé quelque chose ?

C’était assez peu probable dans la mesure où Rose n’avait jamais rien commandé de sa vie, et qu’on la voyait assez mal, à 75 piges, se mettre à commander des trucs sur Amazon (dont elle n’avait d’ailleurs jamais entendu parler, pas plus qu’elle ne savait se servir d’un ordinateur ou un smartphone). Non seulement elle n’avait jamais rien commandé, mais personne ne lui avait jamais rien envoyé de plus conséquent qu’une lettre ou une carte postale, lesquelles lettres provenaient la plupart du temps d’organisme de recouvrement de créances diverses, et rarissimes cartes postales des quelques résidus familiaux dont elle pouvait encore se prévaloir ici et là.

Elle a dit : Non.

Il a posé sa baguette sur la table de la cuisine et dit : Donne-moi ça.

Elle lui a donné le paquet, il a constaté que ses nom et adresse figuraient bien sur l’étiquette, puis, après avoir examiné l’objet sous toutes ses coutures, il s’est dit qu’il y avait peu de chances qu’il s’agisse d’une bombe (qui se donnerait la peine de faire sauter sa femme et un crétin comme lui ?) et s’est résolu à l’ouvrir. À noter que le paquet ne faisait aucune mention d’un quelconque expéditeur.

Je vous fais grâce des détails et autres péripéties concernant l’ouverture de ce colis, sachez seulement que l’expéditeur en question n’avait pas lésiné sur les moyens pour en garantir l’intégrité.

Quelques minutes plus tard, le temps que les époux Barbet retrouvent leurs esprits (surtout Rose, qui avait fané d’un coup en découvrant le pot aux roses — c’est d’un goût douteux, je vous l’accorde, mais il faut bien décompresser un peu quand les événements se précipitent), les forces de l’ordre, votre serviteur en tête, débarquaient dans la place toutes sirènes hurlantes. Le grand jeu à l’américaine, avec crissements de pneus et portières qui claquent. Il y avait aussi, histoire de gonfler un peu le niveau sonore, les services d’urgence pour ranimer Rose qui gisait sans connaissance dans le canapé à fleurs du salon.

Thierry et le colis étaient là aussi, bien sûr, plus quelques voisins attirés par le vacarme qu’il a fallu faire dégager à coups de pompe dans le cul.

Thierry, pour se remettre de ses émotions, était en train de siffler une flasque de cognac bon marché.

Il faut dire que le contenu du colis n’était pas des plus ordinaires.

Pour faire court, il s’agissait de ce qu’on appelle communément une tête, humaine en l’occurrence.

Naturellement, lorsqu’on trouve une tête (humaine surtout, mais on se poserait aussi des questions s’il s’agissait d’une tête de chien ou de mouton) dans un colis, la première chose à faire est de savoir à qui elle appartient, ou plutôt appartenait, car son propriétaire n’est plus en état de la réclamer.

Il faut aussi se demander comment et pourquoi cette tête a atterri dans ce colis, et accessoirement, quand ce colis a été réceptionné par quelqu’un, qui le lui a adressé et pourquoi à lui en particulier.

Telles sont les interrogations auxquelles l’enquêteur se voit rapidement confronté.

Pour mener à bien sa mission, il dispose d’un certain nombre de ressources scientifiques, parmi lesquelles le précieux concours d’un médecin légiste, étrange personnage qui, à défaut de soigner les gens, tente de comprendre pourquoi ils sont morts.

Voilà comment la mystérieuse tête s’est retrouvée sous le regard expert du docteur Zaahid Shirani, personnage complexe aux yeux de braise pour qui la mort n’avait aucun secret. Ses grands-parents, des intellectuels hindous, avaient fui le Bengale après la partition de 47, pour se réfugier à Londres d’abord, après un parcours riche en rebondissements, puis en France.

Si l’on s’en tient aux connaissances actuelles de la communauté scientifique en matière de vie et de mort, un être humain quel qu’il soit n’a que très peu de chances de survivre à l’ablation de sa tête. Mais il peut aussi être mort bien avant qu’on la lui coupe, chose que Shirani devrait s’employer à déterminer.

La réponse n’a pas tardé à tomber : oui, il était mort bien avant qu’on la lui coupe.

Et comment était-il mort ?

Il serait, en l’absence de corps, difficile de répondre à cette question.

Par contre, la tête présentait quelques particularités qui pouvaient se révéler intéressantes dans le cadre d’une enquête policière.

J’ai ouvert la porte du Frigo et une forte odeur de shit m’a aussitôt sauté au visage.

Le Doc se trouvait un peu plus loin, pétard au bec, en train de faire joujou avec la tête.

J’ai dit : Ça va, Doc ?

Il a tourné vers moi son beau visage buriné par les embruns du Pacifique Sud (cet enculé passait toutes ses vacances à Bora-Bora, à faire du bateau et bouffer de la langouste) et répondu : On fait aller, inspecteur.

Il a sorti le pétard de son bec et l’a tendu dans ma direction : Vous en voulez ?

\textsc{Moi} : Jamais ! Ça me file des migraines et des boutons sur la gueule.

\textsc{Lui} : Vous avez tort, c’est du tout premier choix. Je le fais venir tout spécialement des fertiles vallées du Cachemire. Une goutte de Feni, peut-être ?

À chacune de mes visites, il me proposait un verre de Feni, un alcool de cajou très populaire dans son pays de sauvages, et ma réponse était toujours la même : Non merci.

J’avais commis l’erreur, par charité chrétienne, de tremper une fois mes lèvres dans cette abomination liquide, et je vous prie de croire que l’expérience m’avait servi de leçon.

Je lui ai demandé s’il avait du nouveau concernant la tête que je lui avais confiée.

Il a répondu, avec la moue du type qui n’a pas pour habitude de se casser le cul pour des prunes et pour qui avoir du nouveau est d’une telle banalité qu’il ne convient même plus d’en faire état : Approchez, je vais vous montrer.

Il m’a alors expliqué que la boîte crânienne en question (celle d’un homme qu’une quarantaine d’années pourvu d’une dentition exécrable) avait été méticuleusement décalottée, un peu à la façon d’un œuf à la coque, et entièrement vidée de son contenu. Pourquoi faire me direz-vous ? Eh bien tout simplement pour remplacer ledit contenu par un autre, à savoir des croquettes pour chien à l’agneau et au riz dont la marque restait à déterminer. Après quoi la partie supérieure du crâne avait été soigneusement remise en place et hermétiquement soudée, de sorte qu’il fallait vraiment se pencher de près sur la question pour déceler quoi que ce soit.

J’ai sorti un Hemingway Short Story que je me suis aussitôt inséré entre les dents, avant de l’allumer à la flamme de mon briquet (c’est un perfecto très facile à allumer), et j’ai demandé au Doc : Une idée, Doc ?

\textsc{Le Doc} : Je suis légiste, pas psychiatre.

\textsc{Moi} : Vous pensez que seul un dingue peut avoir fait une chose pareille ?

\textsc{Lui} : Non, pas forcément. C’est peut-être juste un type qui aime les chiens et a le sens de l’humour. Est-ce qu’il faut être dingue pour tuer quelqu’un ? Bien sûr que non. Beaucoup de gens ont envie de le faire et certains ne parviennent pas toujours à se retenir. Est-ce qu’il faut être dingue pour farcir le crâne de quelqu’un avec des croquettes pour chien à l’agneau et au riz ? J’ai envie de dire non, pas davantage.

Votre humble serviteur, nettement plus détendu après avoir tiré quelques bouffées de cigare : Soit. Mais vous ne pensez pas qu’il faut être dingue pour envoyer une tête au premier venu ? Les Barbet sont des gens sans histoire, totalement insignifiants, qui ont été manifestement choisis au hasard.

\textsc{Le Doc} : C’est assez drôle, non ?

\textsc{Moi} : Vous trouvez ?

\textsc{Lui} : Tant qu’à faire de s’amuser à farcir des têtes avec des croquettes pour chien, autant les envoyer au premier venu. C’est une question de logique. Une logique un peu particulière, certes, mais une logique quand même. Dites donc, mon vieux, je pense que je vais aller faire un tour rue de la Cloche, demain soir. Vous m’accompagnez ?

\textsc{Moi} : Une autre fois, Doc. Je n’ai vraiment pas la tête à ça.

Pour info, parce que je suppose que vous brûlez d’en savoir plus, je peux vous confier qu’il se trouve rue de la Cloche un endroit connu des initiés sous le nom de Narcisse Rose. Cet endroit, admirable s’il en est, n’est ni plus ni moins que ce qu’il faut bien appeler un lupanar. Mais attention, pas n’importe lequel. C’est un endroit à l’ancienne, très classe, dans le plus pur respect des traditions de la Troisième République (temps béni des maisons closes, avant que Marthe Richard\nf{Marthe Richard, née Marthe Betenfeld (1889--1982), aventurière, espionne et femme politique française. Ancienne prostituée et espionne pendant la Première Guerre mondiale, conseillère municipale de Paris en 1945, elle déposa l’amendement qui entraîna la fermeture des maisons closes en France par la loi du 13 avril 1946. Sa biographie recèle de nombreuses affabulations. \source{fr.wikipedia.org/wiki/Marthe\_Richard}}, péripatéticienne repentie et mythomane notoire, plus pathétique qu’aristotélicienne, ne réclame officiellement leur fermeture en 45 et ne l’obtienne un an plus tard), d’un goût très sûr, un peu orientalisant, fréquenté uniquement, tant au niveau de la clientèle que du personnel, par des gens de toute première qualité, en gros de la pute fermière élevée au grain et du gentleman premier choix.

Je ne suis pas très porté sur la chose, mais il m’arrive, dans mes moments de déprime les plus intolérables, quand j’ai lu tous les livres et fumé tous les cigares, d’aller trouver refuge dans les bras d’une de ces filles qui font profession du bien-être de leurs concitoyens. Mais bon, ça va, je ne vais pas non plus entrer dans les détails de ma vie privée, ni m’excuser parce qu’il m’arrive parfois de me sentir seul au point de rechercher la compagnie d’une personne physiquement attractive et affectivement neutre. Un peu de fraîcheur, même abondamment souillée par ces pratiques impies que la nature inflige à ses subordonnés, ne fait pas de mal de temps à autre.

Zaahid Shirani avait parlé de logique, et la logique voulait que la plaisanterie ne s’arrête pas en si bon chemin.

C’est maintenant que je vais être obligé de vous parler, même si je n’en ai pas la moindre envie, du dénommé Dylan Passereau, chauffeur routier de son état, grossier personnage au physique assez peu avantageux et aux capacités intellectuelles proches de l’inexistence, ce qui n’est bien évidemment, je m’empresse de le préciser, pas le cas de tous les chauffeurs routiers, même si on ne peut décemment nier qu’un certain nombre d’entre eux s’inscrivent pleinement dans cette définition. Car c’est bien lui, Dylan Passereau, homme de route, qui avait, pas plus tard que le lendemain, hérité d’un second colis, renfermant cette fois les organes génitaux d’une personne de sexe masculin conservés dans un bocal de liquide incolore qui s’est avéré plus tard n’être ni plus ni moins que de l’alcool ménager.

Il avait aussitôt appelé les forces de l’ordre, et comme il était le second, dans un même périmètre (il résidait à moins de cinq cents mètres de chez les Barbet), à recevoir un colis de ce genre, on pouvait facilement imaginer que le service trois pièces dans le formol et la tête farcie de croquettes pour chien appartenaient au même individu.

Les analyses ont confirmé que c’était le cas.

Les flics, à commencer par moi, ont enquêté, et, à force de recoupements, sont arrivés à cerner deux ou trois suspects aussitôt placés sous surveillance. D’autre part, on avait appris que le père Clément Vidal, 45 ans, curé de l’église Notre Dame du Perpétuel Secours, avait disparu sans laisser de trace quelque temps avant que la vague de colis commence à déferler sur la ville.

Les analyses ont également confirmé que la tête et les organes génitaux, plus quelques menus articles réceptionnés ici et là par des habitants ulcérés de la ville, appartenaient bel et bien au père Vidal. Au rayon des dommages collatéraux, il est à noter que le pied droit de l’intéressé avait causé la mort de Valérie Renou, retraitée de la fonction publique dont le cœur d’artichaut n’avait pas supporté la vision d’horreur qui lui était imposée.

Solange Jacquard était ce qu’on appelle une personne très pieuse. Piteuse, aussi, car, à bientôt quatre-vingt-dix ans, elle était dans un état de conservation pour le moins discutable.

Solange allait à la messe tous les dimanches, bien sûr, et hurlait les cantiques avec une voix de fausset qui faisait se lézarder les murs de la petite église où elle avait ses habitudes (et les tympans de ses voisins, lesquels s’étaient progressivement éloignés, de sorte qu’elle se retrouvait maintenant seule dans le périmètre déclaré zone sinistrée par le diocèse). Elle avait toujours été très laide, ce qui n’avait pas facilité son rapprochement avec les humains, mais n’était pas rancunière pour un sou, ce qui lui avait permis d’entretenir d’excellentes relations avec Dieu, lequel lui devait bien ça compte tenu de l’existence misérable qu’il l’avait contrainte à mener depuis le jour de sa naissance.

Tous les troisièmes jeudis du mois, la vieille taupe allait se confesser. Allez savoir pourquoi, elle était quasi quotidiennement la victime éplorée de vilaines pensées à caractère sexuel qui pesaient lourdement sur sa conscience passablement délabrée. Oui, ça peut sembler bizarre à quatre-vingt-dix ans, mais il n’était pas rare que son regard vitreux s’attarde exagérément sur la poitrine débordante ou la croupe avenante de telle ou telle frétillante jeune fille. Attirée par les personnes de son sexe depuis sa plus tendre enfance, cette conne n’avait jamais trouvé le moyen de peloter une belle paire de miches ou bouffer une chatte bien juteuse à pleines dents, d’où une certaine aigreur qui suintait par tous les pores de sa peau de vieille fille desséchée. Il lui arrivait aussi, pendant ses crises de délirium nocturnes, telle une vierge en proie aux assauts du démon, de se fourrer des cierges et des courgettes dans le fion.

Heureusement pour elle, tous les troisièmes jeudis du mois, le père Beaubois l’attendait à dix-sept heures tapantes pour écouter religieusement ses inepties et redonner à son âme putréfiée un semblant de fraîcheur divine.

Ce jeudi-là, donc, comme à son habitude, Solange Jacquard, plus laide et aigrie que jamais, la chatte épilée de frais car elle ne supportait pas d’avoir un rat d’égout dans la culotte (et surtout elle aimait sentir le contact du tissu sur ses chairs), s’est pointée à son rendez-vous, le souffle court et le pied creux. Elle avait hâte de lui annoncer que de nouveaux locataires, les Brochard (elle était allée renifler le nom sur la boîte aux lettres), venaient d’emménager en face de chez elle.

Laurent et Frédérique Brochard, la cinquantaine, avaient deux enfants, un garçon et une fille. Le garçon devait avoir dans les treize-quatorze ans, la fille pas loin de dix-sept ou dix-huit. C’était cette dernière, d’une beauté assez dommageable collatéralement, qui avait retenu l’attention de Solange, au point qu’elle qui ne dormait déjà pas beaucoup passait maintenant ses nuits à s’astiquer le bouton. Les vilaines pensées voletaient autour d’elle comme un essaim de mouches à merde, se posant avec insistance sur les points les plus sensibles de sa personne, ne lui laissant aucun répit. À ce stade de décomposition morale, seul le père Beaubois pouvait encore son âme des flammes de l’enfer.

Solange connaissait le chemin. Après quelques rapides signes de croix en direction de l’autel et des autorités religieuses statufiées ici et là, elle est allée droit sur le confessionnal où l’attendait le père Beaubois.

Elle s’est agenouillée à l’endroit réservé aux pénitents, a débité les formules d’usage, et attendu que le père Beaubois l’autorise à vider son sac. Mais le père Beaubois, dont elle devinait le visage à travers la grille, est resté de marbre. Peut-être s’était-il endormi à force d’attendre. Elle a répété les formules d’une voix plus forte, mais le père Beaubois n’a pas esquissé l’ombre d’un mouvement ni daigné ouvrir la bouche pour l’autoriser à poursuivre. Peut-être qu’il ne dormait pas mais qu’il était mort. Après tout, on peut faire une crise cardiaque n’importe où. Peut-être qu’il avait reçu le diable en personne en confession et que son cœur avait lâché, comme celui du père Merrin dans L’Exorciste, quand cette saleté de Pazuzu l’insulte copieusement et traite sa mère de pute.

Après avoir beuglé en vain pour attirer son attention, elle s’est résolue à s’extraire de son logement pour aller s’assurer que le père Beaubois était encore de ce monde.

D’une main tremblante, elle a ouvert la porte centrale et aussitôt poussé un long hurlement qui s’est répercuté sans fin sous les arches de l’église, heureusement déserte à cette heure-ci.

Le père Beaubois était bien là, assis à sa place, mais on lui avait ouvert le ventre, et tout ce qui se trouvait à l’intérieur se trouvait maintenant dans un seau en plastique positionné entre ses jambes. Solange Jacquard, qui n’avait pas gerbé depuis ses dix-sept ans, année de sa première et dernière cuite, a aussitôt recraché les trois parts de tarte aux pommes qu’elle avait ingurgitées avant de venir, lesquelles parts de tarte à moitié digérées ont atterri dans le seau du père Beaubois, avec les restes de blanquette de son repas de midi.

Je suis arrivé une demi-heure plus tard, en compagnie de Titus Beaugendre, mon plus fidèle lieutenant, et d’une escouade de la police scientifique. Il s’est avéré qu’en plus de ses pieds et mains, le père Beaubois avait été délesté de ses organes génitaux (lesquels ne lui servaient en principe pas à grand-chose, mais on sait aujourd’hui que le vœu de chasteté est loin d’être unanimement respecté par les membres du clergé, et ce quelle que soit leur taille). Ensuite, son crâne d’œuf avait été ouvert avec la même précision chirurgicale que celui du père Vidal, également vidé de sa matière cérébrale, puis farci avec des croquettes pour chien dont il y avait tout lieu de penser qu’il s’agissait de croquettes à l’agneau et au riz de la marque Waterflox, uniquement disponible en boutique de luxe pour animaux (le genre de boutique où on trouve aussi des casquettes en velours avec des trous pour les oreilles et des petits blousons en cuir pour chien). Enfin, la calotte prélevée sur le crâne lors de son ouverture, d’un diamètre sensiblement équivalent à celle qui recouvrait jadis la tonsure des clercs (sans doute pour qu’ils n’attrapent pas un rhume de cerveau en déambulant dans les travées glaciales des édifices religieux), avait été soigneusement remise à sa place et la cicatrice maquillée avec une telle habileté qu’elle était presque impossible à distinguer.

La vieille bique tenait tellement à rester auprès de son cher curé qu’on a dû l’expulser à coups de pompe dans le cul. Déjà qu’elle avait foutu de la tarte plein la blanquette, on n’allait pas la laisser continuer à saboter la scène de crime. Après quoi, alors que j’avais instamment donné l’ordre de ne laisser entrer personne, pas même le Pape, une espèce de prélat, genre nonce, évêque, cardinal ou Dieu sait quoi, s’est pointé, avec sa calotte mauve sur le crâne, ses narines pleines de poils, ses oreilles décollées et ses yeux exorbités derrière d’épaisses lunettes à monture d’écaille, et a commencé à gesticuler et couiner comme un beau diable en disant qu’il tenait à être informé de tout et tenu au courant heure par heure du déroulement de l’enquête. Naturellement, il n’était pas question que la presse entende parler de cette regrettable affaire, sans doute l’œuvre d’un fou échappé de l’hôpital psychiatrique le plus proche. On l’a foutu dehors aussi, même s’il a été un peu difficile à manœuvrer que la vieille gouine, jurant qu’il se plaindrait de nos méthodes en haut lieu. Titus, homme de conviction mais anticlérical primaire capable de réactions d’une rare violence en présence d’une soutane, tenait absolument à lui défoncer la gueule, de sorte qu’il m’a fallu user de toute ma force de persuasion pour le ramener à de meilleurs sentiments. Une fois le calme revenu, on a pu se remettre à bosser dans de bonnes conditions, même si un public nombreux, alerté par le va-et-vient incessant des sirènes et gyrophares, commençait à se masser de façon inquiétante aux portes de l’édifice.

Certaines personnes portent bien leur nom, d’autres pas.

Dylan Passereau, par exemple, portait assez mal le sien.

Vous vous rappelez de lui ? C’est le routier qui avait reçu la bite et les couilles du père Vidal dans un bocal.

Alors que le passereau est généralement un petit animal frais et léger qui virevolte de branche en branche en sifflant des airs entraînants, Dylan était quant à lui une sorte de monstre préhistorique haut de deux bons mètres et taillé à la serpe dans un fût de séquoia, pourvu de surcroît d’une voix caverneuse qui renvoyait aux premiers âges de l’humanité. Plus proche du gorille des montagnes que de l’oiseau-mouche, il émanait de sa personne une forte odeur de tripes à la mode de Caen. Ses épaules étaient si larges qu’il explosait systématiquement toutes les vestes qu’il tentait d’enfiler. Allez savoir pourquoi, sans doute parce qu’il me faisait l’effet d’un pithécanthrope capable des abominations les plus régressives, je n’avais jamais réellement cru à son histoire de bocal de couilles. Il m’était instantanément venu à l’esprit qu’il avait très bien pu se les envoyer à lui-même, histoire de narguer la police en se faisant passer pour une victime du Brain Catcher, surnom que j’avais trouvé pour notre tueur de curés en raison de sa manie de prélever la cervelle de ses victimes. Sans me vanter, je passais pour avoir le nez assez creux lorsqu’il s’agissait de démasquer les imposteurs, et Passereau me semblait avoir le profil idéal de l’enculé de service.

Naturellement, je ne m’étais ouvert de mes doutes à personne, tenant à m’occuper moi-même de Passereau au cas où j’arriverais à établir sa culpabilité, ce dont je ne doutais guère. Même Titus, qui me filait régulièrement des coups de main pour expurger la planète de ses abcès les plus purulents, n’était pas au courant. Avec lui, Maël Robineau, Samuel Girard et Grégoire Lussier, j’avais fondé une petite entreprise de nettoyage spécialisée dans le tri sélectif des ordures. Les heureux élus faisaient l’objet d’un traitement de faveur dont les modalités variaient en fonction de leurs méfaits, mais qui se terminait invariablement par leur mise en pièces et la dispersion de celles-ci dans les égouts de l’histoire de l’humanité. Nous formions une petite équipe très efficace, mais chacun de nous se réservait le droit de chasser en solitaire. C’est très agréable de planifier, monter minutieusement une opération et la voir se dérouler sans accroc, mais il arrive parfois qu’un lièvre surgisse à l’improviste, vous passe entre les jambes, et il serait dommage de le laisser filer. Le côté monstrueux de Passereau, inhabituel et digne des films d’horreur les plus réfrigérants, stimulait mon instinct de limier, et j’avais décidé qu’il serait pour moi et moi seul.

J’étais certain, à 99,99 pour 100, qu’il était le Brain Catcher.

Par chance il travaillait à l’international, avec comme destinations principales l’Allemagne, la Belgique et le Luxembourg, ce qui fait que lorsqu’il partait on pouvait être trois ou quatre jours, voire davantage, sans voir sa sale gueule de troll des cavernes dans les parages. Figurez-vous, preuve que la nature fait vraiment n’importe quoi, qu’il avait quand même réussi à trouver une femme et lui faire un enfant, une fille en l’occurrence. Naturellement, la femme en question avait rapidement pris conscience de l’erreur monumentale qu’elle avait faite en s’accouplant avec ce dégénéré, et s’était empressée de prendre la fuite avec sa fille sous le bras. C’était une chose dont je ne pouvais que me féliciter, d’abord pour elles parce qu’elles avaient sans doute échappé au pire, ensuite pour moi car le fait qu’il vive seul facilitait grandement mon travail.

J’ai profité de son absence pour aller faire un tour chez lui, de nuit bien sûr, parce que la nuit tous les chats sont gris et que je suis moi-même un gros matou affichant un certain penchant pour la bouteille. La preuve, c’est que juste avant cette visite je m’étais envoyé les trois quarts d’une bouteille de Gevrey-Chambertin signé Clovis Cuvillier \& Fils à Morey, lequel divin nectar m’avait placé dans l’exact état d’insouciance recherché pour mener à bien cette mission, à risque modéré il faut bien le dire. Inutile de préciser que bien au chaud dans le fond de ma poche, mon fidèle Manu était prêt à pointer le bout de son canon et cracher le plomb à la moindre alerte.

J’ai utilisé une clé passe-partout (achetée pas cher sur Internet, j’en avais plusieurs qui me permettaient de venir à bout de la plupart des serrures, hormis peut-être quelques rares modèles de fabrication tchécoslovaque datant du rideau de fer et de la cession de la Ruthénie subcarpatique à l’URSS en 45) pour m’introduire sans dommage dans le nid d’amour de mon routier préféré. Le Brain Catcher, si c’était bien lui, ne devait pas savoir que quelqu’un avait violé son intimité pendant son absence. Il devait continuer à mener sa petite vie bien tranquille sans se douter de rien, jusqu’au jour où je lui tomberais dessus sans crier gare et lui ferais regretter le jour où son père et sa mère s’étaient rencontrés avant de forniquer pour donner naissance à une des pires abominations jamais conçues par des voies naturelles. L’idée peut sans doute choquer quelques intellectuels de gauche, universitaires affiliés à la LCR de Gérard Filoche et autres membres du Nouveau Parti Anticapitaliste farouchement attachés au respect des droits de l’homme et autres valeurs aussi obsolètes que la liberté, l’égalité et la fraternité, toutes choses avec lesquelles la populace se récure copieusement le fondement, mais il semble évident que certaines personnes devraient être interdites de reproduction (et dans le cas où ils seraient arrêtés après avoir mis bas dans quelque endroit tenu secret, éliminés sans autre forme de procès en même temps que leur immonde progéniture).

La bicoque, située dans un des quartiers les plus pourris de la ville (il suffisait de prononcer son nom pour que les gens se mettent à claquer des dents et transpirer à grosses gouttes), dégageait la même odeur de tripes à la mode de Caen que son propriétaire. Notre homme n’avait pas de chien, ou plutôt n’en avait plus, car même les animaux ne supportaient pas sa présence et se carapataient à la première occasion. Peu de temps après le départ de sa femme et sa fille, Pancho, un Ratier de Majorque dans la force de l’âge, d’humeur habituellement enjouée, avait disparu à son tour. Outre des croquettes Waterflox, croquettes de luxe pour chien gâté, je cherchais une planche à découper spéciale être humain avec les outils appropriés, des congélateurs pleins à craquer, plus des murs entiers d’étagères surchargées de conserves comme on en trouve chez les péquenauds qui passent leur temps à cueillir des haricots, des patates et des citrouilles, zigouiller des lapins, des poules et des cochons, quand ils ne sont pas dans la forêt en train de défourailler sur tout ce qui bouge au calibre 12.

Je n’ai pas trouvé de croquettes Waterflox, ni de congélos pleins à craquer (juste un réfrigérateur rempli de canettes de bière bas de gamme et de bouffe avariée), jusqu’au moment où je me suis retrouvé face à une minuscule porte planquée derrière une tenture bouffée aux mites à l’effigie de Ganesh. Figurez-vous que le dieu de la sagesse, seigneur des catégories, principe du nombre et protecteur des moissons, était en train de danser ou sautiller sur une espèce de pouf en agitant sa trompe et ses deux paires de bras potelés autour de son corps grassouillet, ses principaux attributs dans les mains. Au moins, en Inde, les dieux n’ont pas peur du ridicule, ce qui vaut d’ailleurs pour les Indiens en général, qui adorent les vaches maigres, les bus bondés avec des gens sur le toit, les couleurs criardes, les comédies musicales ineptes et se baigner tout habillés dans les eaux croupies du Gange.

Cette porte, qui menaçait de tomber en poussière si on s’avisait de la brusquer, n’était autre que celle de la cave, endroit qui dégageait une forte odeur de moisi, fosse septique et rat crevé, cocktail qui ne donnait pas franchement envie de s’aventurer plus loin. Surtout que pour descendre, il fallait emprunter un escalier en bois dont chaque marche semblait sur le point de s’effondrer quand on posait le pied dessus. Pour couronner le tout, un interrupteur, assez vieux pour figurer en tête de gondole dans le musée international des interrupteurs, était relié à une ampoule crasseuse pendant nue au bout d’un fil décharné, laquelle émettait juste assez de lumière pour ne pas se marcher dessus dans la pénombre.

J’ai dû mettre en marche la torche de mon téléphone pour me frayer un chemin dans ce trou à rats. Après avoir fureté dans tous les coins et recoins, chose qui ne m’a permis de dénicher que quelques insectes rampants et autres araignées manifestement peu satisfaits de se voir dérangés dans leurs activités sournoises, je me suis vu dans l’obligation de conclure à l’improductivité de ma démarche. Il n’y avait rien dans cette cave susceptible d’incriminer mon principal suspect, ce qui le faisait passer de son statut initial à individu lambda, certes particulièrement moche, vulgaire et inquiétant, mais apparemment indemne des faits ignobles que je brûlais d’envie de coller sur son dos d’une largeur impressionnante. Rien, dans ce que j’avais vu ici, ne pouvait laisser penser que Dylan Passereau était le Brain Catcher, même s’il avait le profil idéal pour ce genre de cas de figure.

Je suis rentré chez moi la queue entre les jambes, au volant de ma Renault Kangoo de dix ans d’âge, ce qui je vous le concède n’est pas spécialement une bagnole de super flic, avec la ferme intention de pousser plus avant mes investigations sur les étranges cas des pères Vidal et Beaubois. Ce n’était pas tous les jours qu’on retrouvait des hommes d’Église en pièces détachées dans des bocaux (qu’on aurait pu étiqueter, par exemple, «~Les Conserves du Vatican~», et distribuer gratuitement aux fidèles lors des grandes occasions).

J’ai retrouvé ma bouteille de Gevrey aux trois quarts vide qui m’attendait bien sagement sur la table de la cuisine, et en passant près d’elle je l’ai clairement entendue me chuchoter dans le creux de l’oreille, d’une voix déchirante qui n’aurait pu laisser personne indifférent, moi encore moins compte tenu de la passion que je nourrissais pour les vins de Bourgogne : finis-moi avant d’aller te coucher. Cette proposition tombait d’autant mieux que j’avais bien besoin d’un petit remontant. J’ai donc sifflé le verre de vin en question et, histoire de digérer un peu mieux la déconvenue que je venais d’encaisser, me suis servi un petit Cognac dans la foulée, que j’ai siroté religieusement, le cul profondément enfoncé dans mon fauteuil club préféré, un Short Story au bec, en écoutant la Huitième de Chostakovitch\nf{Dmitri Chostakovitch (1906--1975), compositeur soviétique né à Saint-Pétersbourg. Sa Huitième Symphonie en ut mineur op. 65 (1943), créée en pleine Seconde Guerre mondiale, est une œuvre sombre et d’une tension extrême, souvent interprétée comme une méditation sur l’horreur du conflit. Elle fut un temps bannie par le régime stalinien. \source{fr.wikipedia.org/wiki/Dmitri\_Chostakovitch}}, œuvre majeure du maître de Saint-Pétersbourg dont la tension dramatique me paraissait idéalement adaptée aux turbulences qui m’agitaient intérieurement.

La question qui revenait sans cesse à mon esprit, ainsi qu’une hideuse ritournelle aux intonations grinçantes, n’était autre que celle-ci : pourquoi s’en prendre à des prêtres ?

Je doutais fort, même si je dispose d’un sens de l’humour à toute épreuve, qu’il s’agisse d’un exercice de style ou une simple déclaration d’anticléricalisme primaire. On sait aujourd’hui que les prêtres n’ont pas toujours le comportement irréprochable qu’on serait en droit d’attendre de leur sacerdoce. Nombre d’entre eux semblent avoir le plus grand mal à résister aux attraits de la chair. Non seulement beaucoup n’y résistent pas, mais il semblerait que certains s’y adonnent avec une coupable frénésie, fort peu en accord avec la rigueur morale qu’on attend d’eux. J’en veux pour preuve les récentes autant que stupéfiantes révélations concernant l’abbé Pierre\nf{Henri Grouès, dit l’abbé Pierre (1912--2007), prêtre catholique français, fondateur du mouvement Emmaüs (1949) et auteur de l’appel de l’hiver 1954 en faveur des sans-abri. À partir de 2024, de nombreuses victimes ont témoigné de violences sexuelles qu’il leur aurait infligées, des faits couverts pendant des décennies par son entourage et par l’Église. \source{fr.wikipedia.org/wiki/Abb\%C3\%A9\_Pierre}}, héros de la cause chrétienne que l’on croyait pourtant au dessus de tout soupçon. Même si la vie m’a appris qu’on ne peut raisonnablement se fier à personne, j’en suis resté comme deux ronds de flan. Derrière son sourire bonasse entouré de poils drus, Castor méditatif (son totem chez les scouts, assez bien trouvé quand on sait à quel point le castor travaille avec sa queue) dissimulait les noires pensées d’un des pires prédateurs sexuels de la chrétienté. Sous sa soutane frétillait un goupillon assez peu orthodoxe, une anguille sournoise qui n’attendait qu’une occasion de se glisser entre les cuisses des filles. À ses nombreuses décorations, Légion d’honneur, Croix de guerre et autres, je serais d’avis d’ajouter celle de grand officier de l’Ordre national des Violeurs patentés et autres Abuseurs sexuels compulsifs. Cela dit, je ne vais pas m’acharner sur lui ni sur sa hiérarchie qui, désireuse de se focaliser uniquement sur la bonté de son âme, a couvert ses agissements pendant des décennies. Finalement, et c’est tout à son honneur, sa lutte acharnée contre la misère incluait aussi la misère sexuelle, celle des prêtres en particulier. De même que pour nombre de nos personnalités les plus en vue, qu’il s’agisse de nos plus grands acteurs, hommes de science ou dirigeants, aujourd’hui poursuivies par des scandales à répétition dont on peine à connaître l’ampleur et la nature exactes, ne faut-il pas tolérer quelques petits égards de conduite eu égard aux services inestimables rendus à la Nation ? Le célibat des prêtres pose un problème de fond, nous le savons tous, car on ne voit pas très bien pourquoi il faudrait s’abstenir de se tripoter la nouille et tirer sa crampe sous prétexte de servir Dieu. D’un autre côté, il semble assez difficile de ne pas voir une certaine équivoque dans le fait de prêcher la retenue des sens d’une part, sinon l’abstinence pure et simple, et d’autre part abuser de ses prérogatives pour satisfaire outrageusement ses plus bas instincts. Il m’est souvent arrivé, en croisant quelque moine courbé sous le poids de sa charge méditative et soumis plus que tout autre au vœu de chasteté, de l’imaginer en train de s’adonner sans retenue aux plaisirs solitaires dans le secret de sa cellule, quitte à se flageller ensuite jusqu’au sang pour expier sa faute. Il est bien cruel, quand on sait à quel point l’être humain ne vit que pour la recherche effrénée de sa satisfaction immédiate et personnelle, d’exiger de l’homme d’Église qu’il renonce à toute forme de jouissance, sinon celle de l’extase mystique dont la nature pourrait d’ailleurs elle-même prêter à confusion.

À la lumière de ces réflexions, il devenait facile d’imaginer les raisons qui pouvaient pousser quelqu’un à s’en prendre à des prêtres. J’admets que le fait de les couper en morceaux et leur farcir la tête de croquettes pour chien laissait planer de sérieux doutes sur la santé mentale de l’auteur des faits. Pour des raisons évidentes, notamment le fait que les femmes ne fassent plus la cuisine, aient perdu toute expérience dans le domaine de la découpe des viandes et le farcissage des légumes (ne parlons pas des boîtes crâniennes), il y avait fort peu de chances que ces crimes atroces soient l’œuvre de l’une d’entre elles, même la plus repoussante des camionneuses, la plus épouvantable des gouines, la plus monstrueuse des brouteuses de minous. Je penchais davantage en faveur de la vengeance d’un individu de sexe mâle, homosexuel pourquoi pas, en tout cas un homme dont lui-même ou un être cher, un fils par exemple (qui aurait mis fin à ses jours ou sombré dans la démence, laissant son géniteur dans le désespoir le plus extrême), aurait été victime des agissements inqualifiables et répétés d’un homme d’Église apparemment au-dessus de tout soupçon.

Pour conforter cette hypothèse, il importait donc de se livrer sans plus attendre à une expertise approfondie de la vie privée des ecclésiastiques concernés.

C’est ce que j’ai commencé à faire dès le lendemain, en sollicitant une entrevue avec la plus haute autorité du diocèse, en la personne éminemment respectable et en tout point admirable de monseigneur Mathéo Riqueti, cardinal de son état.

Mathéo Riqueti était un homme de belle allure, au regard aussi pénétrant que la bite de l’abbé Pierre dans le cul d’une novice terrassée par son magnétisme et sa ferveur religieuse, habitué à ce que les gens se prosternent à ses pieds et boivent ses paroles avec avidité. La première chose que j’ai pensé, en le voyant, c’est que si ça se trouve il passait le plus clair de son temps à enculer des enfants de chœur dans les coins sombres des églises (qui n’en manquent pas, il faut bien le dire). Okay, je retire ce que j’ai dit, honte à moi, je cours de ce pas me suspendre à des crochets de boucher et me rouler dans les braises pour expier mes vilaines pensées. Satan, sors de ce corps !

Il m’a reçu dans un petit salon sympathique, avec fauteuils aussi absorbants que du PQ triple épaisseur, tentures de velours rouge et meubles en bois précieux servant de supports à des bibelots raffinés.

Je lui ai dit que j’enquêtais sur les meurtres des pères Vidal et Beaubois, que j’avais des raisons de penser que ce n’était pas par hasard qu’on les avait pris pour cible, et que j’aurais aimé en savoir un peu plus sur leur compte. Leur vie privée, notamment.

Ses épaules se sont quelque peu ratatinées et sa tête est rentrée dans son cou. Un frisson d’angoisse imperceptible (pour tout œil aussi diaboliquement avisé que le mien) a parcouru son visage poupin surmonté d’une élégante collerette de cheveux grisonnants.

\textsc{Moi} : Waterflox, ça vous parle ?

\textsc{Lui} : Non. C’est quoi ?

\textsc{Moi} : Une marque de croquettes pour chien.

\textsc{Lui} : Quel rapport avec les pères Vidal et Beaubois ?

\textsc{Moi}, sortant un Short Story de la poche de ma veste : Ça vous dérange si je fume ?

\textsc{Lui} : Oui, je suis vraiment désolé. Donc ?

\textsc{Moi}, reniflant le cigare avec envie avant de le remballer : Donc quoi ?

\textsc{Lui} : Quel rapport avec les pères Vidal et Beaubois ?

\textsc{Moi} : On leur a farci le crâne avec des croquettes pour chien de la marque Waterflox. Des croquettes de luxe à l’agneau et au bœuf. Vous n’êtes pas censé le savoir, on n’en a pas parlé dans les journaux.

\textsc{Lui} : Dans ce cas, je ne vois pas pourquoi vous me posez la question.

\textsc{Moi} : Déformation professionnelle, histoire de vérifier que l’info n’a pas fuité.

\textsc{Lui} : L’œuvre d’un dément, à n’en pas douter.

\textsc{Moi} : C’est aussi mon avis.

\textsc{Lui} : Ou du diable en personne !

\textsc{Moi} : Si c’est le cas, on va avoir du mal à l’attraper.

\textsc{Lui} : Quoi qu’il en soit, je ne vois pas très bien ce que je peux faire pour vous, à part vous proposer un petit verre de vino santo. Celui-ci vient du domaine Badia à Coltibuono\nf{Badia a Coltibuono (abbaye de la «~bonne récolte~»), domaine viticole situé en Toscane dans le Chianti Classico, au nord de Gaiole in Chianti. Fondé au XI\textsuperscript{e} siècle par les moines bénédictins de l’ordre de Vallombrosa, il produit des vins rouges et un \textit{vin santo} réputé.\source{fr.wikipedia.org/wiki/Badia\_a\_Coltibuono}}, fondé il y a plus de mille ans par les moines bénédictins de Vallombrosa\nf{L’ordre de Vallombrosa, branche réformatrice des Bénédictins, fut fondé vers 1038 par saint Jean Gualbert près de Florence. L’abbaye-mère de Vallombrosa, dans les Apennins toscans, est encore habitée par des moines. L’ordre, dit «~silvestrin~» ou «~vallombrosain~», insiste sur la pénitence et la prière contemplative. \source{fr.wikipedia.org/wiki/Ordre\_de\_Vallombreuse}}, au nord de Gaiole. Il mûrit six ans dans des petits fûts de châtaignier avant d’être commercialisé.

\textsc{Moi} : Je ne dis pas non.

Il a hurlé : Dardariel, s’il vous plaît !

Sorti de nulle part, un loufiat en soutane s’est approché en rampant plus bas que terre : Monseigneur ?

\textsc{Monseigneur} : Apportez-nous deux verres de vino santo, s’il vous plaît.

\textsc{Le loufiat} : Tout de suite, Monseigneur.

Quelques instants plus tard, j’avais en main un petit verre finement ciselé rempli du précieux nectar aux reflets cuivrés, apporté par un Dardariel qui s’était évaporé avant que j’aie eu le temps de cligner des yeux.

\textsc{Moi}, reniflant le breuvage avec délectation : À votre avis, qu’est-ce qui peut pousser un détraqué à s’attaquer à des hommes d’Église ?

\textsc{Lui} : On a coutume de dire que les voies du Seigneur sont impénétrables. J’ajouterai que celles du démon le sont tout autant.

\textsc{Moi} : Vraiment ?

\textsc{Lui} : Mais oui.

\textsc{Moi} : Écoutez, cher monsieur, je ne vais pas y aller par quatre chemins. Je sais que l’Église, comme l’Armée, a le goût du secret. On lave son linge sale en famille, avec ses propres règles, on se croit au-dessus des lois. Vous n’avez pas le fond de culotte très propre, loin s’en faut, mais vous faites tout pour empêcher les mauvaises odeurs d’arriver jusqu’aux narines du profane. Je ne suis pas spécialement porté sur la religion, vous l’aurez compris, mais deux curés viennent de se faire étriper avec une violence rare. On les a découpés en morceaux et on leur a farci le crâne avec des croquettes pour chien Waterflox. Vous n’avez pas de chien, Monseigneur ?

\textsc{Lui}, sirotant tranquillement son vino santo : Les chiens demandent beaucoup d’attention, inspecteur, et mes obligations envers Dieu et les Hommes ne me laissent aucun répit. C’est tout juste si j’ai le temps de faire une petite partie de golf de temps à autre.

\textsc{Moi} : Alors écoutez bien ceci : les croquettes pour chien Waterflox bénéficient de recettes élaborées par les meilleurs cuisiniers du monde. Il s’agit ici d’agneau à la sauce hoisin accompagné de riz délicatement parfumé au jasmin, de nature à flatter les truffes les plus exigeantes, une recette réservée aux chiens de race tout spécialement concoctée par Anada Sintawichai, un des meilleurs chefs de Bangkok. Le docteur Zaahid Shirani, légiste à la PJ qui a examiné les corps avec la plus grande attention et que vous connaissez peut-être, étant donné que lui-même joue au golf avec assiduité, est formel : ces putains de croquettes pour chien sont un produit de luxe réservé aux bourses les plus rebondies, et le travail pratiqué au niveau des boîtes crâniennes des victimes relève de la plus haute technicité.

\textsc{Lui} : Donc, si je comprends bien, le tueur a des notions de médecine et les moyens de ses ambitions. Pourquoi pas un chirurgien ?

\textsc{Moi} : Oui, comme pour Jack l’Éventreur\nf{Jack l’Éventreur (\textit{Jack the Ripper}), tueur en série non identifié qui assassina au moins cinq prostituées dans le quartier de Whitechapel, à Londres, entre août et novembre 1888. Son identité n’a jamais été établie avec certitude. L’hypothèse d’un meurtrier aux connaissances chirurgicales fut avancée dès l’époque.\source{fr.wikipedia.org/wiki/Jack\_l\%27\%C3\%89ventreur}}. J’y ai pensé, figurez-vous. Mais on n’est pas à Whitechapel et ce n’est pas pour ça que je suis venu vous voir. Ce que je veux savoir, c’est si les pères Vidal et Beaubois ont fait l’objet de signalements concernant des faits à caractère sexuel.

En entendant ces mots, le visage de monseigneur Mathéo Riqueti, jusqu’ici d’une parfaite impassibilité, est soudain retombé tel un vieux soufflé dans le fond de sa gamelle.

\textsc{Lui} : Vous voulez parler de…

\textsc{Moi} : D’attouchements sur mineurs, par exemple. Je pense plus particulièrement à des sodomies de jeunes garçons ou des fillettes contraintes de pratiquer des fellations pendant les cours de catéchisme.

\textsc{Lui} : Vous êtes ignoble !

\textsc{Moi} : Non, mais je fais un métier difficile dans un monde qui l’est.

\textsc{Lui} : Je dispose effectivement d’un certain nombre d’informations concernant les pères Vidal et Beaubois. Je suis prêt à vous en faire part, à titre personnel, pour l’avancement de votre enquête, mais je vous demanderai de garder ça pour vous. Vous savez que notre traverse depuis quelque temps une période difficile, accumulant scandale sur scandale, chose qui pourrait, à terme, conduire à la complète désertion de nos locaux. Si Dieu est parfait, irréprochable, rempli d’amour et de bonté, il arrive que ses émissaires le soient un peu moins. Ce ne sont que des hommes, après tout, avec leurs qualités et leurs défauts, des défauts contre lesquels ils luttent avec le concours de la foi et toute la force de leurs convictions. Mais il arrive parfois que leur volonté faiblisse face aux appels répétés de la chair, que la tentation soit trop forte. Le père Vidal est~-- était~-- un homme remarquable, dévoué corps et âme au salut de ses ouailles, mais je me suis laissé dire qu’il lui arrivait parfois de céder à la tentation. Plus d’une fois, les bénévoles chargés de l’entretien de son église l’ont surpris en train de se livrer à des actes que la morale réprouve. Les rares plaintes déposées ont été classées sans suite, rayées des fichiers sur l’insistance des plus hautes autorités religieuses. Moi-même, eu égard à ses états de service, suis parfois intervenu en sa faveur.

\textsc{Moi} : Et le père Beaubois ?

\textsc{Lui} : Une ordure de la pire espèce, mais avec des relations très haut placées qui lui ont toujours permis de s’en sortir sans une égratignure.

\textsc{Moi} : Il aimait les petits culs ?

\textsc{Lui}, embarrassé : La formule n’est pas très élégante, c’est le moins qu’on puisse dire.

\textsc{Moi} : Parce que vous trouvez élégant de tringler des gamins ?

\textsc{Lui} : Non, pas davantage. D’après ce que je sais, le père Beaubois n’avait pas de préférence marquée en la matière. C’était un opportuniste qui faisait feu de tout bois, sans distinction d’aucune sorte du moment que la victime n’avait pas encore atteint sa majorité.

\textsc{Moi} : Je vois. Alors que le père Vidal…

\textsc{Lui} : À ma connaissance, le père Vidal ne s’intéressait qu’aux petits garçons.

\textsc{Moi} : De toute façon, il ne fait pour moi guère de doute que l’auteur des crimes est un individu de sexe mâle. Il a dû avoir affaire aux pères Vidal et Beaubois et décidé de passer à l’acte après avoir longuement ruminé sa vengeance. Vous savez, quand on passe sa jeunesse à se faire enculer, on en garde souvent quelques mauvais souvenirs. Je pense qu’il est grand temps que l’Église commence à payer un peu le laxisme dont elle fait preuve depuis des siècles, je dirais même la lâcheté impardonnable qu’elle a tendance à afficher en toute circonstance. Votre vin est excellent mais je mentirais en disant que m’inspirez la moindre trace de sympathie, Monseigneur. Vous êtes comme tous ces curetons qui cachent sous leur soutane des montagnes de secrets inavouables. Votre voix est douce comme le miel mais votre cœur plus dur et sec que le granit. Allez vous faire foutre !

\textsc{Lui} : Dardariel !

Je ne vous l’ai peut-être pas dit, n’en n’ayant point vu l’intérêt sur le coup, mais Dardariel avait une des pires sales gueules qu’il m’ait été donné de voir au cours d’une existence qui ne m’avait pourtant pas ménagé sur le sujet. Non seulement il était le majordome de Riqueti, son homme à tout faire y compris sans doute se faire enculer le soir au fond des bois en chantant des cantiques, mais il remplissait aussi avec une certaine efficacité les fonctions de chauffeur et garde du corps.

\textsc{Dardariel} : Monseigneur ?

\textsc{Monseigneur} le roi des enculés : Cet odieux personnage vient de m’insulter et me manquer de respect comme rarement quelqu’un s’est permis de le faire. Frappez-le, s’il vous plaît.

Dardariel portait une tenue à mi-chemin entre la robe de bure à capuche et le costume de ninja. Je pense qu’il avait dû faire du catch ou un truc dans le genre avant d’entrer dans les ordres. Ou alors Monseigneur l’avait rencontré dans le Marais et décidé de le prendre (par derrière de préférence) à son service après une rapide formation aux textes sacrés et autres rituels religieux.

\textsc{Dardariel} : Avec joie, Monseigneur.

Je ne sais pas pourquoi, je ne lui avais rien fait, mais Dardariel n’avait pas l’air de me porter dans son cœur. Sans doute que ma gueule de parfait enfoiré de flic sournois ne lui revenait pas.

Dardariel s’est approché et planté devant moi : Levez-vous, s’il vous plaît.

\textsc{Moi} : Non, je suis très bien assis.

\textsc{Lui} : Comme vous voudrez.

Il m’a attrapé par le colbac et projeté à travers la pièce.

C’était la première fois qu’un membre du clergé me faisait subir un tel affront. J’ai senti la haine monter en moi tel un tsunami dans un de ces films-catastrophes de merde que l’industrie du cinéma américain tourne à la chaîne à grand renfort d’acteurs à chier, d’effets 3D et de connerie artificielle du style Chatte-J’ai-pété.

J’ai dit, en ramassant mes morceaux et essayant de les remettre dans le bon ordre : vous savez ce qu’il en coûte de s’en prendre à un fonctionnaire de police dans l’exercice de ses fonctions ?

\textsc{Lui} : Vous auriez dû y penser avant de manquer de respect à Monseigneur.

\textsc{Moi} : Penser à quoi ?

Ses yeux ont fait trois tours dans leurs orbites, et j’ai compris que son cerveau avait un peu de mal à suivre le rythme. Il n’avait certainement pas été recruté pour ses capacités intellectuelles.

Par contre, niveau qualité des articulations, j’ai pu goûter à la dureté de ses poings quand il m’a envoyé une droite en pleine mâchoire. Pas très appuyée, la droite, mais tout de même assez pour me clouer le bec pendant quelques dixièmes de seconde (ce qui n’est pas chose facile, vous l’aurez sans doute remarqué).

Quand je l’ai vu rappliquer dans ma direction, alors que j’en étais encore à me remettre les dents en place dans leurs alvéoles, je me suis dit qu’il était temps de passer aux choses sérieuses. Rapide comme la foudre, ma main a filé dans ma poche revolver et sorti Manu qui bouillait d’impatience de se joindre à la fête. Il avait des fourmillements dans la détente.

Manu ratait rarement ses entrées, et je crois pouvoir dire que celle-ci a été appréciée à sa juste valeur. Dardariel d’abord, qui fondait sur moi avec la rapidité du faucon, a stoppé net sa progression. Il faut dire que Manu le regardait droit dans les burnes, de quoi faire réfléchir même le plus abstinent des moines. Monseigneur ensuite, qui s’est ratatiné dans le fond de son fauteuil comme s’il avait vu le diable en personne.

\textsc{Moi} : Franchement, messieurs, vous commencez à me taper sur le système, au propre comme au figuré. Je vais essayer de prendre sur moi et faire comme s’il ne s’était rien passé. Dardariel, mon petit Dardariel, viens donc voir par ici.

Dardariel s’est approché, à reculons, ses petits poings tellement serrés qu’on pouvait entendre les jointures craquer à des lieues à la ronde. Il avait beau être con comme un balai, il s’était rendu compte que mon invitation n’avait rien de cordial.

Dès qu’il a été à portée de main, je lui ai balancé un bon coup de crosse en pleine gueule, histoire de lui faire comprendre, ainsi qu’à son connard d’employeur, que les forces de l’ordre gardaient le contrôle de la situation quoi qu’il advienne, et que ce n’était certainement pas une bande de curetons mal ensoutanés qui allaient foutre le bordel dans ma circonscription.

Je pense que c’est un signe : quand les hommes d’Église commencent à se conduire comme la pire des racailles de banlieue, alors il est grand temps de changer d’air. Mais pas question pour autant d’aller me faire chier sur Mars avec Elon Musk, Trump, Poutine, Bolsonaro, Meloni et toutes les faces de pet dictatoriales et nazifiantes de la planète. Il va falloir trouver une alternative. Laquelle, je ne sais pas encore, mais j’y travaille.

\textsc{Moi} : Écoutez, Monseigneur (mot prononcé avec tout le mépris dont j’étais capable), il va me falloir le maximum de détail concernant les plaintes déposées à l’encontre des pères Vidal et Beaubois. Qui, quand et pourquoi. Si vous me mangez gentiment dans la main, sans faire de vaguelettes, j’essayerai de passer l’éponge sur vos agissements et le fait que vous êtes une grosse pourriture qui pue et me retourne le cœur à chaque que je pose les yeux dessus. Dans le cas contraire, je vous promets une campagne de pub dont vous aurez du mal à vous remettre.

Apparemment convaincu par la solidité de mes arguments, Riqueti avait fini par lâcher des infos qui allaient m’être d’une grande utilité pour la suite de mon enquête.

Sauf que, comme tous les enfoirés de son espèce, drapés ou non des oripeaux de la religion, non seulement il n’avait aucune confiance en son prochain, ce que je pouvais difficilement lui reprocher, mais il avait la rancune tenace.

Ce même jour, en fin d’après-midi, on avait fait une descente dans la cité des Alisiers pour remuer un peu la merde. On s’était fait repérer dès notre arrivée, méchamment caillasser, et on avait préféré battre en retraite plutôt que de risquer le mauvais coup. En désespoir de cause, on s’était rabattu sur un pédophile dont le dossier traînait au-dessus de la pile depuis quelques semaines. On avait gentiment sonné à sa porte, comme des gens civilisés, et ce gros tas de merde était venu ouvrir en slip et débardeur crasseux, des morceaux de chips plein la barbe. Il était en train de s’empiffrer devant la télé. On lui a dit qu’on savait des choses sur lui, il a dit qu’il ne voyait pas de quoi on voulait parler, qu’il avait effectivement eu quelques petits problèmes par le passé, mais que ça lui avait servi de leçon et qu’il avait définitivement mis un terme à sa carrière d’exhibitionniste à la sortie des écoles. L’ennui, c’était qu’on avait des informations qui semblaient prouver le contraire. Son ordi était sous surveillance depuis un bout de temps, et ses connexions à répétition sur des sites pédocriminels les plus dégueulasses du web ne plaidaient pas en sa faveur. Titus Beaugendre, mon homme de confiance, compagnon d’infortune, toujours partant pour la grande aventure, carrossé comme un ours et monté comme un taureau (on aurait dit qu’il pissait avec un tuyau d’arrosage), s’était fait tripoter dans les vestiaires de son club de foot quand il avait entre huit et dix ans. Il avait arrêté le foot mais l’expérience lui avait laissé un très mauvais souvenir, de sorte que la vision d’un pédophile, même de très loin, de dos et dans le brouillard, lui causait des migraines insupportables. Quand l’autre a commencé à faire le malin, il n’a pu s’empêcher de lui coller quelques crochets dans le gras du bide pour lui faire fermer sa gueule. Le représentant de l’ordre passe la majeure partie de son temps sous tension, il faut bien qu’il décompresse quand l’occasion se présente. On a saisi le matos et embarqué le pédophile, histoire de le cuisiner un peu à tête reposée.

Le soir même, après une journée de boulot bien remplie comme vous avez pu le constater, je rentrais tranquillement chez moi, avec la satisfaction du devoir accompli, quand je me suis rendu compte que quelque chose clochait dans le périmètre. Dans le rétro de ma Kangoo, plus exactement. Sauf à être devenu complètement parano, ce qui n’était pas impossible vu les cadences infernales auxquelles nous étions soumis, une voiture me suivait. Une Alfa Romeo Giulietta noire aux vitres teintées, le genre de caisse pilotée neuf fois sur dix par une tête à claques qui se prend pour Fangio et finit encastré dans le premier platane qui passe à sa portée. Nous, les vrais flics de terrain, disposons d’un sixième sens qui nous permet de savoir instantanément si quelqu’un nous suit ou pas. Cela dit, conscience professionnelle oblige, j’ai fait trois fois le tour du quartier pour m’en assurer. Il me suivait, ou alors quelqu’un le suivait lui aussi et il habitait le même quartier que moi. Mais la plupart des flics, vous le savez comme moi, ont toutes les peines du monde à croire aux coïncidences. Je ne faisais pas exception à la règle.

Je me suis garé près de chez moi, suis descendu tranquillement de ma voiture, comme si de rien n’était, et me suis dirigé vers l’entrée. Comme je m’y attendais, l’Alfa m’a dépassé au ralenti pour aller se garer à quelques mètres de là.

Je résidais au sixième étage d’un immeuble qui, s’il ne datait pas exactement d’hier, n’en présentait pas moins toutes les garanties nécessaires en termes de confort et d’habitabilité. Il y avait l’eau courante, le gaz, l’électricité et du double vitrage aux fenêtres. Il y avait même un ascenseur dans lequel on pouvait loger deux ou trois personnes (pas trop grosses) sans risquer la chute libre. Comme il tombait régulièrement en rade, produisant des bruits atroces qui donnaient tout sauf envie de monter dedans, je pouvais garder la forme en effectuant des allers et retours dans les escaliers. Ce n’était pas le cas de ma voisine d’en face, une vieille bique sèche comme un coup de trique avec des yeux de fouine, une langue de pute et des touffes de cheveux irrégulièrement plantées sur le crâne, qui se retrouvait en grande difficulté à chaque fois que ce foutu ascenseur déclarait forfait. Ses enfants venaient la voir une fois tous les trente-six du mois et ses petits enfants ne rappliquaient que pour lui taxer du fric (qu’elle ne manquait pas de leur donner, sachant pertinemment qu’ils ne viendraient plus sans ça). Une fois, je l’avais croisée à mi-chemin dans l’escalier, à l’agonie avec sa baguette sous le bras et ses poireaux qui dépassaient du cabas. Au lieu de l’achever, ce qui aurait été un acte de charité chrétienne de haute valeur morale, j’avais commis l’erreur de lui proposer mon aide. Depuis, elle surveillait mes allées et venues et me mettait à contribution dès qu’elle arrivait à me mettre le grappin dessus. Quand je n’étais pas de corvée de courses, ou autre, elle m’attrapait par la manche et me tirait chez elle pour avaler un verre de porto. Le porto, c'est sucré, pas trop alcoolisé, les vieux adorent ça. Elle avait aussi trois chats à peu près aussi vieux qu’elle qui chiaient et pissaient partout. L’un d’eux, un rouquin vaguement tigré avec des oreilles bouffées aux mites, s’échappait dès qu’il pouvait pour venir chier sur le paillasson devant ma porte. D’ailleurs, en règle générale, il s’échappait dès qu’il pouvait pour faire régner la terreur dans le quartier. Cette sale bête s’appelait Korax. Un nom de dragon, de créature maléfique, je me demandais bien où la vieille était allée chercher ça. Je m’en plaignais régulièrement à l’intéressée qui m’offrait un verre de porto pour se faire pardonner, une piquette proche du vinaigre qui me filait des aigreurs d’estomac pour le restant de la journée. Je pense qu’on aurait pu décimer une ville entière avec son porto. Peut-être qu’elle le fabriquait elle-même avec du sang de cochon et de l’acide sulfurique, mais il était impossible d’imaginer qu’elle telle horreur soit en vente libre dans le pays des droits de l’homme, l’art de vivre et la gastronomie. Je commençais à me demander si la vieille, sous ses allures de grand-mère inoffensive, n’était pas en fait une redoutable sorcière ayant survécu aux bûchers de l’Inquisition. Korax me détestait. Il y avait des tas d’endroits où il aurait pu chier et pisser sans faire chier le monde, mais il tenait absolument à le faire devant ma porte. Je voyais qu’il me détestait quand il me regardait fixement avec ses grands yeux verts, avec un air de défi, en remuant nerveusement la queue et poussant des grognements rauques comme s’il allait se jeter sur moi. Il avait au moins trois siècles et je me disais qu’il allait bien finir par crever. Mais non, son aversion pour moi le maintenait solidement accroché à la vie, lui donnait la force de continuer à se battre pour me pourrir la mienne. Quels que soient ses sentiments à mon égard, je dois admettre que Korax n’était pas un chat comme les autres. Par exemple, croyez-le ou non, il avait appris à prendre l’ascenseur. Il savait sauter sur le bouton pour l’appeler, et une fois à l’intérieur, sauter à nouveau sur le bouton pour le faire aller et venir. Il lui arrivait aussi, de plus en plus souvent (il commençait à se faire vieux), de se faufiler entre les jambes des gens pour profiter du voyage. Quand je dis les jambes des gens, je pense aux miennes en particulier, entre lesquelles il adorait se faufiler, pas tellement parce que mes jambes sont des jambes entre lesquelles il est plus agréable de se faufiler que d’autres, mais parce qu’il savait que la manœuvre m’exaspérait et que sa présence m’horripilait. À tel point que des fois, quand je le voyais roder dans les parages, prêt à se jeter dans mes pattes pour profiter de l’ascenseur en faisant semblant de ronronner, je préférais me farcir les six étages à pinces. Mais au lieu de lâcher l’affaire, il se les farcissait lui aussi, en se frottant dans mes jambes pour essayer de me faire tomber. Contrairement au chien, animal grégaire au tempérament servile, le chat n’aime pas les gens, qu’il tient pour responsables d’avoir fait de lui une carpette à sa mémère, tout juste bon à faire ses pattes sur un coussin en ronronnant bêtement. Dans animal domestique, il y a animal et domestique. Le chien est plus domestique qu’animal, tendance larbin dévoué corps et âme à son maître, alors que le chat est un animal qui non seulement déteste les gens, mais se déteste lui-même pour s’être laissé avilir par une espèce qu’il estime, sans doute à juste titre, très inférieure à la sienne. Jamais il ne nous pardonnera ce que nous avons fait de lui, et jamais il ne se pardonnera de nous avoir laissé faire. Reste que dans son esprit, c’est lui a domestiqué l’homme et pas le contraire. À ce titre, il a tous les droits et droit à toutes les déférences. Caractériel et sournois de nature, veule et manipulateur, autoritaire, il faut être d’une naïveté sans bornes pour lui accorder la moindre confiance. C’est d’ailleurs, n’en doutons pas, la principale raison de la fascination qu’il exerce sur les gens, plus ou moins conscients d’avoir entre les mains une grenade dégoupillée qui peut leur péter à la gueule à tout instant. Quand un chat vous regarde de travers en plissant les yeux, cela signifie grosso modo : quel dommage que je ne fasse pas trente ou quarante kilos de plus, car j’aurais le plus grand plaisir à t’arracher la tête et te pisser dans le cou.

Depuis le dernier passage du réparateur, trois semaines plus tôt, l’ascenseur semblait reparti pour une nouvelle vie. Il marchait du feu de Dieu, gravissant les étages à la vitesse d’une fusée et les dévalant à celle d’un skieur de haute montagne, de sorte que c’était tout juste si on avait le temps de monter à bord avant d’être arrivé. Des rumeurs circulaient comme quoi des gens auraient été coupés en deux en essayant de monter à bord. Apparemment, la vioque et son maudit chat avaient échappé au sinistre. Certes, l’engin produisait toujours des bruits suspects et autres turbulences dignes d’un avion de chasse en plein passage du mur du son, mais c’était sa façon d’exprimer mécaniquement sa joie d’avoir retrouvé la pleine et entière jouissance de ses facultés motrices.

Je suis monté dedans (ou plutôt glissé furtivement à l’intérieur, en essayant de ne pas me faire couper en deux) et j’ai appuyé sur le 6. Mais une fois arrivé en haut, au lieu de rentrer chez moi et m’envoyer un ou deux verres de blanc (une bouteille de Chablis Grenouilles m’attendait au frais, vous pensez si j’étais un tantinet pressé de lui rendre hommage, d’autant que la journée avait été particulièrement douce pour la saison et que la soif n’avait cessé de me tirailler) comme j’aurais aimé pouvoir le faire, je me suis planté devant l’ascenseur et j’ai attendu patiemment qu’il redescende, après avoir sorti Manu de sa cachette et vérifié qu’il était chargé à bloc, prêt à cracher son venin à la plus légère sollicitation de mon index droit (lequel, bénéficiant d’une autonomie relative, était sujet à de fréquentes crispations réflexes en période de stress). On n’arrête pas un éléphant avec du 6.35, mais, à bout portant, ça reste une arme tout à fait dissuasive pour le commun des mortels. Comme de juste, Korax, la bête immonde, féline incarnation des forces démoniaques sur terre, digne héritier du chat égyptien au regard fourbe et à la griffe acérée, s’est pointé dans mon dos. Avant même que j’aie pris conscience de sa présence, il était déjà en train de se frotter copieusement dans mes jambes (je n’ai rien dit pour ne pas faire d’esclandre, sachant qu’il était capable de piquer une crise si je tentais de faire obstacle à sa volonté), laissant des touffes de poil roux sur mes bas de pantalon d’une fraîcheur jusqu’ici impeccable. Il pensait peut-être que j’étais sur le point de monter dans l’ascenseur. Plus probablement, il avait reniflé que la situation était tendue et jugé par conséquent le moment idéalement choisi pour venir fourrer son museau dans mes affaires. Je n’ai pas vu d’où il sortait exactement, mais j’étais prêt à parier qu’il venait d’aller lâcher une belle grosse prune bien juteuse sur mon paillasson, petit cadeau de bienvenue dans lequel j’avais mis le pied si souvent que j’avais épuisé mes ressources en matière de calcul mental. Le pire c’était le matin, quand je sortais de chez moi la tronche encore ensuquée de sommeil et les paupières mi-closes. J’avais beau connaître la chanson, il m’arrivait encore, à une fraction de seconde près, de tomber dans le piège.

Je n’ai pas eu à attendre bien longtemps.

Quelqu’un a appelé l’ascenseur, et je ne doutais pas que ce quelqu’un se trouvait au rez-de-chaussée et qu’il s’agissait de l’homme à la Giulietta. En voyant l’ascenseur prendre le large, Korax m’a jeté un regard mauvais et s’est éloigné en trottinant en direction des escaliers, avec cette façon particulière qu’ont les chats de trottiner en levant la queue comme si ça les faisait marrer de vous montrer leur trou du cul. Ce trou du cul fonctionne comme un troisième œil qui vous fixe avec sa pupille marronnasse en ayant l’air de dire : je t’emmerde, connard. Mais comme je vous l’ai dit, Korax savait se servir de l’ascenseur. Il n’avait donc pas besoin de moi pour le prendre, mais adorait m’imposer sa présence de félin sournois et arrogant. Il adorait me faire chier, par exemple gratter ses puces dans ma direction dans l’intention manifeste qu’elles atterrissent sur moi et me vident de mon sang. Mais un jour ou l’autre, à force de me pousser à bout, de jouer avec mes nerfs, il aurait affaire à Manu, son œil de cyclope, son petit corps en acier trempé, froid et glacé comme la mort (glacé aurait sans doute suffi, mais deux précautions valent mieux qu’une), et surtout sa détente si hypersensible qu’il suffisait qu’un moucheron passe à proximité pour la déclencher (d’ailleurs moi-même, quand je le trimballais dans ma poche, j’évitais soigneusement tout mouvement brusque pour éviter de prendre une balle perdue).

Comme prévu, l’ascenseur est monté jusqu’au sixième.

Ça aurait aussi bien pu être la mère Ouvrard, Maria de son prénom, ma voisine d’en face et diabolique maîtresse de cet enfoiré de Korax, âme damnée des forces du Mal, la Voisin du sixième, connue pour empoisonner ses victimes avec de fortes doses de porto frelaté.

Tout comme ça aurait pu être Marc-Antoine Jacquinot, le prof de philo qui résidait à l’autre bout du couloir, un type d’une politesse exemplaire, irréprochable, mais dont le comportement laissait clairement entendre qu’il ne tenait aucunement à faire plus ample connaissance avec les autres locataires. C’était quelqu’un, voyez-vous, qui passait le plus gros de son temps à l’intérieur de lui-même, et n’en sortait que pour planer à mille lieues au-dessus des vicissitudes de l’existence, solitaire et majestueux dans le noir ciel de la connerie humaine, si loin qu’il avait fini par perdre tout contact avec la triste réalité de la vie bassement matérielle et intellectuellement indigente de ses contemporains. C’était le genre de type qui n’avait pas la télé et vivait dans le noir en permanence, rideaux tirés, s’éclairant à la bougie et se nourrissant de fruits et gâteaux secs, ainsi que de plats préparés qu’il réchauffait au micro-ondes, seule et unique concession de sa part au monde moderne, et consommait directement dans leur emballage, l’idée même de salir une assiette et se voir contraint de la laver lui apparaissant comme une aberration sans nom. Il n’avait bien entendu ni femme ni enfant, et encore moins d’animal domestique, chien, chat ou canari, tortue ou poisson rouge, lièvre ou musaraigne. La mère Ouvrard, toujours à la recherche de nouvelles victimes, avait maintes fois essayé de lui faire monter ses courses dans l’escalier, quand l’ascenseur faisait des siennes, s’affichant à mi-chemin du trajet, recroquevillée sur les marches telle une vieille chenille en voie de décomposition, une crotte de chien abandonnée dans le caniveau, physiquement au bout du rouleau, à l’article de la mort. Mais Jacquinot, qui n’était pas tombé de la dernière averse, avait toujours très habilement refusé, d’une voix grave et tranquille suintant la gentillesse, l’amour de l’autre, se confondant en excuses bouleversantes d’humilité, mettant en avant une faiblesse extrême liée à toutes sortes de problèmes de santé, de tares chroniques qui laissaient la médecine perplexe, à commencer par une série de hernies discales inexplicables qui le faisaient cruellement souffrir depuis son plus jeune âge. Folle de rage, dont elle ne laissait rien paraître, la vieille sorcière rêvait de lui administrer quelques doses de porto qui mettraient un terme définitif à ses souffrances. Même Korax, qui n’était pourtant pas du genre timide, hésitait à s’approcher de lui. Il faut dire que les rares fois où il s’y était risqué, il s’était pris des coups de la lourde sacoche en cuir qui accompagnait Jacquinot dans la majeure partie de ses déplacements.

Mais je savais, animé par ce sixième sens qui est l’apanage des plus fins limiers, qu’il ne s’agissait pas de lui ni de la mère Ouvrard.

Aussi, quand la porte de l’ascenseur s’est ouverte, n’ai-je pas été autrement surpris de me retrouver nez-à-nez avec ce cher Dardariel, exécuteur des basses œuvre de monseigneur Mathéo Riqueti. Figurez-vous que ce crétin, qui s’attendait à tout sauf à se retrouver aussi vite en face de moi, était en train de visser tranquillement un réducteur de son au bout d’un Glock 17 à canon fileté. Autrement dit, la visite qu’il s’apprêtait à me rendre n’avait rien de courtoise.

Quand il a vu Manu, auquel il avait déjà présenté et qui lui avait laissé un douloureux souvenir au coin de la mâchoire, et qu’il a entendu ma voix chaude et onctueuse lui intimer l’ordre de mettre les mains aussi haut que possible au-dessus de sa tête de nœud et sortir de l’ascenseur en marchant sur des œufs s’il ne voulait pas repeindre l’habitacle avec sa cervelle de moineau, il a réalisé un peu tard que ses compétences en matière de filature laissaient fortement à désirer.

Dès qu’il a été dehors, j’ai récupéré le Glock et lui ai attaché les mains dans le dos avec la paire de menottes que je trimballais sur moi en permanence.

Je lui ai demandé : T’as un permis de port d’arme, mon petit Dardariel ?

Il a répondu : Oui, je fais du tir au stand de la police.

Ce à quoi j’ai rétorqué : Et un permis de connerie, t’en as un aussi ?

Il m’a regardé en serrant les dents, et son visage était aussi chargé de haine que le ciel de Pompéi était chargé de cendres et autres vapeurs toxiques après l’éruption du Vésuve en 79 après J.-C.

J’ai remballé Manu, non sans lui avoir auparavant tendrement caressé la crosse en signe d’indéfectible affection, puis j’ai retourné Dardariel et lui ai collé le canon de son Glock dans le creux des reins : On va aller faire un tour. Si tu fais le moindre geste suspect, respire trop fort ou lâche un pet de travers, je te bute sans hésiter.

On allait monter dans l’ascenseur quand Korax, émissaire velu des forces occultes qui gouvernent le monde, s’est ramené, la queue en l’air et le trou de balle bien en évidence.

Il m’a toisé, et pour la première fois j’ai cru déceler une lueur de respect, sinon de sympathie, dans le fond de son regard.

Une fois en bas, j’ai dit à Dardariel : Passe devant, je te suis. Et rappelle-toi ce que je t’ai dit : au moindre pet de travers, je te bute.

Korax est resté dans l’ascenseur, puis il a sauté sur le 6 pour remonter gratter à la porte de la mère Ouvrard, laquelle devait être en train de somnoler comme une grosse merde en robe de chambre et pantoufles devant Qui veut gagner des morpions ou Les fions de l’amour.

Le cerveau de Dardariel tournait à plein régime pour imaginer une issue favorable à la situation. Après tout il n’avait rien fait de mal, à part se retrouver dans l’ascenseur d’un immeuble qui se trouvait, par le plus grand des hasards, être celui dans lequel résidait Djeferson Beauvais, inspecteur spécialisé dans la traque des criminels les plus endurcis, tueurs en série, violeurs d’enfants et autres saloperies du même genre. Certes il était, au moment des faits, porteur d’un Glock 17 et se trouvait précisément en train de l’équiper d’un silencieux quand Beauvais lui était tombé dessus, mais il était détenteur d’un permis de port d’arme, et, à sa connaissance, aucune loi n’interdisait spécifiquement de visser un silencieux à un pistolet dans un ascenseur, même s’il reconnaissait volontiers l’existence d’endroits sans doute mieux adaptés à ce type d’activité. Il était, d’autre part, titulaire d’une licence séminariale (il était l’auteur d’une thèse sur les mouvements pentecôtistes américains et les dérives sectaires du Renouveau charismatique), inspiré par et d’un diplôme de prêtre garantissant la parfaite moralité de sa personne. Ajoutez à cela qu’il ne travaillait rien moins que pour le compte de monseigneur Mathéo Riqueti, évêque du Sanctuaire de Ddarr (lieu saint de l’Église catholique et romaine) et ami personnel du \foreignlanguage{italian}{cardinale di tutti li cardinali} Prospero Cangelosi, chef de chœur au Vatican et bras droit de Sa Sainteté le Pape, et vous conviendrez qu’il est difficile, sinon inconvenant, de suspecter un tel homme de quelque mauvaise intention que ce soit, funeste à plus forte raison. Il exigerait la relaxe immédiate et les excuses publiques du sieur Djeferson Beauvais, voire sa radiation définitive des services d’élite de la police nationale. Sa rétrogradation au niveau de simple agent de la circulation, condamné à faire le pied-de-grue à la sortie des écoles, ne serait pas pour lui déplaire.

Je lui ai demandé : T’es garé où ?

Il a fait, avec un geste du menton en direction de l’endroit : Là-bas, un peu plus loin.

\textsc{Moi} : Okay, on y va.

\textsc{Lui} : Vous allez faire quoi ?

\textsc{Moi} : On va faire un tour en bagnole.

Quand la Giulietta a été en vue, je lui ai demandé, en appuyant ma demande d’une légère pression sur le Glock : Où sont les clés ?

\textsc{Lui} : Dans la poche de ma veste.

\textsc{Moi} : Laquelle, je te prie ?

\textsc{Lui} : La droite.

J’ai pris les clés et ouvert le coffre de la Giulietta.

\textsc{Lui} : Vous faites quoi, là ?

\textsc{Moi} : Tu vois bien, j’ouvre le coffre.

\textsc{Lui} : Pourquoi faire ?

\textsc{Moi} : Je t’ai dit qu’on allait faire un tour en bagnole. Allez, monte là-dedans.

\textsc{Lui} : On va où ?

\textsc{Moi} : Faire un tour en bagnole. J’ai toujours rêvé de conduire une Alfa. Je t’en prie, installe-toi.

\textsc{Lui}, de plus en plus inquiet : Pourquoi il faut que je monte dans le coffre ?

\textsc{Moi} : Parce que je te le demande. Je n’ai aucune confiance en toi, alors je préfère te savoir dans le coffre pendant que je conduis.

\textsc{Lui} : Je peux savoir où on va ?

\textsc{Moi} : Monte, tu verras bien.

Il a fini par monter, non sans rechigner dans les grandes largeurs, et dès qu’il a été à l’intérieur, après avoir pris soin de vérifier que personne n’était en train de reluquer alentour, je me suis permis, à sa plus grande surprise (tout est allé très vite et il n’a pas eu le temps de formuler la moindre objection), de lui loger trois balles dans le buffet qui ont définitivement mis un terme à ses activités de mercenaire de Dieu.

L’avantage d’habiter un quartier tranquille, c’est qu’on peut flinguer les gens en toute sérénité, sans se soucier de témoin éventuel qu’il faudrait flinguer à son tour et ainsi de suite jusqu’à ce qu’il n’y ait plus personne sur terre. C’est sans doute ce qui arrivera un jour, mais je ne me sentais pas vraiment d’attaque pour éliminer un par un huit milliards et quelque de témoins gênants. Le jour venu, ce n’est pas un par un que les gens crèveront mais par treize à la douzaine, décimés par des armes chimiques, bactériologiques ou nucléaires, voire naturelles si la planète finit par se fâcher pour de bon comme il semble qu’elle soit en train de le faire.

Je me suis glissé au volant de la Giulietta et j’ai appelé Titus pour lui annoncer que je rappliquais dare-dare avec un truc dans le coffre dont je comptais bien me débarrasser au plus vite.

Le truc en question s’appelait Dardariel, et il s’agissait d’une sorte de moine de combat qui bossait pour Mathéo Riqueti. Cet enfoiré s’était pointé à mon domicile dans l’intention manifeste de me buter. Seulement je l’avais repéré en moins de deux et lui avait tendu un petit guet-apens dans lequel il s’était précipité comme un débutant. J’avais aussi pas mal avancé sur le Brain Catcher et les croquettes Waterflox. Passereau, même s’il avait le profil de l’emploi et avait un temps possédé un Ratier de Majorque, était selon moi définitivement hors de cause, même si j’étais certain qu’on aurait pu trouver des trucs pas très reluisants sur son compte en fouillant dans les recoins et autres interstices nauséabonds de sa pitoyable existence. De toute façon, ne serait-ce que financièrement parlant, les croquettes Waterflox, n’étaient pas à la portée de ses bourses.

Par contre, d’après cette ordure de Riqueti, lequel s’était pas mal fait tirer l’oreille avant de cracher le morceau, les pères Vidal et Beaubois faisaient plus que tripoter des enfants de chœur et enfiler louveteaux et jeannettes pendant les camps de vacances. Car comme le disait fort justement le regretté Lord Robert Stephenson Smyth Baden-Powell of Gilwell\nf{Robert Baden-Powell (1857--1941), général britannique, héros du siège de Mafeking lors de la guerre des Boers (1899--1902). Fondateur du mouvement scout mondial (\textit{Scouting for Boys}, 1908) et des Louveteaux (1916). Grand Commandeur de l’ordre royal de Victoria, il mourut au Kenya en 1941. \source{fr.wikipedia.org/wiki/Robert\_Baden-Powell}}, héros de la guerre des Boers et Grand Commandeur de l’ordre royal de Victoria : «~Les louveteaux ont une recette simple pour se procurer du bonheur : ils s’arrangent pour en donner aux autres~». De quoi entrevoir des perspectives quasiment illimitées et rayonnantes de lumière céleste pour tout prédateur sexuel digne de ce nom. Aux dernières nouvelles, Vidal et Beaubois faisaient partie d’une organisation pédocriminelle incluant non seulement des ecclésiastiques plus ou moins gradés dans la hiérarchie religieuse, mais aussi des individus venant de toutes disciplines et horizons dont les identités, si elles venaient à être connues du grand public, pourraient créer un séisme sans précédent sur le territoire national et même au-delà. J’ignorais ce que Riqueti savait au juste et jusqu’à quel point il était impliqué, mais sa tentative pour me faire taire prouvait que l’affaire était d’une gravité hors norme.

Titus Beaugendre avait deux caractéristiques dont je ne vous ai pas encore parlé, non que je veuille faire des secrets mais tout simplement parce que l’occasion ne s’était pas encore présentée : un, c’était un homme de couleur, noire en l’occurrence, et deux, c’était un homme marié. Le fait qu’il soit de couleur n’a pas de réelle importance pour la suite de l’intrigue, inexistante par ailleurs (je n’aime pas tout ce qui est intrigue, mystère et cachotterie, coup de théâtre et rebondissement de mes deux), mais je tenais à le signaler pour que les choses soient claires. Si je ne l’avais pas écrit noir sur blanc, il y a de grandes chances que votre cerveau de lecteur de race blanche l’ait imaginé comme un mâle caucasien issu des chasseurs-cueilleurs du Paléolithique supérieur ou des bergers Yamnaya des steppes pontiques de basse Volga, et cela, je ne puis me résoudre à l’accepter. D’autant qu’une telle méprise, sans être nécessairement catastrophique, pourrait néanmoins conduire à des contresens regrettables.

D’origine africaine, donc (Sierra Leone en l’occurrence, descendant d’un loyaliste noir des King’s American Dragoons débarqué à Freetown à la fin du XVIIIe siècle), et marié, Titus vivait avec femme (Bérénice) et enfants (deux, un garçon et une fille, Paul et Virginie) dans un pavillon de banlieue (hérité de ses parents, je vous épargne leurs noms et le reste, ça n’a aucun intérêt pour la suite de l’enquête, enquête dont on se contrefout accessoirement, mais bon, c’est un autre sujet) assez moche mais doté d’un petit lopin de terre pour faire pousser des patates et des laitues, ce que notre homme s’employait à faire dès qu’il avait un moment de libre. Sauf que des moments de libres, il en avait rarement, entre son boulot de flic et ses activités parallèles de justicier masqué, de sorte que c’était plus généralement Bérénice et les enfants (Paul surtout, car Virginie, férue de littérature, préférait se plonger dans la lecture assidue des œuvres complètes de Rousseau, Montesquieu, Voltaire, Sainte-Beuve, et bien sûr Bernardin de Saint-Pierre\nf{Jacques-Henri Bernardin de Saint-Pierre (1737--1814), écrivain et botaniste français, auteur de \textit{Paul et Virginie} (1788), roman idyllique situé à l’île Maurice, et d’\textit{Études de la nature} (1784). Élève du père La Caille à Caen, il devint l’ami et le disciple de Jean-Jacques Rousseau. \source{fr.wikipedia.org/wiki/Bernardin\_de\_Saint-Pierre}}, ancien élève de l’abbé La Caille\nf{Nicolas-Louis de La Caille (1713--1762), astronome français, prêtre catholique, professeur au Collège Mazarin à Paris. Il dirigea l’expédition au cap de Bonne-Espérance (1750--1754) qui lui permit de cataloguer dix mille étoiles australes et de mesurer un arc de méridien dans l’hémisphère sud. \source{fr.wikipedia.org/wiki/Nicolas-Louis\_de\_La\_Caille}} au collège des Jésuites de Caen) qui s’occupaient des patates et des laitues, ainsi d’ailleurs que d’à peu près tout dans la maison.

Il était précisément en train de fumer une clope dans le jardin quand il a reçu mon appel concernant Giulietta et son encombrant chargement. Dès qu’elle me voyait rappliquer, Bérénice savait que j’allais lui emprunter son mari et qu’elle ne le reverrait plus que tard dans la nuit, quand il rentrerait (ivre ou non) sans faire de bruit (ou en essayant d’en faire le moins possible) avant de la rejoindre subrepticement sous la couette. Elle voyait en moi un trublion de la vie domestique, un empêcheur de tourner en rond, un casse-couille de première. Pour elle, son mari était un homme droit et honnête que ses mauvaises fréquentations amenaient parfois à déraper, lesquelles mauvaises fréquentations se résumaient essentiellement à un seul et unique nom : le mien. Malgré tout, même si elle rêvait régulièrement de voir mon corps pendu au bout d’une corde ou réduit en charpie dans un amas de tôle, elle ne pouvait s’empêcher s’éprouver pour moi une certaine tendresse mêlée de fascination morbide, du même genre que celle que Titus nourrissait à mon égard. Il était, je dois le dire, le seul ami d’enfance qui me soit resté fidèle. Très tôt, alors que nous n’étions encore que des adolescents en proie aux affres de la puberté (et que j’étais personnellement couvert d’acné qui a laissé des traces sur ma peau), j’ai décelé chez lui un sens aigu de la justice et de réelles aptitudes à la violence. À de nombreuses reprises, il m’a sauvé la mise alors que j’étais en passe de me faire massacrer par un adversaire d’une bande rivale. Je le suis un peu plus maintenant, mais à l’époque je n’étais pas très costaud. Hargneux, oui, mais pas très costaud. Et aussi assez habile intellectuellement, ce qui me permettait de remplir les fonctions de chef d’équipe et cerveau des opérations.

Naturellement, il fallait attendre que la nuit soit bien noire pour aller déposer le colis dans un endroit tranquille.

D’irrésistibles effluves de coulis (colis, coulis, certains mots tissent entre eux d’étranges résonances) de tomates fraîches, huile d’olive et viande de bœuf rôtie parvenaient à mes naseaux frémissants dans l’air du soir. La fenêtre de la cuisine était grande ouverte, et c’était le chemin qu’ils empruntaient pour arriver jusqu’à moi. Pas besoin d’avoir un nez crochu, fumer la pipe et s’appeler Sherlock Holmes pour comprendre que la maîtresse des lieux était en train de concocter sa légendaire sauce bolognaise, une sauce dans laquelle elle mettait, outre ses talents culinaires, tout son amour et sa sensualité. Je suis aussitôt allé lui présenter mes salutations respectueuses, laissant discrètement pendre ma langue pour signifier que le pauvre célibataire que j’étais n’avait rien à manger dans son frigo, ce qui n’était pas totalement faux.

\textsc{Titus}, qui connaissait le langage des langues pendantes, a proposé : Tu mangeras bien avec nous.

\textsc{Moi}, pauvre célibataire aussi sournois qu’un serpent glissant sans bruit telle une babouche furtive dans le sable du désert chauffé à blanc : Je ne voudrais pas déranger.

\textsc{Bérénice} : Tu sais bien que tu ne nous déranges jamais, Djef.

Elle mentait presque aussi bien que moi.

J’ai ajouté : Je crois que si je devais n’emporter qu’une seule chose sur une île déserte pour y passer le restant de mes jours, ce serait les spaghetti bolognaise de Bérénice.

J’exagérais sans doute un peu, mais le fait est qu’entre ça et les œuvres complètes du grand poète élisabéthain Edmund Spenser\nf{Edmund Spenser (vers 1552--1599), poète anglais de la période élisabéthaine, auteur de \textit{La Reine des fées} (\textit{The Faerie Queene}, 1590--1596), vaste épopée allégorique en douze livres projetés dédiée à la reine Élisabeth I\textsuperscript{re}. Il est considéré comme le «~prince des poètes~» de la Renaissance anglaise. \source{en.wikipedia.org/wiki/Edmund\_Spenser}}, pour lequel j’ai pourtant une admiration sans bornes (je pense notamment à son chef-d’œuvre La Reine des fées), je n’aurais pas hésité pas un seul instant.

Titus, qui n’attendait que ça, s’est précipité à la cave (on aurait dit le piaf dans Bip Bip et Vil Coyote) pour en revenir quelques fractions de seconde plus tard en compagnie de deux fiasques de Chianti Rodolfo Dragone, un breuvage pour lequel il affichait une appétence proche de l’addiction. Moi un peu moins, étant donné son acidité décapante, mais je comptais sur la cuisine de Bérénice pour atténuer le feu qui ne manquerait pas de me déchirer les entrailles après quelques verres de cette piquette machiavélique. Quant à Titus, sa corpulence de gorille des montagne s’accompagnait d’une résistance hors du commun à l’alcool. Il aurait pu avaler les deux fiasques cul sec sans ressentir la moindre trace d’ébriété, alors que des gens normaux comme vous et moi n’auraient pas survécu à l’ingestion de la première. Après être passés par toutes les couleurs de l’arc-en-ciel, nous nous serions effondrés comme des masses pour ne plus jamais refaire surface. Eh oui, la vie n’est pas juste, car si nous ressentons tous les effets délétères de la soif ou la désespérance, nous ne sommes pas tous égaux devant l’alcool.

Vers minuit, soit deux fiasques de Rodolfo Dragone plus tard (finalement pas si mauvais ; les premiers verres sont un peu raides, je vous le concède, mais avec un peu de persévérance on finit par lui découvrir des mérites insoupçonnés), Paul et Virginie (respectivement 14 et 16 ans pour celles et ceux que ça intéresse) avaient regagné leurs chambres depuis longtemps, la compagnie des adultes n’étant pour eux qu’une péripétie assez désagréable à laquelle il convenait de se soustraire au plus vite. Bérénice, sans doute soucieuse d’en apprendre un peu plus sur nos projets et s’assurer qu’on n’allait pas faire de connerie majeure dès qu’elle aurait le dos tourné, nous avait tenu la jambe un peu plus longtemps avant de tirer sa révérence, vaincue par nos blagues graveleuses de mâles alpha et nos itérations inflexibles concernant le secret professionnel et le fait que moins on en sait mieux on se porte.

Oui, nous avions des choses à faire, effectivement, mais le seul fait d’aborder le sujet, même de façon évasive comme c’était le cas, constituait déjà une grave entorse au devoir de réserve auquel nous étions soumis.

Non, l’opération ne présentait pas de danger particulier, et encore moins pour des professionnels de notre trempe.

Pourquoi la mener au milieu de la nuit ?

Eh bien, même si je n’étais pas censé répondre à cette question, le fait est que certaines choses nécessitent un peu plus de discrétion que d’autres, d’où l’obligation de travailler de nuit plutôt que s’exposer en plein jour. On pouvait, à la rigueur, classer ces activités nocturnes dans la catégorie des heures supplémentaires, évidemment non rémunérées pour cause de restrictions budgétaires drastiques. Suite à un certain nombre de malversations, hausse des matières premières, déboires commerciaux, fuites des capitaux, cupidité des actionnaires et autre train de vie somptuaire au plus haut niveau de l’État, les finances publiques étaient au bord du gouffre et le Peuple tout entier (surtout les pauvres, bien sûr, qui sont les plus nombreux et habitués à se faire mettre bien profond avec le sourire comme s’il s’agissait d’une marque d’affection de la part de leurs exploitants) allait une fois de plus devoir mettre la main à la poche pour leur sauver la mise. Question de solidarité avec ses dirigeants, lesquels, dès que la navire flamboyant de la Nation voguerait à nouveaux sur les flots argentés d’un avenir radieux, avaient fait la promesse de lui renvoyer grassement l’ascenseur, l’art de faire des promesses et ne jamais les tenir étant la première chose qu’on apprend dans les grandes écoles d’administration.

Et alors ?

Alors rien. Simplement, certaines affaires se traitent dans la pénombre plutôt que la lumière aveuglante des projecteurs. Il se trouvait que Titus et moi-même, sans être des agents secrets à part entière, devions parfois remplir des missions officiellement inexistantes en dehors de nos heures de travail. Rien, pas même le Chianti ou les somptueux spaghetti bolognaise dont elle venait de nous régaler, ne serait de nature à nous délier la langue. Même sous la torture, par exemple si on nous avait agrafé les paupières et obligés à regarder en boucle Super Mario Bros.\nf{\textit{Super Mario Bros.} (1993), film de science-fiction réalisé par Rocky Morton et Annabel Jankel, adaptation du jeu vidéo de Nintendo. Avec Bob Hoskins et Dennis Hopper, il fut un échec commercial et critique retentissant, régulièrement cité parmi les pires adaptations de jeux vidéo. \source{fr.wikipedia.org/wiki/Super\_Mario\_Bros.\_(film)}} de Rocky Morton et Annabel Jankel, jamais, je dis bien jamais, même si après vingt-quatre heures non-stop de Super Mario Bros. il avait fallu s’enquiller tous les pires films d’exploitation du cinéma pré-code hollywoodien, comme par exemple l’histoire tragique de cette secrétaire devenue pute dans Safe in Hell de William Wellman, jamais nous n’aurions livré la moindre information concernant une affaire en cours.

Et non, Bérénice, je suis désolé mais le monde n’est pas qu’un parc d’attraction peuplé de mascottes inoffensives et surdimensionnées qui font le bonheur des petits et des grands, le tout sur fond de musique de fête foraine affligeante et odeurs poisseuses de barbe à papa. Derrière les masques souriants se dissimulent des grimaces de super-vilains qui n’attendent qu’une occasion de faire régner la terreur sur terre. Et il se trouve, par le plus grand des hasards, que Titus et moi sommes précisément chargés, sinon de mettre un terme définitif à leurs agissements, au moins de modérer leurs ardeurs.

Et moi, quand je modère, je modère sans modération.

Vaincue par nos arguments fallacieux, ou plutôt profondément exaspérée par le tissu de conneries que nous n’avions cessé de lui tricoter depuis des heures, Bérénice avait fini par aller se coucher, laissant le champ libre à nos élucubrations judiciaires.

Une heure plus tard environ, la seconde fiasque de Chianti arrivant à son terme, j’ai sorti un Hiram \& Solomon Fellow Craft Gran Toro de la poche de ma veste, le partenaire idéal pour se remettre les idées en place en fin de soirée. Il n’était bien entendu pas question de fumer à l’intérieur, la fumée de cigare (et l’odeur de tabac froid qu’elle laisse derrière elle) étant considérée par la majeure partie des gens (hommes et femmes confondus, même si ces dernières, habituées à ne respirer que des odeurs flatteuses de cosmétiques, sont nettement plus intransigeantes sur le sujet) comme une des pires agressions olfactives qui soient. Comme je ne tiens pas à m’énerver inutilement, je ne ferai pas de commentaire sur la question. Titus, qui était lui-même tenu de griller ses clopes à l’extérieur pour ne pas pourrir l’atmosphère en plus de nuire à la santé de sa femme et ses enfants, a proposé qu’on aille se jeter quelques verres de Monte Negro (un grogue cap-verdien qu’on venait de lui faire découvrir et sur lequel il ne tarissait pas d’éloges) dans le jardin, histoire de fignoler les derniers préparatifs de notre expédition.

Le plan était simple et juteux : j’allais reprendre le volant de l’Alfa, avec son précieux chargement de merde ecclésiastique dans le coffre, et me rendre dans un no man’s land tellement pourri que même les rats et les cafards l’avaient déserté depuis longtemps. Je connaissais un terrain vague, parsemé de carcasses de voitures désossées, situé aux confins d’une zone industrielle abandonnée, livrée à la poussière, aux ronces, chiendent et détritus en tout genre, qui ferait parfaitement l’affaire. De son côté, avec son propre véhicule (un break 307 SW), Titus me collerait au train jusqu’à destination, ce qui lui permettrait non seulement de me donner un coup de main pour le gros œuvre, mais aussi de me ramener à bon port une fois la sale besogne accomplie. Par «~sale besogne~» j’entendais : mise à feu de Giulietta en vue de la destruction totale dudit véhicule et son contenu par les flammes. Ainsi, sa carcasse calcinée jusqu’à l’os viendrait s’ajouter à celles déjà présentes sur les lieux.

Le voyage s’est déroulé sans encombre, et une fois sur place j’ai ouvert le coffre pour vérifier que tout se passait pour le mieux dans le plus infect des mondes. D’autant que Titus, histoire de lui rendre un dernier hommage avant son départ pour l’au-delà, tenait à voir le macchabée.

Imaginez ma surprise en découvrant que le macchabée en question ne l’était pas tout à fait, en dépit des trois balles que je lui avais logées dans le buffet. Ratatiné dans le coffre, il nous dévisageait avec des yeux injectés de sang en tentant de nous agripper avec ses doigts crochus qui s’agitaient vainement dans le vide. D’autre part, des gargouillis sortaient de sa bouche en même temps que des bulles de bave ensanglantée. Même pour quelqu’un comme moi, habitué à voir des trucs horribles, c’était assez désagréable à regarder.

Titus a dit : Je croyais que tu l’avais buté.

Ce à quoi j’ai répondu, de bonne foi : Moi aussi.

\textsc{Lui} : Qu’est-ce qu’on fait, maintenant ?

\textsc{Moi} : On referme le coffre et on le fait cramer vivant.

En entendant ces mots, les gargouillis ont redoublé d’intensité dans le gosier du moine et ses yeux se sont mis à danser la salsa dans le fond de leurs orbites. Manifestement, brûler vif dans le coffre d’une voiture ne correspondait pas du tout à l’idée qu’il se faisait de la mort.

Titus semblait sensiblement du même avis.

Il a dit : C’est barbare, non ?

J’ai répondu : Oui, mais je te rappelle que les curés ne se sont pas gênés pour cramer des tas de pauvres filles sous prétexte que c'étaient des sorcières. Il serait peut-être temps de leur rendre un peu hommage, non, tu ne crois pas ?

\textsc{Lui} : Aux sorcières ?

\textsc{Moi} : Oui, aux sorcières et toutes les autres victimes de l’Inquisition.

\textsc{Lui} : Quand même, ça se fait pas de brûler un moine.

\textsc{Moi} : D’ailleurs, à propos de sorcières, tu ne trouves pas un peu bizarre qu’il soit encore en vie avec trois balles dans le corps ?

\textsc{Lui} : Maintenant que tu le dis.

\textsc{Moi} : La dernière fois que j’ai ouvert le coffre, il était bel et bien mort, je peux te l’assurer. Et maintenant, il ne l’est plus.

Titus s’est signé à plusieurs reprises avant de s’enfuir à toutes jambes en agitant les bras et poussant des cris dans une langue que je ne connaissais pas.

J’ai hurlé : Titus, reviens ici tout de suite !!!

Il s’est ramené, penaud, et a dit : Tu crois que c’est un mort-vivant ?

Voilà ce qui arrive quand on s’enquille non-stop les onze saisons de The Walking Dead sans boire autre chose que de la bière ni bouffer autre chose que des poignées de chips. J’entends d’ici les mauvaises langues insinuer que je présente Titus, qui est une personne de couleur (noire et l’occurrence, le jaune étant pris par les Asiatiques, le rouge par les Indiens, le vert par les Martiens, le bleu par les Schtroumpfs et les Na’vis, et le blanc par les Aryens et autres Indo-européens suprémacistes occidentaux), comme un sauvage craintif et superstitieux, un colosse aux pieds d’argile bête comme ses pieds, un singe à peine descendu de son cocotier, persuadé qu’un type peut revenir à la vie après avoir reçu trois balles dans le buffet et séjourné pendant des heures dans le coffre d’une Alfa Romeo (et ce n’est pas une question de marque, le résultat aurait été le même dans le coffre d’une Fiat ou une BM). C’est d’autant plus ridicule que non seulement Titus est un ami d’enfance, preuve que tout petit déjà je ne m’arrêtais pas à des considérations de ce genre, et ensuite parce que j’ai toujours été un fervent défenseur des droits de l’homme et du citoyen, y compris la femme, et, dans une moindre mesure, l’animal quel qu’il soit, à part peut-être quelques espèces profondément nuisibles dont il ne viendrait à l’idée de personne de sauver l’existence.

Voilà pourquoi, après avoir posé une main virile et paternelle sur l’épaule d’un Titus claquant des dents et flageolant des genoux à la seule vue du moine qui agitait ses longs doigts crochus en roulant des yeux dans le coffre, je lui ai solennellement déclaré, d’une voix calme et posée, aussi noire et profonde qu’une pinte de Guinness dans un pub de Kilmore Quay (ah le cri des fous de Bassan le soir au coin du feu) : Titus, mon ami, comme les vampires et les loups-garous, les morts-vivants n’existent pas. Par contre, le corps humain est capable de bien des choses et on est encore loin d’être au courant de tout. Je pensais avoir tué cet abruti, mais il n’était pas tout à fait mort et il s’est réveillé pendant le trajet. Voilà tout.

\textsc{Lui} : Tu crois ?

\textsc{Moi} : J’en suis sûr.

\textsc{Titus} : N’empêche qu’on ne peut tout de même pas le brûler vif !

\textsc{Moi} : Non, bien sûr, tu as raison. Il faut l’achever avant de procéder à l’incinération.

Je lui ai tendu le Glock : Tu veux t’en charger ?

\textsc{Lui} : Non, merci. Il faut leur tirer une balle dans la tête, sinon on n’est pas sûr qu’ils sont morts.

\textsc{Moi} : Qui ça, ils ?

\textsc{Lui} : Les morts-vivants.

\textsc{Moi} : Ça n’est pas un mort-vivant, Titus. C’est juste un vivant qui n’est pas encore tout à fait mort.

Pendant qu’on discutait de savoir s’il était mort ou vivant, ou les deux à la fois, le dénommé Dardariel se vidait de son sang dans le coffre de l’Alfa, ce qui ne prêtait pas à conséquence puisque personne n’aurait jamais à le nettoyer. Il était de mon devoir d’être humain et responsable de mettre un terme aux souffrances dudit Dardariel. J’ai approché le Glock de sa tête, positionné le canon contre sa tempe, puis, sans me soucier de ses yeux hagards qui tentaient désespérément de me faire comprendre que ce que je m’apprêtais à faire était un acte ignoble totalement indigne de l’espèce dont j’étais en principe un représentant légal, après quelques secondes qui n’étaient pas d’hésitation mais simplement le fait de quelqu’un qui est ennemi de la précipitation et tient à ajuster son coup au millimètre près, j’ai appuyé sur la détente.

Tandis que sa cervelle giclait joyeusement aux alentours, les yeux de Dardariel se sont révulsés une dernière fois dans leurs orbites, comme des rats pris au piège, puis ses paupières se sont closes définitivement sur un monde qui tournerait largement aussi bien, sinon mieux, sans lui. Cela dit, compte tenu de sa conduite parmi nous, il avait du souci à se faire concernant l’accueil que les autorités célestes allaient lui réserver (Saint Pierre allait être furax et l’envoyer moisir au Purgatoire pendant les cent cinquante prochaines générations). Moi aussi j’en avais, du souci à me faire, sauf que je savais pertinemment que non seulement il n’y avait pas de vie après la mort, mais qu’il n’y en avait pas non plus avant, ou alors si peu que ça ne valait même pas la peine d’en parler.

On avait, en passant, rempli quelques bidons d’essence et fait le plein de l’Alfa à une station-service. Un réservoir plein à craquer nous assurait une crémation réussie, dans les règles de l’art. Les bidons en question se trouvaient dans le coffre de la 307. On a attendu quelques instants, histoire d’être sûrs que Dardariel n’allait pas se réveiller encore une fois, sortir du coffre en titubant, avec ses yeux injectés de sang, son crâne explosé et sa cervelle qui lui sortait par les oreilles et les trous de nez, et se jeter sur nous pour nous dévorer les entrailles et nous transformer en zombies. J’ai profité de l’occasion pour rallumer le Hiram \& Solomon qui croupissait entre mes dents depuis un certain temps. Titus, en homme prévoyant qui avait le sens des voyages et de la distraction (il n’avait pas son pareil pour organiser les pique-niques, les excursions en montagne et les anniversaires pour les gosses avec plein de bonbons et de ballons multicolores, autant de choses dont la seule évocation suffisait à me donner des sueurs froides), avait pensé à amener le fond de la bouteille de Monte Negro, de sorte qu’on pouvait se rincer le gosier à la fraîche tout en gardant un œil sur le macchabée. Pour un peu, ça aurait été un moment de poésie à en chialer des larmes de sang.

Je sentais que l’émotion n’était pas loin de nous submerger tel un raz de marée venu des Antipodes, aussi me suis-je dit à l’intérieur de moi-même qu’il était peut-être temps de songer à allumer le barbecue.

Ne pouvant garder une idée aussi brillante pour moi, je m’en suis ouvert à Titus, qui n’a rien trouvé de mieux à faire que pondre la réplique suivante : je me sentirais plus tranquille si tu lui tirais encore une ou deux balles dans la tête.

\textsc{Moi} : Si ça peut te faire plaisir.

Le mort était déjà mort et re-mort, mais peu importe, je lui ai logé quelques dizaines de grammes de plomb supplémentaires dans la cervelle. Il n’a pas bronché d’un millimètre, à part que ce qui avait été autrefois sa tête ressemblait maintenant à un steak haché baignant dans une sauce pas très ragoûtante.

Je me suis demandé si les comportements irrationnels dont il nous arrivait de faire preuve n’expliquaient pas en grande partie le parcours chaotique de notre espèce sur terre, lequel parcours, à force de cahoter, risquait de finir par se casser la gueule pour de bon. C’était le genre de fulgurance molle du genou qui me traversait parfois l’esprit, au plus fort de la nuit, surtout quand je m’apprêtais à faire cramer quelqu’un dans le coffre d’une bagnole. Il y a soixante-six millions d’années (à quelques mois près), un caillou (un gros, très gros caillou, d’une dizaine de kilomètres de diamètre) en provenance de Jupiter s’est écrasé sur le Yucatan, creusant un trou de cent kilomètres de diamètre par trente de profondeur, de quoi envoyer assez de cochonneries dans l’atmosphère pour soumettre la planète à une série d’évènements climatiques dévastateurs tels que : effet de gril dans un premier temps, avec flambée des températures et retombées de particules incandescentes qui crament tout sur leur passage, puis formation d’un écran de poussières toxiques en haute atmosphère qui plonge la planète dans le froid et les ténèbres pendant des décennies, le tout suivi d’une remontée brutale des températures due à la présence massive de gaz à effet de serre.

En gros, tout ce qui vit à la surface passe à la trappe, à commencer par nos bons gros vieux dinosaures qui s’ébrouaient tranquillement à la surface du globe. Voilà comment, alors que ça n’a aucun rapport direct avec l’histoire qui nous occupe, il m’arrive de songer à ce qui serait advenu si l’astéroïde avait raté sa cible et l’extinction K-T n’avait jamais eu lieu. Peut-être que les dinosaures seraient toujours parmi nous et qu’on pourrait aujourd’hui se balader avec un Tricératops ou un T-Rex miniature en laisse en guise d’animal de compagnie. On se demande aussi pourquoi Asteriornis maastrichtensis, sorte de poulet préhistorique qui hantait les basses-cours du Crétacé, ou encore Vegavis iaai, sympathique canard dont les restes vieux de soixante-dix millions d’années ont été retrouvés au nord de la terre de Graham, dans l’Antarctique, ont survécu alors qu’ils n’avaient à priori aucun endroit où se cacher, contrairement à celles et ceux qui pouvaient se réfugier dans les profondeurs de la terre ou la mer. Ce séisme a-t-il eu une influence directe sur l’Évolution au point que sans lui, par exemple, l’espèce humaine n’aurait jamais vu le jour ? Aux dernières nouvelles, après Morotopithecus bishopi en Ouganda et Nyanzapithecus alesi au Kenya, Archicebus achilles, un singe chinois de cinquante-cinq millions d’années à peine plus gros qu’une souris, est ce qui se fait de plus vieux en matière de primate. Peut-être que toute cette joyeuse bande de primates n’aurait jamais vu le jour sans l’astéroïde de Chicxulub.

Ce qui pose la question suivante : l’Évolution a-t-elle un plan de carrière, prévoyant à court ou moyen terme l’émergence de telle ou telle espèce ? Est-ce que, quoi qu’il arrive, nous devons nécessairement passer par ici pour en arriver là ? Fascinant, non ? Oui, mais aussi complètement chiant, raison pour laquelle je vous propose de retourner sans plus tarder à nos moutons, lesquels, je le rappelle en passant, descendent tous du Mouflon oriental, lui-même descendant du Mouflon chinois Ovis shantungensis, ce qui ne nous rajeunit pas.

J’ai dit, avec une légère pointe d’agacement dans la voix : C’est bon, on peut y aller, maintenant ?

\textsc{Titus} : Oui, c’est bon.

On a récupéré les bidons d’essence et arrosé la caisse de fond en comble, des jantes au toit ouvrant en passant par la boîte à gants et la banquette arrière, avec une attention toute particulière pour Dardariel qui, en plus de son sang et sa cervelle, baignait littéralement dans le carburant. Il restait quelques allumettes extra longues dans la boîte qui me servait à allumer mes cigares. J’en ai craqué quelques-unes que j’ai dispatchées un peu partout dans le véhicule, lequel s’est aussitôt mis à flamber dans une ambiance pestilentielle de fin du monde.

Il ne nous restait plus qu’à prendre congé avant que le cocktail Molotov nous explose à la gueule.

Adieu Giulietta, belle italienne aux formes arrondies et courbes affriolantes, telle une Gina Lollobrigida\nf{Gina Lollobrigida (1927--2023), actrice et photographe italienne, icône du cinéma des années 1950--1960. \textit{La Proie des vautours} (\textit{Buona Sera, Mrs. Campbell}, 1968, réal. Melvin Frank) est l'une de ses dernières grandes comédies, où elle joue une Italienne qui a prétendu à trois Américains que chacun est le père de sa fille. \source{fr.wikipedia.org/wiki/Gina\_Lollobrigida}} dans La Proie des vautours ou une Sophia Loren\nf{Sophia Loren (née en 1934), actrice italienne, première comédienne non anglophone à remporter l'Oscar de la meilleure actrice (\textit{La Ciociara}, 1961). \textit{La Comtesse de Hong-Kong} (\textit{A Countess from Hong Kong}, 1967) fut le dernier film de Charlie Chaplin comme réalisateur, dans lequel elle donna la réplique à Marlon Brando. \source{fr.wikipedia.org/wiki/Sophia\_Loren}} dans La Comtesse de Hong-Kong (cette dernière référence permet de boucler en douceur la boucle avec Ovis shantungensis, le mouflon chinois dont je parlais précédemment, même s’il n’est bien évidemment pas question de comparer Sophia Loren à une quelconque sorte de ruminant).

