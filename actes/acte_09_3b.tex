
Par contre, le docteur Sébastien Charrier, lui, était bien là.

Je lui ai dit : Docteur Charrier ! Si je m’attendais à vous trouver ici !

\textsc{Charrier} : On se connaît ?

\textsc{Moi} : Vous ne vous rappelez pas ?

\textsc{Charrier} : Me rappeler de quoi ? On s’est déjà vu quelque part ?

\textsc{Moi} : L’affaire Alena Benesch, ça vous dit quelque chose ?

\textsc{Charrier} : Attendez voir… Mais oui, bien sûr, cette petite pute qui avait essayé de me faire chanter en prétendant que je lui avais fait bouffer la chatte par Hermann, mon dogue allemand !

\textsc{Moi} : Oui. C’est moi qui étais chargé de l’enquête. Il va bien, au fait ?

\textsc{Lui} : Qui ? Hermann ?

\textsc{Moi} : Oui.

\textsc{Lui} : La pauvre bête est morte de chagrin il y a quelques années de cela. Je crois qu’elle s’était beaucoup attachée à cette petite. Le coup de foudre existe aussi chez les animaux, vous savez.

\textsc{Moi} : J’ignorais.

\textsc{Lui} : Oui, je sais, on pense toujours que ce ne sont que des brutes épaisses incapables de sentiment. Eh bien il n’en est rien, ils sont beaucoup plus sensibles qu’on ne le pense.

\textsc{Moi} : Je suis vraiment désolé.

\textsc{Lui} : C’est gentil à vous. Je l’ai enterré au fond du jardin et vais quotidiennement me recueillir sur sa tombe.

\textsc{Moi} : C’est l’avantage d’avoir une grande propriété.

\textsc{Lui} : Vous connaissez la Frétoise ?

\textsc{Moi} : J’y suis allé une fois ou deux. Très bel endroit, à la fois authentique et élégant.

Lui, manifestement ému : Oui, un joyau historique rénové avec passion au cœur d’un environnement préservé. Nous sommes actuellement en train d’aménager le colombier pour y faire une chambre d’ami. C’est une bonne idée, vous ne trouvez pas ?

\textsc{Moi} : Excellente

\textsc{Lui} : Il faudra venir dîner un de ces soirs. Vous êtes marié ?

\textsc{Moi} : Non, pas encore.

\textsc{Lui} : Une fiancée, alors. Ravissante, je suppose. Il faudra penser à nous l’amener.

\textsc{Moi} : Je n’y manquerai pas.

\textsc{Lui} : Quoiqu’il en soit, pour en revenir à cette petite garce d’Alena Benesch, je continue de penser qu’elle n’a eu que ce qu’elle méritait. Il m’est arrivé de penser que j’avais peut-être été un peu trop dur avec elle. Je ne suis pas un mauvais homme, vous savez, et toujours de mon mieux pour réparer mes torts. Si torts il y a, bien entendu. La vérité, c’est que je ne me sens coupable de rien en ce qui la concerne. Toujours est-il que l’autre jour, croyez-le ou non, elle est venue sonner à ma porte en disant qu’elle était dans une misère noire et avait besoin d’un petit coup de main.

\textsc{Moi} : Et qu’est-ce que vous avez fait ?

\textsc{Lui} : Ça reste entre nous, mais ma chère épouse, qui est une vraie salope soit dit en passant, adore me regarder baiser avec une autre femme. On a tous nos petites manies, n’est-ce pas. Alena Benesch a pris un peu de poids, c’est vrai, mais elle est encore tout à fait comestible. Je lui ai proposé de la sodomiser sous les yeux de ma femme, si elle n’avait rien contre le fait de se faire défoncer le cul par un ancien interne des hôpitaux de Paris. En échange, je pourrais essayer de faire jouer mes relations pour qu’elle retrouve un poste dans une clinique privée. J’ai un ami qui adore les femmes plutôt bien en chair, un violeur notoire qui a pour habitude d’abuser de ses patientes quand elles sont dans les vapes. Je ne sais que ça ne se fait pas, mais c’est un ami et je ne me vois pas le balancer aux flics. D’autant que la plupart d’entre elles ne se souviennent de rien à leur réveil. Je vous choque ?

\textsc{Moi} : Un peu, oui. Et celles qui se souviennent, je peux savoir ce que vous en faites ?

\textsc{Lui} : On les envoie chez le psy.

\textsc{Moi} : Un ami à vous, je suppose ?

\textsc{Lui} : Evidemment. Comme elles n’ont que de très vagues souvenirs sur lesquels elles sont incapables de mettre un nom ou un visage, le psy leur explique qu’elles sont en pleine bouffée délirante, sans doute liée à l’un ou l’autre de ces traumas d’enfance mal gérés, ou pas gérés du tout parce que totalement passés sous les radars, qui refont surface après des années d’enfouissement, comme des saletés de zombies qui refoulent du bec et tentent de vous bouffer tout cru. Mais dieu merci, on n’est plus au Moyen Âge. De nos jours, on peut être cinglé sans se retrouver en train de griller sur un bûcher. Les chercheurs bossent comme des dingues, pour des salaires de misère, et nous disposons de molécules de plus en plus sophistiquées pour remettre un peu d’ordre dans les cerveaux détraqués.

\textsc{Moi} : N’empêche qu’il abuse d’elles.

\textsc{Lui} : Oui, on peut dire ça. Mais à ce moment-là, on peut aussi dire qu’il les viole avec ses doigts en procédant aux examens d’usage.

\textsc{Moi} : Oui, enfin, ce n’est pas tout à fait la même chose. Là, les viole carrément avec sa bite.

\textsc{Lui} : Je sais, c’est moche. Très moche, même, mais il prétend avoir un meilleur diagnostic avec sa verge qu’avec les autres outils dont il dispose, trop grossiers à son goût. Je sais que c’est faux, qu’il ment, qu’il se ment à lui-même, mais il n’est pire sourd que celui qui ne veut rien entendre. J’ai beau essayer de lui ouvrir les yeux, il s’obstine dans le déni. Je lui dis : Roman (il s’appelle Roman), mon ami, je t’en supplie, va voir un psy. Un jour ou l’autre, une patiente va se réveiller pendant que tu es en train de l’ausculter avec ta bite, en tout bien tout honneur, et il va en résulter un de ces putains de scandales qui éclaboussent la profession toute entière. Pense à tes collègues, tous ces gens qui se battent becs et ongles pour que les gens se bourrent de médocs jusqu’à cent ans. Il me répond : oui, je ferais bien quelques séances d’hypnose, mais j’ai peur de me faire violer pendant mon sommeil. Je ne fais aucune confiance à tous ces enfoirés de psys ! Vous le voyez, on n’en sort pas. À propos, vous êtes toujours dans la police ?

\textsc{Moi} : Oui, plus ou moins.

\textsc{Lui} : Dans ce cas, je compte sur votre discrétion. Vous savez ce que c’est : les gens sont méchants, ils n’aiment pas les riches. Dès que vous gagnez un peu plus de pognon qu’eux, ils font tout ce qui est en leur pouvoir pour vous mettre des bâtons dans les roues. Heureusement qu’ils n’en ont aucun, sinon celui de descendre dans la rue pour agiter des banderoles et se gargariser de slogans anticapitalistes, sans quoi ils seraient pires que tous ces dictateurs qui dirigent le monde d’une main de fer. Je peux savoir ce qui s’est passé, ici ?

\textsc{Moi} : Accident de la circulation.

\textsc{Lui} : Belle boucherie !

\textsc{Moi} : Oui. Vous pensez qu’elle s’en sortira ?

\textsc{Lui} : Contusions multiples, hémorragie interne au niveau de la cavité abdominale, possible trauma crânien, j’en passe et des meilleurs. Elle est dans la coma. Nul ne peut dire quand elle en sortira, si elle en sort, et encore moins dans quel état. Vous lui vouliez quoi, à cette petite ?

\textsc{Moi} : L’interroger. J’ai de bonne raison de penser qu’elle est impliquée dans la disparition d’un collègue de travail, qui se trouve aussi être un de mes plus proches amis.

\textsc{Lui} : J’ai moi-même perdu un excellent ami.

\textsc{Moi} : Vraiment ?

\textsc{Lui} : Oui, très récemment. Il s’appelait Rémi Durand et était gynécologue.

\textsc{Moi} : Ce nom me dit quelque chose.

\textsc{Lui} : Bien évidemment, que ça vous dit quelque chose. Si vous avez enquêté sur la Frétoise, vous n’ignorez pas que Rémi y passait quasiment tous ses week-ends, plus une bonne partie des vacances scolaires et ses congés de maternité.

\textsc{Moi} : Je vois. Qu’est-ce qui s’est passé, au juste ?

\textsc{Lui} : On l’a retrouvé pendu dans son garage.

\textsc{Moi} : Pendu ???!!!!!!!!!!

\textsc{Lui} : Dans son garage, oui.

\textsc{Moi} : D’après mon expérience personnelle, qui est tout de même loin d’être négligeable, il est assez rare que les gens se pendent dans leur garage, ou alors seulement s’ils sont victimes de harcèlement sur les réseaux sociaux ou viennent d’apprendre fortuitement qu’ils sont atteints d’une maladie grave qui ne leur laisse que quelques heures à vivre. Ils ont généralement bien trop de respect pour leur voiture pour lui imposer une telle humiliation.

\textsc{Lui} : N’est-ce pas. Au lieu de ça, Rémi était en pleine forme et venait tout juste de s’offrir une Porsche Cayman GT4 RS dont il était extrêmement fier.

\textsc{Moi} : C’est troublant, en effet.

\textsc{Lui} : Sincèrement, mon ami, vous pensez vraiment qu’un type qui a une Porsche Cayman GT4 RS dans son garage a la moindre envie de mettre fin à ses jours ?

\textsc{Moi} : Quelle couleur, la Cayman GT4 RS ?

\textsc{Lui} : Jaune, avec pack Clubsport, réservoir de 90 litres et trousse de premiers secours en alcantara !

\textsc{Moi} : Une merveille.

\textsc{Lui} : Absolue ! Vous êtes comme moi, n’est-ce pas ?

\textsc{Moi} : Comment ça ?

\textsc{Lui} : Vous ne croyez pas un instant à la thèse du suicide.

\textsc{Moi} : Le suicide d’un gynéco bien dans sa peau qui vient de s’offrir une Porsche Cayman GT4 RS jaune avec pack Clubsport, réservoir de 90 litres et trousse de premiers secours ?

\textsc{Lui} : Oui, et toutes les options disponibles, l’intérieur full cuir bien évidemment, mais aussi les rétros extérieurs à capteurs de pluie, la reconnaissance des panneaux de signalisation, les tapis de sol en peau de fesse, les jantes en or massif, le système audio Bose et les ceintures de sécurité couleur chair !

\textsc{Moi} : Terrifiant ! J’ai beau tourner et retourner le problème dans tous les sens, je ne vois pas comment un type qui a tout ça dans son garage pourrait avoir la moindre envie de se suicider.

\textsc{Lui} : Et imaginez que ce même type soit marié à une jeune femme somptueuse dont il pourrait être le père ! Vous croyez vraiment qu’un tel homme aurait envie de mettre fin à ses jours ?

\textsc{Moi} : En aucun cas. Vous avez contacté le procureur ?

\textsc{Lui} : Bien évidemment, vous me prenez pour qui ! J’ai exigé qu’une expertise médico-légale soit diligentée dans les plus brefs délais, et j’ai demandé à y assister personnellement.

Moi, sortant un Hemingway Short Story de ma poche : Vous avez raison, on n’est jamais mieux servi que par soi-même. Vous fumez ?

\textsc{Lui} : Vous avez enquêté sur moi, non ?

\textsc{Moi} : Un peu, oui. Enquête de routine, vous savez ce que c’est.

\textsc{Lui} : Dans ce cas, vous devez savoir que je ne crache pas sur un petit cigare de temps à autre.

\textsc{Moi} : Et c’est tout à votre honneur. Tu en veux un aussi, Greg ?

\textsc{Greg} : Je ne voudrais surtout pas déranger.

\textsc{Moi} : Mais pas du tout, voyons, qu’est-ce que tu vas imaginer. C’est juste que comme je sais que tu ne fumes pas, ou quasiment pas, je me suis dit qu’il n’était peut-être pas complètement indispensable de te proposer un cigare.

\textsc{Greg} : Je ne fume pas en temps normal, c’est tout à fait vrai. Mais aujourd’hui, les circonstances sont suffisamment exceptionnelles pour que je fasse une exception à la règle.

\textsc{Moi} : C’est la raison pour laquelle, en dépit des éléments dont je viens de te faire part, que je me suis permis de te demander si tu voulais toi aussi un cigare.

\textsc{Greg} : Dans ce cas, je te répondrai ceci : oui, Djef, même s’il est vrai que je ne fume pas ou quasiment pas, c’est avec le plus grand plaisir que je vais accepter le cigare que tu m’offres si gentiment.

\textsc{Moi} : Bien. On en était où, nous ?

\textsc{Charrier} : Vous veniez de me proposer un cigare.

\textsc{Moi} : Que vous aviez accepté, c’est bien ça ?

\textsc{Lui} : C’est bien ça.

\textsc{Moi} : Et avant ?

\textsc{Lui} : Avant, on était en train de parler de la mort suspecte de mon ami Rémi Durand, gynécologue ayant pignon sur rue qui n’avait aucune raison de mettre fin à ses jours, et ce d’autant moins qu’il était marié à très jolie fille beaucoup plus jeune que lui et venait tout juste de s’offrir une Porsche Cayman GT4 RS jaune avec pack Clubsport, réservoir de 90 litres et trousse de premiers secours en alcantara, le nec plus ultra en matière de chic automobile.

\textsc{Moi} : Vous pensez qu’on l’a tué ?

\textsc{Lui} : Je ne vois pas d’autre explication.

\textsc{Moi} : Il s’agit peut-être d’un accident.

\textsc{Lui} : Vous plaisantez ?

\textsc{Moi} : On ne sait jamais. Imaginez un type qui décide d’aller dans son garage pour bricoler un truc au plafond, accrocher une corde, par exemple. Il monte sur un tabouret, et, pour une raison ou pour une autre, perd l’équilibre et se retrouve avec la corde enroulée autour du cou. Il se débat, tente désespérément de se raccrocher au tabouret. Mais il donne un coup de pied dans le tabouret, le tabouret tombe, et notre homme se retrouve pris au piège.

\textsc{Lui} : Ridicule !

\textsc{Moi} : Ou alors, il a peut-être eu, pendant un bref instant, l’idée de mettre fin à ses jours. On croit connaître ses amis, et on découvre parfois qu’ils nous ont caché des choses pendant des années. François Vérove, alias «~le Grêlé~», a vécu pendant trente-cinq sans attirer les soupçons. Ancien gendarme, bon père de famille, qu’est-ce que vous croyez qu’il s’est passé quand ses proches ont appris qu’il s’agissait en fait d’un tueur en série pédophile de la pire espèce, responsable d’au moins une bonne demi-douzaine de meurtres, et sans doute beaucoup plus, la liste exacte de ses victimes n’ayant jamais pu être établie avec certitude ?

\textsc{Lui} : Il sont tombés des nues, je suppose.

\textsc{Moi} : Peut-être que votre ami Rémi vous cachait des choses, lui aussi. Vous seriez surpris d’apprendre le nombre de gens qui ont une double vie.

\textsc{Lui} : Vous insinuez que Rémi était un tueur en série pédophile ?

\textsc{Moi} : Pas le moins du monde. Enfin, on ne sait jamais. Vous m’avez dit qu’il avait une femme beaucoup plus jeune que lui. Peut-être qu’elle le trompait et qu’il ne l’a pas supporté.

\textsc{Lui} : Elle ne l’a jamais trompé. Elle couchait avec d’autres hommes, c’est vrai, mais Rémi était parfaitement au courant et couchait lui aussi avec d’autres femmes, en toute transparence.

\textsc{Moi} : La vôtre, par exemple.

\textsc{Lui} : Par exemple. Mais j’ai souvent couché avec la sienne. J’ai toujours considéré comme une chose parfaitement normale que mes amis couchent avec ma femme, et m’autorisent à faire de même avec la leur. Je ne sais pas si vous êtes marié, lieutenant…

\textsc{Moi} : Commandant.

\textsc{Lui} : … commandant, mais si vous l’êtes vous devez savoir à quel point il est rébarbatif de coucher toujours avec la même femme, quel soit l’amour qu’on lui porte. L’amour et le sexe sont deux choses totalement différentes, qui ne devraient rien avoir à faire ensemble. Le désir est une chose, l’amour en est une autre, et la confusion qui règne entre les deux est la source de nombreux problèmes. Vous pouvez vivre avec quelqu’un toute votre vie, et continuer à l’aimer, mais certainement pas à le désirer comme au premier jour. Si certains y arrivent, tant mieux pour eux, mais moi ce n’est pas mon cas. Je ne vois pas au nom de quoi je devrais m’interdire d’être attiré par d’autres femmes et de coucher avec elles si le désir est réciproque. Quel spectacle pathétique de voir tous ces vieux types, mariés depuis des siècles à une femme qui ne ressemble physiquement plus à rien, baver comme des malades sur le cul des gamines qui passent à leur portée. Mais vous savez ce qui me fait le plus marrer ?

\textsc{Moi} : Non, dites-moi.

\textsc{Lui} : C’est de voir les jeunes mariés errer dans les rayons des supermarchés.

\textsc{Moi} : Mais encore ?

\textsc{Lui} : Madame marche en tête, et monsieur suit, l’air grognon et la mine déconfite, poussant un caddie rempli jusqu’aux ouïes de couches-culottes, lait en poudre et toute une ribambelle de produits ultra transformés bourrés d’huile de palme hydrogénée, amidon modifié, agents de texture, isolats de protéines et autres perturbateurs endocriniens diversement cancérigènes. Il a pourtant toutes les raisons d’être heureux, le jeune père de famille : il vient d’acheter un pavillon avec un petit lopin de terre pour faire pousser deux patates et trois petits pois, de changer de bagnole et d’accéder aux joies de la paternité. Seulement voilà, madame vient de prendre vingt kilos en neuf mois et il sait pertinemment qu’elle ne réussira jamais à s’en débarrasser. D’autant qu’elle ne fera pas le moindre effort pour ça, pour la bonne et simple raison qu’elle veut un autre enfant et part du principe que c’est pas la peine de suer sang et eau pour perdre vingt kilos si c’est pour en reprendre trente dans la foulée. Et elle se dit aussi que si son mari l’aime, il aura toujours autant envie d’elle même si elle ressemble à un éléphant de mer. Du coup, elle va garder ses vingt kilos et en reprendre une bonne vingtaine de plus pendant sa prochaine grossesse, perdant définitivement toute chance de retrouver un jour sa taille d’antan. Résultat des courses : ils vont commencer à s’engueuler, divorcer dans un an, deux ou trois si tout va bien, et madame va aller s’inscrire à la salle de sport du coin dans l’espoir de trouver un nouveau pigeon pour lui témoigner un peu d’affection. Vous ne trouvez pas que ça fait froid dans le dos ?

\textsc{Moi} : Vu comme ça, ce n’est effectivement pas très engageant.

\textsc{Lui} : Franchement répugnant, vous voulez dire !

\textsc{Moi} : J’aimerais qu’on en revienne à Rémi. Vous n’avez rien remarqué de bizarre pendant les jours ou les semaines qui ont précédé son décès ?

\textsc{Lui} : Non, rien du tout. Je vous le répète, Rémi allait parfaitement bien et n’avait aucune raison de mettre fin à ses jours. Il est évident que quelqu’un l’a tué en essayant de faire passer le crime pour un accident, et je ne doute pas que l’analyse médico-légale le confirmera.

\textsc{Moi} : Vous lui connaissez des ennemis ?

\textsc{Lui} : Les riches ont des tas d’ennemis.

\textsc{Moi} : Vous, peut-être ?

\textsc{Lui} : Moi ? Vous êtes fou !

\textsc{Moi} : Vous m’avez dit qu’il couchait avec votre femme. On a vu des gens en tuer d’autres pour moins que ça.

\textsc{Lui} : Je vous ai dit aussi que je m’en foutais, et que je couchais aussi avec la sienne. Il nous arrivait aussi de coucher tous ensemble, si vous voulez tout savoir.

\textsc{Moi} : Drôles de pratiques.

\textsc{Lui} : Je ne vous demande pas de vous joindre à nous.

\textsc{Moi} : Je peux vous poser une question ?

\textsc{Lui} : Si vous y tenez. Au fait, je ne sais pas si je vous l’ai dit, mais ce cigare est excellent. Il vient de Cuba, je suppose.

\textsc{Moi} : Non, de République dominicaine. Vous avez entendu parler d’Arturo Fuente ?

\textsc{Lui} : Non.

\textsc{Moi} : C’est un de ces émigrés espagnols qui ont fui Cuba pendant la guerre hispano-américaine. Il a atterri à Tampa, en Floride, et s’est lancé dans la fabrication de cigares avec des feuilles en provenance de Cuba. Une production d’abord confidentielle, limitée à quelques milliers de cigares par an roulés dans le salon et la cuisine par les membres de la famille. Ensuite, quand ça commençait à plutôt bien marcher, il y a eu le Che, Castro et la révolution cubaine, avec pour conséquence la rupture des relations diplomatiques entre Cuba et les USA. Du coup, la manne cubaine s’est tarie. Fuente a donc commencé à se fournir au Mexique et à Porto Rico, avec un succès mitigé, avant de changer son fusil d’épaule et aller s’installer au Nicaragua, nouvel Eldorado du cigare et dictature bananière sous contrôle américain. Mais à la fin des années 70, la révolution sandiniste a éclaté, les Somoza ont été foutus à la porte, et la fabrique Fuente, emblème d’une époque révolue, a été entièrement détruite par les flammes. Nouvel exil à Santiago, en République dominicaine, avec pour tout bagage un solide savoir-faire et une volonté farouche de tout casser. Aujourd’hui, associée à la famille Newman, Fuente produit des dizaines de millions de cigares par an, dont quelques uns parmi les plus réputés et onéreux de la planète. Celui que vous êtes en train de fumer, par exemple, le Short Story de la série Hemingway, est un vibrant hommage à l’écrivain qui a passé une bonne partie de sa vie à Cuba et appréciait tout particulièrement les Cohiba, une des plus prestigieuses marques de cigares.

\textsc{Lui} : Les meilleurs, à ce qu’il paraît.

\textsc{Moi} : Les plus chers, en tout cas. Oui, c’est ce que disent les snobs qui fument pour se donner un genre et n’y connaissent rien. C’était sans doute vrai avant l’embargo, et jusqu’à la fin des années 70 ou 80, mais depuis le cigare a fait son chemin un peu partout dans le monde et l’hégémonie cubaine n’est plus d’actualité, notamment en ce qui concerne le rapport qualité-prix. En cause le développement des marchés internationaux comme l’Inde, la Chine et le Moyen-Orient. Le pays n’arrive plus à suivre, avec pour conséquences une tendance à la surproduction, une baisse notoire de la qualité de fabrication et une hausse constante des prix. Les cigares autrefois abordables ont pulvérisé toutes les limites de la décence tarifaire. Le SIGLO VI de Cohiba, par exemple, grand classique s’il en est, se négocie aujourd’hui aux alentours de cent-dix euros pièce, et certaines séries spéciales montent à trois ou quatre cent. D’autre part, même s’il représente toujours dans l’imaginaire collectif la référence absolue en matière de cigare, force est de constater que le havane peine à se renouveler, innover tant sur la plan de la forme que du fond, tandis que les autres rivalisent de créativité pour exciter les papilles du consommateur.

\textsc{Lui} : Si vous le dites.

\textsc{Moi} : Je l’affirme haut et fort et ne cesserai de le clamer jusqu’à mon dernier souffle, n’en déplaise aux crétins prétentieux qui ne jurent que par le havane !

\textsc{Lui} : Grand bien vous fasse.

\textsc{Moi} : Maintenant, si vous le voulez bien, j’aimerais vous poser une petite question. Rien de personnel, rassurez-vous.

\textsc{Lui} : De quoi s’agit-il ?

\textsc{Moi} : De l’individu assis au volant de cette voiture.

Je parlais de Noé Desmarais, qui n’avait pas bougé un cil depuis le début de la conversation.

\textsc{Lui} : Vous voulez savoir s’il est mort, c’est ça ?

\textsc{Moi} : J’aimerais bien, oui.

\textsc{Lui} : C’est un ami à vous ?

\textsc{Moi} : Pas exactement, mais c’est une histoire un peu longue à raconter. Tout ce que je peux vous dire, c’est qu’il arrivait en face quand la Mini a essayé de doubler la Fiesta. Et ça a fait un grand BOUM !

Lui, agitant le truc qui pendait à son cou : Vous savez ce que c’est ?

\textsc{Moi} : Oui, un stéthoscope.

\textsc{Lui} : Mais pas n’importe lequel. C’est un Redmann, la Rolls du stéthoscope. Avec ça, vous pouvez entendre respirer un moucheron et battre le cœur d’un ver de terre, qui en possède cinq soit dit en passant.

\textsc{Moi} : Et alors ?

\textsc{Lui} : Alors cet homme est mort, il n’y a aucun doute là-dessus.

\textsc{Moi} : Vous en êtes sûr ?

\textsc{Lui} : Sûr et certain. Vous n’oseriez tout de même pas mettre en doute mes compétences ?

\textsc{Moi} : Loin de moi cette idée absurde.

\textsc{Lui} : Dans ce cas, vous ne m’en voudrez pas de prendre congé. J’ai encore des tas de vie à sauver qui m’attendent.

FIN

Provisoire, bien entendu.

Il est toujours extrêmement douloureux de mettre un point final à un récit qui a occupé de longs mois de votre existence, pompé une bonne partie de votre énergie et mobilisé toutes vos facultés créatrices. C’est comme dire au revoir à un vieil ami, lui serrer une dernière fois la main sans savoir si on le reverra un jour. On la garde longtemps dans le creux de la sienne, comme un petit animal blessé, on se refuse obstinément à la lâcher. Et puis on rentre chez soi, triste, et on avale une belle assiette de rognons de veau à la crème pour se donner une contenance, tenter d’oublier que toutes les choses ont une fin, les meilleures comme les pires, ce qui est une bonne chose pour les pires mais moins bonne pour les bonnes.

Sous la pression de mes fans, qui commencent à trouver le temps long (vous m’avez manqué, vous aussi), et surtout de mon éditeur qui a grand besoin de renflouer les caisses de sa modeste entreprise (les temps sont durs pour tout le monde, et ce serait pour lui un crève-cœur de devoir vendre son Ferretti Custom Line 97 ou hypothéquer sa villa de Saint-Raphaël pour sauver les meubles), je me vois dans l’obligation de remettre à plus tard un certain nombre des affaires en cours.

Je pense notamment à Jaya, la fille adorée de mon ami Zaahid Shirani, tombée entre les griffes d’un certain Simon Keskula, guide spirituel et maître incontesté d’un secte post-apocalyptique connue sous le nom d’Alliance de la Révélation. J’ai promis à Zaahid de tout mettre en œuvre pour que Jaya rentre au bercail et que le monstre qui la tenait sous sa dépendance soit définitivement mis hors d’état de nuire. Et comme je suis un homme de parole, je vais faire ce que j’ai dit. Et surtout, je ne manquerai pas de vous narrer par le détail comment, par quel stratagème machiavélique, ruse subtile et technique d’infiltration digne des meilleurs services de renseignement, et accessoirement usage immodéré de la force, sinon la violence la plus aveugle et éthiquement condamnable, je serai parvenu à mes fins.

En attendant, je sais qu’une double question vous ronge la cervelle aussi sûrement qu’un rat affamé s’attaque à un morceau de gruyère ou un vieux quignon de pain : Repentance Whittingham, alias la Gardienne de la Nuit ou la femme de ménage la plus rapide du monde, est-elle sortie du coma, et quid de Titus Beaugendre, porté disparu après une rencontre tant fortuite que suspecte avec la demoiselle en question ?

Eh bien… on n’en sait trop rien, à vrai dire, mais je ne manquerai pas de vous le faire savoir si j’apprends quelque chose à ce sujet. Tout ce que je sais pour l’instant, et c’est assez maigre je vous le concède, c’est qu’elle a disparu du jour au lendemain de sa chambre d’hôpital. Et comme je doute fort qu’elle ait été capable de le faire par ses propres moyens, l’action d’un tiers n’est pas à exclure.

Pour ce qui est de Titus, je vous propose un petit flashback juste avant le mot FIN, au moment où le docteur Charrier nous a annoncé que lui et son Redmann, la Rolls du stéthoscope, étaient formels sur le fait que Noé Desmarais ne ferait plus jamais joujou avec les allumettes, ni d’ailleurs avec quoi que ce soit d’autre, l’envie de faire joujou avec quelque objet ou organe que ce soit lui étant définitement passée. Desmarais refroidi, la joyeuse petite bande de néonazis des Disciples de la Colère était en partie démantelée. Ne restait plus qu’à exterminer les sieurs Monteil et Jégou, individus peu recommandables auxquels j’entendais bien réserver un traitement à la hauteur de leurs exploits. Pour ce faire, je m’étais dit que ça pourrait être sympa de transformer ma salle de bain en chambre à gaz. On enlevait Monteil et Jégou, on les obligeait à revêtir un pyjama rayé, on les affamait pendant quelques semaines, puis, quand ils commençaient à sentir si mauvais que même les mouches à merde s’enfuyaient à tire-d’aile à leur approche, on leur offrait une petite douche gratuite au Zyklon B, le célèbre insecticide à base de cyanure de la Deutsche Gesellschaft fur Schadlingsbekampfung (il faut reconnaître que les Allemands ont un certain talent pour créer des mots de quinze kilomètres de long totalement imprononçables pour toute personne non germanique, à tel point que je me demande si ce n’est pas une des raisons principales de leur manque de popularité et relatif isolement sur la scène internationale, outre le fait qu’ils construisent des voitures rapides très appréciées des trafiquants de drogue, boivent beaucoup de bière et sont nuls en cuisine). Après quoi on en faisait des steaks hachés, merguez, saucisses et andouillettes, et on organisait une grande fiesta dans le quartier avec barbecue à gogo jusqu’à épuisement des stocks. On joignait l’utile à l’agréable, et je doute fort que la police s’amuserait à aller fourrer son nez dans les cuvettes de chiottes du voisinage. Je ne connais pas les effets du piment et des épices sur les composés organiques, mais je suppose qu’il est tout à fait possible de rechercher des traces d’ADN dans une merguez ou une chipolata. Si on le faisait plus souvent, quelque chose me dit qu’on pourrait avoir des surprises de taille.

