\textsc{Moi} : Pourquoi j’ai l’impression que vous vous foutez de ma gueule ?

Lui, tout sourire : Je ne me permettrais pas, monsieur le commissaire.

Une des choses qui m’auraient fait le plus plaisir, à ce moment-là, aurait été d’être pilote de chasse, que des touristes français (venus massacrer des lions, des éléphants et des rhinos) soient sauvagement assassinés et décapités par des bushmen dans le désert du Kalahari, et que la guerre soit déclarée à la Namibie par un président de la République aux abois, en totale perdition, disposant d’une cote de popularité frisant le ridicule, suspecté de corruption passive, fraude fiscale, trafic de drogue, proxénétisme aggravé, prise illégale d’intérêts, détournement de fonds publics et association de malfaiteurs, prêt à tout, y compris rayer un pays de la carte, pour redorer son blason aux frais de la princesse. J’aurais quitté les lieux sans dire un mot, rejoint ma base, serais monté dans mon Rafale et revenu larguer tout mon stock de bombes et de missiles sur le Caribbean Hôtel, ne laissant à sa place qu’un cratère fumant jonché de gravats et de cadavres entremêlés.

Greg, que je n’avais pas entendu arriver, s’est pointé derrière moi et m’a donné quelques coups de coude dans le dos pour me signaler sa présence et son intention de s’entretenir avec moi.

\textsc{Moi} : Quoi, qu’est-ce qui se passe ?

Il a alors attiré mon attention, de façon non verbale, en le pointant du doigt, sur un fait qui venait de se produire et pouvait se révéler d’une importance capitale pour la suite de notre enquête : une porte d’ascenseur venait de s’ouvrir, non loin d’une girafe en train de brouter les feuilles en plastique d’un arbre factice, et de cette ouverture venait de surgir ce qui ressemblait comme deux gouttes d’eau à une femme de chambre poussant un chariot adapté à l’exercice de ses fonctions.

Croiser une femme de chambre est chose assez courante dans un hôtel, je vous le concède, mais ce qui la rendait particulièrement intéressante dans le cas présent, c’était que la femme de chambre en question ressemblait elle-même comme deux gouttes d’eau à Repentance Whittingham, alias Atiena, la Gardienne de mes deux.

Repentance semblait parfaitement détendue, comme une fille à qui tout sourit dans la vie et qui n’a à priori aucune raison de s’inquiéter pour son avenir.

Je dis bien «~à priori~», car toute personne qui a quelque chose à se reprocher et croise mon chemin a du souci à se faire pour son avenir. D’ailleurs, toute personne qui croise mon chemin, même si elle n’a rien à se reprocher, a du souci à se faire pour son avenir. Je peux très bien, sur un coup de tête, parce que je me suis levé du pied gauche ou de la couille droite, parce qu’il tombe sur le monde un fin crachin qui m’indispose, parce que ma femme me trompe avec mon chien, parce que j’ai écumé sans succès toute ma garde-robe à la recherche d’une paire de chaussettes qui ne soit pas trouée au niveau du gros orteil, pour tout un tas de raisons passant par tous les stades de l’insignifiance, décider de lui loger une balle de crâne, avant d’exécuter tous les témoins de la scène (ce qui a chaque fois constitue une nouvelle scène qui peut avoir de nouveaux témoins, et ainsi de suite, de sorte qu’il faut parfois éliminer tout un quartier, voire une ville entière et une bonne partie de la campagne environnante, pour se débarrasser enfin du dernier témoin, ou, si je suis dans un bon jour, ce qui n’arrive pratiquement jamais, lui laisser la vie après lui avoir coupé la langue et crevé les yeux et les tympans pour m’assurer de son innocuité, ce qui représente un surcroît de travail non négligeable pour finalement pas grand-chose, sachant que la plupart des gens préfèrent largement être morts plutôt que vivre dans les conditions épouvantables que je viens de décrire, sans voir ni entendre quoi que ce soit, ni être en capacité de prononcer le moindre mot, même s’il n’y a pas grand-chose à dire quand on en est réduit à un tel degré d’infirmité), regagner tranquillement mon domicile avec mes fringues couvertes de sang, me faire couler un bain, mettre un CD de Funki Porcini (The Mulberry Files, par exemple, ou At The Edge Of The World), me glisser dans l’eau tel un reptile sournois, une larve machiavélique, un cafard dans le plus simple appareil, et me laisser glisser sans résistance dans la tiédeur parfumée de l’oubli en sirotant un grand verre de cognac.

\textsc{Greg} : Tu vois ce que je vois ?

\textsc{Moi} : Oui, c’est Repentance Whittingham.

\textsc{Lui} : Qui ça ?

\textsc{Moi} : Atiena, la Gardienne de l’ennui. Son vrai nom, c’est Repentance Whittingham.

\textsc{Lui} : Comment tu sais ça ?

\textsc{Moi} : C’est Dumo qui me l’a dit.

\textsc{Lui} : Qui ça ?

\textsc{Moi} : Dumo, le réceptionniste.

\textsc{Lui} : On fait quoi ?

\textsc{Moi} : On va juste lui poser quelques questions, gentiment, poliment, comme des gens bien éduqués, et on verra bien ce qu’il en ressort.

\textsc{Lui} : Et s’il n’en ressort rien ?

\textsc{Moi} : On surveille l’entrée, on attend qu’elle sorte, on l’embarque et on l’emmène au labo pour la cuisiner à l’abri des regards indiscrets.

Greg, outré : Mais c’est parfaitement illégal.

\textsc{Moi} : En effet.

\textsc{Lui} : C’est contraire aux droits de l’homme, la femme, et tout ce qui s’ensuit. C’est très grave.

\textsc{Moi} : Si on s’en tient au code pénal, il s’agit effectivement d’un enlèvement passible d’une lourde peine de prison, surtout s’il s’accompagne d’actes de torture et de barbarie.

\textsc{Greg} : Désolé, mais je ne mange pas de ce pain-là. Ma réputation et mon honneur sont en jeu, sans parler de ma licence professionnelle.

\textsc{Moi} : Ne t’en fais pas, j’assume l’entière responsabilité de la mission. En cas de pépin, il te suffira de nier avoir eu connaissance de mes agissements, j’abonderai dans ton sens et tu t’en sortiras sans une égratignure. Mais trêve de bavardage. Cet endroit est une véritable jungle, nous ne sommes pas les bienvenus, et le personnel fera tout ce qui est en son pouvoir pour nous mettre des bâtons dans les roues, voire nous éliminer et faire disparaître nos corps dans les profondeurs de l’hôtel. Il faut agir vite, pendant que la cible est encore en vue, sinon on va perdre sa trace au détour d’un couloir. Elle a l’avantage du terrain, qu’elle arpente depuis de longues années et dont elle connaît chaque recoin par cœur.

\textsc{Greg} : On n’en sait rien.

\textsc{Moi} : Quoi ?

\textsc{Lui} : Si elle travaille ici depuis de longues années.

\textsc{Moi} : Non, mais on peut le supposer. On doit le faire, même, car il ne faut jamais sous-estimer l’adversaire. Allons-y !

Au moment où je prononçais ces mots, le regard de Repentance Whittingham a croisé le mien.

Il ne lui a pas fallu plus d’une demi-seconde pour me reconnaître et comprendre que ma présence en ces lieux n’était pas forcément la meilleure nouvelle de l’année.

Elle a lâché son chariot et s’est précipitée vers l’entrée avec une telle célérité (je ne saurais la quantifier au juste, mais je pense qu’on ne devait pas être loin de celle de la sonde Parker Solar Probe soumise à l’influence gravitationnelle de Vénus) que Greg et moi en sommes restés comme deux ronds de flan, bras ballants, yeux écarquillés et bouche grande ouverte, l’air si profondément idiot qu’on n’a pas été étonnés de voir tout le personnel présent éclater de rire et se foutre ouvertement de notre gueule.

Le temps de reprendre nos esprits, ravaler notre rancœur et prendre sur nous pour ne pas dégommer tout le monde à coups de flingues, et on s’est lancés à la poursuite de la fugitive.

Repentance Whittingham s’est engouffrée dans une Cooper S garée un peu plus loin, le genre de petite bombe qui permet de se faire la malle en vitesse quand on n’a pas la conscience tranquille, horriblement envie de pisser, ou simplement de se faire plaisir sur le bitume quand on est amateur de sensations fortes. Vous en connaissez beaucoup, vous, des femmes de chambre qui roulent en Cooper S ? Moi pas. On est déjà sur de la femme de chambre de qualité supérieure, haut de gamme, pur porc, élevée au grain sous la mer (d’où ces petits arômes iodés fleurant bon la moule à marée basse qui font la joie des gastronomes en culotte courte, en slip, ou même à poil pour les plus intrépides), ou alors une femme de chambre qui, en plus de passer une partie de son temps à faire des lits et récurer des chiottes, se livre à d’autres activités pour arrondir ses fins de mois (ce qui n’est pas une fin en soi, ni en soie, les humoristes apprécieront). À moins, bien sûr, qu’il ne s’agisse d’une femme qui a ou vient d’hériter d’une somme d’argent assez conséquente, ou de gagner au loto, et continue néanmoins à exercer sa profession parce qu’il s’agit pour elle non pas de quelque chose d’essentiellement alimentaire, avant tout destiné à la nourrir elle et le gamin trisomique qu’elle élève seule depuis le décès par arme à feu de son conjoint alcoolique et violent, mais d’une authentique passion qu’elle n’entend pas abandonner, même pour tout l’or du monde.

Repentance Whittingham a fait rugir le moteur de son petit bolide, pas très content d’être réveillé en sursaut (le genre de moteur qui n’est pas du matin, préfère largement s’exprimer à la tombée de la nuit sur les grands boulevards ou les petites routes de campagne), puis s’est enfuie en laissant dans son sillage des traces de gomme brûlée additionnées d’un épais nuage de gaz d’échappement, riches, comme chacun le sait, en monoxyde de carbone, dioxyde d’azote et autres hydrocarbures aromatiques polycycliques très mauvais pour la santé. Mais la petite peste se foutait royalement que les gens s’intoxiquent en respirant ses déjections. Tout ce qui l’intéressait, c’était de rouler comme une dingue dans les rues de la ville, sans se soucier un instant des dommages collatéraux, vies brisées et familles en pleurs que sa conduite irresponsable pouvait occasionner.

Si vous voulez mon avis, Repentance Whittingham n’était pas encore prête à se repentir, implorer la clémence des instances supérieures de l’univers, demander pardon à qui que ce soit, ni même éprouver le moindre soupçon d’un commencement de début de vague regret pour toutes les mauvaises actions dont elle s’était rendue coupable au cours de son existence entièrement dévolue à la domination d’autrui et sa soumission sans réserve à ses exigences les plus dégradantes, injustes, obscènes et farfelues.

De mon côté, après m’être quelque peu pris les pieds dans le tapis et avoir bien failli m’empaler sur la corne du buffle qui trônait à l’entrée, je déployais à présent l’ensemble de mes facultés motrices et capacités énergétiques pour rejoindre mon véhicule au plus vite.

J’étais plus ou moins dans les temps, jusqu’au moment où Greg a mis le pied sur une crotte de chien et ripé en beauté avant de s’étaler de tout son long dans le caniveau. Cette chute s’est accompagnée d’une bordée d’injures et de grossièretés que je n’aurai pas la faiblesse de reproduire en ces lignes, mais sachez qu’elles s’adressaient non seulement à la gent canine, ce ramassis de bâtards attardés tout juste bons à se renifler le cul et gueuler sans raison, mais surtout leurs soi-disant propriétaires, maîtres ou appelez ça comme vous voudrez, sombres crétins et abrutis de première classe qui considèrent l’espace public comme des chiottes à clébard. D’après Greg, on devait non seulement leur infliger une forte amende en cas de non-ramassage de crotte, mais aussi les condamner à ingérer l’objet du délit en mastiquant longuement chaque bouchée.

N’ayant rien d’autre sous la main pour nettoyer la semelle de sa chaussure, Greg, toujours en jurant comme un charretier, s’est mis à la racler furieusement contre le bord du trottoir, maudissant la négligence des usagers et l’incapacité des élus locaux à assurer la propreté de leur territoire. Dans ces conditions, l’interdiction massive et définitive de tout canidé dans l’espace public lui apparaissait comme la seule mesure raisonnable à prendre, le chien devant être réservé à un usage purement domestique, comme les fonctions de gardiennage ou d’ami de substitution pour les gens qui vivent dans la solitude la plus extrême, ou encore la pratique de la chasse en milieu forestier, endroit où ces sales bêtes peuvent pisser et chier à tout bout de champ sans que cela prête à conséquence, au même titre que les renards dont ils sont les proches cousins. Le chat, par contre, qui jamais ne s’abaisserait à chier devant tout le monde et encore moins laisser sa crotte à l’abandon, avait toute sa place dans la cité, d’autant que sa présence dissuadait les rats et autres rongeurs malintentionnés de s’installer ouvertement dans nos murs.

Cette opération de nettoyage nous a pris quelques précieuses secondes supplémentaires, de sorte que Repentance Whittingham, alias la Gardienne de la Nuit et accessoirement la femme de chambre la plus rapide du monde, avait déjà quelques longueurs d’avance sur nous quand nous nous sommes enfin lancés à sa poursuite.

Mais, grâce aux deux cent cinquante bourrins survitaminés attelés à ma charrette, il ne m’a pas fallu longtemps pour la rattraper. Le seul détail véritablement incommodant dans cette affaire, c’était que l’habitacle de la Kangoo, en dépit des efforts désespérés de Greg pour s’en défaire, empestait la merde de chien.

Me voyant débouler dans son rétro, elle a accéléré à son tour, prenant tous les risques pour me distancer, et je dois bien admettre qu’elle était loin d’être maladroite avec un volant entre les mains. Si un jour on la virait de son emploi de femme de chambre, elle pourrait toujours se recycler dans la course automobile. Cela dit, on n’était pas sur la Whaanga Coast ou au Panzerplatte. En ville, les limitations de vitesse sont réduites au plus bas et les panneaux de signalisation nombreux pour les faire respecter. Cela dit, faire partie de la noble famille des Représentants de l’Ordre présente tout de même quelques avantages, comme par exemple celui de disposer d’un gyrophare et une sirène pour s’affranchir des règles de circulation en vigueur. L’usager doit être averti quand des cinglés roulent à tombeau ouvert dans les rues de la cité. Il doit savoir que les forces de l’ordre (au même titre que celles du feu et de la santé), toujours sur la brèche pour garantir sa sécurité au péril de la leur, sont prioritaires quand des actions de ce type sont engagées sur la chaussée. Même si mon véhicule de service, qui se trouvait aussi être mon véhicule personnel (un ludospace très agréable en conduite normale et pratique pour transporter des objets lourds et encombrants), n’était pas une réplique exacte de la Pursuit Special de Mad Max, elle n’en constituait pas moins une arme de destruction massive une fois lancée à pleine vitesse dans les rues de la ville. J’ai donc mis en branle le système d’avertissement sonore et lumineux dont j’étais le dépositaire.

Au volant de sa Cooper S, totalement insensible aux appels à la raison que je multipliais à son égard, Repentance Whittingham filait comme une flèche et je lui suçais la roue (à défaut d’autre chose) au plus près, ne laissant entre nos deux véhicules que l’épaisseur d’une feuille de papier à cigarette. Dans les films d’action, les gens se collent au cul, roulent portière contre portière, et n’hésitent pas à froisser de la tôle pour arriver à leurs fins. Ils s’en foutent, c’est la production qui paye. Dans le cas présent, je le répète, il s’agissait d’un véhicule que j’avais payé de ma poche, rubis sur l’ongle, le prix d’une bouchée de très bon pain, je vous le concède, puisque je l’avais acheté d’occasion, mais dans lequel j’avais par la suite investi de confortables sommes d’argent pour le transformer en authentique machine de guerre habilement dissimulée sous les dehors inoffensifs d’un utilitaire sans autre prétention que celle de l’être, utile. On sait par expérience que sapiens sapiens (environ deux millions de glandes sudoripares, vitesse de pointe aux alentours de 45 km/h, inventeur de la bombe atomique, du petit salé aux lentilles et du saut à l’élastique) est très souvent au moins aussi attaché à sa voiture qu’à sa femme et ses enfants. Il la dorlote, la bichonne, la caresse amoureusement, se mire avec délectation dans sa carrosserie rutilante, s’enfonce jusqu’aux ouïes dans ses sièges en cuir pleine fleur pour savourer le chant rauque et mélodieux de son six cylindres sublimé par une ligne d’échappement optimisée, lui récure les jantes à la brosse à dents et polit les chromes à la peau de chamois. Il ne fait aucun doute, si la chose était matériellement possible, qu’il n’hésiterait pas à s’accoupler avec elle dans les positions les plus torrides du kamasutra automobile. On imagine alors facilement la suite, aussi terrifiante que fascinante : pour une raison quelconque, échappant totalement à l’entendement des apprentis sorciers du transhumanisme, la bagnole tombe enceinte. Après quelques semaines d’une grossesse tumultueuse, elle donne naissance à une sorte de monstre de Frankenstein supersonique, à mi-chemin entre l’homme et la Formule 1. À l’instar du singe et du nain (je ne fais bien entendu aucune analogie qualitative entre l’un et l’autre), la chose est dotée de membres inférieurs nettement plus courts que la moyenne, avec des cuisses très puissantes. Capable de se mouvoir efficacement aussi bien à la verticale qu’à l’horizontale, c’est toutefois dans cette dernière configuration qu’elle produit les accélérations les plus fulgurantes et pulvérise haut la main les records de vitesse des animaux les plus rapides de la planète. Et je parle ici uniquement des mammifères terrestres, comme le guépard, le lièvre et l’antilope, et de la poiscaille, comme l’espadon, le marlin et le thon banane. J’exclus volontairement les volatiles, comme le faucon, l’aigle et le martinet, et plus encore les insectes, minuscules créatures sex pedibus aux performances exceptionnelles, sachant que la vorace cicindèle, par exemple, qui ne dépasse pas les deux centimètres de long, flirte allègrement avec les sept cents kilomètres/heure, à tel point que ses propres yeux n’arrivent plus à suivre et qu’elle peine à distinguer ses proies dans le feu de l’action. Mais s’il fallait à tout prix trouver un alter ego vaguement humain à cette abomination mutante, ultime rejeton des passions déviantes de la sexualité automobile considérée comme un des beaux-arts, c’est clairement du côté des speedsters Jay Garrick, Wally West, Barry Allen, Pietro Maximoff (Vif-Argent), Danica Williams et Hunter Zolomon (alias Zoom ou Reverse-Flash) qu’il conviendrait de se tourner.

Bref, on s’en fout.

Ce qui est certain, c’est que personne, à commencer par moi, n’aurait pu se douter de ce qui allait se passer dans un très proche avenir.

Traquée, comme on le répète toutes les trente secondes dans ces pathétiques émissions de télé pour handicapés mentaux et baltringues en surcharge pondérale spécialisées dans les faits divers sanglants, cold cases et autres affaires judiciaires retentissantes dont tout le monde se contrefout éperdument, Repentance Whittingham a pris tous les risques pour tenter de nous semer. Elle aurait pu, pour rester dans la ligne cinématographique adoptée précédemment, être interprétée par la très contagieuse Antonia Thomas, père anglais, mère jamaïcaine, mélange aux yeux verts hautement détonant, susceptible de déclencher des tempêtes de niveau 5 dans les parties basses de la sphère anatomique : vents violents, slips arrachés et projetés à des lieues à la ronde, couilles tuméfiées, décharges à répétition, danger de priapisme et crise cardiaque. Pour celles et ceux qui ne seraient pas au courant, je me dois de préciser qu’Antonia, depuis quelques années maintenant, exerce la noble profession d’actrice au pays de Très Sa Gracieuse Majesté la Reine de Mes Deux. Hélas, son talent n’est pas reconnu à sa juste valeur par ces cons de rosbifs qui n’y entendent rien à l’art, la bouffe et la beauté féminine, rien à rien en général, sans quoi ils ne passeraient pas leur temps à se prosterner comme des carpettes décérébrées devant le Royal Vampire qui leur suce la moelle pour maintenir ses privilèges et son train de vie pharaonique. La vérité, c’est que ces insulaires, comme tous les gens de leur sorte, sont des asociaux de première classe, arrimés telles des patelles paranoïdes à leur foutu rocher. L’ineptie atavique et l’aveuglement quasi systémique de ses compatriotes, alcooliques pour la plupart il faut bien le dire, ont conduit à cette absurdité stratosphérique qui ne cesse de susciter mon exaspération (et celle, je veux le croire, de toutes les âmes sensibles qui ont encore le sens du beau et vouent, à travers ses productions les plus inimitables et abouties, un culte irréductible à la nature : si on veut la voir (Antonia) en activité et se prendre à rêver de serrer son petit corps frémissant entre ses bras pantelants, lui bouffer le nez à pleine bouche et éventuellement se livrer sur elle à des activités que la morale réprouve (je m’en excuse d’avance auprès d’elle et tous les membres de sa famille, mais je ne fais que traduire le sentiment général des heureux élus qui ont eu la chance de croiser sa route, hommes, femmes et animaux confondus, y compris insectes et batraciens, heureuses la mouche qui pète et la grenouille qui coasse sur son épaule dénudée), on doit malheureusement se satisfaire de somnoler avec une demi-molle devant des teen dramas et autres séries télé aussi diversement mémorables que Misfits et Lovesick, sans oublier Good Doctor aux côtés de Freddie Highmore, alors frais émoulu de la série Bates Motel. À noter que ce même Freddie Highmore, à l’instar d’une Olivia Cooke (délicieuse) ou un Nestor Carbonell (ténébreux à souhait avec ses sourcils épais et son regard de braise), qui lui donnent la réplique dans ladite série (Bates Motel, très réussie au demeurant), peine lui aussi à s’extirper des griffes du petit écran. Seule Vera Farmiga (qui joue le rôle de Norma Louise Bates, la mère de Freddie Highmore dans Bates Motel), née de parents ukrainiens, ancienne de la St. John the Baptist Ukrainian Catholic School de Syracuse, dans l’État de New York, et actrice pour laquelle j’éprouve une appétence aussi bizarrement obsessionnelle que dépourvue de tout caractère sexuel (même si je la trouve excessivement désirable, paradoxe que je peine à expliquer, je l’avoue, et ne tiens même pas spécialement à le faire, tant je préfère que le mystère reste entier, mais que j’aurais néanmoins, s’il fallait absolument tenter de fournir un élément de réponse, tendance à mettre sur le compte de l’aura de maternité quasi virginale qui émane de son adorable personne), s’en sort avec les honneurs. Croyez-moi ou non, mais c’est un bien triste monde que celui dans lequel la plupart d’entre nous survivent avec l’énergie du désespoir, et heureusement qu’il existe des gens comme Antonia Thomas et surtout Vera Farmiga pour badigeonner d’un peu de baume nos cœurs meurtris par les assauts répétés de l’existence.

Donc, comme je vous le disais, l’horreur est montée d’un cran.

Poussée dans ses derniers retranchements, telle une mygale acculée dans le fond de sa tanière, Repentance Whittingham s’est décidée à jouer le tout pour le tout. Refus de priorité, dépassement par la droite, emprunt de voie non autorisée, non respect des règles les plus élémentaires de bonne conduite et du bien circuler ensemble, elle a en quelques minutes pulvérisé tous les records détenus par les pires chauffards de la planète, prouesse d’autant plus remarquable qu’elle n’était à priori pas sous l’emprise de l’alcool ou d’une quelconque drogue, de synthèse ou autre. Franchement, si je n’avais pas eu entre les mains le volant sport de ma Kangoo Interceptor Turbo +, capable de franchir le mur du son par simple pression du gros orteil sur la pédale d’accélérateur, je crois que j’aurais été incapable de suivre le mouvement et obligé de renoncer, la mort dans l’âme et la larme à l’œil, à mettre un terme à la folle cavale de la Gardienne de la Nuit. Mais c’est mal me connaître que de croire que j’allais renoncer aussi facilement. J’avais en effet, outre une formation de cavalier de la Garde républicaine et d’enquêteur subaquatique (niveau bac ou RNCP 4), choses qui, quand on aime l’eau et les chevaux, peuvent toujours servir en cas de crise, suivi les cours de pilotage VRI du commandant Valentin Deschanel, spécialisé dans les go fast et les interventions en zone urbaine à forte densité, star incontesté de la discipline, hélas tragiquement décédé quelques mois auparavant dans un accident de la circulation, et ce alors qu’il rentrait tranquillement chez après avoir fait ses courses à la supérette du coin. Il faut dire qu’il était en T-MAX 530 (quarante cinq chevaux à six mille cinq cent tours/minute, le scooter préféré des dealers) et que le choc avait été d’une violence inouïe (et non pas inuite comme le prétendent certains imbéciles patentés dont on se demande encore comment ils ont réussi à ne pas faire virer de l’école et passer leur bac avec succès, le concept, je le rappelle, n’ayant aucun rapport particulier avec le Groenland et sa population autochtone, pas plus qu’avec la baleine, l’ours polaire et le caribou, même si l’ours polaire, plus encore que le caribou ou le pacifique cétacé, est capable de faire preuve de violence quand il se sent menacé, à noter la rime plutôt riche avec cétacé), j’en veux pour preuve ses mandarines, yaourts aux fruits (il adorait les yaourts aux fruits, personne n’est parfait) et blancs de poireau éparpillés dans tout le périmètre. Une vraie boucherie, et pour la perte sinon d’un ami à proprement parler, même si nous entretenions toujours d’excellentes relations, au moins d’un mentor qui m’avait enseigné les joies de la glisse et du talon-pointe.

Tandis que nous nous trouvions sur une artère passablement fréquentée, à slalomer dangereusement entre les usagers qui nous maudissaient au passage, Repentance Whittingham a tenté une manœuvre particulièrement osée, sinon suicidaire, qui consiste à doubler à l’aveugle dans un virage en priant le ciel que personne n’ait la mauvaise idée de se pointer en face. Ce cas de figure, systématiquement représenté dans les courses-poursuites au cinéma, donne toujours lieu, même si on sait que ça va passer de justesse, à des moments de franche rigolade, surtout quand le pauvre type qui arrivait paisiblement en sens inverse finit les quatre roues en l’air dans le décor, les cheveux en bataille et les fringues en lambeaux, et contemple, dépité, sa belle voiture toute neuve bonne pour la casse. C’est le genre de sitation cruelle et parfaitement injuste qui, reconnaissons-le à notre corps défendant, trouve bien souvent un écho favorable dans le public, preuve que l’être humain est encore loin d’en avoir fini avec ses vieux démons.

Sauf que dans la vraie vie, sans caméra ni perchman, les choses ne se passent pas toujours comme on voudrait. Le scénario n’est pas écrit, on improvise en permanence, et il n’est pas possible de retourner quarante fois la scène pour obtenir satisfaction. La première est la bonne, il faut être au top tout de suite et ne surtout pas compter sur le montage ou les effets spéciaux pour arrondir les angles.

Et ce qui devait arriver arriva : au moment où Repentance effectuait sur les chapeaux de roues son dépassement non autorisé pour cause d’absence totale de visibilité dans un virage particulièrement dangereux, un Sprinter (utilitaire léger de chez Mercedes, ndlr) arrivait en sens inverse. À son volant, se trouvait un triste sire qui lui-même roulait à tombeau ouvert parce qu’il était en retard à son boulot, peintre en bâtiment en l’occurrence. Non content de baigner dans son jus comme une grosse merde fraîchement pondue, il écoutait du Spear of Longinus à fond les ballons, fumait clope sur clope et braillait comme un veau dans l’habitacle enfumé de sa poubelle.

On ne va pas se mentir, se voiler la fesse, tenter de minimiser pieusement les faits : le choc a été d’une violence impitoyable, provoquant un de ces vacarmes épouvantables qui évoquent irrésistiblement un tremblement de terre, une explosion due au gaz ou une attaque de missiles russes, et la projection dans l’atmosphère d’une pluie de débris automobiles comparables aux scories d’une éruption volcanique de grande ampleur.

Disons-le tout net, la Mini ne faisait pas le poids face au Sprinter. Elle a été littéralement pulvérisée sous l’impact. Tout est allé tellement vite que le Sprinter (dont le conducteur, qui avait pris une cuite retentissante la veille, n’était sans doute pas au mieux de sa forme) n’a même pas esquissé la moindre tentative de dégagement. De mon côté, tandis que Greg tétanisé par la peur n’avait même plus la force de hurler, se contentant d’ouvrir un bec de cent pieds de long d’où aucun son ne sortait, j’ai sauté à pieds joints sur le frein et réussi de justesse, grâce à mon sens aigu du pilotage et ma parfaite connaissance de l’arme de destruction massive qui me tenait lieu de véhicule, à éviter le massacre. Après avoir embouti la Mini de plein fouet, le conducteur a cédé à la panique, et le Sprinter a continué sa route en zigzaguant dangereusement avant d’aller s’écraser à grand bruit contre un platane.

Je suis sorti de la voiture (le premier, Greg avait besoin d’un peu de temps pour se remettre de ses émotions) et me suis précipité au chevet de Repentance Whittingham. Enchevêtrée dans un amas de tôle indescriptible, j’ai constaté qu’elle donnait encore quelques vagues signes de vie. Naturellement, j’ai aussitôt appelé les secours. Même si j’avais quelques raisons de lui en vouloir, notamment celle de m’avoir fait risquer ma vie dans cette course-poursuite endiablée, l’idée de voir une aussi belle chose disparaître à tout jamais de la surface de la Terre me semblait intellectuellement irrecevable. Et oui, pas la peine de hurler, je sais très bien que les femmes, et les êtres vivants en général, ne sont pas des choses, même si on pourrait très bien partager le monde entre choses vivantes ou non, le fait d’être en vie se signalant essentiellement par sa nature dégénérative et éphémère. La vie, en sursis permanent dans le couloir de la mort, est une situation des plus inconfortables. L’être humain, conscient de cette précarité, a tôt fait de sombrer dans le doute et la paranoïa. Il tente, par tous les moyens, de prolonger son existence. Mais pourquoi s’acharner à vivre dans un monde aussi hostile, avec une sentence de mort épinglée au milieu du front ? Le danger vient de partout, y compris de l’intérieur, et peut surgir à tout moment, y compris celui où on s’y attend le moins. Malheur à l’inconscient, galvanisé par la jeunesse ou les stupéfiants, et souvent les deux, qui tente de tromper la mort, car elle finira un jour ou l’autre par l’emporter. La chose vivante, condamnée à disparaître, l’est aussi à se reproduire pour se survivre à elle-même. Sans cette fonction essentielle, toute vie est impossible. Elle peut aussi se dupliquer par ses propres moyens, sans l’aide d’un tiers, mais elle ne peut échapper au processus. La chose, par contre, la vraie, parfaite en soi, se suffit à elle-même et n’a nul besoin de ce stratagème pour perdurer. Son existence, sinon définitive, est au moins certaine et indéniable. Elle n’a nul besoin de régénérescence, renaissance, reset ou mise à jour, nul besoin d’évoluer, de s’adapter à son environnement. Elle est intemporelle, exempte de tout questionnement, toute remise en question. C’est ici, sans doute, que s’opère le distinguo entre la chose naturelle et l’objet fabriqué, artificiel. L’objet, en effet, est soumis à un objectif qui le dépérennise, le voue, au même titre que l’être vivant, à l’obsolescence et la disparition, l’obligeant à se transformer sans cesse pour subsister. C’est ainsi que l’objet, la machine, par exemple, transcende sa nature inanimée pour se rapprocher artificiellement du principe vital. À travers l’homme, l’objet se déplace, pense, vit et meurt. Il est, en quelque sorte, l’ultime avatar de son désir d’affranchissement des lois de l’existence. Depuis des millénaires qu’il se torture inutilement les méninges, l’homme n’aspire qu’à une seule chose : devenir une machine, un automate capable, sans le moindre effort, la moindre prise de tête, crise de conscience ou autre, le moindre doute sur la nature de ses actes, d’accomplir des miracles et de battre tous les records, y compris de longévité, de réduire définitivement au silence et l’impuissance sa vieille ennemie la Mort. L’acte sexuel, par exemple, se doit d’être entièrement dévolu au plaisir, et non plus à cette fonction dégradante et avilissante qu’est la reproduction. Ainsi, ce piège odieux tendu par la Nature pour nous contraindre à signer notre arrêt de mort, entériner l’acte de décès de nos années de jeunesse et d’insouciance, se transforme en gag de l’arroseur arrosé. Débarrassé de ces oripeaux d’un autre âge, le plaisir sexuel peut enfin s’exercer sans limite ni contrainte, faire feu de tout bois. Jouissons sans entrave, morale ou autre, et laissons les basses-œuvres de la chose reproductive à celles et ceux qui n’ont d’autre moyen de subsistance que de fabriquer des enfants à la chaîne. Voilà un monde plus juste, où chacun se spécialise dans son domaine de compétence et vit pleinement sa vie sans remords ni regret. Dans ce monde plus juste, les nantis, dont la recherche du plaisir est le principal, sinon seul et unique domaine de compétence, viennent au secours de ceux qui ont voué leur existence à la douleur. Voir les autres comme des objets, des instruments qu’on peut manipuler à loisir pour satisfaire ses exigences, se satisfaire, est en droite ligne du chemin d’excellence que nous nous sommes tracé. Notre objectif, je le rappelle, est de devenir des machines de guerre, utraperformantes, conçues pour dominer et conquérir le monde. Dans un premier temps, bien sûr, avant de s’attaquer au reste de l’univers, et j’en profite au passage pour rendre grâce à Elon Musk (visionnaire sud-africain de race blanche dont le caractère ultra-reproducteur, à priori paradoxal, s’explique uniquement par son appétence pour les femmes jeunes et jolies bien entendu, mais surtout son narcissisme exacerbé et la volonté de se dupliquer à l’infini) d’avoir eu la présence d’esprit de se ménager (à lui et quelques fidèles) des bases arrières dans l’espace afin de ne pas être pris au dépourvu le moment venu, sachant que les gens, hélas bien trop rares, qui voient un peu plus loin que le bout de leur nez, commencent à se trouver singulièrement à l’étroit sur ce lopin de terre étriqué qu’est la planète bleue. Toutes les créatures faibles et geignardes qui auront l’impudence de se mettre en travers de notre marche triomphale seront impitoyablement exterminées.

Et voilà pourquoi cette chère Repentance Whittingham, qui, j’ose le dire, était l’irréfutable incarnation de l’éternel féminin en son sens le plus mythologique (autant dire carrément mytho, archétypal, archi-typique, désespérément romantique, goethien, gothique, et, en un mot comme en cent, au moins en ce qui me concerne car tous les égouts sont dans la nature, assez éloigné de la blonde platine à la Ginger Rogers, d’autant que j’ai une sainte horreur de la comédie musicale et que Fred Astaire m’a toujours fait penser à un sosie virevoltant de Stan Laurel) du terme, n’avait à mon sens nul besoin de se livrer à cette malheureuse tentative de dépassement, au sens propre comme au figuré, laquelle ne pouvait, en définitive, que la confronter brutalement à sa propre finitude.

Quand il a été constaté qu’elle était encore en vie, même si celle-ci ne tenait plus qu’à un fil, Greg et moi, tandis que les badauds (à commencer par le conducteur de la vénérable Ford Fiesta que la Mini avait tenté de doubler) commençaient à affluer de toute part, nous sommes dirigés vers la carcasse du Sprinter.

Le baiser passionné qu’il avait échangé avec le tronc du platane lui avait explosé le moteur et fait voler le parebrise en éclats.

À l’intérieur, se trouvait une vieille connaissance que Greg, pour avoir suffisamment enquêté sur l’affaire Tiago Alvarez (Sally Robinson le harcelait encore quasi quotidiennement pour le pousser à franchir le Rubicon en éliminant l’assassin présumé de son bien-aimé) a parfaitement identifiée dès le premier coup d’œil. Aussi improbable que cela puisse paraître, il s’agissait ni plus ni moins que de l’ignoble néonazi psychopathe, pervers et pyromane Noé Desmarais, fondateur, avec les sieurs Aymeric Jégou et Milo Monteil, de la sympathique petite association de malfaiteurs connue sous le nom de Disciples de la Colère. Personnellement, je préférais la dénomination de Disciples de la Connerie, qu’ils avaient poussé à un rare degré de perfection.

Petite piqûre de rappel pour la route : un beau jour, un certain Léopold Chiasson de Bellisle, comte de son état, s’éprend de son garde-chasse, un jeune dieu répondant au nom de Robert Pleimelding. Dès qu’il voit son petit cul apparaître au coin d’un bois (quand il est à quatre pattes en train de ramasser des châtaignes, par exemple, ou de renifler une truffe avec son flair de lévrier), le comte est tellement excité qu’il serait prêt à enfiler cul sec le premier sanglier qui lui passe à portée d’entrejambe. Car le problème, voyez-vous, c’est que Robert est marié, et que rien ne prouve qu’il ait envie de se faire ramoner la tuyauterie par son employeur (même si à l’époque, je dis ça je dis rien, les gens n’étaient peut-être pas aussi tatillons sur les conditions de travail et les droits du citoyen, autrement dit rechignaient moins à payer de leur personne pour se donner les moyens de réussir dans la vie). Fou de désir, vous l’aurez compris, Bellisle lui sort le grand jeu : havane, cognac centenaire, grosse bûche dans la cheminée du salon, préludes de Chopin avec légers craquements d’époque par un Arthur Rubinstein au sommet de son art, peignoir en soie savamment entrouvert pour laisser deviner une anatomie aussi avantageuse que remarquablement bien conservée, interminable discussion sur la question de savoir lequel du 12 ou du 20 est le meilleur calibre pour la chasse à la bécasse, sachant que le plomb de 8 bourre grasse reste sans doute le meilleur compromis en toute circonstance, etc, etc. Petit hommage en passant à la vieille sorcière de la Madrague récemment disparue, ex-héroïne de La Vérité (vaudeville judiciaire et musical tragique signé Henri-Georges Clouzot), même si, qu’on le veuille ou non, ses amitiés politiques douteuses avec la fille cadette du caliborgnon de la Trinité-sur-Mer risque de ternir durablement l’image de passionaria de la cause animale qu’elle entendait laisser. Bellisle, disposant de moyens quasi illimités pour parvenir à ses fins, ne tarde pas, en dépit de la résistance héroïque qu’elle lui oppose, à réduire sa proie à sa merci. La romance peut commencer, et se prolonger ainsi jusqu’à la mort du comte, qui laisse alors assez d’argent à Robert et sa famille pour vivre confortablement en chantant les louanges et bénissant chaque jour le nom de leur généreux donateur. Dans la foulée, l’aristocrate lui octroie également les quelques cinquante hectares de forêt (et pas de la friche dégueulasse, hein, du terrain vague hérissé d’arbres faméliques tout juste bon à stocker des déchets nucléaires, non, de la bonne vieille forêt bien épaisse regorgeant du plus fin gibier et des meilleurs champignons comestibles poussant comme une bénédiction divine au pied des plus beaux fûts), de forêt, disais-je, que Robert, même s’il n’a plus le pas aussi souple qu’auparavant, continue à arpenter inlassablement pendant ses vieux jours, la pipe au bec, comme il l’a toujours fait et rêve de le faire encore longtemps après sa mort dans les forêts enchantées du paradis. C’est là, au cours d’une de ses balades avec sa fidèle Greta (un drahthaar, croisement de griffon kortals et de braque allemand à poil court, excellent chien d’arrêt qui, le cas échéant, n’hésitera pas un instant à se jeter à l’eau pour repêcher une gélinotte criblée de plombs), son juxtaposé Chapuis Progress Grand Luxe calibre 12/70 (crosse anglaise en noyer premier choix, plaque de couche en bois de rose et sujets animaliers gravés en taille douce sur les contre-platines et le dessous de bascule, une arme de collection qu’il n’aurait jamais eu les moyens de s’offrir sans les largesses de Chiasson) et sa gibecière à rabat en cuir pleine fleur Lazzaro Bernardini (encore du cousu main qui cadre assez peu avec le statut de garde-chasse à la petite semaine de l’intéressé), qu’il tombe sur une vision d’horreur qui restera à tout jamais gravée dans sa mémoire : un corps (humain, le corps), qu’une quelconque entité malfaisante a manifestement tenté de détruire intégralement par le feu.

S’il était envisageable que le sujet, pour des raisons diverses (démence, fanatisme religieux ou mystique particulière, satanisme, pratique du vaudou, sur fond d’addiction à l’alcool ou toute autre drogue dure sous quelque forme que ce soit), ait tenté de mettre fin à ses jours de cette façon que je qualifierai de pour le moins médiévale, il l’était nettement moins qu’il se soit fait désintégrer par des extraterrestres en train de reconnaître les lieux dans la perspective d’une prochaine invasion. Tout bien considéré, le plus crédible restait que le pauvre avait été la victime d’actes de torture et de barbarie de la part d’un ou plusieurs individus qu’il restait à identifier et punir à la hauteur de leurs agissements.

Dépêché sur les lieux en compagnie de Zaahid Shirani, légiste d’origine hindoue avec lequel j’avais, dans un premier temps, noué des relations purement amicales, avant que celles-ci ne se transforment en relations quasi familiales (le ténébreux personnage avait fort habilement mis le grappin sur Tosca, la sœur jumelle de ma dulcinée), j’avais d’abord pensé à un règlement de comptes entre truands, la technique dite «~du barbecue~» étant souvent utilisée par ce genre de clientèle pour tenter de faire entrave à l’action de la justice (infraction, je le rappelle en passant au cas où certains d’entre vous auraient dans l’idée de céder à la tentation, passible au bas mot de trois ans d’emprisonnement et de 75 000 euros d’amende).

C’était compter sans la perspicacité de Shirani, limier diabolique capable de débusquer sans effort la plus microscopique aiguille dans la plus monumentale botte de foin. L’animal avait réussi, Dieu sait comment, à dresser le profil génétique de la victime et le comparer à celui de Tiago Alvarez, un ambulancier gay qui s’était récemment évaporé dans la nature, disparition jugée plus qu’inquiétante sur laquelle, par le plus grand des hasards, enquêtait Grégoire Lussier, un autre mien ami, pour le compte d’une (un) certaine Sally Robinson, sosie de Danny DeVito en jupon. Grâce à Cerqueira, gorille portugais au cerveau de moule qui tentait tant bien que mal de dissimuler son homosexualité à son entourage (je l’avais surpris en flagrant délit de racolage sur la voie publique, et menacé de le foutre en taule et ébruiter son petit secret s’il ne se montrait pas coopératif), à commencer par les membres de sa fine équipe de sympathisants d’extrême-droite, le lien entre Alvarez et Desmarais avait pu être établi. D’après le grand singe en question, c’était Noé Desmarais, suprémaciste blanc, homophobe et pyromane, qui avait carbonisé Alvarez au lance-flammes, en joyeuse compagnie de deux autres ordures de son espèce, Aymeric Jégou et Milo Monteil.

J’ai brandi ma plaque et hurlé à la cantonade : POLICE !!! CIRCULEZ, Y A RIEN À VOIR !!!

On le sait, les gens adorent les trucs scabreux. Ils vous diront que non, mais le fait est que s’il se passe quelque chose d’horrible quelque part, ils ont toutes les peines du monde à ne se précipiter pour jeter un œil. Ils sont comme ces charognards qui reniflent l’odeur de la viande froide et rappliquent ventre à terre pour prendre part au festin. Avant ils se contentaient de regarder, maintenant ils filment, ce qui leur permet d’une part de revoir la scène encore et encore, d’autre part d’en faire profiter celles et ceux qui n’auraient pas eu la chance d’y assister. Avant, ils n’avaient que leur parole à opposer aux sceptiques et aux jaloux, maintenant ils ont les images pour preuve de leur bonne foi. Ces images, d’ailleurs, peuvent leur assurer une petite rente s’ils sont les seuls à les posséder. Les chaînes d’info en continu se feront une joie de les récupérer pour les diffuser en boucle auprès du grand public. Naturellement, plus c’est ignoble et horrifique, et plus le rapport est important. Tout est fait, en ce bas monde, pour flatter les plus bas instincts de la communauté. C’est ainsi que les réseaux sociaux, ramassis de crétins décérébrés et de sociopathes à la petite semaine, peuvent se développer à la vitesse d’une portée de cafards dans un placard rempli de victuailles. Les vautours parlent aux vautours, partagent l’info en temps réel, vingt-quatre heures sur vingt-quatre. Les hyènes ricanent, les requins de la finance entrent dans la danse et se remplissent la panse. Honni soit qui mal y pense, et longue vie en passant à la reine Victoria, au prince de Galles, à Ed Wood (le comte d’Halifax, pas l’auteur de La Fiancée du monstre et Necromania, qualifié par les frères Medved de «~plus mauvais réalisateur de tous les temps~», ce qui est plutôt un compliment venant de réacs dans leur genre), au gentilhomme huissier de (pas à) la verge noire, au duc de Kent et à Naruhito, empereur du Japon, à sa gracieuse épouse Masako Owada, et bien sûr à leur fille Aiko, princesse de Toshi et Grand-cordon de l’ordre de la Couronne précieuse, au même titre que la reine d’Espagne et la princesse Basma de Jordanie, également (pour celles et ceux que ça intéressent, même si je me doute bien qu’ils ne sont légion) Grand-Croix de l’ordre royal de l’Étoile polaire de Suède. Les gens que je viens de citer, citoyens haut-placés de l’univers, quasi divinités vénérées par des peuples tout entiers (lesquels, il faut bien le reconnaître, ne barbotent pas toujours dans l’opulence et ont par conséquent d’autant plus de mérite à admirer des gens qui se goinfrent sur leur dos), bénéficient d’une protection toute particulière pour garantir leur intimité. Mais vous, qui n’avez pas de particule et encore moins de sang royal qui circule dans vos veines, sachez que quel que soit l’endroit où vous vous trouviez (j’allais dire cachiez, sans doute ce que vous auriez de mieux à faire), il y aura toujours quelqu’un pour vous filmer à votre insu. Non pas que votre vie présente un quelconque intérêt, mais le voyeurisme est aujourd’hui parvenu à un tel degré d’omniprésence qu’il est devenu impossible de s’y soustraire. Fini le bon temps où on pouvait agresser une petite vieille en toute sécurité, tabasser un étranger ou harceler sexuellement sa secrétaire sans que la moitié de la planète soit au courant dans les secondes qui suivent. Aujourd’hui, outre les caméras de surveillance qui fleurissent à tous les coins de rues, il faut compter sur les particulier qui vivent en permanence l’œil rivé à l’écran de leur smartphone. L’être humain a toujours été sujet au voyeurisme, c’est vrai, mais ce qui était jadis honteux peut aujourd’hui s’afficher au grand jour en toute légalité. De la même façon, l’exhibitionnisme n’est plus réservé à une certaine catégorie de personnel, comme les politiciens, les artistes ou les gens qui se baladent à poil sous des imperméables même quand il ne pleut pas, de préférence à la sortie des écoles. Non, vous pouvez maintenant être une parfaite nullité totalement dépourvue de charisme et réussir à capter l’attention de millions de followers encore plus nuls que vous. On dit follower parce que suiveur ou suiveuse c’est pas terrible, sinon franchement péjoratif, au même titre que disciple, qui fait un peu gourou d’une secte de demeurés, ou encore abonné, qui fait référence à la téléphonie des années 70 et cadre assez mal avec l’idée de modernité véhiculée par les nouvelles technologies.

Avant, si vous étiez un gros pervers de voyeur, vous aviez vu Psychose trente ou quarante fois, et, sans aller jusqu’à empailler des oiseaux, appliquiez la méthode dite «~Norman Bates~», c’est-à-dire que aviez acheté une perceuse et fait des trous un peu partout dans votre baraque pour vous rincer l’œil. Mais comme vous n’aviez pas de motel perdu au fin fond de la cambrousse, et qu’en plus vous aviez hérité d’une tête qui n’inspirait pas vraiment confiance, le plus dur était de trouver des femmes qui, pour une raison ou pour autre, acceptent de franchir le seuil de votre porte. Par chance, vous aviez un peu de famille, des sœurs, des tantes, des nièces, et même un fils qui s’était dégoté une femme somptueuse qui lui avait donné de nombreux et beaux enfants, dont trois filles pas piquées des hannetons, et qui, malgré le fait que vous sembliez bizarre çà tout le monde (le petit dernier, auquel vous fichiez une trouille bleue, ne se résignait à vous embrasser que contraint et forcé par ses parents, et encore n’aurait-il pas fait pire grimace si on l’avait forcé à rouler une galoche à une vieille méduse échouée sur la plage des Sablettes à La Seyne-sur-Mer), venait vous rendre visite de temps à autre. Vous pouviez alors, l’œil rivé à l’un ou l’autre des nombreux trous qui garnissaient votre intérieur, celui de la salle de bain ou des toilettes notamment, satisfaire pleinement vos répugnantes ambitions. Certes, tout cela demandait une préparation minutieuse et un art consommé de la dissimulation, car si jamais quelqu’un avait découvert votre petit manège, vous pouviez définitivement dire adieu au peu de vie sociale qu’il vous restait, et accessoirement vous préparer à aller finir vos jours en prison, endroit où on pratique encore la séparation des sexes et où on ne dispose de toute façon d’aucun outil pour faire des trous dans les murs. Aujourd’hui, tout se passe comme si un petit malin, un visionnaire et bienfaiteur de l’humanité, conscient de la détresse qui accablait ses semblables et bien déterminé à leur venir en aide, avait trouvé le moyen de faire des trous partout dans le monde pour que tout le monde puisse se rincer l’œil sans limite d’âge ni de distance. Des tas d’exhibitionnistes frustrés, contraint de vivre dans la honte et le déni, ont enfin pu afficher leur différence au grand jour. Tu veux voir mon zizi (pensée émue pour Francky Vincent, inoubliable auteur de La Braguette d’or, Alice ça glisse et La Chatte de la voisine), ma petite chérie ? Non ? Pas de problème, je te le montre quand même. Oui, je sais, tu n’as que dix ans, mais il n’est jamais trop tôt pour s’instruire. Je vais te montrer à quoi ça sert, ce qui se passe quand on le secoue, et tu me remercieras plus tard. Voilà quand même une approche autrement constructive et conviviale du bien vivre ensemble (endroit coloré et agréable, salles bien conçues, belle initiative de la municipalité), qui devrait, sauf imprévu, largement contribuer à apaiser les tensions qui divisent les peuples et dressent inutilement les gens les uns contre les autres. Grâce à Internet, chacun peut enfin vivre sa passion sans se soucier du qu’en-dira-t-on. Ces échanges entre gens de bonne compagnie, hors de tout jugement, sont le lubrifiant qui permet à la machine sociale de tourner sans accroc. Désormais, si un accident se produit quelque part et que ne pouvez y assister, d’abord parce que personne n’a jugé utile de vous prévenir, ce qui est déjà assez grave, et ensuite parce qu’on ne peut pas être partout en même temps, soyez rassuré : même s’il se produit en haute mer, au sommet de l’Himalaya ou aux abords de la planète Mars, les rescapés auront pris soin de filmer la scène sous toutes les coutures. Vous assisterez, comme si vous y étiez, à la lente agonie de celles et ceux qui n’auront pas eu la chance de mourir sur le coup (que vous aurez vu mourir aussi, ne vous inquiétez pas). S’il s’agit d’un naufrage, vous verrez les moins chanceux se faire dévorer par les requins, et le reporter improvisé, animé par une foi intense et la volonté farouche de servir l’info à tout prix, préférera mille fois se faire dévorer lui-même plutôt que de renoncer à sa mission. Et s’il a encore la force, dans un ultime sursaut d’orgueil, de filmer sa fuite effrénée dans les eaux glacées de l’océan, vous verrez, en un sublime ralenti magnifié par les filtres idoines, les terribles prédateurs se rapprocher inexorablement de leur proie et la mettre en pièces jusqu’à ce que soit tranchée, en un ultime gros plan spectaculaire, la main qui tenait courageusement l’appareil hors des flots écumeux et rougis par le sang. Si on est au sommet de l’Himalaya, ce qui peut arriver aussi, quelque part entre le Népal et le Tibet, et que les survivants, pris dans une crevasse et gravement blessés dans leur chute, n’ont pas d’autre choix que de s’entredévorer pour survivre, il y en aura toujours pour filmer pendant que les autres passent à table. Je pense aussi à la Grande Guerre (celle de 14-18 pour les ignares), époque où la photographie n’en était encore qu’à ses premiers balbutiements en noir et blanc. On dispose bien de quelques images rudimentaires, d’une netteté approximative, mais quelle perte pour l’humanité de ne pas avoir pu voir et savoir ce qui se passait réellement dans les tranchées. Les derniers poilus sont morts, et la plupart yoyotaient sérieusement dans les dernières années de leur existence, chose bien sûr tout-à-fait normale dont il ne saurait ne leur être fait grief, surtout quand on a vécu les horreurs de la guerre. Difficile néanmoins, dans ces conditions, de se fier à leur témoignage, même si on ne rendra jamais assez hommage au courage et la ténacité dont ils ont fait preuve pour empêcher le pays de tomber aux mains de ceux qui n’étaient pas encore des enfoirés de nazis mais n’allaient pas tarder à le devenir. Si Manfred von Richthofen, par exemple, alias le Baron Rouge, avait pu filmer tout ce qui s’est passé à bord de son Fokker, je pense que ces images feraient aujourd’hui encore la une du box-office. Pour ce qui est de la planète Mars, avec les transmissions satellites actuelles, on pourrait vivre le massacre en direct sans aucune difficulté, une bière à la main, confortablement installé dans le canapé de son salon, entouré de poupées sexuelles hyperréalistes en provenance de Chine ou du Japon. Une certaine idée du bonheur, un peu déviante, peut-être, mais tellement contemporaine. Il faut vivre avec son temps, nom de Dieu, et le temps est aux pédophiles qui montrent leur bite sur les réseaux sociaux, aux profanateurs de sépultures qui vont pisser sur la tombe de Badinter, aux dirigeants irrédentistes et mégalos qui font main basse sur les richesses du monde en agitant l’épouvantail d’une troisième guerre mondiale, aux tueries de masse dans les écoles et les universités, aux ballots de coke qui s’échouent sur les plages de la Côte d’Opale, aux riches de plus en plus riches, aux pauvres de plus en plus pauvres, aux cons de plus en plus cons, aux morts de plus en plus morts, aux gamines qu’on shoote de force pour les prostituer dans des chambres d’hôtel sordides, aux réfugiés qui se noient dans l’océan (ils n’ont même pas la chance de tomber sur un ballot de coke qui leur permettrait de s’offrir des vêtements secs et des vrais faux papiers), aux gens de plus en plus gros à force de bouffer de la merde (qui est en vente libre, contrairement au crack, aux sels de bain, au fentanyl et à la kétamine, très nocifs je le rappelle, alors que l’ayahusaca et les champignons magiques, produits naturels s’il en est, sont toujours là pour vous procurer voyage et dépaysement à petit prix), aux gens qui s’exhibent à moitié à poil dans des escape games débiles et autres programmes affligeants (liste non-exhaustive pour prendre pleinement la mesure de l’ampleur des dégâts : Bâtard Academy, Qui veut épouser mon fils ? Personne merci, L’Amour est dans ton cul, Norbert crétin d’office, Mariés au premier coup de bite, La Villa des culs brisés, Danse avec les porcs, The Voice : La Poisse, L’île de la perdition, Le Meilleur bon à rien, Crétin Express, Astikoh-Lanta, Top Larbin, Cauchemar dans les latrines avec le trois-quarts centre étoilé de la cuisine française, etc, etc, etc) suivis par des millions de téléspectateurs qui sont à priori des gens normaux comme vous et moi (j’espère bien que non, sinon tout est foutu), j’en passe et des meilleures, des vertes, des pires et des pas mûres.

Donc, comme je le disais avant d’expliquer les raisons de cet emportement, j’ai brandi ma plaque et hurlé à la cantonade : POLICE !!! CIRCULEZ, Y A RIEN À VOIR !!!

Et j’ai ajouté, ce qui n’était pas absolument nécessaire techniquement mais très satisfaisant au niveau du bien-être et l’épanouissement personnel : FOUTEZ LE CAMP, BANDE DE CHAROGNARDS !!!!!!

La plupart de ces enfoirés ont foutu le camp, conformément à mes instructions on ne peut plus claires, mais certains, plus tenaces et affamés que d’autres, ont continué à rôder dans les parages, se planquant derrière les arbres et les tas d’ordures pour s’adonner à leur vice. Je précise que, suite à une grève prolongée des agents de propreté urbaine, plus communément appelés éboueurs, les poubelles vomissaient leurs entrailles et l’espace public s’était transformé en décharge à ciel ouvert, avec diverses conséquences, dont au moins deux passablement préoccupantes sur le plan sanitaire et social : d’une part attirer toute une faune de bestioles peu regardantes sur l’état de fraîcheur de leur alimentation, d’autre part empuantir l’atmosphère de façon significative. Même les oiseaux, qui d’ordinaire se plaisaient à faire des vocalises dans la ramure environnante, avaient déserté la place.

Les pompiers, les flics, les secours sont arrivés, toutes sirènes hurlantes.

Desmarais, assis au volant de sa bagnole comme si de rien n’était, à ceci près qu’il était tout de même légèrement disloqué de toute part, ne présentait aucune blessure apparente. L’idée qu’il ait pu survivre à l’accident était pour moi d’une extrême contrariété. J’étais, vous le savez, bien décidé à le buter. J’avais beau tourner et retourner la question dans tous les sens, je ne voyais aucune circonstance atténuante à accorder à cette ordure. Même s’il avait eu une enfance horriblement malheureuse, si son père l’avait élevé dans le culte du Troisième Reich et obligé à apprendre Mein Kampf par cœur dès son plus jeune âge, il devait disparaître sans laisser de traces. Et comme il y avait de fortes chances que sa femme partage ses opinions, qui n’étaient d’ailleurs pas des opinions mais seulement une expression idéologisée de la haine, la bêtise et la vulgarité, il aurait été prudent de l’exterminer dans la foulée. Même chose pour ses gosses, auxquels qu’il avait vraisemblablement inculqué, comme son père l’avait fait avec lui, les valeurs de l’idéologie nazie, valeurs qu’eux-mêmes se feraient un devoir de transmettre à leur progéniture. Mais peut-être les avait-il découvertes lui-même, comme un grand, après une quelconque tragédie qui l’avait laissé profondément meurtri et gorgé d’amertume dans un monde cruel où il n’avait plus sa place. Imaginez par exemple que sa mère, après avoir pris conscience (un peu tard, malheureusement) que son mari n’était qu’une brute, un alcoolique et un crétin antisémite, raciste et xénophobe, ait cédé aux tentations de l’adultère avec un employé du gaz venu relever les compteurs (et les jupes par la même occasion). C’est déjà très embêtant, car chacun sait que la brute raciste et xénophobe est très à cheval sur les valeurs morales et la fidélité conjugale (surtout en ce qui concerne sa femme), mais si l’employé du gaz en question cumule avec une rare insolence toutes les tares les plus rédhibitoires aux yeux de ladite brute, on coure droit à la catastrophe. Même s’il avait été le parfait aryen dans toute sa splendeur, la chose aurait été difficile à avaler. Mais imaginez, ne serait-ce qu’un instant, qu’il soit juif, arabe, communiste, franc-maçon, bisexuel, et, pour couronner le tout, affligé de quelque handicap mineur ou discrète malformation congénitale (un sexe anormalement développé, par exemple, et capable de soutenir une érection des heures durant sans montrer le moindre signe de faiblesse). Le mari trompé, quand il découvre le pot aux roses (en l’occurrence des roses fanées dont les tiges putréfiées baignent dans l’eau croupie), entre dans une fureur noire. Après être allé se recueillir une dernière fois sur la tombe de Rudolf Hess à Wunsiedel, en Bavière, il endosse son plus bel uniforme de la SS, vérifie que le chargeur de son semi-automatique Luger P08 est plein à craquer, puis se dirige d’un pas ferme vers la chambre d’hôtel où il sait que les amants ont l’habitude de se retrouver pour donner libre cours à leur frénésie sexuelle. Là, il les trouve au lit, en train de forniquer comme des bêtes, et la vision de cette espèce d’animal velu en train de prendre sa femme en levrette le pousse à commettre l’irréparable. Il l’abat d’une balle en pleine tête, chose qui, et c’est pour dire à quel point cet employé du gaz est anormalement constitué, n’entrave en rien sa vigueur sexuelle. Même mort, la tête à moitié arrachée, il continue à limer furieusement la malheureuse couverte de sang et de cervelle qui hurle tant et plus. Désireux de mettre un terme à ses souffrances, notre homme l’abat à son tour d’une balle dans la tête. Mais le monstre, loin de s’arrêter à ce genre de contretemps, continue à besogner le cadavre comme si de rien n’était. Le malheureux, qui n’est alors pas loin de perdre la raison, vide son chargeur sur le zombie qui est toujours en train de s’acharner sur la dépouille de sa volage moitié. Voyant que tout cela est sans effet, et comprenant que le monstre, qui le dévisage en se léchant de façon obscène les babines ensanglantées, et au sujet duquel j’ai indiqué précédemment qu’il était à voile et à vapeur, ne va pas tarder à s’en prendre sexuellement à lui, il recharge son arme et se fait sauter la cervelle. Et, détail sordide que je ne pas encore eu le courage de vous révéler, Desmarais père ne s’est pas rendu seul à cette expédition punitive qui vient de tourner au cauchemar. Non, il a emmené avec lui son fils Noé, treize ans moment des faits, afin qu’il soit témoin de l’indignité de sa mère. Le petit Noé, donc, a assisté à tout. Butalement plongé jusqu’aux ouïes dans la triste réalité d’une existence qu’il n’avait pas choisie, il a vu sa mère adorée mourir des mains de son père, puis son père se donner la mort pour échapper aux assauts de l’amant de sa femme transformé en zombie. Lui-même, Noé, n’a dû son salut qu’à une fuite effrénée dans les couloirs de l’hôtel, avant que le zombie ne soit finalement carbonisé au lance-flammes par le GIGN. On peut comprendre, et je suis tout disposé à le faire, que de tels événements aient influencé durablement le cours de son existence et précipité sa chute. On peut comprendre que sa raison en ait été ébranlée, et qu’il ait développé un certain penchant pour les arts du feu et la pyrotechnie. Néanmoins, quels que soient le trauma originel, l’éducation pourrie et l’état de dégradation mentale auquel il était parvenu, on ne pouvait continuer à le laisser sévir impunément. Face à un détraqué de ce calibre, on ne pouvait en aucun cas se permettre de miser sur la clémence ou les bons sentiments. On aurait pu le mettre en prison, bien sûr, et il aurait passé le restant de ses jours à ruminer sa vengeance. Mais à quoi bon, puisqu’il était irrécupérable, et que même s’il l’avait été, ça aurait de toute façon été un boulet accroché au pied de la société. Même s’il avait cessé de foutre le feu aux gens, il aurait continué à répandre ses idées malsaines et participer à la déliquescence ambiante. On aurait aussi pu le foutre en HP et le bourrer de médocs jusqu’à sa mort. Dans un cas comme dans l’autre, à part faire bosser les toubibs et les gardiens, le retour sur investissement aurait été nul. En humaniste convaincu, je suis et ai toujours été un farouche adversaire de la peine de mort, en ce sens que la mort n’est pas une peine mais une nécessité, au mieux un incident de parcours. C’est sur la mort des uns, le terreau de leur cadavre, que fleurit la vie des autres. Et je suis d’avis que la société ne doit en aucun cas se salir les mains en effectuant les sales besognes nécessaires à sa survie. C’est à certains citoyens, plus dévoués que d’autres à la cause commune, de s’y coller, et la justice n’a rien à voir là-dedans. On peut m’objecter que ce n’est pas à moi de décider qui doit ou non cesser de respirer. Ce serait vrai si je décidais de quoi que ce soit, mais ce n’est pas le cas. Je ne décide de rien du tout, pas plus que le marteau ne décide de taper sur la tête du clou. Je fais ce que j’ai à faire en toute décontraction, en toute humilité. J’essaie de le faire bien, en bon artisan respectueux de la tradition et de ses outils, à l’image de mon père qui, après l’armée, avait œuvré un temps comme tueur à gage pour un caïd de la drogue, avant de partir planter ses choux dans des contrées plus vertes. Même s’il était au service de l’une d’entre elles, et non des moindres, mon père n’exécutait que des ordures, raison pour laquelle j’ai toujours considéré qu’il exerçait une profession de salubrité publique. D’autant qu’il travaillait proprement, sans excès de zèle, ce qui explique sans doute que personne ne soit jamais revenu de l’au-delà pour se plaindre de ses agissements. Comme lui, à ceci près que je fais dans le bénévolat et n’attends aucune espèce de reconnaissance de la part de qui que ce soit, je supprime certaines personnes peu fréquentables parce qu’il faut bien que quelqu’un le fasse, et que cette mission m’a été dévolue par des forces supérieures sur lesquelles je n’ai aucune autorité. Et non, je ne suis pas ce genre de cinglé qui entend des voix lui murmurer dans le creux de l’oreille qu’il doit prendre les armes et endosser sa plus belle, blanche et étincelante armure pour aller défendre la veuve et l’orphelin. Personne ne me parle, ne me dit quoi faire. J’agis en toute simplicité, aussi naturellement que l’agneau tète sa mère avant que le loup ne la croque, en toute simplicité lui aussi.

Cela dit, pour en revenir à la situation présente, je n’avais rien contre le fait que le destin fasse le boulot à ma place. Après tout, je n’étais pas responsable de l’existence de cette tête de nœud, et n’avais donc aucune raison particulière de m’infliger la corvée de le faire disparaître.

Comme beaucoup d’urgentistes souffrant de handicap visuel, le docteur Sébastien Charrier portait des lunettes Lazarus Pierson en titane avec verres antibuée, antireflet, antitout, charnières flexibles et plaquettes nasales antidérapantes, spécialement conçues pour procurer le maximum de confort aux membres du corps médical. L’urgentiste de terrain, confronté à des conditions de travail souvent difficiles, se doit d’être parfaitement équipé pour donner le meilleur de lui-même. Charrier était le genre d’homme qui s’entretient physiquement et intellectuellement pour être toujours au top des ses capacités. Il était marié et père de trois enfants. Sa femme le trompait avec son gynécologue, le docteur Rémi Durand, qui était aussi le meilleur ami de son mari, mais il s’agit là d’un cas de figure tellement banal que de nos jours plus personne n’y prête attention. D’autant qu’on ne pouvait pas dire à proprement parler qu’elle le trompait avec le docteur Rémi Durand, puisque le docteur Charrier était parfaitement au courant de ses agissements. Non seulement il était au courant, mais lui-même couchait avec la femme du docteur Durand, son meilleur ami, lequel meilleur ami était bien entendu lui aussi parfaitement au courant des agissements de son épouse. Charrier était couvert de poil, comme un singe, et sa libido était semblable à une bête sauvage qu’il avait toutes les peines du monde à tenir en laisse. Déjà, quand il était petit, ses parents faisaient venir à la maison des amis avec lesquels ils organisaient des orgies sexuelles où tout le monde était convié, y compris les enfants. Il ne s’agissait pas d’inceste à proprement parler, mais le fait est que sa mère et la plupart de ses frères et ses sœurs l’avaient sucé à de nombreuses reprises. Même son grand-père, qui ne crachait pas sur les petits garçons à l’occasion, lui avait plus d’une fois sucé la bite, tout comme lui-même, dans un souci d’équité et de retour d’ascenseur, avait plus d’une fois sucé la bite et avalé la semence grand-paternelle. C’est dans ce contexte qu’il avait décidé, comme son père, lui-même chirurgien de renom, et son grand-père avant lui, généraliste unanimement apprécié pour ne pas dire vénéré, de consacrer sa vie à sauver celle des autres. Mais quand on passe le plus clair de son temps à côtoyer la mort, il faut bien se détendre un peu en rentrant chez soi. Tout le monde l’avait bien compris, raison pour laquelle des partouzes étaient régulièrement organisées chez les Charrier ou les Durand, tous deux possesseurs de vastes maisons de maîtres dans la campagne environnante. Les Charrier, par exemple, étaient les heureux propriétaires d’un château du XVIIIe (quinze pièces, parc paysager de deux hectares, piscine et dépendances), dans le Loiret. Naturellement, quand on le voyait débarquer comme ça sur les lieux d’une catastrophe, on ne pouvait pas se douter que le docteur Sébastien Charrier était obsédé par les jeunes et jolies infirmières qui gravitaient autour de lui. Tellement qu’il lui arrivait de leur proposer, pour arrondir leurs fins de mois (tout le monde sait que le métier d’infirmière, qui exige non seulement des compétences médicales, mais aussi beaucoup de patience et d’empathie face à des gens en situation de stress et de faiblesse extrême, est très insuffisamment rémunéré et ne suscite plus guère de vocations), de venir passer le week-end avec lui et ses amis à la campagne. Tout ce qu’elles avaient à faire était de venir dans leur tenue de travail, autrement dit en petite tenue sous leur blouse (en coton de préférence, dessous sexy appréciés), et de s’occuper gentiment des pensionnaires. Il leur était également demandé de ne faire aucune différence entre homme et femme, de quelque âge, condition sociale (elles n’avaient pas de souci à se faire, il n’y aurait que des gens bien élevés et très à l’aise financièrement) ou origine ethnique que ce soit, les séminaires en question étant placés sous le signe de la confraternité intergénérationnelle, culturelle et sexuelle entre les peuples. Toute pratique, tant qu’elle ne portait pas (gravement, on ne pouvait jamais totalement exclure tel ou tel suçon ou légère trace de morsure, lesquels, de toute façon, feraient l’objet d’un dédommagement adapté au préjudice) atteinte à l’intégrité physique de la personne, devait être acceptée et satisfaite dans la joie et la bonne humeur la plus exubérante. Si elles respectaient à la lettre ces quelques directives, leur train de vie pourrait connaître une embellie spectaculaire. À elles le coupé sport, les robes de princesses, les bijoux chatoyants, les restaurants étoilés et les voyages en première classe aux Seychelles et à Bora Bora. La plus stricte discrétion était bien entendu de mise, sachant que la plupart des gens sont bien trop étroitement cadenassés dans un carcan de valeurs morales d’un autre âge pour admettre que certains aient impunément accès à des plaisirs auxquels ils n’auront jamais droit.

Jusqu’au jour où l’une des infirmières en question, une certaine Alena Benesch (blonde, le teint pâle, avec des grands yeux noisette et un visage d’ange tombé du ciel), est allée porter plainte au commissariat le plus proche.

Et devinez qui se trouvait dans ce commissariat le plus proche ?

Votre serviteur.

Et c’est à lui qu’on a refilé la patate chaude.

Vêtu de mon plus chouette costume, mon plus charmant sourire aux lèvres, accompagné d’une poignée de flics en uniforme pour renforcer l’aspect solennel de la démarche, je suis allé sonner à la porte de Charrier, dans les beaux quartiers, là où toutes les baraques, ceintes de hauts murs hérissés de tessons de bouteilles ou de clôtures électriques, sont équipées d’alarme dernier cri qui retentissent au moindre frémissement, déclenchant aussitôt le bouclage de la zone et le parachutage sur site des meilleurs éléments du RAID et du GIGN. Des maîtres-chiens lourdement armés patrouillent dans les rues, et des snipers surveillent H24 les alentours du haut de miradors placés à tous les coins de rues.

Il convient également, si l’on ne souhaite pas finir ses jours en fauteuil roulant, de se méfier des pièges à loup dans les parcs et mines antipersonnel dans les allées de jardin.

Eh bien croyez-le ou non, j’ai été reçu comme un cheveu sur la soupe.

C’est tout juste si on ne pas claqué la porte au nez.

En fait, ça n’aurait pas été pire si j’avais été un huissier de justice ou un témoin de Jéhovah .

Je parle d’un témoin de Jéhovah mâle, bien sûr, avec les oreilles décollées et un physique d’expert-compatible, ou une bonne femme de cinquante piges mal fagotée avec des dents jaunes et un fort strabisme divergent.

Parce que je vous fiche mon billet que si j’avais été un témoin de Jéhovah femelle avec la plastique de Scarlett Johansson, Jenna Ortega, Rachel McAdams ou Gal Gadot, le tout emballé dans une blouse d’infirmière ultra sexy avec les boutons prêts à exploser sous la pression des formes généreuses qu’elle s’efforçait désespérément de contenir, on m’aurait déroulé le tapis rouge, offert des fleurs, un verre de Champomy, et aussitôt proposé un emploi à plein temps comme garde d’enfants à domicile, gouvernante, femme de ménage, jardinière de légumes ou n’importe quoi d’autre pour me garder à portée de braguette et tenter de m’entraîner corps (surtout) et âme dans la spirale du vice.

Poussé dans ses retranchements, Charrier a admis qu’il lui arrivait de recevoir des filles dans son château du XVIIIe, mais que celles-ci étaient majeures, traitées avec les mêmes égards que les autres invités, sans que ne soit jamais exigée aucune contrepartie de leur part.

À l’époque j’étais comme toi, lecteur, jeune et fougueux inspecteur, épris de justice et de liberté, révolté à l’idée que les puissants abusent de leurs prérogatives en toute impunité. Je rêvais, à cheval sur mon blanc destrier, engoncé dans ma plus resplendissante armure et coiffé de mon plus bel heaume, de voler au secours de la veuve et l’orphelin en proie aux persécutions de nobles dévoyés, d’usuriers pervers et de crapules sans foi ni loi. D’étranges rumeurs circulaient au sujet du château de la Frétoise, près de Montargis, entre Préfontaines et Corquilleroy.

Quelques années auparavant, une fille avait disparu sans laisser de traces dans le secteur. Son vélo, ainsi qu’une de ses chaussures et une pince à cheveux, avaient été retrouvés sur la route de Nargis, au lieu-dit du Bois au Notaire. La pauvre enfant venait tout juste de fêter ses dix-sept ans. Les recherches n’ont rien donné, mais les allées et venues nocturnes du côté de la Frétoise avaient éveillé les soupçons de plusieurs personnes du voisinage, même si l’endroit est particulièrement isolé. Le propriétaire des lieux, un certain Sébastien Charrier, avait été entendu. Il avait déclaré n’être au courant de rien, mais s’était montré des plus désagréables, traitant les enquêteurs avec dédain, n’hésitant pas à leur faire sentir qu’ils n’étaient que des moins-que-rien indignes de sa compagnie, des pauvres types auxquels il ne se résignait à adresser la parole que contraint et forcé. Le même sort m’a été réservé lorsque je l’ai entendu à mon tour, suite au dépôt de plainte de l’infirmière qui affirmait avoir été violentée durant son séjour au château. Un soir, par exemple, après un repas bien arrosé, on l’avait trainée de force dans le salon et obligée à se dévêtir entièrement devant les invités, tous complètement bourrés et incapables de la moindre retenue. Elle se souvient que certains, à commencer par le docteur Charrier lui-même, avaient le sexe à l’air et se masturbaient outrageusement devant elle. C’est ce même docteur Charrier qui lui avait ensuite copieusement aspergé les parties génitales avec de la crème chantilly, avant de faire venir Hermann, son dogue allemand, pour qu’il nettoie la zone à grands coups de langue sous les éclats de rire et les encouragements obscènes de l’assistance. Ames sensibles s’abstenir. Je vous laisse néanmoins imaginer, si toutefois vous en avez le courage, la terreur de cette jeune femme, livrée à la frénésie d’un monstre de près de quatre-vingt-dix kilos qui aurait très bien pu ne pas se satisfaire de cet amuse-gueule et décidé de passer sans plus tarder au plat de résistance. Quelle horreur ! Si vous ajoutez à cela le caractère extrêmement humiliant et dégradant de la scène, vous transpirez à grosses gouttes et prenez aussitôt la mesure du traumatisme subi.

Naturellement, cette enflure de Charrier a nié les faits avec la dernière énergie, affirmant que la fille mentait pour se faire du fric sur son dos. Selon lui, ce tissu de conneries grosses comme un troupeau d’éléphants risquait de jeter un voile crasseux sur la blancheur immaculée de sa réputation. En conséquence, il a menacé de porter plainte en retour pour diffamation. Son avocat était le genre de Pavarotti du barreau capable de faire passer un curé pédophile pour un honnête serviteur de Dieu. Il a également claironné qu’il allait s’empresser de solliciter ses relations dans les plus hautes sphères de l’État, afin que celles et ceux qui avaient prêté une oreille complaisante à ces élucubrations soient sanctionnés à la hauteur de leurs agissements. Peu de temps après, j’ai été convoqué chez le dirlo qui m’a gentiment expliqué que Charrier était un type comme ça, un héros des temps moderne qui ne comptait pas ses heures pour sauver la vie des gens. Charrier n’était pas le Gilles de Rais de Montargis, le Barbe Bleue de la Frétoise, l’Ogre de Préfontaines qui nourrit ses molosses avec de la chair humaine, et il fallait séance tenante arrêter de l’emmerder avec cette histoire de dogue allemand bouffeur de chatte à la chantilly. Tout cela ne tiendrait pas une seconde devant les tribunaux. D’autant que l’infirmière en question, même si elle en avait toutes les apparences, était loin d’être une princesse de conte de fée, aussi pure et innocente que la rosée du matin ou la fleur des champs fraîchement éclose. Ses états de service faisaient mention d’un certain nombre de délits qui cadraient assez mal avec le numéro de novice de couvent des Ursulines qu’elle avait tenté de nous faire avaler : excès de vitesse, conduite en état d’ivresse et usage de stupéfiants.

Résultat des courses : Alena Benesch a retiré sa plainte et Charrier s’en est sorti sans une égratignure, lavé de tout soupçon, blanc comme la neige qui recouvrait jadis les vastes plaines de notre enfance.

Naturellement, Benesch a été gentiment remerciée par la Direction et priée d’aller exercer ses talents de garde-malade dans un autre établissement, si possible dans une autre galaxie, à des années-lumière de la planète Terre. Charrier, grand seigneur, lui a versé un petit pécule pour l’aider à tenir le coup en attendant de trouver un nouveau job. Pas rancunier pour un sou, il lui a glissé dans le creux de l’oreille qu’il y aurait toujours un bol de soupe pour elle à la Frétoise. Et pas seulement de la soupe : il avait fait le plein de chantilly, et Hermann, au bord de la dépression depuis son départ précipité, la réclamait avec insistance.

Tu te demandes peut-être, ami lecteur dont je connais la sagacité et la soif de transparence, comment je suis au courant de tous ces détails concernant la vie privée de Charrier, notamment son enfance dévoyée au sein d’une famille sexuellement dysfonctionnelle ?

La réponse est simple : quand j’enquête, j’enquête, ce qui signifie que je passe au crible tous les éléments ayant trait à l’enquête en question, y compris les plus insignifiants. J’effectue des recoupements, fouille dans les tiroirs, les archives, les boîtes à chaussures, les armoires à linge et les cartons à chapeau, épluche les livrets de comptes, les carnets de liaison, les journaux intimes, visite les jardins secrets, déterre les cadavres et plonge tête baissée dans les profondeurs existentielles des protagonistes de l’affaire. Ce travail de fourmi, long et ingrat, me permet de débusquer sans coup férir l’ivraie qui se dissimule au sein du bon grain. Ainsi, Charrier avait des frères et sœurs avec lesquels il n’était peut-être pas forcément dans les meilleurs termes du monde, au point qu’il ne parlait quasiment plus à certains d’entre eux. C’est vers eux qu’il fallait se tourner pour obtenir de précieuses informations sur un frère qui ne leur inspirait plus que mépris et répulsion. Il devenait alors possible de reconstituer pas à pas le parcours d’un Charrier, comprendre comment avait pu s’échafauder dans son cerveau malade cette passion dévorante pour les dogues allemands, les infirmières et la chantilly.

Voilà comment, ami lecteur, j’ai pu apprendre toutes ces choses passionnantes sur la vie du docteur Sébastien Charrier, et te donner ainsi l’impression que j’étais dans le secret des dieux, alors que tout cela n’était finalement rien d’autre que le fruit d’un travail lent et minutieux. Cela dit, oui, je peux aussi être un tout petit peu dans le secret des dieux, car, sauf le respect que je te dois, je ne suis pas non plus obligé de tout te dire. C’est encore moi le seul maître à bord de ce rafiot littéraire qui vogue sans relâche sur les flots écumeux de la syntaxe, la rhétorique, la synecdoque, l’oxymore, la litote, la métaphore, l’antonomase, l’allégorie, le truisme et le paradoxe. Donc oui, je ne suis pas obligé de tout de dire, et peux choisir à mon gré le moment de te dévoiler ou pas tel ou tel aspect de l’histoire qui nous occupe.

Charrier m’a immédiatement reconnu.

Il a sorti de sa poche une seringue remplie d’un liquide verdâtre qu’il a essayé de me planter dans le bras.

Je ne sais toujours pas quel était ce liquide verdâtre de merde, de la pisse de rat, du jus de cadavre ou autre chose, mais quelque chose me dit que je serais mort dans d’atroces souffrances s’il avait réussi à me l’injecter.

J’ai esquivé le coup, une courte lutte s’est engagée, à l’issue de laquelle il s’est retrouvé au sol.

J’ai tenté de le ramener à de meilleurs sentiments avec un bon coup de latte dans les parties, mais il a réussi à m’attraper le pied, le tordre et me faire perdre l’équilibre.

Je me suis retrouvé au sol à mon tour, la cheville endolorie, pendant que Charrier se relevait d’un bond avec l’élégance d’un sportif de haut niveau.

Greg s’est jeté sur lui pour l’étrangler, mais Charrier l’a vu arriver et reçu avec une série de coups qui l’ont laissé sans voix, notamment le crochet au foie et l’uppercut à la mâchoire, tous deux assénés avec une précision d’autant plus redoutable que l’anatomie n’avait aucun secret pour lui.

Greg, faisant sienne la devise de l’inspecteur Harry Callahan dans Magnum Force, «~l’homme sage est celui qui connaît ses limites~», n’a pas jugé utile de se révéler après avoir mordu la poussière.

Dans une autre vie, Charrier avait été champion de France universitaire de boxe anglaise. Il avait perdu en vitesse, son jeu de jambes n’était plus ce qu’il était et ses réflexes s’étaient quelque peu émoussés, mais il lui restait encore largement de quoi faire illusion sur un ring.

Pas besoin d’être diplômé de l’ESSEC et encore moins d’être le principal actionnaire de la Royal Caribbean Cruises Ltd. pour comprendre que la situation était en train de se barrer méchamment en couille.

Il m’a fallu développer des trésors de résistance à la douleur et de volonté farouche de survivre dans ce monde cruel qui est le nôtre pour réussir enfin à retrouver cette position qui, au même titre que le cheval, la femme et le dromadaire, compte au rang des plus belles conquêtes de l’homme, je veux bien sûr parler de la bipédie. Grâce à elle, nous avons tout le loisir de conserver la pleine et entière jouissance de nos membres supérieurs, nos mains en particulier, ce qui nous a permis d’accomplir des miracles qui seraient à tout jamais restés hors de portée si nous avions été contraints de nous déplacer à quatre pattes.

C’est alors que j’ai vu Charrier se diriger à grands pas vers moi, les poings serrés et les traits horriblement déformés par la haine, l’envie de me détruire entièrement, me mouliner, me torréfier, me hacher menu, me réduire en cendre, m’éradiquer définitivement de la surface de la Terre (510 millions de kilomètres carrés tout de même, dont 70\% de flotte, rivières, lacs et profondeurs océaniques peuplés de créatures aussi étranges que primitives).

La situation était d’autant plus préoccupante qu’il avait un scalpel à la main, instrument dont les qualités de tranchant ne sont plus à démontrer.

Greg au sol, et apparemment bien décidé à y rester, je ne pouvais plus compter que sur moi.

Et Manu, bien sûr, fidèle serviteur de la Loi qui ne m’avait jamais trahi, ne s’était jamais enrayé, n’avait jamais connu la moindre avarie en plus de vingt ans de bons et loyaux services, vingt longues années d’épreuves traversées côte à côte, la main (ou la crosse, si vous préférez) dans la main, le doigt sur la détente.

Il y a des moments, dans l’existence, où l’heure n’est plus aux conciliabules, atermoiements et autres vaines tergiversations.

Quand l’ennemi fond sur vous, l’écume aux lèvres, et que vous êtes clairement en infériorité numérique, le mieux est encore de faire feu sans se poser de question.

C’est ce que j’ai fait, à deux reprises.

Le premier projectile a raté sa cible, à savoir la tête de Charrier, mais le second a fait mouche.

Son crâne s’est ouvert comme un œuf à la coque, et sa cervelle a été projetée dans les airs, tel un drôle d’objet volant mal identifié.

Elle a effectué quelques tours sur elle-même, avant d’atterrir sur l’épaule d’un bonze qui circulait en trottinette électrique sur le trottoir d’en face. Soyons clair : j’ai eu beau poser la question à tous les témoins qui avaient assisté à la scène, et dieu sait qu’il y en avait un paquet, aucun n’a été en mesure de m’expliquer ce que ce foutu bonze faisait là, en toute illégalité qui plus est, ce qui n’est à priori pas dans le style des bonzes, toujours respectueux des lois, adeptes de la discrétion et désireux de se fondre dans la foule (même si l’espèce de soutane orange dans laquelle ils se trimballent n’est sans doute pas le meilleur moyen d’y parvenir), sachant qu’on n’avait pas vu de bonze dans le secteur depuis au moins trois ou quatre siècles, en admettant qu’on en ait jamais vu un, raison pour laquelle personne ne s’attendait à en voir un, et encore moins au guidon d’une putain de trottinette électrique, le bonze n’étant généralement pas pressé et préférant faire usage de ses pieds, à l’ancienne, pour se transporter d’un point à un autre.

Le bonze a tourné la tête, vu la cervelle sur son épaule, poussé un cri (oui, les bonzes aussi poussent des cris, peut-être pas autant que les gens normaux, les anachorètes ou les membres des autres congrégations, mais ils en poussent aussi), tenté de s’en débarrasser, perdu le contrôle de sa trottinette, réussi de justesse à éviter un bac à fleurs, avant d’aller s’écraser sur une borne anti-stationnement, effectuer un vol plané d’anthologie et se retrouver les quatre fers en l’air au beau milieu de la chaussée. Tandis qu’il tentait de se relever, un bus est arrivé à pleine vitesse et lui a roulé sur la tête, laquelle a explosé comme une vieille citrouille pourrie en répendant son contenu sur le bitume.

Coupez !

En fait non, les choses ne sont exactement passées de cette manière.

Vous le savez comme moi, la réalité est une tambouille désespérément fade. Si on veut lui donner un semblant de saveur, il ne faut pas lésiner sur les épices. C’est d’ailleurs ce que la plupart des gens, dès qu’ils ont un moment de libre, s’emploient à faire.

Le mensonge, par exemple, ou le fait de travestir plus ou moins subtilement la vérité, sont des pratiques courantes en la matière.

Quand Roger Borniche (ex-comique troupier reconverti en flic à la Sûreté puis écrivain à succès) raconte comment il a serré Émile Buisson, alias l’ignoble Fatalitas, ennemi public numéro 1, dans la petite auberge de Normandie où ce dernier était tranquillement en train de déjeuner (pâté de campagne, avec salade et cornichons, tripes au calva, camembert et tarte aux fraises, il s’agit d’un scoop mondial puisque le contenu de ce déjeuner n’avait encore jamais été dévoilé, on se demande d’ailleurs bien pourquoi quand on sait à quel point les gens sont friands de ce genre de détails), il enjolive copieusement la scène, allant jusqu’à faire croire que c’est sa propre femme, Martine, qui a passé les menottes à Buisson (alors, je le rappelle, que deux autres flics, dont c’est le métier de menotter les gens, étaient présents dans la salle). Sacré Roger ! Non, en réalité, Martine n’était là que pour endormir la méfiance de Buisson, individu extrêmement dangereux en permanence sur le qui-vive, et en aucun cas risquer de prendre un mauvais coup en procédant elle-même à son arrestation.

Moi-même, en l’occurrence, qui n’ai rien à envier à Roger Borniche, à ceci près (manquerait plus que ça !) que je n’ai jamais été comique troupier, chansonnier ou agent de sécurité dans un grand magasin (il n’y a pas de sot métier, je vous l’accorde, mais ça montre bien à quel point l’approche du maintien de l’ordre était différente en ce temps-là, même si aujourd’hui encore il n’est pas rare que d’anciens acteurs de seconde zone accèdent aux plus hautes fonctions de l’État), ne rechigne pas à mettre un peu de piment dans le ragoût fadasse de l’existence, quitte à rétablir, une fois la supercherie découverte, l’exacte vérité des faits.

Dans le cas présent, je suppose que la présence d’un bonze en trottinette a dû sembler bizarre aux plus méfiants d’entre vous, d’autant qu’il n’y a aucun monastère dans les environs, et que même s’il y en avait un, il n’est pas du tout certain que les moines auraient l’autorisation d’utiliser un tel moyen de locomotion.

Je vous rassure tout de suite : il n’y en avait pas.

Pas à ma connaissance, en tout cas.

Par contre, le docteur Sébastien Charrier, lui, était bien là.

Je lui ai dit : Docteur Charrier ! Si je m’attendais à vous trouver ici !

\textsc{Charrier} : On se connaît ?

\textsc{Moi} : Vous ne vous rappelez pas ?

\textsc{Charrier} : Me rappeler de quoi ? On s’est déjà vu quelque part ?

\textsc{Moi} : L’affaire Alena Benesch, ça vous dit quelque chose ?

\textsc{Charrier} : Attendez voir… Mais oui, bien sûr, cette petite pute qui avait essayé de me faire chanter en prétendant que je lui avais fait bouffer la chatte par Hermann, mon dogue allemand !

\textsc{Moi} : Oui. C’est moi qui étais chargé de l’enquête. Il va bien, au fait ?

\textsc{Lui} : Qui ? Hermann ?

\textsc{Moi} : Oui.

\textsc{Lui} : La pauvre bête est morte de chagrin il y a quelques années de cela. Je crois qu’elle s’était beaucoup attachée à cette petite. Le coup de foudre existe aussi chez les animaux, vous savez.

\textsc{Moi} : J’ignorais.

\textsc{Lui} : Oui, je sais, on pense toujours que ce ne sont que des brutes épaisses incapables de sentiment. Eh bien il n’en est rien, ils sont beaucoup plus sensibles qu’on ne le pense.

\textsc{Moi} : Je suis vraiment désolé.

\textsc{Lui} : C’est gentil à vous. Je l’ai enterré au fond du jardin et vais quotidiennement me recueillir sur sa tombe.

\textsc{Moi} : C’est l’avantage d’avoir une grande propriété.

\textsc{Lui} : Vous connaissez la Frétoise ?

\textsc{Moi} : J’y suis allé une fois ou deux. Très bel endroit, à la fois authentique et élégant.

Lui, manifestement ému : Oui, un joyau historique rénové avec passion au cœur d’un environnement préservé. Nous sommes actuellement en train d’aménager le colombier pour y faire une chambre d’ami. C’est une bonne idée, vous ne trouvez pas ?

\textsc{Moi} : Excellente

\textsc{Lui} : Il faudra venir dîner un de ces soirs. Vous êtes marié ?

\textsc{Moi} : Non, pas encore.

\textsc{Lui} : Une fiancée, alors. Ravissante, je suppose. Il faudra penser à nous l’amener.

\textsc{Moi} : Je n’y manquerai pas.

\textsc{Lui} : Quoiqu’il en soit, pour en revenir à cette petite garce d’Alena Benesch, je continue de penser qu’elle n’a eu que ce qu’elle méritait. Il m’est arrivé de penser que j’avais peut-être été un peu trop dur avec elle. Je ne suis pas un mauvais homme, vous savez, et toujours de mon mieux pour réparer mes torts. Si torts il y a, bien entendu. La vérité, c’est que je ne me sens coupable de rien en ce qui la concerne. Toujours est-il que l’autre jour, croyez-le ou non, elle est venue sonner à ma porte en disant qu’elle était dans une misère noire et avait besoin d’un petit coup de main.

\textsc{Moi} : Et qu’est-ce que vous avez fait ?

\textsc{Lui} : Ça reste entre nous, mais ma chère épouse, qui est une vraie salope soit dit en passant, adore me regarder baiser avec une autre femme. On a tous nos petites manies, n’est-ce pas. Alena Benesch a pris un peu de poids, c’est vrai, mais elle est encore tout à fait comestible. Je lui ai proposé de la sodomiser sous les yeux de ma femme, si elle n’avait rien contre le fait de se faire défoncer le cul par un ancien interne des hôpitaux de Paris. En échange, je pourrais essayer de faire jouer mes relations pour qu’elle retrouve un poste dans une clinique privée. J’ai un ami qui adore les femmes plutôt bien en chair, un violeur notoire qui a pour habitude d’abuser de ses patientes quand elles sont dans les vapes. Je ne sais que ça ne se fait pas, mais c’est un ami et je ne me vois pas le balancer aux flics. D’autant que la plupart d’entre elles ne se souviennent de rien à leur réveil. Je vous choque ?

\textsc{Moi} : Un peu, oui. Et celles qui se souviennent, je peux savoir ce que vous en faites ?

\textsc{Lui} : On les envoie chez le psy.

\textsc{Moi} : Un ami à vous, je suppose ?

\textsc{Lui} : Evidemment. Comme elles n’ont que de très vagues souvenirs sur lesquels elles sont incapables de mettre un nom ou un visage, le psy leur explique qu’elles sont en pleine bouffée délirante, sans doute liée à l’un ou l’autre de ces traumas d’enfance mal gérés, ou pas gérés du tout parce que totalement passés sous les radars, qui refont surface après des années d’enfouissement, comme des saletés de zombies qui refoulent du bec et tentent de vous bouffer tout cru. Mais dieu merci, on n’est plus au Moyen Âge. De nos jours, on peut être cinglé sans se retrouver en train de griller sur un bûcher. Les chercheurs bossent comme des dingues, pour des salaires de misère, et nous disposons de molécules de plus en plus sophistiquées pour remettre un peu d’ordre dans les cerveaux détraqués.

\textsc{Moi} : N’empêche qu’il abuse d’elles.

\textsc{Lui} : Oui, on peut dire ça. Mais à ce moment-là, on peut aussi dire qu’il les viole avec ses doigts en procédant aux examens d’usage.

\textsc{Moi} : Oui, enfin, ce n’est pas tout à fait la même chose. Là, les viole carrément avec sa bite.

\textsc{Lui} : Je sais, c’est moche. Très moche, même, mais il prétend avoir un meilleur diagnostic avec sa verge qu’avec les autres outils dont il dispose, trop grossiers à son goût. Je sais que c’est faux, qu’il ment, qu’il se ment à lui-même, mais il n’est pire sourd que celui qui ne veut rien entendre. J’ai beau essayer de lui ouvrir les yeux, il s’obstine dans le déni. Je lui dis : Roman (il s’appelle Roman), mon ami, je t’en supplie, va voir un psy. Un jour ou l’autre, une patiente va se réveiller pendant que tu es en train de l’ausculter avec ta bite, en tout bien tout honneur, et il va en résulter un de ces putains de scandales qui éclaboussent la profession toute entière. Pense à tes collègues, tous ces gens qui se battent becs et ongles pour que les gens se bourrent de médocs jusqu’à cent ans. Il me répond : oui, je ferais bien quelques séances d’hypnose, mais j’ai peur de me faire violer pendant mon sommeil. Je ne fais aucune confiance à tous ces enfoirés de psys ! Vous le voyez, on n’en sort pas. À propos, vous êtes toujours dans la police ?

\textsc{Moi} : Oui, plus ou moins.

\textsc{Lui} : Dans ce cas, je compte sur votre discrétion. Vous savez ce que c’est : les gens sont méchants, ils n’aiment pas les riches. Dès que vous gagnez un peu plus de pognon qu’eux, ils font tout ce qui est en leur pouvoir pour vous mettre des bâtons dans les roues. Heureusement qu’ils n’en ont aucun, sinon celui de descendre dans la rue pour agiter des banderoles et se gargariser de slogans anticapitalistes, sans quoi ils seraient pires que tous ces dictateurs qui dirigent le monde d’une main de fer. Je peux savoir ce qui s’est passé, ici ?

\textsc{Moi} : Accident de la circulation.

\textsc{Lui} : Belle boucherie !

\textsc{Moi} : Oui. Vous pensez qu’elle s’en sortira ?

\textsc{Lui} : Contusions multiples, hémorragie interne au niveau de la cavité abdominale, possible trauma crânien, j’en passe et des meilleurs. Elle est dans la coma. Nul ne peut dire quand elle en sortira, si elle en sort, et encore moins dans quel état. Vous lui vouliez quoi, à cette petite ?

\textsc{Moi} : L’interroger. J’ai de bonne raison de penser qu’elle est impliquée dans la disparition d’un collègue de travail, qui se trouve aussi être un de mes plus proches amis.

\textsc{Lui} : J’ai moi-même perdu un excellent ami.

\textsc{Moi} : Vraiment ?

\textsc{Lui} : Oui, très récemment. Il s’appelait Rémi Durand et était gynécologue.

\textsc{Moi} : Ce nom me dit quelque chose.

\textsc{Lui} : Bien évidemment, que ça vous dit quelque chose. Si vous avez enquêté sur la Frétoise, vous n’ignorez pas que Rémi y passait quasiment tous ses week-ends, plus une bonne partie des vacances scolaires et ses congés de maternité.

\textsc{Moi} : Je vois. Qu’est-ce qui s’est passé, au juste ?

\textsc{Lui} : On l’a retrouvé pendu dans son garage.

\textsc{Moi} : Pendu ???!!!!!!!!!!

\textsc{Lui} : Dans son garage, oui.

\textsc{Moi} : D’après mon expérience personnelle, qui est tout de même loin d’être négligeable, il est assez rare que les gens se pendent dans leur garage, ou alors seulement s’ils sont victimes de harcèlement sur les réseaux sociaux ou viennent d’apprendre fortuitement qu’ils sont atteints d’une maladie grave qui ne leur laisse que quelques heures à vivre. Ils ont généralement bien trop de respect pour leur voiture pour lui imposer une telle humiliation.

\textsc{Lui} : N’est-ce pas. Au lieu de ça, Rémi était en pleine forme et venait tout juste de s’offrir une Porsche Cayman GT4 RS dont il était extrêmement fier.

\textsc{Moi} : C’est troublant, en effet.

\textsc{Lui} : Sincèrement, mon ami, vous pensez vraiment qu’un type qui a une Porsche Cayman GT4 RS dans son garage a la moindre envie de mettre fin à ses jours ?

\textsc{Moi} : Quelle couleur, la Cayman GT4 RS ?

\textsc{Lui} : Jaune, avec pack Clubsport, réservoir de 90 litres et trousse de premiers secours en alcantara !

\textsc{Moi} : Une merveille.

\textsc{Lui} : Absolue ! Vous êtes comme moi, n’est-ce pas ?

\textsc{Moi} : Comment ça ?

\textsc{Lui} : Vous ne croyez pas un instant à la thèse du suicide.

\textsc{Moi} : Le suicide d’un gynéco bien dans sa peau qui vient de s’offrir une Porsche Cayman GT4 RS jaune avec pack Clubsport, réservoir de 90 litres et trousse de premiers secours ?

\textsc{Lui} : Oui, et toutes les options disponibles, l’intérieur full cuir bien évidemment, mais aussi les rétros extérieurs à capteurs de pluie, la reconnaissance des panneaux de signalisation, les tapis de sol en peau de fesse, les jantes en or massif, le système audio Bose et les ceintures de sécurité couleur chair !

\textsc{Moi} : Terrifiant ! J’ai beau tourner et retourner le problème dans tous les sens, je ne vois pas comment un type qui a tout ça dans son garage pourrait avoir la moindre envie de se suicider.

\textsc{Lui} : Et imaginez que ce même type soit marié à une jeune femme somptueuse dont il pourrait être le père ! Vous croyez vraiment qu’un tel homme aurait envie de mettre fin à ses jours ?

\textsc{Moi} : En aucun cas. Vous avez contacté le procureur ?

\textsc{Lui} : Bien évidemment, vous me prenez pour qui ! J’ai exigé qu’une expertise médico-légale soit diligentée dans les plus brefs délais, et j’ai demandé à y assister personnellement.

Moi, sortant un Hemingway Short Story de ma poche : Vous avez raison, on n’est jamais mieux servi que par soi-même. Vous fumez ?

\textsc{Lui} : Vous avez enquêté sur moi, non ?

\textsc{Moi} : Un peu, oui. Enquête de routine, vous savez ce que c’est.

\textsc{Lui} : Dans ce cas, vous devez savoir que je ne crache pas sur un petit cigare de temps à autre.

\textsc{Moi} : Et c’est tout à votre honneur. Tu en veux un aussi, Greg ?

\textsc{Greg} : Je ne voudrais surtout pas déranger.

\textsc{Moi} : Mais pas du tout, voyons, qu’est-ce que tu vas imaginer. C’est juste que comme je sais que tu ne fumes pas, ou quasiment pas, je me suis dit qu’il n’était peut-être pas complètement indispensable de te proposer un cigare.

\textsc{Greg} : Je ne fume pas en temps normal, c’est tout à fait vrai. Mais aujourd’hui, les circonstances sont suffisamment exceptionnelles pour que je fasse une exception à la règle.

\textsc{Moi} : C’est la raison pour laquelle, en dépit des éléments dont je viens de te faire part, que je me suis permis de te demander si tu voulais toi aussi un cigare.

\textsc{Greg} : Dans ce cas, je te répondrai ceci : oui, Djef, même s’il est vrai que je ne fume pas ou quasiment pas, c’est avec le plus grand plaisir que je vais accepter le cigare que tu m’offres si gentiment.

\textsc{Moi} : Bien. On en était où, nous ?

\textsc{Charrier} : Vous veniez de me proposer un cigare.

\textsc{Moi} : Que vous aviez accepté, c’est bien ça ?

\textsc{Lui} : C’est bien ça.

\textsc{Moi} : Et avant ?

\textsc{Lui} : Avant, on était en train de parler de la mort suspecte de mon ami Rémi Durand, gynécologue ayant pignon sur rue qui n’avait aucune raison de mettre fin à ses jours, et ce d’autant moins qu’il était marié à très jolie fille beaucoup plus jeune que lui et venait tout juste de s’offrir une Porsche Cayman GT4 RS jaune avec pack Clubsport, réservoir de 90 litres et trousse de premiers secours en alcantara, le nec plus ultra en matière de chic automobile.

\textsc{Moi} : Vous pensez qu’on l’a tué ?

\textsc{Lui} : Je ne vois pas d’autre explication.

\textsc{Moi} : Il s’agit peut-être d’un accident.

\textsc{Lui} : Vous plaisantez ?

\textsc{Moi} : On ne sait jamais. Imaginez un type qui décide d’aller dans son garage pour bricoler un truc au plafond, accrocher une corde, par exemple. Il monte sur un tabouret, et, pour une raison ou pour une autre, perd l’équilibre et se retrouve avec la corde enroulée autour du cou. Il se débat, tente désespérément de se raccrocher au tabouret. Mais il donne un coup de pied dans le tabouret, le tabouret tombe, et notre homme se retrouve pris au piège.

\textsc{Lui} : Ridicule !

\textsc{Moi} : Ou alors, il a peut-être eu, pendant un bref instant, l’idée de mettre fin à ses jours. On croit connaître ses amis, et on découvre parfois qu’ils nous ont caché des choses pendant des années. François Vérove, alias «~le Grêlé~», a vécu pendant trente-cinq sans attirer les soupçons. Ancien gendarme, bon père de famille, qu’est-ce que vous croyez qu’il s’est passé quand ses proches ont appris qu’il s’agissait en fait d’un tueur en série pédophile de la pire espèce, responsable d’au moins une bonne demi-douzaine de meurtres, et sans doute beaucoup plus, la liste exacte de ses victimes n’ayant jamais pu être établie avec certitude ?

\textsc{Lui} : Il sont tombés des nues, je suppose.

\textsc{Moi} : Peut-être que votre ami Rémi vous cachait des choses, lui aussi. Vous seriez surpris d’apprendre le nombre de gens qui ont une double vie.

\textsc{Lui} : Vous insinuez que Rémi était un tueur en série pédophile ?

\textsc{Moi} : Pas le moins du monde. Enfin, on ne sait jamais. Vous m’avez dit qu’il avait une femme beaucoup plus jeune que lui. Peut-être qu’elle le trompait et qu’il ne l’a pas supporté.

\textsc{Lui} : Elle ne l’a jamais trompé. Elle couchait avec d’autres hommes, c’est vrai, mais Rémi était parfaitement au courant et couchait lui aussi avec d’autres femmes, en toute transparence.

\textsc{Moi} : La vôtre, par exemple.

\textsc{Lui} : Par exemple. Mais j’ai souvent couché avec la sienne. J’ai toujours considéré comme une chose parfaitement normale que mes amis couchent avec ma femme, et m’autorisent à faire de même avec la leur. Je ne sais pas si vous êtes marié, lieutenant…

\textsc{Moi} : Commandant.

\textsc{Lui} : … commandant, mais si vous l’êtes vous devez savoir à quel point il est rébarbatif de coucher toujours avec la même femme, quel soit l’amour qu’on lui porte. L’amour et le sexe sont deux choses totalement différentes, qui ne devraient rien avoir à faire ensemble. Le désir est une chose, l’amour en est une autre, et la confusion qui règne entre les deux est la source de nombreux problèmes. Vous pouvez vivre avec quelqu’un toute votre vie, et continuer à l’aimer, mais certainement pas à le désirer comme au premier jour. Si certains y arrivent, tant mieux pour eux, mais moi ce n’est pas mon cas. Je ne vois pas au nom de quoi je devrais m’interdire d’être attiré par d’autres femmes et de coucher avec elles si le désir est réciproque. Quel spectacle pathétique de voir tous ces vieux types, mariés depuis des siècles à une femme qui ne ressemble physiquement plus à rien, baver comme des malades sur le cul des gamines qui passent à leur portée. Mais vous savez ce qui me fait le plus marrer ?

\textsc{Moi} : Non, dites-moi.

\textsc{Lui} : C’est de voir les jeunes mariés errer dans les rayons des supermarchés.

\textsc{Moi} : Mais encore ?

\textsc{Lui} : Madame marche en tête, et monsieur suit, l’air grognon et la mine déconfite, poussant un caddie rempli jusqu’aux ouïes de couches-culottes, lait en poudre et toute une ribambelle de produits ultra transformés bourrés d’huile de palme hydrogénée, amidon modifié, agents de texture, isolats de protéines et autres perturbateurs endocriniens diversement cancérigènes. Il a pourtant toutes les raisons d’être heureux, le jeune père de famille : il vient d’acheter un pavillon avec un petit lopin de terre pour faire pousser deux patates et trois petits pois, de changer de bagnole et d’accéder aux joies de la paternité. Seulement voilà, madame vient de prendre vingt kilos en neuf mois et il sait pertinemment qu’elle ne réussira jamais à s’en débarrasser. D’autant qu’elle ne fera pas le moindre effort pour ça, pour la bonne et simple raison qu’elle veut un autre enfant et part du principe que c’est pas la peine de suer sang et eau pour perdre vingt kilos si c’est pour en reprendre trente dans la foulée. Et elle se dit aussi que si son mari l’aime, il aura toujours autant envie d’elle même si elle ressemble à un éléphant de mer. Du coup, elle va garder ses vingt kilos et en reprendre une bonne vingtaine de plus pendant sa prochaine grossesse, perdant définitivement toute chance de retrouver un jour sa taille d’antan. Résultat des courses : ils vont commencer à s’engueuler, divorcer dans un an, deux ou trois si tout va bien, et madame va aller s’inscrire à la salle de sport du coin dans l’espoir de trouver un nouveau pigeon pour lui témoigner un peu d’affection. Vous ne trouvez pas que ça fait froid dans le dos ?

\textsc{Moi} : Vu comme ça, ce n’est effectivement pas très engageant.

\textsc{Lui} : Franchement répugnant, vous voulez dire !

\textsc{Moi} : J’aimerais qu’on en revienne à Rémi. Vous n’avez rien remarqué de bizarre pendant les jours ou les semaines qui ont précédé son décès ?

\textsc{Lui} : Non, rien du tout. Je vous le répète, Rémi allait parfaitement bien et n’avait aucune raison de mettre fin à ses jours. Il est évident que quelqu’un l’a tué en essayant de faire passer le crime pour un accident, et je ne doute pas que l’analyse médico-légale le confirmera.

\textsc{Moi} : Vous lui connaissez des ennemis ?

\textsc{Lui} : Les riches ont des tas d’ennemis.

\textsc{Moi} : Vous, peut-être ?

\textsc{Lui} : Moi ? Vous êtes fou !

\textsc{Moi} : Vous m’avez dit qu’il couchait avec votre femme. On a vu des gens en tuer d’autres pour moins que ça.

\textsc{Lui} : Je vous ai dit aussi que je m’en foutais, et que je couchais aussi avec la sienne. Il nous arrivait aussi de coucher tous ensemble, si vous voulez tout savoir.

\textsc{Moi} : Drôles de pratiques.

\textsc{Lui} : Je ne vous demande pas de vous joindre à nous.

\textsc{Moi} : Je peux vous poser une question ?

\textsc{Lui} : Si vous y tenez. Au fait, je ne sais pas si je vous l’ai dit, mais ce cigare est excellent. Il vient de Cuba, je suppose.

\textsc{Moi} : Non, de République dominicaine. Vous avez entendu parler d’Arturo Fuente ?

\textsc{Lui} : Non.

\textsc{Moi} : C’est un de ces émigrés espagnols qui ont fui Cuba pendant la guerre hispano-américaine. Il a atterri à Tampa, en Floride, et s’est lancé dans la fabrication de cigares avec des feuilles en provenance de Cuba. Une production d’abord confidentielle, limitée à quelques milliers de cigares par an roulés dans le salon et la cuisine par les membres de la famille. Ensuite, quand ça commençait à plutôt bien marcher, il y a eu le Che, Castro et la révolution cubaine, avec pour conséquence la rupture des relations diplomatiques entre Cuba et les USA. Du coup, la manne cubaine s’est tarie. Fuente a donc commencé à se fournir au Mexique et à Porto Rico, avec un succès mitigé, avant de changer son fusil d’épaule et aller s’installer au Nicaragua, nouvel Eldorado du cigare et dictature bananière sous contrôle américain. Mais à la fin des années 70, la révolution sandiniste a éclaté, les Somoza ont été foutus à la porte, et la fabrique Fuente, emblème d’une époque révolue, a été entièrement détruite par les flammes. Nouvel exil à Santiago, en République dominicaine, avec pour tout bagage un solide savoir-faire et une volonté farouche de tout casser. Aujourd’hui, associée à la famille Newman, Fuente produit des dizaines de millions de cigares par an, dont quelques uns parmi les plus réputés et onéreux de la planète. Celui que vous êtes en train de fumer, par exemple, le Short Story de la série Hemingway, est un vibrant hommage à l’écrivain qui a passé une bonne partie de sa vie à Cuba et appréciait tout particulièrement les Cohiba, une des plus prestigieuses marques de cigares.

\textsc{Lui} : Les meilleurs, à ce qu’il paraît.

\textsc{Moi} : Les plus chers, en tout cas. Oui, c’est ce que disent les snobs qui fument pour se donner un genre et n’y connaissent rien. C’était sans doute vrai avant l’embargo, et jusqu’à la fin des années 70 ou 80, mais depuis le cigare a fait son chemin un peu partout dans le monde et l’hégémonie cubaine n’est plus d’actualité, notamment en ce qui concerne le rapport qualité-prix. En cause le développement des marchés internationaux comme l’Inde, la Chine et le Moyen-Orient. Le pays n’arrive plus à suivre, avec pour conséquences une tendance à la surproduction, une baisse notoire de la qualité de fabrication et une hausse constante des prix. Les cigares autrefois abordables ont pulvérisé toutes les limites de la décence tarifaire. Le SIGLO VI de Cohiba, par exemple, grand classique s’il en est, se négocie aujourd’hui aux alentours de cent-dix euros pièce, et certaines séries spéciales montent à trois ou quatre cent. D’autre part, même s’il représente toujours dans l’imaginaire collectif la référence absolue en matière de cigare, force est de constater que le havane peine à se renouveler, innover tant sur la plan de la forme que du fond, tandis que les autres rivalisent de créativité pour exciter les papilles du consommateur.

\textsc{Lui} : Si vous le dites.

\textsc{Moi} : Je l’affirme haut et fort et ne cesserai de le clamer jusqu’à mon dernier souffle, n’en déplaise aux crétins prétentieux qui ne jurent que par le havane !

\textsc{Lui} : Grand bien vous fasse.

\textsc{Moi} : Maintenant, si vous le voulez bien, j’aimerais vous poser une petite question. Rien de personnel, rassurez-vous.

\textsc{Lui} : De quoi s’agit-il ?

\textsc{Moi} : De l’individu assis au volant de cette voiture.

Je parlais de Noé Desmarais, qui n’avait pas bougé un cil depuis le début de la conversation.

\textsc{Lui} : Vous voulez savoir s’il est mort, c’est ça ?

\textsc{Moi} : J’aimerais bien, oui.

\textsc{Lui} : C’est un ami à vous ?

\textsc{Moi} : Pas exactement, mais c’est une histoire un peu longue à raconter. Tout ce que je peux vous dire, c’est qu’il arrivait en face quand la Mini a essayé de doubler la Fiesta. Et ça a fait un grand BOUM !

Lui, agitant le truc qui pendait à son cou : Vous savez ce que c’est ?

\textsc{Moi} : Oui, un stéthoscope.

\textsc{Lui} : Mais pas n’importe lequel. C’est un Redmann, la Rolls du stéthoscope. Avec ça, vous pouvez entendre respirer un moucheron et battre le cœur d’un ver de terre, qui en possède cinq soit dit en passant.

\textsc{Moi} : Et alors ?

\textsc{Lui} : Alors cet homme est mort, il n’y a aucun doute là-dessus.

\textsc{Moi} : Vous en êtes sûr ?

\textsc{Lui} : Sûr et certain. Vous n’oseriez tout de même pas mettre en doute mes compétences ?

\textsc{Moi} : Loin de moi cette idée absurde.

\textsc{Lui} : Dans ce cas, vous ne m’en voudrez pas de prendre congé. J’ai encore des tas de vie à sauver qui m’attendent.

FIN

Provisoire, bien entendu.

Il est toujours extrêmement douloureux de mettre un point final à un récit qui a occupé de longs mois de votre existence, pompé une bonne partie de votre énergie et mobilisé toutes vos facultés créatrices. C’est comme dire au revoir à un vieil ami, lui serrer une dernière fois la main sans savoir si on le reverra un jour. On la garde longtemps dans le creux de la sienne, comme un petit animal blessé, on se refuse obstinément à la lâcher. Et puis on rentre chez soi, triste, et on avale une belle assiette de rognons de veau à la crème pour se donner une contenance, tenter d’oublier que toutes les choses ont une fin, les meilleures comme les pires, ce qui est une bonne chose pour les pires mais moins bonne pour les bonnes.

Sous la pression de mes fans, qui commencent à trouver le temps long (vous m’avez manqué, vous aussi), et surtout de mon éditeur qui a grand besoin de renflouer les caisses de sa modeste entreprise (les temps sont durs pour tout le monde, et ce serait pour lui un crève-cœur de devoir vendre son Ferretti Custom Line 97 ou hypothéquer sa villa de Saint-Raphaël pour sauver les meubles), je me vois dans l’obligation de remettre à plus tard un certain nombre des affaires en cours.

Je pense notamment à Jaya, la fille adorée de mon ami Zaahid Shirani, tombée entre les griffes d’un certain Simon Keskula, guide spirituel et maître incontesté d’un secte post-apocalyptique connue sous le nom d’Alliance de la Révélation. J’ai promis à Zaahid de tout mettre en œuvre pour que Jaya rentre au bercail et que le monstre qui la tenait sous sa dépendance soit définitivement mis hors d’état de nuire. Et comme je suis un homme de parole, je vais faire ce que j’ai dit. Et surtout, je ne manquerai pas de vous narrer par le détail comment, par quel stratagème machiavélique, ruse subtile et technique d’infiltration digne des meilleurs services de renseignement, et accessoirement usage immodéré de la force, sinon la violence la plus aveugle et éthiquement condamnable, je serai parvenu à mes fins.

En attendant, je sais qu’une double question vous ronge la cervelle aussi sûrement qu’un rat affamé s’attaque à un morceau de gruyère ou un vieux quignon de pain : Repentance Whittingham, alias la Gardienne de la Nuit ou la femme de ménage la plus rapide du monde, est-elle sortie du coma, et quid de Titus Beaugendre, porté disparu après une rencontre tant fortuite que suspecte avec la demoiselle en question ?

Eh bien… on n’en sait trop rien, à vrai dire, mais je ne manquerai pas de vous le faire savoir si j’apprends quelque chose à ce sujet. Tout ce que je sais pour l’instant, et c’est assez maigre je vous le concède, c’est qu’elle a disparu du jour au lendemain de sa chambre d’hôpital. Et comme je doute fort qu’elle ait été capable de le faire par ses propres moyens, l’action d’un tiers n’est pas à exclure.

Pour ce qui est de Titus, je vous propose un petit flashback juste avant le mot FIN, au moment où le docteur Charrier nous a annoncé que lui et son Redmann, la Rolls du stéthoscope, étaient formels sur le fait que Noé Desmarais ne ferait plus jamais joujou avec les allumettes, ni d’ailleurs avec quoi que ce soit d’autre, l’envie de faire joujou avec quelque objet ou organe que ce soit lui étant définitement passée. Desmarais refroidi, la joyeuse petite bande de néonazis des Disciples de la Colère était en partie démantelée. Ne restait plus qu’à exterminer les sieurs Monteil et Jégou, individus peu recommandables auxquels j’entendais bien réserver un traitement à la hauteur de leurs exploits. Pour ce faire, je m’étais dit que ça pourrait être sympa de transformer ma salle de bain en chambre à gaz. On enlevait Monteil et Jégou, on les obligeait à revêtir un pyjama rayé, on les affamait pendant quelques semaines, puis, quand ils commençaient à sentir si mauvais que même les mouches à merde s’enfuyaient à tire-d’aile à leur approche, on leur offrait une petite douche gratuite au Zyklon B, le célèbre insecticide à base de cyanure de la Deutsche Gesellschaft fur Schadlingsbekampfung (il faut reconnaître que les Allemands ont un certain talent pour créer des mots de quinze kilomètres de long totalement imprononçables pour toute personne non germanique, à tel point que je me demande si ce n’est pas une des raisons principales de leur manque de popularité et relatif isolement sur la scène internationale, outre le fait qu’ils construisent des voitures rapides très appréciées des trafiquants de drogue, boivent beaucoup de bière et sont nuls en cuisine). Après quoi on en faisait des steaks hachés, merguez, saucisses et andouillettes, et on organisait une grande fiesta dans le quartier avec barbecue à gogo jusqu’à épuisement des stocks. On joignait l’utile à l’agréable, et je doute fort que la police s’amuserait à aller fourrer son nez dans les cuvettes de chiottes du voisinage. Je ne connais pas les effets du piment et des épices sur les composés organiques, mais je suppose qu’il est tout à fait possible de rechercher des traces d’ADN dans une merguez ou une chipolata. Si on le faisait plus souvent, quelque chose me dit qu’on pourrait avoir des surprises de taille.

