\chapter{Acte 8}

\noindent Trois quarts d’heure plus tard, après avoir déposé Sam chez lui (humainement parlant, c’était à peu près l’équivalent d’une grenade dégoupillée en permanence, raison pour laquelle j’étais~-- et je n’étais pas le seul, Greg partageant très largement mon opinion sur le sujet, et Nathan aussi dans une moindre mesure sans doute liée à son quotient intellectuel réduit et sa façon plus terre-à-terre de voir les choses~-- assez pressé de m’en débarrasser), je me suis retrouvé au 157 rue des Anus en fleur (adresse fantaisiste je le rappelle, étant tenu à la plus extrême discrétion pour tout ce qui touche à ma vie privée), sur le trottoir devant mon domicile, en train de m’échiner sur le digicode de la porte d’entrée de l’immeuble. Je connaissais par cœur le numéro gagnant, bien entendu, mais il m’arrivait de plus en plus souvent de l’oublier, tout ou partie, au même titre d’ailleurs que mon numéro de carte de crédit et tous les numéros en règle générale, y compris les dates de naissance de mes proches et plus proches amis (heureusement en nombre limité). Ce triste état de fait m’avait conduit à les noter, lui et quelques autres, sur un petit morceau de papier que je trimballais discrètement dans le fond de ma poche, soigneusement plié dans le portefeuille flambant neuf que Zarina, désespérée de me voir utiliser envers et contre tout la vieille relique qui m’avait accompagné jusqu’à présent, avait jugé bon de m’offrir. Cette vieille relique avait traversé avec moi maintes épreuves, dont certaines parmi les plus douloureuses de mon existence, et je n’ai pas honte de dire, au risque de passer pour un demeuré, que j’y étais aussi viscéralement attaché qu’un enfant à son doudou. Zarina, bien décidée à couper le cordon quasi ombilical qui me reliait à mon vieux portefeuille, m’avait placé face à un ultimatum sans équivoque : c’était lui ou moi. Choc, séisme, avis de tempête, alerte cyclonique, vents à trois cents kilomètres/heure, pluie diluvienne de grêlons gros comme des œufs d’autruche, la terre s’effondrait sous mes pieds et je plongeais dans un abîme de tumulte et de magma incandescent !!! Face à un tel dilemme, je m’étais accordé un délai de réflexion de quelques semaines avant de donner ma réponse. Pendant ce laps de temps, tel Forest Whitaker dans Ghost Dog, Corey «Raekwon» Woods gravissant un par un les échelons menant à la salle des Cinq Dragons ou encore San Te affrontant le général Sien Ta dans la 36e Chambre de Shaolin, je suis parti au sommet de la montagne afin de méditer sur le sens de la vie, seul, la tête dans les nuages, vêtu d’un simple pagne en raphia dans la plus pure tradition hawaïenne, où j’ai vécu entouré de quelques rares animaux qui sortaient de leurs repaires à la nuit tombante pour m’assister dans ma quête de l’absolu. Elles avaient bien compris, les pauvres bestioles sans avenir, minables créatures entièrement soumises au joug impitoyable de la nature, à sa brutalité bestiale, que j’étais le phare autour duquel il convenait de se rassembler pour entrevoir une vague lueur d’espoir dans les ténèbres du néant. Pour moi la question était simple : étais-je oui ou non un homme de bien, d’amour et de bonté, gorgé de tolérance et boursouflé par l’empathie, bouffi de générosité, tumescent d’affection, abritant un cœur obèse dans sa poitrine gonflée d’orgueil, prêt à toutes les concessions, à renier l’essence même de son être, la substantifique moelle de sa nature profonde, pour faire de sa vie en couple une de ces réussites qui inspirent les générations futures, leur insufflent la conviction que tout n’est pas perdu alors même que les bombes pleuvent sur leurs têtes et que les flammes ravagent le monde. C’est ainsi qu’on a pu, à une époque pas si lointaine, me voir errer dans les forêts hostiles, parler avec les bêtes, danser au clair de lune avec les chouettes et les blaireaux, m’abreuver à la source des ravins, glaner des baies et creuser la terre avec mes ongles crasseux à la recherche de larves et vers de terre pour assouvir ma faim, dormir à même le sol sur un lit de feuilles mortes. Au terme de cet exil, je suis rentré chez moi, ai solennellement aspergé d’essence mon vieux larfeuille adoré, puis j’y ai mis le feu et l’ai regardé brûler sans regret, avec, au plus, une légère touche d’humidité dans le fond de l’œil. Je savais qu’une page du grand livre de ma vie, best-seller de mon histoire personnelle, chronique de la désespérance et recueil de poèmes aux strophes atrophiées, vers solitaires à la rime mal arrimée, rafiot littéraire en perdition, venait de se tourner, et que mon portefeuille-doudou élimé jusqu’à la corde faisait partie de ces objets sacrificiels dont la transsubstantiation cathartique précipite la renaissance de l’initié. J’ai ouvert grand les bras, laissé venir à moi les petits enfants~-- et les simples d’esprit~-- de l’Amour et dit à Zarina, qui était tombée à genoux et pleurait toutes les larmes de son corps en léchant goulûment mes sandales Valentino studshield fisherman en cuir de veau : tu vas tomber enceinte, et un fils tu enfanteras ; et maintenant ne bois ni vin ni liqueur forte (tu parles, Charles !), ni ne mange rien d’impur (ah bon, même pas des beignets de poisson-pénis ou des yeux de mouton à la kazakhe ?), car cet enfant sera consacré à Dieu dès le ventre de sa mère jusqu’au jour de sa mort. Rassurez-vous, je plaisantais, il n’était nullement dans mes intentions de me reproduire, et surtout pas à l’identique. À elle seule, ma présence constituait déjà une insulte au bon goût, un scandale naturel, une aberration évolutionnelle. Malgré l’esprit pervers qui régissait la plupart de mes actes, je ne voulais pas qu’une bande de petits moi-mêmes se répande comme une trainée de poudre à travers le monde, semant la mort et la destruction sur son passage. Bon, je ne dis pas que Zarina n’avait pas quelques petites arrières-pensées en tête, mais il n’était pas question pour moi, au moins pour l’instant, de prêter la moindre oreille à ses supplications. Vous m’imaginez, moi, Djeferson Beauvais, commandant de police dont les décorations prestigieuses s’entassent sur le revers de la veste, unanimement reconnu et salué par sa hiérarchie comme un leader de premier plan, un meneur d’hommes comme on n’en fait plus depuis Alexandre le Grand, Simon Bolivar, Churchill, Gengis Khan, Clemenceau, Hannibal et Charles Quint, vous m’imaginez en train de changer des couches et trimballer une poussette dans les rues de la ville, l’air idiot, un sourire béat sur mon visage horriblement déformé par les joies de la paternité ? De raconter des histoires idiotes à un gamin qui refuse obstinément de fermer l’œil à trois heures du mat ? Me déguiser en père Noël et rester coincé dans la cheminée avec ma hotte sur le dos ? Me ronger les ongles jusqu’au sang à la moindre poussée de fièvre ? Me taper les goûters d’anniversaires, les spectacles de fin d’année ? Me faire virer de chez moi comme un malpropre pour les booms d’ados pendant lesquelles ces petits cons vident le bar, vomissent et chient partout ? Me faire traiter de tous les noms parce que j’ai eu le malheur de faire une réflexion sur la tenue d’untel ou les horaires de sortie de tel autre ? Voir mes gosses raser les murs quand je les emmène à l’école parce qu’il ont les boules de se trimballer avec un vieux con dans mon genre ? Me saigner aux quatre veines pour eux et les voir me planter comme une merde à la première occasion ? Leur enseigner principes et valeurs pour les retrouver en train de dealer du shit à la sortie du collège ? Leur acheter un nouveau smartphone hors de prix tous les deux ou trois mois parce qu’ils se sont assis dessus ou l’ont laissé tomber dans la cuvette des chiottes ? Flipper ma race et tourner en rond comme un fauve en cage parce qu’il est une minuit deux et que ma fille devait rentrer à minuit ? Etre obligé de rencontrer les parents de son petit copain et faire comme si on était les meilleurs amis du monde alors que je sais pertinemment qu’ils ne seront plus ensemble dans trois jours (ma fille et son petit copain, ses parents je ne sais pas) ? Rencontrer les parents du petit copain de ma fille, découvrir que sa mère est une bombe atomique, une arme de destruction massive ultra radioactive, le genre qui porte des sous-vêtements à l’uranium enrichi et vous fait exploser les globes oculaires si vous avez le malheur de poser les yeux sur certaines parties charnues de son anatomie, que naturellement elle n’est pas insensible à mon charme (et comment le pourrait-elle, la pauvresse, loin de moi l’idée de lui jeter la pierre), et qu’il va par conséquent être assez difficile de ne pas sombrer dans le vice et la dépravation dignes des plus vils animaux (au risque de me faire démonter par le mari qui travaille dans le bâtiment, mesure deux mètres, pèse cent trente kilos et n’a jamais lu une ligne de Hume, Locke, Popper ou Aristote) ? M’apercevoir avec effroi que le petit copain de ma fille a quarante-cinq ans de plus qu’elle (et avec satisfaction qu’il est extrêmement riche, fils unique, célibataire et sans enfant) ? Mais aussi, plus généralement, se voir vieillir dans les yeux de ses gosses, les voir grandir, changer et vous pousser lentement dans la tombe pour prendre votre place, se reproduire à leur tour et comprendre enfin les souffrances que vous avez endurées en silence (plus ou moins) durant toutes ces années passées à les chérir et les choyer, lesquelles ont passé à la vitesse de l’éclair, comme une balle qui vous transperce le cœur, comme si le temps s’était accéléré rien que pour vous faire chier, vous empêcher de profiter pleinement de moments rares et précieux que vous ne retrouverez plus jamais, même pas quand vos gosses vous refileront les leurs pour aller au resto ou partir en week-end, sous prétexte qu’il faut aussi que vous profitiez d’eux alors qu’ils sont complètement pourris-gâtés, caractériels, insupportables, et que leurs parents sont trop contents de s’en débarrasser pour souffler un peu, alors que vous, toujours à la traine, à la ramasse, à côté de la plaque, vous les aviez sur le dos sept jours sur sept et vingt-quatre heures sur vingt-quatre, sans personne à qui les refiler pour éviter de sombrer corps et âme dans les affres de la folie, ou alors juste un papy ou un tonton pédophile qui avait la trique en les faisant sauter sur ses genoux et les tripotait dans votre dos (et je ne vous compte pas les dommages collatéraux du type tentatives de suicide et années de psychanalyse à cinquante balles de l’heure). Vous m’imaginez sérieusement vivre tout ça, ces années de cauchemar, uniquement pour faire plaisir à une femme que j’adore, certes, vénère par-dessus tout, pour l’instant, mais dont je serai peut-être en train de disperser les morceaux aux quatre vents dans quelques années ? Non, si je faisais ça, mes gosses ne me le pardonneraient jamais.

J’ai finalement réussi à taper le code, entrer, monter dans l’ascenseur et gagner le sixième sans encombre. J’entends par «sans encombre» que ledit ascenseur, d’une vétusté à toute épreuve, ne s’est pas subitement décroché pour s’enfoncer dans les profondeurs de la terre telle une bombe anti-bunker GBU-57 de l’US Air Force, fascinant concentré de technologie développé par Lockheed Martin, Boeing et la Northrop Grumman Corporation. Si vous êtes un psychopathe qui veut devenir le maître du monde et pense aller s’installer à six pieds sous terre pour se livrer en toute tranquillité à ses petites activités illicites, n’y pensez plus. Avec la GBU-57, les dictateurs en herbe, génies du mal mégalos et autres tarés du même genre ne sont plus en sécurité nulle part. N’imaginez surtout pas que cinquante mètres de béton armé vous protégeront de la justice divine exercée par le bras vengeur et doré à l’or fin de l’HPPM, l’Homme le Plus Puissant du Monde. Elon Musk, qui a choisi d’aller s’exiler dans l’espace pour préparer son grand retour sur Terre, l’a bien compris, ce qui fait de lui l’apprenti-dictateur 2.0 neuroatypique le plus hype de l’univers, même si ses affinités politiques discutables et sa propension psychotique à repeupler le monde avec ses propres rejetons (une quinzaine à ce jour, qui portent tous des noms à coucher dehors) ne plaident pas nécessairement en sa faveur. À propos de la GBU-57, je rappelle au passage que douze de ces sympathiques engins ont été largués sur Fordo dans la nuit du 21 au 22 juin 2025, l’objectif revendiqué par un Donald Trump ivre de puissance et à haute valeur psychiatrique ajoutée étant d’empêcher Ali Khamenei, le guide suprême iranien (dont les dernières mises à jour en matière de droits de l’homme et de la femme en particulier remontent à la civilisation de Jiroft et aux premiers Aryens, je ne parle bien évidemment pas de ces cons de Nazis mais des Aryens de la période védique de l’âge du bronze), de fabriquer des bombes atomiques pour les balancer sur la gueule de Benyamin Netanyahou, autre guide suprême… enfin, premier ministre israélien d’extrême-droite visé par un mandat d’arrêt de la CPI pour crimes de guerre. Au même titre, bien sûr, que les terroristes du Hamas et une bonne partie des dirigeants de cette planète, l’autre partie, tenue par des impératifs commerciaux et une certaine image de marque à préserver, ne pouvant malheureusement pas s’en donner autant à cœur-joie qu’elle le souhaiterait, dans l’attente d’une guerre mondiale qui permettrait enfin à tout le monde de faire joujou avec ses têtes nucléaires et se tripoter la nouille atomique jusqu’à l’explosion finale, l’orgasme actinique fukushimiquement pur, l’ultime éjaculation qui rayerait définitivement toute forme de vie de la surface de la Terre, hormis peut-être quelques vagues espèces primitives à l’avenir incertain.

Je ne pensais pas (non, vraiment pas, tant j’étais certain d’avoir connu mon lot de souffrance pour la journée, une journée qu’il me tardait de voir s’achever comme rarement il m’avait tardé de voir une journée s’achever, alors que vous savez comme moi qu’il existe d’autres journées dont on aimerait qu’elles durent éternellement, ces journées qui font dire que le temps passe trop vite et réfléchir amèrement sur la vacuité de l’existence et la finitude des choses), en arrivant au sixième, qu’il me faudrait encore affronter de terribles épreuves avant de pouvoir enfin me glisser dans la chaleur tiède et réparatrice de mes draps.

J’ai vu une ombre glisser furtivement dans la pénombre du couloir, tel un requin fantôme dans les eaux les plus noires, profondes et mystérieuses de l’océan Pacifique, des eaux dans lesquelles glissent encore, et sans doute depuis la nuit des temps, des créatures dont l’existence défie l’imagination et qu’on croirait tout droit sorties du cerveau détraqué d’un Lovecraft, un Sheridan Le Fanu ou un Montague Rhodes James pour n’en citer que quelques uns.

Cette ombre venait de l’autre bout du couloir, un endroit où peu de gens osaient s’aventurer tant circulaient à son sujet d’étranges légendes (l’immeuble ne datait pas d’hier) et rumeurs de phénomènes surnaturels ayant entraîné un certain nombre de décès inexpliqués, et qui était précisément celui dans lequel Marc-Antoine Jacquinot, le prof de philo neurasthénique au physique de planche à pain maladive et myope, avait choisi de s’établir. Jacquinot lui-même, avec son éternelle serviette en cuir patiné par des décennies de bons et loyaux services, prêtait le flanc avec indifférence aux racontars les plus insidieux tant de la part de ses collègues que de la population estudiantine du lycée dans lequel il exerçait. Profondément philosophe dans l’âme, Jacquinot savait que son séjour sur terre serait de courte durée et n’avait pas de temps à perdre à se prendre le chou avec ses concitoyens. N’étant moi-même pas dépourvu de ce que j’appellerai une certaine forme de pessimisme raisonnable, j’avais tenté à de nombreuses reprises, sans réel (aucun, si vous préférez) succès, de nouer contact avec lui, ne serait-ce qu’au titre des bonnes relations de voisinage. Il avait su, en quelques mots d’une parfaite courtoisie, me faire comprendre que, malgré les apparences, il n’était en aucune façon à la recherche d’un ami, une âme-sœur ou quoi que ce soit d’autre qui s’apparente de près ou de loin à un être humain. Même s’il arpentait d’un pas traînant les couloirs sombres et glacés d’une existence avec laquelle il ne se sentait aucune affinité particulière, il entendait bien continuer à profiter jalousement d’une solitude chèrement acquise à la force du poignet, poignet à l’extrémité duquel se trouvait une main qui, aux dires des mauvaises langues, était le seul et unique partenaire avec lequel il ait jamais entretenu une quelconque relation à caractère sexuel.

J’ai essayé d’allumer mais la lumière ne marchait pas.

La lumière ne marche jamais quand une scène dramatique se profile à l’horizon.

C’est en principe à ce moment-là qu’une musique sinistre fait son entrée en scène pour mettre le spectateur sous pression. Cette musique allie le plus souvent sons d’une gravité extrême, plus bas que bas, comme sortis d’outre-tombe, et couinements suraigus qui mettent les nerfs en pelote. Pour parfaire le tout, des effets percussifs soudains et largement amplifiés s’abattent sur vous à l’improviste.

Comme on a coutume de dire dans ces cas-là : je n’y voyais pas plus loin que le bout de mon nez, assez court du reste.

Sans une aide extérieure, j’aurais été rigoureusement incapable de trouver le trou de la serrure de la porte d’entrée de mon appartement, ou alors il m’aurait fallu tâtonner des heures durant pour y parvenir. C’est sans doute ce qui se serait passé si j’avais vécu au dix-sept ou dix-huitième siècle, mais j’avais la chance de vivre au vingt-et-unième, à la glorieuse époque de la téléphonie mobile, et d’avoir dans ma poche ce qu’on appelle un smartphone, littéralement un téléphone intelligent, fin et racé, connu aussi sous les noms de mobile multifonction et terminal de poche, véritable couteau suisse des temps modernes, capable de satisfaire toutes vos exigences en un temps record. Prendre des photos de qualité professionnelle alors que vous êtes complètement nul et disposez au mieux de la sensibilité artistique d’une moule à marée basse ? C’est possible. Envoyer des messages chiffrés alors que vos connaissances en mathématiques se limitent à la table de multiplication par 2, et encore par temps clair quand Neptune est en conjoncture favorable avec Vénus ? C’est possible. Avoir des tas d’amis qu’on n’a jamais vu, qui affichent des photos de profil bidons, postent des gros plans de leur bite en érection et débitent des insanités par treize à la douzaine ? C’est possible, bien sur, et même très à la mode, grâce à nos réseaux sociaux dernier cri, offrant à prix cassé toutes les avancées en matière de navigation en eaux troubles et hypertextualisation des relations quotidiennes. Grâce à nous, non seulement vous ne serez plus jamais seul, mais vous ferez partie d’une cybersociété mondiale toujours prête à répondre à vos attentes et vous niquer dans les grandes largeurs. On attend avec impatience le modèle «plancha» permettant de se faire griller des saucisses sans fumée dans les salles d’attente et les transports en commun.

L’ombre en question, celle qui glissait furtivement dans le couloir en provenance de l’appartement de Marc-Antoine Jacquinot, «l’ermite du sixième» comme on l’appelait, était celle d’une créature que je ne connaissais que trop bien : Korax, le chat de la mère Ouvrard, animal sournois et malfaisant qui ne perdait jamais une occasion de nuire à son prochain. Il avait fait de l’immeuble son territoire et entendait bien y faire régner une seule et unique loi : la sienne.

Il est passé devant moi, m’a jeté un regard mauvais, puis s’est dirigé nonchalamment, trop nonchalamment, en faisant exprès de prendre tout son temps pour bien me faire chier, me narguer, vers l’autre bout du couloir, celui où se trouvait l’appartement de la mère Ouvrard, l’ignoble vieille bonne femme qui sentait le rance et empestait la méchanceté à plein nez, la Voisin, la Giulia Tofana, l’Hélène Jégado, la Marie Besnard, la Chisako Kakehi du sixième qui empoisonnait ses victimes au porto frelaté, une infâme piquette qu’elle achetait à vil prix à la supérette du coin. S’il existait un semblant de justice dans ce pays, il y a belle lurette qu’elle aurait dû être condamnée à la réclusion criminelle à perpétuité, à l’isolement pour éviter de contaminer le reste de la colonie. Soit maudite jusqu’à l’os, mère Ouvrard, et puissent les flammes de l’enfer te calciner les miches à petit feu jusqu’à la fin des temps ! Oui, je sais ce que tu vas me dire : je suis seule au monde, jamais personne ne vient me voir, je n’ai ni famille, descendance ni ami, de moula non plus et y a longtemps que j’ai arrêté de bédave, ma vie est si laide et inhumaine que les roses fanent à ma vue et les rivières se tarissent. Sur la vie de ma mère, même les serpents, les rats et les cafards me fuient ! Je suis la plus immonde des larves qui aient jamais été pondues par le cloaque putréfié d’une femme de mauvaise vie, un tel concentré d’abjection que la pourriture elle-même fait figure de nectar à côté de moi. Ouais, je sais tout ça, et pourtant j’estime que j’ai droit moi aussi à minimum sinon d’amour, je demande pas la lune, au moins de considération et de respect.

Bon, okay, j’exagère un peu. D’accord, Maria Ouvrard était une de ces femmes comme on n’aimerait pas en croiser une au fond d’un placard, un bois ou un cimetière, notamment la nuit, et on se demande encore à quoi pouvaient bien ressembler ses parents pour engendrer une horreur pareille. Quelle mouche les avait piqués ? Néanmoins, qu’on le veuille ou non, elle avait bien dû être jeune, elle aussi, à un moment ou à un autre. Ce n’est pas une garantie en soi, mais enfin tout de même, on est souvent surpris de voir à quel point une adorable fillette peut se transformer, en l’espace de quelques décennies, en chose plus ou moins monstrueuse, improbable croisement entre l’éléphant de mer et le poisson chauve-souris à lèvres rouges (sale gueule, celui-là), puis en vieille sorcière hideuse et maléfique. Ça vaut aussi pour les hommes, bien entendu, je ne voudrais pas qu’on me traite de phallocrate. C’est juste, si on veut, que la marge de manœuvre est plus étroite pour l’homme en raison du niveau de mocheté qu’il parvient à entretenir tout au long de son existence. Le moment critique, à savoir la puberté, impacte durablement son apparence. Il est alors tellement laid, sans queue ni tête (surtout sans tête, d’ailleurs, parce que la queue tourne à plein régime), que le temps ne peut que lui être bénéfique. Si, vous serez d’accord avec moi que pas mal de types sont moins laids vieux que jeunes, même s’ils restent de toute façon tellement immondes qu’on se demande comment une femme peut avoir envie d’y toucher. Le désespoir, sans doute, ou la déformation professionnelle. Bref, quels que soient les griefs qu’on pouvait avoir à son égard, la mère Ouvrard restait une créature de Dieu. Ou du diable, peu importe, elle restait une créature tout court, et, à ce titre honorifique, ne méritait pas qu’on l’écrase comme une vulgaire punaise malodorante ou la cloue comme une vieille chouette (moche) à la porte de sa chambre. D’autant qu’ils n’en avaient plus pour bien longtemps, elle son suppôt à quatre pattes, à empuantir l’atmosphère du sixième étage. Un jour on la retrouverait morte dans son appartement, de même que Korax qui serait mort empoisonné en essayant de la becqueter. Fin de l’histoire, au diable les varices !

Par chance, à cette heure avancée de la nuit, elle dormait à poings fermés, car rien n’aurait été plus désagréable, pour ne pas dire traumatisant, que de voir surgir sa vieille tête parcheminée dans l’embrasure de sa porte grinçante.

J’aurais dû aller me coucher, mais au lieu de ça j’ai trouvé bizarre de voir Korax sortir de là et me suis dirigé toute affaire cessante vers le fond du couloir, scrutant les alentours à la lumière de mon téléphone.

Une fois devant la porte de Jacquinot, je me suis rendu compte qu’elle était entrebâillée.

Aussitôt, une giclée de sueur froide m’a parcouru la nuque et glissé le long du dos jusqu’à la raie du cul.

Sans être tout à fait désagréable, la sensation trahissait quand même une certaine anxiété de ma part, savamment entretenue par les ténèbres environnantes et le silence pesant qui planait sur les lieux (oui, on peut être pesant et planer, la chose n’est point contraire aux règles de la physique).

Il n’était bien évidemment pas normal que la porte de Jacquinot soit entrebâillée à trois heures du matin, alors que, de jour comme de nuit, elle ne l’était jamais, sauf bien entendu quand il entrait ou sortait de chez lui, moments particuliers où il entrebâillait alors sa porte juste assez pour pouvoir se glisser dans l’ouverture. De cette façon, si un observateur se trouvait, par hasard ou non, dans les parages, il lui était impossible de distinguer ce qui se trouvait à l’intérieur. À force de s’adonner à ce petit exercice, Jacquinot était devenu presque aussi plat que la sacoche en cuir vintage qu’il trimballait en permanence au bout de son bras. À propos de ses bras (petite anecdote en passant pour détendre l’atmosphère, assez irrespirable il faut bien le dire), j’avais remarqué qu’il les avait nettement plus longs que la moyenne, ceux d’un homme normal lui arrivant peu ou prou à mi-cuisse, à l’endroit où se trouvent les poches de son pantalon s’il en porte un, alors que les siens lui arrivaient pratiquement aux genoux, d’un côté comme de l’autre. Tant et si bien que cette fameuse sacoche en cuir qu’il trimballait en permanence ne se trouvait jamais très loin de racler le sol, raison pour laquelle il avait choisi, ou s’était fait fabriquer sur mesure (ou avait hérité de son père qui avait le même genre de bras que lui), un modèle d’une confection assez inhabituelle, nettement plus long que haut.

J’ai poussé la porte et demandé d’une voix mal assurée, conscient que le fait qu’une porte soit entrouverte ne donnait pas pour autant le droit à quelqu’un de s’introduire chez quelqu’un d’autre sans y être invité : Il y a quelqu’un ?

Apparemment non.

J’ai néanmoins insisté : Monsieur Jacquinot, vous êtes là ?

Finalement, après avoir posé plusieurs fois la même question et avoir obtenu à chaque fois la même absence de réponse, je me suis décidé à entrer pour de bon.

Mon premier travail a été de mettre la main sur l’interrupteur le plus proche, tâche plutôt aisée qu’il ne m’a pas fallu plus de quelques secondes pour mener à bien.

Jacquinot aurait pu se trouver dans l’entrée, étalé de tout son long, face contre terre, baignant dans une mare de sang s’échappant de son crâne fracturé de part en part, ou encore des multiples plaies par arme blanche présentes la quasi totalité de son corps sans vie. Pire encore, seule sa tête aurait pu se trouver là, les yeux hagards et la langue pendante, les restes de son corps affreusement mutilé gisant un peu partout dans l’appartement, un bras par ci, une jambe par là, la bite dans le congélo et les couilles dans le vide-ordures, ces deux derniers points constituant la signature tristement célèbre du tueur en série connu sous le nom de «tueur du sixième», monstre insaisissable dont la principale caractéristique est de commettre ses forfaits uniquement au sixième étage des immeubles dans lesquels il s’introduit. Une fois, un jour où il devait être particulièrement en forme et remonté à bloc contre la société (espérons que les experts nous en apprendront un peu plus sur lui quand on aura réussi à le coincer), il ne s’est pas contenté d’un seul appartement mais a décimé la totalité de l’étage, enfants et animaux compris (le visage des enfants peint en jaune citron, comme s’il baignait dans un océan de lumière estivale, et les animaux coupés en deux dans le sens de la longueur).

Si un tel carnage s’était produit, vous pensez bien que Korax aurait été le premier sur les lieux pour se goinfrer des restes de la victime.

Mais tel n’était pas le cas.

D’interrupteur en interrupteur, j’ai poursuivi mon inspection et constaté que l’endroit était bien tel que je me l’étais imaginé : rempli de poussière et de toiles d’araignées, de la vaisselle sale s’entassant dans l’évier de la cuisine, des restes alimentaires trainant ici et là, abandonnés au détour d’un placard ou une commode et tartinés d’une généreuse couche de moisi, des bougies spectrales dégoulinantes de cire fondue, des meubles branlants dont on osait à peine effleurer portes et tiroirs de peur de les voir s’effondrer sous ses doigts. Même chose pour les chaises sur lesquelles seul un fou furieux suicidaire se serait risqué à poser ses fesses, les fauteuils défoncés et le canapé qui crachait ses tripes et servait vraisemblablement de repaire à toute une faune à laquelle il valait mieux éviter de songer si on ne voulait pas finir à l’asile le plus proche. Ce mobilier préhistorique, que Jacquinot avait dû hériter de quelques lointains ancêtres vivant à l’âge de pierre, il ne lui était manifestement jamais venu à l’idée de s’en débarrasser, comme tout être sain d’esprit l’aurait fait. J’imagine qu’il ne voulait pas froisser la susceptibilité des défunts, lesquels l’observaient attentivement depuis l’enfer où leurs méfaits les avaient conduits, et n’auraient pas manqué, en le voyant bazarder toute cette merde, d’exprimer leur désaccord en déplaçant des objets, faisant s’ouvrir et se refermer les portes de façon intempestive, entendre des bruits bizarres et souffler un vent glacial dans les couloirs de l’appartement. Ensuite, ils auraient pris possession de Jacquinot qui se serait mis à pisser partout et débiter des obscénités indignes de sa condition. Il aurait alors fallu faire venir un vieux prêtre cardiaque et un jeune prêtre en pleine crise de foi pour tenter de mettre un terme à l’infestation. Naturellement, les choses se seraient mal passées, le vieux prêtre serait mort en pleine action, le crucifix à la main, et le jeune aurait vendu son âme au diable avant de se jeter dans la cage d’escalier, dévaler les six étages sur le dos et se retrouver en bas avec la nuque brisée, tout ça pour une vague histoire de mobilier périmé.

Je savais Jacquinot grand amateur de littérature, mais ce que je découvrais au fur et à mesure de ma visite (je devrais dire mon exploration, car je me sentais de plus en plus dans la peau de Raymond Maufrais se frayant un chemin à coups de machette dans la jungle du Mato Grosso) dépassait de loin mes espérances les plus folles : tous les murs étaient recouverts de livres du sol au plafond, fascinant spectacle digne du plus majestueux des édifices religieux ! Car quand je dis tous les murs, c’est bien tous les murs, y compris, chose assez inhabituelle vous en conviendrez, ceux des toilettes et de la salle de bain. Je veux bien qu’on trouve quelques bouquins dans les toilettes, les gens ayant tendance à se faire chier quand ils chient et profiter de l’occasion pour parfaire leur culture, mais de là à ce que les murs soient couverts de livres du sol au plafond, il y a un pas que je n’avais encore jamais vu franchi. D’ailleurs, la plupart du temps, il ne s’agit pas, dans les toilettes des gens, de livres mais de revues, magazines, journaux plus distrayants, avec des images et moins de mots, moins à même de contrarier l’extrême concentration nécessaire à l’expulsion de matières qui ont parfois tendance à se faire désirer, faire preuve d’une certaine mauvaise volonté pour quitter le nid douillet dans lequel elles reposaient jusqu’ici.

Pour autant que l’on sache, l’être humain a toujours entretenu des relations particulières avec son caca. Il faut dire que là où l’animal peut se permettre de déféquer en toute tranquillité sans se soucier de rien, se torcher par exemple, parce que tout le monde, à commencer par lui, se fiche comme d’une guigne qu’il se promène le nez au vent avec de la merde au cul, suivi en permanence par un essaim de mouches irrésistiblement attirées par l’odeur infecte qu’il dégage, l’être humain, lui, se doit de sacrifier à des règles d’hygiène très strictes. Dès son plus jeune âge, alors qu’il n’est encore qu’un nourrisson totalement dépendant du bon vouloir de ses parents, ces derniers lui enseignent avec autorité qu’il est hors de question de chier dans son froc comme si de rien n’était. Comme d’une part il est incapable de se rendre aux toilettes par ses propres moyens, et d’autre part ne voit aucune raison valable de le faire, ils l’obligent à porter des couches et se relaient à son chevet pour les changer aussi souvent que nécessaire, chose qu’ils font de plus ou moins bonne grâce suivant leurs appétences en la matière et leurs disponibilités respectives. Avant on lavait les couches, ce qui représentait une charge de travail supplémentaire pas toujours très agréable, maintenant on les jette, le progrès se caractérisant par une accumulation de plus en plus massive de déchets dont on sait de moins en moins quoi faire, ce qui oblige au mieux à se creuser la tête pour trouver un moyen de les éliminer, au pire (qui est toujours certain) à les enterrer discrètement dans des endroits déserts en croisant les doigts pour qu’ils ne refassent jamais surface et ne contaminent pas le sous-sol avec leurs émanations malsaines. C’est ainsi que l’enfant, dès sa naissance, prend l’habitude de se chier dessus et baigner de longues heures dans la tiédeur rassurante et légèrement poisseuse de ses excréments, chose qui l’amène tout naturellement à les considérer avec une certaine bienveillance, d’autant que l’odeur forte et familière qui s’en dégage ne l’incommode nullement, contrairement à ses parents et son entourage qui témoignent d’une certaine gêne dès qu’ils la détectent. Bébé, au contraire, est ravi de se retrouver les quatre fers en l’air dès que sa couche est pleine, de gesticuler les pattes écartées en attendant que papa-maman le délestent de son panier bien garni pour le remplacer par un autre qu’il se fera une joie de remplir à la première occasion. Il exhibe fièrement ses organes génitaux et son petit trou de balle tout crotté, sachant que ses parents leur témoignent le plus vif intérêt. Ce caca qu’ils aiment tant, c’est avec la plus vive satisfaction qu’il le leur offre, en quantité illimitée et sous une forme diversement liquide et parfumée. Et puis, un beau jour, le ton change, on lui fait comprendre qu’il ne pourra pas continuer toute sa vie à chier dans des couches parce que ses parents en ont ras le cul (façon de parler) de trimballer des sacs de merde à longueur de journée. Quelle déception, pour ne pas dire trahison. Lui qui pensait pouvoir encore, à trente-cinq ou quarante ans, exhiber fièrement ses organes génitaux et son cul plein de merde sous le nez de ses parents émerveillés, tombe de haut. On a beau lui expliquer que ça reviendra peut-être quand il sera vieux, tout pourri et sur le point de casser sa pipe, mais que ce ne sera plus papa-maman qui feront le job parce qu’ils seront morts et enterrés depuis longtemps (et puis que de toute façon même s’ils étaient encore en vie ils n’auraient aucune envie de faire), il n’en tire aucune consolation. Non seulement on lui dit qu’il est grand temps d’arrêter de se chier dessus, mais on insiste lourdement sur le fait que ce qu’il tenait pour quelque chose de précieux, le meilleur de lui-même issu des fondements les plus intimes de sa personne, un authentique trésor qu’il offrait de bon cœur à ses parents pour les remercier de lui avoir donné la vie, n’est en réalité qu’une chose parmi les plus infectes et répugnantes qui soient, une pure abomination que ses parents ont pris courageusement sur eux pout tenter d’éliminer jusque dans ses moindres recoins, traquer dans ses plus infimes retranchements. Lui qui pensait que ses parents avaient soigneusement conservé les tonnes d’excréments qu’il leur offrait avec joie depuis des lustres, apprend avec stupéfaction qu’ils n’avaient de cesse de s’en débarrasser avec dégoût sitôt la récolte effectuée. Quelle désillusion, quel temps perdu, et surtout quelle duplicité de la part de gens qu’il pensait proches de lui et soucieux de préserver jalousement les témoignages d’affection qu’il leur dispensait en toute innocence. Maintenant qu’il est en âge de marcher, on lui explique que le caca c’est caca et qu’il n’est pas question de continuer à emmerder le monde avec ça. C’est tellement caca qu’il devra, pour éviter tout contact avec ses vêtements, porter des slips pendant le restant de ses jours, le slip étant une sorte de couche-culotte extrafine qui n’a pas vocation à recueillir le gros des troupes mais seulement limiter les dégâts en cas de fuite (à noter que le délit de fuite urinaire ou fécale ne donne pas lieu à des poursuites judiciaires). S’il semble avéré que les rois de France se soulageaient en public, devant une foule compacte qui se pressait pour assister au spectacle nauséabond et pétaradant de son seigneur et maître, de telles pratiques sont aujourd’hui révolues et sévèrement réprimées par la Loi, sauf peut-être dans le cadre de certaines réunions privées à la moralité douteuse. De nos jours, il existe, pour se soulager, des endroits prévus à cet effet, doté d’un système de verrouillage intérieur permettant à l’utilisateur de ne pas se faire surprendre en pleine action, la bite à la main ou la crotte au cul. De tels endroits portent des noms aussi divers et évocateurs (et souvent au pluriel) que toilettes, cabinets, latrines, sanitaires, commodités, petit coin et lieu d’aisance, plus quelques noms argotiques tels que goguenots, chiottes ou tartisses, ce dernier étant aujourd’hui sorti de l’usage courant. On parle aussi, à propos d’un tel endroit dépourvu de siège, de chiottes à la turque, ce qui semblerait indiquer que nos amis turcs ne sont pas très à cheval sur le confort et ne craignent pas de se chier (du latin cacare, «évacuer des excréments») sur les bottes. Chez nous, gens civilisés, on prend le plus grand soin de son postérieur et met toutes les chances de son côté pour joindre l’utile à l’agréable (d’où la fréquente présence de livres, revues, et, pourquoi pas si on dispose de l’espace nécessaire, un mini-bar et un téléviseur dernier cri, même s’il ne faut pas non plus que ce soit trop confortable sans quoi les gens risqueraient de plus jamais remettre le nez dehors, préférant rester bien au chaud dans leur merde plutôt que de retourner affronter les frimas de la vie quotidienne). Et surtout, quand on a fini, on ne se contente pas de se rhabiller et retourner vaquer à ses occupations comme si de rien n’était, un sourire de contentement sur le visage, sans faire le moindre cas des gens qui se pincent le nez et grimacent autour de soi. Non, avant de sortir, on s’essuie soigneusement le fion avec un papier lui aussi spécialement prévu à cet effet, le plus souvent de couleur rose ou blanche, orné ou non de motifs assez sommaires, plus ou moins rugueux, présent dans les toilettes sous la forme de rouleaux de feuilles pré-coupées aisément détachables. Reste, vous l’aurez compris, que s’enfermer à double tour dans un placard pour satisfaire un besoin aussi naturel que légitime pose un problème de fond (et même de fondement) sur lequel il ne serait peut-être inutile de s’interroger de temps à autre, histoire de ne pas entretenir outre mesure ce qui s’apparente clairement à un déni systémique aux conséquences potentiellement dévastatrices. Car enfin, je vous le demande à vous, gens de bonne volonté à l’esprit clair et lucide, quelle honte y a-t-il à chier, et pourquoi le fait de satisfaire un besoin aussi élémentaire se retrouve-t-il soudain frappé d’opprobre, au même titre d’ailleurs que l’orifice qui sert à mener l’action ? Chez beaucoup, cette interrogation ne trouve pas réponse, ce qui les conduit, hagards et dépités, à s’enfermer dans une sorte de névrose obsessionnelle (ou névrose de contrainte, en l’occurrence le fait d’être obligé de se cacher pour chier, et, par extension, s’attacher à faire disparaître toute mauvaise odeur de sa personne, voire disparaître tout court, tirer la chasse sur son existence vaine et inutile) et développer des troubles comportementaux plus ou moins graves. La chose est d’autant plus sensible que sa configuration elle-même prête à confusion. La nature, pour laquelle j’ai beaucoup de respect par ailleurs, n’a-t-elle pas fait preuve d’un peu trop de légèreté en plaçant l’anus dans le voisinage immédiat des organes génitaux ? Quelle mouche (à merde) l’a donc piquée pour qu’elle aille, elle d’ordinaire si subtile et mesurée, se fourvoyer dans un tel salmigondis anatomique ? Le cas de la vulve est encore plus prégnant, dans la mesure où bite et couilles sont des organes bénéficiant d’une certaine forme d’externalité. Mais la vulve, même si son architecture habile et fonctionnelle fait d’elle un organe tout à fait digne d’intérêt, n’en reste pas moins, qu’on le veuille ou non, un orifice, certes moins étroit que le fion, et aussi plus accessible car contrairement à lui elle ne dispose pas de la protection joufflue des fesses, mais un orifice tout de même, une excavation, un creux, un trou, une dépression, voire un gouffre, un précipice, une tombe, appelez ça comme vous voudrez. Au même titre que la bouche, me direz-vous, les trous de nez aussi bien que ceux des oreilles, car si on le veut on peut effectivement voir des orifices partout (et certains ne se privent pas d’en faire usage dans le feu de l’action, qu’il s’agisse des aisselles, d’un simple entrecuisse ou tout autre dispositif vaguement concave utilisé en guise de substitut), mais vous remarquerez comme moi que cette fois, la nature n’a pas commis l’impair de placer la bouche à côté du fion. Ce serait, à mon sens, tout aussi pratique pour manger, mais peut-être pas pour faire d’autres choses comme jouer de la flûte ou chanter des cantiques à tue-tête, l’expression «à tue-tête» n’ayant du même coup plus aucune raison d’être. Il faudrait alors la remplacer par «à tue-fion» ou «à tue-fesses». On notera aussi, dans ce contexte de sexualité débridée, que bouche et orifice anal entretiennent des relations dont l’étroitesse ne le cède en rien à l’incongruité, Mais, pour en revenir à nos moutons, vous ne m’enlèverez pas de l’idée que cette coupable proximité entre deux organes que tout oppose, je veux bien sûr parler de la vulve et le fion, est l’archétype même de ce qui s’appelle une erreur de la nature, d’autant plus manifeste que l’une et l’autre jouissent d’un statut très particulier au sein de la sphère anatomique. Tous deux, en effet, ont une importance capitale sur le plan vital, non dépourvue de symbolique qui plus est. L’un sert en quelque sorte à se vider, évacuer les scories de l’appareil digestif, tandis que l’autre sert à faire le plein de vie, régénérer l’espèce. Pourtant, qu’il s’agisse de l’étron ou du nouveau-né, tous deux sont éjectés de la même manière, à ceci près que les affres exprimées pendant l’accouchement sont sans commune mesure avec celles correspondant à l’expulsion délicate de quelque boudin récalcitrant, à un point tel que seule une farouche volonté de donner la vie peut permettre d’y souscrire en toute sérénité.

Mais comment, pour le dire vulgairement (chose dont j’ai horreur mais qui autorise parfois des raccourcis saisissants), par quelle aberration cosmique un trou qui produit de la merde au kilomètre a-t-il pu se retrouver à côté d’un trou qui produit de la vie, autrement dit la chose la plus belle et noble qui soit ? Reconnaissez que de là à penser que la vie c’est de la merde, il n’y a qu’un pas. Je veux bien que la nature ait le sens du contraste, mais tout de même ! On ne s’étonne plus, dès lors, que certains individus, déjà largement traumatisés par le fait d’être obligés de s’enfermer dans un placard pour chier, présentent des troubles du comportement qui confinent à la démence pure et simple, et Dieu sait pourtant que j’ai les idées larges en la matière. On trouve par exemple, sous le nom de code 302.9 de la section des paraphilies non spécifiées du DSM (l’encyclopédie des troubles mentaux soigneusement tenue à jour par l’Association Américaine de Psychiatrie, tant il est vrai que les pauvres ont fort à faire de ce côté-là, j’en veux pour preuve le cinglé à peau rouge et cheveux jaunes qu’ils ont trouvé le moyen de placer à la tête de l’État et qui n’arrête pas de taxer tout le monde, jouer au golf, envoyer des bombes sur l’Iran, se prendre la tête avec des milliardaires néonazis sud-africains et construire des centres de détention pour migrants dans des trous paumés infestés d’alligators), la coprophilie, pratique qui consiste à placer ou faire entrer des excréments dans sa bouche, les mastiquer plus ou moins longuement avant de les ingérer ou les recracher suivant la gravité des cas. Certains, parmi les scatophiles les plus modérés, se contentent de jouer avec, les manipuler avec délectation, pouvant aller jusqu’à s’en barbouiller le visage ou s’en enduire le corps. Le Marquis de Sade, dans ses Cent Vingt Journées de Sodome ou l’École du Libertinage, s’est longuement (avec délectation) penché sur la question. Cet ouvrage, rédigé sur trente-trois feuillets de papier vergé crème et bleu collés bout à bout jusqu’à constituer un rouleau de près de douze mètres de long, alors que Sade se trouvait en résidence à la Bastille pour faits de débauche extrême à la limite de la criminalité la plus débridée, est aujourd’hui considéré comme le catalogue de perversions sexuelles le plus abject et complet jamais réalisé par la main de l’Homme. Véritable compilation des pires abominations en la matière, ce rouleau de PQ satanique a longtemps été dissimulé aux yeux du monde, dans le but de protéger l’espèce humaine, essayer de sauver son âme dévoyée des flammes de l’enfer, empêcher coûte que coûte que de telles horreurs tombent entre des mains innocentes. Sans doute rendu fou par un sentiment de toute-puissance lié à son rang, à moins que la consanguinité n’ait quelque chose à voir là-dedans, Sade s’est rendu coupable des pires atrocités, se livrant sans retenue à des passions telles que même son génie littéraire ne parvenait plus à les transcender. Son style, proche de l’écriture automatique, déversant un flot continu d’immondices à la face du lecteur tétanisé (ébahi, et parfois même, il faut malheureusement bien le dire, excité et inspiré), trahissait une démence que les mots n’arrivaient plus à contenir. Et parmi ces immondices se trouvaient bien évidemment, de façon récurrente jusqu’à l’écœurement, outre le sperme, le sang et même le pus, les matières fécales pondues par les anus les moins ragoûtants de la planète. Selon Freud et moi-même, car je le rejoins volontiers sur ce point (avec peut-être, quand même, en ce qui me concerne, un léger surcroît de finesse dans la vision holistique des choses), l’apprentissage de la propreté anale n’est pas sans conséquence sur la vie de l’enfant. L’enfant en bas âge, dépourvu de dents, doit se contenter de boire du lait et avaler des aliments liquides. Et quand il chie, non seulement ça ne sent pas le muguet, mais il en fout partout dans sa couche parce que ses selles manquent de consistance. Autrement dit, quand il chie, et selon qu’il est une fille ou un garçon, il s’en fout plein la chatte, la bite et les couilles. Normalement, quand on grandit, on s’essuie le trou de balle et on passe à autre chose. Dans le cas présent, quand l’heureux élu, le père ou la mère en l’occurrence, la nourrice le cas échéant ou toute autre personne autorisée, ouvre le paquet, il ou elle est contraint de procéder à des travaux de nettoyage de grande ampleur, d’autant plus techniques que les surfaces concernées offrent de nombreux coins et recoins pas toujours faciles à atteindre. En d’autres termes, la merde s’accumule dans les plis et il faut frotter sec pour en venir à bout. On imagine facilement que ce genre de traitement occasionne des sentiments confus chez l’enfant, qui associe tout naturellement le fait de chier, abondamment de préférence (le nettoyage dure plus longtemps), au fait de se faire tripoter les organes génitaux, chose qui lui procure bien évidemment un certain plaisir. Pourquoi ? Eh bien mais tout simplement parce la nature, certainement l’entité la plus manipulatrice qui soit, a mis au point ce stratagème diabolique qui consiste à faire en sorte que la reproduction soit synonyme de plaisir extrême, de jouissance absolue. Quoi qu’il en soit, même si ces notions de propreté et reproduction sont encore bien abstraites pour lui, l’enfant en bas âge comprend très vite que plus il chie copieusement et plus il sera récompensé par papa-maman qui passeront du temps à lui récurer les parties génitales. Avant, c’était le plus souvent maman qui s’y collait, de sorte que la plupart des filles étaient lesbiennes et des garçons rêvaient de coucher avec leur mère. Maintenant, les papas s’y mettent aussi, de sorte que la plupart des filles sont hétéros et des garçons homos. Si les deux participent à relative égalité, ce qui est préconisé par la nouvelle charte de la vie en couple, on peut espérer avoir affaire à des enfants raisonnablement bisexuels (à voile et à vapeur comme on disait avant, quand on avait le sens de la métaphore maritime et ferroviaire), ce qui est encore le moyen le plus sûr de prendre son pied dans toutes les circonstances. cela semble d’une logique imparable, frappé violemment au coin du bon sens le plus inaltérable, et pourtant vous devez garder solidement en mémoire que je ne dispose d’aucune information permettant de le confirmer ou l’infirmer officiellement. Il ne s’agit, j’en ai bien peur, que de spéculations dont la gratuité ne saurait en aucun cas légitimer l’abus.

Cela dit, à l’heure de la désinformation ouvrant la voie à toutes les formes de paranoïa complotiste, je ne vois aucune raison de s’en priver.

Mais ce n’est pas de ça que je voulais vous parler initialement.

Non, ce sur quoi je voulais m’arrêter, c’est sur le fait que je me trimballais une terrible envie de chier depuis des lustres, et que celle-ci, qui était plus ou moins passée au second plan depuis une dizaine de minutes, s’était brutalement réveillée à la vue des toilettes de Jacquinot, endroit, comme je vous l’ai dit (ou pas, je ne sais plus), lui aussi recouvert de livres du sol au plafond.

Parmi ceux-ci, outre de nombreux ouvrages consacrés à la philosophie, la psychiatrie et la psychanalyse, parmi lesquels les œuvres complètes de Freud, Fliess (je recommande à tous la lecture des «relations entre le nez et les organes génitaux féminins présentés selon leur signification biologique», indispensable) et Breuer (ses remarquables travaux sur l’hystérie, notamment le cas de la militante féministe et polyglotte Bertha Pappenheim, alias Anna O.), se trouvait aussi tout ce que la littérature érotique avait produit de plus excitant depuis la plus haute Antiquité.

Comme je n’avais rien remarqué de suspect pendant mon expédition, je me suis dit que rien ne m’empêchait, puisque j’étais sur place, de couler un bronze dans les toilettes de Jacquinot. J’aurais aussi bien pu le faire chez moi, je vous l’accorde, mais l’envie était devenue tout à coup si impérieuse que je courais le risque de perdre une partie du chargement en cours de route. Ç’eut été dommage, vous en conviendrez. J’ajoute qu’il y avait là un tas d’ouvrages que je ne connaissais pas, des auteurs dont je n’avais même~-- en dépit de ma culture encyclopédique~-- pour ainsi dire jamais entendu parler, une foultitude de livres rares à n’en pas douter, et, cerise sur le gâteau, un certain nombre d’opuscules poussiéreux et autres grimoires antiques qui ressemblaient furieusement à des éditions originales d’une valeur plus ou moins inestimable. Qu’est-ce qu’ils foutaient dans les chiottes, mystère et boule de gomme, mais toujours est-il que mon niveau d’excitation intellectuelle avait atteint des sommets que seul ma dramatique envie de chier pouvait lui disputer.

C’est donc mu par une avidité certaine que je me suis introduit dans la place, ai dégrafé mon pantalon, fait tomber mon slip (le modèle Brando en jersey de coton bi-stretch à imprimé léopard de chez Dolce \& Gabbana, pas du meilleur goût je vous l’accorde, mais c’était Zarina qui me l’avait offert et il coûtait une couille, j’étais donc obligé de le porter de temps en temps pour amortir l’investissement), me suis assis sur la cuvette des chiottes et ai aussitôt balancé un flot de chiasse digne des chutes du Niagara.

Dès lors, il ne me restait plus qu’à tendre le bras pour attraper le volume de mon choix.

J’ai hésité un instant entre le Roman de Violette de la marquise Henriette de Mannoury d’Ectot (alias la Vicomtesse de Cœur Brûlant) et Le Diable au corps du chevalier Andréa de Nerciat (aka «docteur Cazzoné»), personnage obscur dont j’avais maintes fois croisé le nom mais n’avais jamais lu la moindre ligne. Le moment était venu de corriger cet oubli, ce qui ne m’empêcherait bien évidemment pas, si la séance s’éternisait, de jeter un œil sur le roman de cette chère Violette, créature à laquelle il n’était pas difficile d’imaginer qu’il arrivait tout un tas d’aventures parmi les plus affriolantes.

Un quart d’heure plus tard, j’étais toujours dans le Diable au corps (à ne pas confondre avec l’opéra-bouffe d’Ernest Blum et Raoul Toché ou le chef-d’œuvre de Radiguet), dans les toilettes de Jacquinot. J’avais fini de décharger mon abominable cargaison depuis un certain temps, je me sentais nettement mieux et ne voyais aucune raison valable d’interrompre ma lecture. J’avais, je dois le reconnaître, la fâcheuse habitude, que j’imagine partagée par beaucoup de gens dans mon genre, même si je reste un exemplaire assez unique de ce que l’humanité peut produire de meilleur, ceci dit en toute modestie bien sûr, j’avais, disais-je, la fâcheuse habitude de m’éterniser sur le trône au point d’avoir tellement de fourmis dans les jambes qu’il me devenait pratiquement impossible de marcher. Les bestioles me dévoraient les pattes et je risquais de me foutre par terre à chaque fois que je posais le pied au sol. Toute personne m’ayant vu sortir des toilettes dans cet état n’aurait pas manqué d’éclater de rire d’une part, tant le spectacle est ridicule, et d’autre part, au vu du contexte, s’interroger sérieusement sur l’état de ma santé mentale. Dieu merci, tout ne tardait pas à rentrer dans l’ordre.

«Dans cette attitude, il a le superbe cul sur les yeux et sa bouche est croisée de cette entaille magique où la Nature a fixé le siège des voluptés. En même-temps, l’intéressant et fier boute-joie se dresse contre les yeux de Nicole, déjà provoquée par une langue qui n’est pas la gauche et peu complaisante langue de l’automate Hilarion.»

Voilà très exactement ce que j’étais en train de lire, rédigé dans le style inimitable du XVIIIe, quand un bruit suspect a attiré mon attention, faisant immédiatement retomber la tension palpable qui était en train de s’installer discrètement au niveau de la partie la plus bas-ventrale de mon humble personne.

Je suis comme tout le monde, je n’aime pas tellement que des bruits suspects attirent mon attention, surtout quand je suis en train de chier dans les chiottes de quelqu’un d’autre, que ce quelqu’un d’autre, que je connais à peine (bonjour, au revoir, joyeux Noël et bonne année), ne m’a pas invité à lui rendre visite, lire ses livres rares, et encore moins utiliser ses chiottes sans autorisation préalable. J’aurais aussi bien pu passer un coup de fil, en m’excusant de l’heure tardive, pour l’avertir que j’étais sur le point de débarquer pour utiliser ses toilettes, sous le prétexte fallacieux que les miennes étaient bouchées, remplies de merde jusqu’aux ouïes avec une chasse d’eau hors service, ou encore occupées par quelqu’un d’autre, ma compagne, par exemple, créature de rêve qui préparait idéalement le bollito misto à la viande de poulain et la couenne de porc, ou encore encore, pourquoi pas, une entité démoniaque et grimaçante avec des yeux injectés de sang et la bouche remplie d’asticots, un extraterrestre (genre pas commode, Alien ou Predator, et je vous prie de croire que quand vous tombez nez-à-nez avec un de ces deux-là en train de chier dans vos toilettes à trois heures du matin, vous vous excusez du dérangement, refermez doucement la porte et vous éloignez aussi vite et aussi loin que vos jambes flageolantes peuvent vous porter), ou tout simplement un inconnu qui les avait prises en otage et exigeait une rançon exorbitante pour les libérer.

Ô crotte ! suspends ta chute, comme aurait dit Lamartine dans Le lac, un lac de merde en l’occurrence, un océan de matière fécale dont les effluves fleuraient bon le poisson pourri et la crevette en rupture de chaîne du froid, afin que je puisse tendre l’oreille et mettre un nom sur le bruit que j’entends !

Le bruit se rapprochait, et il ne fallait pas avoir fait cinq millions d’années d’études ou être titulaire d’un diplôme de MOF (Meilleure Oreille de France) pour se rendre compte que ce bruit était celui que des pieds humains chaussés de chaussures à semelles de cuir produisent habituellement lorsqu’ils se déplacent sur des lattes de plancher.

Dans quelques instants, la personne à qui appartenaient ces pieds, personne dont il y avait tout lieu de penser qu’elle n’était autre que Marc-Antoine Jacquinot, prof de philo notoirement dépressif qui foutait la trouille à ses élèves~-- et pas seulement eux, il me la foutait à moi aussi et à la plupart des gens qui avaient le malheur de croiser sa route au détour d’un cimetière, un terrain vague, une allée de bibliothèque ou une voie sans issue~-- et ne se déplaçait jamais sans une sacoche en cuir d’apparence antédiluvienne, cette personne allait débarquer et me découvrir en fâcheuse posture dans ses chiottes, à trois heures du matin, la crotte au cul et le Diable au corps entre les mains.

J’imaginais sans peine son visage (au nez aquilin, au regard terne chaussé d’horribles lunettes vintage à monture dorée qui lui donnaient des airs de médecin-chef de camp de la mort adepte de la vivisection et la stérilisation de masse, au menton tapissé de barbe semblable à de la moquette élimée et au crâne aussi dégarni qu’un buffet de mariage après la fête) déformé par la stupeur et la consternation de me découvrir moi, son voisin de palier avec lequel il n’entretenait que des relations de la plus stricte convenance, assis sur la cuvette de ses chiottes à trois heures du matin. Le choc pouvait, s’il était cardiaque, l’emporter directement dans la tombe. L’avantage, vu qu’il n’avait apparemment ni ami ni relation d’aucune sorte, c’était que personne ne le regretterait. Il n’y aurait, comme pour Mozart, pour suivre son cercueil qu’une poignée de chiens galeux et de clodos endimanchés, plus, éventuellement, quelques adolescents boutonneux, chevelus et binoclards, fripés à la va-comme-je-te-pousse avec des vestes trop grandes et des pantalons trop courts, qui voyaient en lui un maître absolu du nihilisme relativiste au sens le plus démocritien, kafkaïen, polysémique et destouchien du terme.

Dans la situation pour le moins délicate qui était la mienne, et compte tenu du temps extrêmement court dont je disposais pour mettre sur pied une stratégie plus créative, j’en suis arrivé à la conclusion que le mieux était encore de m’enfermer dans les toilettes, ce que j’ai fait sans hésiter, courageusement.

Les chaussures à semelle de cuir sont arrivées et se sont arrêtées à ma hauteur.

De mon poste d’observation, je pouvais entendre la respiration de la personne à qui appartenaient les pieds qui se trouvaient dans ces chaussures, personne qui se tenait juste derrière la porte, ou devant si on se plaçait de son point de vue, et qui, à en juger par le temps qu’elle prenait pour faire entendre le son de sa voix, exprimer ses interrogations et faire valoir ses droits, devait se poser pas mal de questions sur la conduite à adopter.

J’ai vu la poignée bouger et j’ai prié pour que la porte, construite dans un matériau de piètre qualité et à peu près aussi épaisse qu’une feuille de papier à cigarette, résiste à la pression.

La poignée a bougé à nouveau, avec une vigueur accrue trahissant un certain énervement de la part de mon visiteur. Mais peut-être faisait-il tout simplement lui aussi l’objet d’une envie pressante, et pouvait-on alors comprendre, en admettant qu’il s’agisse bien du propriétaire des lieux, que le fait de trouver la porte de ses chiottes fermées de l’intérieur lui occasionne une certaine forme de contrariété, laquelle pouvait aussi, si la situation ne se réglait pas au plus vite, se transformer en fureur noire elle-même susceptible d’entraîner un déchaînement de violence sans précédent. Même si, de la part d’une personne aussi flasque, absente et dépourvue de relief que Jacquinot, toute idée de déchaînement de violence sans précédent semblait relever davantage du thriller horrifique au scénario inexistant (voir exemple ci-dessous) que d’une vision claire et détaillée de la triste réalité des choses.

Exemple ci-dessous, qui s’apparente à une note en bas de page mais ne se trouve pas en bas de page (une petite coquetterie que je m’autorise de temps à autre, quand la Lune est en conjonction avec Saturne, période qui favorise la concentration et la confiance en soi, de préférence à l’heure où la Sittelle de Krüper rejoint à petits coups d’ailes pressés son nid dans les noires forêts de pins de l’Azerbaïdjan), et ce pour la bonne et simple raison que je sais pertinemment que personne ou presque ne lit les notes en bas de page, ce qui signifie que si on veut qu’elles soient lues il ne faut pas les mettre en bas de page mais en plein milieu, ne pas chercher à les ostraciser, les reléguer au second plan, mais au contraire les inclure de plein pied dans un récit auquel non seulement elles appartiennent, mais dont elles viennent également préciser onctueusement le sens et renforcer en douceur la pertinence : le professeur Roman Bozhko, un virologue d’origine ukrainienne d’une petite cinquantaine d’années (il a une jambe plus courte que l’autre, des hémorroïdes, et surtout un syndrome de Moersch et Woltman qui l’oblige à se bourrer de benzodiazépines), qui bosse pour un labo P4 top secret du genre Unité 731 (quand les Japs violaient des femmes à la chaîne et se livraient à des expériences horribles sur des êtres humains pour développer des armes bactériologiques) ou Zagorsk-6 (même chose pour les Russes, mais avec des babouins~-- attention, je n’ai pas dit que les Russes violaient des babouins, même s’il faut s’attendre à tout avec eux), découvre avec effroi que sa femme le trompe avec un prof de philo dépressif et totalement dépourvu de charisme. Bozhko pète un câble et met au point, à l’insu de ses employeurs dont on ne sait pas très bien qui ils sont ni d’où ils viennent, un virus qui s’attaque exclusivement aux profs de philo dépressifs et les transforme en tueurs psychotiques soumis à des accès de violence incontrôlable, accès qui les poussent, notamment, à s’en prendre à des femmes enceintes pour leur ouvrir le ventre et dévorer ce qui se trouve à l’intérieur.

Autant dire que les chances de voir Jacquinot se transformer en tueur psychotique et cannibale étaient aussi minces que celles de le voir se transformer en loup-garou, vampire, mouche géante, créature du marais ou entité maléfique venue d’ailleurs, même si, dans le noir et vu de dos, il pouvait aisément flanquer la trouille à n’importe qui. D’autant, je le rappelle, qu’il avait en permanence avec lui cette sacoche bizarre dont personne ne connaissait le contenu et qui ressemblait étrangement à la mallette de chirurgien de sir William Gull, proche de la reine Victoria un temps suspecté d’être Jack l’Éventreur (voir From Hell des frères Hughes, inspiré des écrits de Stephen Knight, obscur complot mêlant franc-maçonnerie et famille royale d’Angleterre sur fond de misère sociale et prostitution).

La poignée a cessé de bouger et une petite voix s’est fait entendre derrière la porte : Il y a quelqu’un ?

Ce à quoi j’ai répondu au débotté : C’est occupé !

\textsc{La voix} : Ah, pardon !

\textsc{Moi} : Je vous en prie.

Quelques instants de silence s’en sont suivis, aussi lourds et chargés d’électricité qu’un ciel d’orage, à l’issue desquels la voix s’est fait entendre de nouveau : Excusez-moi de vous déranger, mais… je peux savoir qui vous êtes ?

\textsc{Moi} : Et vous-même ?

\textsc{La voix} : Marc-Antoine Jacquinot, j’habite ici.

\textsc{Moi} : Ah, très bien. Je suis Djeferson Beauvais, votre voisin de palier. On s’est déjà croisé dans l’ascenseur.

\textsc{La voix} : Le commissaire de police ?

\textsc{Moi} : C’est ça.

\textsc{La voix} : Enchanté. Mais dites-moi, je peux savoir ce que vous faites dans mes toilettes à cette heure-ci ?

\textsc{Moi} : Une envie pressante, je suis vraiment désolé.

\textsc{La voix} : Vous n’avez pas de toilettes chez vous ?

\textsc{Moi} : Si, bien sûr, mais elles sont malencontreusement bouchées de chez bouchées. Vous savez ce que c’est, à force de chier dedans, encore et encore, la merde s’accumule et les ennuis commencent.

\textsc{La voix} : Non, je ne sais pas. Je mange très peu et vais très rarement à la selle, pour tout vous dire.

\textsc{Moi} : Je vois. En ce qui me concerne, j’adore chier et ne m’en prive guère, un véritable tube digestif sur pattes. Il y a des gens qui se mordent la queue, c’est bien connu, à défaut d’être assez souple pour se la sucer. Mais moi, si je pouvais avoir la bouche directement reliée au trou de balle, comme dans The Human Centipede, je serais le plus heureux des hommes. Je ne vous choque pas, j’espère ?

\textsc{La voix} : Du tout.

\textsc{Moi} : Vous avez vu le film ?

\textsc{La voix} : Du tout non plus, mais le concept est intéressant philosophiquement parlant.

\textsc{Moi} : Vous avez une très jolie collection de livres rares et anciens, en tout cas.

La voix, traversée de bizarres inflexions aquatiques comme s’il y avait une fuite d’eau à l’intérieur : Oui, j’en suis assez fier.

\textsc{Moi} : J’ajoute que vos toilettes sont très spacieuses et agréables pour s’adonner à la lecture.

\textsc{La voix} : Oui, il m’arrive souvent d’y aller uniquement pour ça. Je peux vous poser une question un peu indiscrète ?

\textsc{Moi} : je vous en prie, faites comme chez vous.

\textsc{La voix} : Justement. Je peux savoir comment vous avez fait pour entrer chez moi ?

\textsc{Moi} : Question légitime à laquelle je vais me faire un devoir de répondre dans les meilleurs délais.

\textsc{La voix} : C’est très aimable à vous. J’ose espérer que rien de grave ne vous amène ici.

\textsc{Moi} : Rassurez-vous. Si je me suis permis d’entrer, c’est uniquement parce que la porte était ouverte. Tenaillé par une violente envie de chier, je ne me voyais pas aller sonner à la porte de cette chère madame Ouvrard, Maria de son prénom, pour me soulager. D’abord il est tard, ou tôt, ensuite son chat me déteste et se ferait une joie de me crever les yeux, et enfin la seule idée de visiter les toilettes de cette charmante vieille dame me soulève le cœur aussi sûrement qu’une bourriche d’huîtres avariées. Je crois que je préférerais encore me retenir jusqu’à la fin des temps, quitte à exploser et engloutir la Terre sous un déluge de merde !

\textsc{La voix} : Comme je vous comprends. Vous dites que ma porte était ouverte, c’est bien ça ?

\textsc{Moi} : En effet. La chose a attiré mon attention, du reste. J’ai beau être un être humain avec une bouche, un tube digestif et un trou de balle, je n’en reste pas moins, avant toute considération d’ordre scientifique ou rationnel, un flic jusqu’au plus profond de mon âme, un limier que son flair infaillible mène par le bout du nez sur les chemins les plus étroits et les pistes les plus escarpées de la vérité.

\textsc{La voix} : Ce que vous venez de dire est assez joli.

\textsc{Moi} : Je suis poète à mes heures. Non, ce que je voulais dire, c’est que j’ai vu votre porte ouverte, que j’ai trouvé ça bizarre et me suis dit qu’il vous était peut-être arrivé quelque chose. Une fois sur place, j’ai été saisi par une violente envie de chier. Vos toilettes me tendaient les bras, ou la cuvette, et j’ai pensé que vous ne verriez pas d’inconvénient à ce que je les utilise, vu que les miennes bouchées. J’ai eu tort ?

\textsc{La voix} : Non, bien sûr. Quel genre d’homme serais-je si j’interdisais l’accès à mes toilettes à un des mes semblables au cœur de la tourmente.

\textsc{Moi} : C’est bien ce que je me suis dis, et je suis rassuré de voir qu’il ne vous est rien arrivé. Vous étiez sorti ?

\textsc{La voix} : Oui, comme tous les premiers vendredis soir du mois. J’anime un café philo au Monocle, rue Marcel Duclos. Laurent Jaubert, le patron, est un garçon très sympathique et ouvert à la conversation. C’est aussi le sosie exact du grand acteur français Paul Meurisse, jusqu’au timbre de la voix, particulièrement savoureux. Vous avez vu la série des Monocle ?

Moi, cédant à l’envie de mentir sans raison, par pur vice, alors que j’avais vu et revu la série des Monocle un nombre incalculable de fois : Non.

\textsc{La voix} : C’est bien dommage.

\textsc{Moi} : Oui.

\textsc{La voix} : Donc vous ne connaissez pas Paul Meurisse.

\textsc{Moi} : Si, je l’ai vu dans L’Armée des ombres, de Melville.

\textsc{La voix} : Ah, très bien.

\textsc{Moi} : Et aussi dans Le Cri du cormoran le soir au-dessus des jonques, de Michel Audiard, avec Michel Serrault, Jean Carmet et Bernard Blier.

\textsc{La voix} : Excellent. Ce n’est pas son meilleur film, mais Meurisse joue le rôle d’un chef de gang assez savoureux, il faut bien le dire, inspiré de Jean-Pierre Melville, précisément, avec ses lunettes noires et son chapeau. C’est aussi la première apparition de Gérard Depardieu dans un long-métrage.

\textsc{Moi} : Si vous le dites. Ecoutez, tout cela est très intéressant, mais je n’aime pas trop bavarder à travers une porte. Laissez-moi cinq petites minutes, le temps de me refaire une beauté, et je suis à vous.

\textsc{La voix} : Oui, bien sûr.

Je me suis torché et rhabillé en vitesse. Bien inspiré par une longue expérience et une connaissance profonde de la nature fécale des choses, j’avais pris soin d’ouvrir la fenêtre en grand avant de procéder à la vidange complète des huit mètres d’intestins qui garnissaient mon abdomen, en plus du foie, de la rate et l’estomac. L’odeur, même si elle était de nature à terrasser des personnes fragiles ou des animaux de petite taille tels que rongeurs, insectes, symphyles et autres pauropodes, restait supportable pour un individu normalement constitué, sain d’esprit (condition indispensable pour affronter une telle épreuve, et je dois reconnaître que j’avais quelques doutes à ce sujet concernant mon interlocuteur) et ne souffrant d’aucune pathologie respiratoire chronique.

J’ai ouvert la porte et me suis retrouvé nez-à-nez avec Jacquinot, sa mallette pourrie de tueur en série londonien, sa petite bouche cruelle agitée de rictus incessants, comme s’il cherchait en permanence à détacher des trucs coincés entre les dents, son nez d’oiseau de proie, et, last but not least, son regard vitreux scintillant mollement derrière ses lunettes vintage à monture dorée. N’ayant pas eu l’opportunité de me laver les mains, j’ai fait fi des usages en m’abstenant de lui tendre l’une d’entre elles pour qu’il la serre chaleureusement, chose qu’il n’aurait de toute façon vraisemblablement pas faite.

Ses yeux se sont rapidement promenés sur moi, comme des mouches bleues sur un étron fraîchement pondu, puis sont allés se fixer sur un point précis à l’intérieur des chiottes.

Il a alors pointé son doigt en direction du point en question, poussé un cri étouffé, et finalement réussi à vaguement articuler, d’une voix si déchirante qu’elle aurait pu tirer des larmes au plus retors, chauve, ondiniste et nécrophile des huissiers de justice : NOM DE DIEU DE BORDEL DE MERDE !!!!!!!!!

Moi, tournant la tête vers l’endroit en question, couvert de livres comme le reste de la pièce, sans remarquer quoi que ce soit de particulier (je précise au passage que je ne l’aurais jamais cru capable d’une telle vulgarité) : Quoi ?

\textsc{Lui} : VOUS NE VOYEZ PAS ?

\textsc{Moi} : Non, quoi donc ?

\textsc{Lui} : IL Y A UN TROU !!!!!!!

\textsc{Moi} : Ah bon, où ça ?

Lui, se dirigeant vivement vers l’emplacement désigné, moi à ses trousses : Mais là, voyons, entre La Métaphysique du strip-tease de Denys Chevalier et le Satiricon de Pétrone !!!!!!

\textsc{Moi} : Eh bien ?

\textsc{Lui} : Il y a un trou, vous ne voyez pas !

\textsc{Moi} : Vous voulez parler de ce léger espace entre les deux ?

\textsc{Lui} : Pas si léger que ça ! C’est l’endroit où se trouvait Glicère, ou la Philosophie de l’Amour de Nicolas Cammaille-de-Saint-Aubin, une édition originale de 1796, tirée à une centaine d’exemplaires dont la plupart ont aujourd’hui disparu.

\textsc{Moi} : Ça vaut cher ?

\textsc{Lui} : Dans les deux mille euros.

\textsc{Moi} : Ah oui, en effet, c’est pas donné. Vous êtes bien certain qu’il était là ?

\textsc{Lui} : Vous me prenez pour qui, un amateur ? Bien évidemment, que j’en suis certain ! Je connais par cœur l’emplacement de chacun de mes livres, et il y en a plus d’un millier dans cette pièce. ET LÀ, TENEZ !!!!!!!!!!!!!!

\textsc{Moi} : Quoi encore ? Ne me dites pas que….

\textsc{Lui} : Si, regardez, là, entre les Dialogues des courtisanes de Lucien de Samosate et Margot la ravaudeuse de Fougeret de Monbron, vous ne remarquez rien ?

\textsc{Moi} : Il y a un léger espace ?

\textsc{Lui} : Oui, comme vous dites ! C’est là que se trouvait une très rare édition originale de la Lettre à la Présidente de Théophile Gautier, une tenancière de bordel connue sous le nom d’Apollonie Sabatier qui a beaucoup inspiré Baudelaire et dont on dit qu’elle aurait servi de modèle à L’Origine du monde, le fameux tableau de Courbet. Il n’en existe qu’une cinquantaine d’exemplaires dans le monde, édités sous le manteau dans les années 1850, dont la valeur se situe aujourd’hui entre cinq et six mille euros. C’est une catastrophe !

\textsc{Moi} : Je comprends mieux.

\textsc{Lui} : Quoi ?

\textsc{Moi} : Pourquoi la porte était ouverte. Il est évident qu’un ou plusieurs individus ont profité de votre absence pour faire main basse sur vos livres rares. J’espère que vous êtes bien assuré.

\textsc{Lui} : Vous savez aussi bien que moi comment fonctionnent les assurances. Vous payez plein pot pendant des lustres, et quand c’est à eux de le faire ils font tout pour minimiser les frais. Je vais toucher un prix de gros pour l’ensemble, un lot de consolation tout au plus. Car voyez-vous, monsieur…

\textsc{Moi} : Beauvais. Djeferson Beauvais, avec un D comme désir ou doryphore, et un seul F. Mon père était un fervent admirateur de Thomas…

\textsc{Lui} : Car voyez-vous, monsieur Beauvais, il y a la valeur marchande des choses, qu’on peut évaluer avec une certaine précision en fonction de données matérielles et statistiques, et celle qu’on leur accorde pour des raisons plus personnelles, existentielles, métaphysiques et autres.

\textsc{Moi} : Sans doute, oui. Ce qu’on appelle le préjudice moral.

\textsc{Lui} : En termes de justice, oui. Mais il n’y a pas que la justice, dans la vie. Il y a des choses qui ne sont pas de son ressort, même si elle entend se mêler de tout et régler tous les problèmes à coups de peines de prison et compensations financières. Je vous en foutrai, moi, des dommages et intérêts ! Rien ne remplacera jamais les trésors qui m’ont été volés !

\textsc{Moi} : Il faudra penser à faire une liste complète des objets dérobés.

\textsc{Lui} : Ne vous en faites pas pour ça, je connais tout par cœur.

\textsc{Moi} : Il y a des livres partout, et je pense que les voleurs ont fait le tour de l’appartement.

\textsc{Lui} : Sans aucun doute. Par exemple, je suis prêt à parier que mon édition originale des Passions de l’âme de Descartes, datée de 1649, soit un an avant la mort de l’auteur à Stockholm, estimée aujourd’hui entre quinze et vingt mille euros, a disparu.

\textsc{Moi} : C’est probable, oui.

\textsc{Lui} : Certain

\textsc{Moi} : En principe, quand on a des objets de valeur chez soi, on fait installer un système d’alarme.

\textsc{Lui} : Est-ce que j’ai une tête à faire installer des systèmes d’alarme ?

\textsc{Moi} : Non, pas vraiment. Mais les assurances risquent de se faire tirer l’oreille pour rembourser.

\textsc{Lui} : L’argent n’a aucune valeur pour moi. J’ai hérité la plupart de ces livres de mon père, mon grand-père et mon arrière-grand-père, qui étaient tous des collectionneurs avertis, c’est comme si je les avais perdus une seconde fois.

\textsc{Moi} : Il y a quand même chose qui me turlupine.

\textsc{Lui} : Quoi ?

\textsc{Moi} : Vous n’avez aucun ami, n’est-ce pas ?

\textsc{Lui} : Je ne vois pas à quoi ils me serviraient. Mes seuls amis sont mes livres. C’est à eux que je me confie, et c’est eux qui m’apportent le réconfort dont j’ai besoin.

\textsc{Moi} : Pas de famille non plus, je suppose ?

\textsc{Lui} : Quelques vagues oncles et tantes que je vois une fois tous les dix ans ou vingt ans, principalement aux enterrements, le leur ou celui des autres.

\textsc{Moi} : Vous allez aux enterrements ?

\textsc{Lui} : Ça m’est arrivé une fois ou deux. Contrairement à ce qu’on pourrait penser, je n’ai aucune appétence particulière pour la mort et la religion. Les gens vivent, meurent, apparaissent et disparaissent, pas de quoi en faire tout un plat.

\textsc{Moi} : Ni femme ni enfant, bien évidemment.

\textsc{Lui} : Surtout pas ! Ils vous pompent toute votre énergie et vous vous retrouvez au fond du trou avant d’avoir eu le temps de comprendre ce qui vous arrivait ! Non, la solitude est un mal nécessaire si on veut approfondir un peu les raisons de sa présence sur terre, comprendre les tenants et aboutissants ce cette chose absurde qu’on appelle la vie. Je dois reconnaître que ça fait des années que je bosse comme un chien et que je ne suis toujours pas plus avancé, ou à peine. Pour ce qui est des enfants, j’en ai bien assez comme ça dans mes salles de classe.

\textsc{Moi} : Ça vous dérange si je fume ?

\textsc{Lui} : Oui.

\textsc{Moi} : C’est sans importance. De toute façon, je n’avais aucune envie de fumer. Non, voyez-vous, ce qui me turlupine le plus, dans cette histoire, et c’est le flic qui parle et non plus le simple voisin, c’est comment, alors que vous avez une vie sociale proche du zéro absolu, ni femme ni amis qui pourraient fourrer leur nez dans vos affaires, des malfrats ont pu savoir que vous gardiez chez vous, sans aucune protection qui plus est, tous ces livres rares et autres éditions originales inestimables ?

\textsc{Lui} : On se le demande, en effet.

\textsc{Moi} : Car à voir les livres qui ont été volés, on voit que ces bandits connaissaient leur affaire. Ils sont venus ici en sachant pertinemment ce qu’ils allaient y trouver.

\textsc{Lui} : À n’en pas douter. J’irai porter plainte dès demain au commissariat du quartier.

\textsc{Moi} : Le plus tôt sera le mieux. Il se trouve que j’ai quelques relations au sein de l’OCBC, l’Office central de lutte contre le trafic des biens culturels, je tâcherai de faire en sorte que votre dossier se retrouve en haut de la pile. La chose peut sembler anodine de prime abord, mais les voleurs travaillent le plus souvent pour le compte d’antiquaires et marchands d’art véreux qui financent indirectement le grand banditisme et le terrorisme. À moins, bien sûr, qu’on ait affaire à un type qui travaille en solo, un amoureux des belles-lettres qui cherche à agrandir sa collection sans bourse délier. C’est moins grave, si on veut, mais ça n’en reste pas moins du vol.

\textsc{Lui} : Vous en connaissez ?

\textsc{Moi} : Quoi ?

\textsc{Lui} : Des gens de ce genre.

\textsc{Moi} : Moi non, mais mes collègues certainement. C’est leur métier de suivre toutes les pistes, traquer les criminels, notamment sur Internet, le Dark Web en particulier. C’est parfois des gens comme vous, qui n’ont l’air de rien, des citoyens exemplaires, au-dessus de tout soupçon, qui brûlent d’une passion dévorante pour l’art et n’ont pas d’autre moyen de la satisfaire que de s’approprier des choses qui ne leur appartiennent pas. Si vous les interrogez à ce sujet, ils vous répondront qu’ils n’ont rien fait de mal, dans la mesure où leur intérêt exclusif et désintéressé pour la chose les rend seuls dignes de sa possession. Vous avez vu Les Pleins Pouvoirs, de et avec Clint Eastwood ?

\textsc{Lui} : Non.

\textsc{Moi} : Le héros de ce film est précisément un type de ce genre, qui vole des tableaux de maîtres pour avoir le plaisir et le privilège d’être le seul à les admirer.

\textsc{Lui} : Vous m’en direz tant.

\textsc{Moi} : C’est un peu le cas avec vos livres, non ?

Lui, posant sur moi un regard bleu délavé chargé de mépris et de consternation : Je ne les ai pas volé, à ce que je sache.

\textsc{Moi} : Non, bien sûr, mais les aimez profondément.

\textsc{Lui} : J’y suis viscéralement attaché, c’est vrai, et leur valeur marchande n’a rien à voir là-dedans. Comme je vous l’ai dit, j’ai hérité de la plupart de ces ouvrages, en tout cas tout ceux que mon salaire de prof ne m’aurait jamais permis de m’offrir. Je me fiche que vous preniez pour un cinglé, un maniaque, un ermite qui passe son temps enfermé chez lui dans la pénombre, entouré de grimoires poussiéreux et de déchets alimentaires en voie de putréfaction. J’aimerais qu’on les retrouve, mais je survivrai à leur disparition. Je ne dis pas que ce sera facile, que je ne serai pas la proie d’une puissante et atrocement douloureuse sensation de manque, mais je survivrai. J’ai survécu à bien pire que ça, vous pouvez me croire.

\textsc{Moi} : Ah bon ?

\textsc{Lui} : Oui. Mais inutile d’insister, vous n’en saurez pas davantage.

\textsc{Moi} : Ne vous en faites pas, je n’ai aucunement l’intention d’insister. Non que ça ne m’intéresse pas, encore que, mais la journée a été longue, interminable, même, et plus vite je serai dans mon lit, mieux je me porterai. Pour ce qui est de vos bouquins, j’essayerai de voir ce que je peux faire. Pas grand-chose, je le crains, le rayon culturel n’étant pas le mieux pourvu de la police nationale, d’ailleurs assez dépourvue en général. Cela dit, si j’étais vous, je sécuriserais un minimum les lieux. Pour quelques dizaines d’euros, on trouve des caméras de surveillance connectées qui permettent de voir et entendre ce qui se passe chez soi à distance et de recevoir des messages d’alerte en cas d’intrusion. Vous avez un téléphone portable ?

\textsc{Lui} : Bien sûr, oui. Vous me prenez pour quoi ? Un homme préhistorique ?

\textsc{Moi} : Je ne sais pas, vu que n’avez pas la télévision.

\textsc{Lui} : Pourquoi faire ? Regarder des jeux et des feuilletons débiles, m’enquiller des kilomètres de spots publicitaires avilissants et m’abrutir devant des chaînes d’info en continu qui servent d’organes de propagande à des milliardaires d’extrême-droite ? Très peu pour moi, merci.

\textsc{Moi} : Je suis tout à fait de votre avis.

\textsc{Lui} : Etre vautré devant un écran et regarder défiler des images en bouffant du pop-porn ne correspond pas vraiment à l’idée que je me fais de l’existence.

\textsc{Moi} : Et vous avez parfaitement raison. Bon, c’est pas que je m’ennuie, mais il faut vraiment que j’aille me coucher, maintenant. Je suis éreinté.

Jacquinot, d’une voix, qui jusqu’ici procurait une sensation à peu près aussi agréable que de se faire râper vigoureusement la couenne avec du papier de verre, devenue soudain aussi suave et onctueuse qu’une épaisse couche de crème pâtissière : Vous voulez boire un verre ? Je crois qu’il me reste une bouteille de Bisset.

\textsc{Moi} : De quoi ?

\textsc{Lui} : De Bisset, un apéritif au quinquina, un arbuste originaire de la cordillère des Andes. C’est un peu comme Byrrh, Dubonnet ou le Cap Corse de Mattei, une mistelle de cépages locaux dans laquelle on fait macérer de l’écorce de quinquina et des plantes en proportions variables. Très populaire à la Belle Époque, aujourd’hui totalement passée de mode. La quinine est un antidouleur et un antipaludéen naturel, mais c’est aussi un excitant dont l’usage intempestif peut avoir des conséquences graves. Le président Félix Faure, par exemple, est mort à l’Élysée dans les bras de sa maîtresse, le demi-mondaine Marguerite Japy. La coquine était goulue et le président avait besoin de fortifiant pour survivre à ses assauts. Sauf que cette fois il en est mort, manifestement. Ça vous dit ?

Moi, me dirigeant ostensiblement vers la sortie : Merci, sans façon.

\textsc{Lui} : Il est un peu tard, c’est vrai.

Moi, alors qu’il se tenait devant moi et entravait ma progression : Un peu, oui. Pardon, excusez-moi.

\textsc{Lui} : Ce sera pour une autre fois, alors.

Je n’en demandais pas tant, la distance respectueuse qui prévalait jusqu’ici dans nos relations me convenant en réalité tout à fait, même s’il m’arrivait parfois, comme tout le monde et pour me donner bonne conscience, de dénoncer la solitude grandissante dont mes contemporains faisaient l’objet en dépit d’une proximité accrue jusqu’à l’excès, l’overdose sociale, la surcharge pondérale existentielle, de regretter la convivialité tribale des temps anciens, déplorer l’absence de communication entre voisins, absence de communication dont on a la plupart du temps toutes les raisons de se féliciter compte tenu de la relative absence d’intérêt, pour ne pas dire la nullité stratosphérique du voisinage en question, et absence de communication sournoisement encouragée par l’hyperconnectivité ambiante, le fantasme communautaire : Oui, nous aurons certainement l’occasion de nous recroiser dans l’ascenseur ou l’escalier. Pardon…

\textsc{Lui} : Que nenni, je me ferai une joie de vous inviter chez moi ! (Et d’ajouter tout sourire, apparemment satisfait de son trait d’humour :) Comme ça vous ne serez pas obligé de pénétrer par effraction.

Moi, courant presque pour échapper aux griffes de Jacquinot : La porte était ouverte. Jamais je ne me serais permis autrement.

Lui, posant sur moi son regard délavé de reptile bigleux, d’une voix dégoulinante de chantilly périmée : Et vous avez bien fait ! Sans cela, je n’aurais sans doute jamais su qu’un voisin aussi instruit et passionnant résidait à deux pas de chez moi.

Je me suis abstenu de lui dire que mon rythme de lecture n’excédait pas les deux ou trois livres par an, tout comme j’ai omis de préciser qu’il m’arrivait fréquemment, même pas à ma grande honte, de ne lire que le début et la fin des ouvrages en question, voire les acheter, les remiser dans un coin et ne les ouvrir que six ou sept ans plus tard, en tombant dessus par hasard après avoir oublié leur existence. Non, tous ces détails sans importance devaient rester à tout jamais enfouis sous une épaisse couche de vernis protecteur si je ne voulais pas que mon aura de détective esthète et cultivé (et parfois un tantinet pédant, à la façon d’un Sherlock Holmes ou un Poirot de la grande époque) ne tombe en poussière tel un vampire exposé aux premières lueurs de l’aube.

J’ai donc glissé un pied dans l’embrasure de la porte, que j’étais enfin parvenu à entrouvrir, et laissé négligemment tomber aux pieds de mon interlocuteur énamouré : Vous êtes trop aimable. Je possède en effet un niveau de culture générale qui n’est pas complètement négligeable, je vous l’accorde, mais je ne me considère en aucune façon comme quelqu’un de particulièrement instruit ni passionnant.

Lui, tout sourire, dévoilant une dentition approximative à tout point de vue, tant par la disposition, la forme et les manquements que la couleur, d’un jaune pisseux du plus mauvais effet, très universitaire, à des années-lumière de celle d’un Tom Croûte, un Bras de Bite ou même un George Clownesque refait à neuf (jeux de mots dont il est assez difficile de rendre toute la finesse dans la langue de Shakespeare, Fitzgerald, Poe, Salinger ou Harriet Beecher Stowe, et sans doute plus encore dans le dialecte du New Hampshire tel qu’on le parle encore~-- de moins en moins, hélas~-- dans le comté de Sullivan, je m’en excuse d’avance auprès du traducteur qui aura la lourde charge de rendre cet ouvrage accessible aux anglophones) : Et vous, cher voisin, vous êtes bien trop modeste ! Si vous avez besoin d’un livre, parler à quelqu’un ou quoi que ce soit d’autre, n’hésitez pas à venir me voir.

Moi, prenant le fuite : Je n’y manquerai pas. (Traduction : Tu peux toujours courir, face de pet ! Plutôt mille fois crever dans d’atroces souffrances que de remettre les pieds dans ton gourbi !)

\textsc{Lui} : Au revoir, monsieur Beauvais. À très bientôt, j’espère.

«L’espoir fait vivre, pauvre con !»

Voilà en substance ce que je me suis dit en prenant congé de Jacquinot, ce qui n’était pas une façon très polie de le remercier de son accueil, je vous le concède, mais n’en traduisait pas moins avec une troublante exactitude les sentiments étranges et pénétrants que sa présence faisait pousser tels des champignons vénéneux sur le terreau fertile de mon imagination débordante. Je dis «étranges», et pour tout dire tout à fait inexplicables, de l’ordre du mystère le plus insondable, aussi épais que la couche de fond de teint sur la tronche d’une rombière de Neuilly-sur-Seine, parce que le fait est que (formule pas très gracieuse littérairement parlant, mais ça le fait si on a un bon timing de lecture), excepté un certain état de délabrement plus ou moins généralisé qui pouvait, à la longue, susciter l’aversion de la plupart de ses semblables (notamment un physique ingrat dont il n’était, sinon par négligence et désintérêt total pour ce qui touchait aux apparences, aucunement responsable), le pauvre vieux n’avait pas fait grand-chose pour mériter autant d’acrimonie de la part de quelqu’un comme moi, à priori sympathique, ouvert sur la différence et parfaitement à même de faire preuve de tolérance et de compréhension dans les situations les plus désespérées.

Le trajet de retour, quoique bref, fut éprouvant, car je sentais peser sur moi, telle la hache du bourreau prête à s’abattre sur ma nuque, le regard tranchant de Jacquinot qui m’observait depuis le pas de sa porte, tout à la joie de s’être fait un nouvel ami, lequel nouvel ami était d’ailleurs le seul puisqu’il n’en avait à ma connaissance pas d’autre. Et cette lourde responsabilité, celle d’être le seul et unique ami de Jacquinot, quitte à passer pour une ordure de premier choix, un cloporte au cœur sec, une larve immonde baignant dans une mare de pus sanguinolent, je vous avouerai franchement que je n’étais pas prêt à l’assumer. Au fond de lui-même, le pauvre vieux se félicitait que ce regrettable incident (le vol de ses précieux bouquins) lui ait permis de faire plus ample connaissance avec son voisin de palier. En effet, même s’il était à priori difficile de ranger ce dernier dans la catégorie des intellectuels de haut vol, il n’en cachait pas moins, sous des dehors rugueux, un esprit sinon brillant, n’exagérons rien, au moins affûté et ouvert à des perspectives dépassant de très loin de cadre de ses attributions. Il ne doutait pas un instant que ce nouvel ami, tout acquis à sa cause, userait de toutes les attributions en question pour l’aider à retrouver ses trésors disparus. Il va sans dire que pour ma part, en dépit des belles paroles prononcées dans le feu de l’action, je n’avais pas la moindre intention de lever le petit doigt pour lui, pas plus que je n’avais l’intention d’aller boire un verre chez lui et encore moins de l’inviter à le faire chez moi. Au contraire, comme nous avions parfaitement réussi à le faire jusqu’à présent, j’avais la ferme intention que nous continuions à vivre comme des étrangers et ne nous adresser la parole qu’en de très rares occasions, par exemple quand nous nous retrouvions coincés ensemble dans l’ascenseur et n’avions par conséquent pas le moyen de faire autrement (enfin si, on aurait pu, mais c’est quand même plus difficile de ne pas parler à quelqu’un quand vous vous retrouvez coincé avec lui dans un espace restreint pour un temps indéterminé ; l’étroitesse des lieux et l’atmosphère irrespirable font que vous êtes rapidement amenés à échanger quelques mots, ne serait-ce que pour discuter de la meilleure stratégie pour échapper à ce cauchemar). Les quelques mots que nous échangions annuellement suffisaient à mon bonheur, et il m’avait maintes fois fait comprendre que cela suffisait au sien. Pourquoi changer une équipe qui gagne, risquer sur un malentendu, une occurrence hasardeuse et sans avenir, de briser une absence totale d’amitié qui avait fait ses preuves et nous apportait à l’un et l’autre la plus entière satisfaction ? C’est peu dire que je me maudissais la curiosité, ou plutôt la déformation professionnelle qui m’avait poussé à m’introduire dans l’antre de cette créature visqueuse à l’haleine de putois, et plus encore cette intempestive envie de chier qui m’avait poussé à utiliser ses toilettes. J’aurais mille fois mieux fait de m’abstenir, quitte à chier dans mon froc et regagner piteusement mon logis, sachant que je devrais fournir des explications à ce sujet. Je mettais cette boulette (et ce paquet de merde) sur le compte de l’alcool, dont j’avais quelque peu abusé au cours de la soirée, et jurais qu’on ne m’y reprendrait pas.

C’est donc en rasant les murs, tel un affreux mille-pattes se déplaçant furtivement dans l’obscurité, que j’ai longé le couloir jusqu’à chez moi, me suis retrouvé devant ma porte, ai introduit (non sans difficulté, encore sous le coup de l’alcool et l’émotion) ma clé dans la serrure, et fait effectuer à ladite clé un tour complet dans le sens inverse des aiguilles d’une montre. Cet acte, dont l’apparente simplicité n’en mobilise pas moins une certaine technicité (j’en veux pour preuve qu’il n’est pas si facile de crocheter une serrure multipoints sans savoir-faire ni équipement adéquat), a eu pour effet de déverrouiller la situation et me permettre de regagner enfin mes chères pénates. Celles-ci, vous le savez maintenant, m’étaient d’autant plus chères que m’attendait à l’intérieur cette authentique et intrépide aventurière des temps modernes qui avait su, à force de patience et de ténacité, se frayer un chemin à travers la jungle de ma vie sentimentale. On pourrait aussi parler de désert, bien sûr, mais je préfère parler de jungle, parce que dans le désert on se fait chier, un peu comme sur un radeau de fortune au milieu de l’océan, alors que dans la jungle on se fait chier aussi, c’est vrai, à essayer de sauver sa peau par tous les moyens, mais au moins on n’a pas le temps de s’ennuyer (vous allez me dire qu’en mer il y a toujours moyen de survivre à des tempêtes ou des attaques de requins, mais quand même c’est pas pareil, c’est plus plat). C’est ainsi, sortie tout droit de The Lost City of Z, que l’implacable héroïne, après avoir déjoué les pièges les plus sournois et manqué mille fois de se faire dévorer par les cannibales du coin, avait fini par découvrir, tapi dans un écrin de verdure soigneusement dissimulé au milieu d’un inextricable enchevêtrement de plantes carnivores et autres lianes toxiques, ce temple inviolé que toute femme digne de ce nom rêve secrètement de profaner au moins une fois dans sa vie, je veux bien sûr parler de mon cœur.

Et elle avait parfaitement réussi son coup, avec une efficacité redoutable, au point que le célibataire endurci que j’étais s’était ramolli aussi sûrement une vieille tranche de pain trempée dans du lait (ou tout autre sorte de liquide alcoolisé ou non, ça marche aussi).

Je suis rentré sans faire de bruit, allé avaler un grand verre d’eau à la cuisine, me laver les dents à la salle de bain, après quoi je me suis dirigé vers la chambre en essayant tant bien que mal de ne pas faire grincer le plancher, lequel n’était pas de la première jeunesse et prenait un malin plaisir à hurler dès qu’on posait le pied dessus.

Et c’est là, croyez-le ou non, qu’elle m’attendait au pied du lit, dans le plus simple appareil, hormis ce gode-ceinture long et effilé, de couleur rose pâle, dont il nous arrivait d’user dans nos ébats. Je m’étais promis de ne donner aucun détail sur ma vie privée, ses aspects les plus intimes notamment, et voilà que je me retrouve au pied du mur. J’aimerais par conséquent que tout ce que je vais dire ici, ou fortement sous-entendre, ne sorte jamais de ces pages. Ce sera notre petit secret, si vous le voulez bien, et n’allez surtout pas vous faire des idées : il ne s’agissait que d’un petit jeu entre nous, somme toute bien innocent, à mille lieues des turpitudes sacrilèges que vos esprits mal tournés vont s’empresser d’imaginer.

Toujours est-il que quand je l’ai vue dans cet accoutrement, les premiers mots qui sont sortis de ma bouche, dans un souffle, ont été les suivants : Oh non, chérie, pas ce soir !

Ce à quoi elle a répondu : C’est à cette heure-ci que tu rentres ?

J’avais à peine la force de parler : Cette journée a été la plus longue et éreintante de toute mon existence.

Elle, l’air sévère : Ça fait des heures que je t’attends !

\textsc{Moi} : Je t’avais dit que je rentrerais tard, si toutefois je revenais vivant de cette aventure.

Elle, intraitable : Tout de même, tu mérites une punition.

\textsc{Moi} : Tu n’es pas contente que je rentre vivant de cette aventure ?

\textsc{Elle} : Si, ça va me permettre de te donner la punition que tu mérites.

Moi, aussi déterminé que mon état de fatigue générale, proche du dépôt de bilan et la liquidation totale, me le permettait : Désolé, mais ce sera pour une autre fois.

\textsc{Elle} : Tu sors d’où, d’abord ?

\textsc{Moi} : De chez le voisin.

\textsc{Elle} : Jacquinot ?

\textsc{Moi} : Oui, Jacquinot.

\textsc{Elle} : Je peux savoir ce que tu foutais chez Jacquinot ?

La chère âme, d’ordinaire si tendre et délicate, pouvait se transformer en véritable harpie. Sur une échelle de zéro à dix, j’estimais mes chances de survie à deux ou trois.

Moi, tout en me désapant : C’est une longue histoire. Je te la raconterai demain, si tu veux bien. Enlève ce truc et viens me rejoindre au lit.

\textsc{Elle} : Tout s’est bien passé ?

\textsc{Moi} : Non, pas vraiment. Mais je t’en supplie, laisse-moi me coucher et dormir quelques heures. Je te raconterai tout ça demain.

\textsc{Elle} : Tu ne veux vraiment pas que t’en mette un petit coup pour t’aider à t’endormir ?

\textsc{Moi} : Merci, ça ira. En plus, je ne suis pas tout à fait à l’aise de ce côté-là.

\textsc{Elle} : Ah bon ? Comment ça ?

\textsc{Moi} : Je suis barbouillé, ballonné. Et j’ai la chiasse, si tu veux tout savoir, à tel point que j’ai dû aller me soulager toute affaire cessante chez Jacquinot.

\textsc{Elle} : Tu n’as pas digéré le bollito misto à la viande de poulain et couenne de porc ?

\textsc{Moi} : Faut croire que non, mon amour. C’est bien la première fois que ça m’arrive. Je pense que c’est dû au stress, à la fatigue.

\textsc{Elle} : Tu travailles trop, voilà tout. Et tu ne pouvais pas venir te soulager ici ?

\textsc{Moi} : Tu te doutes bien que si je suis allé me soulager chez Jacquinot, c’est que je n’avais pas le choix de faire autrement. Personne ne s’amuse à aller chier chez Jacquinot si une autre solution s’offre à lui. Il a une superbe collection de livres anciens, soit dit en passant.

\textsc{Elle} : C’est vrai ?

Moi, nu : Et comment, que c’est vrai ! Et on lui en a piqué une bonne partie ce soir, raison pour laquelle j’étais chez lui. On se couche ?

Elle, plus douce que la mousse qui pousse en douce dans la cambrousse : On se couche.

\textsc{Moi} : Enlève ce truc, tu veux. je n’ai vraiment pas la tête à ça.

\textsc{Elle} : Oui, mon chéri. J’avais pensé te faire une petite surprise, mais je vois bien que ce n’est pas le bon moment.

Moi, assis sur le bord du lit : Non, vraiment pas. Cela dit, t’es vraiment un amour. Approche.

Elle a enlevé le… enfin, ce que vous savez, et s’est approchée en ondulant telle une charmeuse de serpent (j’entendais, dans le fond de mon crâne transformé en caisse de résonance improvisée, façon calebasse, cucurbitacée, balafon, tambour d’eau et nébuleuse de l’Œuf pourri, un petit air de pipeau qui me vrillait les synapses et me faisait des nœuds dans la cervelle).

J’ai posé mes mains sur ses fesses, qu’elle avait aussi rondes et fermes que celles de Melanie Nunes Fronckowiak (élue Miss-Plus-Beau-Cul-du-Monde en 2008, après la somptueuse Kristina Dimitrova en 2007, pour info je rappelle que ce grand concours international, organisé par la marque Sloggi, réunissait le gratin de la fesse mondiale), et j’ai commencé à lui bouffer le nombril. J’adorais lui bouffer le nombril, c’était quelque chose que je n’aurais pas hésité un seul instant à placer en tête de liste de mes activités favorites, bien avant la pêche à la truite, l’homicide avec préméditation et l’écoute en boucle de l’adagio assai du Concerto en sol de Ravel (lequel, peu de temps après la composition de ce chef-d’œuvre intemporel, allait malheureusement devoir mettre un terme à ses activités pour cause de paralysie supranucléaire progressive). Il y a toute sorte de nombril, plus ou moins appétissant, mais le sien, véritable petite perle nacrée nichée au creux d’un écrin soyeux, représentait pour moi ce qui se faisait de mieux en la matière. En temps normal, il suffisait que je m’attaque à ce délicieux amuse-bouche pour devenir aussitôt la proie d’un appétit dévorant. Mais pas ce soir. Non, ce soir son nombril me faisait l’effet d’un vieux chewing-gum insipide à force d’avoir été mâché et remâché. Je me suis demandé ce que je foutais là, en train de léchouiller le nombril d’une femme que je connaissais à peine, ce que tout le monde foutait là, pourquoi tous ces gens continuaient à s’agiter vainement à la surface d’un caillou perdu au fin fond de l’univers, pourquoi on avait des bras, des jambes, des fesses, des langues, des yeux, des nez et des oreilles, à quoi rimaient toutes ces conneries qu’on était obligé de supporter à longueur de journée, pourquoi on se cassait le cul à naître si c’était pour disparaître quelques années plus tard, pourquoi on ne pouvait pas naître une bonne fois pour toutes et vivre jusqu’à la fin des temps sans avoir à se reproduire à tout bout de champ, pourquoi on n’arrêtait pas de se poser idiotes, et surtout pourquoi je n’étais pas déjà au lit en train de ronfler comme un bienheureux en attendant la fin du monde, la fin de tout en admettant qu’il y ait eu un début à quoi que ce soit. Tous ces concepts foireux m’insupportaient et ne m’inspiraient qu’une longue série de bâillements à m’en décrocher la mâchoire, tandis que ma langue, semblable à une limace anémique, une larve exsangue engluée dans sa propre bave, s’épuisait à récurer le bouton d’or de ma fleur des champs préférée.

