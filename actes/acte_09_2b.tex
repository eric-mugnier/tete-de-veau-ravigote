Je me suis raclé le fond de la gorge, et Dumo, le crétin des Alpes au crâne bosselé trempé de sueur, a enfin daigné jeter un œil sur mon humble personne.

Il a dit : Bonjour monsieur.

J’ai répondu : Bonjour Dumo.

Il a ajouté : Que puis-je faire pour votre service ?

J’ai immédiatement reconnu le son de sa voix : c’était celle que j’avais entendue cette nuit dans l’interphone, quand notre joyeuse petite bande était venue sonner à la porte du Caribbean.

L’accueil n’avait pas été des plus chaleureux, et ne l’était pas davantage maintenant.

J’ai demandé : Vous me reconnaissez ?

\textsc{Lui} : Pas le moins du monde, monsieur. Vous désirez une chambre ?

\textsc{Moi} : Oui.

\textsc{Lui} : Désolé, monsieur, mais l’hôtel est complet.

\textsc{Moi} : Ça m’étonnerait.

\textsc{Lui} : Vraiment, monsieur ?

\textsc{Moi} : C’est à vous que j’ai eu affaire cette nuit, n’est-ce pas ?

\textsc{Lui} : Cette nuit, monsieur ? Ça m’étonnerait, je n’étais pas de service.

\textsc{Moi} : Vous l’étiez.

\textsc{Lui} : Vous croyez ?

\textsc{Moi} : J’en suis certain. Tout comme je suis certain que vous vous rappelez très bien de moi.

\textsc{Lui} : Si vous le dites.

\textsc{Moi} : Donc, vous savez qui je suis.

\textsc{Lui} : Pas le moins du monde, monsieur. Mais la première chose qu’on vous apprend, à l’école hôtelière, c’est de ne jamais contredire le client, quelles que soient les inepties qu’il profère.

\textsc{Moi} : Dans ce cas, on doit vous apprendre aussi qu’il ne faut jamais contredire un policier dans l’exercice de ses fonctions.

\textsc{Lui} : Monsieur est policier ?

\textsc{Moi} : Commissaire Beauvais, de la police judiciaire.

\textsc{Lui} : Policier ou pas, je vous assure que l’hôtel est complet.

\textsc{Moi} : Je ne pense pas, non.

\textsc{Lui} : C’est votre droit le plus strict.

\textsc{Moi} : Et je sais très bien que c’est à vous que j’ai parlé cette nuit. Vous m’avez parlé de Gustav Nachtigal\nf{Gustav Nachtigal (1834--1885), médecin et explorateur allemand. Il traversa le Sahara et explora le Sahel dans les années 1860--1870, puis joua un rôle décisif dans l’établissement du protectorat allemand sur le Cameroun, le Togo et le Sud-Ouest africain (actuelle Namibie) en 1884, au nom du chancelier Bismarck. \textit{Source :} \textup{fr.wikipedia.org/wiki/Gustav\_Nachtigal}}, Bismarck\nf{Otto von Bismarck (1815--1898), homme d’État prussien et allemand, surnommé le «~Chancelier de fer~». Artisan de l’unification allemande en 1871, il dirigea l’Empire allemand jusqu’en 1890 et joua un rôle majeur lors de la conférence de Berlin (1884--1885) qui organisa le partage colonial de l’Afrique. \textit{Source :} \textup{fr.wikipedia.org/wiki/Otto\_von\_Bismarck}}, Göring\nf{Hermann Göring (1893--1946), militaire et homme politique nazi allemand. Maréchal du Reich et commandant de la Luftwaffe, il fut l’un des principaux dirigeants du Troisième Reich. La Namibie (alors Sud-Ouest africain allemand) fut gouvernée un temps par son père Heinrich Ernst Göring, nommé commissaire impérial. Condamné à mort à Nuremberg, il se suicida la veille de son exécution. \textit{Source :} \textup{fr.wikipedia.org/wiki/Hermann\_G\%C3\%B6ring}}, et expliqué que cet hôtel est officiellement rattaché à l’ambassade de Namibie et interdit aux Blancs.

\textsc{Lui} : Exact. Et il me semble aussi vous avoir demandé de revenir avec un mandat en bonne et due forme. Vous l’avez ?

\textsc{Moi} : Non.

\textsc{Lui} : Dans ce cas, je ne suis pas certain de pouvoir faire grand-chose pour vous. D’autant que je suis très occupé, comme vous pouvez le constater.

\textsc{Moi} : Vous m’avez également affirmé être un métis du Cap, fils d’une mère française et d’un héros de la bataille de Salt River\nf{La bataille de Salt River (1510) est une confrontation entre des marins portugais conduits par Francisco de Almeida et des guerriers Khoïkhoï près du Cap. Elle se solda par la mort d’Almeida et d’une cinquantaine de Portugais, constituant l’une des premières résistances armées victorieuses des peuples d’Afrique australe contre les Européens. \textit{Source :} \textup{fr.wikipedia.org/wiki/Francisco\_de\_Almeida\#La\_mort\_d\%27Almeida}}. Au cours de cette bataille, restée légendaire dans la longue et douloureuse histoire de l’Afrique du Sud, les indigènes ont vaincu les Portugais et forgé leur réputation de courage et de férocité.

\textsc{Lui} : Je n’ai rien d’autre à rajouter.

\textsc{Moi} : Je ne suis pas venu en ennemi. Il se trouve que je suis à la recherche d’un collègue et ami d’enfance, et que j’ai de bonnes raisons de penser qu’il est venu ici cette nuit.

\textsc{Lui}, passant la main sur son crâne trempé de sueur avant de l’essuyer sur sa bedaine : Je vous l’ai dit, cet hôtel est interdit aux Blancs.

\textsc{Moi} : Mon ami est Noir, descendant en droite ligne des King’s American Dragoons du colonel Benjamin Thompson\nf{Benjamin Thompson, comte de Rumford (1753--1814), physicien et inventeur américano-britannique. Loyaliste pendant la guerre d’Indépendance américaine, il commanda les King’s American Dragoons avant de s’établir en Europe. Membre de la Royal Society, il est connu pour ses travaux sur la chaleur et la thermodynamique ainsi que pour ses nombreuses inventions pratiques, dont des améliorations décisives des cheminées et fourneaux. \textit{Source :} \textup{fr.wikipedia.org/wiki/Benjamin\_Thompson}}, comte de Rumford, membre de la Royal Society et auteur d’un remarquable Essai sur les cheminées, avec des propositions pour les améliorer, les empêcher efficacement de fumer, économiser le combustible, et rendre les habitations plus confortables et salubres.

\textsc{Lui} : Intéressant.

\textsc{Moi} : C’est comme je vous le dis.

\textsc{Lui} : Dans ce cas, il est peut-être venu ici. Je peux vérifier dans le registre, si vous me dites son nom.

\textsc{Moi} : Titus Beaugendre, comme un beau gendre.

J’ai sorti mon téléphone et affiché une photo de Titus, prise dans son jardin alors qu’on était en train de faire griller des steaks par une belle soirée d’été.

\textsc{Moi} : Tenez, c’est lui.

\textsc{L’autre}, jetant un coup d’œil à la photo : Ça ne me dit rien. Attendez, je consulte le registre.

Dans la foulée, j’ai affiché la photo de la Gardienne de la Nuit, prise à son insu peu de temps avant qu’elle disparaisse avec Titus dans les profondeurs de la nuit : J’ai tout lieu de penser qu’il se trouvait en compagnie de cette personne.

\textsc{Lui}, levant le nez de son registre : Ça ne me dit rien non plus. Vous connaissez son nom ?

\textsc{Moi} : Elle a dit s’appeler Atiena, et exercer la profession de gardienne de la nuit.

\textsc{Lui} : Gardienne de la nuit, dites-vous ?

\textsc{Moi} : Oui.

\textsc{Lui} : Drôle de métier.

\textsc{Lui} toujours, m’arrachant presque le téléphone des mains : Permettez !

\textsc{Moi} : Je vous en prie.

\textsc{Lui}, fixant la photo avec attention, le visage de plus en plus déformé par la perplexité : On dirait…

\textsc{Moi} : On dirait ?

\textsc{Lui} : On dirait Repentance.

\textsc{Moi} : Repentance ?

\textsc{Lui} : Repentance Whittingham. Drôle de nom, n’est-ce pas ?

\textsc{Moi} : Pour le moins. D’origine anglaise, je suppose ?

\textsc{Lui} : Aucune idée. Tout ce que je sais, c’est qu’elle travaille ici comme femme de chambre. Une très belle femme, du reste.

\textsc{Moi} : Vous en êtes certain ?

\textsc{Moi} : Que c’est une très belle femme ? Oui, pour autant que je puisse en juger. Après, c’est une question de goût, bien sûr.

\textsc{Moi} : Non. Je veux dire : vous êtes certain qu’elle travaille comme femme de chambre dans cet établissement ?

\textsc{Lui}, me restituant le téléphone et retournant à son registre : Oui, si c’est bien Repentance. Mais j’avoue que la ressemblance est assez troublante. Par contre, je ne vois aucune trace d’un quelconque Titus Beaugendre.

\textsc{Moi} : Il a pu donner un faux nom.

\textsc{Lui}, refermant le registre et plongeant avec vigueur ses yeux légèrement globuleux dans les miens, comme s’il cherchait à m’hypnotiser : C’est possible. Mais à moins qu’il soit venu déguisé, il me semble que je reconnaîtrais son visage.

\textsc{Moi}, incapable de détacher mon regard du sien : Vous m’avez dit que vous n’étiez pas de service cette nuit.

J’ai senti ma gorge se nouer, comme si des mains invisibles étaient en train de m’étrangler.

\textsc{Lui} : J’ai dit ça ?

\textsc{Moi}, avec la désagréable impression de me tenir en équilibre sur des jambes en caoutchouc : Vous l’avez dit.

\textsc{Lui} : Cette nuit, vous dites ?

\textsc{Moi} : Excusez-moi, mais je ne me sens pas très bien.

\textsc{Lui} : Vous dites que je n’étais pas de service cette nuit ?

\textsc{Moi}, transpirant à grosses gouttes : Il fait une chaleur à crever, ici, vous ne trouvez pas. Vous n’auriez pas un mouchoir à me prêter, par hasard ?

Il pouvait difficilement refuser, vu qu’il en avait des tonnes.

\textsc{Lui} : Mais si, bien sûr.

\textsc{Moi} : Merci, c’est très aimable à vous.

\textsc{Lui} : Mais je vous en prie, c’est tout naturel.

J’ai détaché mon regard du sien, me suis épongé le front, et j’ai commencé à me sentir un peu mieux.

\textsc{Lui} : Ça va ?

\textsc{Moi} : Mieux, merci.

\textsc{Lui} : Cette nuit, disiez-vous ?

\textsc{Moi} : Quoi, cette nuit ?

\textsc{Lui} : Vous disiez que je n’étais pas de service cette nuit ?

J’ai senti qu’il tentait à nouveau d’exercer ses talents de prêtre vaudou à mon encontre, mais j’ai réussi à parer le coup.

\textsc{Moi} : Non, c’est vous qui l’avez dit.

\textsc{Lui} : J’ai dit ça ?

\textsc{Moi} : Oui, vous l’avez dit.

Une lueur de désagrément s’est baladée sur son visage, tel un mille-pattes qui court le long des murs à la tombée du jour.

\textsc{Lui} : Dans ce cas, je me suis trompé. Ce n’est pas cette nuit que je n’étais pas de service, mais celle d’avant. Nous avons des horaires très irréguliers, il n’est pas toujours facile de s’y retrouver.

Je commençais à reprendre du poil de la bête : Je vois. Et cette Repentance Machin-Chose, elle est de service aujourd’hui ?

\textsc{Lui}, affichant un sourire aussi avenant qu’un banc de piranhas dans les eaux troubles de l’Amazone : Whittingham, monsieur. Une très jolie femme, avec des yeux d’une couleur assez indéfinissable.

\textsc{Moi} : Un vert très pâle, si je ne m’abuse.

\textsc{Lui} : Je ne saurais dire exactement. L’hôtel compte près d’une centaine d’employés, vous comprendrez que je ne peux pas me souvenir de la couleur des yeux de chacun d’eux.

\textsc{Moi}, de nouveau en pleine possession de mes moyens : Je comprends admirablement. Mais vous n’avez pas répondu à ma question.

\textsc{Lui}, exhibant une rangée de crocs qui n’auraient pas dépareillé dans le bec d’un requin-bouledogue : Et pourquoi le ferais-je ? Vous avez un mandat, une commission rogatoire ? je suis accusé de quelque chose ?

\textsc{Moi} : Rien de tout ça, rassurez-vous. Il s’agit juste d’une petite discussion entre gens de bonne compagnie, tout ce qu’il y a de plus informelle.

\textsc{Lui} : C’était quoi, déjà, votre question ?

\textsc{Moi} : Est-ce que cette Repentance Whit-je-sais-plus-quoi est de service aujourd’hui ?

\textsc{Lui} : Whittingham, monsieur. «~La colonie du peuple blanc~», en anglais.

\textsc{Moi} : Vous avez fait des études de langues ?

\textsc{Lui} : Disons que j’en parle un certain nombre.

\textsc{Moi} : Et donc ?

\textsc{Lui} : Donc rien. La compagnie cherchait un réceptionniste polyglotte, elle a fait appel à mes services.

\textsc{Moi} : Je parlais de Repentance Whittingham.

\textsc{Lui} : Très belle femme.

\textsc{Moi} : Oui. Mais elle est de service, aujourd’hui ?

\textsc{Lui} : Je ne sais pas.

\textsc{Moi} : Vous avez sans doute l’emploi du temps des employés.

\textsc{Lui} : Oui, je l’ai, en effet. Mais je ne suis pas certain d’avoir envie de le consulter. D’ailleurs, je ne suis pas là pour ça. Je suis là pour accueillir les gens et faire en sorte qu’ils passent un agréable séjour dans notre établissement.

\textsc{Moi} : Vous m’obligeriez.

\textsc{Lui} : Vous vouliez savoir si votre ami était ici, il ne l’est pas, je crois que nous avons fait le tour de la question. Si vous voulez réserver une chambre, pour vous ou quelqu’un d’autre, il y a un petit hôtel très sympathique à deux rues d’ici. Je peux vous donner son nom et son adresse, si vous le désirez.

