
\noindent Quelques jours plus tard, les journaux ont parlé d’un corps retrouvé carbonisé dans le coffre d’une voiture, au cœur d’une vaste friche industrielle abandonnée bien connue pour servir de lieu de rendez-vous à toute une faune de pervers qui venaient se tripoter la nouille entre les tas d’immondices.

L’incendie avait été d’une telle violence que la voiture et le cadavre avaient été nettoyés jusqu’à l’os.

L’inspecteur Djeferson Beauvais, chargé de l’enquête, n’avait pas caché sa terreur face à une scène de crime qu’il estimait digne des pires scènes de guerre et attentats perpétrés contre des populations innocentes par des fanatiques religieux de la pire espèce. Le mot «~pire~» (qui rime avec empire et vampire, ça n’a pas de lien direct mais quand même ça fait réfléchir) était celui qui revenait le plus fréquemment dans sa bouche, et il confessait avoir acquis, au fil du temps et des abominations sans cesse plus atroces auxquelles il était confronté, la certitude que quelque chose ne tournait pas rond chez l’être humain. D’après lui, ça posait un problème de fond sur lequel il faudrait bien que l’humanité décide de se pencher un jour ou l’autre. Et le plus tôt serait le mieux, car au train où allaient les choses, il ne nous resterait bientôt plus que nos yeux pour pleurer.

Avis largement partagé par le docteur Zaahid Shirani, légiste vedette de la PJ dont le regard pénétrant et les interventions régulières sur les plateaux de télé réjouissaient les amateurs d’énigmes policières et crimes sanglants, qui ne cachait pas la difficulté de faire correctement son travail dans des conditions aussi extrêmes.

Djeferson Beauvais, interrogé par la presse sur fond de carcasses de voitures, sol jonché de détritus et arbres faméliques tendant leurs membres décharnés vers un ciel éternellement gris, s’est déclaré fermement décidé à tout mettre en œuvre pour découvrir le fin mot de l’affaire, ajoutant avoir toute confiance en Zaahid Shirani pour mettre la main sur les indices les plus infimes pouvant conduire à l’arrestation du monstre responsable de cette tragédie.

Vous l’aurez compris, outre le fait que j’étais très embêté d’avoir à me rechercher moi-même pour me traduire devant la justice, j’étais aussi très remonté contre Riqueti, que je tenais pour personnellement responsable de la majeure partie de mes emmerdements. S’il n’avait pas envoyé ce tueur pour me dessouder, je n’aurais pas eu besoin de le faire cramer dans le coffre de sa voiture. À celles et ceux qui objecteraient que je n’étais peut-être pas obligé de le faire cramer dans le coffre de sa putain de voiture, je répondrai que l’objection ne tient pas face à la réalité du fait que tout cela n’aurait jamais eu lieu si cet imbécile n’était pas venu chez moi dans l’intention de me faire passer de vie à trépas. La chronologie fait loi, et je ne vois absolument pas en quoi le fait d’avoir fait cramer cet individu dans le coffre de sa voiture infirmerait la validité de mon raisonnement.

Je lui ai donc filé le train, à cet enfoiré, et quand j’ai vu qu’il se dirigeait vers sa luxueuse maison de campagne située dans un environnement protégé et ceinte de murs infranchissables, avec pour seule compagnie son nouveau chauffeur au volant de sa BMW Série 7 High Security (blindage renforcé, vitres multicouches résistant aux tirs de gros calibre, intérieur luxe tout confort, moteur V8 capable d’accélérations décoiffantes, vision infrarouge, technologie de pointe et communications sécurisées), je me suis dit que c’était le moment rêvé d’aller lui poser quelques petites questions sur ses activités récentes, notamment le fait qu’il n’hésite pas à payer des gens pour en tuer d’autres, chose ô combien décevante de la part d’un homme censé représenter Dieu sur terre. On ne jouait pas dans la même catégorie, c’est vrai, mais ce n’était pas un cureton de luxe qui allait me chier dans les bottes.

Non, ce qui m’inquiétait le plus, outre les murs infranchissables de la propriété qu’il allait bien falloir que je me démerde pour franchir d’une manière ou d’une autre, c’était le remplaçant de Dardariel, un type aussi large que haut, avec une tronche de primate mal dégrossi, des oreille en chou-fleur et des battes de baseball à la place des bras. Avec son costard-cravate et son chapeau de cowboy sur la tête, sans doute la seule fantaisie que lui autorisait son employeur, il ressemblait à un flic ricain genre Texas ranger qui tire sur tout ce qui bouge au moindre battement de cil. Et vu sa largeur d’épaules, il était certainement aussi capable de couper des bûches en deux avec le tranchant de la main et assommer un bœuf à coups de bite. Cette fois, j’avais intérêt à la jouer fine si je ne voulais pas me retrouver entre quatre planches.

La veille, et croyez bien que je n’en suis pas fier, j’étais au Narcisse Rose avec Zaahid Shirani et on avait passé la soirée à boire des Girofliers du Clair de Lune (un cocktail à base de rhum, champagne et clou de girofle, spécialement conçu pour faire sortir les morts de leurs tombes et les asticots de leurs claquos) jusque tard dans la nuit. On avait aussi pas mal parlé de la Compagnie des Indes orientales\nf{La Compagnie des Indes orientales fut fondée en plusieurs versions concurrentes~: britannique (\textit{East India Company}, 1600), néerlandaise (\textit{VOC}, 1602) et française (1664, sous Colbert). Instruments du capitalisme colonial, ces sociétés par actions détenaient le monopole du commerce avec l'Asie. La version anglaise gouverna l'Inde de facto jusqu'en 1858. \source{fr.wikipedia.org/wiki/Compagnie\_des\_Indes\_orientales}}, sujet qui semblait le passionner, et surtout du Bangladesh, pays d’où étaient originaires ses grands-parents vénérés, ainsi que du périple invraisemblable qu’ils avaient accompli pour arriver jusqu’à nous. Pour couronner le tout, on avait dû subir les assauts à répétition d’une paire de jumelles florentines, Tosca et Zarina Brizzi, lesquelles, mues par le vice propre aux filles de leur espèce, tenaient absolument à ce qu’on finisse la soirée à quatre dans leur suite du Jade Mountain Hôtel de l’avenue des Rossignols (très bel endroit au demeurant, dans lequel je n’avais encore jamais eu le privilège de mettre un orteil), offre que j’avais poliment déclinée pour des raisons éthiques d’une part, ayant été élevé dans le respect des traditions et valeurs morales en vigueur dans ce pays, pratiques d’autre part, étant promis à une grosse journée le lendemain. En soi, ma seule présence dans ce lieu de perdition était déjà la preuve d’un grave manquement de ma part aux exigences de ma profession, et je me verrais contraint, sitôt de retour dans la pénombre glacée de ma cellule, de me flageller jusqu’au sang pour obtenir le pardon de Dieu le Père, Christ ressuscité et tous les saints apôtres de la Parole divine. Je dois néanmoins reconnaître, à mon corps défendant, que Zarina exerçait sur moi une attraction qui n’avait rien à voir avec les montagnes russes ou le train fantôme. Non, on était plus proche des forces cosmiques qui régissent l’univers. Cela dit, quand on est pourvu d’une volonté d’acier, comme c’est mon cas, on arrive fort bien à se sortir de ce genre de traquenard sexuel dans lequel la nature essaye sans cesse de nous entraîner, la garce.

Shirani, très en verve, m’avait parlé du corps calciné dans le coffre de l’Alfa, réduit peu ou prou à l’état de cendre, et je lui avais répondu avec aplomb qu’il s’agissait sans doute d’un banal règlement de compte entre dealers, comme cela était malheureusement fréquent dans le quartier. Quant à la Giulietta, elle appartenait à un certain Kévin Charrier, domicilié au 18 rue des Nénuphars, que je m’étais fait un plaisir de soumettre à un interrogatoire musclé. En toute méchanceté gratuite, il va sans dire, car il était évident que le particulier, vendeur chez Alfa, n’avait strictement rien à voir dans l’affaire. Au contraire, il s’était fait piquer sa voiture le jour-même et, en bon citoyen responsable et respectueux des lois, était allé aussitôt s’en plaindre au commissariat le plus proche. Là, le planton de service, un petit gros au crâne dégarni et la moustache alerte, lui avait déclaré entre deux bâillements que tout serait diaboliquement mis en œuvre pour la retrouver dans les meilleurs délais. Il s’agissait d’une affaire d’ampleur nationale, d’une priorité absolue qui allait mobiliser les plus hautes instances et les troupes d’élite de la Nation.

Et puis, chemin faisant, on en était venus à parler de choses très intimes concernant la vie du légiste, des choses que je savais déjà plus ou moins, rumeur oblige, mais qu’il ne m’avait encore jamais fait l’honneur de me confirmer de vive voix, et je dirais même très vive voix, assez pour couvrir le niveau sonore de la musique d’ambiance. D’excellente qualité, du reste, la musique en question, dans le plus pur style Five Spot Café, le fameux club de jazz situé dans le Lower East Side de Manhattan\nf{Le \textit{Five Spot Café}, ouvert en 1956 au 5 Cooper Square dans le Lower East Side de Manhattan, fut la scène centrale du jazz moderne et du free jazz entre 1956 et 1967. Thelonious Monk y tint une résidence historique en 1957~; Ornette Coleman y créa le scandale de son quartette en 1959. \source{fr.wikipedia.org/wiki/Five\_Spot\_Caf\%C3\%A9}}. À l’époque, je parle de la fin des années 50, le quartier était une zone sinistrée dans laquelle les honnêtes gens auraient préféré se faire amputer des deux jambes plutôt que d’y mettre un pied. La plupart des résidents étaient des Noirs, et une bonne partie de ces Noirs étaient des musiciens de jazz. Pas le genre pépère à la Glenn Miller, Benny Goodman ou Duke Ellington, ce truc de riches pour s’encanailler dans des bars à putes. Non, un genre plus méchant, sauvage, imprévisible, laissant la part belle à l’imagination débridée de types qui en avaient gros sur la patate, en perdition pour la plupart. Pour eux, l’instrument était une arme qui leur permettait de pleurer en silence ou hurler à la mort, suivant l’état d’ébriété dans lequel ils se trouvaient. Leurs noms ? Vous les connaissez aussi bien que moi, sinon il est grand temps de vous mettre à écouter autre chose que du Sofiane Pamart et du Ibrahim Maalouf : Charlie Parker, John Coltrane, Shadow Wilson, Cecil Taylor, Johnny Griffin, Roy Haynes, Ahmed Abdul-Malik, Don Cherry, Ornette Coleman, Miles Davis, pour ne citer que les plus célèbres, et bien sûr le grand Thelonious Monk\nf{Thelonious Sphere Monk (1917--1982), pianiste et compositeur de jazz américain originaire de Rocky Mount (Caroline du Nord). Figure de proue du bebop, son style harmonique déconcertant et ses silences calculés en firent une des personnalités les plus singulières du jazz. Il vécut les six dernières années de sa vie chez Pannonica de Koenigswarter dans le New Jersey, presque muet. \source{fr.wikipedia.org/wiki/Thelonious\_Monk}} qui reste une énigme pour nombre d’amateurs. C’est dans ce foutoir savamment désorganisé, expressionniste, abstrait, dans lequel trainaient également toute une faune d’artistes dégénérés dans le genre de ceux qu'Hitler rêvait d’exterminer en même temps que les Juifs, les communistes, les francs-maçons et les homosexuels, qu’est arrivée «~la Baronne~», Pannonica de Koenigswarter\nf{Kathleen Annie Pannonica de Koenigswarter, née Rothschild (1913--1988), surnommée «~la Baronne~». Issue de la branche britannique des Rothschild, elle devint la principale mécène du jazz à New York, hébergeant et soutenant Charlie Parker (mort dans sa chambre en 1955) et Thelonious Monk (qui vécut chez elle ses six dernières années). \source{fr.wikipedia.org/wiki/Pannonica\_de\_Koenigswarter}}, Nica pour les intimes, riche héritière au caractère bien trempé, folle de jazz, militante des droits de l’homme, grande amie des minorités oppressées et des Noirs en particulier. À titre d’exemple, c’est dans sa suite de l’hôtel Stanhope\nf{Charlie Parker (1920--1955), saxophoniste alto américain surnommé «~Bird~», figure fondatrice du bebop. Il mourut le 12 mars 1955 à 34 ans dans la chambre new-yorkaise de Pannonica, à l’hôtel Stanhope. Le médecin légiste, stupéfait par l’état de son corps ravagé, lui attribua entre 50 et 60 ans. \source{fr.wikipedia.org/wiki/Charlie\_Parker}}, en face de Central Park, que Charlie Parker, usé par la drogue, rendra son dernier souffle à trente-quatre ans (il en paraissait le double). Née Rothschild, Nica avait à dix ans quand son père vénéré, banquier et naturaliste fou, s’est tranché la gorge dans un moment de déprime. Charles\nf{Charles Rothschild (1877--1923), banquier londonien de la maison Rothschild et entomologiste passionné. Il décrivit plusieurs centaines d’espèces de puces, dont \textit{Xenopsylla cheopis}, principal vecteur de la peste bubonique. Atteint de troubles mentaux sévères (possiblement une encéphalite léthargique), il se donna la mort en 1923 dans une maison de repos. \source{fr.wikipedia.org/wiki/Charles\_Rothschild}}, comme son frère Lionel, vouait une passion immodérée à l’entomologie. Quand il n’était pas en train de glander dans son bureau de Thames Street, dans la City, il battait la campagne un filet à la main avant de rentrer chez lui au pas de gymnastique pour épingler ses trouvailles dans des boites en carton. Il avait aussi une passion dévorante pour les puces, ces saletés de suceuses de sang capable de sauter plus de trois cents fois leur taille, soit l’équivalent pour un être humain d’un bond au-dessus de la Vostok Tower\nf{La Vostok Tower (374 m), achevée en 2016 dans le quartier d’affaires de Moskva-City (Complexe de la Fédération) à Moscou, est le deuxième plus haut gratte-ciel de Russie. Son nom évoque le programme spatial soviétique Vostok. \source{fr.wikipedia.org/wiki/Vostok\_Tower}} du Complexe de la Fédération, à Moskva-City. Sur les quelques vingt mille espèces aujourd’hui répertoriées, nombre qui file le vertige et montre à quel point les parasites les plus assoiffés de sang disposent d’une formidable capacité d’adaptation, et je ne dis pas ça pour nous, Charles Rothschild en a décrit quelques centaines, ce qui n’est déjà pas si mal pour l’époque même si ça ne suffit pas à faire de lui un naturaliste de premier plan. D’autant qu’il était blindé de thune et disposait de tout le loisir nécessaire pour jouer les Jean-Henri Fabre\nf{Jean-Henri Fabre (1823--1915), entomologiste et naturaliste français, auteur des dix volumes des \textit{Souvenirs entomologiques} (1879--1907). Victor Hugo le surnomma «~l'Homère des insectes~». Son nom est associé à plus de 200 espèces et genres d'insectes. \source{fr.wikipedia.org/wiki/Jean-Henri\_Fabre}}. Mais bon, ce n’est pas si courant pour un banquier de se passionner pour les insectes, même si l’esprit affûté ne manquera pas de voir un lien direct entre les puces qui sucent le sang des gens et les banquiers qui leur sucent leur pognon. Pannonica s’appelait en réalité Kathleen Annie Pannonica de Rothschild, Pannonica étant le seul resté à la postérité, et elle avait épousé le baron Jules de Koenigswarter, ancien de Janson-de-Sailly et diplômé de l’École des Mines, héros de la Campagne de Tunisie et du débarquement à Cavalaire-sur-Mer sous les ordres du général de division Jean Touzet du Vigier\nf{Jean Touzet du Vigier (1896--1969), général de corps d'armée français. Il commanda la 1re Division blindée lors de la Campagne de Tunisie (1943) puis le débarquement de Provence (août 1944) à Cavalaire-sur-Mer à la tête du 1er Corps de la 1re Armée française. \source{fr.wikipedia.org/wiki/Jean\_Touzet\_du\_Vigier}}. Il a une bonne tête mais ce n’est pas forcément le mec le plus marrant qui soit. En même temps, comme il n’est jamais là, sa présence ne constitue pas un obstacle majeur à leur union. Pannonica, prénom étrange s’il en est, vient de Pannonie, ancienne province de l’Empire romain correspondant plus ou moins à l’actuelle Slovénie, région où Charles s’est offert le luxe de mettre la main sur une nouvelle espèce de papillon de nuit. Le 1er juin 1954, Thelonious Monk, 37 ans, originaire de Rocky Mount en Caroline du Nord (état qui, je le rappelle, appliquait entre 1927 et 74 le programme de stérilisation forcée contre les Noirs et les Indiens, entreprise génocidaire validée par le juge à la Cour suprême des États-Unis Oliver Wendell Holmes\nf{Oliver Wendell Holmes Jr. (1841--1935), juge associé à la Cour suprême des États-Unis (1902--1932). Dans l'arrêt \textit{Buck v.\ Bell} (274 US 200, 1927), il rédigea la décision validant la stérilisation forcée, concluant~: «~Three generations of imbeciles are enough.~» Cette décision ne fut jamais officiellement annulée. \source{fr.wikipedia.org/wiki/Oliver\_Wendell\_Holmes\_Jr.}}), est à Paris pour participer au troisième Salon du Jazz, salle Pleyel\nf{La salle Pleyel, inaugurée en 1927 au 252 rue du Faubourg-Saint-Honoré à Paris (8e), est l’une des grandes salles de concert françaises. Le \textit{Salon du jazz} du 1er juin 1954 y réunit pour la première fois de nombreux jazzmen américains devant un public parisien, dont Thelonious Monk, Sidney Bechet et Gerry Mulligan. \source{fr.wikipedia.org/wiki/Salle\_Pleyel}}, en même temps que Sidney Bechet, le Gerry Mulligan Quartet, Jonah Jones, Claude Luter et Martial Solal. Il se fait copieusement huer par la foule en délire, qui le traite de singe débile et maniaque, avant de se faire éreinter par la critique qui émet elle aussi l’idée qu’il serait plus à sa place à sauter de branche en branche dans la jungle qu’à jouer du piano dans une salle de concert. Dans son malheur, il croise celle qui deviendra sa plus fidèle groupie, indéfectible et riche amie, j’ai bien sûr nommé Pannonica de Koenigswarter, qui décide aussitôt de le prendre sous son aile et faire en sorte qu’il ne manque plus jamais de rien. Il faut savoir que Monk souffrait des mêmes désordres mentaux que Charles, le père de Nica, et que leur comportement était similaire à bien des égards, ce qui explique sans doute pourquoi elle a pris soin de lui pendant près de trente ans, dans l’espoir de ne pas le voir se trancher la gorge à son tour. C’est ainsi qu’il a passé les six dernières années de sa vie chez elle, dans un état de délabrement physique et moral croissant, sans toucher à un piano et se pissant dessus toutes les trente secondes pour cause de dérèglement aigu de la prostate, et s’est éteint pour ainsi dire dans ses bras de sa bienfaitrice, dans le New Jersey (comme pas mal d’autres musiciens noirs, qui adoraient mourir dans le New Jersey, sans doute un des endroits les plus agréables pour mourir aux USA, et plus spécialement dans le comté de Bergen, à Englewood, où dès 1906 l’écrivain et militant socialiste Upton Sinclair\nf{Upton Sinclair (1878--1968), romancier américain, auteur de \textit{La Jungle} (1906), réquisitoire contre les abattoirs de Chicago qui provoqua une réforme sanitaire fédérale. Socialiste militant, il fonda en octobre 1906 à Englewood (New Jersey) l’\textit{Helicon Home Colony}, communauté utopique détruite par un incendie d’origine suspecte en mars 1907. \source{fr.wikipedia.org/wiki/Upton\_Sinclair}} fonde sa Helicon Home Colony, sorte de communauté pré-hippie ravagée par un incendie d’origine suspecte peu de temps après son installation).

Donc, comme je vous le disais, alors même que les sœurs Brizzi nous reluquaient avec insistance en se passant ostensiblement la langue sur les lèvres (et j’étais terrorisé tant je sentais que celle de gauche, dont je ne savais pas encore qu’elle s’appelait Zarina, mourait d’envie de me dévorer tout cru, et encore plus tant je devais bien admettre que cette perspective cannibalistique exerçait sur moi un véritable attrait), Zaahid Shirani m’a confié avec un soulagement manifeste (merci le Giroflier du Clair de Lune, breuvage magique qui n’a pas son pareil pour délier les langues les plus ratatinées dans le fond de leur gosier) qu’il avait une fille, Jaya (enfin, je savais déjà qu’il avait une fille et qu’elle s’appelait Jaya, mais il s’était toujours montré extrêmement évasif à son sujet et j’avais toujours eu la délicatesse de ne pas insister), qui lui causait bien du souci depuis des années, plus précisément depuis que sa mère Faustina, d’origine galicienne (elle avait vu le jour à Saint-Jacques-de-Compostelle), avait avalé assez d’alcool et de cochoncetés pharmaceutiques pour décimer un troupeau d’éléphants dans la force de l’âge.

Jaya avait douze ans au moment des faits, et, sans doute parce que le destin, d’une cruauté sans limites, prend plaisir à s’acharner sur des proies faciles, c’était elle qui avait découvert le corps en rentrant du collège. Elle avait passé une bonne demi-heure à essayer de ranimer sa mère, avant le retour de son père qui n’avait eu d’autre choix que de constater le décès. Suite à cette tragédie, le comportement de Jaya, brillante et pleine de vie jusqu’alors, avait commencé à changer, et ses résultats scolaires à piquer du nez. Il aurait pu s’agir d’un épisode transitoire, somme toute bien compréhensible dans le contexte, mais la situation, loin de s’améliorer, n’avait fait qu’empirer, les tentatives pour enrayer la chute n’ayant donné strictement aucun résultat.

À dix-sept ans, elle avait fait une tentative de suicide peu convaincante pour alerter sur la détresse psychologique qui ravageait son existence, telle une meute de rats rongeant peu à peu sa santé mentale. Zaahid avait pris la mesure de la chose et s’était efforcé d’agir en conséquence, ne faisant qu’aggraver la situation au fur et à mesure de ses initiatives, toutes plus malheureuses les unes que les autres.

L’année suivante, Jaya était en totale perdition, rupture scolaire et avec la société en général, et le conflit avec son père avait atteint sinon un point de non-retour, au moins de difficulté majeure à espérer s’en sortir un jour. Autant dire qu’ils ne se parlaient quasiment plus, les rares tentatives dans ce sens aboutissant inévitablement à un échange de hurlements laissant craindre le pire. Zaahid n’était pas d’une fidélité à toute épreuve, c’est vrai, il aimait le sexe, à deux ou à plusieurs, surtout avec des professionnelles, et il lui arrivait parfois de faire preuve d’un caractère difficile, sans parler d’accès de violence qu’il peinait à contrôler en dépit de sa consommation élevée de cannabis, mais de là à lui reprocher la mort de sa femme, il y avait un pas que Jaya ne s’était pas gênée pour franchir allègrement.

Aujourd’hui âgée de vingt-deux ans, méconnaissable, Jaya multipliait les addictions et n’hésitait pas à se prostituer pour satisfaire ses besoins. Elle avait défoncé toutes les barrières, arraché toutes les digues, pulvérisé tous les cadres. C’est dans ces conditions abominables qu’elle avait croisé la route d’un certain Simon Keskula, fondateur de l’Alliance de la Révélation, secte post-apocalyptique établie au fin fond des Alpes de Haute-Provence, au pied de la Montagne de Lure. Keskula, depuis longtemps dans le collimateur de la MIVILUDES\nf{Mission interministérielle de vigilance et de lutte contre les dérives sectaires, créée par décret en mai 2002 auprès du Premier ministre. Elle observe et analyse les mouvements à caractère sectaire et conseille les pouvoirs publics et les particuliers. \source{fr.wikipedia.org/wiki/Miviludes}}, n’en était pas à son coup d’essai, puisqu’on lui devait déjà la création du Monster Gang (une organisation genre X-Men censée regrouper des gens aux capacités hors du commun) et de l’Ordre de la Lune Noire (structure opaque en relation avec des groupuscules d’extrême droite), tous deux de sinistre mémoire et frappés d’interdiction par les autorités compétentes. Lesquelles, jusqu’à présent, se cassaient les dents sur l’Alliance de la Révélation, qui semblait bénéficier de la protection de certaines personnalités influentes.

Mais ceci est une autre histoire, aussi tragique que passionnante, sur laquelle je ne manquerai pas de revenir ultérieurement.

La BM a emprunté une voie privée qui conduisait à l’entrée de la propriété.

J’ai continué mon chemin comme si de rien n’était, avant de faire demi-tour et revenir me positionner dans les parages, hors de vue de Riqueti et sa bande de malandrins qui auraient pu être tentés d’observer les alentours à la jumelle.

L’entrée, en plus de ses grilles de quatre mètres de haut, étant équipée de caméras de surveillance interdisant formellement de se livrer à toute forme d’excentricité inconsidérée, j’ai jugé plus intelligent de faire le tour et trouver une brèche pour m’introduire discrètement dans le périmètre. Il fallait, pour atteindre le mur d’enceinte, s’aventurer dans une jungle d’orties et arbustes épineux qui n’avait pas grand-chose à envier, question impénétrabilité, à la forêt de Bwindi, en Ouganda\nf{La forêt impénétrable de Bwindi, en Ouganda, est inscrite au patrimoine mondial de l'Unesco depuis 1994. Elle abrite environ la moitié de la population mondiale de gorilles des montagnes (\textit{Gorilla beringei beringei}), soit environ 400 individus sur les 700 recensés à l'état sauvage. \source{fr.wikipedia.org/wiki/For\%C3\%AAt\_imp\%C3\%A9n\%C3\%A9trable\_de\_Bwindi}}, où vit la plus grande population de gorilles des montagnes encore en activité, espèce à laquelle appartenait vraisemblablement le nouveau chauffeur (lui-même potentiellement en voie de disparition) de Riqueti. Ce dernier (Riqueti, pas son chauffeur que je n’avais pas encore le plaisir de connaître, et je vous avouerai que je n’étais spécialement pressé de combler ce manque) avait clairement évoqué l’existence d’une organisation pédocriminelle de grande ampleur, connue sur OnionLand sous le nom de League of the Unknown Ribbon (la LUR, ce qui ne veut pas dire grand-chose car on ne voit pas très bien à quoi le ruban inconnu en question fait allusion, à moins que ce ne soit celui entourant le paquet-cadeau d’enfants innocents livrés à la barbarie de leurs tortionnaires) à laquelle les pères Vidal et Beaubois auraient appartenu. J’avais aussitôt mis mon pote Maël Robineau, expert en cybercriminalité, pirate informatique à ses heures et justicier à la petite semaine, sur le coup. Il avait épluché l’Oignon (le dark web, NDLR), notamment la nuit (lorsque sa femme et ses enfants dormaient à l’étage), dans le sous-sol de sa maison transformé en base secrète et centre de recherches ultrasensibles équipé de tout le matériel de pointe en matière de sécurité informatique et sécurité tout court (légèrement paranoïaque sur les bords, il avait fait installer une porte blindée avec serrure biométrique à empreinte digitale, du genre de celles qu’on trouve à Fort Knox ou à la banque de France), mais ses investigations n’avaient rien produit de réellement concluant. Je ne doutais pas de la réalité de cette organisation, sur laquelle j’aurais peut-être l’occasion de me pencher un jour ou l’autre, mais j’étais persuadé que Riqueti me dissimulait des choses de toute première importance. Faisait-il lui-même partie de cette organisation ? J’avais tendance à penser que non, sans quoi il n’aurait pas pris le risque de m’en parler, ou alors juste pour le plaisir de me coller un contrat sur la tête, ce qui n’était pas impossible de la part d’un esprit tordu comme le sien.

Quoi qu’il en soit, j’étais bien décidé à lui tirer les vers du nez une bonne fois pour toutes, voire carrément lui couper le nez en question au ras des joues avec un couteau rouillé et le lui enfoncer bien profond dans la rondelle, ne serait-ce que pour l’amener à réfléchir sur ce qu’il avait fait de sa vie et s’il avait des raisons d’en être fier.

Après une bonne demi-heure de recherches, à me démener dans la végétation en regrettant à chaque instant de n’avoir pas pensé à emporter une machette, j’ai fini par dénicher une vieille porte en bois qui n’avait pas dû être utilisée depuis l’attribution du prix Nobel de philosophie à Jean Baptise Pourri pour ses travaux sur la structure discontinue de la connerie. Elle était tellement vermoulue qu’elle ne tenait plus que par l’opération du Saint-Esprit, lequel (car immenses sont ses pouvoirs) se manifestait en l’occurrence sous la forme d’un exosquelette de lierre dont la chose (porte) était entièrement recouverte. J’ai entrepris d’arracher courageusement de grosses touffes du lierre en question, ce qui m’a pris un certain temps et permis de m’esquinter gentiment les mains, puis j’ai appuyé de tout mon poids sur la porte qui s’est affaissée en faisant un bruit bizarre évoquant furieusement celui que produit un éléphant de mer en glissant sur une plage de Californie.

À ce stade, je me suis demandé pourquoi je m’amusais à passer par derrière alors qu’il m’aurait suffi de passer par la porte principale en présentant ma carte de flic. Je me suis demandé : mais quelle est donc cette force obscure qui te pousse à prendre des chemins de traverse, emprunter des sentiers sinueux et semés d’embûches alors qu’il te suffirait de passer par l’entrée principale, celle que tous les gens normaux empruntent le plus naturellement du monde ?

N’ayant pas trouvé de réponse satisfaisante à cette question, sinon que je voulais bénéficier d’un effet de surprise incompatible avec une pratique standard, j’ai continué ma route.

J’étais à présent dans un parc, du genre de ceux que les gens riches aiment bien avoir chez eux pour frimer devant les invités et faire la bamboche au clair de lune, et j’ai profité de l’occasion pour m’assoir sur un banc qui tendait ses petites planches moussues à mon fessier douloureux (oui, je ne vous l’ai pas dit et aurais préféré ne pas avoir le faire pour éviter de passer pour un manche, mais j’avais glissé sur une pierre, étais parti en vol plané et m’étais rétamé lourdement sur le cul). Ledit banc étant invisible de la propriété, j’avais tout le loisir de m’allumer un petit cigare et savourer quelques bouffées avant de passer à la phase deux de mon opération commando. J’avais aussi tout le loisir d’appeler Maël et Titus pour leur demander de me rejoindre au plus vite, car j’avais maintenant la très nette et désagréable impression que mon entreprise, plus que téméraire, flirtait avec le suicidaire. Autrement dit j’étais venu, avec pour seul et unique partenaire Manu, un 6.35 à sept coups auquel j’attachais une grande valeur sentimentale mais dont l’efficacité létale était pour le moins discutable, me jeter dans la gueule du loup. Manu, un Le Français dans sa version de base «~modèle de poche~», avait appartenu à mon grand-père Philibert, communiste et résistant de la première heure, qui s’en était servi à plusieurs reprises pour faire sauter le caisson à des traîtres ou des nazis (eh oui, il n’avait pas toujours été le vieillard paisible et amateur de pêche à la mouche qu’il était en fin de carrière). Je le voyais souvent chez lui étant enfant, dans la vitrine où il reposait, et je passais des heures à le contempler. Un jour, mon grand-père m’a raconté l’histoire de cet objet, permis de le toucher, le prendre en main, expliqué son fonctionnement, et finalement autorisé à faire mes premiers cartons dans la cour. Je devais avoir une douzaine d’années, et n’étais alors qu’un bambin innocent aux pupilles dilatées par la curiosité. Plus tard, peu de temps avant sa mort, il me l’a officiellement légué, évènement qui figure encore aujourd’hui parmi les plus marquants de mon existence. Naturellement, pour les missions à haut risque nécessitant une puissance de feu supérieure, je disposais de mon arme de service, bien sûr, laquelle était malheureusement restée à la maison, au même titre que mon Desert Eagle 44 magnum dont je commençais à regretter amèrement la présence.

Le banc était confortable, aussi doux à mon coccyx qu’un tapis de mousse fraîche et parfumée dans laquelle s’ébrouent de minuscules insectes aux couleurs vives. C’est donc en toute décontraction que j’ai allumé un Gurkha Ghost Shadow, cigare dominicain emballé dans une feuille d’Arapiraca brésilien sombre et huileuse, et pris le temps de me détendre un peu avant d’appeler Maël. La nuit commençait à tomber, lentement mais sûrement, sur mes frêles épaules de justicier masqué de l’ombre qui erre le balai à la main, tel un super-technicien de surface, dans les ruelles sombres et humides de la cité tentaculaire pour faire régner un semblant de propreté dans un monde voué à la crasse et la putréfaction. Supertechno à Goddam City, jeu de mots foireux pour un monde foireux. Même si cette fois il ne s’agissait pas d’une ruelle sombre et humide, infestée de rats et cafards comme des poissons dans l’eau parmi ordures et immondices, mais d’un repaire perdu au fond des bois dans lequel les Forces du Mal semblaient bien avoir établi leur QG.

J’ai dit : Maël ?

\textsc{Maël} : Oui.

\textsc{Moi} : Salut, c’est Djef.

\textsc{Lui} : Salut, Djef.

\textsc{Moi} : Je suis chez Riqueti.

\textsc{Lui} : C’est cool.

\textsc{Moi} : Pas trop, non. L’endroit est une vraie forteresse. J’ai dû passer par derrière pour m’introduire.

\textsc{Lui} : Qu’est-ce que je peux faire pour toi ?

\textsc{Moi} : T’es dispo, là ?

\textsc{Lui} : Négatif. Ma fille a quarante de fièvre et ma femme n’est pas là. Pourquoi ?

\textsc{Moi} : Eh ben, à dire la vérité, je me sens un peu seul, ici. Je me suis dit que tu pourrais venir me tenir compagnie.

\textsc{Lui} : Désolé, vieux, mais ça ne va pas être possible. Barre-toi si tu sens que c’est trop dangereux, on y retournera en force une autre fois.

\textsc{Moi} : Laisse tomber, je vais me démerder.

\textsc{Lui} : T’as essayé Greg ?

Pour info, Grégoire (Lussier) faisait aussi partie de notre joyeuse bande de justiciers d’opérette. Diplômé de l’Essec, analyste M\&A chez Reckless \& Knot (banque surnommée «~La Machine à Laver~» parce que les affaires qu’elle traitait n’entraient pas toujours dans le cadre de la plus stricte légalité), il avait, le jour où cette dernière s’était retrouvée impliquée jusqu’aux ouïes dans ce qu’il est convenu d’appeler un scandale financier de grande ampleur, pris conscience qu’il faisait un boulot de merde dans un monde de merde et décidé de tout plaquer pour voler au secours de la veuve et l’orphelin, surtout quand la veuve était jeune et jolie, riche de préférence, et l’orphelin héritier d’une immense fortune comme Bruce Wayne, Largo Winch ou Steve Jobs. Profondément humaniste, il avait soudain compris qu’il ne pouvait pas continuer à se goinfrer éhontément pendant que d’autres crevaient la dalle et se faisaient piétiner par des gens comme lui, des requins de la finance qui nageaient en cercle autour des navires en perdition, attendant qu’ils sombrent pour dévorer leurs occupants. Il avait donc, après avoir été licencié avec de copieuses indemnités, ouvert un cabinet de détectives privés, lequel lui servait également de couverture pour des opérations plus radicales. Privé est un métier à risque, on ne le répétera jamais assez. La preuve en est qu’il se trouvait présentement dans une chambre d’hôpital, après avoir été blessé par balle dans l’exercice de ses fonctions.

\textsc{Moi} : Il est à l’hosto.

\textsc{Maël} : Ah bon ? Comment se fait-il que je ne sois au courant de rien ?

\textsc{Moi} : Je viens seulement de l’apprendre, figure-toi. Apparemment c’est arrivé il y a deux jours, pendant qu’il enquêtait sur une affaire d’adultère à priori tout ce qu’il y a de banal. Sauf que l’amant n’est pas du genre commode et lui a tiré dessus alors qu’il allait procéder au constat.

\textsc{Lui} : C’est grave ?

\textsc{Moi} : Il a été touché au bras, rien de grave à priori. Greg savait que le type avait un casier long comme le bras, qu’il trafiquait avec des gens peu recommandables et avait l’habitude de se trimballer avec un calibre sur lui. Il s’était équipé en conséquence et a finalement réussi à flinguer ce connard après une course poursuite dans les escaliers.

\textsc{Lui} : Il est mort ?

\textsc{Moi} : Non, pas tout à fait, il s’en tire avec un genou en vrac et la rate explosée. Les flics attendent qu’il se rétablisse pour lui coller une inculpation de tentative de meurtre sur le dos.

\textsc{Lui} : Ben merde, alors !

\textsc{Moi} : Comme tu dis. Le client, un bijoutier plein aux as de la rue Edgar Morel, est ravi, et Greg va empocher une rallonge substantielle en remerciement de ses bons et loyaux services. Il paraît même que le gars lui aurait demandé de buter aussi sa femme dans la foulée, en faisant passer ça pour un accident. Greg lui a répondu sèchement que le métier de tueur à gage exigeait des compétentes qu’il n’avait pas, et qu’il préférait faire comme s’il n’avait rien entendu.

\textsc{Maël} : Le fait est qu’il n’a jamais tué personne pour de l’argent.

\textsc{Moi} : C’est exact.

\textsc{Lui} : Et Sam ?

\textsc{Moi} : Il est en vacances à Zanzibar, tous frais payés dans une villa 5 étoiles avec vue sur mer, piscine, plage privée, lit king size, clim, minibar, coffre-fort et jardin luxuriant. J’ai cru comprendre qu’il a rendu quelques petits services à un homme d’affaires tanzanien qui a su se montrer particulièrement reconnaissant, preuve que tes frères africains ne sont pas tous des sauvages qui vénèrent des idoles païennes et coupe les couilles des albinos pour s’en faire des grigris.

(NB : je reviendrai plus tard~-- ou pas, suivant mon humeur et les besoins de l’action~-- sur le cas du capitaine Samuel Girard, appartenant lui aussi à notre joyeuse petite bande de justiciers d’opérette, pour la plupart dotés d’une moralité douteuse, certes, mais terriblement humains et sympathiques, même si parfois amenés à commettre des actes plus ou moins répréhensibles aux yeux d’une justice hélas trop souvent aveugle, molle du genou, ou tout simplement dépassée par les événements, tant les affaires à traiter sont nombreuses et les réductions de personnel incompatibles avec le serein exercice d’une profession toujours plus corsetée et exigeante tant sur le plan ethnique qu’éthique et tactique.)

\textsc{Lui} : Il y en a qui ont de la chance.

\textsc{Moi} : Tu l’as dit, bouffi. C’est pas des honnêtes travailleurs comme toi et moi qui auront un jour les moyens d’aller se dorer la pilule sous les Tropiques ! Et pourtant c’est pas faute de se décarcasser, tu seras d’accord avec moi.

\textsc{Maël} : Ouais. Reste plus que Titus, dans ce cas.

\textsc{Moi} : Ouais, je vais l’appeler, mais il va encore se faire sonner les cloches par sa femme.

\textsc{Maël} : Je te souhaite bonne chance. Cela dit, tu devrais pouvoir t’en sortir sans trop de difficulté avec le logiciel que je t’ai installé.

\textsc{Moi} : J’y compte bien.

J’ai raccroché et tiré quelques bouffées de cigare en regardant la nuit dégouliner sur le paysage comme du chocolat fondu sur une glace à la vanille (la comparaison est osée et passablement vide de sens, mais bon, je me suis dit que ça mettrait un peu de poésie pâtissière dans un monde aussi sec qu’une vieille tranche de jambon racornie sur le bord d’une assiette ébréchée).

Puis j’ai appelé Titus, sachant pertinemment que j’allais encore déchaîner la foudre dans le ciel sans nuage (ou presque) de sa vie de famille exemplaire.

J’ai dit : Titus ?

\textsc{Titus} : Oui.

\textsc{Moi} : Salut, c’est Djef.

\textsc{Lui} : Salut, Djef.

\textsc{Moi} : Je suis chez Riqueti.

\textsc{Lui} : C’est cool.

\textsc{Moi} : Pas trop, non. L’endroit est une vraie forteresse. J’ai dû passer par derrière pour m’introduire.

\textsc{Lui} : Qu’est-ce que je peux faire pour toi ?

\textsc{Moi} : T’es dispo, là ?

\textsc{Lui} : J’allais allumer le barbecue.

\textsc{Moi} : Laisse tomber le barbecue, vieux, je suis dans la merde.

\textsc{Lui} : T’as pensé à ma femme ?

\textsc{Moi} : J’y pense jour et nuit.

\textsc{Lui} : Et mes gosses, tu y as pensé à mes gosses ?

\textsc{Moi} : Et comment, que j’y ai pensé ! Je n’y penserais pas davantage s’il s’agissait de mes propres enfants. Cela dit, pour une fois, ça ne leur fera pas de mal de bouffer autre chose que des saucisses et des merguez.

\textsc{Lui} : J’ai prévu des steaks hachés et du boudin créole.

\textsc{Moi} : Tu sais quoi ?

\textsc{Lui} : Non. Quoi ?

\textsc{Moi} : Je me suis pointé ici ne pensant que ça allait être une partie de plaisir, mais je me rends compte que ça risque d’être un peu plus compliqué que prévu. J’ai vraiment besoin de toi.

\textsc{Lui} : Bérénice va être furax.

\textsc{Moi} : Elle comprendra si tu lui dis que c’est une question de vie ou de mort.

\textsc{Lui} : Mes gosses vont être furax.

\textsc{Moi} : Tu sais quoi ?

\textsc{Lui} : Non. Quoi ?

\textsc{Moi} : Tu allumes le barbecue et tu viens me rejoindre après. Bérénice n’aura plus qu’à faire cuire la barbaque, c’est quand même pas sorcier de poser trois saucisses sur une grille.

\textsc{Lui} : Je te trouve bien arrogant.

\textsc{Moi}, estimant que le terme «~arrogant~» n’était peut-être pas le plus approprié mais jugeant préférable de n’en pas faire état : Tu viens ou tu viens pas ?

\textsc{Lui}, après un temps d’hésitation, sans doute dû au fait qu’il avait d’abord songé à m’insulter copieusement avant de se raviser : Comme si j’avais le choix.

\textsc{Moi} : Non, en effet.

\textsc{Lui} : Quoi, non ?

\textsc{Moi} : Non, en effet, tu n’as pas le choix.

\textsc{Lui} : Bien sûr que non, que je ne l’ai pas.

\textsc{Moi} : Ben non.

\textsc{Lui}, bougon : Quand même, tu pourrais prévenir avant.

\textsc{Moi} : Je ne savais pas que j’allais venir.

\textsc{Lui} : Ben voyons.

\textsc{Moi} : Si, je t’assure. J’effectuais une petite filature de routine, pépère, juste comme ça au cas où, et je m’apprêtais à remballer quand j’ai vu Riqueti monter dans la bagnole. Je pensais même passer boire l’apéro chez toi, figure-toi.

\textsc{Lui} : Ah bon ?

\textsc{Moi} : Oui, j’avoue que l’idée m’a traversé l’esprit.

\textsc{Lui} : Ben t’aurais pu me prévenir.

\textsc{Moi} : J’allais le faire, figure-toi. Sauf que quand j’ai vu Riqueti monter dans la bagnole, mon sang n’a fait qu’un tour et j’ai décidé de le suivre.

\textsc{Lui} : T’as des idées bizarres, parfois.

\textsc{Moi}, tirant sur mon cigare les fesses au frais dans la mousse : Ouais, je sais.

\textsc{Lui}, retrouvant un semblant de calme après la bourrasque d’énervement qui venait de le traverser (Titus avait bien des défauts, comme celui de s’emporter facilement, mais pas celui d’être rancunier, ce qui signifie qu’il pouvait vous hacher menu sous le coup de l’émotion, puis le regretter aussitôt et passer le restant de son existence à recoller les morceaux à la pince à épiler) : Et t’es où, au juste ?

\textsc{Moi} : Au château de Marmont. Tu vois où c’est ?

\textsc{Lui} : Non.

\textsc{Moi} : Tu vois Draguillon ?

\textsc{Lui} : Vaguement, oui.

\textsc{Moi} : T’as un GPS, non ?

\textsc{Lui} : Ouais, bien sûr.

\textsc{Moi} : Eh ben t’as qu’à le mettre.

\textsc{Lui} : Et je tape quoi ?

\textsc{Moi} : Je viens de te le dire : château de Marmont. C’est pas vraiment un château, du reste. Plutôt une grosse baraque bourgeoise en forme de château.

\textsc{Lui} : Ouais. Un château, quoi.

\textsc{Moi} : Si on veut.

\textsc{Lui} : C’est loin ?

\textsc{Moi} : Non, tout près de Draguillon.

\textsc{Lui} : Et je te retrouve où ?

\textsc{Moi} : Appelle-moi quand t’arrives, je te donnerai la marche à suivre.

\textsc{Lui} : Okay, je me mets en route. Essaie de te tenir à carreau en attendant.

\textsc{Moi} : Je vais faire le maximum.

Il est allé voir sa femme qui s’affairait en cuisine à préparer une salade composée à base de riz, tomate, oignon, œuf dur et poivron, s’est approché d’elle aussi doucement que sa stature imposante le lui permettait, puis lui a glissé dans le creux de l’oreille, la voix chargée d’appréhension : Chérie ?

\textsc{Bérénice}, sans lever la tête de son saladier : Oui ?

\textsc{Titus} : Je suis désolé, mais…

\textsc{Elle}, tournant vers lui un visage déjà empourpré par la colère : Quoi encore ?

\textsc{Lui} : Djef vient de m’appeler.

\textsc{Elle} : Et alors ?

\textsc{Lui} : Il est au château de Marmont.

\textsc{Elle} : Il a gagné au loto ?

\textsc{Lui} : Non. C’est juste que…

\textsc{Elle} : Que quoi, mon chéri ?

\textsc{Lui} : Ben… il a des ennuis, à ce que j’ai cru comprendre.

\textsc{Elle}, narquoise : Non ? C’est vrai ?

\textsc{Lui} : Il semblerait, oui.

\textsc{Elle}, d’une voix d’une braise qui couve sous la cendre, prête à s’enflammer au moindre courant d’air : Rappelle-moi, tu es marié avec qui ? Avec Djef ou avec moi ?

\textsc{Lui}, muscles tendus, prêt à prendre la fuite si une fourchette ou un rouleau à pâtisserie se dirigeait vers lui à une vitesse anormalement élevée : Avec toi, chérie, tu le sais bien.

\textsc{Elle} : Et tu préfères passer la soirée qui ? Avec Djef ou tes enfants et moi ?

\textsc{Lui} : Les enfants et toi, chérie, tu le sais bien. C’est juste que Djef se trouve dans une situation assez délicate, et…

\textsc{Elle}, tailladant une pauvre tomate innocente à grands coups de couteau de boucher : Djef est toujours dans une situation délicate, vingt-quatre heures sur vingt-quatre, c’est sa spécialité de se fourrer dans des situations délicates ! Et qui est-ce qu’il appelle ensuite ? Je te le donne en mille ? Toi, toujours toi, et toi tu rappliques à chaque fois qu’il te sonne ! Ne me dis pas qu’il n’a pas d’autres amis que toi, des amis qui ne seraient pas obligés de planter femme et enfants à chaque fois que Djef appelle parce qu’il est dans une situation délicate ! Y en a marre, de Djef ! Il ferait mieux de se trouver une femme et faire des gosses, au lieu de se fourrer dans des situations délicates et faire chier le monde à tout bout de champ pour qu’on l’aide à s’en sortir ! À croire qu’il le fait exprès !

\textsc{Titus}, écaillant un œuf avec ses gros doigts pour bien montrer que lui aussi était capable de s’occuper de tâches ménagères, ces petites choses de la vie quotidienne en apparence insignifiantes, humbles, sinon franchement casse-couille, mais qui constituent en réalité le fertile substrat d’une vie de couple réussie : Oui, c’est vrai, tu as raison. (Règle numéro un : ne jamais contrarier une femme qui tient un couteau de boucher.) Je crois savoir qu’il est inscrit sur des sites de rencontre pour célibataire de moins de cinquante ans.

\textsc{Bérénice} : Tu parles !

\textsc{Titus} : Si, je t’assure. C’est un garçon très sensible, tu le sais aussi bien que moi. L’ennui, c’est qu’il fait peur aux filles avec ses yeux de fou et les trucs bizarres qu’il raconte en permanence.

\textsc{Elle} : En effet, il n’est pas très net.

\textsc{Lui} : Complètement cinglé, tu veux dire ! (Règle numéro deux : faire semblant d’abonder dans son sens.)

\textsc{Elle}, un poil plus détendue : N’empêche qu’il suffit qu’il te siffle pour que tu rappliques ventre à terre !

\textsc{Titus} : Je te trouve un peu sévère, pour le coup. Je te rappelle qu’il m’a déjà sauvé plusieurs fois la mise, et qu’il n’hésiterait pas à le refaire si l’occasion se présentait.

\textsc{Elle} : Justement ! Ce type a un don pour attirer les embrouilles, et je n’ai pas envie que ça se termine mal un jour l’autre.

\textsc{Titus} : Oui, mon amour.

\textsc{Elle} : Te fiche pas de moi !

\textsc{Lui} : Je me fiche pas de toi, je sais très bien que c’est pas facile de vivre avec un flic.

\textsc{Elle} : Surtout un flic qui fait des heures supplémentaires. Non rémunérées, bien entendu.

\textsc{Lui} : Je sais. Mais dans ce métier, on ne va pas loin si on commence à compter ses heures. On n’est pas au théâtre, le crime ne fait jamais relâche.

Bon, je n’étais pas là pour entendre ça, mais je trouve que c’est plutôt pas mal dit pour un type qui n’est pas (objectivement, sans jugement de valeur) ce qui se fait de mieux en matière d’explication de texte et littérature comparée.

C’est à ce moment-là que Virginie, seize ans, qui était dans sa chambre en train de lire le Traité sur la Musique de saint Augustin (elle jouait du piano depuis l’âge de dix ans), a fait son apparition dans la cuisine. Elle avait entendu du bruit et tenait à s’enquérir de la situation.

\textsc{Virginie} : Tout va bien ?

\textsc{Bérénice} : Très bien. C’est juste que ton père va encore nous planter là pour aller faire les quatre cents coups avec son copain Djef.

\textsc{Titus} : Je suis vraiment désolé, mais il s’agit d’un cas de force majeure.

\textsc{Bérénice} : Question de vie ou de mort.

\textsc{Titus} : C’est ça. Je ne devrais sans doute pas vous le dire, mais il se trouve en ce moment-même chez un individu dont on ne connaît pas encore exactement le niveau de dangerosité. Il pensait s’en sortir tout seul, raison pour laquelle il n’avait pas jugé bon de me prévenir, mais il craint des complications et pense que nous ne serons pas trop de deux pour y faire face. C’est mon devoir de lui prêter main forte.

\textsc{Bérénice}, narquoise : Oui, bien sûr.

\textsc{Virginie} : Il est gentil, Djef.

\textsc{Bérénice} : Oui, très. Mais il est aussi très chiant.

\textsc{Titus}, consultant sa montre : Je suis d’accord avec vous sur toute la ligne, les filles : il est très chiant et très gentil, et vous êtes mes deux princesses adorées sans qui la vie n’aurait aucun intérêt pour moi.

\textsc{Bérénice} : Mais tu dois partir tout de suite pour voler au secours de ton ami Djef, c’est ça ?

\textsc{Titus} : C’est ça.

Son téléphone a sonné.

Et devinez qui c’était ?

C’était moi, bien sûr, l’empêcheur de tourner en rond, le briseur de paix dans les ménages : Titus ?

\textsc{Lui} : Quoi encore ?

\textsc{Moi} : T’es où ?

\textsc{Lui} : Chez moi.

\textsc{Moi} : T’es pas encore parti ?

\textsc{Lui} : Si, presque.

\textsc{Moi} : Qu’est-ce que tu fous, bordel ?

\textsc{Lui} : J’ai une femme et des enfants, au cas où tu ne l’aurais pas remarqué.

\textsc{Moi} : Contrairement à moi qui suis libre comme l’air, je sais, merci. Désolé de te déranger en plein conseil de famille, mais il se passe des choses, ici.

\textsc{Lui} : Ah bon ? Rien de grave, j’espère.

\textsc{Moi} : Hélas non. Mauvaise nouvelle pour vous : je suis encore en vie. Par contre, Riqueti et son chauffeur viennent de s’absenter. Je me suis dit qu’on pourrait en profiter pour fouiller la baraque.

\textsc{Lui} : C’est une bonne idée.

\textsc{Moi} : Merci. Fais vite, je t’attends.

\textsc{Lui} : Je suis là dans une demi-heure.

\textsc{Moi} : Okay. Je vais explorer un peu les alentours en attendant.

J’ai raccroché et commencé à fureter dans le secteur.

Le parc était cossu, les arbres centenaires, la piscine olympique, les dépendances nombreuses. Il y avait même une petite chapelle dans laquelle Monseigneur devait donner des messes privées pour les notables du coin. Des messes noires, peut-être, durant lesquelles, coiffé de sa plus belle mitre, il sacrifiait des nourrissons, chiait sur des rondelles de pain azyme, pissait dans des calices, buvait du sang humain et s’accouplait avec des animaux, le tout pour s’attirer les bonnes grâces de Lucifer et accéder à l’immortalité. Peut-être même que c’était à cet endroit précis que se tenaient les sessions extraordinaires de la League of the Unknown Ribbon, sinistre association de malfaiteurs dont les ramifications souterraines s’étendaient bien au-delà de l’Hexagone.

Ladite chapelle était fermée à clé.

Qu’à cela ne tienne, je disposais de tout le matériel nécessaire pour venir à bout de la serrure.

Quelques instants plus tard, donc, j’étais à l’intérieur de la chapelle. J’en ai fait rapidement le tour sans rien déceler de suspect. Je m’attendais au moins à trouver, je ne sais pas, moi, des crucifix à l’envers ou une statue de Lucifer avec des pieds de bouc, des ailes de chauve-souris et un braquemart comac poli par des années de dévotion tant orale que manuelle. Mais rien de tout ça, pas même un petit bocal avec un fœtus à l’intérieur ou une bible satanique écrite avec du sang de lépreux sur de la peau humaine rongée par la petite vérole. Pas la moindre bouteille de vin de messe remplie de pisse, effigie de la Vierge avec des nichons énormes, des lunettes de soleil et une carotte dans le cul. Je dois dire que j’attendais un peu mieux de Riqueti et sa bande de dégénérés de la LUR. Même si, pour être tout à fait exact, je ne disposais pour l’instant d’aucune preuve tangible de son appartenance à une telle organisation, raison pour laquelle (découvrir de telles preuves) je me trouvais ici.

Trois quarts d’heure environ après notre dernier échange téléphonique, j’ai reçu un appel de Titus me signalant qu’il venait d’arriver. Il avait repéré ma bagnole et s’était garé à côté.

Je lui ai dit que j’étais passé par derrière pour rentrer, mais que le chemin était un peu compliqué, pour ne pas dire digne d’un parcours du combattant dans une jungle infestée de serpents et truffée de mines antipersonnel, et que par conséquent, vu que les locataires étaient partis, j’allais le faire passer par la porte principale, ce qui nous permettrait de gagner du temps et lui éviterait de revenir chez lui en lambeaux et de se faire tabasser par sa chère et tendre. Grâce à mon passe de compète, l’opération ne prendrait que quelques secondes.

Ou minutes, peut-être, parce que la serrure s’est révélée un peu plus coriace que prévu.

J’étais en train de m’échiner dessus quand mon téléphone a sonné. J’ai laissé sonner.

J’ai fini par triompher de la serrure et laisser entrer un Titus qui commençait manifestement à trouver le temps long, d’autant qu’il avait promis à sa femme de faire aussi vite que possible. À ce rythme-là, il n’était pas près de retrouver la chaleur du foyer que je l’avais forcé à abandonner.

J’ai jeté un coup d’œil à mon téléphone : un O6 inconnu avait laissé une trace sur ma messagerie. La tentation de l’écouter était trop forte. C’était Zarina Brizzi, la fille que j’avais rencontré la veille au Narcisse Rose, et à qui j’avais, dans un moment d’égarement, commis l’erreur de refiler mon numéro de téléphone. Elle n’avait pas perdu de temps. Elle me disait qu’elle et sa sœur avaient passé une excellente soirée en notre compagnie (j’étais là-bas avec Zaahid Shirani), qu’elle-même, à titre personnel, avait particulièrement apprécié ma conversation, et qu’elle espérait qu’on pourrait se revoir dans les plus brefs délais. C’était d’autant plus étrange que je n’avais pas dû prononcer plus de dix ou quinze mots dans la soirée, et n’avais rien fait, à mon sens tout cas, pour susciter un tel engouement. Il faut croire que mon charme dévastateur s’exerçait à mon corps défendant. Cela dit, j’aurais mauvaise grâce de prétendre que Zarina, sa beauté plastique indéniable, son esprit d’une vivacité peu commune et son accent italien aussi doux et enveloppant qu’un manteau en plumes de marabout Yves Saint Laurent\nf{Yves Saint Laurent (1936--2008), couturier français né à Oran. Directeur artistique de Dior à 21 ans, il fonda sa propre maison en 1961. Il est notamment crédité d'avoir introduit le smoking féminin (1966) et popularisé le prêt-à-porter de luxe. \source{fr.wikipedia.org/wiki/Yves\_Saint\_Laurent\_(couturier)}} (élevé dans le respect du bien-être animal et des cycles biologiques, je parle du marabout), me laissaient de marbre.

\textsc{Titus} : Qui c’est ?

\textsc{Moi} : Une fille que j’ai rencontrée hier soir au Narcisse Rose.

\textsc{Lui} : Ah ah, les affaires reprennent, on dirait.

\textsc{Moi} : Ne dis pas de conneries !

\textsc{Lui} : Elle s’appelle comment, si ce n’est indiscret ?

\textsc{Moi} : Zarina.

\textsc{Lui} : C’est joli, Zarina.

\textsc{Moi} : Oui, bon, on s’en fout. Amène-toi, on a déjà perdu assez de temps.

\textsc{Lui} : Quand est-ce que tu me la présentes ?

\textsc{Moi} : T’as fini, oui !

\textsc{Lui} : Excuse-moi d’être content pour toi. En plus, je pense qu’une femme ne serait pas trop pour remettre un peu d’ordre dans ton existence.

\textsc{Moi} : On a échangé trois mots autour d’un verre, c’est peut-être un peu tôt pour publier les bans.

\textsc{Lui} : N’empêche que t’as l’air drôlement mordu.

\textsc{Moi} : Je ne vois pas ce qui te fait dire ça.

\textsc{Lui} : Je ne sais pas, la façon que t’as d’en parler avec des étoiles plein les yeux.

\textsc{Moi} : Tu te fous de moi ?

\textsc{Lui} : Non, c’est vrai, t’as des étoiles plein les yeux.

\textsc{Moi}, le fusillant du regard : Laisse tomber, tu veux.

\textsc{Lui}, pénible : Tu vas la revoir bientôt ?

\textsc{Moi} : Écoute, vieux, je t’aime bien, t’es comme un frère pour moi, mais si tu continues comme ça, c’est toi qui vas en voir, des étoiles !

\textsc{Lui} : Okay, ça va, pas la peine de s’énerver. On fait comment, alors ?

\textsc{Moi} : C’est pas compliqué : tu montes la garde pendant que je visite la maison.

\textsc{Lui} : Et s’ils reviennent ?

\textsc{Moi} : Tu me fais signe et on se barre en toute discrétion.

\textsc{Lui} : En toute discrétion ?

\textsc{Moi} : Oui, en toute discrétion. T’as pas oublié ton flingue, j’espère ?

\textsc{Lui}, sortant son Glock 17 et me l’agitant sous le nez : Tu me prends pour qui, pépère ? Bien sûr que non, que je l’ai pas oublié !

\textsc{Moi} : Parfait.

\textsc{Lui} : Sauf que je ne pense pas en avoir besoin si on se barre en toute discrétion, comme tu dis.

\textsc{Moi} : On ne sait jamais.

J’ai fait le tour de la bicoque, Titus sur mes talons, et rapidement trouvé ce que je cherchais : le moyen de m’introduire en toute illégalité dans un domicile qui n’était pas le mien. Ce moyen était une porte, et on y accédait en descendant quelques marches d’escalier, tellement étroites et abruptes qu’on était, à moins de chausser du 32, ce qui correspond peu ou prou à la pointure d’un gamin de six ans (pour info, je chaussais du 44 et Titus un bon 50), obligé de se déplacer en crabe, au risque de se niquer la cheville et se foutre la gueule par terre à tout moment. En d’autres termes, ladite porte était clairement celle de la cave. Celle-ci avait quand même la particularité d’être blindée, preuve que la cave en question devait recéler des trésors liquides avec lesquels le propriétaire des lieux n’avait aucune envie des mécréants se rincent la glotte. Monseigneur était fin gourmet, aimait arroser ses plats des meilleurs crus. Pas de problème, j’étais moi-même féru de gastronomie et de vins fins et veloutés. On dit souvent que c’est autour d’une bonne table que les grandes causes se négocient. Et j’ai envie de m’écrier : mais oui, bien sûr, c’est tellement vrai ! Si Riqueti, au lieu de m’envoyer ses tueurs, m’avait fait les honneurs de sa cave, je vous fiche mon billet que les hommes de goût que nous sommes auraient su se montrer suffisamment adultes pour balayer leurs dissensions d’un revers de la patte. Alors oui, c’est vrai, je n’étais pas venu pour ça (comme souvent, d’ailleurs, je ne savais pas très bien pourquoi j’étais venu, ce que je faisais là, il s’agissait tout au plus d’une vague idée, le pressentiment qu’une tragédie d’ampleur hellénistique était en train de se nouer dans l’ombre propice de quelque profond caveau), mais je vous prie de croire que si d’aventure tel ou tel précieux flacon s’avisait de me faire de l’œil, je n’allais certainement pas passer à côté en faisant comme si je n’avais rien vu. Merde, ce n’est pas parce que certains prennent les cochons de lait pour des sangliers albinos (croyez-le ou non, mais le cas s’est vu en forêt de Beaumont-le-Roger, dans l’Eure, bien triste histoire sur laquelle je n’aurais malheureusement pas le plaisir de m’appesantir davantage, étant donné que le présent ouvrage ne traite ni d’élevage ni de chasse) qu’il faut se laisser marcher sur les sabots ! En un mot comme en cent, quatre en l’occurrence : l’occasion fait le lardon.

J’ai sorti mon matos de cambrioleur, avec Titus dans mon dos qui commençait à manifester certains signes de nervosité (en dépit de sa stature de grand singe, et vous me connaissez assez pour savoir que je ne dis pas ça parce qu’il s’agit d’une personne de couleur, c’était quelqu’un d’assez nerveux, qui prenait sur lui en permanence pour ne pas tout exploser sur son passage, comme cette porte, par exemple, qu’il se serait fait une joie de défoncer à coups de poings si je n’avais pas été là pour le ramener à la raison), et aussitôt attaqué à la serrure avec tout le doigté dont j’étais capable, c’est-à-dire énormément.

Elle était coriace, mais moi aussi.

Après dix bonnes minutes de tripatouillage intensif, elle a rendu les armes.

Après la porte, l’escalier, toujours aussi étroit et d’autant plus casse-gueule que s’y ajoutait une bonne dose d’humidité supplémentaire, s’enfonçait dans les profondeurs de la terre. Par chance, un interrupteur situé en haut à droite permettait de faire toute la lumière sur la situation. En tant que chef des opérations, c’était à moi que revenait le privilège de passer le premier.

J’ai dit à Titus, qui maugréait derrière moi : Fais gaffe, ça glisse.

\textsc{Lui} : Merci, j’avais pas remarqué.

\textsc{Moi} : Je n’aime pas trop te sentir dans mon dos.

\textsc{Lui} : Je protège tes arrières.

\textsc{Moi} : Mouais. Et si tu glisses, je dégringole avec toi.

\textsc{Lui} : La confiance règne.

Quand vous avez un mec qui chausse du 50 et qui est obligé de se plier en quatre pour descendre un escalier aussi étroit qu’un trou de souris, et que ce mec se trouve dans votre dos suffisamment près pour que vous puissiez sentir son souffle vous rafraîchir la nuque, j’estime qu’on est en droit de se faire un brin de muguet (alternative florale à «~un peu de souci~»).

Environ trois mètres plus bas, comme il ne fallait pas être un génie pour s’en douter, se trouvait une cave qu’on aurait pu aisément transformer en boîte de nuit ou court de tennis. L’espace était occupé par des tonneaux vides, certains étant retournés pour faire office de tables et permettre ainsi la dégustation in situ des crus les plus prometteurs, et les murs couverts de casiers de bouteilles dont la totalité (je parle des bouteilles) devait bien avoisiner, à vue de nez, les deux ou trois mille, soit de quoi plonger en état de sidération tout amateur un peu sérieux.

J’ai dit à Titus, la voix brisée par l’émotion : Je crois que je vais en avoir pour un moment.

\textsc{Lui} : Comment ça ?

\textsc{Moi} : Tu ne crois quand même pas que je vais repartir sans jeter un œil là-dessus.

\textsc{Lui} : Jeter un œil ?

\textsc{Moi} : Oui, et prélever quelques échantillons de valeur que nous aurons le plaisir de partager ou revendre sur Internet pour arrondir nos fins de mois difficiles.

\textsc{Lui} : T’as vraiment aucune moralité !

\textsc{Moi} : Non, mais j’ai soif. Soif de vivre et déguster les meilleurs vins, que ce ne soit pas toujours les mêmes qui en profitent.

\textsc{Lui} : Bérénice adore le Bordeaux.

\textsc{Moi} : Je crois qu’on devrait pouvoir trouver ce qu’il faut. Tu sais quoi ?

\textsc{Lui} : Non ?

\textsc{Moi} : Changement de plan.

\textsc{Lui} : Pourquoi, il y en avait un ?

\textsc{Moi} : Plus ou moins. Tu sais ce que je pense ?

\textsc{Lui} : Non, et j’aimerais autant ne pas le savoir.

\textsc{Moi} : Je pense qu’on se complique bien trop l’existence.

\textsc{Lui} : Ah bon.

Je lui ai tendu les clés de ma caisse : Oui. Tu vas remonter, aller chercher ma caisse et venir te garer carrément ici, le cul à la porte de la cave.

\textsc{Lui} : Je te rappelle qu’on n’est pas chez nous.

\textsc{Moi} : On s’en fout. Il faut savoir se faire plaisir, de temps à autre.

\textsc{Lui} : Et si les autres rappliquent ?

\textsc{Moi}, avec des intonations dans la voix que je ne me connaissais pas, rendu fou par la vue des bouteilles qui scintillaient devant mes yeux exorbités comme autant d’étoiles qu’il suffisait de tendre le bras pour atteindre : On va faire aussi vite que possible.

\textsc{Lui} : D’accord, mais s’ils reviennent et qu’ils nous trouvent en train de piquer leurs bouteilles ?

\textsc{Moi} : On avisera. Pour info, j’ai appris récemment que Riqueti est originaire de Spezzano, le siège du clan Terracciano.

\textsc{Lui} : Connais pas.

\textsc{Moi} : Faudrait voir à te tenir un peu courant, ma vieille. Le clan Terracciano, c’est la mafia calabraise. T’as entendu parler de la mafia calabraise, quand même ?

\textsc{Lui} : Oui, bien sûr.

\textsc{Moi} : Alors tu sais que c’est pas des tendres. Si j’en crois mes informateurs, Riqueti bosse pour eux et se sert de son influence pour renforcer celle de l’organisation.

\textsc{Lui} : Tu crois ?

\textsc{Moi} : J’en suis sûr. Si je te dis Piero Bottaro ?

\textsc{Lui} : Je te réponds inconnu au bataillon.

\textsc{Moi} : Et Santo Termine ?

\textsc{Lui} : Qui ça ?

\textsc{Moi} : Santo Termine. Lui et Bottaro sont des pontes de la ’Ndrangheta\nf{La ’Ndrangheta, organisation criminelle originaire de Calabre (Italie du Sud), est considérée par Europol et l’Office des Nations unies contre la drogue et le crime comme la mafia la plus puissante et la plus internationale du monde. Son chiffre d’affaires annuel est estimé à plus de 50 milliards d’euros, essentiellement issu du trafic de cocaïne. \source{fr.wikipedia.org/wiki/’Ndrangheta}}, elle-même en cheville avec la mafia albanaise. Termine est spécialisé dans le trafic d’œuvres d’art, les paris clandestins et l’immobilier, Bottaro dans la fausse monnaie, les comptes offshore et la fraude aux subventions européennes. J’ai mis Maël sur le coup et j’ai la preuve, photos à l’appui, que Riqueti, Termine et Bottaro se connaissent depuis des lustres. Donc Riqueti est une ordure et on ne va pas se gêner pour lui piquer son pinard !

\textsc{Titus} : Et s’il rapplique avec son garde du corps ?

J’ai peut-être pas l’air, comme ça, mais je suis le genre de type très malin qui a toujours qui a toujours une longueur d’avance sur les évènements, un joker dans le fond de sa poche. Je ne dirais pas que je suis capable de lire l’avenir dans le fond d’un verre de Chambertin, mais pas loin.

\textsc{Moi}, sortant deux chiffons noirs des poches de mon pantalon : S’ils rappliquent, on enfile ça et on leur tombe sur le dos par derrière. On les maîtrise, on les attache dans un coin, et on finit tranquillement ce qu’on a à faire.

\textsc{Lui} : C’est quoi ?

\textsc{Moi}, triomphant d’ingéniosité : Des cagoules, mon vieux ! Avec ça sur la tronche, ils ne sauront jamais qui a fait le coup.

\textsc{Lui} : On aurait peut-être pu les mettre tout de suite.

\textsc{Moi} : Pourquoi faire ?

\textsc{Lui} : Il y a peut-être des caméras de surveillance qui nous filment depuis le début.

\textsc{Moi} : Et tu crois sans doute que je n’y ai pas pensé.

\textsc{Lui} : Je sais pas. Tu y as pensé ?

\textsc{Moi} : Je croyais que tu me connaissais un peu mieux, depuis tout ce temps. Bien sûr, que j’y ai pensé !

J’ai sorti mon téléphone et lui ai mis sous le nez : Tu sais ce que c’est, ça ?

Lui, levant les yeux au ciel : Oui, un téléphone.

\textsc{Moi} : Exact.

J’ai tripoté le téléphone pendant quelques secondes et lui ai à nouveau collé sous le nez : Et ça, tu sais ce que c’est ?

\textsc{Lui}, jetant un œil blasé sur l’objet : Euh…. non, c’est quoi ?

\textsc{Moi} : Le progrès, mon vieux. Avec ce programme dernier cri, je peux scanner les systèmes de surveillance et les désactiver. C’est précisément ce que j’ai fait pendant que tu discutais avec ta femme et prenais tout ton temps pour te ramener ici.

\textsc{Lui}, interloqué : Merde, alors !

\textsc{Moi}, essayant de ne rien laisser paraître de l’autosatisfaction qui m’agitait intérieurement : Ouais. Des caméras, il y en a effectivement un peu partout ici. Mais grâce à ce petit bijou de technologie, j’ai pu localiser le logiciel de surveillance et le pirater avec une facilité déconcertante dès que Riqueti s’est absenté. Même chose pour les alarmes censées prévenir les flics en cas d’intrusion. Autrement dit, on peut se balader dans le périmètre en toute tranquillité. Je ne te cache pas que c’est Maël qui m’a filé le tuyau, installé le programme et indiqué la marche à suivre.

\textsc{Lui} : Impressionnant !

\textsc{Moi} : Ouais, c’est assez fascinant. Plus on essaye de se protéger et plus on s’expose, en quelque sorte. Allez, va chercher la caisse et viens te garer aussi près que possible pour qu’on n’ait pas trop à se fatiguer en déménageant le butin.

Une heure plus tard, le coffre de la Kangoo était plein à craquer de tout ce que j’avais pu trouver de plus rare et prestigieux dans la cave de Riqueti (ne vous inquiétez pas, j’aurai l’occasion d’y revenir).

Je ne vous l’ai peut-être pas dit, mais ma caisse, même si elle n’avait pas grand-chose à voir avec l’Aston DB5 de ce crétin de James Bond ou le Tumbler de cet autre abruti notoire de Bruce Wayne, était quand même équipée de gadgets très utiles pour sauver ses fesses en cas de coup dur. Par exemple, un jeu de plaques minéralogiques rotatives me permettait de dissimuler ma véritable identité en un battement de cils. Ce tour de force s’effectuait grâce à un interrupteur habilement dissimulé parmi les commandes du tableau de bord, et on ne pouvait espérer détecter la supercherie qu’en se livrant à un examen approfondi des plaques en question. C’était Nathan Lussier (un des nombreux frères de Greg), garagiste de son état, féru de nouvelles technologies et de nazi porn (Train spécial pour Hitler, Ilsa la louve des SS, Camp N°7, Le Lac des morts-vivants, autant de chefs-d’œuvre absolus réservés à un public averti), en plus du bel canto (fan de Donizetti, il écoutait Lucia di Lammermoor en boucle en jouant de la pompe à vide et de la clé de 12) et de la chasse à l’arc (c’était son côté comte Zaroff, sauf que lui utilisait un arc à poulies Martin Cougar Vintage MT-6 en fibre de carbone, le même que Stallone dans John Rambo, quatrième volet de la saga avec un John Rambo qui commence sérieusement à piquer du nez dans sa soupe), qui avait procédé à l’installation avec tout le génie dont il était capable. L’optimisation (la plupart du temps, reconnaissons-le, parfaitement illégale) de véhicules ordinaires était sa grande passion. Par exemple, le moteur à essence 1,6 16v de ma fidèle Kangoo développait à l’origine une petite centaine de poneys. C’était bien suffisant pour doubler un poids lourd et effectuer ses livraisons en temps et en heure, mais totalement ridicule pour espérer se livrer à des courses-poursuites délirantes dans des conditions extrêmes telles que centres-villes aux terrasses bondées et trottoirs encombrés de gens en fauteuil roulant et de mères de familles avec des landaus, autoroutes à contresens ou routes de montagne sinueuses bordées de falaises d’un côté et de précipices de l’autre. Eh bien, croyez-le ou non, mais après être passé entre les mains magiques de Nathan, le bloc pouvait maintenant se prévaloir d’un troupeau de cent cinquante mustangs gonflés à bloc qui piaffaient d’impatience d’exploser les chronos. Grâce à Nathan, l’utilitaire léger était devenu une authentique bombe de l’asphalte. Il avait bien sûr fallu procéder aux aménagements nécessaires en termes de fiabilité et de sécurité (je ne plaisante pas avec ça), mais j’avais insisté pour que rien ne se fasse au détriment de la discrétion exemplaire qui avait toujours été ma règle de conduite : pas d’échappement tonitruant, de roues de camion, aileron de requin et autres prises d’air intempestives. Pour tout le monde, j’étais et devais rester le parfait brave type qui roulait sans accroc, respectait scrupuleusement les limitations de vitesse et les panneaux de signalisation, n’oubliait jamais d’attacher sa ceinture, mettre son clignotant et vérifier la pression de ses pneus avant de partir en vacances (chose qu’il ne faisait bien évidemment jamais vu que son compte en banque était dans le rouge dès le 13 du mois).

Histoire de ne pas prendre le risque de se retrouver coincés comme des rats si l’évêque du Sanctuaire de Ddarr et son gorille des montagnes revenaient à l’improviste, j’ai suggéré à Titus d’aller remettre la Kangoo à sa place, pas trop près pour ne pas attirer l’attention des fouineurs éventuels, mais pas trop loin non plus pour qu’on ne soit pas obligés de se taper des kilomètres au pas de charge pour s’exfiltrer de la zone. Mes affinités avec le sport se limitaient à quelques rares disciplines, et la course à pied n’en faisait pas partie.

Telle était la première partie de sa mission, qu’il avait acceptée d’assez mauvaise grâce, estimant sans doute qu’elle n’était pas à la hauteur de ses capacités. D’autant que je lui avais, avec une insistance de nature à taper sur les nerfs du plus flegmatique des interlocuteurs, intimé l’ordre de rouler sur des œufs pour ne pas risquer d’esquinter mon précieux chargement.

La seconde (partie de sa mission), tout aussi périlleuse, était la suivante : rester en haut, trouver un emplacement stratégique, surveiller activement les alentours et me prévenir aussitôt en cas de danger. J’avais une totale confiance en Titus, mais vous savez ce que c’est, avec le stress, on se met facilement à radoter et ergoter sur des sujets mineurs. Et du stress, en dépit de l’excitation qui m’habitait en pensant à la cargaison de ma Kangoo et autres objets de valeur que je ne manquerais pas de dénicher en fouillant la maison (et de voler aussi, oui, je savais que c’était mal, au moins sur le plan de la morale chrétienne, mais les arcanes de la notion de propriété échappaient encore en grande partie à mon entendement, même si je parvenais quand même, sans trop de difficulté, à me mettre dans la peau de celui qui se fait dévaliser et en éprouve un certain ressentiment), j’en étais quand même porteur d’une quantité respectable.

Je m’apprêtais à gagner le rez-de-chaussée quand quelque chose a attiré mon attention. C’était à peine audible, mais, en tendant l’oreille, on pouvait distinguer des bruits bizarres s’apparentant à des gémissements ou une respiration étouffée, un peu comme si quelqu’un était en train de scier du bois dans une pièce adjacente avec une scie en mousse, ou encore de s’échiner à monter une mayonnaise avec une brosse à dents. Je me suis dirigé vers l’endroit d’où semblaient provenir les bruits en question, à savoir un des murs les plus glaireux et répugnants qu’il m’ait été donné de voir (et je vous prie de croire que j’en ai vu, des murs glaireux et répugnants, au cours de ma longue carrière de fonctionnaire véreux), contre lequel, au prix d’un effort titanesque, j’ai néanmoins réussi à coller une oreille. J’ai longtemps hésité entre la droite et la gauche, pour finalement choisir la droite, laquelle m’est apparue sur le moment, pour des raisons que je ne saurais décrire avec précision, comme la plus performante et la moins fragile des deux. Il faut savoir que mes oreilles et moi étions nés le même jour, à la même heure et au même endroit, de sorte que nous formions une équipe très soudée, comme les cinq doigts de la main qui eux-mêmes se connaissent depuis toujours et entretiennent une relation très étroite, quasi fusionnelle, leur permettant d’accomplir des prouesses techniques qui laissent sans voix la plupart des habitants de cette planète, et ne manqueraient certainement pas de clouer le bec à la plupart des touristes extraterrestres de passage sur terre.

Et la réponse est : oui, il y avait quelqu’un, à n’en pas douter.

En y regardant de plus près (faisant fi de l’odeur de marée basse qui s’en dégageait, mêlant poissons morts et vieilles touffes de varech pourri), je me suis rendu compte que ce mur n’en était pas vraiment un, mais plutôt une cloison enduite d’un revêtement lui donnant toutes les apparences du mur qu’il n’était pas.

En m’éclairant avec mon téléphone, j’ai fini par découvrir une protubérance suspecte sur laquelle j’ai décidé d’appuyer, certain qu’il s’agissait d’un mécanisme d’ouverture habilement dissimulé dans le décor.

Bingo ! La cloison a coulissé sur quelques dizaines de centimètres, pas de quoi laisser passer un troupeau de gnous lancés à pleine vitesse avec une meute de lions à ses trousses (plus quelques guépards et une escadrille de vautours en soutien aérien), mais assez pour qu’une personne affichant un tour de taille raisonnable (comme moi, par exemple) puisse se glisser à l’intérieur sans avoir à rentrer son ventre.

Non loin de là, j’ai avisé un interrupteur qui ne demandait qu’à être activé, ce que je me suis empressé de faire, n’éprouvant aucune jouissance particulière à rester dans le noir plus longtemps que nécessaire.

Et la lumière fut.

Et là, franchement, que s’écrier d’autre que (je m’en excuse d’avance auprès de toutes les âmes sensibles et éprises de justice divine que ces paroles pourraient choquer) : NOM DE DIEU DE BORDEL DE MERDE DE FILS DE PUTE D’ENCULÉ DE SA RACE !!!

Dans le fond de la pièce, il y avait ce qui ressemblait comme deux gouttes de sueur à une chaise électrique, dans une présentation à peine plus évoluée que celle qui avait servi à faire griller le jeune George Junius Stinney Jr.\nf{George Junius Stinney Jr. (1929--1944), 14 ans, fut le plus jeune condamné à mort exécuté aux États-Unis au \textsc{xx}e siècle. Son procès, expédié en une journée sans défense effective devant un jury exclusivement blanc, fut jugé inique dès l'époque. En 2014, un tribunal de Caroline du Sud annula sa condamnation et le déclara officiellement innocent. \source{fr.wikipedia.org/wiki/George\_Stinney}} (14 ans) le 16 juin 44 au pénitencier de Columbia. Soupçonné d’avoir sauvagement assassiné deux gamines parties cueillir des maypops dans la campagne ensoleillée du comté de Clarendon, en Caroline du Sud, George Stinney avait commis trois erreurs fatales : 1 : les deux gamines en question (Betty June Binnicker et Mary Emma Thames, respectivement âgées de 11 et 8 ans) étaient blanches ; 2 : il habitait près de chez elles et leur parlait de temps en temps ; et 3, la pire de toutes : il avait eu la mauvaise idée de naître noir, ce qui signifie qu’il a été «~interrogé~» par des flics blancs sans l’assistance d’un avocat, et qu’il s’est ensuite retrouvé devant une salle d’audience exclusivement composée de suprémacistes blancs surexcités à l’idée de bouffer du nègre. Les gens de couleur, privés de leurs droits civiques, étaient invités à rester tranquillement chez eux s’ils ne tenaient pas à se faire lyncher. De toute façon, ce n’était que partie remise, car ils finiraient tôt ou tard par croiser le Klan qui se ferait un plaisir de foutre le feu à leurs baraques et les pendre haut et court.

Saucissonné sur cette chaise, entièrement nu, un individu respirait bruyamment, difficilement, dans un état proche de l’inconscience, le corps couvert d’ecchymoses. Son état général, plus qu’alarmant, laissait présager le pire dans un très proche avenir. Cela dit, en faisant vite, on devait encore pouvoir le sauver.

Je me suis approché, dans l’idée de lui soutirer quelques informations sur son identité et les raisons de sa présence ici, et lui ai donné quelques petites tapes sur l’épaule pour tenter de le faire revenir à lui.

Il a poussé un grognement, assez flippant je dois dire, puis, au prix d’un effort manifestement surhumain, comme si elle avait été lestée de plomb, il a soulevé une paupière et posé sur moi un regard vitreux dans lequel j’ai vu ou cru voir passer, aussi furtive qu’un Lockheed Martin F-22 Raptor dans le ciel de Téhéran, une vague lueur d’espoir. Quand je dis «~soulevé une paupière~», j’entends soulevé de quelques centièmes de millimètres, ce qui ne lui permettait pas de se faire une idée très précise de la situation. Assez, cependant, pour comprendre que le visage sympathique et bienveillant qui se tenait en face de lui n’était pas celui de l’un ou l’autre de ses habituels tortionnaires.

Il a commencé à s’agiter un peu, un filet de bave au coin des lèvres, et sa mâchoire a produit une série de craquements suspects, attestant qu’il s’efforçait de former des mots pour exprimer son point de vue.

J’ai dit : Du calme, mon vieux, tout va bien.

C’est le genre de répartie qui ne veut pas dire grand-chose, sinon rien, mais qu’on se sent néanmoins obligé de prononcer quand on se retrouve seul dans une cave devant un type à poil attaché à une chaise électrique.

Un gargouillis est sorti de sa bouche, en même temps qu’un filet de bave ensanglantée.

J’en ai remis une couche : Ne vous inquiétez pas, je vais vous sortir de là.

Et c’est là que mon téléphone a sonné, que j’ai décroché, ne serait-ce que parce que ce sont des choses qui se font quand son téléphone sonne, et entendu la voix de Titus qui chuchotait, manifestement en proie à quelque chose qui sans être tout à fait de la panique n’en était pas très éloigné : Djef ?

\textsc{Moi} : Oui ?

\textsc{Lui} : Tu m’entends ?

\textsc{Moi} : Oui, je t’entends. Tu devineras jamais ce que j’ai trouvé ici.

\textsc{Lui} : Remonte en vitesse, les autres sont rentrés !

\textsc{Moi} : T’es sûr ?

\textsc{Lui} : Oui, je suis sûr ! La BM est là !

\textsc{Moi} : Et c’est maintenant que tu me préviens !

\textsc{Lui} : J’étais en train de couler un bronze dans un bosquet. Je les ai pas entendus rentrer.

\textsc{Moi} : Ils t’ont vu ?

\textsc{Lui} : Je crois pas, non.

\textsc{Moi} : OK, j’arrive !

\textsc{Lui} : Magne-toi !

J’ai dit au type sur la chaise : Je suis obligé de partir. Mais je vais bientôt revenir, ne vous en faites pas.

J’étais sur le point de faire demi-tour quand j’ai entendu une voix dans mon dos.

Elle s’exprimait avec un léger accent italien, et j’avais déjà eu l’occasion de l’entendre : Je peux savoir ce que vous faites chez moi, inspecteur ?

Je me suis retourné d’un coup, et retrouvé en face de Riqueti et son gorille à chapeau de cowboy et oreilles en chou-fleur. Je n’ai rien contre les cowboys, si ce n’est que la plupart sont quand même des grosses brutes sans cervelle qui ont piqué leur terre aux Indiens (mais vous allez me dire que c’est de bonne guerre, qu’après tout les Indiens, s’ils avaient été moins nuls et arriérés, attachés à des croyances stupides, ne se seraient peut-être pas fait mettre la misère et retrouvés entassés comme des clodos dans des réserves insalubres), mais je n’ai jamais pu encaisser le chou-fleur. Déjà, tout petit, mon crétin de père devait me tabasser jusqu’au sang pour m’en faire avaler une bouchée. En plus de ses oreilles indigestes, le gorille avait un nez tellement écrasé qu’il lui recouvrait presque la totalité du visage. C’était le genre de type qui foutait la trouille même de dos, et quand vous étiez derrière lui, vous imploriez le ciel pour qu’il ne se retourne pas.

Riqueti a dit : Doucement, inspecteur, pas de geste brusque. Fouillez-le, Niccolo.

Donc, le singe s’appelait Niccolo et répondait à son prénom, preuve que, contre toute apparence, il n’était pas totalement dépourvu d’intelligence.

Il s’est approché, avec ses mains énormes qui ressemblaient à des gants de boxe sans gants de boxe (et dans une de ces mains il y avait ce qui ressemblait furieusement à un pistolet semi-automatique Smith \& Wesson Bodyguard 380, un jouet qu’il valait mieux éviter de laisser traîner dans une chambre d’enfant), et n’a pas mis longtemps à dénicher Manu qui tentait de se planquer dans le fond de ma poche tel un petit animal apeuré.

Après quoi, il m’a décoché un petit sourire avec des dents impeccables qui n’étaient manifestement pas d’origine. Je préférais ça plutôt qu’il me décoche une droite en pleine poire.

\textsc{Riqueti} : Donnez-le-moi, je vous prie.

Niccolo lui a remis le flingue.

\textsc{Riqueti}, examinant Manu sous toutes les coutures : Quelle drôle de petite chose.

\textsc{Moi} : C’est Manu.

\textsc{Lui} : Manu ?

\textsc{Moi} : Oui, une arme de collection made in France.

\textsc{Lui} : Vous croyez qu’on peut tuer quelqu’un, avec ça ?

\textsc{Moi} : Avec un peu de bonne volonté.

Il a braqué le flingue dans ma direction et j’ai cru ma dernière ou avant-dernière heure arrivée. J’avais beau tenter de me persuader que Manu, reconnaissant son seigneur et maître (cet homme remarquable qui l’avait sauvé des oubliettes de la grande histoire des armes à feu et avec lequel il avait partagé tant de bons moments à la chasse aux nuisibles), refuserait de faire feu, j’avais un peu de mal à y croire.

Au dernier moment, juste avant d’appuyer sur la détente, Riqueti a détourné le canon.

\textsc{Lui}, apparemment enchanté de la bonne blague qu’il venait de me faire : Mais dites-moi, c’est qu’il marche du feu de Dieu, votre petit engin !

J’ai préféré, pour ne pas risquer de me laisser aller à un mouvement d’humeur qui n’aurait fait qu’envenimer la situation, garder le silence.

\textsc{Riqueti} : Bien, trêve de plaisanterie. Je peux savoir à qui vous parliez ?

\textsc{Moi} : Qui ça ? Moi ?

\textsc{Lui} : Oui, vous, au téléphone.

\textsc{Moi} : Un ami.

\textsc{Lui} : Un ami ? Donnez-moi ce téléphone, s’il vous plaît.

\textsc{Moi} : Non, c’est privé.

\textsc{Lui} : Ici aussi c’est une propriété privée. Ce n’est manifestement pas ça qui vous arrête. Le téléphone, Niccolo.

Niccolo m’a arraché le téléphone de la main, intervention qui m’a fait à peu près le même effet que si une murène avait jailli de son trou pour m’arracher un bras, et l’a tendu à Riqueti qui a jeté un coup d’œil à la liste d’appels et rappelé le dernier numéro.

Titus a décroché : Allo ?

Silence.

\textsc{Titus} : Djef ?

Silence.

\textsc{Titus} : Djef, tu m’entends ?

\textsc{Riqueti} : Oui.

\textsc{Titus} : Qu’est-ce que tu fous, bordel ? Pourquoi tu me rappelles ?

Silence.

\textsc{Titus}, de plus en plus interloqué : Djef ?

\textsc{Riqueti} : Ne vous inquiétez pas, monsieur…. monsieur comment, déjà ?

\textsc{Titus} : Vous n’êtes pas Djef !

\textsc{Riqueti} : Non, je ne suis pas Djef. Mais ne vous en faites pas, il est entre de bonnes mains. Les miennes, en l’occurrence, et celles de Niccolo.

Comprenant enfin qu’il était en train de se faire mener en bateau, et prenant du même coup conscience de la gravité de la situation, Titus a aussitôt raccroché.

Je n’ai pas eu le temps de vous dire, tant j’étais pris dans le feu de l’action, que Riqueti tenait un chien en laisse.

Ce chien, à poil ras, de taille moyenne, affublé de longues oreilles pointues et d’une queue de rat, ne semblait pas le moins du monde intéressé par ma présence. J’en connais qu’autres qui auraient grogné et montré les dents, mais lui était au-dessus de ça. Il me considérait comme une quantité négligeable et tenait à me le faire sentir, exprimer par son détachement hautain que je n’avais pour lui pas plus d’importance qu’une vieille feuille de laitue ou une pomme blette.

\textsc{Riqueti}, voyant que je reluquais son clebs : Je vous présente Terzo, inspecteur. C’est un Cirneco de l’Etna\nf{Le Cirneco dell’Etna est une race de lévrier sicilienne parmi les plus anciennes d’Europe, attestée dès 2 500 av.\ J.-C. par des pièces de monnaie et des mosaïques. La villa romaine du Casale (Piazza Armerina, Sicile, \textsc{iv}e s.), inscrite au patrimoine mondial de l’Unesco, en présente de remarquables représentations de chasse. Aristote le mentionne dans son \textit{Histoire des animaux} (livre IX). \source{fr.wikipedia.org/wiki/Cirneco\_dell\%27Etna}}, un lévrier de très haute lignée dont on retrouve la trace dès l’Antiquité. En Sicile, par exemple, sur les murs de la villa du Casale à Chiazza, on peut voir des scènes de chasse parfaitement conservées sur lesquelles il figure. Vous noterez la noblesse de ses traits. Aristote lui-même en a fait une description assez convaincante dans son Histoire des animaux. Mais je suppose que tout cela vous indiffère.

\textsc{Moi} : Totalement, en effet.

\textsc{Lui} : Vous avez tort, Terzo est un remarquable limier. Je suis sûr qu’il pourrait vous en remontrer à bien des égards.

\textsc{Moi} : Pour chasser le lapin, peut-être. Personnellement, je m’attaque à des proies nettement plus grosses.

\textsc{Lui} : Bien trop pour vous, je le crains. Vous avez un mandat de perquisition ?

\textsc{Moi} : Un quoi ?

\textsc{Lui} : Ne faites pas l’imbécile. Vous n’avez pas de mandat, vous n’êtes pas en service, pour moi vous n’êtes qu’un citoyen lambda pris en flagrant délit de violation de domicile. La légitime défense m’autorise à faire justice moi-même.

\textsc{Moi} : En vertu de quoi ? De l'Édit de Nantes ?

\textsc{Lui} : Non, de l’article 122-6 du Code Pénal qui m’autorise à repousser, de nuit, toute entrée par effraction, violence ou ruse, dans un lieu habité, le mien ne l’occurrence. Le cas de figure me semble avéré.

\textsc{Moi} : Mon cul, oui ! Vous savez ce qu’il en coûte de s’en prendre à un représentant de l’ordre ?

\textsc{Lui}, une grimace censée ressembler à un sourire aux lèvres : De l’ordre, c’est vous qui le dites. Moi, je dirais plutôt un représentant du chaos, un agent du désordre.

Puis, s’adressant à Niccolo et son Bodyguard 380 : Niccolo, mon ami, allez donc faire une tournée d’inspection pour voir si tout se passe bien. J’ai dans l’idée que notre ami n’est pas venu seul.

Niccolo a hoché la tête et s’est éloigné en trottinant comme un bouledogue en rut. Tout ce que j’avais entendu de lui, jusqu’à présent, se résumait à quelques vagues grognements laissant à penser qu’il maîtrisait difficilement l’usage de la parole.

\textsc{Moi} : Vous êtes sur une pente savonneuse, monseigneur.

\textsc{Lui} : Vraiment ?

\textsc{Moi} : Oui, mais il est encore temps de tout arranger. Vous me rendez mon arme, vous me laissez partir et je ferme les yeux sur vos activités illicites. On ne vous l’a peut-être pas dit, mais la peine de mort est interdite en France depuis le 9 octobre 81. Autrement dit, même pour jouer, il est interdit de faire griller des gens dans une cave.

\textsc{Lui} : Je suppose, par contre, qu’il est autorisé de les faire griller dans le coffre d’une voiture.

\textsc{Moi} : On ne va pas se chamailler pour si peu. Vous me laissez partir, et tout le monde continue à faire griller des gens où bon lui semble. Je suppose que vous avez d’excellentes raisons de vous en prendre à cet individu.

\textsc{Lui} : Excellentes, en effet.

\textsc{Moi}, désignant le type sur la chaise, lequel sortait progressivement de sa torpeur et tentait, toujours sans réel succès, d’articuler quelque chose qui ressemble vaguement à un langage connu : Et on peut savoir lesquelles ?

Un certain nombre de petites choses commençaient à tournoyer dans ma cervelle comme des mouches autour d’une charogne. Ce qui m’avait mis la puce à l’oreille, si j’ose dire, c’était le chien. Jusqu’à présent, je n’avais rien trouvé permettant de relier Riqueti au Brain Catcher, mais maintenant j’avais Terzo, chien de race de son état, dont j’imaginais assez bien qu’il n’était pas du genre à bouffer de la pâtée premier prix. Par contre, les croquettes Waterflox à l’agneau et au riz d’Anada Sintawichai, à peu près aussi chères au kilo que le caviar de Beluga albinos ou la Tuber magnatum piémontaise, pouvaient fort bien constituer son ordinaire.

\textsc{Lui} : Disons qu’il a commis des erreurs regrettables, et que le moment est venu pour lui de payer.

\textsc{Moi} : Quel genre d’erreurs ?

\textsc{Lui} : Du genre sexuel, si vous voyez ce que je veux dire.

\textsc{Moi}, après quelques instants d’un silence si épais qu’un pet de moucheron aurait fait l’effet d’une déflagration thermonucléaire : C’est vous, n’est-ce pas ?

Un sourire sorti des tréfonds de son âme dévoyée a transformé son visage jusqu’ici relativement anodin en authentique preuve de l’existence du diable : Moi quoi ?

\textsc{Moi} : C’est vous, le Brain Catcher !

\textsc{Lui} : Je ne vois pas de quoi vous voulez parler. Je suis monseigneur Mathéo Riqueti, évêque du Sanctuaire de Ddarr et ami personnel du cardinal Prospero Cangelosi, le chef de chœur au Vatican. Je ne vois rien de répréhensible là-dedans.

\textsc{Moi}, à propos du type sur la chaise : Et lui, qui est-ce ?

\textsc{Riqueti}, le visage toujours affreusement déformé par cette forme de démence rare qui atteint parfois les ecclésiastiques en fin de carrière : Un homme d’Église, lui aussi. Il s’agit du père Marian Granet, si vous tenez vraiment à tout savoir.

\textsc{Moi}, d’une voix claire et nette : Vous êtes le Brain Catcher !

\textsc{Lui} : Mmmoui, j’en ai vaguement entendu parler, en effet. Sachez que je n’ai rien à voir avec ce triste personnage.

\textsc{Moi} : N’empêche que c’est bien vous qui tuez des prêtres et leur farcissez le crâne avec des croquettes Waterflox !

\textsc{Lui} : Waterflox, dites-vous ? C’est bizarre, c’est justement la marque préférée de Terzo. Vous pensez que mon chien a quelque chose à voir dans cette terrible histoire ?

\textsc{Moi} : Je pense surtout que vous êtes complètement cinglé !

\textsc{Lui} : Vous n’êtes pas sans savoir que la définition de la maladie mentale reste extrêmement sujette à caution. Mais dites-moi, inspecteur, avez-vous déjà assisté à une exécution ?

\textsc{Moi} : Oui, bien sûr.

\textsc{Lui} : Vraiment ?

\textsc{Moi} : Oui, même que la plupart du temps c’est moi qui fais office de bourreau.

\textsc{Lui} : Je vous parle d’une exécution sur la chaise électrique.

\textsc{Moi} : Merci, j’avais compris. Non, évidemment, je n’ai jamais assisté à une exécution de ce genre.

\textsc{Lui} : Et que diriez-vous d’y assister ?

\textsc{Moi} : Vous n’allez pas me dire que cette relique est en état de marche ?

\textsc{Lui} : Cette relique, comme vous dites, est une exacte réplique, à un léger détail près, de la chaise sur laquelle s’est assis le tueur en série sado-maso, pédophile et cannibale Albert Fish\nf{Albert Fish (1870--1936), surnommé «~le Vampire de Brooklyn~» ou «~l'Homme gris~», tueur en série américain reconnu coupable d'au moins trois meurtres d'enfants. Masochiste autoproclamé, il s'était planté des aiguilles dans le périnée~; elles provoquèrent des courts-circuits lors de son exécution à Sing Sing le 16 janvier 1936. \source{fr.wikipedia.org/wiki/Albert\_Fish}}, alias le Vampire de Brooklyn, le 16 janvier 1936. On lui reproche notamment d’avoir étranglé la petite Grace Budd, 10 ans, avant de la découper en morceaux et la manger entièrement. Outre le fait qu’il entendait des voix lui ordonnant de violer, castrer et tuer des petits garçons, il adorait s’enfoncer des aiguilles dans le rectum et se fouetter jusqu’au sang avec une planche à clous de sa confection. Les aiguilles en question ont provoqué des courts-circuits pendant l’exécution, obligeant, pour le plus grand bonheur de Fish, le bourreau à s’y reprendre à plusieurs fois pour arriver à ses fins. L’homme que vous voyez assis sur cette chaise, le père Granet, prétend lui aussi que c’est la voix de Dieu qui lui ordonne de s’en prendre à des petits garçons. Il prétendait tellement que j’ai dû lui en couper une bonne partie pour le faire taire. Vous voyez ce bouton rouge, là-bas ?

\textsc{Moi} : Non.

\textsc{Lui} : Mais si, vous le voyez.

\textsc{Moi} : Je ne tiens pas du tout à le voir.

\textsc{Lui} : Il suffit d’appuyer dessus pour faire rôtir le père Granet, et c’est vous-même qui allez vous en charger.

Le père Granet, qui n’avait plus de langue mais avait encore ses oreilles, s’agitait de plus en plus, parfaitement conscient du caractère critique de sa situation. Il aurait aimé se plaindre, exprimer des objections, faire valoir son point de vue, mais le moignon de langue qui se tortillait dans le fond de sa gorge tel un rat pris au piège ne lui permettait pas de faire entendre sa voix de façon satisfaisante. Il était, je l’ai dit, entièrement nu, et la chaise avait été conçue de telle sorte qu’il lui était possible de se soulager sans avoir à se déplacer. En clair, il était assis sur une chaise percée avec un seau positionné en dessous pour recueillir ses déjections, à la façon d’une cuvette de chiotte.

Alors je ne sais pas ce qui s’est passé, si ses muscles se sont relâchés ou quoi, toujours est-il qu’il s’est mis à uriner en même temps qu’il continuait à tenter désespérément de s’exprimer, les deux formant un mélange que je n’hésiterai pas à qualifier d’assez indigeste. En matière de spectacle avilissant, de façon de rabaisser un homme plus bas que terre, le forcer à renoncer à toute espèce de dignité, Riqueti avait atteint un degré de raffinement qui témoignait d’un réel sens de l’esthétique.

Cela dit, je n’étais pas prêt pour autant à lui servir d’homme de main, d’où ma réponse ferme et définitive : C’est hors de question !

\textsc{Lui} : Vous êtes en mon pouvoir, maintenant, et n’avez pas d’autre choix que d’accéder à mes désirs. Sauf si vous préférez vous assoir sur cette chaise après lui, et mariner dans vos excréments pendant un temps indéterminé avant que quelqu’un se décide à mettre un terme à vos souffrances. Si vous ne le faites pas pour lui, faites-le pour vous. Vous n’avez pas hésité à carboniser Dardariel dans le coffre de sa voiture, je ne vois pas ce qui vous empêche d’en faire autant avec lui. Si ça peut vous décider, sachez que le père Granet n’est pas seulement un pédophile de la pire espèce, mais aussi un assassin qui n’hésite pas à sacrifier des nourrissons pour s’attirer les bonnes grâces de Lucifer. Il perpétue l’exercice occulte des messes noires, à l’image d’un Eustache Blanchet\nf{Eustache Blanchet (?--v. 1440), prêtre poitevin, aumônier de Gilles de Rais. Accusé d'avoir participé à des invocations démoniaques, il fut condamné lors du procès de 1440 mais gracié après avoir témoigné contre Gilles de Rais. \source{fr.wikipedia.org/wiki/Gilles\_de\_Rais}}, un Joseph-Antoine Boullan\nf{Joseph-Antoine Boullan (1824--1893), abbé français défroqué, fondateur de la Société réparatrice. Accusé de satanisme et de pratiques occultes, il servit de modèle au chanoine Docre dans \textit{Là-bas} (1891) de Huysmans. \source{fr.wikipedia.org/wiki/Joseph-Antoine\_Boullan}}, un Étienne Guibourg\nf{Étienne Guibourg (v. 1610--1686), abbé français impliqué dans l'affaire des Poisons (1679--1682). Accusé d'avoir célébré des messes noires pour la marquise de Montespan, maîtresse de Louis XIV, il mourut emprisonné sans jugement à la forteresse de Besançon. \source{fr.wikipedia.org/wiki/\%C3\%89tienne\_Guibourg}}, ou plus récemment un Matthias Schuster, faux pasteur mais véritable escroc qui non seulement couchait avec sa sœur et les enfants qu’il avait eus avec elle, mais s’adonnait à des rituels sataniques dans les ruines de l’abbaye de Grérac, en Corrèze, dont il avait fait l’acquisition pour une bouchée de pain en même temps qu’un hameau abandonné situé en plein cœur de la forêt de Montaulogne, soi-disant pour les remettre en état. Je suppose que vous en avez entendu parler.

\textsc{Moi} : Ça me dit vaguement quelque chose, en effet.

Pendant ce temps, Niccolo avait repéré Titus qui se dissimulait entre deux haies de thuyas, le Glock en main.

Il s’est approché en silence, par derrière, et lui a collé le canon de son flingue entre les omoplates, tout en disant ceci : Laisse tomber ton flingue, doucement, tout doucement, et mets les mains en l’air.

Titus a laissé tomber le Glock, mais au lieu de lever sagement les mains, il s’est brusquement retourné et a saisi le bras qui tenait le Bodyguard.

Une lutte s’est engagée, au cours de laquelle Niccolo a accidentellement appuyé sur la détente, provoquant une détonation qui a attiré l’attention de Riqueti, lequel, je vous le rappelle, me tenait en joue avec Manu.

\textsc{Riqueti}, en entendant le coup de feu, a eu le mauvais réflexe de tourner la tête.

Telle la mangouste qui se jette sur le serpent, j’ai profité de l’occasion pour lui balancer un coup de latte dans les parties. Les saintes couilles de notre ami l’évêque ont fait le yoyo dans son slip. Riqueti est tombé à genoux, la gueule ouverte, les yeux exorbités, avant de se rouler par terre en poussant des beuglements de veau à l’abattoir. Au passage, il a laissé tomber Manu et la laisse du clebs, lequel, jusqu’ici d’une placidité à toute épreuve, je dirais même d’une indifférence rare, s’est instantanément transformé en bête fauve tout droit sortie d’une époque où bestialité et sauvagerie exacerbées régnaient en maîtresses absolues sur la terre de nos ancêtres.

Pourvu d’une détente assez impressionnante, le canidé est passé à trois mètres au-dessus de ma tête au moment même où je me penchais pour ramasser Manu. Il a fini sa course à l’autre bout de la pièce, et le temps qu’il réussisse à freiner en catastrophe, faire demi-tour et repartir à la charge, j’avais eu largement le temps de récupérer mon vieux compagnon de route.

Je lui ai serré chaleureusement la crosse, sur laquelle étaient gravées les initiales de mon grand-père maternel, PC pour Philibert Chéron (et Parti Communiste, auquel il appartenait effectivement, de toutes les forces de sa foi vibrante en l’avenir de l’Homme et la perspective d’un monde meilleur, sans misère ni injustice), le père de ma génitrice adorée Valentine Chéron (VC, oui), résistant de la première heure et tueur de nazis digne (même s’il n’était pas juif, à ma connaissance en tout cas) d’un Aldo l’Apache ou un Donny Donowitz, et lui ai appuyé une première fois sur la détente qu’il avait souple et onctueuse.

Terzo ne se trouvait plus qu’à quelques mètres de moi quand la première balle, tirée au jugé je dois bien le dire, l’a atteint à la cuisse. Il a poussé un jappement de douleur tout à fait déchirant mais continué à courir comme si de rien n’était, comme si sa fureur le rendait insensible à la douleur.

Pourquoi ne pas le dire, au risque de passer pour un crétin sentimental d’une naïveté crasse à la limite de la niaiserie la plus inacceptable, mais il se trouve que j’ai une certaine tendresse pour les animaux. Au même titre que nous, ils font partie de la grande famille du vivant et tâchent de tirer au mieux, avec les moyens rudimentaires dont ils disposent, leur épingle du jeu sadique auquel ils sont contraints de participer. Cela dit, même si je savais fort bien que ce pauvre Terzo agissait davantage comme une machine à tuer décérébrée qu’un authentique criminel dont la seule et unique joie consiste à infliger de la souffrance à ses semblables, je n’allais pas me laisser égorger comme un mouton pendant la fête de l’Aïd. L’instinct de survie n’est-il pas la force principale qui régit notre existence ? Il faut, en permanence, se protéger d’une multitude d’agressions de toutes natures, dont on ne sait jamais trop quand ni comment elles vont nous tomber sur le dos, ce qui tend à générer du stress et de l’angoisse pouvant aller jusqu’à la psychose et la paranoïa (laquelle va nous pousser à détruire ce qui nous entoure à titre préventif, nous chuchoter dans le creux de l’oreille «~tue untel ou untel si tu veux sauver ta peau~»). Les mécanismes de défense sont au cœur de notre activité, et l’être humain, conscient de cette triste réalité, a mis au point un certain nombre de dispositifs très utiles en cas de danger. Et parmi ceux-ci, même si on ne pouvait plus parler à son égard de fleuron de la technologie moderne, le revolver Le Français de Manufrance (ici en modèle de poche 6.35, l’équivalent national du Walther thuringeois) restait un exemple tout à fait convaincant de l’ingéniosité déployée par l’être humain pour se protéger de l’adversité.

Les deuxième et troisième coups sont partis quasiment coup sur coup, c’est le cas de le dire, et je dois admettre en toute modestie que tous deux ont atteint leur cible, je veux bien sûr parler de ce pauvre Terzo qui ne faisait finalement rien d’autre qu’effectuer son boulot de chien, en toute innocence.

Hélas, tout a une fin, et jamais plus personne ne l’entendrait aboyer, chose qu’il ne faisait de toute façon que très rarement, compte tenu de son pédigrée et son éducation hors pair. C’est avec beaucoup de dignité, après un ultime soubresaut, qu’il a rendu son âme à Anubis, dieu des chiens, auquel il ressemblait d’ailleurs comme deux gouttes d’eau (au même titre que deux espèces très proches, le chien de garenne des Canaries d’une part, redoutable chasseur de lapin s’il en est, et surtout le lévrier de Malte qui serait un descendant direct de Tesem, le chien de Khéops, deuxième pharaon de la IVe dynastie dont la pyramide se dresse à Gizeh, non loin de celles de Khéphren, son fils, et Mykérinos, le fils de son fils).

Riqueti, qui commençait tout juste à se remettre de ses émotions, a poussé un long cri d’effroi en voyant son animal adoré étendu dans la poussière. Quelqu’un de plus charitable que moi lui aurait sans doute laissé le temps de pleurer à loisir son compagnon disparu, mais j’étais d’assez mauvaise humeur et peu enclin à faire preuve de compassion. En guise de condoléances, je lui ai braqué mon flingue sous le nez, et ordonné de s’étaler face contre terre, les bras en croix.

Pendant ce temps, Titus était aux prises avec Niccolo.

À première vue, la bataille semblait inégale. D’un côté Titus, force de la nature dépassant le mètre quatre-vingt-dix et capable de décorner un buffle à mains nues, et de l’autre Niccolo, son chapeau de cowboy ridicule, ses oreilles en chou-fleur et son nez épaté, d’une carrure impressionnante, certes, et doté de paluches surdimensionnées, mais dont la taille s’apparentait à celle de ce qu’on aurait pu appeler un «~nain de grande taille~», c'est-à-dire grand pour un nain mais totalement ridicule pour un être humain normalement constitué. Attention, je ne prétends aucunement que les nains sont des êtres humains anormalement constitués. Non, je dis juste que certaines particularités physiques permettent de les distinguer à coup sûr du commun des mortels. Et qui sait, peut-être un jour régneront-ils en maîtres sur la Terre, ce qui ne serait d’ailleurs pas une mauvaise idée car on pourrait réduire considérablement la hauteur sous plafond des appartements et loger ainsi beaucoup plus de monde. Mais bon, on n’en est pas encore là (cela dit, petite prévision gratuite en passant, je ne serais pas étonné que l’être humain rétrécisse dans les siècles à venir, jusqu’à retrouver une taille proche de celle qu’il avait à la préhistoire).

Donc, si on s’en tenait aux dimensions des belligérants, la bataille semblait inégale. Sauf que Niccolo était un boxeur hors pair, aussi difficile à toucher qu’une anguille. Titus, après un corps-à-corps viril pendant lequel il avait pu juger de la force hors du commun de son adversaire, était néanmoins parvenu à lui faire lâcher son engin (je parle bien sûr de son Smith \& Wesson Bodyguard 380, 380 pour 380 ACP, soit le 9 mm court conçu dès 1903 par le très inventif John Moses Browning\nf{John Moses Browning (1855--1926), armurier américain, l'un des inventeurs les plus prolifiques de l'histoire des armes à feu. Outre la cartouche .380 ACP (1903), il conçut le Colt M1911, le Browning Hi-Power, la mitrailleuse M2 «~Ma Deuce~» et le Winchester Model 1894. \source{fr.wikipedia.org/wiki/John\_Moses\_Browning}} pour équiper le Colt Pocket Hammer). Un Titus, je le rappelle, qui avait lui-même laissé tomber son Glock quelques instants auparavant.

Niccolo a entamé les débats par un direct au corps qui s’est échoué sur les abdos de Titus, naturellement bien doté de ce côté-là. Si c’était moi qui avais pris ce coup, je crois que je serais encore allongé par terre à regarder danser les étoiles dans le ciel. Il était évident, à voir la façon dont il se déplaçait, que Niccolo avait de nombreuses années de boxe derrière lui. Titus s’est dit qu’un crochet au foie suivi d’une droite à la mâchoire serait une combinaison intéressante à essayer. En face de lui, Niccolo dansait avec la légèreté d’une libellule, distribuant au passage des coups dont la plupart faisaient mouche. Le crochet au foie de Titus n’est pas arrivé à destination, et la mâchoire tant convoitée n’était plus là quand son poing s’est présenté. Pendant ce temps, les jabs continuaient à pleuvoir et atteindre leur cible avec une précision diabolique. Titus, même s’il était loin d’avoir inventé l’eau tiède, était quand même pourvu de facultés intellectuelles lui permettant de se faire une idée raisonnable de la situation. Et le moins qu’on puisse dire, c’est que la sienne était tout sauf enviable. Il ressemblait à un de ces sacs de sable sur lesquels les boxeurs frappent à tour de bras pour s’entraîner. À aucun moment il n’avait le temps de s’organiser tant Niccolo faisait preuve d’une rapidité déconcertante. Il pensait éviter un coup, et venait au contraire s’empaler sur un autre qu’il n’avait pas vu venir. Sa résistance était mise à rude épreuve et la défaite semblait inévitable.

C’est alors qu’un coup de théâtre, de ceux qui ont fait la renommée des tragédies antiques, s’est produit.

Alors que Niccolo continuait à tabasser gentiment sa proie, s’amuser avec elle, la réduire méthodiquement à l’impuissance, il a mis le pied dans un trou. Sa cheville a émis un craquement de mauvais augure, et l’expression de joie sadique qui tapissait son visage s’est aussitôt changée en voile noir et grimaçant. D’un coup, qui ne devait rien à Titus, le Rudolf Noureev\nf{Rudolf Noureev (1938--1993), danseur et chorégraphe soviétique, considéré comme le plus grand danseur de ballet du \textsc{xx}e siècle. Il fit défection à l’Ouest le 17 juin 1961 à l’aéroport du Bourget (Paris) lors d’une tournée du Ballet Kirov, puis dirigea le Ballet de l’Opéra de Paris (1983--1989). \source{fr.wikipedia.org/wiki/Rudolf\_Noure\%C3\%AFev}} des rings s’est transformé en pantin désarticulé incapable de mettre un pied devant l’autre, boitillant de façon ridicule en essayant de se protéger de son mieux.

Vous pensez bien que Titus, en bon opportuniste qu’il était, a profité de l’occasion pour lui balancer tout ce qu’il avait dans la tronche, avec une violence décuplée par l’humiliation qu’il venait de subir. Une avalanche de coups s’est abattue sur Niccolo, lequel n’a pas tardé à mettre un genou, puis les deux à terre.

Titus, après lui avoir, en guise de cerise sur le gâteau, décoché un dernier coup de tatane dans les ratiches, a récupéré son Glock et le Bodyguard. Il était maintenant seul maître à bord, et l’autre le regardait avec des yeux remplis d’un mélange scintillant de haine et de désespoir.

Titus lui a collé le canon du Bodyguard contre la tempe, et il a dit : Bouge pas.

\textsc{Niccolo} : Tu fais le malin, maintenant, mais tu m’aurais jamais eu si je m’étais pas pris les pieds dans le tapis.

\textsc{Titus} : C’est vrai que t’es coriace.

\textsc{Niccolo} : Laisse-moi filer et on en reste là. Tu peux garder mon flingue en souvenir, si tu veux.

\textsc{Titus}, au moment même où son téléphone se remettait à sonner : Y a intérêt, que je vais le garder !

Il a décroché, tout en surveillant l’autre du coin de l’œil, et entendu ma voix chaude et enveloppante comme un manteau de fourrure en léopard des neiges qui lui murmurait dans le creux de l’esgourde : Titus ?

\textsc{Lui} : Ouais.

\textsc{Moi} : Alors, t’en es où ?

\textsc{Lui} : Je suis maître de la situation.

\textsc{Moi} : Il est mort ?

\textsc{Lui} : Non, mais je le tiens en joue.

\textsc{Niccolo}, entre ses dents : T’as vraiment eu de la chance, enfoiré !

\textsc{Moi} : Parfait, alors bute-le.

\textsc{Lui} : Tu crois ?

\textsc{Moi} : Oui, et après viens me rejoindre en bas. Tu ne devineras jamais ce que j’ai trouvé !

\textsc{Lui} : Vas-y, raconte.

\textsc{Moi} : Bute-le et viens me rejoindre en bas.

\textsc{Lui} : T’es sûr qu’il faut vraiment le buter ?

\textsc{Niccolo}, qui commençait à se faire du souci : Quoi ? Buter qui ?

\textsc{Titus} : Ta gueule, toi !

\textsc{Moi} : Comment ça, ta gueule ?

\textsc{Titus} : Non, pas toi, c’est l’autre qui me parle. T’es sûr qu’il faut vraiment le buter ?

\textsc{Moi} : Qu’est-ce que tu veux en faire ?

\textsc{Titus} : Je sais pas. On pourrait le laisser filer.

\textsc{Niccolo} : Oui, bonne idée. Vous en faites pas, je dirai rien à personne.

\textsc{Moi} : Pour qu’il aille nous balancer à la première occasion ? Certainement pas, non !

\textsc{Titus} : Tu crois qu’il ferait ça ?

\textsc{Niccolo}, de plus en plus inquiet pour son avenir : Quoi ? Faire quoi ?

\textsc{Titus} : Ferme-la, je t’ai dit !

\textsc{Moi} : C’est à moi, que tu parles ?

\textsc{Titus} : Non, c’est l’autre abruti qui arrête pas de me parler ! Commence vraiment à me casser les oreilles, celui-là !

\textsc{Moi} : Tu ne veux pas le buter ?

\textsc{Lui}, manifestement en proie à des questionnements existentiels qui n’étaient pas monnaie courante chez lui : J’ai pas dit ça.

\textsc{Moi} : T’es tombé amoureux de lui, ou quoi ?

\textsc{Lui} : Dis pas de conneries !

\textsc{Moi} : Non, parce que ça arrive parfois que des mecs se battent et tombent amoureux l’un de l’autre. Si c’est le cas, je vous souhaite tous les vœux de bonheur.

\textsc{Lui} : Okay, ça va.

\textsc{Moi} : Quoi ?

\textsc{Lui} : C’est bon, je vais le buter.

\textsc{Niccolo} : Ah non, ça va pas recommencer ! J’ai rien fait, moi, je suis juste un homme de main. Je vois pas pourquoi je devrais trinquer à la place des autres !

\textsc{Moi} : Je savais que je pouvais compter sur toi.

\textsc{Lui} : Tu restes au téléphone ?

\textsc{Moi} : Pourquoi faire ?

\textsc{Lui} : Je sais pas. J’aimerais juste que tu restes au téléphone pendant que je le bute.

\textsc{Moi}, compréhensif : Tu veux que je t’encourage un peu pendant que tu appuies sur la détente ?

\textsc{Lui} : Oui, je sais pas. Je le sens pas, cette fois.

\textsc{Niccolo} : Si tu le sens pas, faut pas le faire ! Il s’agit quand même de la vie d’un homme, merde !

\textsc{Moi} : Allez, vas-y, mon vieux. Prends ton temps, respire à fond, et quand tu sens que tu es prêt tu appuies sur la détente, tranquillement, sans forcer.

\textsc{Titus}, essayant de se remonter le moral : C’est vrai, ça, j’ai déjà buté plein de gens de sang-froid.

\textsc{Moi} : Bien sûr, des tas ! Et dis-toi bien que tous l’avaient amplement mérité.

\textsc{Lui} : Ouais, t’as raison.

\textsc{Moi} : Bien sûr que j’ai raison.

\textsc{Lui} : Je comprends vraiment pas ce qui se passe. En plus, ce type a vraiment une sale gueule et il vient de me dérouiller sévère. Je devrais lui en vouloir à mort.

\textsc{Moi} : Tu sais, il se passe parfois des choses curieuses, dans l’existence. On a buté des dizaines, voire des centaines de gens, et puis un jour, sans qu’on sache pourquoi, on n’arrive pas à appuyer sur la détente. Est-ce que c’est l’ampleur de la tâche qui nous submerge, l’impression de prêcher dans le désert, de se décarcasser pour des prunes ? Oui, peut-être bien, n’empêche qu’il faut se forcer à retourner au charbon encore et toujours, sans jamais céder au découragement.

\textsc{Titus} : C’est bon, je le sens bien, là !

\textsc{Moi} : Je suis de tout cœur avec toi, ma vieille !

C’est au moment précis où Titus allait enfin faire son devoir de tueur froid et sanguinaire que Niccolo, mu par cet instinct de survie qui trouve sa source au plus profond de notre génome, s’est détendu tel un fauve dans la savane, ou plutôt un phacochère surpris par un fauve dans la savane, et mis à courir de toute la force de ses jambes courtes mais puissantes dans la première direction qui s’offrait à lui.

\textsc{Moi}, commençant à trouver le temps long : Alors, ça vient ?

\textsc{Titus} : Oui, attends, je…

\textsc{Moi} : Quoi encore ?

\textsc{Titus} : Il est en train de se faire la malle !

\textsc{Moi} : Ben qu’est-ce que tu attends ? Vas-y, tire-lui dessus !

Titus a tiré à deux reprises en direction du fugitif, lequel a continué à détaler en zigzaguant comme un lapin avant de disparaître au coin d’un massif de fleurs.

\textsc{Moi} : C’est bon, tu l’as eu ?

\textsc{Titus}, pressant une troisième fois la détente : C’est qu’il est rapide, le bougre !

\textsc{Moi} : On ne peut pas en dire autant de toi.

\textsc{Lui} : Je suis vraiment désolé.

\textsc{Moi} : Tu l’as raté, si je comprends bien ?

\textsc{Lui} : Ouais, je crois bien.

\textsc{Moi} : Eh oui. Voilà ce qui arrive quand on perd son temps en tergiversations oiseuses !

\textsc{Lui} : T’inquiète, il a eu la peur de sa vie. On ne le reverra pas de sitôt. De toute façon, j’ai gardé son flingue.

\textsc{Moi} : Il en a peut-être un autre planqué quelque part.

\textsc{Titus} : Penses-tu ! Il a eu du bol de sauver ses fesses et il ne va pas venir demander son reste.

\textsc{Moi} : Mouais, reconnais que t’as quand même pas été très efficace sur ce coup-là. Bon allez, viens me rejoindre en bas, j’ai des choses intéressantes à te montrer.

Quand Titus s’est pointé, penaud, et qu’il a vu le petit spectacle que je lui réservais, à savoir un chien mort, son maître avec les couilles en vrac et un père Granet plein d’espoir ficelé sur sa chaise électrique, il n’a pu retenir un : Nom de Dieu !

\textsc{Moi} : Comme tu dis ! Est-ce que je peux te demander de tenir cette ordure en joue pendant que je détache le prisonnier ? Tu penses que c’est dans tes cordes ?

\textsc{Lui}, désolé : Oui, je sais, j’ai merdé. Mais ne t’inquiète pas, s’il a le culot de se repointer ici je lui fais sauter le caisson sans sommation.

\textsc{Moi}, commençant à détacher le père Granet qui se trouvait dans un état de délabrement physique (à commencer par son bout de langue manquant) et moral assez impressionnant : Je t’accorde qu’il s’en tire à bon compte, raison pour laquelle je doute fort qu’il remette les pieds ici.

\textsc{Titus} : C’est qui, lui ?

\textsc{Moi} : Un certain père Marian Granet, à ce que j’ai cru comprendre. C’est bien ça, Monseigneur ?

\textsc{Riqueti}, les traits déformés (déjà qu’il n’était pas trop beau au naturel) par une haine féroce : Oui, et aussi une sale ordure qui a violé et assassiné des tas d’enfants innocents !

\textsc{Moi} : Loin de moi l’idée de me comparer à un enfant innocent, mais je vous rappelle que vous avez vous aussi tenté de me faire assassiner, mon cher ami.

\textsc{Lui} : Simple malentendu.

\textsc{Moi} : Fâcheux tout de même.

\textsc{Lui} : Excusez-moi, mais je ne savais pas de quel côté vous étiez.

\textsc{Moi} : Toujours celui de la justice, soyez-en sûr. Figure-toi, mon cher Titus, que notre bon ami le Cardinal a admis à demi-mot être le Brain Catcher.

\textsc{Titus} : Non ?

\textsc{Moi} : Si.

\textsc{Riqueti}, l’air renfrogné des grands jours : Je n’ai rien admis du tout !

\textsc{Moi} : Je suis certain que si on va faire un tour à l’étage, on va trouver des croquettes Waterflox.

\textsc{Titus} : À l’agneau et au riz ?

\textsc{Moi} : Oui, riz au jasmin et agneau mariné à la sauce hoisin, conçues à l’origine par le grand chef Anada Sintawichai pour Tuani, sa chienne Thaï Ridgeback à poil bleu.

\textsc{Riqueti} : Je ne vous pardonnerai jamais d’avoir tué Terzo !

\textsc{Moi} : Oui, je dois reconnaître que ça m’embête un peu. Je ne vois pas pourquoi les animaux doivent toujours payer le prix fort pour la connerie de leurs maîtres. Bon, en même temps, s’ils n’étaient pas programmés pour voler bêtement au secours de leurs supérieurs hiérarchiques, y compris des crétins de la pire espèce qui n’ont aucune réelle estime pour eux, ce genre de chose n’arriverait pas.

Une fois le père Granet sorti de sa chaise, je me suis rendu compte qu’il tenait à peine debout et l’ai posé dans un coin pour vaquer à la suite de mes occupations. J’ai quand même suggéré à Titus de le tenir à l’œil au cas où il reviendrait soudainement à la vie et déciderait de nous jouer un tour à sa façon. Je n’avais pas vraiment de raison de douter de ce que m’avait raconté le cardinal à son sujet, mais je ne pouvais pas non plus l’exécuter sans autre forme de procès. Avant de décider de quoi que ce soit à son sujet, une enquête s’imposait.

Vaguement contrarié (il arrive qu’on se sente vaguement contrarié, sans trop savoir pourquoi, surtout que dans le cas présent les choses se déroulaient plutôt pas mal), j’ai attrapé Riqueti par le bras en disant : Monseigneur, si vous voulez bien vous donner la peine.

\textsc{Lui}, paniqué : Qu’est-ce que vous faites ?

\textsc{Moi} : Je vous en prie, installez-vous.

\textsc{Lui} : Mais vous êtes fou ! Arrêtez ça tout de suite !

\textsc{Moi}, lui enfonçant Manu dans le creux des reins et le poussant dans la chaise : J’ai dit assis !

\textsc{Lui} : Vous n’avez pas le droit de faire ça !

\textsc{Moi} : Vous vouliez bien me forcer à faire griller le père Granet.

\textsc{Lui} : C’était pour rire. De toute façon, cette machine ne fonctionne pas.

\textsc{Moi} : Dans ce cas vous n’avez rien à craindre. Attache-le, Titus.

\textsc{Lui} : Je vous préviens, j’ai le bras long. Tout ça va vous coûter très cher !

\textsc{Moi} : Il faut savoir se faire plaisir de temps en temps.

\textsc{Lui} : Je n’ai rien fait de mal. Le père Granet était consentant, demandez-lui. Il aime les sensations fortes, si vous voyez ce que je veux dire. Cette situation l’excitait au plus haut point, c’est lui qui a insisté pour que je le condamne à la chaise électrique. Il a fait de choses discutables, pas très catholiques, ou trop, et il voulait expier ses fautes dans la douleur. Je lui ai dit : mon cher, s’il n’y a que ça pour vous être agréable, j’ai exactement ce qu’il faut à la maison.

\textsc{Moi} : N’empêche que vous alliez le faire griller.

\textsc{Lui} : Non, juste le torturer un peu à l’électricité pour son plus grand plaisir.

\textsc{Moi} : Je ne sais pas si vous le savez, mais pendant une électrocution la température du corps peut monter jusqu’à 100 degrés, entraînant des effets visuellement assez insupportables. En plus de se pisser et se chier dessus, il n’est pas rare que le supplicié s’enflamme ou que ses yeux soient éjectés de leurs orbites comme des bouchons de champagne ! C’est ce qui est arrivé à Albert Clozza en 1991.

L’avantage, avec le cardinal Riqueti, c’était qu’il était chauve et qu’on pouvait par conséquent s’épargner la corvée de lui raser le crâne avant de le mettre sous tension. Restait maintenant à espérer pour lui que le petit bijou qu’il s’était fait fabriquer fonctionnait aussi bien qu’il l’avait déclaré précédemment, sachant que la chaise restait quand même un des instruments de torture les plus barbares jamais mis au point par l’être humain pour se débarrasser de son prochain. À tel point qu’aujourd’hui quasiment plus personne ne l’utilise, à part quelques états parmi les plus sympathiques, les plus accueillants et les moins racistes du territoire américain, je veux bien sûr parler de la Virginie, de la Caroline du Sud, du Tennessee et de l’Alabama, ou griller un Noir de temps à autre fait encore partie des traditions et de ces bons vieux souvenirs qu’on se plaît à évoquer le soir au coin du feu, la pipe au bec et un verre de George Dickel à la main.

Le manuel de l’utilisateur, rédigé dans une langue facile d’accès et généreusement enrichi d’illustrations de qualité, détaillait la marche à suivre pour s’assurer d’une cuisson optimale, notamment les endroits où il était préférable de placer les électrodes, les spécialistes s’étant souvent écharpés à ce sujet. Si le sommet du crâne bien dégagé arrivait toujours en tête du classement pour la première électrode, la question de savoir s’il valait mieux placer la seconde sur le haut ou le bas de la jambe, voir le pied, faisait toujours débat. Dans le cas présent, le constructeur préconisait les testicules, ce qui techniquement n’était pas une mauvaise idée mais soulevait quand même quelques légers problèmes d’éthique auxquels il semblait assez peu perméable. Cela dit, à titre expérimental, j’ai décidé d'opter pour cette solution.

\textsc{Moi} : Il y a quand même une petite question que j’aimerais vous poser, Monseigneur.

\textsc{Lui}, de mauvais poil : Fichez-moi la paix !

\textsc{Moi} : Croyez-moi, ça vous fera du bien de soulager votre conscience.

\textsc{Lui} : Elle va très bien, merci !

\textsc{Moi} : J’aimerais savoir pourquoi vous vous en prenez à des hommes d’Église, les découpez en morceaux que vous expédiez aux quatre coins de la ville, et leur farcissez le crâne avec des croquettes pour chien ?

\textsc{Lui} : Qu’est-ce qui vous fait croire que c’est moi ?

\textsc{Moi} : Qui d’autre ?

\textsc{Lui} : Je ne sais pas, moi, un fou ! Je ne suis pas fou, et vous êtes sur le point de commettre une très grave erreur judiciaire.

\textsc{Moi} : Et vous, vous étiez sur le point de faire griller le père Granet.

\textsc{Lui} : Pas du tout. Comme je vous l’ai dit, il s’agissait d’un jeu à caractère sexuel.

\textsc{Moi} : Ben voyons ! Et ce morceau de langue que vous lui avez coupé ?

\textsc{Lui} : Il arrive que les choses dérapent dans le feu l’action.

\textsc{Moi} : Ce morceau de langue, je ne serais pas étonné que quelqu’un le reçoive bientôt dans la boîte aux lettres.

\textsc{Lui} : Vous n’avez aucune preuve !

\textsc{Moi} : Mais vous, vous avez un chien. Aviez, pardon.

\textsc{Lui} : Et alors ?

\textsc{Moi} : Alors je suis prêt à parier que vous le nourrissiez avec des croquettes Waterflox. Je n’aurai aucun mal à en trouver ici.

\textsc{Lui} : Et alors, ça ne prouve rien ! Des tas de chiens mangent des croquettes Waterflox.

\textsc{Moi} : À 3000 balles le kilo ? Je crois pas, non.

\textsc{Lui} : Je parle de chiens de race qui ont la chance de vivre avec autre chose que des maîtres chômeurs et alcoolos.

\textsc{Moi} : D’autre part, je suis certain que si je fouille dans ce gros congélateur que j’entrevois là-bas dans le fond de la pièce, je vais trouver des tas de bocaux avec des vrais morceaux de curés à l’intérieur.

Riqueti s’est recroquevillé dans sa chaise en me regardant de travers. J’avais vu des photos de déments prises dans des asiles psychiatriques de sinistre renom, aujourd’hui laissés à l’abandon pour la plus grande joie des urbex en mal de sensations fortes, eh bien le cardinal ressemblait tout à fait à l’un d’entre eux. Il aurait fait merveille avec une camisole de force sur le dos.

\textsc{Moi} : Je ne comprends pas ce qui peut pousser un homme d’Église à s’en prendre à d’autres hommes d’Église.

\textsc{Lui} : Je n’ai pas toujours été un homme d’Église, jeune homme. J’ai été enfant, comme tout monde, et victime des agissements pervers d’un de ces hommes d’Église dont vous parlez. Plusieurs, même, mais un en particulier qui m’a laissé un très mauvais souvenir. Pour faire court, je l’ai menacé de tout dire à mes parents et il a tué Leo, le Cirneco de l’Etna que mon père m’avait offert pour mon cinquième anniversaire, une bête superbe qui était mon plus fidèle ami et confident, à laquelle je tenais comme à la prunelle de mes yeux. Ensuite, il a dit que si ça ne suffisait pas il s’en prendrait à Nina, ma sœur, puis à mon père et ma mère si nécessaire. J’étais terrorisé et n’ai jamais rien dit à personne. Mais plus tard, je me suis juré d’avoir ma revanche sur tous ces enfoirés, raison pour laquelle je suis entré dans les ordres avec la ferme intention de me hisser au plus haut niveau de la hiérarchie. De l’intérieur, je n’aurais aucun mal à savoir qui faisait quoi et prendre les décisions qui s’imposaient. Pour vous dire la vérité, je me fiche complètement de Dieu, Jésus et toutes ces conneries. Ma seule et unique motivation a toujours été d’étancher ma soif de vengeance.

\textsc{Moi} : Pourquoi les découper en morceaux ?

\textsc{Lui} : Il avait fait la même chose avec Leo.

\textsc{Moi} : Et leur farcir la tête avec des croquettes pour chien ?

\textsc{Lui} : Petit touche de créativité personnelle, rapport à l’expression «~mettre du plomb dans la cervelle~». Moi, c’est des croquettes pour chien, de qualité supérieure, vous l’aurez noté. On a bien le droit de s’amuser un peu, non ?

\textsc{Moi} : Bien sûr. D’ailleurs, si vous n’y voyez pas d’inconvénient, on va maintenant passer au clou du spectacle.

\textsc{Lui} : Faites ce que vous voudrez. De toute façon, Terzo est mort et je n’ai pas la force de continuer à vivre sans lui. Vous promettez de me farcir le crâne avec des croquettes pour chien une fois que je serai mort ?

\textsc{Moi} : Désolé, mais je n’ai pas que ça à faire.

\textsc{Lui} : Dommage.

\textsc{Moi} : Il ne nous reste plus qu’à vous souhaiter bon voyage.

Ainsi s’achevait, avec le cardinal Mathéo Riqueti, la sanglante épopée du Brain Catcher.

Au moment de partir, une divine surprise nous attendait, preuve que les justes sont toujours récompensés : alors que nous passions au détour d’une allée, traînant avec nous le père Granet qui suait sang et eau pour mettre un pied devant l’autre, on est tombés sur ce bon vieux Niccolo adossé à un arbre, en train de se vider tranquillement de son sang. Titus n’était pas aussi nul que je le pensais, et qu’il le croyait lui-même, car il avait, tout en suivant une conversion au téléphone (pas passionnante, au demeurant), réussi un tir létal sur une cible mouvante, ce qui n’est pas donné à tout le monde. Niccolo, dans sa fuite, avait pris une balle dans le dos, et se trouvait à présent entre la vie et la mort. Comme on n’avait pas de temps à perdre, qu’il n’avait manifestement aucune chance de s’en tirer et que même si ça avait été le cas on ne l’aurait certainement pas laissé faire, Titus s’est à nouveau servi du flingue de Niccolo pour mettre un point final au roman de gare de son existence, bien trop longue succession de pages mal écrites distillant un mortel ennui. Après quoi il a méticuleusement essuyé le Bodyguard avant de replacer l’objet entre les mains de son propriétaire. L’engin avait dû servir à buter pas mal de gens, et Titus n’avait aucun intérêt à ce que son nom soit associé à son palmarès. L’heure était venue de rendre à César ce qui lui appartenait, et César allait pouvoir aller griller en enfer avec son mentor Riqueti, lequel, lorsqu’il aurait le cul transformé en rôti de porc cramé jusqu’à l’os, aurait peut-être un petit moins envie de faire joujou avec les appareils électriques.

Quant au père Granet, à en croire ses explications rendues quelque peu vaseuses par son état physique général, il ne s’agissait aucunement de l’ignoble pédophile dont Riqueti nous avait vanté les mérites. Son seul tort, dans l’affaire, était d’avoir des penchants sexuels d’une nature un peu particulière, certes, mais qui, s’exerçant toujours entre personnes majeures et consentantes, ne sortaient pas du cadre de la légalité. Bien sûr, à titre personnel, il ne m’inspirait que la plus vive répulsion, mais je ne me sentais pas qualifié pour juger de ses agissements (lesquels, selon moi, étaient du ressort de la psychologie la plus intime), et la seule chose que je pouvais lui conseiller, à part consulter au plus vite les meilleurs spécialistes de la confusion mentale, était de se tenir aussi loin de moi qu’il lui était possible de le faire, avec tout le respect que j’avais pour lui et le calvaire qu’il venait d’endurer.

Et aussi, bien évidemment, de tenir sa langue, ou sa moitié de langue, s’il tenait à la conserver, et je ne doutais pas qu’il suivrait mes prescriptions à la lettre. Même pour un type comme lui, sensible aux délices de la douleur, qui ne se sentait jamais plus en vie que lorsqu’il était assis sur les genoux de la mort, l’expérience avait été profitable. Il allait, à présent, réfréner ses pulsions les plus morbides et s’en tenir à des pratiques moins traumatisantes, plus consensuelles. Il avait pris conscience, au cours de cette mésaventure qui avait bien failli lui être fatale, que l’existence avait un prix au-delà duquel il n’était pas disposé à s’engager.

Titus et moi, en tout point fidèles à la réputation d’excellence qui était la nôtre, allions le déposer à l’hôpital le plus proche, où il chanterait de sa plus belle voix la chanson suivante : enlevé par un sadique alors qu’il rentrait chez lui à la tombée de la nuit, il avait été soumis à de multiples sévices avant de mettre à profit un instant d’inattention de son geôlier pour réussir à s’échapper. Il ignorait totalement l’endroit où il se trouvait et avait couru des heures durant dans la forêt avant de trouver une issue. Il avait ensuite, alors qu’il errait, dans un état second, sur une route secondaire dont il serait bien incapable de préciser la localisation, été pris en stop par un individu qui l’avait conduit jusqu’ici. L’individu en question, dépourvu de tout signe particulier permettant de le distinguer du commun des mortels, était pressé et ne tenait pas s’attirer de complications inutiles, raison pour laquelle il était aussitôt reparti, fin de l’histoire. Naturellement, tout cela s’était passé comme dans un de ces cauchemars dont on se réveille en sueur, le cœur battant, les yeux écarquillés dans la pénombre, et il était bien incapable de donner des informations précises sur les conditions de sa détention. D’autant que son bourreau, chaque fois qu’il lui rendait visite, non seulement portait un masque de Grand Méchant Loup de film d’horreur qui lui couvrait intégralement le visage, mais, à l’instar d’un Batman ou un Dark Vador, s’exprimait à travers un dispositif sonore qui rendait sa voix méconnaissable.

Enfin, et j’en terminerai là, j’ai embarqué quelques flacons de Bourgogne supplémentaires avant de prendre congé, la seule idée qu’ils puissent tomber entre de mauvaises mains me rendant littéralement ivre de rage. J’estimais, entre ma première razzia et celle-ci, avoir sauvé du naufrage les plus remarquables d’entre eux, même si je m’étais vu, bien à contrecœur, dans l’obligation de faire des choix qui avaient été pour moi autant de coups de poignard dans le dos de ma passion pour le jus de raisin fermenté. Ôôôôô fermentus, fermentatis, fermentatorum, laisse à tout jamais déferler dans mon gosier avide les flots tumultueux de ta splendeur fruitée aux saveurs incomparables ! Fasse que, à jamais enchaîné au flux continu de tes liquides bienfaits, jamais mon âme ne connaisse ce dessèchement intérieur qui ronge les meilleurs d’entre nous ! Amen.

