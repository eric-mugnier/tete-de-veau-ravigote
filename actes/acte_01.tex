\chapter{Acte 1}

\noindent Une décharge d’un milliard de volts a déchiré le ciel comme un vieux chiffon bouffé aux mites, aussitôt suivie d’une déflagration à vous faire se dresser les cheveux sur la tête et s’éjecter les dents des gencives.

Quelques heures plus tôt, drainés par des bourrasques dignes de l’apocalypse, les nuages avaient rappliqué en masse et plongé le secteur dans les ténèbres.

Gulav (Gulav Behram, dit «le Kurde») était une des pires pourritures qui aient jamais vécu à la surface de cette terre de merde, ce trou à rat qui avait vu ces cons de Romains clouer Jésus sur une croix et ces fumiers de Nazis exterminer les Juifs par treize à la douzaine. Depuis, les réjouissances n’avaient cessé de s’enchaîner à un rythme frénétique. L’horreur montait en puissance, les limites de l’abjection semblaient pouvoir être repoussées indéfiniment.

Personne ne savait d’où il sortait et tout le monde s’en foutait, à commencer par lui. Quelqu’un qui n’a aucun avenir n’a pas à se soucier de son passé. Il vivait de rapine, vol, viol, escroquerie, extorsion et autres saloperies du même genre.

En un mot comme en cent, Gulav était né pour faire le mal et n’avait cessé de s’employer à mériter pleinement ses galons de pourriture hors norme. Il donnait envie de dégueuler à tout le monde, mais les gens se retenaient pour ne pas finir avec un couteau dans le dos. En bon clébard amateur de levrette et de chiennes en chaleur, Gulav était un vicieux qui faisait ses coups par derrière. Un pro, quoi.

Or donc, ce jour-là, vla ti pas qui faisait un temps à ne pas mettre un chien dehors, précisément, mais cette sous-merde de Gulav, cet étron magistral chié dans la douleur par l’anus même de Satan qui s’était planqué tel un ver solitaire infernal dans les intestins de sa mère, était pire que le pire charognard vomi par les entrailles de la putréfaction. Il était, à lui seul, la preuve irréfutable de la non-existence de Dieu. Désolé, je m’en excuse aussi platement qu’une limande auprès des grenouilles de bénitier, cathos de droite antisémites et autres enculeurs de louveteaux, mais si Dieu avait créé une bouse comme Gulav, alors il ne méritait pas qu’on lui adresse la moindre prière, sinon celle de se faire tout petit dans son coin et fermer sa gueule à tout jamais.

Gulav s’est dit que c’était le moment idéal pour se faire une petite bicoque. Il la surveillait depuis un bout de temps et savait que les proprios s’étaient fait la malle. En vacances, certainement. Déjà que c’était des retraités qui n’en foutaient pas la rame, ils trouvaient encore le moyen de se payer du bon temps au frais de la princesse. Pendant que les jeunes se cassaient le cul pour trois francs six sous, ces fossiles vivants s’en foutaient plein la gueule au club Med ou au camping de Palavas-les-Flots. Gulav leur préparait une petite surprise pour leur retour. Non seulement il allait tout saccager, mais il ne se gênerait pas pour marquer son territoire en se soulageant sur le tapis du salon. Arsène Lupin laissait sa carte de visite, lui, c'était de la merde.

Comme à son habitude, Gulav s’est pointé dans une voiture volée, une caisse tellement pourrie que son propriétaire ne se donnerait même pas la peine de signaler sa disparition. De toute façon, personne ne se donnerait la peine de la rechercher.

Il a profité d’une brève accalmie pour se ruer hors de la poubelle, le pied-de-biche à la main. Gulav était d’une laideur telle que même les cafards dégueulaient en croisant son chemin. Les seules filles auxquelles il pouvait prétendre, sans avoir recours au viol sur fond de tortures à faire passer le marquis de Sade pour un bienfaiteur de l’humanité, étaient les paumées défoncées jusqu’à l’os qui n’auraient pas hésité à s’accoupler avec un troupeau de sangliers en échange d’une dose de came. Les gens n’imaginent pas, avachis dans le canapé en simili de leur médiocrité quotidienne, à quel point il se passe des trucs moches à deux pas de chez eux. Et s’ils parviennent malgré tout à se l’imaginer, ils s’en foutent. Au contraire, le fait de savoir que des gens crèvent la gueule ouverte dans le caniveau participe à leur bonheur. Ils ne vous le diront jamais, bien sûr, et finiront même par se convaincre que c’est faux, mais le fait est que le malheur des autres contribue très largement à leur félicité.

C’était un de ces petits pavillons de merde comme on en trouve un peu partout dans les endroits destinés à accueillir les gens qui ne souhaitent pas s’entasser dans des immeubles de cinquante étages à l’isolation douteuse. Ils préfèrent pourrir dans un cercueil individuel, en tout point semblable à celui du voisin (hormis quelques petites touches personnelles dont on se passerait volontiers), mais soigneusement délimité par des barrières physiques plus ou moins hermétiques que nul ne doit s’aviser de franchir sans laissez-passer sous peine de prendre une décharge de calibre 12 à travers la gueule. Ils veulent un petit lopin de terre avec des nains de jardin, une fontaine en plastoc, une cabane branlante pour ranger leurs outils et une pelouse tondue au ras des pâquerettes. Ils veulent la complète panoplie du parfait connard qui a tout bien réussi dans sa vie, y compris sa mort. Ses frais d’obsèques lui ont coûté tellement de pognon qu’il est pressé de crever pour en profiter. Je serais d’avis, pour économiser un  peu d’espace, d’enterrer tous ces crétins insignifiants dans leurs jardins, sous les chrysanthèmes et bégonias qu’ils arrosent à l’eau de source à longueur de journée. Les locataires s’y succéderaient et chacune de ces cryptes des temps modernes aurait droit à son petit cimetière personnel.

La plupart de ces boîtes à chaussures sont aujourd’hui occupées par des vieux (ou futurs vieux) qui votent (très) à droite et ne savent pas se servir d’un téléphone portable. Ils détestent les étrangers et les gens (surtout les étrangers) qui se garent devant chez eux. Ils ont des gosses, parce qu’avoir des gosses faisait partie de leur plan de carrière, leur vie rêvée de crétins obsolètes voués à la décrépitude. D’autant qu’avoir des gosses ne demande pas de qualités particulières : il suffit d’avoir un équipement en bon état de marche et de faire les poubelles pour trouver une conjointe. Dès qu’ils ont fini par dénicher un emploi à peu près stable, ils se sont fait construire (certains l’ont construit de leurs propres mains, on n’est jamais mieux servi que par soi-même) un de ces foutus pavillons pour former à leur image une portée de rejetons édentés qui porteront à leur tour le flambeau de la connerie vers des horizons insoupçonnés. Je devrais plutôt dire leur nid, qu’il aurait sans doute fallu dératiser avant qu’il soit trop tard. Leurs gosses ont quitté le navire dès qu’ils ont été en âge de le faire. En fait de navire, il s’agissait plutôt d’une coquille de noix sur un océan de merde balayé par les bourrasques incessantes d’un destin aussi haineux qu’une meute de chiens enragés. Aujourd’hui, ces mêmes gosses, qui ont aussi des gosses et un foyer, attendent qu’ils crèvent pour solder la bicoque et se payer du bon temps avec le fric. Ils sont la copie conforme de leurs parents, le portable en plus (auquel ils s’accrochent comme des désespérés pour tenter de donner un sens à leur vie, se sentir autre chose qu’un pion en perdition sur un échiquier trop grand pour lui), et se ruinent pour élever des gosses qui leur fausseront compagnie à la première occasion. Ils ont, accroché au mur de leur salon, le calendrier idéal d’une vie réussie, et entendent bien cocher toutes les cases avant que le croque-mort referme sur leur tronche déconfite le couvercle de leur boîte en sapin.

On pourrait dire que Gulav n’avait jamais eu de chance. On pourrait dire aussi qu’il était tellement con que tout ce qu’il entreprenait ne pouvait faire autrement que tourner rapidement au vinaigre.

Ce soir-là, je n’étais pas en service. Je dis ça parce que je suis flic, et en service la plupart du temps comme la plupart des flics. Flic est un boulot de merde, mais comme tous les boulots sont des boulots de merde, je me suis dit que quitte à faire un boulot de merde autant se balader librement avec un flingue à la ceinture. Et puis, à part soldat, c’est le dernier métier où vous avez le droit de tuer des gens en toute impunité. Sauf que si vous êtes soldat, vous êtes obligé de vivre H24 avec d’autres crétins au crâne rasé dans une caserne à la con. C’est quand même plus cool de vivre dans le vrai monde avec ses propres fringues sur le dos, sans avoir à marcher au pas, ramper dans la boue et se faire postillonner à la gueule à longueur de journée par une brute épaisse à haleine de chiotte. Et puis à l’armée, pour tuer des gens, il faut aller à la guerre, et là, vous avez de fortes chances d’en prendre une avant d’avoir eu le temps de vous faire plaisir. Trop risqué. Bien sûr que les malfrats aussi sont dangereux, mais dans la police on peut se permettre de les flinguer sans raison et bricoler une petite mise en scène pour enfumer une hiérarchie plus ou moins complaisante.

Ce vieux Gulav avait un casier long comme le bras, mais, allez savoir pourquoi, il finissait toujours par se retrouver dehors. Oui, je sais, c’est le mal du temps de foutre les prisonniers dehors. Plus assez de place dans les prisons. Un jour, il faudra demander à l’habitant d’en prendre un ou deux chez lui pour participer à l’effort collectif. À charge pour lui de les attacher dans le cave ou le grenier. Pour ce qui est de Gulav, il était tellement chiant que même les taulards n’en voulaient plus. Quand il se retrouvait en cabane, il ne fallait pas plus de deux ou trois jours pour que les détenus se mutinent et fassent une grève de la faim pour exiger son expulsion. Donc on le foutait dehors en le priant de se tenir à carreau. Naturellement il n’en faisait rien, parce qu’il était bien trop con pour comprendre un traître mot de ce qu’on lui racontait. On en avait tellement marre de voir sa gueule qu’on finissait par ne même plus se donner le mal de lui courir après.

Je m’étais souvent dit qu’un jour ou l’autre, si je ne voulais pas devenir fou à force de tourner en boucle avec cet abruti dans le paysage, je devrais me résoudre à mettre un terme définitif à ses pitoyables activités.

Il semblait que ce jour soit enfin arrivé.

Par le plus grand des hasards, puisque je rentrais paisiblement chez moi après une petite balade en ville.

Cette bouse sur pattes avait mal choisi son jour pour faire des siennes. Quand je dis «choisi» c’est une façon de parler, parce qu’en réalité il ne se passait pas une journée sans qu’il trouve le moyen de faire chier le monde d’une façon ou d’une autre.

Donc, quand j’ai vu l’autre crétin des Alpes garer sa poubelle sur le trottoir d’en face, un sourire est apparu sur mon visage, découvrant une rangée de quenottes à faire pâlir d’envie le grand méchant loup en personne (et je ne parle même pas de l’Hydre de Lerne, Alien, Smaug et Godzilla, vulgaires animaux de compagnie tout juste bons à bouffer des croquettes, miauler au coin du feu et se tortiller d’aise quand on leur gratte le bide).

Je me suis garé un peu plus loin et l’ai regardé se faufiler dans les ténèbres liquides tel le reptile abject qu’il n’avait cessé d’être depuis le jour maudit de sa naissance, neuf mois environ après que son abruti alcoolique de géniteur avait cru malin de fourrer son zguègue dans le cloaque maternel, sorte de jungle marécageuse infestée de bestioles qui auraient fait s’enfuir à toutes jambes le plus endurci des aventuriers. Même un Livingstone (je parle de David, pas de Jonathan le goéland qui n’aurait pour rien au monde accepté de mettre une plume dans ce bourbier) aurait longuement hésité avant de tenter l’expédition. Il n’était pas rare, au détour d’un buisson malodorant grouillant d’araignées venimeuses et autres grenouilles tueuses, d’y croiser des créatures mutantes à mi-chemin entre le tapir et l’anaconda, l’alligator et le singe hurleur, la panthère noire et la sangsue de cinq mètres de long. Ce terrain vague, cette zone de non-droit hantée par les serviteurs du Mal aurait dû être éradiquée de la carte depuis bien longtemps. D’ailleurs, quand je dis «cru malin» (cf. ci-dessus), je devrais plutôt dire «tragique concours de circonstances qui l’avait, au détour d’une beuverie dans l’un ou l’autre de ces bars louches où il avait ses habitudes, conduit à commettre un geste irréparable dont l’Humanité n’allait pas tarder à payer le prix fort». Si la Nature avait ne serait-ce qu’un semblant d’intelligence, elle ferait en sorte que de telles abominations n’aient aucune chance de se produire. Au lieu de ça,  cette conne pousse ses ressortissants à forniquer à tout-va sans se soucier des conséquences.

Grand seigneur, je lui ai laissé un peu de temps pour se mettre à l’aise, prendre ses marques. Après quoi, j’ai vérifié que «Manu», mon 6.35 Manufrance (arme de collection, chargeur sept coups qu’il convient de vider d’une traite sur l’agresseur pour s’assurer d’une réelle efficacité létale, pas très moderne, je vous l’accorde, voire vieillot, mais j’aime flinguer français) était chargé à bloc et suis, avec la souplesse d’un félin d’une grâce infinie (un peu empâté, le félin, diront les mauvaises langues), sorti de ma caisse pour lui emboîter le pas.

Je vous rappelle que des rideaux de flotte dégringolaient du ciel illuminé par un feu d’artifice d’éclairs tonitruants, situation assez désagréable sur le plan épidermique, mais globalement favorable pour progresser de façon discrète dans un environnement hostile. Si quand même, vous ne m’enlèverez pas de l’idée que ce genre de quartier discret aux trottoirs impeccables, peuplé de patriotes armés jusqu’aux dents, reste un environnement assez hostile. Souvent âgés et insomniaques, maniaques, paranoïaques, durs de la feuille et à moitié aveugles, pétris de ressentiment à l’égard d’une existence qui n’a cessé de leur pourrir consciencieusement la vie, les patriotes en question ont tendance à tirer dans le tas au moindre pet de travers.

Comme indiqué précédemment, Gulav était une saleté de clébard sournois qui faisait ses coups par derrière. Au lieu de passer par devant, il avait préféré faire le tour de la baraque et s’attaquer à la porte de derrière, laquelle donnait sur une vague terrasse et un jardinet ceint d’une haie de thuyas du plus bel effet. Et à en juger par l’état de celle-ci (la porte de derrière, pas la haie de thuyas taillée au cordeau), il n’avait pas à proprement parler choisi la méthode douce. Au lieu de travailler la serrure en souplesse, comme tout monte-en-l’air qui se respecte, il l’avait démontée au pied-de-biche en arrachant la moitié de la porte. Ce type était la honte de la profession, et l’intersyndicale des cambrioleurs m’avait adressé plusieurs suppliques pour faire cesser le massacre.

Trempé comme une soupe, le 6.35 à la main, je me suis faufilé à l’intérieur. J’avais tout du héros des temps modernes, l’espion sur l’échiquier international du Mal, le cavalier noir, le fou prend la reine (par derrière), la tour dans le cul du roi, le reptile subtil qui se glisse en silence dans les interstices du vice pour que surgisse in extremis la justice.

L’endroit, même pour un héros des temps modernes lancé à pleine vitesse dans les corridors de la vengeance, était d’un classicisme déprimant qui ne donnait aucune envie de s’attarder. Se sauver en courant, plutôt. Aucune personnalité, que des objets bas de gamme accumulés au cours d’une existence totalement dépourvue d’intérêt, aussi morne et plate qu’un paysage de Wallonie. Il fallait vraiment que Gulav ait du temps à perdre pour s’introduire là-dedans, ou alors il disposait d’informations laissant entendre qu’un magot était caché quelque part. Mais c’était aussi improbable que de voir une soucoupe volante atterrir dans le jardin et une fournée de Télétubbies en descendre pour ramasser des choux avant de rentrer chez eux. Gulav était du genre menu fretin : un petit collier par-ci, une bague par-là, une petite culotte fraîchement portée, quelques pièces de monnaie suffisaient à son bonheur. Son pied, il le prenait en violant l’intimité des gens ou les gens eux-mêmes à l’occasion, les dames d’un certain âge de préférence, bien mûres, à la limite du blet, qu’il prenait plaisir à martyriser avant de prendre la fuite en saccageant tout sur son passage, non sans leur avoir auparavant tatoué un G entre les omoplates avec la pointe de son couteau à cran d’arrêt. Un G comme Gulav (ou encore Gibbon, Gencive, Gaz puant, Génocide, Glauque, Gonorrhée, Gras du bide, Gymnosperme, etc), sa carte de visite sur peau humaine.

Après une rapide visite du propriétaire, j’ai trouvé mon Gulav à quatre pattes dans le salon en train de braquer le faisceau de sa torche sous une commode Louis XV. Du faux, bien sûr, du Louis le Bien-Aimé made in Taïwan in 1995 par des enfants en bas âge élevés à l’alcool de riz et à coups de trique dans le bas du dos. Je ne sais pas ce que cet abruti cherchait et m’en foutais royalement. Seule la vue réjouissante de son postérieur gentiment offert à la semelle de ma chaussure offrait pour moi un quelconque intérêt.

Il s’est pris son coup de pompe pleine bourre et est allé gentiment s’écraser contre la commode Louis XV dont il inspectait les dessous avec tant d’attention. Il a laissé tomber la torche au passage. J’ai profité de ce qu’il était en train de rassembler ses esprits pour la ramasser et la lui braquer en pleine face.

Il s’est mis à cligner des yeux, l’air bovin, la gueule ouverte, le nez en sang, et une série de grognements dont j’ai été incapable de décrypter la signification éventuelle est sortie de son gosier.

\textsc{Votre serviteur} (pour info, je m’appelle Djeferson Beauvais, avec un D et un seul F, je vous le dis maintenant, ce sera fait) : Ça va, Gulav ?

Allez savoir pourquoi, mon père était un fan de Thomas Jefferson, riche propriétaire terrien et planteur de Charlottesville, dans le comté d’Albemarle en Virginie, mathématicien, naturaliste (il fréquentait le baron von Humboldt, membre de l’Académie des sciences et président de la Société géographique de Paris), horticulteur, inventeur (d’une machine à faire les nouilles et du cylindre de chiffrement polyalphabétique qui porte son nom), lockiste, rousseauiste et polyglotte, troisième président of the United States of America après George Washington et le peu connu John Adams, pote de Condorcet et d’Alembert (du temps où il était ambassadeur à Paris), grand amateur de bonne chère, notamment les macaronis, les gaufres, le muscat de Frontignan et le vin en général, spécialités qu’il se fait un devoir de ramener dans ses valises pour les faire découvrir à ses concitoyens enthousiastes. Esclavagiste (on lui attribue un cheptel de près de six cents esclaves), certes, comme tout bon Américain du Sud qui se respecte, et pas que les Américains du Sud, du reste, même s’ils étaient et sont restés parmi les plus réfractaires à l’émancipation des Noirs, l’exploitation des races dites «inférieures» étant alors une pratique assez répandue dans le monde, mais progressiste et nullement insensible aux charmes de la communauté noire, des femmes (jeunes) notamment, et tout particulièrement ceux d’une certaine Sally Hemings, laquelle, si on en croit la rumeur et surtout les analyses ADN réalisées à la fin des années 90 sur certains de ses descendants, lui aurait donné au moins un fils, élevé dans le plus grand secret dans sa propriété de Monticello. Mon paternel (paix à son âme corrompue et maintes fois souillée par le péché) étant manifestement infoutu d’orthographier correctement le nom de ce glorieux personnage, ou désireux de le franciser ou pire encore de faire preuve d’originalité, chose qui ne lui ressemblait pas, j’avais hérité de ce prénom ridicule.

\textsc{Lui} : Qu’est-ce que vous faites là, commandant ?

\textsc{Moi} : Rien, je passais dans le coin, j’ai vu de la lumière, je me suis dit : tiens, je parie que ce vieux Gulav est encore en train de faire des siennes. J’entre, je jette un œil et là bingo qui je vois à quatre pattes dans le salon en train de fureter sous la commode ? mon vieux copain Gulav, plus laid et répugnant que jamais ! Tu cherches quelque chose de particulier ?

Mon Gulav, chafouin : j'ai perdu une boucle d’oreille.

\textsc{Moi} : Il serait temps que t’arrêtes de tapiner, mon lapin, c’est plus de ton âge. En plus, ta perruque est de travers, on dirait une vieille folle.

\textsc{Lui} : Vous allez me laisser partir ?

\textsc{Moi}, lui agitant le 6.35 sous le nez : Et comment ! Je vais même t’aider à le faire. Tu sais qui c’est, ça ?

Il a hoché la tête en faisant la moue (un truc horrible qui lui donnait l’air d’un vieux mérou constipé) pour signifier que non seulement il n’en savait rien, mais qu’il s’en fichait comme de sa dernière dent creuse.

\textsc{Moi} : C’est Manu, mon fidèle équipier. Et Manu, il ne peut vraiment pas te saquer.

\textsc{Lui} : On peut trouver un arrangement, chef. Je vous file la moitié du butin et vous me laissez filer.

\textsc{Moi} : C’est tentant, je l’avoue. Sauf que Manu n’est pas du tout de cet avis.

C’est là qu’il a essayé de se jeter sur moi.

C’était bien essayé mais Manu a été plus rapide. Perso, j’aurais pu me contenter de lui sonner la cloche à coups de crosse, mais Manu était du genre hypersensible, surtout au niveau de la détente. Le coup est parti très vite (900 km/h environ) et la balle est entrée direct dans le cœur du sujet, celui de Gulav en l’occurrence. Comme j’ai déjà eu le plaisir de vous l’expliquer, le 6.35 est un calibre sympathique mais peu efficace en termes de rentabilité mortifère, raison pour laquelle une seconde balle s’est aussitôt échappée du canon en direction du crâne de l’intéressé, suivie d’une troisième qui allée s’échouer dans une zone assez indistincte mais néanmoins sensible de sa masse corporelle.

Il s’est affalé et mis à couiner comme un porc à l’agonie. Vraiment aucune dignité. Je ne suis pas spécialiste de la question, mais à vue de nez il lui restait quelques minutes à se tordre douleur avant de crever. J’aurais pu tirer un fauteuil, allumer un cigare (j’avais justement un Don Carlos de Fuente qui trainait dans le fond de ma poche) et le regarder tranquillement se vider de son sang, mais il commençait à se faire tard et j’avais hâte de retrouver la quiétude de mon foyer, mes vieux livres poussiéreux et la tiédeur de mes draps.

J’ai donc, dans un accès d’humanité assez inhabituel, vidé le restant du chargeur dans sa carcasse, mettant du même coup un arrêt définitif à la carrière de Gulav Behram, dit «le Kurde». Plus jamais les vieilles dames en chemise de nuit ne verraient débarquer sa sale gueule dans l’embrasure de leurs portes, plus jamais il ne prendrait congé de ses copains de beuverie en leur plantant un couteau dans le dos, plus jamais il ne fendrait des crânes à coups de hache ni ne volerait des bagnoles pour défoncer les devantures de magasin. Une nouvelle ère commençait.

Cela dit, je suis un perfectionniste. Certains salopent tout et repartent sans se soucier des conséquences, moi j’ai été élevé dans le respect des autres et moi-même à travers eux. Je n’allais pas laisser traîner cette grosse merde de Gulav au milieu du salon. Je l’ai attrapé par les pieds, tirer jusqu’à l’entrée, devant laquelle il avait eu la riche idée de garer sa poubelle, et installé au volant comme si de rien n’était. Avant de partir, j’ai refait le plein et lui ai relogé une demi-douzaine de bastos dans le buffet, histoire d’être bien certain qu’il n’allait pas de réveiller et se faire la malle.

Celui ou celle qui allait tomber sur cette grosse andouille truffée de plombs le lendemain en promenant son chien allait faire une drôle de tête.
