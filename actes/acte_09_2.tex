
Vous rigolez, mais ça s’est déjà vu que des gens déplacent des choses pendant la nuit pour vous faire croire que vous perdez la boule. Dans les vieux films policiers, par exemple, il n’est pas rare qu’une blonde pulpeuse avec des yeux en velours bleu, des dents de vampire, des lèvres de sangsue, une taille de guêpe et deux obus à la place des seins, engage, pour une raison X ou Y (le plus souvent parce que la pauvre chérie est victime d’un odieux chantage qui risque de faire capoter son mariage avec son mari richissime et vieillissant, ce qui aurait pour conséquence désastreuse de la priver de toute ressource et l’obliger à retourner racoler le micheton dans les bars des hôtels de luxe), un détective privé alcoolique au charme ravageur qui ne se départit jamais de son humour caustique et sa décontraction à toute épreuve. Un beau soir, la créature arrive en courant dans le bureau de ce dernier, toujours impeccablement coiffée en dépit d’une course effrénée sous une pluie battante, pour lui annoncer qu’elle vient d’être témoin d’un meurtre horrible et qu’elle préfère venir le trouver lui plutôt que la police car elle ne tient aucunement à être mêlée officiellement à cette sombre histoire. Après quelques hésitations, durant lesquelles le privé pose avec insistance son regard acéré sur les courbes vertigineuses de la fille, il consent à enfiler son imper (faute de mieux), allumer une clope, poser son chapeau sur sa tête, et la suivre sur les lieux du drame afin de vérifier l’authenticité de ses dires. Tous deux prennent place dans sa grosse voiture cabossée, traversent la ville en sens inverse, toujours sous la pluie et sans desserrer les dents (la tension est palpable, l’atmosphère en prise directe sur le triphasé, on sent bien qu’il ne faudrait pas grand-chose pour qu’ils se jettent l’un sur l’autre et que le détective lui roule une de ces putains de pelles d’anthologie dont il a le secret sans même enlever sa clope de sa bouche et tout en gardant un œil avisé sur la route), et quand ils arrivent sur place, le cadavre a disparu sans laisser la moindre trace. Le sang a été nettoyé et les meubles et objets renversés remis en place comme si de rien n’était. Du coup la fille passe pour une affabulatrice bonne pour l’hôpital psychiatrique, se lance dans des explications à n’en plus finir, toutes aussi dépourvues de sens les unes que les autres, et finalement, histoire de ne pas s’être déplacé pour rien, le privé lui roule une pelle d’anthologie avec la clope au bec et le chapeau sur la tête, avant, grand seigneur, de la ramener discrètement chez son mari cocu (parce qu’elle le trompe, bien entendu, et plutôt deux fois qu’une, la salope, avec tout ce qui lui tombe sous la chatte, du jardinier à tablette de chocolat au chauffeur ancien boxeur en passant par le puceau de service et le comptable érotomane).

Mais ce qui arrive assez fréquemment avec des cadavres ou des objets de valeur dans des appartements haussmanniens du huitième arrondissement de Paris, des villas en bord de mer, des cabanes au fond des bois ou même de simples chambres de bonne situées au dernier étage d’immeubles insalubres, se produit nettement plus rarement avec des hôtels entiers, particuliers ou non, car même si vous disposez de moyens exceptionnels pour déplacer un tel établissement dans son entièreté, il vous faudra impérativement opérer à une heure très avancée de la nuit. Et même de nuit, si l’hôtel, comme c’est généralement le cas des hôtels, particuliers ou non, est situé dans ce qu’on appelle le centre-ville ou sa périphérie immédiate, je doute fort que vous arriviez à le déplacer par la voie des airs sans attirer l’attention de personne. Ne serait-ce que celle des forces de l’ordre, par exemple, qui effectuent des rondes régulières dans les centres-villes, ou encore celle des fêtards attardés qui, même déchirés au point d’éprouver les plus vives difficultés à mettre un pied devant l’autre, seraient les premiers surpris de voir un tel édifice passer au-dessus de leurs têtes échevelées.

J’ai fait le tour du pâté de maisons un certain nombre de fois, que je n’ai pas jugé nécessaire de compter, avant que quelqu’un se décide enfin à monter dans sa putain de bagnole, glisser la clef de contact dans la fente prévue à cet effet, démarrer le moteur et se déplacer d’un lieu à un autre pour des raisons qui lui appartenaient et ne m’intéressaient en aucune façon.

Se déplacer a toujours fait partie des préoccupations majeures de l’espèce humaine, et elle le fait maintenant de plus en plus vite pour occuper le maximum d’espace en un minimum de temps, rêvant d’ubiquité, téléportation et autres colonies interstellaires pour assouvir ses coupables penchants. La question qui se pose est la suivante : que faire quand on est une espèce éminemment invasive qui ne dispose d’aucun autre prédateur qu’elle-même ? L’autorégulation est-elle une option viable ? Peut-on légitimement compter sur la tempérance d’une espèce dont la voracité légendaire ne laisse aucune place à l’expectative ou l’introspection, sinon pour une poignée de marginaux au crâne rasé qui vivent reclus dans des ruines vieilles de plusieurs siècles (pour vivre heureux vivons caché, c’est bien cul nu) ? Arrêtons de manger de la viande, les animaux sont nos amis pour la vie et en plus c’est mauvais pour la santé, s’époumone la jeunesse 2.0 qui se teint les cheveux en rose fluo, flotte dans des fringues XXL et revendique le droit de pouvoir jouer au foot quand on est une fille et à la poupée quand on est un garçon, ainsi que la liberté de choisir tel ou tel sexe si l’on estime, notamment à l’adolescence quand on est en plein quête de soi et recherche de la vérité existentielle, que le sien n’est pas conforme. On se demande bien au nom de quoi une fille ne pourrait pas venir au monde avec des attributs masculins, et inversement un garçon avec ce qui fait habituellement les charmes de la féminité. La nature fait des erreurs, c’est à nous qu’il revient de les corriger. Elle en fait même un peu trop, raison pour laquelle il serait peut-être temps, alors que la cinquième génération de téléphone sans fil vient de voir le jour et que nous n’avons aucunement l’intention de nous arrêter en si bon chemin, de songer à la sortir du jeu une bonne fois pour toutes. Après des centaines de millions d’années de règne sans partage, l’heure de la retraite a sonné. Je bande donc je suis, peut-être, mais pas forcément un garçon. Et oui, c’est vrai, j’ai une jolie petite paire de loches bien rebondies qui prend le frais dans mon Lise Charmel à balconnet, mais ce n’est pas pour autant que je suis une fille. Merde, il est temps de mettre un terme à cette vision étriquée du genre humain ! La nature est vieille, dépassée, ses injonctions n’ont plus aucun sens dans le contexte actuel, elle a perdu le contact avec les nouvelles générations, l’élève a dépassé le maître, Dieu est mort, vive Dieu ! La tyrannie des sexes, qui veut qu’on soit maçon ou garagiste quand on est un garçon et danseuse nue ou secrétaire de direction quand on est une fille, a fait son temps. Quant à cette pratique d’un autre âge que j’évoquais précédemment, qui consiste à élever des animaux dans des conditions déplorables et les assassiner pour se repaître de leur chair en toute impunité, elle encourage la bestialité qui est en nous et nous rabaisse au rang des plus vils représentants d’un passé à jamais révolu. Devenons de paisibles herbivores, abandonnons tout esprit de compétitivité, toute trace d’ego malsain, d’individualisme forcené, éclairons-nous à la bougie et arrêtons de boire du sang comme des vampires poussiéreux. Nous sommes des milliards sur terre, tous animés des meilleures intentions, et je ne doute pas qu’avec un peu de bonne volonté il soit possible de faire en sorte que tout se passe pour le mieux dans le meilleur des mondes. Nous venons en paix, amis Terriens, et apportons dans nos valises les nouvelles technologies de la félicité éternelle. Il y a trop, bien trop longtemps que vous vous entretuez bêtement (pour des raisons d’une telle frivolité que nous peinons encore à comprendre la nature réelle de vos motivations), il vous faut maintenant apprendre à vivre en bonne intelligence. Enculez-vous les uns les autres, a dit en substance notre Sauveur bien-aimé avant de finir cloué sur une croix entre deux malfrats et de repartir au Ciel la queue entre les jambes, et ne faites pas aux truies ce que vous n’aimeriez point qu’on vous fasse, bande de putois malodorants ! Vous n’aimeriez pas qu’on vous mange, n’est-ce pas, même si certains d’entre vous l’ont fait pendant un certain temps avant de prendre conscience que cette pratique n’était pas sans doute la mieux adaptée à l’établissement d’une paix durable entre les peuples. Alors laissez ces pauvres bêtes tranquilles, laissez-les se gaver de glands, gambader joyeusement dans la luzerne, et chassez de votre tête cette obsession de vouloir à tout prix les transformer en jambon, saucisson, pâté de tête et andouillette.

Mais je m’égare, une fois de plus (et je ne vous cache pas que parler de jambon, saucisson, pâté de tête et andouillette, m’a donné une solide envie de me restaurer sans plus attendre).

Pour en revenir à ce que je disais précédemment, avant de m’embarquer dans cette diatribe futuriste qui, je l’imagine, risque de faire grincer quelques dents aussi bien dans les clapets réactionnaires de l’extrême-droite décomplexée que la bouche en cœur des bobos de la Rive Droite, un de nos plus gros problèmes sur terre est qu’il y a tellement de voitures partout qu’il n’y a plus moyen de se garer nulle part. S’il y a encore moyen de trouver une place à la cambrousse, entre deux troupeaux de vaches garés en double file, la chose est devenue quasiment impossible dans les centres-villes saturés du 21e siècle.

C’est ainsi que nous étions des dizaines, que dis-je des dizaines, des centaines à tourner autour du Caribbean Hôtel, dont certains qui tournaient jour et nuit depuis des semaines et avaient fini par développer un tel degré d’exaspération qu’il aurait suffi de la plus minuscule étincelle pour mettre le feu aux poudres et déclencher une guerre civile comme on n’en avait pas connu depuis la Fronde et le régime de Vichy. Aussi opportuniste que la hyène qui profite d’un instant d’inattention du guépard pour lui subtiliser la gazelle qu’il vient de passer des heures à traquer dans la brousse, j’ai profité de ce qu’une personne handicapée (il s’agissait en fait d’un individu parvenu à un tel degré d’obésité que ses jambes disparaissaient presque entièrement sous une cascade de plis graisseux) remonte péniblement dans son SUV pour me substituer à elle en toute illégalité. Particulièrement bien équipé, je disposais en effet, en plus d’une affichette sur laquelle était inscrit en lettres majuscules «~INTERVENTION POLICE~» que je plaçais bien en évidence sur le tableau de bord, d’un macaron HANDICAPÉ qui me permettait d’outrepasser régulièrement mes droits, y compris, je l’avoue humblement, quand je n’étais pas spécialement en intervention. Après tout, le maintien de l’ordre est une chose essentielle si on veut espérer que la société ait une chance de survivre aux dissensions internes et autres dérèglements intestinaux responsables des flatulences qu’elle émet en permanence. J’ajoute que les handicapés, physiques et mentaux, du reste, avec tout le respect que j’ai pour eux (j’ai moi-même d’innombrables amis handicapés avec lesquels je passe d’excellents moments, à tel point que c’est tout juste si je vois la différence avec mes amis valides, pour la plupart à peine plus intéressants, même si c’est tout de même plus facile de s’adonner aux joies de l’escalade ou faire un footing en forêt de Rambouillet avec une personne valide qu’un cul-de-jatte), doivent prendre conscience que leurs problèmes personnels, aussi cruels et injustes soient-ils, je suis le premier à le reconnaître, ne doivent pas pour autant entraver l’action du bras séculier de la justice. Je suis tout à fait pour qu’on leur réserve des places ici et là, mette tout en œuvre pour leur faciliter la tâche au maximum, ne les priver d’aucunes des réjouissances auxquelles ont droit les gens normaux, mais je ne tiens pas à les avoir dans les pattes à tout bout de champ. Imaginez, par exemple, que vous êtes en train de courser un pickpocket ou un vendeur à la sauvette dans les rues de la cité, chose qui nous arrive malheureusement plus souvent qu’à notre tour à nous autres gens d’armes et de police, et qu’un handicapé vous barre la route en se traînant comme une limace en plein milieu du trottoir. Vous faites quoi ? Vous l’évitez soigneusement, au risque de laisser filer le contrevenant, ou le traitez sans discrimination, comme n’importe quel citoyen, autrement dit lui foncez dessus et le percutez violemment sans vous soucier un seul instant des conséquences ? Il y a des moments, dans l’existence, où il faut savoir faire des choix qui ne sont pas toujours agréables et faciles à assumer, et si un jour on doit me couper les deux bras ou les deux jambes (ou les deux, enfin les quatre, plus la bite et tout ce qui dépasse), eh bien j’essaierai de me faire aussi discret que possible pour ne pas emmerder le monde, tel Raymond Burr dans L’Homme de fer (pour le cas où je devrais continuer à enquêter cloué dans un fauteuil roulant, au risque que tous les criminels se foutent de ma gueule et rêvent de me pousser dans les escaliers).

Bon, blague à part (parce que je blaguais, bien sûr, vous n’avez tout de même pas cru un seul instant que j’étais à ce point dépourvu de sens civique que je n’hésitais pas à me garer sur les places réservées aux handicapés, des gens qui n’ont pas eu de chance dans la vie et méritent bien un minimum de compassion de la part de celles et ceux qui jouissent de la totalité de leurs facultés), je n’ai pas eu à user ou abuser de mes prérogatives pour réussir à me garer. Car en effet, non loin de l’entrée principale du Caribbean, un espace entre deux citadines d’entrée de gamme attendait qu’on vienne l’occuper, chose que je me suis empressé de faire séance tenante, exécutant avec maestria un créneau à montrer en boucle dans toutes les écoles de conduite. Ce que je veux dire par là, c’est que s’il existait un Nobel du créneau, récompense que malheureusement aucune autorité compétente n’a encore songé sérieusement à attribuer, les gens étant tous des abrutis qui ne voient pas plus loin que le bout de leur nez, j’aurais pu empocher les onze millions de couronnes suédoises sans la moindre difficulté, soit environ un million d’euros qui m’auraient permis de mettre un peu de beurre dans les épinards desséchés de mon ordinaire. Mais s’il arrivait, par extraordinaire, que le comité Nobel norvégien soit assez con pour accorder à Donald Trump le prix tant convoité (mais c’est avec un immense soulagement que j’apprends à l’instant même, preuve qu’il y a encore un vague semblant de justice en ce bas monde, que le prix Nobel de la Paix vient d’être décerné à Maria Corina Machado, femme politique qui tente courageusement, au péril de sa liberté et sans doute sa vie, de faire obstacle aux sordides manigances de l’ignoble Nicolas Maduro, l’actuel président du Venezuela, prêt à toutes les exactions pour se maintenir au pouvoir et continuer à s’enrichir honteusement sur le dos de ses concitoyens), s’il arrivait, disais-je, qu’une telle aberration (à peu près aussi absurde, si vous voulez mon avis, que l’idée de voir des requins tomber du ciel ou des parasites extraterrestres mal intentionnés atterrir dans une forêt de Caroline du Sud) se produise, je foncerais aussitôt à Oslo (en voiture bien sûr, au volant de ma fidèle et puissante Kangoo, et il va de soi que j’effectuerais un créneau parfait devant le numéro 51 de la rue Henrik Ibsen) afin d’exiger que le Nobel du Créneau soit créé immédiatement pour m’être remis dans la foulée.

C’est avec un ensemble parfait, au millimètre près, comme si on avait répété la scène pendant des mois sous la direction d’un des plus grands chorégraphes de tous les temps (même si je ne suis pas certain que Balanchine, Cunningham, Béjart ou Pina Bausch auraient accepté de faire la choré de deux types insignifiants et mal réveillés en train de sortir d’une Kangoo à peu près aussi glamour qu’un crapaud barbotant dans une flaque d’eau croupie), que Greg et votre serviteur, tels deux Titans nés des amours coupables d’Ouranos et Gaïa (je rappelle quand même que Gaïa est plus ou moins la mère d’Ouranos, ce qui fait de cette union l’inceste fondateur de la mythologie grecque, inceste d’où sont issus, outre les Titans, une série de monstres comme les Cyclopes et les Hécatonchires, et que c’est encore du ventre de Rhéa, fécondée par son propre frère Cronos, que sortiront les dieux de l’Olympe, somme toute une belle bande de dégénérés qui ne font pas franchement honneur à la profession), sommes sortis du véhicule sus-mentionné. Au cinéma, la scène aurait été tournée au ralenti, en slow motion, comme disent nos amis américains qui ont toujours une longueur d’avance sur tout (ou de retard, suivant l’endroit où on se trouve), et servie accompagnée d’un morceau de choix comme L’entrée des dieux au Walhalla, scène finale de L’Or du Rhin de Richard Wagner, lui-même connu pour être une assez belle ordure prête à tout pour assouvir ses plus bas instincts, à commencer coucher avec la fille de Franz Liszt, Cosima, âgée de vingt-quatre ans de moins que lui, qui est aussi, accessoirement, la femme de son meilleur ami, le pianiste, compositeur (assez peu doué il est vrai) et chef d’orchestre Hans von Bülow. Comme quoi les plus sombres histoires d’inceste et de trahison n’empêche pas le génie de s’exprimer, pour le meilleur, le pire, le pire du meilleur et le meilleur du pire.

Greg (qui s’était mis au tango depuis quelques semaines, histoire de rencontrer des femmes superbes, bien sûr, au regard de braise, à la taille de guêpe et la croupe incendiaire, mais aussi de faire un peu d’exercice pendant ses rares heures de loisir, ce qui ne serait pas du luxe car il avait pris pas mal de bide ces derniers temps), s’était offert, en plus d’une paire de chaussures de danse en cuir souple avec talons de 22 mm à absorption de chocs et semelles antidérapantes, un Bersa Thunder 380 CC, spécialement conçu pour assurer une protection discrète en toute circonstance sans pour autant renoncer à une redoutable puissance de feu. Connu aussi aussi sous le nom de 9 mm court, avec une douille de 17,3 mm (l’une des nombreuses munitions créées dans les années 1910 par le regretté John Moses Browning, un petit gars de l’Utah, membre de l’Eglise de Jésus-Christ des saints des derniers jours, qui avait un foutu sens des affaires et n’ignorait pas que l’homme est un loup pour l’homme, et que malgré tout l’amour que notre Seigneur exige qu’on lui porte il vaut quand même mieux éviter de tourner le dos à son prochain), le calibre 380 ACP vous permettra d’exploser en toute fraternité le crâne d’un individu qui aurait la mauvaise idée de s’en prendre à votre intégrité physique ou au contenu de votre portefeuille, que je vous souhaite aussi dodu et rembourré que les fesses de Kim Kardashian, Nicki Minaj, Jennifer Lopez et Kylie Jenner réunies. J’ajoute que sa poignée ergonomique en polymère texturé viendra se réfugier dans le creux de votre main tel un chaton craintif et ronronnant. Quand on sait à quel point les narcotrafiquants sont des gens qui aiment consommer local, et ont une telle foi dans la qualité de leurs produits qu’ils n’hésitent pas à les exporter aux quatre coins du monde, on comprend mieux que les membres des cartels sud-américains soient de fervents adeptes de la marque Bersa, originaire de Buenos Aires, de même que nos amis allemands ne jurent que par le Walther PPK, et italiens par le 80X Cheetah de Beretta. Dieu que les gens peuvent être chauvins !

Et chauvin, je l’étais sans doute aussi (même si peut-être pas autant que nos amis allemands et italiens, lesquels n’ont d’ailleurs pas toujours été nos amis, il faut bien le dire, voir NBPPBP, Note en Bas de Page Pas en Bas de Page), puisque c’était entre les mains expertes d’une manufacture d’armes et cycles française, sise en la bonne ville de Saint-Etienne et jadis réputée pour la robustesse et la fiabilité de ses fusils de chasse, que j’avais choisi de remettre ma vie.

\textsc{NBPPBP} : Je pense surtout à nos amis allemands qui sont allés jusqu’à s’installer chez nous sans nous demander notre avis, dans le Nord d’abord, puis l’ensemble du pays, hormis la Corse et les départements du Sud-Est réservés à Mussolini. Fils de militant socialiste révolutionnaire, puis ancien instituteur devenu dictateur sous le nom de Guide Suprême de la République Sociale Italienne (Duce en italien), ce dernier sera désavoué par le Grand Conseil (avec la bénédiction du roi qui s’irrite de son omniprésence), arrêté et emprisonné dans les Abruzzes, au Campo Imperatore. Par chance, Hitler a vent de ses emmerdements, et comme il n’est pas du genre à laisser tomber ses amis dans la débine, il appelle son vieux pote Otto Skorzeny, SS-Hauptsturmführer farouchement anticommuniste de son état, et envoie un commando des forces spéciales du tristement célèbre Sonder Lehrgang Oranienburg pour libérer Musso (ou Mumu, comme l’appellent ses rares amis ploutocrates, tous sexuellement déviants, Mumu lui-même ne cachant pas ou peu une certaine appétence pour les très jeunes filles).

De retour aux affaires, Mumu s’autoproclame Président Directeur Général en Chef de la République de Salò (ou des Salauds, suivant l’orthographe retenue), régime vaguement cryptocommuniste sous tutelle nazie qui est loin de faire l’unanimité dans le pays.

En avril 45, sous la pression des Alliés qui entendent bien récupérer les vins de Rinaldi, Conterno, Mascarello et Giacosa, la mozarella di bufala, le moliterno truffé, le jambon de Parme, la mortadelle et le guanciale, sans parler des trésors artistiques proprement dit présents dans tous les coins et recoins de cette région bénie des dieux, il décide de fuir en emportant deux mallettes dont le mystérieux contenu reste aujourd’hui encore marqué d’un point d’interrogation (peut-être de la truffe blanche d’Alba ou du Storico Ribelle de 10 ans d’âge). Arrêté par la Résistance italienne, condamné à mort par le Comité de libération nationale, il est exécuté, en même temps que sa maîtresse et âme damnée Clara Petacci, par des partisans dans une ferme des environs de Dongo, non loin du lac de Côme, endroit idyllique s’il en est mais funeste pour les amants maudits du fascisme, dont les corps sans vie finiront tristement pendus par les pieds sur la piazza Loreto de Milan (avec ceux de Nicola Bombacci, Alessandro Pavolini et Achile Starace).

La mallette n’a jamais été retrouvée, mais on se demande bien ce que le Duce trimballait avec tant d’acharnement dans son ultime cavale. La célébrité de celui ou celle qui retrouvera ce trésor, sans doute escamoté par les partisans (à moins que Musso, comprenant que tout était fini, ne l’ait enfoui quelque part avant de sombrer dans les abîmes de l’Histoire), est assurée.

Pendant des années, la Manufacture d’Armes de Saint-Etienne a été une référence dans le monde de la Mort, tant sur le plan humain qu’animal. Ses fusils de chasse, par exemple (Falcor, Robust, Simplex et Idéal, des noms qui font rêver et traduisent assez bien une certaine idée de la destruction), jouissaient d’une excellente réputation de fiabilité et solidité, et ses armes de guerre faisaient le bonheur des courageux jeunes gens qui avaient mis leur vie au service de la Nation. En échange de leurs bons et loyaux services, la Patrie reconnaissante mettait à leur disposition du matériel de qualité pour exterminer son prochain dans les meilleures conditions, avec le maximum de confort et d’efficacité. En plus de ses modèles propres, issus du génie créatif de ses ingénieurs (des artistes de la mort, virtuoses de l’homicide, disciples de Zénon d’Élée, Anaxagore et Mélissos, pour qui le tir à balle réelle était un art que l’on se devait de porter à son plus haut degré de finitude dénazifiée, l’expression la plus ontologiquement pure de la vérité au sens présocratique et phénoménologique du terme), Manufrance (nom commercial de la Manufacture d’armes et cycles de Saint-Etienne, pionnière de la vente par correspondance) restera à jamais dans les mémoires pour avoir assemblé des armes aussi légendaires que le Beretta M12, le G3 de Heckler \& Koch et le lance-roquettes antichar LRAC F1 de la Luchaire Défense SA, société anonyme au capital de 4 millions de francs. Mais pour moi, outre le Chassepot de 1866 et le canon de 75 de 1897, véritables fleurons d’une approche plus moderne, progressiste, pour ne pas dire humaniste de la guerre, le chef-d’œuvre absolu de la Manufacture d’Armes de Saint-Etienne reste incontestablement celui que j’appelle affectueusement Manu, le petit Manu, autrement dit le pistolet automatique Le Français dans sa version de poche, calibre 6.35, jouet que le commissaire Ottavioli gardera précieusement sur lui jusqu’à la fin des années 70 (avant de passer à des modèles plus consistants pour s’adapter à la puissance de feu croissante des nouvelles générations de malfrats). Je sais que les jeunes d’aujourd’hui pensent que toutes les choses qui se sont passées avant le jour de leur naissance sont les reliques nauséabondes d’une époque révolue, mais je tenais à leur faire savoir, quitte à passer pour un vieux con dépravé, un vieux débris obsolète, qui était vraiment Pierre Ottavioli. Figure du 36, c’était d’abord un homme d’honneur et un de ces flics à l’ancienne comme on n’en fait plus, à l’image d’un Charles Pellegrini, un Roger Marion, un Robert Broussard ou encore un Marcel Guillaume (le modèle du Maigret de Simenon). En ce temps là, somme toute pas si lointain, flics et voyous se donnaient la réplique dans une ambiance, sinon d’admiration réciproque et d’estime à proprement parler, au moins de respect mutuel, ce qui poussait chacun à se surpasser au profit d’une cause commune qui dépassait largement le cadre exigu de la simple individualité : celle du grand banditisme, grandeur qui ne définissait pas la violence extrême et aveugle qui s’y exerçait, mais la qualité supérieure (et parfois volontiers franchouillarde, un tantinet nationaliste, je vous le concède, au sens pur porc du terme, les expressions «~à l’ancienne~» et «~qualité supérieure~» désignant aussi bien des produits du terroir tels que le saucisson sec ou la blanquette de veau) de ses intervenants.

Voilà comment Greg Lussier, Bersa Thunder, le petit Manu et moi-même nous sommes retrouvés devant la porte du Caribbean Hôtel, une entrée majestueuse trônant au centre d’une de ces majestueuses façades à colonnades qu’un Louis Le Vau, un Robert de Cotte, un Charles Le Brun, un François Mansart, un Claude Perrault ou encore un Jacques Gondouin de Folleville, pourquoi pas (le «~bon Gondouin~», comme l’appelait Louis XV, d’abord jardinier du château de Choisy avant de se lancer dans l’architecture sous l’égide du Roi qui semblait le tenir en haute estime), auraient été fiers d’inscrire au catalogue de leurs réalisations les plus significatives, même s’ils n’avaient encore qu’une très vague idée de ce que serait un jour le style colonial dans toute sa rigidité phallique et son paternalisme débonnaire. Force, hélas, et croyez bien que je suis le premier à le déplorer, était de constater que l’édifice avait perdu une bonne partie de sa superbe. La façade, sans un ravalement d’urgence, risquait de s’effondrer à tout moment, ensevelissant au passage d’innocentes victimes dont le seul tort aurait été de se trouver là au mauvais moment. S’ensuivrait alors, avec une implacable dynamique, le cours normal des choses, en partant du postulat que la catastrophe se produirait de nuit plutôt que de jour, l’obscurité étant un excellent adjuvant de l’angoisse, la terreur et la dramaturgie : cacophonie des sirènes hurlant à tout va, féérie lumineuse des gyrophares multicolores, présence massive des forces de l’ordre pour sécuriser le périmètre et permettre aux personnels de santé de s’acquitter au mieux de leur mission, arrivée tonitruante des vautours surexcités de l’info en continu, forêt de micros tendus aux rares témoins oculaires et survivants de ce qui pourrait bien rester dans les anus comme une des pires tragédies de l’histoire de l’humanité (après le tremblement de terre de Shaanxi en 1556, l’explosion de La Valette en 1634, le naufrage du Scipion dans la baie de Samana en 1782, la catastrophe du Victoria Hall en 1883, le vol 123 de la Japan Airlines en 1985, l’effondrement du toit de la patinoire de Bad Reichenhall en 2006 et la terrible bousculade d’Antananarivo avant un concert de Paul Bert Rahasimanana~-- alias Rossy~-- en 2019), corps sans vie évacués sur des civières, membres épars récupérés ici et là et aussitôt congelés dans le vain espoir d’être un jour restitués à leurs propriétaires, badauds en pantoufles et robes de chambre prêts à vendre père et mère pour apercevoir ne serait-ce qu’une goutte de sang ou un morceau de cervelle sur la chaussée, téléphones portables en surchauffe et vidéos de l’événement faisant le tour de la planète en une fraction de seconde. Tristesse envahissante du monde, nullité cosmique et décomplexée de l’espèce humaine en voie de décomposition, sidération intersidérale, folle envie d’appeler Dieu en PCV (laid moins de vain temps ne pleuvent pas qu’au naître, homme aime titre qu’un nombre inca le cul glabre de choses dont ils mourirons singe ah mais avoir an tendu pas relais, l’orthographe, par exemple, sachant que nous aussi casserons nos pipes sans jamais avoir entendu parler d’un nombre d’autant plus incalculable de choses qu’elles ne cesseront de s’accumuler pendant les siècles et les siècles qui suivront notre mort, siècles dont j’ai malheureusement toutes les raisons de penser qu’ils ne seront peut-être pas aussi nombreux que prévu à suivre le cortège du temps) pour le questionner une nouvelle fois sur la nature exacte de ses motivations, tenter une dernière fois de comprendre par quelle aberration il s’est mis en tête de créer, à son image paraît-il (Genèse 1:26-28 LSG), ce qui n’est soit dit en passant pas très flatteur pour lui, une espèce aussi débile et dérisoire que la nôtre. Si le but était de nous faire passer par toutes les étapes de la médiocrité pour arriver enfin à quelque chose de présentable, alors on peut dire que nous n’en sommes encore qu’au tout début de notre évolution. Par contre, si le but était d’expier à travers nous quelque faute originelle qu’il aurait lui-même commise, alors on peut dire que l’objectif est entièrement atteint et qu’il serait peut-être temps de songer à mettre un terme à nos souffrances, chose que nous sommes par ailleurs tout à fait en mesure de réaliser par nos propres moyens. J’ai toujours dit, et je le maintiens avec la plus extrême vigueur, que tout ce que nous faisons et accomplissons avec tant de fierté n’est finalement qu’une maigre resucée à visage humain des œuvres de la nature, dont nous ne faisons que reproduire les faits et gestes en les adaptant à nos besoins, lesquels sont d’autant plus importants que s’impose chaque jour davantage l’évidence de notre inaptitude à vivre en harmonie avec le monde. Chacune de nos actions est soumise à l’utilisation d’une prothèse correspondante, une voiture pour rouler, un bateau pour naviguer, une fusée pour aller dans l’espace, des couverts pour manger, des verres pour boire, des armes pour tuer, des écrans pour voir ce qui se passe autour de nous, des chambres pour dormir, des salles de bain pour se laver, des sex-toys pour forniquer (même si c’est haram de s’en servir et si le guide suprême iranien Ali Khamenei s’est fendu d’une fatwa à leur encontre), des tables pour poser des trucs dessus, des chaises pour s’assoir, des chaussures pour marcher, des claviers pour écrire, des mots pour le dire, etc, etc, etc. Nous sommes tous des infirmes de naissance qu’on équipe de prothèses sans cesse plus sophistiquées pour les transformer en sportifs de haut niveau. Les performances sont remarquables, si on veut, mais le prix à payer bien trop élevé pour la majeure partie d’entre nous. En nous dotant des moyens intellectuels nécessaires pour surpasser notre condition, la nature s’est tiré une balle dans le pied. Sa légitimité a pris du plomb dans l’aile, elle a vu ses prérogatives contestées et son champ d’action se transformer lentement en peau de chagrin. Elle a pris conscience que si l’Homme en avait un jour les moyens, il n’hésiterait pas à la détruire, comme il n’hésite pas à détruire tout ce qui fait obstacle à ses ambitions démesurées. Lui, dans le même temps, s’est rendu compte que la nature ne lui était plus d’aucune utilité. Non seulement elle n’avait cessé de lui mettre des bâtons dans les roues, l’obligeant à surmonter des épreuves qui menaçaient jusqu’à la survie de son espèce, mais les rares satisfactions qu’elle lui procurait en échange ne faisaient que renforcer sa servitude et exacerber sa frustration. À défaut de vivre caché, sur une île déserte ou reclus entre les hauts murs de quelque monastère situé au sommet d’une montagne inaccessible, il lui fallait, pour vivre heureux et donner la pleine mesure de ses capacités, construire un monde à son image, entièrement artificiel, dont il pourrait contrôler le fonctionnement jusque dans ses moindres rouages. Quant à cette belle intelligence, cette conscience exceptionnelle qui lui servait soi-disant à accomplir des miracles, elle-même devait renoncer au naturel pour se lancer à corps perdu dans les délices de l’artifice. La machine, à terme (même si elle le fait déjà dans de nombreuses situations), est vouée à remplacer l’être humain, bien trop fragile et approximatif dans tous les secteurs d’activité. Car enfin, si le rêve de l’Homme a toujours été de vaincre la mort, il est évident qu’il lui faut d’abord vaincre la vie pour y parvenir. Ce n’est que lorsqu’il aura percé les mystères de l’obsolescence programmée qu’il pourra enfin s’affranchir des limites du temps. Il pourra alors vivre le cœur léger, même si ce cœur n’est qu’une machine, et envisager l’avenir sans cette épée de Damoclès de la Mort suspendue en permanence au-dessus de la tête. Comment, je vous le demande, se consacrer sereinement à une tâche si vous savez que tout peut s’arrêter d’un instant à l’autre ? La nature, en nous condamnant à vivre dans cette incertitude, dans l’urgence d’une fin aussi certaine qu’imprévisible, a fait acte de cruauté absolue. Car même si elle constitue parfois une source de motivation, cette urgence est d’abord et avant tout un instrument de torture diabolique. C’est ainsi que nous mettons au monde des enfants, que nous sommes censés aimer plus que tout au monde, en sachant pertinemment qu’il nous faudra les abandonner à leur triste sort. Il ne faut pas s’étonner que le principe même de la reproduction, véritable rouleau-compresseur de l’existence, machine à broyer du vivant qui nourrit les enfants du sang de leurs parents, soit aujourd’hui dans le collimateur des nouvelles générations. Là encore, cette fatalité qui nous condamne à engendrer notre propre succession, si elle peut sembler flatteuse à première vue, revient en réalité à signer notre arrêt de mort et entériner le fait que nous allons passer le restant de nos jours à assister au spectacle pitoyable de notre déchéance. Et personne, croyez-le bien, ne se privera de vous faire sentir à quel point vous ne servez plus à rien, si tant est que vous ayez jamais servi à quelque chose, sinon vous plier en quatre sans jamais vous plaindre pour que votre progéniture ne manque de rien. Et si un jour votre enfant vient vous trouver et vous dit qu’il n’a pas demandé à naître, répondez-lui que vous non plus n’avez rien demandé et vous seriez volontiers passé de sa présence si vous aviez su à quoi il allait ressembler. Il se peut alors, pour se venger bêtement d’une adversité qu’il ne soupçonnait pas, que, dans un geste exagérément théâtral, le rejeton en question décide de mettre fin à ses jours. Dites-vous bien, dans ce cas-là, que vous n’êtes pas davantage responsable de sa mort que vous ne l’étiez de son existence, et que toutes ces considérations oiseuses ne seront bientôt plus de mise dans le nouveau monde merveilleusement virtuel et orgasmiquement artificiel, véritable feu d’artifice de joie de vivre cybernétique sur fond de neurosciences sexy dopées à la myéline homéostatique, que nous nous proposons de créer. Plus personne, alors que tout le monde savait pertinemment que vous n’étiez pas en mesure de les assumer, ne viendra vous reprocher d’avoir failli à vos devoirs parentaux au profit de vos ambitions personnelles. Les psychologues autoproclamés de la connerie institutionnelle vous le répètent assez comme ça, se gargarisant à l’envi d’éléments de langage auxquels eux-mêmes ne comprennent pas un traître mot : vos enfants ne sont pas les vôtres, ne vous appartiennent pas, alors personne ne viendra vous demander de les fabriquer vous-même (souvent au prix d’atroces souffrances, la nature, que son principe d’économie, sinon de radinerie, pousse à entasser le maximum d’accessoires dans un minimum d’espace, n’ayant pas jugé nécessaire de doter la femme d’un orifice digne de ce nom pour mettre son enfant au monde, ce qui signifie que donner la vie a longtemps été synonyme de perdre la sienne) et vous en sentir responsable jusqu’à la fin de vos jours. La nature, qui n’a eu de cesse de nous harceler depuis des millénaires, sera domestiquée jusqu’au moindre brin d’herbe, la moindre touffe de poil, réduite au silence le plus abyssal, et il nous sera alors possible de promener virtuellement des microbes en laisse sur les trottoirs de l’infinitésimal. Et quand on en aura marre d’être éternel, que le temps aura totalement disparu et que Dieu lui-même sera venu publiquement reconnaître qu’il n’existe pas et n’a jamais existé ailleurs que notre imagination dévoyée, il nous suffira d’exercer une légère pression sur notre nombril pour mettre un terme à notre inexistence.

Le hall du Caribbean Hôtel était tellement vaste qu’un McDonnell-Douglas C-17 Globemaster III de l’US Air Force aurait pu s’y poser sans problème si la porte d’entrée avait été ne serait-ce qu’un poil plus large. De la même façon, avec une aisance comparable, un troupeau de buffles d’Afrique au grand complet aurait pu y séjourner en toute quiétude si le carrelage et la moquette avaient été remplacés par de l’herbe, n’ayant aucunement à redouter les balles des riches chasseurs occidentaux qui rêvent d’accrocher des têtes coupées de Big Five sur les murs de leurs résidences hors de prix. Et ceci pour une raison très simple : le Caribbean Hôtel était, sinon interdit aux Blancs, au moins réservé aux gens de couleur (au sens large du terme, c’est à dire que les Na’vi, Yondu, le Dr Manhattan~-- à ne bien évidemment pas confondre avec le Mr Manatane de Benoît Poelvoorde~-- et même l’ignoble Yellow Bastard de Sin City pouvaient y être admis en montrant patte blanche, au même titre qu’un Hellboy ou encore un Géant Vert, que son épiderme verdâtre et l’abominable odeur de maïs en boîte qui se dégage de sa personne ont définitivement mis au ban de la société), ce qui revenait sensiblement au même. Ce statut particulier, nettement discriminatoire, n’était finalement pas pire que d’interdire aux pauvres l’accès à nombre de manifestations culturelles ou écoles privées sous le prétexte qu’ils n’ont pas les moyens de se les offrir (ce qui entretient incidemment l’idée que ce sont tous des crétins incultes et sans avenir, proposition inacceptable même si non totalement dépourvue de fondement). On fait semblant de s’en émouvoir, d’en dénoncer l’injustice, mais la vérité c’est que la discrimination par l’argent ne choque plus personne depuis belle lurette, à commencer par les pauvres eux-mêmes qui trouvent tout naturel de claquer des fortunes pour aller applaudir des gens qui gagnent en une fraction de seconde ce qu’il leur faut des mois de labeur intensif, ingrat et notoirement sous-payé pour acquérir.

À cette heure matinale, l’endroit était désert.

Quelques vagues grooms en costume folklorique s’agitaient ici et là, sans doute pour faire croire qu’il se passait quelque chose alors qu’il ne se passait strictement rien, je peux en témoigner sur la vie de feu ma grand-mère maternelle adorée Alexandrine Chéron, née Lemaître, décédée en juin 2022 alors qu’elle survolait la cordillière des Andes à bord du Cessna Skylane 182 jaune canari qu’elle s’était offert pour son quatre-vingt-septième anniversaire. Cette femme, une femme de tête qui avait toujours placé l’indépendance au-dessus de tout et ne s’en était jamais laissé compter par tous les beaux-parleurs qui avaient croisé sa route (exception faite de mon grand-père Philibert, dont le charme ravageur, le sens aigu de la probité et les moyens financiers assez conséquents avaient eu raison de sa résistance), cette femme, disais-je, restait pour moi l’archétype absolu de l’aventurière intrépide au physique de reine de beauté. Les photos d’elle que j’avais vu quand elle n’était encore qu’une adolescente frondeuse ou une splendide jeune femme dont le regard bleu d’acier ne laissait planer aucun doute sur le caractère farouche et la soif de liberté, m’avaient fait forte impression. Aujourd’hui encore, parvenu à un certain degré de maturité dans l’existence, il m’est difficile de regarder ces photos, de qualité très médiocre pour la plupart, sans faire aussitôt l’objet de turbulences intérieures d’une violence inexplicable. Certes, c’était ma grand-mère et je l’adorais, mais de là à me mettre à chialer comme un gosse dès que je tombe sur une photo d’elle en maillot de bain, je pense qu’il y a tout de même un pas qui ne devrait pas être franchi avec une telle allégresse. Je me souviens, quand j’étais petit, qu’elle était encore un très belle femme pour son âge. Je dirais même anormalement, comme si un philtre de jeunesse éternelle la protégeait des ravages du temps. Bon, je reconnais que celui-ci avait fini par la rattraper, car à bientôt cent ans elle ressemblait quand même davantage à une vieille momie édentée qu’à une biche au teint frais comme la rosée du matin. Cela dit, au milieu des ruines subsistaient encore quelques reliques des splendeurs du passé, aussi scintillantes que des pépites dans le lit boueux d’une rivière.

Greg a dit, la voix traversée par un vieux frisson de peur ancestrale telle que l’homme n’en avait plus connu depuis que le dernier spécimen d’ours de Deninger s’est éteint dans une grotte de la sierra d’Atapuerca, près de Burgos : Je déteste cet endroit.

Ce à quoi j’ai répondu immédiatement, sans lui laisser le temps de s’enfoncer davantage dans les profondeurs sombres et humides de l’effroi : Pas moi.

C’était vrai, du reste, je le trouvais plutôt sympathique, cet endroit. Sauf peut-être la décoration, qui laissait quelque peu à désirer. En effet, le propriétaire des lieux, sans doute frappé de démence, n’avait rien trouvé de mieux à faire que de transformer l’endroit en une espèce de vieux musée de province rempli d’objets poussiéreux tout droit sortis d’un film d’horreur des années 50. Les animaux empaillés, par exemple, faisaient un peu désordre dans un établissement de cette catégorie, revendiquant un niveau de standing à priori incompatible avec la présence d’un groupe de hyènes au milieu du salon. L’hippopotame non plus n’avait rien à faire là, pas plus que les antilopes, le léopard en train de déchiqueter un phacochère (les entrailles étaient particulièrement bien imitées), ou encore le vautour qui déployait ses ailes au-dessus de la Réception, prêt à se jeter sur le client venu réclamer sa clé.

\textsc{Greg} : On se croirait dans Psychose.

\textsc{Moi} : En plus exotique.

Je sais qu’il existe, de nos jours enténébrés, des jeunes gens qui n’ont jamais vu un film en noir et blanc, ni entendu parler d’Alfred Hitchcock et encore moins de Robert Bloch.

Grand admirateur de Lovecraft avec lequel il entretient une longue relation épistolaire, Bloch est pourtant une des figures majeures de la littérature fantastique et horrifique américaine. Passionné par les histoires de monstres en tout genre, il s’intéresse de près à une certaine catégorie de prédateurs sexuels qui n’hésitent pas à tuer pour assouvir leurs fantasmes déviants. Un certain Edward Theodore Gein, par exemple, vient de défrayer la chronique. Il semblerait que le décès de sa mère, une fanatique protestante qui détestait les hommes, ait eu un effet désastreux sur sa personnalité. Après sa mort, le fiston, alors âgé de trente-neuf ans, commence à péter très sérieusement les plombs. Les gens normaux, au moins ceux qui croient en Dieu et une vie après la mort, se rendent au cimetière pour fleurir les tombes et prier pour le salut des âmes de leurs défunts. Pas lui. Quand il a constaté, malgré ses suppliques répétées et ses incantations au clair de lune, que sa mère ne semblait pas décidée à refaire surface, aux grands maux les grands remèdes, il est revenu avec sa plus belle pelle pour la sortir de terre. Il a ramené son butin à la ferme et s’est livré sur lui a des pratiques que la morale réprouve. Et le jour où il s’est lassé de son jouet, mu par des pulsions dans le détail desquelles je préfère ne pas entrer (je m’en voudrais qu’un enfant innocent, tombé par hasard sur cet ouvrage, se retrouve traumatisé par sa lecture), il est retourné au cimetière pour s’approvisionner. Que des cadavres de femmes, bien sûr, qu’il rapportait jalousement chez lui pour se fabriquer des trophées tous plus macabres les uns que les autres. Si vous aviez des envies bizarres, comme équiper votre salon avec un canapé en cuir de femme ou vous balader dans les rues de la ville avec une veste du même matériau sur le dos, c’est Eddie qu’il fallait aller voir. Nul doute, avec tous les cinglés en liberté, qu’il aurait pu se faire pas mal de fric en vendant ses créations au lieu de les garder pour lui. Mais s’il était doué pour la couture (peut-être pas autant que Paul Poiret ou Jean Patou, mais il avait son petit savoir-faire), il n’avait aucun sens du commerce et ne tenait aucunement à ce que ses activités s’ébruitent. Certains, du fait de la nature quelque peu discutable de leurs activités, sont condamnés à la clandestinité, ce qui est un moindre mal comparé aux risques qu’ils encourent. Toujours est-il que ce qui n’était au début qu’un passe-temps bien innocent, censé lui changer les idées et l’aider à supporter les affres de la solitude, est rapidement devenu une quête obsessionnelle. Il lui en fallait toujours plus, et les ressources que la nature met généreusement à notre disposition se révèlent parfois largement insuffisantes. En clair, les gens ne mouraient pas assez vite pour suivre le rythme effréné de sa créativité. Eddie, qui n’avait jusqu’ici connu que les plaisirs solitaires en essayant d’échapper au regard accusateur des crucifix disséminés un peu partout dans la baraque, venait de découvrir avec émerveillement les joies de l’amour physique avec une vraie femme parfaitement consentante et entièrement soumise à ses désirs, qu’il pouvait profaner par tous les orifices sans que sa mère, désormais transformée en abat-jour, descente de lit et autre rideau de douche, vienne le menacer des foudres de l’enfer. Il pouvait désormais faire tout ce qu’il voulait, même si les chairs faisandées qu’il malaxait avaient parfois tendance à se déliter sous ses doigts, tout comme il n’était pas toujours très agréable de fourrer sa langue dans des cavités buccales débordant d’asticots. Pour toutes ces raisons (pénurie de matière première et besoin de chaleur humaine), Eddie s’est mis à rêver de faire l’amour à des femmes encore tièdes qu’il aurait lui même choisies, au lieu de se contenter d’articles de récupération ayant depuis longtemps dépassé leur date de péremption. Il se fait la main sur Mary Hogan, de Pine Grove, dont il garde la tête pour se rappeler des bons moments passés en sa compagnie, avant de jeter son dévolu sur Bernice Worden, une sexagénaire qu’il croise régulièrement dans les rues de Plainfield et à laquelle il n’ose déclarer sa flamme, ne disposant pas du bagage technique nécessaire pour s’exprimer clairement et avoir la moindre chance de retenir l’attention d’une femme aussi belle et raffinée. En désespoir de cause, il met un terme brutal à son existence, la ramène dans sa tanière et peut enfin, à l’abri des regards et sans craindre le ridicule, lui témoigner toute l’étendue de la passion qui le consume jour et nuit. Mais quand ils s’étonnent de ne plus la voir et apprennent l’étrange disparition de Bernice, des voisins signalent à la police avoir vu à plusieurs reprises un type bizarre rôder autour de chez elle. Ce type bizarre, c’est Ed Gein, un gars du coin qui vit seul dans une ferme pourrie des environs de la ville. C’est le fils de George et Augusta Gein, le petit dernier d’une fratrie de deux garçons. Ils sont tous morts sauf lui. Il n’est plus tout à fait le même depuis le décès de sa mère, à laquelle il vouait une adoration sans bornes. C’était une maîtresse femme qui menait son monde à la baguette, une protestante rigoriste qui ne tolérait pas les écarts de conduite. Eddie avait la réputation d’un gars à la limite du handicap mental, mais toujours bien poli et prêt à rendre service. Quand les flics ont débarqué chez lui, ils ont été saisis par une pestilence telle que le contenu de leur estomac leur est aussitôt remonté dans le fond de la gorge. Ensuite, ils ont eu droit à une visite guidée du petit musée des horreurs qu’Eddie s’était aménagé à domicile. Il ne leur a pas fallu longtemps pour comprendre qu’ils venaient de tirer le gros lot.

C’est ainsi que Robert Bloch, quand il a eu vent de l’affaire, s’est attelé à la rédaction que ce qui allait devenir son roman le plus célèbre : PSYCHOSE. Alfred Hitchcock, le petit gros qui adorait foutre la trouille aux gens et fantasmer sur les créatures de rêve, blondes pour la plupart, qu’il engageait pour tourner dans ses films, lit le livre, et, en bon maniaque sexuel féru de psychanalyse qu’il est, décide aussitôt de l’adapter à l’écran. Gein n’est plus le péquenaud attardé du Wisconsin qui a fait frissonner le pays tout entier, mais le jeune et timide Norman Bates, célibataire endurci qui tient un motel pas très réjouissant en bordure de nationale. Ténébreux à souhait, posant sur les êtres et les choses un regard d’une étrange fixité, Norman prend soin de sa vieille mère malade et empaille des oiseaux pendant son temps libre. On aurait tendance à lui donner le bon dieu sans confession, mais il a aussi des petites manies qui pourraient facilement le faire passer pour un vilain garçon. Par exemple, et cela n’a rien à voir avec une quelconque passion contrariée pour le bricolage, il adore faire des trous dans les murs pour mater les clientes en petite tenue. Il faut dire que sa libido et ses conditions de vie difficiles lui occasionnent de sérieux troubles du comportement. Sévère et très à cheval sur les principes (à défaut d’autre chose), sa mère se transforme instantanément en redoutable bras armé de la justice divine pour châtier les Messaline qui viennent agiter leurs appas sous les yeux exorbités de son petit chéri d’amour. Lui-même, conscient des agissements de la vieille femme qui n’a hélas plus toute sa tête, se doit de tout mettre en œuvre pour la protéger. Il n’est donc pas rare que des visiteurs un peu trop curieux disparaissent sans laisser de trace.

Je citerai aussi, dans un style nettement plus gore et pétaradant, le cultissime Massacre à la tronçonneuse de Tobe Hooper : une famille de cannibales vit dans la nostalgie du passé, le bon vieux temps où l’abattoir du coin faisait vivre tout le comté. Ils ont un sens de la décoration assez particulier, en rapport avec leur ancienne profession de boucher. Le papy, qui doit avoir dans les cent cinquante ans et survit au grenier dans des conditions d’hygiène déplorables, n’avait pas son pareil pour assommer un bœuf à coups de marteau. Il passe le plus clair de son temps à somnoler dans son fauteuil, mais la vue d’un bon plat de tripes ou une belle andouillette suffit à le ramener à la vie et lui redonner la pêche de ses vingt ans. Cinq potes en vacances, deux filles et deux garçons plus un troisième qui n’est autre que le frère handicapé de l’une d’entre elles, débarquent dans les environs de Round Rock, au Texas. Sur la route, ils font la connaissance de divers membres de la famille, dont le père à la station-service, un type bizarre avec une tête de rat, et un des fils débiles qu’ils font l’erreur de prendre en stop. Ce dernier, qui n’a manifestement pas toute sa tête, s’intéresse de très près à Franklin, le frère handicapé de Sally Hardesty, la copine de Jerry. Quand le débile, après avoir tenu des propos inquiétants, sort un couteau pour se tailler un steak dans le cul de Franklin, les cinq gens lui demandent de prendre congé. En guise de réponse, il s’entaille le creux de la main en rigolant et menace de tuer tout le monde. Ils finissent par réussir tant bien que mal à le faire sortir, mais le débile, qui se balade avec des animaux morts dans son sac, les maudit ouvertement et imprime l’empreinte sanglante de sa main en guise de signe cabalistique sur la carrosserie du van. Pendant ce temps, les autorités alertent l’opinion sur le fait que les profanations de sépultures se multiplient dans le secteur, accompagnées d’actes blasphématoires particulièrement scandaleux et répugnants. En effet, après avoir été extraits de leurs tombes, ce qui représente déjà une grave atteinte à leur intégrité, les cadavres sont ensuite installés comme des épouvantails dans le cimetière. Seuls des gens souffrant de graves troubles mentaux, sans doute membres d’une secte s’adonnant à des orgies sexuelles et des sacrifices de nourrissons, sont capables de telles horreurs. Les cinq jeunes gens, partis à la recherche de la maison où Sally et Franklin ont passé une partie de leur enfance, finissent par mettre la main dessus. Grande est leur déception quand ils constatent qu’il n’en reste que des ruines. Mais quand on est jeune, beau, qu’on a toute la vie devant soi, qu’on n’a pas un gramme de graisse sur le corps, qu’on a des tablettes de chocolat quand on est un garçon et un petit cul rond comme un ballon moulé dans un micro-short en jean quand on est une fille, il faut un peu plus que ce genre de contretemps pour entamer votre bonne humeur. C’est ainsi que Kirk et Pam, l’autre sympathique petit couple de notre fine équipe de randonneurs estudiantins, décident d’aller faire trempette dans le doux ruisseau clapotant aperçu en cours de route. Ce faisant, ils aperçoivent, au détour d’un sentier tortueux, une propriété isolée dont le moins qu’on puisse dire est qu’elle n’inspire pas franchement confiance. Ils l’ignorent encore, et nous aussi même si on commence à entrevoir la sinistre vérité à travers le voile trompeur de la joie de vivre et l’insouciance, mais cette bicoque délabrée n’a rien de la petite maison dans la prairie. On a tous en tête cet endroit idyllique où des gens charmants vous accueillent à bras ouverts comme si vous faisiez partie de la famille depuis toujours. Vous tombez en panne en rase campagne, frappez à la porte, une mère de famille sexuellement attirante (en anglais Mother I’d Like to Fuck, NDLR) vous ouvre la porte, les cheveux en bataille et le décolleté largement ouvert sur des perspectives vertigineuses dignes de la Vallée des plaisirs, et vous offre aussitôt une énorme part de tarte à la citrouille que vous n’avez pas intérêt à refuser si vous ne voulez pas qu’elle vous arrache les yeux avec ses ongles de trente centimètres de long taillés en pointe. Vous vous dites «~Pourquoi moi, pourquoi tant de bonheur, est-ce que la chance serait enfin en train de tourner en ma faveur ?~», et elle vous annonce que le type encadré dans le photo, celui-là même que vous regardez avec insistance, les yeux remplis de crainte parce qu’il a l’air d’une brute épaisse et que vous ne pouvez vous empêcher de penser qu’il ne sera peut-être pas enchanté de vous trouver en compagnie de sa femme en rentrant d’une dure journée de labeur, elle vous annonce qu’il est décédé l’année dernière, pulvérisé par un poids lourd sur l’interstate 35, une des routes les plus meurtrières de tout le Texas. Vous aimeriez sauter de joie jusqu’au plafond, sombrer sans retenue dans la plus infâme concupiscence, mais, Dieu merci, la bienséance et la volonté de ne pas passer pour un sale con et un gros porc lubrique vous empêchent de décoller. Quoi qu’il en soit, il semblerait en effet que la chance soit en train de tourner en votre faveur. Et attendez, ce n’est pas tout. Vous lui posez prudemment la question de savoir s’il elle lui a trouvé un remplaçant, et elle vous répond, tout en posant sur vous un regard tropical qui vous transforme le bas-ventre en fourmilière, que non, elle n’a pas la tête à ça, que c’est encore trop tôt pour songer à la bagatelle, alors que des vieux bouts de vêtements et des lambeaux de chair pourrie s’accrochent encore aux os de ce pauvre Edward qui s’agite dans le fond de sa tombe. Bien sûr, vous ne comprenez que trop bien, il faut laisser du temps au temps, répondez-vous en baissant pudiquement les yeux sur les profondeurs de son décolleté et vous fustigeant intérieurement de l’absence totale de moralité qui vous caractérise, ajoutant d’une voix de fausset qu’il commence à se faire tard et que vous n’allez pas tarder à prendre congé, même si vous n’avez nulle part où aller et courez le risque d’atterrir dans un motel pourri tenu un jeune homme féru de taxidermie. Que nenni, s’insurge-t-elle en projetant sur vous une onde de choc parfumée qui vous transporte au septième ciel, ce n’est pas parce que ce pauvre Edward est mort dans des circonstances tragiques qu’elle doit pour autant renoncer aux valeurs de charité chrétienne qui ont toujours été les siennes. Vous trépignez de joie intérieurement en entendant ces paroles de réconfort, aussi douces qu’un tapis de mousse sous vos pieds nus. Un bonheur n’arrive jamais seul, écrivait Marie de Rabutin-Chantal, marquise de Sévigné, dans une lettre à sa fille datée du 20 novembre 1676, et vous n’allez pas tarder à en faire l’expérience. En effet, la porte s’ouvre et une jeune fille, plus belle que le jour et la nuit réunis, fait sont entrée dans la pièce. Et cette charmante jeune fille, sommairement emballée dans une robe à fleurs qui ne cache pas grand-chose de son anatomie dévastatrice, c’est tout simplement Candy, la fille de Janet, la jeune femme qui a eu la gentillesse de vous offrir l’hospitalité en plus d’une généreuse part de tarte à la citrouille (vous détestez ça mais vous forcez quand même à l’avaler jusqu’à la dernière miette, conscient que cette mise en bouche pourrait bien constituer les prémisses d’un festin autrement réjouissant).

Voilà comment les choses pourraient se passer si nous vivions dans un monde meilleur, et surtout si quelqu’un d’autre que Tobe Hooper avait écrit et réalisé le film. Dans le cas présent, vous le savez parce que vous avez déjà vu le film ou n’êtes tout simplement pas tombé de la dernière averse, la résidence en question, à peu près aussi accueillante qu’une décharge à ciel ouvert infestée de rats et de cafards gros comme des rats, n’est autre que celle des cannibales de service, la famille Sawyer. Kirk va être le premier à y passer, suivi de près par Pam qui finit dans le congélo après avoir été empalée sur un croc de boucher. Je vous passe les détails, mais je vous promets que la scène ne manque pas de piquant. La curiosité est un vilain défaut, et si vous en doutez encore, faites confiance à Bubba Sawyer pour vous en faire la démonstration aussi sonore que tranchante. Géant consanguin doté d’un cerveau de batracien, Bubba sait qu’il

est moche comme un pou trisomique et ne supporte pas son reflet dans un miroir. Il porte un masque en peau humaine, arraché au crâne de l’une de ses nombreuses victimes. Bûcheron dans l’âme, il découpe les voyageurs de passage à la tronçonneuse avant de les refiler à son frère Drayton, le cuisinier de la famille, qui se charge de les transformer en fromage de tête (bon, ça, le fromage de tête, avec des pickles, des cornichons et un bon verre de vin blanc sec) et chili con carne (un des meilleurs de la région, si l’on en croit les voyageurs qui ont eu le privilège de le déguster, et surtout de survivre à la dégustation). Après Kirk et Pam, c’est au tour de Jerry, le petit copain de Sally parti à leur recherche, de se faire défoncer la gueule à coups de marteau par Bubba Sawyer, plus connu dans le milieu SM sous le nom de Leatherface. Sally, restée avec son frère handicapé pour veiller sur lui, n’a aucune nouvelle de ses amis et voit la nuit tomber avec une anxiété grandissante. Je suppose que Hooper, quand il a écrit le scénario, s’est dit que ce serait sympa de voir un type en fauteuil roulant se faire découper à la tronçonneuse par un géant demeuré avec un masque de cuir sanguinolent sur le visage. S’en prendre avec une telle violence à une créature sans défense ne pourrait que faire date dans l’histoire du septième art et du film d’horreur en particulier, genre mineur que ses plus ardents défenseurs considèrent comme le nec plus ultra de la création artistique. Il a également compris que le film serait encore plus crédible si l’on insinuait perfidement qu’il n’était en aucun cas une fable horrifique issue du cerveau malade d’un réalisateur indépendant plus ou moins détraqué, mais la relation fidèle et sans complaisance d’événements s’étant réellement déroulés quelque part au fin fond du Texas. Oui, braves gens, tandis que vous ronflez paisiblement dans vos pavillons de banlieue, bien à l’abri derrière vos portes blindées et vos système d’alarme dernier cri, le 357 Magnum sous l’oreiller, il existe encore des endroits reculés où la sauvagerie la plus extrême s’exerce en toute liberté. Pendant que vos sirotez vos cocktails au bord de vos piscines à quatre-vingt mille dollars, insensibles à la misère du monde, des jeunes gens dans la fleur de l’âge se font découper à la tronçonneuse par des mongoliens de deux mètres de haut au visage masqué. Pensez-y, quand vous vous gaverez de pop corn devant la télé, et n’oubliez pas de vous renseigner à deux fois avant de partir à l’aventure. Vous vous croyez plus malin que les autres, mais pourriez vous aussi vous exposer à de très graves dangers. Et vous, parents, tremblez quand vos enfants décident de partir en vacances à l’autre bout du monde, certains que la fraîcheur de la jeunesse leur ouvrira toutes les portes et les préservera du mauvais sort. Elle pourrait tout aussi bien leur ouvrir grand les portes de l’enfer.

Dans la même veine, je me ferai également une joie de mentionner le très glauque et méconnu Maniac de William Lustig : Franck Zito ne fait pas partie de ces gens qui ont eu la chance de vivre dans un foyer harmonieux, avec des parents aimants qui ne passent pas leur temps à picoler et s’engueuler, et élèvent leurs enfants dans le respect des saines valeurs du travail et l’amour du prochain. Non. Quand le père de Franckie n’était pas en train de s’arsouiller au bistrot, il rentrait à la maison pour tabasser sa femme et ses gosses. Un jour pas fait comme un autre, il s’est barré et on ne l’a jamais revu. Quant à la mère de Franckie, elle ne valait guère mieux que son père. Elle-même largement alcoolique et dépendante de nombreuses substances prohibées, elle se vendait au plus offrant pour arrondir ses fins de mois. Contrairement à d’autres prostituées dans son genre, elle bossait à domicile et le petit Franckie assistait fréquemment aux ébats de sa mère et ses amants de passage, une «~clientèle~» dont je vous laisse imaginer l’élégance naturelle et la distinction. En dépit de son comportement inadapté et totalement dépourvu de chaleur humaine, Franckie avait élevé sa mère au rang d’idole absolue et rêvait du jour où les gros porcs qui lui passaient sur le corps trouveraient leur juste châtiment. Ce n’était pas elle la coupable, mais ces ordures qui profitaient de sa faiblesse pour assouvir leurs coupables penchants. Et lui aussi était coupable de ne pas pouvoir la protéger. Sa mère reçoit des hommes chez elle, mais tous n’ont pas les mêmes intentions. Certains, par exemple, s’intéressent davantage à Franckie qu’à sa mère. Ils vont même jusqu’à faire des offres très alléchantes pour s’octroyer le droit de faire de lui ce que bon leur semble. Sa mère, toujours à court de fric pour payer ses doses, explique à Franckie qu’il va devoir faire à ces messieurs certaines des choses qu’elle-même fait avec eux, notamment avec sa bouche. Franckie hésite, mais sa mère insiste en disant qu’ils risquent de lui faire du mal s’il ne le fait pas. Il ne voudrait pas que les vilains messieurs fassent du mal à sa maman, n’est-ce pas ? S’il est bien sage, maman lui fera un gros câlin quand ils seront partis. Des années plus tard, longtemps après la mort de sa mère, Franck a développé de sérieux troubles de la personnalité. En clair, il est complètement barré, incapable de se comporter normalement en société. Comme Hugh Hefer, il vit entouré de mannequins. Sauf qu’il ne vit pas à Beverly Hills, dans un manoir de deux mille mètres-carrés avec (entre autres) salle de sport, piscine, tennis, zoo privé et parc paysager avec grotte comme dans les contes de fées, et que ses amis ne s’appellent pas John Lennon, Elvis Presley, Alec Baldwin, Sylvester Stallone, Pamela Anderson, Cameron Diaz, Leonardo DiCaprio ou Kim Kardashian. Oh non. Franckie n’a aucun ami, sa vie est loin d’être un conte de fées, et ses mannequins à lui n’ont rien à voir avec les créatures plantureuses au sourire éclatant qui s’agglutinent comme des mouches autour du patron de Playboy. Ses mannequins à lui ne marchent pas, ne parlent pas, ne respirent pas. Ils ne sont pas morts, non, c’est juste qu’ils n’ont jamais été en vie, même s’ils ont toutes les apparences de la réalité. Ce sont juste des objets, des poupées à taille humaine qu’on habille pour exposer dans les boutiques de fringues. Plastiquement parfaites, certes, mais surtout parfaitement en plastique. C’est précisément cette inertie, alliée à cette troublante ressemblance, qui peut donner des idées à certains individus sexuellement perturbés. Et même à d’autres, qui n’ont à priori rien de commun avec les pervers en question, mais sont juste fatigués d’avoir à composer avec les exigences d’un ou une partenaire en chair et en os (vous le voyez, j’essaie d’être aussi inclusif que possible, conscient que les fanatiques de la non-binarité intersexuelle et transidentitaire seraient trop contents de planter ma tête au bout d’une pique au moindre faux pas). Pour beaucoup d’hommes, confrontés à la difficulté croissante de vivre en couple (depuis, pour faire court, que la femme n’est plus cette petite chose fragile et soumise qu’on pouvait engrosser à loisir et asservir en toute tranquillité), la femme parfaite serait une poupée sexuelle ou un androïde hyperréaliste offrant tous les avantages d’un être humain sans en avoir les inconvénients. Notons, au passage, qu’une telle évolution des mœurs éviterait bon nombre de féminicides perpétrés par des conjoints alcooliques et violents. Car si un être vivant ne vous appartient pas, il n’en va pas de même pour un humanoïde en silicone. Même les prédateurs sexuels les plus abjects, comme l’ont prouvé les récentes affaires de ventes de poupées pédopornographiques sur Shein, AliExpress et Amazon, pourraient y trouver leur compte. Non seulement de nombreux enfants ne seraient plus traumatisés par leurs agissements, et des familles brisées dans la foulée, mais de nombreuses vies seraient épargnées, sachant que les pires d’entre eux n’hésitent pas à tuer pour s’assurer du silence de leurs victimes. Sur le plan de la déontologie, je reconnais que la pilule est assez difficile à avaler. C’est déjà assez moche de savoir qu’on vit entouré de porcs qui ne songent qu’à abuser de nos enfants (certains vous diront que c’est encore plus moche de ne pas savoir qui ils sont, et ils n’auront sans doute pas complètement tort), on ne va pas en plus leur fournir de quoi assouvir leurs fantasmes en toute légalité. Cela reviendrait, en quelque sorte, à officialiser leur existence et accepter de vivre avec eux en bonne intelligence. C’est d’autant plus difficile à imaginer qu’on sait pertinemment qu’il y aura toujours des brebis encore plus galeuses que les autres pour s’affranchir des règles en vigueur, jetant le discrédit sur toute la communauté de gentils pédophiles respectueux qui se contentent de tripoter des poupées en silicone.

Lustig, après de rapides débuts dans le porno, s’est tourné vers le cinéma de genre, le psycho-killer movie en l’occurrence. Un beau jour de l’année 1979, il est allé trouver son pote Joe Spinell, abonné aux seconds rôles de flics véreux et truands sans envergure, pour lui proposer de se glisser dans la défroque peu avenante de Franck Zito, propriétaire d’une boutique de mannequins et tueur de femmes psychotique obsédé par le souvenir de sa mère prostituée. Spinell, qui n’avait sans doute rien de mieux à foutre, a accepté, et s’est impliqué dans le film au point de participer à l’écriture du scénario. Vivre avec des femmes en plastique, c’est bien, mais ça ne suffit pas à faire un film d’horreur digne de ce nom. Pas mal de trucs ont déjà été faits, il va falloir trouver quelque chose d’un peu original si on veut avoir une chance de sortir du lot et décrocher un Oscar. À l’époque, la plupart des tueurs écumaient les magasins de bricolage pour s’équiper. Agrafeuse, burin, clé à molette, disqueuse, faucille, fourche, hache, machette, marteau, masse, meuleuse, pelle, perceuse, pied à coulisse, pioche, pistolet à clous, poinçon, tournevis, tronçonneuse, scie sauteuse, spatule, tout y passait dans la joie et la bonne humeur. Si on avait pu tuer quelqu’un à coups de tire-bouchon, ouvre-boîte ou lime à ongles, je vous assure qu’on ne se serait pas gêné pour le faire. J’estime toutefois que le couteau, qui peut sembler désuet, sinon vulgaire à première vue, reste une valeur sûre, à condition bien évidemment qu’il soit de taille conséquente et manipulé par quelqu’un qui connaît son affaire. Et c’est exactement ce que se sont dit nos deux compères : Zito, qui n’a aucun goût particulier pour le bricolage, va se servir d’un bon vieux couteau de chasse des familles. Et à quoi peut bien servir un bon vieux couteau de chasse des familles, hein, je vous le demande ? Eh bien à dépecer des animaux, par exemple, ou tailler des morceaux de bois pour en faire des armes redoutables, comme Stallone dans Rambo. Mais Zito crèche à New York, pas dans la jungle, et ce n’est pas dans les rues de la Grosse Pomme qu’on risque de croiser une biche ou tapir. Il fallait donc trouver quelque chose de plus en rapport avec le profil de sociopathe schizophrène de l’intéressé.

Spinell, avec sa gueule de type qu’on n’avait pas envie de croiser de nuit au détour d’une ruelle sombre et humide, au pavé glissant et aux trottoirs envahis de sacs poubelles éventrés, était un grand fan de western. Comme vous le savez, les westerns sont ces films qui évoquent les différents aspects de la conquête de l’Ouest. Les Rosbifs ont débuté avec treize colonies sur la côte Est, tandis que les Français étaient plutôt bien installés au centre du pays. Ils s’étaient même fait quelques potes indiens pas trop regardants avec lesquels ils commerçaient pour tenter s’intégrer harmonieusement dans le paysage, même s’il était clair que ça risquait de péter un jour ou l’autre. Plus au Sud, on trouvait les Espagnols qui continuaient à creuser des trous un peu partout pour trouver de l’or. Il y avait aussi quelques Hollandais qui gravitaient dans le secteur, arrivé dans le sillage de Peter Stuyvesant, mais ils ne faudrait pas longtemps pour le rejeter à la mer en cas de conflit. On trouvait un peu de tout, en fait, mais pas en quantité suffisante pour représenter une réelle menace. Tous ces gens bricolaient dans l’espoir de tirer un jour leur épingle du jeu. Un jour, les treize colonies en ont eu marre des exigences de la Couronne et décidé de s’affranchir une bonne fois pour toutes de ses directives. Ils avaient pris tous les risques et entendaient bien créer un nouvel empire dont ils seraient les seuls maîtres après Dieu. La Mère Patrie resterait à tout jamais dans le fond de leur cœur, bien entendu, tatouée à l’encre rouge de la conquête normande et la révolte des barons, mais ils avaient aujourd’hui d’autres chattes à fouetter, et non des moindres. Dans la foulée, ils ont viré les Français et le reste, laissant les Espagnols se dépatouiller avec l’Amérique du Sud. Je vous la fais courte, historiquement approximative, mais c’est grosso modo comme ça que les choses se sont passées. Comme ils n’occupaient qu’une toute petite partie du très vaste territoire sur lequel ils venaient de poser le pied, ils ont repris leur bâton de pèlerin et décidé d’aller voir ce qui se passait un peu plus loin. Ils n’ont pas tardé à se rendre compte que, en dehors des colons blancs et esclaves noirs qu’ils avaient amenés dans leurs valises, il existait une race d’autochtones dont il n’allait peut-être pas être aussi facile de se débarrasser. Même s’il s’agissait de sauvages peinturlurés tels qu’on en avait maintes fois croisés dans les expéditions passées, ils risquaient de sérieusement entraver la marche triomphale de la civilisation occidentale vers l’avenir radieux du libéralisme économique. Dès qu’un brave type de colon sans histoire s’installait quelque part, tranquille, avec sa petite famille et ses quelques têtes de bétail qui broutaient paisiblement dans le champ voisin, ces abrutis lui tombaient dessus en poussant des hurlements stridents et saccageaient tout sur leur passage, allant jusqu’à violer les femmes et enlever les enfants pour les élever à la sauce indienne. Déjà qu’il fallait se coltiner les hors-la-loi qui foutaient la merde dans les saloons, se battaient en duel à tous les coins de rues et détroussaient les voyageurs (il leur arrivait aussi de violer les femmes), ça allait devenir difficile de faire son beurre s’il fallait avoir en permanence cette bande d’emplumés sur le dos. La tâche était d’autant plus ardue qu’ils n’étaient pas constitués en une seule armée, qu’on aurait pu vaincre en une seule fois, mais une multitude de tribus éparpillées aux quatre coins du pays. L’avantage, c’était que les tribus en question n’étaient pas toujours en très bons termes, ce qui permettait, avant bien sûr de se retourner contre eux, de pactiser avec les uns pour affaiblir les autres. Et comme c’était des sauvages, c’est à dire des gens qui vivaient sous des huttes en peau de lapin, se trimballaient les couilles, lâchaient des caisses dans la plus totale insouciance, fumaient des pipes de trois mètres de long et ne disposaient que d’un armement rudimentaire pour faire valoir leurs droits à la terre de leurs ancêtres, il ne devrait pas être trop difficile de leur faire fermer leur grande gueule une bonne fois pour toutes. Car dis-toi bien ceci, ô vil suppôt de l’impérialisme américain dont l’existence même représente un grave danger pour la survie de l’espèce à laquelle j’ai la malchance d’appartenir : il existait, dans les antiques contrées de la lointaine Amérique, des kyrielles de tribus qui se partageaient les vastes plaines de l’Ouest, du Texas à l’Etat de Washington en passant par la Lorraine, l’Arizona, l’Idaho, le Colorado, le Montana, l’Iowa, le Nouveau-Mexique, le Nevada, l’Oregon, l’Utah, l’Arkansas, le Kansas tout court, le Wyoming et l’entrée de service. En plus ou moins bonne intelligence, il est vrai, car il leur arrivait fréquemment de se taper sur la gueule comme des gamins de cinq ans qui se disputent un jouet. Ces grands enfants avaient de longues conversations avec la lune, les pierres et les ruisseaux, dansaient avec les loups, et respectaient chaque chose, même la plus insignifiante, comme un membre à part entière de leur famille. Chaque année, ils attendaient le retour des bisons pour remplir les frigos, faire le plein de peau et viande séchée, et les bisons étaient tellement nombreux qu’il y en avait largement assez pour tout le monde. Leur vie s’écoulait ainsi, au gré du vent et des saisons. Ils n’en demandaient pas davantage et pensaient que les choses dureraient ainsi jusqu’à la fin des temps. Quand il a vu débarquer l’homme blanc, avec ses chemises carreaux, ses jeans et ses santiags, l’Indien a tout de suite compris que les jours heureux étaient terminés. Il faut dire que l’homme blanc s’est vite montré envahissant, totalement irrespectueux des usages, coutumes et règles en vigueur, arrogant, violent, dominateur, certain de la supériorité de son espèce, se comportant comme le roi du monde en pays conquis. Il a bien essayé de résister, mais son matos de fortune ne faisait pas le poids face à la puissance de feu de l’envahisseur. Le temps de décocher une flèche et il s’était déjà pris trois balles dans le buffet. La lutte était inégale et s’est achevée comme elle devait s’achever : par la victoire écrasante du pot de fer contre le pot de terre, sa prise de contrôle totale du territoire et sa relégation aux archives de la vieille nation indienne, réduite à survivre sur les lopins de terre indignes généreusement alloués par le gouvernement fédéral des Etats Unis d’Amérique. Le Visage Pâle est arrivé («~et je vis un cheval fauve, et celui qui était monté dessus avait nom la Mort, et l’Enfer suivait après lui~», Apocalypse 6:8) et il a dit : vous êtes ici chez moi, faites vos valises et foutez le camp. Fini les chants et les danses au clair de lune au son du tambour, les discussions interminables au coin du feu et la chasse au bison dans les vertes prairies du Wyoming et de l’Oklahoma. On est venu ici pour s’établir, entre gens civilisés, on n’a pas besoin de sauvages pour tenir la chandelle. On va faire des trous partout pour trouver de l’or et du pétrole. Ouais, avec des grosses machines qui fument et font du bruit. Grâce à ça on va devenir très riche et on va transformer vos terrains vagues en cités radieuses et tentaculaires. On va aussi construire des lignes de chemin de fer pour transporter nos marchandises d’un bout à l’autre du pays, des grosses bagnoles pour mettre plein d’essence dedans et une route pour aller de Chicago à Santa Monica. On va vous la mettre bien profond et vous ne pourrez rien faire pour nous en empêcher. Car la terre n’appartient à personne, même si les membres d’une même famille y sont enterrés depuis des générations, et comme elle n’appartient à personne, elle appartient à tout le monde, et donc à moi en particulier. Tu ne croyais tout de même pas que l’état de grâce allait durer éternellement, que t’allais continuer à te la couler douce sous ton tipi, caracoler dans les Rocheuses et fumer le calumet de la paix pendant que nous, fleurons de l’espèce humaine et garants de la civilisation, on allait rester sans rien dire coincé sur un rocher de l’autre côté de l’Atlantique ? Non, mon gars, on n’a pas fait tout ce chemin pour enfiler des perles et chanter des cantiques. Donc, je te le dis tout net, ce qui va se passer maintenant, c’est qu’on va te parquer dans des endroits pourris où même les rats et les cafards ne voudraient loger pour rien au monde. Mais comme on n’est pas des monstres, on va te filer du tord-boyau à volonté pour que tu puisses passer tes journées à picoler pour oublier que tu t’es fait entuber dans les grandes largeurs. Des questions ? Quoi ? Si nous aussi on picole ? Bien sûr, qu’on picole, mais nous on a de bonnes raisons pour le faire. Autre chose ? Non ? Dans ce cas ramassez vos cliques, vos claques, et foutez-moi le camp d’ici ! L’aigle s’est envolé, le coyote s’est tu, le chien de prairie a regagné sa tanière, le loup ne chante plus sous la lune (non, il n’a plus envie depuis que les Indiens sont partis), mais soyez au moins certains d’une chose, farouches guerriers des sauvages plaines de l’Ouest, c’est que vos noms resteront à jamais gravés dans nos mémoires, la mienne et celle de tous ceux qui croient encore en la justice et la liberté (rires) : Cheyennes, Sioux, Pawnees, Kiowas, Crows, Shoshones, Arapahos, Comanches, Cherokees, Apaches, Têtes Plates, Nez Percés, Pieds Noirs, Culs d’Oursins et Sacs à Puces. Vos âmes pures chevauchent désormais dans les grandes plaines de l’au-delà aux côtés de Wakan Tanka, Pah, Shakouroun, du corbeau, du vieil homme et du grand lièvre. Amen, tchuss, Allahu akbar et gode save the Gouine !

Spinell (je rappelle à toutes fins utiles qu’on est en train de parler de Maniac, le film de Lustig) n’avait pas fait les années d’études nécessaires à l’obtention d’un diplôme d’historien en bonnet difforme (ce qui n’est pas mon cas non plus, je vous rassure tout de suite, pas plus que celui de Lustig ou Caroline Munro, l’actrice principale du film, et je ne parle même pas de Gail Lawrence, Kelly Piper et Sharon Mitchell), mais il avait vu suffisamment de westerns pour savoir à quoi s’en tenir au sujet des Amérindiens. Et il adorait ça, les Amérindiens. Lui-même, d’ailleurs, avec sa sale gueule de psychopathe au look ringard et aux yeux exorbités, aurait très bien pu tourner dans un western horrifique du genre The Wind, Dead in Tombstone ou The Burrowers, ou encore, dans un style plus classique, l’intemporel Jesse James contre Frankenstein. Que celles et ceux qui ne l’ont pas vu au moins une bonne douzaine de fois soient immédiatement traduits en justice et bannis à tout jamais des salles de cinéma de France et de Navarre ! En l’an de grâce 1966, Jesse James contre Frankenstein et Billy the Kid contre Dracula sortent coup sur coup. L’émotion est grande (nulle), la foule (ne) se presse (pas), la critique se déchaîne (l’ignore totalement). On doit cette prouesse technique à William Beaudine, génie méconnu dont personne n’a jamais entendu parler et tout le monde se contrefout éperdument. Baudine vient du cinéma muet, où il avait l’habitude d’enchaîner les films par treize à la douzaine. Je considère pour ma part que la réputation de films comme La Parade du rire, Le Retour de Philo Vance et l’inénarrable Gorille de Brooklyn mériterait d’être revue à la hausse, n’en déplaise aux amateurs des frères Dardenne, Xavier Dolan et Bruno Dumont. On me murmure dans l’oreillette qu’il avait prévu d’enchaîner avec Doc Holliday contre la Momie et Calamity Jane contre Jack l’Eventreur, films qu’il n’a malheureusement pas eu le temps de tourner pour des raisons de santé. Il va de soi que je me ferais un plaisir de le faire moi-même si j’avais un peu de temps et d’argent devant moi. J’appellerais l’agent de Scarlett Johansson et je lui dirais : j’ai un rôle pour Scarlett. Lui : Ah bon ? Quelle heureuse nouvelle ! Moi : Oui, le vais tourner Calamity Jane contre Jack l’Eventreur avec mon téléphone portable et j’ai pensé à elle pour le rôle principal. Je vais aussi écrire la musique du film et jouer le rôle de Jack l’Eventreur. Lui : Pas con. Moi : Et tu sais quoi ? Lui : Non. Moi : J’ai pensé à Scarlett pour le rôle de Calamity Jane. Lui : Excellente idée, je vais voir si elle est disponible. Je te rappelle dans une heure. Moi : Okay, pas de problème. Il faut savoir que lui et moi on est comme cul et chemise, même si c’est généralement plutôt moi qui fait le cul et lui la chemise. Lui, une heure plus tard comme convenu : Désolé, vieux, mais ça ne va pas être possible. Moi : Non ? Lui : Si. Elle doit tourner dans un remake de L’Exorciste signé Mike Flanagan. Moi : Elle doit jouer quoi ? Lui : On ne sait encore pas trop. Peut-être Regan MacNeil, devenue une séduisante jeune femme, qui doit à nouveau faire face à des démons bien décidés à lui faire dire des horreurs et tourner la tête à 360 degrés. Ou alors Linda, la fille que le père Karras et Sharon Spencer, la secrétaire de Chris MacNeil, ont eu en secret, et qui se transforme en succube sous l’influence de Lamashtu, une divinité summérienne qui n’est autre que la propre femme de Pazuzu. Moi, déçu : Bon, tant pis. Je vais voir si Leven Rambin est disponible. Lui : La fille du Dr Sloan dans Grey’s Anatomy ? Moi : Oui, je l’ai trouvée très bonne dans Mank de David Fincher. Lui : Elle a joué dedans ? Moi : Oui, on l’aperçoit dans une scène ou deux. Elle crève l’écran, je trouve. Lui : Mmmmouais, je suis pas sûr. Prends plutôt Amanda Seyfried, à ce moment-là. Ou Sydney Sweeney (NDLR : recordwoman du cri le plus long et l’infanticide le plus spectaculaire dans Immaculate de Michael Mohan). Moi : OK, merci, c’est pas grave. De toute façon, je ne pense pas que Scarlett aurait accepté de jouer gratos dans mon film. Lui : Je ne pense pas non plus. Mais on ne sait jamais, Scarlett est une jeune femme tout à fait imprévisible.

Bon sang, je n’en reviens pas qu’on ait pu passer en une fraction de seconde des Sioux à Sydney Sweeney, laquelle, soit dit en passant, pourrait se révéler tout à fait attractive dans un western gore où elle porterait des jeans American Eagle et jouerait la fille d’un pionnier enlevée par des Indiens cannibales. D’ailleurs, je suis à peu près certain que si Franck Zito avait croisé Sydney Sweeney dans les rues de New York, il aurait aussitôt fait une fixette sur elle et rêvé jour et nuit de lui offrir une petite coupe. Et quand je parle de petite coupe, il ne s’agit bien évidemment pas d’une petite coupe de Champagne, millésimé ou non, mais d’une coupe de cheveux. D’après Hérodote, les Scythes, farouches guerriers venus des steppes pontiques d’Anatolie, avaient pour habitude de boire le sang de leurs victimes. Grand bien leur fasse, il ne faisait pas toujours très chaud dans le secteur, les combats étaient souvent longs et éprouvants, on peut comprendre qu’ils aient eu besoin de se détendre un peu après l’effort. C’est le genre de situation où on est bien content de boire un petit verre de vin chaud, par exemple. Comme ils n’en avaient pas sous la main, une bonne rasade de sang frais faisait l’affaire. Ils auraient pu s’en tenir là, mais non. Après avoir étanché leur soif, ils tranchaient la tête de leur ennemi pour la ramener au chef qui les félicitait en leur donnant de grandes tapes dans le dos. Les rires et les chants fusaient autour du feu qui crépitait gaiment sous la voûte étoilée, dans une saine ambiance de camaraderie où tous les excès étaient autorisés. Ensuite, les femmes et les enfants étaient violés par les participants qui, il faut bien le dire, puaient méchamment de la gueule à force de boire du sang et s’empiffrer de viande crue. En guise de souvenir, il était d’usage de conserver le cuir chevelu de la victime. Une fois débarrassée de ses impuretés, la chose était utilisée en tant que serviette pour la table, ce qui ne manquait quand même pas d’une certaine classe. Et quand on était un valeureux guerrier et qu’on en avait accumulé plein au cours de ses nombreuses campagnes, il était de bon ton de les coudre ensemble pour s’en fabriquer des vêtements. Un petit gilet bien seyant, par exemple, ou une paire de moufles. Ed Gein n’aurait pas fait mieux, sauf que lui préférait déterrer des cadavres plutôt que risquer de prendre un mauvais coup sur un champ de bataille. Zito, pour sa part, se contentait plus modestement de scalper des filles pour habiller le crâne de ses mannequins, leur donner un peu de cette humanité qui leur faisait cruellement défaut.

Et enfin, last but not least comme disent nos amis anglo-saxons, incontestables spécialistes du genre, l’inoubliable Silence des agneaux de Jonathan Demme (d’après le bouquin de Thomas Harris, journaliste et écrivain de seconde zone qui a eu la riche idée de transformer un psychiatre réputé, brillant, cultivé, mélomane et gastronome, en sociopathe manipulateur, sadique et cannibale), dans lequel un certain Jame Gumb, alias Buffalo Bill, jeune homme perturbé par une identité sexuelle mal définie, lui aussi adepte des travaux de couture, enlève des femmes pour leur voler leur peau. Il utilise notamment, pour les faire monter dans son van, le coup dit «~du bras dans le plâtre~», un stratagème emprunté à Ted Bundy. Ce dernier, élevé par les parents de sa mère qui lui ont longtemps fait croire qu’ils étaient se parents et qu’elle était sa sœur, n’a jamais réussi à s’intégrer parfaitement au monde des gens normaux. D’autant moins, si l’on en croit certaines des rumeurs sournoises qui couraient dans le voisinage, que son grand-père était son père, autrement dit qu’il était le fruit pourri d’une relation incestueuse entre sa mère et son grand-père. Le coup dit «~du bras dans le plâtre~» est librement inspiré des pratiques de Ted. Ce dernier, en effet, n’avait pas de van, mais une Coccinelle jaune. Il n’avait pas non plus le bras dans le plâtre, technique un peu complexe à mettre en œuvre, mais en écharpe. Ce qui est certain, en tout cas, c’est qu’il se pointait sur les campus et se servait de cette prétendue infirmité pour s’attirer les bonnes grâces de ses victimes. Il leur demandait de l’aider à charger des objets lourds et encombrants dans le coffre de sa voiture, et dès qu’elles étaient affairées à se rendre utiles il en profitait pour les agresser lâchement par derrière. Teddy était un petit malin qui avait un gros problème avec les femmes. Il avait aussi toutes sortes de vilains penchants qui le tenaient éloigné d’une vie épanouie en société. C’est dommage, car en dépit d’un monosourcil pas très gracieux dont il n’a semble-t-il jamais jugé utile de se départir, Bundy appartenait plutôt à la catégorie des beaux gosses charismatiques qui n’ont qu’à claquer des doigts pour que les plus belles femmes leur tombent dans les bras, la tête en arrière, les yeux révulsés, le souffle court et la bouche entrouverte. Rien à voir avec la plupart de nos tueurs en série hexagonaux, qui sont vieux, moches, puent du bec et n’ont finalement pas d’autre choix que s’en prendre à des enfants en bas âge ou des handicapés. Un Emile Louis, par exemple, aurait eu beau se pointer sur des campus avec le bras en écharpe dans une Coccinelle jaune, je doute fort qu’une étudiante, même grosse et moche, ait consenti à lever le petit doigt pour l’aider à charger des objets lourds et encombrants dans le coffre de sa voiture. Son physique ingrat et son QI de mouche à merde l’ont contraint à s’attaquer à des proies sans défense, de pauvres créatures qu’il reluquait dans le rétro de son bus avant de les agresser sexuellement et les laisser pour mortes dans des endroits déserts. De la même façon, un Michel Fourniret, malin comme un renard mais vieux et surtout terriblement moche, aussi appétissant qu’un rat d’égout tombé d’une benne à ordures, devait recourir aux services de sa compagne demeurée pour embarquer des gamines dans sa camionnette et les conduire à leur mort. Un peu à la façon d’un Albert Fish, alias le Vampire de Brooklyn, l’Ogre de Wysteria, the Grey Man ou encore le Moon Maniac, cet espèce de taré made in USA, oiseau de malheur à tête de charognard qui aimait torturer les enfants et s’enfoncer des aiguilles dans le cul. Entendez par là que les Etats Unis ont eux aussi eu leur lot de types moches et déjantés dont la seule apparence aurait suffi à vous faire sauter à pieds joints dans le premier vaisseau en partance pour l’espace. On ne peut pas éternellement se flageller, même s’il est clair que les tueurs yankees ont globalement plus de classe que les nôtres. Un Ed Kemper, for exemple, deux mètres pour cent cinquante kilos de barbaque et un QI de surdoué, donne forcément à réfléchir, surtout quand on sait qu’il s’est payé le luxe, après avoir assassiné ses grands-parents, massacré un certain nombre d’étudiantes et planté des fléchettes dans la tête coupée de sa mère, de se rendre gentiment à la police. On se dit qu’il ne faut finalement pas grand-chose pour que l’esprit humain se mette à dysfonctionner dans les grandes largeurs. Parce que si tel n’est pas le cas, si dysfonctionnement il n’y a pas, alors on est en droit de se poser de sérieuses questions sur l’avenir de notre espèce.

\textsc{Greg} : On se croirait dans Psychose.

\textsc{Moi} : En plus exotique.

Greg, étreignant nerveusement la crosse du Bersa Thunder 380 CC dissimulé sous sa veste : J’ai peur.

\textsc{Moi} : Ne t’en fais pas, je m’occupe de tout.

\textsc{Lui} : C’est bien ce qui me fait peur.

Comme à peu près tout ce qui se trouvait dans l’hôtel, à part les animaux empaillés, Greg et moi, le type de la Réception appartenait à la catégorie de ce qu’on appelle pudiquement les «~personnes de couleur~». Greg, qui était en train de glisser lentement dans l’alcoolisme dit «~mondain~», n’était plus très loin d’appartenir à la catégorie très convoitée des personnes de couleur rouge, comme les Amérindiens, bien sûr (encore que cela ne concerne originellement qu’une très petite partie d’entre eux qui avaient pour habitude de s’enduire la peau d’ocre rouge), mais aussi la grande communauté des victimes de coups de soleil (qui ne constituent pas une ethnie à proprement parler), et surtout la seule et unique personne de Donald Trump (le fameux «~rouge Trump~» qu’on retrouve non seulement sur le visage de l’intéressé, mais aussi les «~power ties~», casquettes et autres produits dérivés en vente libre sur le Trump Store).

Sur le revers de sa veste, le Réceptionniste portait un badge sur lequel était inscrit «~Dumo~». J’en ai déduit qu’il s’agissait de son prénom, lequel, à une lettre près (un B en l’occurrence), aurait été celui du sympathique éléphanteau des studios Disney, animal avec lequel, mis à part un certain embonpoint, il ne partageait pas de ressemblance directe. En effet, il avait des oreilles minuscules, à peine plus développées que des branchies, et un nez court et épaté, comme si on avait tapé dessus pendant des heures et des heures avec un rouleau à pâtisserie, un maillet ou tout autre objet contondant, en bois de préférence, autrement dit un organe aux antipodes de cette chose longue et majestueuse qu’on appelle une trompe.

Personne n’a jamais vu un œuf avec des cheveux (sauf en Chine, bien sûr, terre de tous les mystères y compris les plus absurdes et incongrus, je pense notamment à cette manie qu’ils ont de bouffer des œufs bouillis des heures durant dans de la pisse de collégien, pratique tout de même un peu étrange dont on peine à entrevoir clairement la finalité, sauf en Chine, disais-je, où un villageois des environs de Quanzhou, dans le sud-est du pays, a découvert un œuf couvert de poil dans le ventre d’une truie, curieux objet dont la valeur marchande, s’il s’agit bien de la concrétion attendue, pourrait avoisiner le million de dollars), mais on s’attend à en trouver, ne serait-ce quelques uns, sur le crâne d’un être humain. Même celui d’un Chinois chauve, un bonze, par exemple, ou un vieil herboriste du Sichuan, sur lequel on finit toujours par en dénicher un qui traîne au détour d’un pli ou une oreille. Et pourtant, aussi incroyable que cela puisse paraître, je vous fiche mon billet de longues heures de recherche, à la loupe ou au microscope, n’auraient pas révélé la présence du moindre élément de ce type sur le crâne de Dumo, entièrement revêtu d’une substance laquée plus proche de la boule de billard que du cuir chevelu.

Par ailleurs, je ne sais pas si vous avez déjà essayé de pianoter sur le clavier d’un ordinateur avec des boudins créoles à la place des doigts (ou des saucisses de Toulouse), mais même si vous ne l’avez jamais fait, je suis certain que vous n’aurez aucun mal à vous représenter la difficulté de la tâche. Et compatir du même coup, en bon chrétien, bouddhiste ou musulman que vous êtes (je milite en effet, à défaut de l’abandon pur et simple de toute espèce de religion, hormis peut-être le vaudou, la santeria et accessoirement le Temple Satanique de Greaves \& Jarry à Salem, pour une approche plus œcuménique, mystique, néo-romantique et libérale de la Foi, enfin délivrée de ses chaînes et autres clous christiques plantés dans les chairs tuméfiées de l’Espérance et la Rédemption), au calvaire de Dumo, qui se trouvait exactement dans la très pénible situation que je viens d’essayer, au travers de métaphores charcutières à mon sens assez pertinentes, de décrire aussi fidèlement que possible. Car en effet, le pauvre vieux s’échinait à pianoter sur un clavier dont les touches étaient dix fois trop petites pour ses extrémités digitales. Forcément, la virtuosité s’en trouvait grandement altérée, sa bonne humeur également, et l’ampleur de la tâche mobilisait l’entièreté de ses facultés intellectuelles. Son état de concentration, extrême, n’ouvrait aucune brèche sur le monde extérieur. On aurait pu forcer l’entrée de l’hôtel au bulldozer sans qu’il s’en aperçoive, et un troupeau de buffles lancés à pleine vitesse aurait pu faire de même sans obtenir davantage de résultat. Des gouttes de sueur perlaient sur son front, avant de descendre le long de ses joues, tels des petits animaux rampants laissant une trace humide dans leur sillage, et s’écraser lourdement en contrebas, sur le clavier de l’ordinateur, ce qui obligeait Dumo à l’essuyer sans arrêt avec des mouchoirs en papier, les mêmes (enfin d’autres, sortis des mêmes paquets) dont il se servait pour s’éponger le visage aussi souvent que possible avant de les jeter sans précaution dans la corbeille sise à proximité (il ne prenait pas le temps de viser, autant dire que les trois quarts atterrissaient à côté). De la même façon que certains se rongent les ongles, manipulent un objet quelconque, remuent frénétiquement telle ou telle partie de leur corps (le plus souvent la jambe, comme s’ils étaient pressés de s’enfuir), ou se triturent fébrilement une mèche de cheveux (toujours la même, comme si elle était rattachée à certaines terminaisons nerveuses déterminantes pour le succès de l’opération), lui se caressait machinalement le haut du crâne avec le plat de la main. Ce faisant, il se retrouvait avec une main pleine de sueur qu’il devait à son tour essuyer (avec un de ces mouchoirs en papier dont il faisait un usage compulsif, comme je l’ai indiqué, mais aussi très souvent avec certaines parties parmi les plus accessibles de ses propres vêtements, avec les conséquences désastreuses que l’on imagine en termes d’apparence et de propreté), avant de la tremper à nouveau dans la sueur de son crâne, l’essuyer à nouveau, et ainsi de suite, devenant ainsi la proie du jeu de dupe dont il était le principal artisan.

N’entrevoyant aucune issue favorable à cette effroyable tragédie, je me suis vu, l’espace d’un court instant, dégainer Manu et tirer une balle dans la tête de ce pauvre Dumo.

Et me suis aussitôt ravisé, bien entendu, conscient que le remède aurait quand même été quelque peu disproportionné. J’ajoute que des témoins se trouvaient sur les lieux, assez peu nombreux, certes, mais non moins vigilants, qui se seraient fait une joie de témoigner en ma défaveur. Quand on est, comme moi, quelqu’un dont l’essentiel de l’activité consiste à envoyer ses concitoyens croupir derrière les barreaux d’une cellule, il ne fait pas bon se retrouver dans la même situation. Tous n’attendent qu’une seule chose, en dehors de la libération conditionnelle ou la réduction de peine : vous voir débarquer pour vous mettre en pièces. Le monde carcéral est une poubelle dans laquelle on entasse les rebuts de la société, ses déjections, les corps étrangers qu’on extirpe de son épiderme. On aimerait tirer la chasse une bonne fois pour toutes, mais l’éthique commande de les maintenir en vie, leur offrir le gîte et le couvert avant de les relâcher dans la nature avec l’espoir que ce séjour à l’ombre leur aura rafraîchi les idées. Mais si vous placez des voleurs, des violeurs et des assassins dans le même périmètre, qu’est-ce qui va se passer à votre avis ? Eh bien soit ils s’entretuent, ce que tout le monde souhaite intérieurement sans oser publiquement l’avouer, soit ils se serrent les coudes entre confrères et ressortent de là gonflés à bloc comme jamais, bien décidés à unir leurs forces pour prendre leur revanche sur la société, lui faire payer au centuple le préjudice subi. En enfer, les enfants de chœur virent leur cuti. Rien de tel, pour former des criminels endurcis, que de leur tanner le cuir en prison. Le raisin de la haine fermente pour engendrer le nectar du mal. L’ennui, c’est qu’on ne sait pas quoi en faire, et que les entretenir ad vitam æternam finit par coûter cher à la société. Imaginez un instant que tout délinquant, aussi jeune et minime soit-il, soit éliminé physiquement au moindre faux pas, réduit en cendres et rayé des cadres de la civilisation, en partant du principe que les chances qu’il s’amende sont nulles, et que même si par miracle il y parvenait, le jeu n’en vaut de toute façon pas la chandelle. Autant former tout de suite des gens instruits et compétents plutôt que se casser le cul (et la tirelire) à essayer de rattraper de justesse des repris de justice. Pourquoi, en gros, ne pas travailler à l’américaine ? On sait que la police ne peut pas être partout, en permanence à tous les coins de rues, l’agent du maintien de l’ordre n’a pas quatre bras et encore moins le don d’ubiquité, alors pourquoi ne pas permettre aux honnêtes gens de faire eux-mêmes le sale boulot. Vous travaillez d’arrache-pied pour gagner honnêtement votre vie, vous revenez tranquillement du cinéma ou du restaurant après une dure journée de labeur, comme les parents de Bruce Wayne, et vous vous faites sauvagement agresser par un junkie nauséabond qui en veut à votre portefeuille et aux bijoux de votre femme. Bienvenue à Gotham City. Vous attendez quoi, qu’il vous assassine froidement dans une ruelle sombre et humide ? Non, bien sûr : vous sortez votre flingue (vous ne sortez jamais sans lui, il est votre ami pour la vie)et faites sauter le caisson du rat d’égout sans autre forme de procès (je sais bien qu’il faut que les juges et les avocats gagnent leur croûte, mais ce serait bien qu’ils ne le fassent pas au détriment de notre santé). Et hop, vous faites d’une paire deux couilles : non seulement vous sauvez votre portefeuille et accessoirement votre femme et (surtout) ses bijoux (qui vous ont coûté un bras, parce que vous, contrairement à certains qui ne s’embarrassent pas de principes, vous payez vos dettes rubis sur l’ongle), mais en plus vous participez gracieusement au ramassage des ordures ménagères, lesquelles, on le sait, sont de plus en plus envahissantes dans nos cités laissées à l’abandon. D’accord, vous êtes une grosse merde et votre conscience vous gratte un peu le cul pendant un jour ou deux, mais au moins vous assumez vos responsabilités civiques et méritez pleinement votre qualificatif de citoyen modèle (et pouvez toujours aller vous confesser au curé de la paroisse qui se fera un plaisir de vous donner l’absolution). Fini de faire le tri, on ratisse large et on fait le ménage à la louche, tout doit disparaître. Même chose pour le petit voyou qui vole une barre de céréales chez l’épicier du coin : négatif, votre Horreur, je plaide coupable, mon client est une ordure et on ne va pas s’amuser à lui taper gentiment sur les doigts alors qu’on sait pertinemment qu’il va recommencer à la première occasion. On le tue tout de suite, on gagne du temps, de l’argent, et tout le monde est content (sauf lui et sa famille, bien sûr, mais ça on s’est fout, c’est de leur faute, ils n’avaient qu’à mieux l’éduquer, et d’ailleurs on va éradiquer le nid dans la foulée). Je ne sais pas, moi. J’essaie juste de trouver des solutions pour éviter que nos femmes et nos enfants se fassent tripoter dans les transports en commun, nos petits vieux dépouiller par des individus peu scrupuleux (alors que les croquemorts s’en chargent très bien en toute légalité). Oui, d’accord, pour le petit voyou qui a volé une barre de céréales chez l’épicier du coin, on n’est peut-être pas obligé de le tuer tout de suite. On peut commencer par lui couper la main qui a commis le larcin, histoire de lui faire comprendre que ce n’est pas parce qu’on n’a pas d’argent pour s’acheter à manger qu’il faut se croire autorisé à dérober le bien d’autrui. Dans ce cas-là, on ne mange pas, voilà tout, ou alors, si on a vraiment très faim, on se trouve un travail honnête pour subvenir à ses besoins. Ce n’est pas le travail qui manque, vous savez. Bon, il est certain que compte tenu de son niveau d’études, le petit voyou en question ne va pas pouvoir prétendre à un salaire mirobolant. Mais il gagnera tout de même suffisamment pour s’acheter honnêtement une jolie petite pomme, et il la mangera avec d’autant plus de satisfaction qu’elle aura de la valeur pour lui. Si elle représente la moitié de son salaire, par exemple, vous pouvez être certain qu’il la mangera avec énormément de satisfaction, en savourant chaque bouchée jusqu’à l’extase. Alors que le riche, lui, le pauvre, ne prend plus aucun plaisir à manger une pomme, ou alors seulement si elle recouverte d’une feuille d’or et si on a pris soin de remplacer les pépins par des diamants. Comme le pauvre, le riche ne prend plaisir à manger une chose que si elle représente une partie non négligeable de son salaire, ce qui signifie qu’il n’est pas évident de se nourrir quand on a la malchance de gagner cent ou cent cinquante mille euros par mois. Même la truffe blanche ou le caviar à cinq mille balles le kilo, il faut déjà en avaler pas mal avant de commencer à ressentir un semblant de bien-être, une vague sensation de plaisir. D’autre part, le riche ne prend plaisir à être riche que s’il en permanence le référent de la misère sous les yeux, de même que le pauvre ne prend plaisir à l’être que s’il a en permanence le référent de l’abondance et la richesse sous les yeux. Et c’est exactement ce qu’on s’efforce de lui fournir, avec la presse people, les jeux débiles où des animateurs blindés de fric se foutent ouvertement de sa gueule, les centres commerciaux rutilants, les offres soi-disant spéciales et les supermarchés qui créent l’illusion de l’abondance en débordant de produits identiques dont seule la présentation diffère. Il se repaît des faits et gestes des riches et des puissants, se saigne aux quatre veines pour assister à des matchs de foot où les joueurs gagnent trois ou quatre mille fois le SMIC, et dans le même temps se plaint que la vie est trop chère et fonce sur les flics qui tentent de le contrôler alors qu’il roule bourré à cent à l’heure dans les rues de la ville. Il estime sans doute, en compensation de la vie de merde qu’il a courageusement accepté d’assumer, avoir droit à une certaine tolérance de la part de Nation reconnaissante. Le problème, c’est que la Nation ne lui reconnaît que le droit de se taire, estimant pour sa part que lui accorder le droit de vivre est déjà un privilège inestimable (mais un mal nécessaire à la survie du régime). Il se comporte d’autant plus mal qu’il n’a pas grand-chose à perdre, à part sa vie, à laquelle il ne tient pas plus que ça, ou sa liberté, qui ne lui sert pas à grand-chose s’il ne peut pas rouler bourré, insulter les gens et dégrader le bien d’autrui. On pourrait croire, quand il s’émerveille devant une voiture de luxe qu’il ne pourra jamais s’offrir, une star qu’il ne pourra jamais approcher, une actrice somptueuse qu’il ne pourra jamais soumettre à ses exigences sexuelles (et heureusement pour elle, la pauvre), que c’est de la convoitise. Mais non, ça le rend vraiment heureux et fier d’être pauvre, d’appartenir à la noble corporation des traîne-savates alcooliques et édentés (les fameux «~sans-dents~» du président Hollande, politicien médiocre mais humoriste réputé), et il ne changerait de catégorie sociale pour rien au monde. Si on lui donnait dix millions de dollars en petites coupures usagées dans un sac poubelle, il ne les prendrait pas. Enfin si, il les prendrait, mais il ne saurait pas quoi en faire, le pauvre. Autant donner de la confiture à un cochon. Il continuerait à boire de la piquette, mais il en boirait dix fois plus compte tenu de son budget illimité, et son espérance de vie, déjà aussi mince qu’une feuille de papier à cigarette (cigarettes qu’il fume par paquets entiers, de même qu’il s’adonne sans retenue à tous les vices à sa disposition), s’en trouverait réduite d’autant. Il essayerait bien de le placer à droite à gauche, sur les conseils avisés de quelque escroc en col blanc, de s’acheter une grosse voiture, une grosse bagnole, une grosse maison avec une grosse femme et des gros enfants, mais il se retrouverait vite entouré d’une nuée de parasites, contraint de se séparer de tous ses «~amis~» qui ne seraient en réalité qu’une bande de profiteurs de la pire espèce, totalement dénués de scrupules, et ne pourrait espérer aucun salut de la part des riches, lesquels continueraient éternellement à le considérer comme un corps étranger et opposer une fin de non-recevoir à ses tentatives de rapprochement. Sa vie, qui était déjà un pur calvaire, tournerait au cauchemar, car il se verrait contraint, après que sa femme ait demandé le divorce et tenté de le soulager de la moitié de sa fortune, de terminer ses jours dans la solitude la plus effroyable, passer ses journées à picoler des grands crus au bord de sa piscine et rouler sans but sur les hauteurs de Cannes au volant de sa Maserati flambant neuve. Quelle horreur ! La raison pour laquelle les riches, les vrais, les de pères en fils depuis des générations, n’ont que des amis riches, qui ne sont pas des amis, du reste, car il est bien entendu que le riche n’a pas d’ami, mais seulement des amis riches, autrement dit des amis qui n’en sont pas, au mieux des compagnons d’infortune et de perdition, hermétiques à la compassion mais rompus à toutes les subtilités du secret bancaire et la fiscalité paradisiaque, c’est précisément que les riches, qui le sont déjà par définition, et ce depuis longtemps, le plus longtemps possible de façon à ne laisser planer aucun doute sur le sujet, n’ont à priori aucune raison d’abuser d’eux. Le riche et le pauvre doivent être maintenus à bonne distance l’un de l’autre, ni trop proche ni trop éloignée, de sorte qu’ils puissent continuer à s’observer dans l’irrespect mutuel et l’aversion réciproque. C’est sur cet équilibre instable que repose l’avenir de la civilisation occidentale, dont on peut craindre, effectivement, qu’elle ne s’écroule à tout moment. Et j’ai envie de dire à mes amis développés, tant sur le plan de la morphologie, la technologie et la culture, très supérieurs à la moyenne dans leur approche de l’existence, mélomanes d’exception et lecteurs assidus des plus grands philosophes, qui, soudain pris de court par la déferlante qui menace de les engloutir, cherchent désespérément mon regard dans les ténèbres pour se raccrocher à quelque chose de profondément humain, sensible et empathique : oui, mes frères, c’est mal barré, mais tout n’est pas perdu. Grâce à l’oncle Sam, l’Armée rouge et le Soviet suprême, et même sans l’appui du gnome coréen à tête de poupon maléfique qui nous déteste cordialement, j’affirme sans une once de tremblement dans la voix que nous disposons de suffisamment d’ogives nucléaires pour maintenir à distance les hordes de pauvres et autres déshérités qui se pressent à nos portes. Disparaissez Marcheurs blancs, émissaires de la Mort, et laissez-nous jouir en toute liberté de nos Livrets A, LEP et PEL. Foutez-nous la paix et laissez-nous toucher en paix nos allocations familiales, revenus de solidarité active et autres primes de Noël. Non, vous ne toucherez pas à nos comptes en Suisse, pas plus qu’à nos épouses et encore moins nos filles, et n’espérez pas tremper un jour vos fesses ramollies dans nos piscines, vos mouillettes impies dans de jaune de nos œufs, ni vos lèvres perfides dans le champagne de nos coupes.

J’ai chargé Greg de faire le guet.

Comment j’ai fait ça ?

Rien de plus simple : je l’ai pris solennellement à part, dans un renfoncement de la cage d’escalier digne d’un palais des mille et une nuits, et, après lui avoir fait prêter allégeance à la cause et jurer fidélité à ma personne, je lui ai confié la lourde responsabilité de surveiller nos arrières, autrement dit me signaler immédiatement tout fait ou geste qui lui semblerait un tant soit peu suspect.

Quel grand moment d’émotion ! Je le vois encore fondre en larmes, tel un gros bébé au visage anguleux et la ceinture abdominale légèrement relâchée, l’exercice n’étant pas son fort, et s’écrouler comme une merde à mes pieds, bouleversé par l’immense honneur que je lui accordais. Je ne voudrais pas entrer dans les détails, mais il avait toujours souffert d’un manque de reconnaissance total de la part de son père, qui n’avait cessé de le dénigrer et lui faire sentir que, quoi qu’il fasse, il ne parviendrait jamais à se hisser à la hauteur de ses exigences. C’est dur pour fils de se faire traiter comme une sous-merde par son propre géniteur, celui auquel on pense devoir la vie (alors qu’il n’y est pour rien, en fait), c’est pas terrible pour la confiance, et il faut souvent des années de psychanalyse pour s’en remettre, et encore jamais complètement. Ce type était clairement une ordure qui n’aurait jamais dû avoir d’enfant. Seulement voilà, la nature se fiche que les ordures, assassins, nazis, trafiquants de drogue, politiciens corrompus et autres, se reproduisent au même titre que les honnêtes gens. Vous pouvez être le type le plus dégénéré qui soit, vous finirez toujours par trouver une fille aussi déglinguée que vous qui sera ravie d’offrir son ventre à vos ébats. La nature ne fait pas de différence entre ses enfants, tous ont les mêmes chances de se reproduire, et elle fait en sorte que chacun trouve chaussure à son pied, même s’il pue des pieds et s’il s’agit d’une vieille godasse trouée sortie d’une poubelle. Elle a donné à tous le pouvoir de niquer, le mode d’emploi et les outils pour le faire. Et elle a fait en sorte, histoire d’être bien sûre de ne pas rater son coup, que tous ne pensent qu’à ça vingt-quatre heures sur vingt-quatre, sans distinction de revenus, quotient intellectuel, familial ou autre. Et sans se balader avec la trique au vent vingt-quatre heures sur vingt-quatre, ce qui pourrait se révéler rapidement pénible et handicapant. D’où le coup de l’engin à géométrie variable, facile à ranger discrètement dans un fond de culotte pour ne pas donner l’alerte en permanence, créer un état de panique générale impossible à endiguer. La femme, par contre, avec ses seins non rétractables, dispose d’un handicap certain en la matière. Facile à repérer, il lui est d’autant plus difficile d’échapper à la concupiscence de ses concitoyens. On peut dire que la nature lui a joué un tour pendable. C’est la fable du pot de miel et du saumon. Imaginez qu’on vous dise : Tu vois cette forêt ? Elle est infestée d’ours bruns, des gros qui n’ont qu’une seule idée en tête : bouffer. Et maintenant, tu vois cette table ? Dessus, il y a un pot de miel et un saumon. Tu choisis l’un ou l’autre, ce que tu préfères, tu le poses en équilibre sur ta tête, et tu vas te balader au milieu des ours. Il est clair que vous avez toutes les chances de vous faire arracher la tête à tous les coins d’arbres. Depuis, la vie de la femme est un véritable cauchemar, un parcours du combattant qui force l’admiration. À moins de s’emmailloter sous trente ou quarante couches de fringues, de dissimuler son visage et l’abondante chevelure soyeuse et parfumée dont elle est naturellement pourvue, elle est condamnée à vivre dans un monde où ses chances de survie, sexuellement parlant, flirtent avec le zéro absolu. Les Arabes, qui l’obligent à se couvrir de la tête aux pieds avant de mettre le nez dehors, ne laissant la place qu’à une ouverture grillagée ou une fente minuscule pour lui permettre de respirer et voir sans être vue, sont en fait d’ardents protecteurs de la femme contre leur propre tyrannie. Non seulement ils reconnaissent implicitement qu’ils sont tous des obsédés sexuels incapables de contrôler leurs pulsions, ce qui est déjà une belle preuve de courage et d’abnégation, de recul sur soi (même s’il ne s’accompagne pas nécessairement d’un sens de l’humour à toute épreuve), mais ils lui indiquent (de façon assez autoritaire, il est vrai) le moyen de se prémunir contre leurs ardeurs, sachant qu’eux-mêmes, à moins de se crever les yeux, sont incapables d’avoir une approche saine et dépassionnée de la plastique féminine. Un banc de morues ne peut pas évoluer sans protection au milieu des récifs. Les requins sont partout, affamés (ou pas, du reste, car le requin a toujours faim, qu’il ait déjà mangé ou non), et la moindre écaille qui scintille à l’horizon les plonge aussitôt dans un état de frénésie proche de la démence. Même s’il n’y a pas énormément de requins dans le golfe Persique, et si la morue n’est pas le plat favori des Iraniens, nos amis Arabes, nobles descendants du royaume de Saba et des Omeyyades, ont compris la nécessité de protéger la femme de la violence de leur désir. Pas question pour eux d’envoyer des filles en hot-pants de cheerleaders se trémousser au milieu d’une caravane de bédouins qui viennent de traverser le désert à dos de chameau, des gens dont les burnes, sous l’effet de la chaleur et des chocs répétés, ont atteint un tel degré de dilatation qu’elles sont susceptibles d’exploser à tout moment, comme les graines du cornichon d’âne ou de l’herbe à Robert qu’on effleure par inadvertance. Ce n’est que dans la plus stricte intimité, à l’abri des regards concupiscents de la meute en chaleur, que la femme peut se dévoiler. L’efficacité est indéniable, mais on peut critiquer la méthode, qui est aux antipodes de notre façon de procéder. En effet, nous avons décidé de traiter le mal par le mal. Puisque nous sommes tous des gros porcs lubriques incapables de contrôler nos pulsions, autant y aller à fond et affronter nos démons à bras-le-corps. Le danger, en dissimulant la femme, c’est que cette occultation ne fasse qu’exacerber les passions et entraîner l’imaginaire vers des horizons aussi dangereux qu’insoupçonnés. Si vous sentez qu’on vous cache quelque chose, ou cherche à vous le cacher, votre envie de savoir est d’autant plus vive, intense, et peut rapidement virer à l’obsession. Vous imaginez des choses qui ne sont pas conformes à la réalité et courez droit à la déception, la frustration. Sauf, bien sûr, si vous vous gardez de toute extrapolation intempestive et conservez intact votre sens du merveilleux. Mais cela n’est pas donné à tout le monde. Pragmatiques, nous avons décidé qu’il valait mieux jouer la carte de la transparence : aux premiers rayons de soleil (oui, il ne fait pas toujours une chaleur à crever dans nos contrées) sur fond de ciel bleu dégagé et d’oiseaux qui chantent dans les arbres à la végétation naissante, la femme, à l’instar de l’homme, pourra elle aussi se trimballer à moitié nue dans les rues de la cité, confiante dans le fait que nous saurons garder nos distances et éviter regards et réflexions salaces concernant tout ou partie de son anatomie. Ne sommes-nous pas des gens civilisés, qui n’avons nul besoin de recourir à de grossiers stratagèmes pour conserver élégance et dignité ? Après tout, les Africains arrivent très bien, dans leurs contrées lointaines, à vivre en permanence entourés de femmes largement dévêtues sans développer aucun ressentiment particulier. Tout se passe très bien pour eux, ils s’accouplent quand bon leur semble, de la façon qui leur plaît, et personne ne trouve rien à redire à la situation. C’est bien la preuve, si on n’est pas complètement con, qu’on n’est pas obligé de couvrir sa femme de la tête aux pieds pour éviter les problèmes. De toute façon, quoi qu’on fasse et qui qu’on fesse, il y aura toujours des emmerdeurs pour s’affranchir des règles et traverser en dehors des clous.

J’ai relevé Greg, l’ai serré fort dans mes bras, tel un père qui verrait son enfant pour la dernière fois, ou une maman ours qui serrerait dans ses bras son bébé ours avant de le laisser partir seul dans les profondeurs de la forêt infestée de méchants chasseurs alcooliques et réactionnaires, puis me suis dirigé d’un pas lourd mais décidé vers la Réception.

Nous étions, je le rappelle, en territoire ennemi. Le coup pouvait venir de n’importe où, n’importe quand, et se présenter sous n’importe quelle forme, y compris la plus innocente, comme ces enfants qui surgissent dans le djebel et s’avancent vers vous avec une bombe à la ceinture, le sourire aux lèvres et les bras chargés de dattes. Je te raconte pas le méchoui, sidi. Alors oui, je sais ce que vous allez me dire : pas la peine faire d’en faire des caisses, de sombrer dans le délire paranoïaque des références au djihad et à l’apocalypse selon Saint-Maclou ! On se calme et on boit frais à Saint-Tropez, comme disait le regretté Max Pécas, petit producteur de navets connu pour la qualité de ses bulbes (La Baie du désir, Je suis une nymphomane, Marche pas sur mes lacets, Mieux vaut être riche et bien portant que fauché et mal foutu, etc). Vous êtes juste deux touristes mal réveillés dans le hall d’un hôtel de luxe, en plein jour, armés jusqu’aux dents qui plus est, on ne voit donc pas très bien ce qui pourrait vous arriver, même s’il est clair que vous n’êtes pas à proprement parler les bienvenus dans le contexte, compte tenu de la connerie qui vous anime et la prétention abyssale qui vous caractérise, sans parler de la couleur de peau blafarde et grassouillette d’asticot avec laquelle vous avez eu le toupet de venir au monde. Certes, ce n’est pas de gaité de cœur qu’on vous voit fouler le plancher ancestral de cet édifice prestigieux, mais ce n’est pas comme si vous étiez deux chérubins prépubères égarés dans un congrès de pédophiles, ou, pire encore, deux Juifs handicapés, homosexuels et communistes ayant atterri, suite à une cascade d’événements tous plus rocambolesques les uns que les autres, en plein milieu d’une réunion de nostalgiques du Troisième Reich.

C’est donc plein d’espoir et la poitrine gonflée d’orgueil que Greg, rasséréné par les propos lénifiants (j’aimerais, au risque d’altérer la fluidité de la narration, avoir une pensée émue pour le jeune adulte dont le niveau de vocabulaire~-- et de culture générale~-- n’excède pas celui d’un enfant de cinq ans d’il y a cinquante ans et qui, s’accrochant à chaque phrase tel un naufragé à une planche de bois vermoulu, essaie malgré tout courageusement de lire ce livre, au demeurant moins hermétique que Finnegans Wake, Le Roi pâle ou La Maison des feuilles, et lui dire que non, tout n’est pas perdu, à condition bien sûr qu’il cesse immédiatement d’ingurgiter des torrents de soupe indigeste avec des gros morceaux de caca qui pue à l’intérieur sur X, TikTok, WhatsApp, Instagram et les autres, sans quoi il parviendra à un niveau~-- level, je traduis en anglais par charité chrétienne, pour l’aider à ne pas décrocher totalement~-- de décérébration si stratosphérique que même ses propres enfants, qu’il aura bien évidemment achetés en solde sur Internet et renvoyés plusieurs fois pour vice de forme, au risque de se retrouver avec des articles ne correspondant plus du tout à ses attentes, inexistantes de toute façon, ne le reconnaîtront plus) que je venais (je vous avais prévenu qu’on allait perdre en lisibilité) de lui déverser à flux tendu dans le creux de l’oreille, s’est élancé, tel un jeune faon qui s’éveille dans la forêt enchantée de ses ancêtres, bercé de mille senteurs et sonorités à la fois étranges et familières qui l’émerveillent autant qu’elles l’interloquent, sur le sentier lumineux (rien à voir avec le parti du camarade Gonzalo, je vous rassure tout de suite) de l’épanouissement personnel, respirant à pleins poumons l’air nouveau de la confiance retrouvée.

Après quoi je me suis présenté à l’accueil, arborant le sourire satisfait du voyageur qui vient d’effectuer une confortable traversée de l’Atlantique en jet privé, entouré d’un essaim d’hôtesses vibrionnantes laissant flotter dans leur sillage un subtil bouquet d’exhalaisons paradisiaques, et j’ai attendu que Dumo, toujours très affairé à son travail, daigne s’intéresser à moi.

Ce qu’il n’a pas fait, manquant à tous ses devoirs avec une impudence désarmante de naturel.

Mais aussi terriblement irritante.

Du coup, ma pression artérielle est montée en flèche.

Et il m’arrive, quand ma pression artérielle monte en flèche, de faire des choses que je ne fais pas en temps normal, ou nettement

moins, comme par exemple tatouer mes initiales au fer rouge sur la poitrine des gens, leur arracher les yeux et les remplacer par des balles de ping pong, leur couper le nez, la langue et les oreilles avec des ciseaux rouillés, ou encore leur ouvrir la boîte crânienne à l’emporte-pièce pour leur siroter la cervelle à la paille.

Voilà comment je me suis retrouvé dans un état d’exaspération que je ne me souvenais plus avoir atteint depuis le jour où Zarina, après avoir englouti un nombre indéterminé de Girofliers du Clair de Lune (cocktail dévastateur à base d’amaretto, grappa, sirop de framboise et clou de girofle), m’avait sauvagement agressé sous le prétexte fallacieux que j’aurais, je dis bien «~aurais~», car je ne me souviens absolument pas de l’avoir fait, approché sa sœur d’un peu trop près. Accusation profondément injuste, perfide et mensongère, dont je peine aujourd’hui encore, longtemps après les faits, à me relever (mais ça va, hein, j’ai suivi une thérapie assez musclée à base d’électrochocs et produits stupéfiants qui me permet enfin d’entrevoir le bout du tunnel, même si je passe encore par des phases de mélancolie semblables à d’épaisses nappes de brume qui colle à la peau et s’insinue au plus profond de votre être). Car tenez-vous bien, à en croire les dires de l’écervelée, nos lèvres (les miennes et celles de Tosca, sa sœur jumelle, en tout point semblables aux siennes, hormis peut-être une imperceptible inflexion à la commissure, dessinant, en de très rares occasions, une microscopique et fugace fossette en sortie de joue) se seraient pratiquement abouchées au cours d’un échange passionné concernant les relations pour le moins sulfureuses du roi Zog d’Albanie avec une certaine Tatiana Visirova, subtil mélange d’épices chinoises, roumaines et russo-polonaises savamment assemblées en une seule et même petite personne passablement dévergondée, assez peu douée pour les études, mais suffisamment à l’aise pour se produire dans le plus simple appareil sur la scène des Folies-Bergère et déclencher des tonnerres d’applaudissements à chacune de ses apparitions.

J’ai pris sur moi pour ne pas sortir Manu et exploser le crâne bosselé du crétin des Alpes qui me faisait face.

Vers la fin du XVIIIe siècle, à l’aube des sports d’hiver et la révolution technologique, les premiers touristes parviennent au sommet des monts alpins et découvrent avec stupeur qu’ils ne sont pas seuls. D’étranges créatures, simiesques, contrefaites, et affligées pour la plupart de goitres spectaculaires, vivent dans ces contrées reculées. Il devient rapidement évident que ces créatures, en dépit de leurs malformations et leur intelligence limitée, ne représentent aucun danger particulier. De retour en ville, ils écument les salons pour faire part de leur découverte, ne lésinant pas sur les détails les plus atroces, n’hésitant pas à forcer le trait pour les besoins de la narration. Le bourgeois frissonne, les jeunes filles se pâment et tombent inanimées dans les bras des beaux parleurs, leurs lèvres entrouvertes exhalant le souffle rauque des désirs inassouvis. En secret, elles rêvent d’étreintes avec ces monstres innommables qu’on imagine membrés comme des taureaux, portant jusqu’aux genoux de lourdes paires de burnes couvertes d’une épaisse fourrure. Qui sont ces humanoïdes ? Sont-ils les descendants des Nains d’Erebor, dont ils ont la taille réduite, les traits grossiers, la silhouette massive et la force de cheval ? Sont-ils consanguins, cannibales, sodomites ? Quelles sont leurs idoles ? S’adonnent-ils à des rites païens placés sous le signe de la nécrophilie et la zoophilie ? On envisage un temps de les exterminer, ou les réduire à l’esclavage, les faire travailler à des tâches ingrates comme des bêtes de somme, puis on choisit de les parquer dans des sanatoriums pour les étudier à loisir et tenter de percer le mystère de leur existence et leur laideur extrême. L’armée envisage un temps de transformer les mâles en super-combattants pour les envoyer sur le front en première ligne en cas de conflit. Leur aspect menaçant devrait suffire à pousser l’ennemi à rebrousser chemin sans demander son reste. Quelques années plus tard, la science livre son verdict : non, il ne s’agit pas d’une espèce d’hommes semi-préhistoriques ayant miraculeusement échappé à la civilisation, et encore moins de créatures fantastiques sorties du ventre de villageoises engrossées par des trolls, des elfes ou des lutins, mais de pauvres hères souffrant de la thyroïde en raison du manque d’iode. Peu festif mais vrai. Grâce aux progrès de la médecine, le crétin des Alpes a aujourd’hui disparu, même s’il y a toujours autant de crétins dans les Alpes que partout ailleurs.

