\textsc{Greg} : Je vais appeler Sally Robinson. Elle sera sûrement ravie d’apprendre que Desmarais n’est plus de ce monde.

\textsc{Moi} : Et moi je vais appeler Bérénice pour lui dire qu’on est toujours sans nouvelles de Titus.

\textsc{Greg} : C’est quand même bizarre, cette histoire.

\textsc{Moi} : On nage en pleine absurdité.

\textsc{Greg} : Je me pose des questions.

\textsc{Moi} : À propos de quoi ?

\textsc{Lui} : Titus. Il aime sa femme, ses gosses, je ne comprends pas pourquoi il a suivi cette fille sans discuter.

\textsc{Moi} : C’est toujours pareil : on pense connaître les gens, et on découvre des zones d’ombre qu’on ne soupçonnait pas.

\textsc{Greg} : Tu penses qu’il est encore en vie ?

\textsc{Moi} : Je n’en ai aucune idée. Peut-être qu’il est enchaîné quelque part, en train de se faire violer et torturer par une bande de suprémacistes blancs homosexuels.

\textsc{Greg} : Ou alors il a décidé que sa vie ne correspondait plus à ses rêves et il a foutu le camp sans laisser d’adresse.

\textsc{Moi} : Il m’a souvent parlé de son envie de retourner en Sierra Leone pour renouer avec ses origines et retrouver la trace de ses ancêtres. Je crois savoir que l’idée ne plaisait pas plus que ça à Bérénice.

\textsc{Greg} : Il t’avait semblé bizarre, ces derniers temps ?

\textsc{Moi} : Pas plus que d’habitude. Il a toujours été un peu mystique. Il y a quelque temps, il s’est découvert une passion pour Edward Tylor\nf{Edward Burnett Tylor (1832--1917), anthropologue britannique considéré comme l’un des fondateurs de l’anthropologie culturelle. Auteur de \textit{La Civilisation primitive} (1871), il fut le premier à définir scientifiquement le concept de culture et développa une théorie de l’animisme comme forme originelle de la religion. Il fut nommé premier professeur d’anthropologie à Oxford en 1896. \source{fr.wikipedia.org/wiki/Edward\_Tylor}}, un anthropologue british du XIX\textsuperscript{e} qui s’intéressait à l’animisme, au totémisme et à la cosmogonie. Titus potassait aussi des trucs sur la culture yoruba et les babalaos.

\textsc{Greg} : Les quoi ?

\textsc{Moi} : Les babalaos. Ce sont des espèces de sorciers qui pratiquent la divination avec des noix de palmier.

Pendant ce temps, les pompiers avaient découpé la Mini pour extraire Repentance Whittingham de son cercueil de tôle.

La femme de ménage la plus rapide du monde avait ensuite été transportée toutes sirènes hurlantes vers l’hôpital le plus proche.

\textsc{Greg} : Et Bérénice ?

\textsc{Moi} : Quoi, Bérénice ?

\textsc{Greg} : Tu lui as parlé, récemment ?

\textsc{Moi} : Bien sûr, que je lui ai parlé récemment ! Elle m’appelle toutes les dix secondes pour savoir ce qui est arrivé à son cher et tendre. D’ailleurs c’est bizarre, ça fait un moment qu’elle ne m’a pas rappelé.

\textsc{Greg} : Elle n’a rien remarqué de particulier, elle non plus ?

\textsc{Moi} : Elle, non. Mais moi, oui.

\textsc{Greg} : Comment ça ?

\textsc{Moi} : J’ai remarqué quelque chose de particulier. Pas à propos de Titus, mais de Bérénice, justement.

\textsc{Greg} : Ah bon ?

\textsc{Moi} : Oui. Depuis quelque temps, Bérénice a un comportement inhabituel, que je ne lui connaissais pas auparavant.

\textsc{Greg} : Tu peux être plus précis.

\textsc{Moi} : C’est assez indéfinissable, en fait. C’est quelque chose dans son attitude générale, sa façon de s’habiller, de parler, et même de se déplacer.

\textsc{Lui} : Elle ne se déplace plus comme avant ?

\textsc{Moi} : Elle se déplace bizarrement, avec une démarche plus féline que d’habitude.

\textsc{Greg} : Plus féline ?

\textsc{Moi} : Oui, plus féline. Je te l’ai dit, c’est assez indéfinissable. Cela dit, en la voyant, je me suis dit que quelque avait changé en elle, comme si elle n’était plus tout à fait la même.

\textsc{Greg} : Des soucis personnels, peut-être.

\textsc{Moi} : Même son odeur n’est plus la même.

\textsc{Greg} : Elle a peut-être tout simplement changé de parfum.

\textsc{Moi} : Ne sois pas idiot. Si tu veux vraiment que je te dise ce que je pense…

\textsc{Greg} : Oui, j’aimerais bien.

\textsc{Moi} : … je vais te le dire, mais arrête de me couper sans arrêt ! Si tu veux vraiment que je te dise ce que je pense, je ne serais pas étonné qu’elle voie quelqu’un d’autre.

\textsc{Greg} : Non ???!!!

\textsc{Moi} : Si.

\textsc{Greg} : Mais alors, ça change tout !

\textsc{Moi} : Comment ça ?

\textsc{Greg} : Si elle voit quelqu’un d’autre, il se peut que Titus soit au courant, ce qui expliquerait son comportement bizarre de ces derniers temps.

\textsc{Moi} : Titus n’était pas bizarre ces derniers temps. Il n’est devenu bizarre qu’hier soir, quand cette fille a débarqué de nulle part. Dès qu’elle est entrée dans son champ de vision, c’était comme si son esprit avait quitté son corps pour aller gambader tel un guépard dans les vastes plaines du Kalahari, royaume de ceux qui suivent l’éclair et ramassent par terre. Tu te rappelles quand elle lui a dit «~toi venir avec moi dans chambre~», l’étrange lueur qu’il y avait dans ses yeux quand elle prononcé ces mots ?

\textsc{Greg} : Les yeux de qui ?

\textsc{Moi} : Ceux de Titus, bien sûr ! Tu le fais exprès, ou quoi ?

\textsc{Lui} : Excuse-moi, mais j’ai un peu de mal à te suivre dans tes divagations. Et non, je n’ai pas vraiment fait attention à la lueur qu’il y avait dans les yeux de Titus quand elle a prononcé ces mots. Je devais être occupé ailleurs. Cela dit, si Bérénice a effectivement un amant comme tu le prétends, ça pourrait expliquer bien des choses.

\textsc{Moi} : Je ne prétends rien du tout. Je dis juste qu’il n’est pas totalement impossible, compte tenu des éléments dont je dispose, que Bérénice ait un amant.

\textsc{Greg} : Quels éléments ?

\textsc{Moi} : Des éléments…

\textsc{Greg} : Tu veux dire des éléments vraiment… vraiment…

\textsc{Moi} : Compromettants, oui. Mais ce serait peut-être à toi de m’en dire un peu plus sur le sujet…

\textsc{Greg}, feignant la surprise : À moi ?

\textsc{Moi} : Oui, à toi. Je crois savoir que Bérénice est une excellente danseuse de tango…

\textsc{Greg} : C’est possible, en effet.

\textsc{Moi} : Ne fais pas comme si tu ne le savais pas.

\textsc{Greg} : J’en ai vaguement entendu parler.

\textsc{Moi} : Un peu plus que vaguement, je pense. Je crois savoir aussi que tu t’es toi-même découvert une passion pour Carlos Gardel\nf{Carlos Gardel (1890--1935), chanteur et acteur franco-argentin, figure légendaire du tango. Surnommé «~El Zorzal Criollo~» (le merle créole), il enregistra plus de 900 chansons et contribua à diffuser le tango dans le monde entier. Il mourut dans un accident d’avion à Medellín à l’apogée de sa gloire. \source{fr.wikipedia.org/wiki/Carlos\_Gardel}}, Rodolfo Biagi\nf{Rodolfo Biagi (1906--1969), pianiste et chef d’orchestre argentin de tango, surnommé «~Manos Brujas~» (les mains ensorcelées). Son style, caractérisé par un rythme syncopé et énergique, fit de son orchestre l’un des plus populaires de l’âge d’or du tango dans les années 1930--40. \source{fr.wikipedia.org/wiki/Rodolfo\_Biagi}}, Francisco Canaro\nf{Francisco Canaro (1888--1964), violoniste, compositeur et chef d’orchestre argentin d’origine uruguayenne, l’un des maîtres du tango. Il fonda son premier orchestre en 1916 et composa plus de 5 000 œuvres, dont de nombreux classiques du genre. Il fut également promoteur du tango en Europe, notamment à Paris dans les années 1920. \source{fr.wikipedia.org/wiki/Francisco\_Canaro}}, El Pibe\nf{Juan Carlos Cobián (1896--1953), dit «~El Pibe~» (le gamin), pianiste et compositeur argentin de tango. Auteur de classiques tels que \textit{Los Mareados} et \textit{Shusheta}, il fut l’un des novateurs du genre et influença de nombreux musiciens de sa génération. \source{fr.wikipedia.org/wiki/Juan\_Carlos\_Cobi\%C3\%A1n}} et Anibal «~El Gordo~» Troilo\nf{Aníbal Troilo (1914--1975), surnommé «~El Gordo~» ou «~Pichuco~», bandonéoniste et chef d’orchestre argentin, l’une des figures les plus aimées du tango. Son orchestre fut au cœur de l’âge d’or du genre dans les années 1940. Compositeur de classiques comme \textit{Sur} et \textit{La última curda}, il est considéré comme le maître incontesté du bandoneón. \source{fr.wikipedia.org/wiki/An\%C3\%ADbal\_Troilo}}, pour n’en citer que quelques-uns.

\textsc{Greg}, de plus en plus mal à l’aise : One day I was at home, et non pas reincarnated as the 7th Prince, en train de bouffer des raviolis en boîte, quand j’ai entendu une chanson de La Tana, Susana Rinaldi\nf{Susana Rinaldi (née en 1935), chanteuse et actrice argentine surnommée «~La Tana~». Considérée comme l’une des plus grandes interprètes de tango, elle se distingue par sa voix grave et son phrasé dramatique. Elle a connu une carrière internationale, notamment en Europe, et s’engage également en politique en faveur des droits de l’homme. \source{fr.wikipedia.org/wiki/Susana\_Rinaldi}} de son vrai nom, à la radio. Stupeur et tremblements, un déclic s’est produit dans ma tête. J’ai aussitôt balancé mes raviolis à la poubelle et décidé de changer de vie. Ma passion pour le tango était née.

\textsc{Moi} : Et tu t’es acheté une paire de Danilo Pinto à mille cinq cent balles !

\textsc{Greg} : Quand on aime on ne compte pas.

\textsc{Moi} : Des Danilo Pinto dont tu ne te sépares quasiment plus. La preuve, tu les as aux pieds ce matin, alors qu’on n’était pas franchement partis pour danser le tango.

\textsc{Greg} : Elles sont très confortables.

\textsc{Moi} : Je n’en doute pas. Mais ça n’explique pas pourquoi tu t’es inscrit dans la même école de danse que Bérénice…

\textsc{Greg} : Je me trompe ou tu es en train d’insinuer des choses pas très catholiques à mon sujet ?

\textsc{Moi} : Je n’insinue pas, j’affirme. J’affirme, mon cher Grégoire, que Bérénice et toi dans comme des Argentins jusque tard dans la nuit au Palazzo Cristal Club de la rue Féret. Des Argentins désargentés, peut-être, mais des Argentins tout de même.

\textsc{Greg} : C’est faux !

\textsc{Moi}, d’un calme olympien : Non, c’est authentiquement vrai.

\textsc{Lui} : Comment tu peux savoir ça ?

\textsc{Moi} : Je le sais, c’est tout.

\textsc{Lui} : C’est Bérénice qui te l’a dit ?

\textsc{Moi} : Ah ah ! Donc tu admets que Bérénice et toi vous trémoussez comme des dingues au Palazzo ?

\textsc{Lui} : On danse, c’est tout.

\textsc{Moi} : Oui, avec la femme de Titus. Titus à qui tu t’es bien gardé de dire que tu dansais le tango avec sa femme. Reconnais que tes méthodes sont particulièrement sournoises.

\textsc{Greg} : Je suppose qu’il est au courant.

\textsc{Moi} : Non, il ne l’est pas. Enfin, pas officiellement. Et ce n’est pas tout.

Greg a ouvert la bouche. J’ai senti qu’il voulait dire quelque chose, plaider sa cause, mais il s’est aussitôt ravisé, comprenant qu’il avait le droit de garder le silence et que tout ce qu’il dirait pourrait être retenu contre lui.

J’ai donc repris le fil de mon réquisitoire : On vous a vu sortir main dans la main et vous roulez des pelles sur le trottoir !

Greg a bruyamment ravalé la salive qui affluait dans le fond de sa gorge : Pardon ?

L’accusation était ignoble, je l’admets, mais hélas non dépourvue de fondement.

\textsc{Moi} : Pelles, galoches, palots, patins, appelle comme tu veux, toujours est-il qu’on vous vu vous boulotter les amygdales !

\textsc{Lui}, feignant assez maladroitement la stupéfaction (même un acteur de sitcom ou téléfilm français aurait largement fait mieux) : Hein ? Quoi ? Comment ? On ? Qui ça, on ?

Comme un bon journaliste ne cite jamais ses sources, un bon flic ne révèle jamais le nom de ses tontons.

Mais à vous, à toi, lecteur (je pense qu’on en est arrivé à un stade de notre relation où on peut raisonnablement envisager de se tutoyer), je peux bien le dire.

Je te connais suffisamment pour savoir, si je ne le fais pas, que tu vas psychoter, te creuser les méninges et douter de tout, y compris de Dieu, l’abbé Pierre, Donald Trump et Vladimir Poutine, t’interroger sur le sens de la vie, les dangers de la 5G, la réalité augmentée, du deep learning et de l’IA générative, jusqu’à en perdre le boire et le manger, le sommeil aussi, passer tes nuits les yeux grand ouverts à fixer le plafond, le cœur battant, à suer sang et eau dans le fond du lit conjugal de moins en moins conjugué, insensible aux appels désespérés de ta femme pour que tu t’intéresses enfin un peu à elle, ses désirs, ses rêves, son corps à l’abandon. Je sais que tu vas errer sans but dans les rues de la ville, telle une bête malade, chercher refuge dans des lieux de perdition, boire plus que de raison, et que tes amis, lassés de tes absences et ton humeur chagrine, vont te quitter un par un. Puis ce sera au tour de ta moitié, devenue au fil du temps ton tiers, ton quart, ton huitième, une fraction de plus en plus insignifiante de toi-même, de faire ses valises. Ayant épuisé les trésors de patience dont elle disposait, elle retournera chez sa mère, laquelle en sera enchantée pour au moins trois raisons : 1. elle vit seule avec sa chatte et se fait chier comme un rat mort ; 2. elle a toujours été extrêmement possessive et aurait voulu garder sa fille pour elle toute seule ; 3. elle t’a toujours considéré comme une erreur de la nature et un bon à rien. Mais ta femme ne partira pas seule : elle emportera avec elle la seule chose qui ne retenait encore accroché du bout des doigts à l’existence, je veux bien sûr parler de tes trois enfants. Quand tu rentrais à la maison, après ce qui aurait dû être une dure journée de labeur, ils te donnaient du «~monsieur~», non parce que vous êtes de la haute et avez conservé les usages de la vieille noblesse française (vous êtes des roturiers de la pire espèce, sans une once de sang royal dans les veines), mais parce qu’ils ne te reconnaissaient tout simplement plus. Et quand je dis «~aurait dû être~» une dure journée de labeur, c’est parce que tu as perdu ton boulot, bien sûr, et passes tes journées assis sur un banc à ruminer tes vilaines et vaines pensées, et envisager le moyen le plus expéditif de mettre un terme à tes souffrances. Tel Jean-Claude Romand\nf{Jean-Claude Romand (né en 1954), imposteur et assassin français. Pendant dix-huit ans, il fit croire à son entourage qu'il était médecin chercheur à l'OMS à Genève, tout en ne travaillant pas et en dilapidant l'épargne de sa famille. Découvert sur le point d'être démasqué, il assassina sa femme, ses deux enfants et ses parents en janvier 1993, avant de mettre le feu à sa maison. Condamné à la réclusion criminelle à perpétuité, son histoire inspira notamment le roman de Emmanuel Carrère \textit{L'Adversaire} (2000). \source{fr.wikipedia.org/wiki/Jean-Claude\_Romand}}, tu iras toquer à la porte des moines de la congrégation de Solesmes, à Notre-Dame de Fontgombault, mais ceux-ci, persuadés d’avoir affaire à un suppositoire de Satan, refuseront de t’accueillir. Par charité chrétienne, sur l’insistance du père-abbé qui est un homme profondément bon et ouvert d’esprit, cette bande de clowns tonsurés te proposera toutefois une place de jardinier au monastère Saint-Joseph de Séguéya, en Guinée-Conakry, poste que tu auras la fermeté de refuser d’une part parce que tu n’as pas la main verte, d’autre part parce que tu as toujours eu une peur bleue des Noirs, lesquels restent pour toi des créatures étranges aux yeux globuleux et à la bouche remplie de dents. C’est ainsi, seul sur le canapé du salon, la libido en berne devant les chaînes d’info en continu qui achèveront de te liquéfier le cerveau, un vieil exemplaire du \textit{Désespéré} de Léon Bloy\nf{Léon Bloy (1846--1917), écrivain et pamphlétaire français catholique, connu pour son style virulent et mystique. \textit{Le Désespéré} (1886), son premier roman fortement autobiographique, dépeint la lutte d'un homme profondément religieux contre la médiocrité bourgeoise et la lâcheté spirituelle. Il exerça une influence notable sur des écrivains comme Georges Bernanos et Jacques Maritain. \source{fr.wikipedia.org/wiki/L\%C3\%A9on\_Bloy}} à portée de main, entouré de bouteilles vides, de toiles d’araignées et de déchets alimentaires en voie de putréfaction, tu dépériras lentement dans l’indifférence générale, après que ton père t’aura déshérité et ta mère se sera désagrégée de chagrin comme une vieille chaussette pourrie. Jusqu’au jour où tes voisins (avec lesquels tu n’as jamais entretenu la moindre relation), incommodés par l’odeur, appelleront la police qui se fera une joie de venir défoncer ta porte et découvrir avec horreur ton cadavre grouillant d’asticots.

Aussi, parce que je te suis reconnaissant des efforts que tu as consentis pour arriver jusqu’ici, je vais te le dire.

En fait, aussi étrange et stupéfiant que cela puisse paraître, je tenais cette information capitale de la bouche même de Sam Girard, cuisinier hors pair et ancien des forces spéciales de l’armée de terre, qui se trouvait résider non loin du Palazzo Cristal Club. C’est là, depuis le bar où il avait coutume de se désaltérer et refaire le monde avec d’autres habitués, assez rugueux pour la plupart, qu’il les avait distinctement aperçus en train de se livrer à des activités qui avait plus à voir avec les bordels de Buenos Aires qu’avec le Rio de la Plata et ses danses locales. Prudent, et habitué au secret comme la plupart des militaires, il ne s’en était ouvert qu’à moi. Pourquoi moi ? Eh bien mais tout simplement parce que je suis une personne de confiance, comme vous n’aurez sans doute pas manqué de le remarquer. Si vous êtes dans la détresse, affective ou autre, je saurai trouver les mots justes pour remettre sur les rails la locomotive cabossée de votre existence. Je dis ça, mais je n’ignore pas que c’est dans la nature des gens de raconter leur vie au premier venu. Ils ne vous voient pas comme un être humain, mais comme une benne à ordures dans laquelle déverser leurs immondices. Ce sont les mêmes qui abandonnent leurs papiers gras sur la chaussée et leurs mégots de clopes dans la forêt, au risque de foutre le feu. Peu importe l’environnement, l’important est de se débarrasser à bon compte de ses déchets. Et si vous avez la faiblesse de prêter ne serait-ce qu’une vague oreille aux divagations de l’un d’entre eux, vous pouvez être certain qu’ils ne seront pas long à faire la queue devant chez vous pour faire leurs besoins sur votre paillasson. Pour ce qui est des soi-disant secrets qu’ils sont censés garder, dites-vous bien que c’est un fardeau dont ils se débarrasseront à la première occasion. Ils vous refilent la patate chaude et reprennent leur petite vie tranquille comme si de rien n’était, la conscience aussi légère et vierge qu’au jour de leurs premiers vagissements. Libre à vous de la refiler à quelqu’un d’autre, qui à son tour la refilera à quelqu’un d’autre, et ainsi de suite jusqu’à ce que tout le monde soit au courant. Au courant de quoi, c’est une autre question. Car il va de soi, au cours du périple mouvementé qui a été le sien, que le récit n’a plus grand-chose à voir avec ce qu’il était à l’origine.

Dans le cas présent, la patate chaude en question était une première main, passée directement du producteur au consommateur, de la terre à l’assiette. Rien à voir avec ces cochonneries bourrées d’additifs tous plus toxiques les uns que les autres. Sam l’avait récoltée à la source, et me l’avait livrée telle qu’elle, sans la nettoyer, lui faire subir le moindre traitement qui aurait pu altérer sa texture et sa saveur. Elle avait de la mâche, du goût, et le mieux, pour les préserver, était de la cuisiner le moins possible. Certains disent qu’on peut sublimer le produit, le rendre encore meilleur qu’il n’est en réalité, mais si une chose est parfaite en soi, ce que les mêmes ne cessent de répéter avec les yeux révulsés et des trémolos dans la voix, alors je ne vois pas l’intérêt d’en rajouter, sinon celui de s’en attribuer indûment les mérites grâce à un tour de passe-passe discutable. Par exemple, le type qui essaie de vous vendre un jambon prétendument incomparable, va d’abord vous beurrer la tartine d’une épaisse couche de superlatifs concernant son produit, sans la moindre pudeur. Il pourrait se contenter de vous en couper une tranche et attendre patiemment votre réaction. Mais non, il prépare le terrain, vous assure que vous allez participez à une expérience unique dont vous sortirez à jamais transformé, vivre un moment de grâce absolue, toucher du doigt les arcanes de la mystique plotinienne et approcher au plus près des plus grands mystères qui agitent l’humanité depuis la nuit des temps. Vous êtes l’élu, celui qui a été choisi pour assister à la Révélation. À tel point, même si vous le pensez profondément, que vous hésiterez à lui dire que son soi-disant merveilleux jambon de pays n’est rien d’autre qu’une merde infâme indigne du plus vil clébard. Pire encore, vous serez dans l’obligation d’en faire l’acquisition alors que vous n’en avez pas la moindre envie. Et comme vous aurez été bien reçu (c’est tout juste si on ne sera pas allé vous chercher au milieu de la rue pour vous traîner dans la boutique), qu’on vous aura déroulé le tapis rouge, traité comme une personnalité de premier plan, vous culpabiliserez d’autant plus de ne pas souscrire à l’offre proposée. Celle-ci, du reste, sera largement prohibitive, pour ne pas dire totalement déconnectée de la réalité, mais c’est un abus dont vous ne pourrez hélas faire mention sans passer aussitôt pour un effroyable radin, un pique-assiette qui profite de la générosité de ses hôtes pour s’empiffrer à bon compte. Ainsi, vous aurez bel et bien assisté à la Révélation de votre propre connerie, la facilité avec laquelle il est possible de vous la fourrer bien profond. Voilà pourquoi, lorsque vous êtes en affaires, et c’est un petit conseil que vous donne comme ça en passant, partez toujours du principe que votre partenaire est une ordure qui ne pense qu’à vous arnaquer, fera tout pour endormir votre vigilance et se barrer avec la caisse à la première occasion. Ne faites confiance à personne, à commencer vous-même, car vous êtes et serez toujours votre pire ennemi, d’autant plus dangereux qu’il est enclin à tout vous pardonner, y compris vos plus monumentales erreurs.

Donc, pour en revenir à notre affaire, je ne pouvais pas répondre à la question de Greg (Hein ? Quoi ? Comment ? On ? Qui ça, on ?) autre chose que : Désolé, mais je ne peux pas te le dire.

\textsc{Greg} : Je ne sais pas qui t’a raconté ça, mais je peux t’assurer qu’il n’y a rien entre Bérénice et moi. Crois-le ou non, mais c’est tout à fait par hasard qu’on s’est retrouvé dans ce club de danse.

\textsc{Moi} : Dans ce cas, je trouve étrange que tu ne m’en aies jamais rien dit.

\textsc{Lui} : L’occasion ne s’est pas encore présentée, voilà tout. C’est tout récent, je te le rappelle.

\textsc{Moi} : Récent ou pas, je pense que si tu avais rencontré Bérénice dans un club de danse, tu n’aurais pas manqué de m’en faire part. À moins, bien sûr, que tu aies une bonne raison de ne pas le faire.

\textsc{Greg}, aussi convaincant qu’un politicien qui affirme n’avoir jamais eu connaissance des agissements de Jeffrey Epstein\nf{Jeffrey Epstein (1953--2019), financier américain reconnu coupable de trafic sexuel de mineures. Il fut arrêté en 2019 et retrouvé mort dans sa cellule dans des circonstances controversées. Ses réseaux impliquant de nombreuses personnalités politiques et économiques mondiales alimentèrent de multiples théories et enquêtes judiciaires. \source{fr.wikipedia.org/wiki/Jeffrey\_Epstein}} alors que son nom revient plus de cinq mille fois dans les dossiers afférents, et n’être allé à Little Saint James\nf{Little Saint James est une île privée des Îles Vierges américaines, surnommée «~l’île de Pédophile~», que Jeffrey Epstein acquit en 1998. Elle servit de cadre à de nombreuses rencontres avec des personnalités de premier plan et à des actes de trafic sexuel. Après l’arrestation d’Epstein, l’île fut saisie par les autorités américaines. \source{fr.wikipedia.org/wiki/Little\_Saint\_James}} qu’une ou deux fois, et encore uniquement pour jouer au tennis, prendre des bains de mer et pêcher le marlin : Je suis vraiment très triste.

\textsc{Moi} : Ah bon ?

\textsc{Lui} : Ben oui, que tu puisses penser une chose pareille de moi.

\textsc{Moi} : Pourquoi tu ne dis pas tout simplement la vérité ?

\textsc{Lui} : Quelle vérité ?

\textsc{Moi} : Que tu te tapes la femme de Titus. En plus de Lou et peut-être une demi-douzaine d’autres, va savoir. À partir de là, on peut tout imaginer.

\textsc{Lui} : Comme quoi, par exemple ?

\textsc{Moi} : Eh bien, par exemple, que Bérénice et toi avez imaginé un plan machiavélique pour vous débarrasser du mari gênant. Vous louez les services de cette prétendue femme de ménage et Gardienne de mes couilles, qui est en réalité une redoutable tueuse à gage, et vous faites croire à tout le monde qu’elle et Titus sont partis couler des jours heureux quelque part sous les Tropiques.

\textsc{Greg}, avec pour la première depuis longtemps une vague touche de sincérité dans le regard : C’est du grand n’importe quoi !

Bon, je reconnais qu’il m’arrive parfois d’avoir des idées farfelues, sinon franchement débiles, et celle-ci figurait en bonne place dans le haut du panier. Si je n’avais pas cette profonde connaissance de moi-même, fruit d’une longue et attentive fréquentation de mon humble personne, j’en viendrais presque à douter de l’état de ma santé mentale. Mais s’il faut douter à tout prix, j’aime autant douter de celle des autres. Greg est un ami et le restera, même s’il couchait avec sa propre mère, ce qui ne risque pas d’arriver puisqu’elle est morte, paix à son âme. Mais peut-être l’ont-ils fait, se sont-ils vautrés tels des hippopotames en rut dans la boue de l’inceste, des sangsues boursouflées de désir dans la vase abjecte de la consanguinité. Si tel est le cas, je n’en ai jamais rien su et préfère n’en jamais rien savoir. Cela dit, Greg m’avait souvent donné cette bizarre impression de traverser l’existence sur la pointe des pieds, en catimini. On ne le sentait pas réellement investi dans les affaires du monde. Je n’irais pas jusqu’à dire qu’il se foutait royalement de tout, ce serait sans doute exagéré, mais il y avait tout de même chez lui une forme de dilettantisme qui flirtait parfois assez dangereusement avec l’absentéisme pur et simple. Je pense que ça ne lui aurait fait ni chaud ni froid si on lui avait annoncé qu’il allait mourir demain. Ses traits ne seraient pas décomposés, ni sa gorge nouée. Contrairement à ces condamnés à mort qui s’accrochent désespérément à la vie et pleurent toutes les larmes de leur corps au moment fatidique, abandonnant au pied du gibet les derniers oripeaux de leur dignité, c’est le cœur léger qu’il aurait offert son cou à la corde du bourreau. Un jour, on lui avait fait cadeau de la vie. C’était, comme à nous tous, son premier cadeau d’anniversaire. Sauf que pour lui, le cadeau en question ressemblait davantage au jackpot d’une loterie infernale dont il aurait été non pas le grand gagnant, l’heureux élu, mais le grand perdant, la principale victime. Quand certains passent le restant de leurs jours à les remercier pour leur avoir donné la vie, permis de gambader sur terre tels des faons émerveillés, d’autres vouent leurs géniteurs aux gémonies et n’aspirent qu’à leur faire payer le prix de cette trahison. Quand on aime vraiment quelqu’un, on ne lui donne jamais le jour. Certes, on n’aura jamais le plaisir de faire sa connaissance, contempler sur ses traits la marque de sa glorieuse lignée, mais au moins on n’aura pas son sang sur les mains. Tel un Marc-Antoine Jacquinot, avec lequel il partageait de nombreux traits de caractère, Greg n’avait jamais jugé utile de faire valoir ses droits à la paternité. Tous deux étaient, quoi qu’on en pense, en parfaite adéquation avec cette tendance de la jeunesse actuelle qui consiste à ne plus faire d’enfant, au point que d’aucuns commencent à s’en inquiéter sérieusement. C’est chronophage, ça coûte bonbon et ça ne rapporte le plus souvent que des emmerdements. D’autre part, avec les familles sans cesse recomposées, faites de bric et de broc, la notion de descendance perd de sa pertinence. D’autant qu’il ne sera bientôt plus nécessaire de payer de sa personne, la reproduction pouvant être assurée in vitro à partir d’une sélection aléatoire d’échantillons prélevés sur l’ensemble de la population, avec tous les avantages d’un brassage ethnique intempestif. Et sans doute même qu’un jour, ovule et spermatozoïde pourront être fabriqués de toutes pièces en laboratoire, ce qui permettra non seulement de s’affranchir du concept de race, problématique s’il en est, mais aussi d’exercer un contrôle total sur la démographie, tant sur la plan de la quantité que la qualité. Il est loin le temps où on accouchait dans la douleur, et où la question se posait souvent de savoir si on devait sauver la vie de la mère ou celle de l’enfant. Ce choix, il faut bien le reconnaître, était rarement favorable à la parturiente, surtout si l’enfant à naître était un mâle (pour un bien, en quelque sorte). Enfin, le monde fait aujourd’hui figure d’un tel bourbier que beaucoup rechignent à y plonger d’innocentes créatures qui n’ont rien demandé à personne. À quoi bon les envoyer au casse-pipe, les condamner à livrer une guerre perdue d’avance. Et surtout, à quoi bon rajouter du malheur au malheur quand il y a déjà de par le monde des tas d’orphelins qui crèvent la gueule ouverte dans le caniveau, sont livrés pieds et poings liés à une horde de prédateurs qui abusent d’eux en toute impunité. Bien sûr, la plupart d’entre eux sont des boules de haine et de ressentiment difficiles à gérer, mais l’heure est venue de se détacher de ces vieux principes éculés de filiation et de maternité. La reproduction doit être assurée de façon mécanique et désintéressée. Après tout, faire des enfants n’est pas un investissement à long terme, une assurance-vie pour de futurs vieux qui n’ont pas envie de crever seuls comme des chiens dans un pavillon de banlieue décrépi. C’est une nécessité biologique, si on veut que l’espèce perdure, qui doit être appréhendée de façon scientifique et non plus sentimentale et affective, avec la niaiserie résignée de ces jeunes parents qui endurent H24 les réflexions idiotes de leur entourage radotant. Notre existence n’est que le fruit du hasard, et nos enfants ne nous appartiennent pas davantage que nous ne nous appartenons à nous-mêmes. Ton corps t’appartient… Mon cul, oui ! C’est pas comme si tu étais allé l’acheter au supermarché du coin, et tu auras beau l’entretenir aussi bien sinon mieux que ta baraque ou ta bagnole, ça ne changera rien à l’affaire. Si tu peux toujours revendre ou refiler ta baraque et ta bagnole à tes enfants, personne ne voudra de ton corps. Ta carcasse sera dévolue aux vautours et tu te feras ratisser jusqu’à l’os, à l’exception bien sûr de tes dents en or, prothèses de hanche et autres babioles imputrescibles qui seront récupérées par la communauté. Car n’oublie jamais ceci, ma frère, mon sœur ou qui que tu sois : ton corps est la propriété de Dame Nature qui te l’a généreusement (plus ou moins, certains auraient des raisons de se plaindre) prêté pour te servir d’enveloppe physique pendant ton séjour sur terre. Il faut le restituer le jour de ta mort, sensiblement dans l’état où tu l’as trouvé le jour de ta naissance, c'est-à-dire peu fonctionnel et totalement dépendant des autres pour assurer sa survie. Sauf que cette fois, les autres n’ont pas la moindre envie de te donner la becquée, te porter sur leurs épaules et te torcher le cul. Ils n’ont plus envie de te courir après quand tu t’enfuis à quatre pattes à travers la maison en ricanant comme un dingue et lâchant des caisses à tout bout de champ. Ils sont fatigués de faire tes courses, hurler parce que ton sonotone déconne, te ramasser parce que tu n’arrêtes pas de tomber (ça te rapproche de la tombe), aller te récupérer au commissariat parce qu’on t’a encore chopé en train de te balader à poil dans la rue, taguer des obscénités sur les murs et pisser sur les devantures de magasins. Ils n’ont qu’une hâte, c’est que tu casses ta pipe et arrêtes de faire chier le monde avec tes jérémiades. Même tes gosses, que tu appelles madame ou monsieur parce que tu ne les reconnais plus, commencent à trouver le temps long. Eux et leurs propres enfants auraient mieux à faire que de passer leurs dimanches à l’EHPAD, mais comme ils savent que tu es capable de les déshériter sur un simple coup de tête, ils continuent à venir pour ne pas risquer de passer à côté du pactole. Mais toi tu t’en fous, parce que tu as cinq ans dans ta tête, et que tu sais que ton grand âge te permet de faire à peu près tout et n’importe quoi sans que personne n’ose lever le petit doigt. Alors tu vas voir le caïd du coin, met de la schnouff dans ta chicorée et t’achètes un flingue pour braquer une banque. Quand les flics arrivent, tu défourailles à tout-va, mais personne ne riposte parce que ça ferait désordre de buter un vieux qui n’a plus toute sa tête. On attend que le chargeur soit vide, puis on vient te chercher pour t’escorter gentiment jusqu’à l’asile le plus proche, asile dont tu ne tarderas d’ailleurs pas à t’évader pour mettre à nouveau la ville à feu et à sang, passer en justice pour dégradation de bien public, et faire marrer tout le tribunal en te foutant de la gueule des juges et désavouant publiquement ton avocat pour cause d’incompétence chronique et inculture caractérisée.

Bref, les jeux sont faits à notre insu, et nous ne sommes que des billes qui virevoltent sur la table de la roulette en espérant tomber sur le bon numéro. Beaucoup, aujourd’hui, semblent assez peu motivés pour participer à ce qui leur apparaît de plus en plus comme un jeu de dupe, une vaste supercherie. La reproduction ne fait plus recette, la contraception bat son plein. Chez les femmes principalement, qui sont en première ligne. Si elles éprouvaient jadis une certaine fierté à tripler de volume en neuf mois et se traîner comme des vaches à lait en priant le ciel que tout se passe bien au moment de l’accouchement, elles ne semblent plus aujourd’hui très réceptives à la démarche. Physiquement parlant, elles préfèrent le style Coca-Cola au style Orangina, la taille de guêpe à l’embonpoint gravidique. Dans le monde moderne, biberon, couche, siège-bébé, poussette et dépression ne font plus rêver personne, les gens ayant autre chose à foutre que passer leur temps à pouponner. La grossesse, vécue comme un parcours du combattant digne des Forces Spéciales dans l’enfer djihadiste du sanctuaire de Tofagala, est une chose beaucoup trop sérieuse pour être confiée à des gens dont ce n’est pas le métier. Seules des femmes surentraînées, au mental d’acier, peuvent se permettre d’affronter cette épreuve sans s’exposer aux ravages du stress post-traumatique. Donner la vie, au même titre que la prendre, exige des compétences qui ne sont pas à la portée du premier venu. Face aux difficultés croissantes de l’existence, la progéniture elle-même semble accessoire, un luxe dispensable réservé à une élite qui a les moyens de ne pas s’impliquer dans la vie familiale et surtout faire élever ses enfants par d’autres, des professionnels spécialement formés pour gérer les parcours scolaires, les conflits de cour de récréation, les chagrins d’amour et les crises d’adolescence.

Voilà pourquoi, pour en revenir à Greg, même s’il brûlait d’une quelconque flamme pour Bérénice, ce qui était fort probable à moins que Sam n’ait été abusé par ses sens (le fait est qu’il avait tendance à picoler, en plus de certaines mauvaises habitudes toxicologiques contractées au cours de ses années de baroud au sein de la Légion), j’avais tout lieu de penser que ce modeste incendie ne survivrait pas à la première averse.

Un flic en uniforme s’est pointé pour me faire part d’une nouvelle de la toute première importance, susceptible d’éclairer l’affaire en cours d’un jour radicalement nouveau. Il avait mis la main, en jetant négligemment un œil (réflexe professionnel) dans le coffre de la Mini, sur une importante quantité de ce qu’il est convenu d’appeler un produit stupéfiant. En l’occurrence, il s’agissait de toute évidence de cet ester méthylique de benzoylecgonine plus connu sous le nom de chlorhydrate de cocaïne, ou cocaïne tout court, alcaloïde tropanique très recherché pour ses propriétés psychoactives. À vue de groin, il y en avait une bonne dizaine de kilos, soit une valeur marchande avoisinant les sept cent mille euros. Il va sans dire que la présence de ce chargement pour le moins compromettant expliquait à lui seul la conduite (dangereuse) de la très belle et sulfureuse Repentance Whittingham, la femme de ménage la plus rapide du monde qui ne se contentait apparemment pas de faire succinctement le ménage dans un hôtel de luxe pour personnes de couleur, hôtel de luxe qui d’ailleurs, au vu des récents événements, n’était peut-être pas seulement, pour reprendre mot pour mot l’excellente définition du Larousse, le respectable «~établissement commercial mettant à la disposition d’une clientèle itinérante des chambres meublées pour un prix journalier~» (et excessivement élevé, ajouterai-je) qu’il prétendait.

Voilà, c’est tout pour l’instant.

Tout ce que je puis ajouter, à l’heure où j’écris ces lignes, c’est qu’on est toujours sans nouvelles de Titus Beaugendre.
%%% 
%%% Est-il besoin de préciser que je m’associe pleinement à sa femme et ses enfants, dont je partage l’affliction, pour dénoncer l’indolence, sinon l’inaction des pouvoirs publics, et exiger que toute la lumière soit faite au plus vite sur %%% cette affaire.
%%% 
%%% Naturellement, vous serez les premiers informés dès que j’en saurai un peu plus à ce sujet.
%%% 
%%% En attendant, je vous souhaite bon vent (dans tous les sens du terme, y compris bien sûr l’expulsion plus ou moins sonore et trébuchante de ces gaz intestinaux qu’il serait dangereux de conserver indéfiniment à l’intérieur de soi).
%%% 
%%% \textsc{PS} : Si je n’aime pas les fins, toujours décevantes et redondantes, inutiles, je n’aime pas non plus les titres, ennuyeux, racoleurs et commerciaux, profondément réducteurs et indigents.
%%% 
%%% Si vous lisez MADAME BOVARY sur une couverture, par exemple, vous viennent aussitôt en tête des adjectifs peu flatteurs comme bovin ou bavarois, lesquels, à moins d’être féru de produits laitiers ou de culture germanique, ne donnent %%% guère envie d’ouvrir le livre.
%%% 
%%% De la même façon, si vous lisez DRACULA, vous viennent aussitôt en tête des images à caractère sexuel pour le moins déplacées, même si, je vous l’accorde, les canines du vampire qui pénètrent dans la chair tendre d’une gorge féminine ne %%% sont pas totalement exemptes de sensualité.
%%% 
%%% J’avais, dans un premier temps, pensé appeler ce livre L’ÉCUME DES ABAT-JOUR, en hommage à Boris Vian\nf{Boris Vian (1920--1959), écrivain, poète, musicien et ingénieur français, figure majeure de Saint-Germain-des-Prés. Auteur %%% notamment de \textit{L’Écume des jours} (1947), il signa également sous le pseudonyme Vernon Sullivan des romans noirs sulfureux. Son titre \textit{L’Écume des jours} inspira ici le calembour \textit{L’Écume des abat-jour}. \source{fr.%%% wikipedia.org/wiki/Boris\_Vian}}, auteur que je connais de nom et de réputation, mais dont je confesse, à ma grande honte, n’avoir jamais lu le moindre livre. Je ne pouvais donc pas, ne serait-ce que par éthique littéraire, retenir ce %%% titre, assez rigolo par ailleurs.
%%% 
%%% J’ai ensuite pensé l’appeler LES GLAPISSEMENTS DE L’ENNUI, en référence aux CROASSEMENTS DE LA NUIT (titre français assez disgracieux de STILL LIFE WITH CROWS) de Douglas Preston\nf{Douglas Preston (né en 1956) et Lincoln Child (né en %%% 1957) sont deux romanciers américains qui collaborent depuis \textit{La Relique} (1995). Spécialisés dans les thrillers mêlant policier, fantastique et horreur, ils ont créé le personnage de l’agent du FBI Aloysius Pendergast. \textit%%% {Still Life with Crows} (2003), traduit en français sous le titre \textit{Croassements de la nuit}, est le quatrième roman de cette série. \source{fr.wikipedia.org/wiki/Douglas\_Preston\_et\_Lincoln\_Child}} \& Lincoln Child, deux %%% auteurs de romans policiers teintés de fantastique que j’apprécie particulièrement, même si leur prose à quatre mains est loin d’être un modèle d’intelligence et de créativité, tant sur la forme que le fond (il est bon, parfois, de %%% s’autoriser une certaine dose de médiocrité). J’ai rapidement laissé tomber l’idée, divertissante, certes, mais sans grand intérêt.
%%% 
%%% J’ai alors pensé, de façon plus impersonnelle, l’intituler LIVRE UN ou MON PREMIER ROMAN, mais la connotation enfantine m’est apparue par trop manifeste.
%%% 
%%% \enlargethispage{\baselineskip}%
%%% Si j’ai finalement, en désespoir de cause, résolu de tisser la métaphore alimentaire, c’est parce que manger (et boire) reste une de mes activités favorites. C’est aussi, accessoirement, la garantie de ne plus avoir à se creuser le chou %%% pour trouver un titre, la gastronomie internationale ne manquant pas de recettes aussi populaires que savoureuses. J’aurais aussi bien pu appeler ce bouquin CASHER BLUES, en référence subtile et odorante au CASHEL BLUE, ce fromage %%% irlandais à pâte persillée, ou JAMBALAYA, best-seller de la cuisine cajun, TOUOP-TOUOP KELONG, recette camerounaise à base de banane plantain et poisson fumé, ou encore MOROS Y CRISTIANOS, du nom de ce délicieux plat cubain à base de %%% riz blanc et haricots noirs, sans que personne n’y trouve rien à redire, ni trahir aucunement la nature conviviale et épicée du présent ouvrage.
%%% 
%%% 
%%% 