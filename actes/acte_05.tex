
\noindent Croyez-le ou non, mais Aymeric Jégou et Tacito Cerqueira étaient deux des pires enfoirés qui aient jamais posé le pied sur notre bonne vieille terre, laquelle en avait pourtant vu des vertes et des pas mûres depuis les quelques milliards d’années qu’elle s’échinait à tourner autour du soleil. Ils travaillaient tous deux pour la même entreprise de déménagement, entreprise qu’ils avaient fondées à deux et dont ils étaient les deux principaux actionnaires et pour ainsi dire les deux seuls employés, hormis une secrétaire répondant au nom de Jessica Millet, sans qu’il soit établi de près ou de loin que son patronyme ait un quelconque rapport avec l’auteur de l’angélus, des glaneuses, de la sieste, du bouquet de marguerites, de l’homme à la houe et du retour du troupeau, autant de valeurs simples dont on appréhende aujourd’hui pleinement l’évidente nécessité.

Jégou et Cerqueira étaient donc, chacun dans leur genre (assez diamétralement opposé, on va le voir), des individus auxquels il n’était pas spécialement conseillé de s’amuser à chercher des poux dans la tête (de toute façon il aurait été difficile d’en trouver vu qu’ils avaient tous les deux la boule à zéro). Sans être à proprement parler une étoile brillant au firmament de l’intelligence, un phare dans la nuit, Jégou était loin d’être bête et pouvait même se montrer retors, tandis que Cerqueira était pourvu d’une capacité de réflexion aussi restreinte que les droits de la femme (aujourd’hui à peine supérieurs à ceux du hérisson) en Afghanistan depuis la prise de Kaboul par les talibans, ou encore la quantité de tissu sur le corps d’une danseuse du Crazy Horse.

Quand nos deux compères n’étaient pas en train de soulever des caisses et trimballer des meubles dans les escaliers, ils officiaient gracieusement en tant que membres du service d’ordre d’un certain parti politique d’extrême-droite dont j’aimerais autant, pour des raisons de dignité personnelle, éviter de dire le nom. Sachez seulement qu’il y est question de la partie haute du visage, située entre les yeux et le haut du crâne, et de quelque chose qui renvoie à l’idée de naissance en général, étymologiquement parlant, mais plus précisément de naissance sur un territoire donné, régi par des règles similaires, des lois auxquelles sont subordonnés tous les habitants dudit territoire, lesquels sont censés parler la même langue, partager des valeurs similaires fondées sur l’expérience d’une histoire commune plus ou moins ancienne, et éprouver une certaine fierté, sinon une réelle satisfaction, à vivre les uns avec les autres. Mais si vous voulez réellement mon avis, eh bien je pense que tout ça c’est surtout de la merde en barre, et que ce genre de structure tient essentiellement sur des faux-semblants et des malentendus.

Sur le plan physique, Jégou et Cerqueira étaient aux antipodes l’un de l’autre. Autant le premier, de nature (cela ne se voyait plus trop maintenant du fait qu’il avait le crâne rasé, au même titre que son comparse), était blond aux yeux bleus, correspondant en tout point au type aryen tel que défini par Ernst Brestrich dans son «~Traité sur les races~», ouvrage controversé aujourd’hui introuvable en librairie (mais vous pourrez le trouver sur Onionland, si vous tenez absolument à vous plonger dans ce torchon), autant le second était brun aux yeux noirs, doté d’une pilosité portugaisoïde assez aux antipodes des canons de la beauté nordique. Cela dit, un physique de ce type, avec monosourcil des plus broussailleux qui plus est, n’avait pas empêché un Rudolf Hess, pourtant notoirement paranoïaque, d’accéder aux plus hautes fonctions du Reich. Pas pour longtemps, il est vrai : en 41, ce triste sire décide de s’envoler vers l’Angleterre pour soi-disant négocier un traité de paix avec Churchill, le roi, la reine et compagnie, personnalités qu’il ne rencontrera bien entendu jamais et traité qui ne sera tout aussi bien évidemment jamais signé. Par contre, il aura tout le loisir de faire connaissance avec le sens de l’hospitalité anglaise, notamment la qualité d’accueil des établissements psychiatriques dans lesquels il sera invité à résider jusqu’à la fin de la guerre, avant d’être expédié à Nuremberg par le premier charter, jugé, condamné à perpète et incarcéré à la prison de Spandau en compagnie de six autres joyeux drilles de son acabit, j’ai nommé les tristement célèbres Karl Dönitz (désigné par Hitler lui-même alors à l’agonie comme son successeur officiel à la tête du Saint-Empire, à la place d’Himmler qui s’était fait la malle et du gros Göring complètement défoncé à la codéine), Erich Raeder (commandant en chef de la Kriegsmarine, libéré en 55 pour raisons de santé), Konstantin von Neurath (ministre des Affaires étrangères, SS-Obergruppenführer et gouverneur de Bohême-Moravie, libéré en 54 pour raisons de santé), Walther Funk (président de la Reichsbank et blanchisseur en chef de l’or et l’argent piqués aux Juifs, libéré en 57 pour raisons de santé lui aussi), Albert Speer (grand architecte du Reich et concepteur du Reichsparteitagsgelände de Nuremberg, libéré en 66 à l’issue de sa peine), et Baldur von Schirach (chef des Jeunesses hitlériennes, amateur d’art et poète à ses heures, je n’en dirai pas davantage, libéré en 66 à l’issue de sa peine). En 1989, à l’âge de 93 ans, Hess se décide enfin à mettre un point final au roman de gare de sa pitoyable existence. Après la libération de Speer et von Schirach en 66, il reste le seul et unique pensionnaire de la prison de Spandau, laquelle sera rasée après sa mort pour éviter d’en faire un lieu de culte pour les nostalgiques du Troisième Reich. Le fils de Hess et filleul de Hitler, Wolf Rüdiger, au même titre que Gudrun Himmler et Edda Göring, a tout fait pour réhabiliter la mémoire de son crétin de père, allant jusqu’à prétendre qu’il avait été assassiné par la CIA pour des raisons politiques aussi obscures que farfelues. Si les autorités ouest-allemandes avaient voulu s’en débarrasser, ce qui aurait pu se comprendre vu qu’elles payaient ses frais de séjour (soit le prix d’une single room au Lutetia pour une cellule de 6 m², multipliez ça par 40 ans à 365 jours et vous aurez une idée de ce que cet enfoiré a coûté à la communauté~-- plus de 20 millions d’euros, au cas où vous n’auriez pas de calculette sous la main), elles n’auraient pas attendu vingt ans pour le suspendre au bout d’un fil électrique. Quant à la CIA (ou les SAS suivant les versions), je pense qu’elle avait autre chose à foutre qu’éliminer un vieillard sénile dont tout le monde se fichait comme de l’an 40. Et malgré tout ça, figurez-vous que le petit-fils, Wolf Andreas, négationniste convaincu, n’a rien trouvé de mieux à faire que de reprendre le flambeau à la mort de son père. Bref, il se passe des choses bizarres dans le cerveau humain, et je pense qu’on n’a pas fini de se creuser les méninges pour essayer de comprendre ce qui se trame à l’intérieur de cette machine infernale.

Et, pour en revenir à ce que je racontais en début de paragraphe, non seulement Jégou s’inscrivait parfaitement dans la catégorie des blonds aux yeux bleus, un blond presque blanc avec des yeux d’un bleu plus bleu que bleu, mais il était aussi sec et nerveux que Cerqueira était humide et détendu. Par «~humide~» j’entends perpétuellement moite, sinon trempé, car il transpirait à grosses gouttes sous l’épaisse couche de poil qui tapissait la majeure partie de son corps. Il était aussi beaucoup plus grand et corpulent que son acolyte, de sorte que tandis que l’autre passait son temps à s’exciter, toujours prêt à mordre comme le roquet hargneux qu’il était, lui conservait au contraire un calme à toute épreuve. Autrement dit, quand il démontait quelqu’un, le dépouillait os par os de chacun de ses membres, jouait de la guitare avec ses tendons et du xylophone avec ses dents, c’était pour ainsi dire sans haine ni violence, se contentant de l’écrabouiller méticuleusement avec les armes de destruction massive qui lui tenaient lieu de mains. Il ne paniquait jamais, ni même ne montrait, au plus fort des affrontements (à coups de barres de fer, manches de pioche, battes de baseball et parfois même couteaux et armes à feu) qui les opposaient régulièrement aux antifas, poseurs d’affiches trotskistes et autres anarcho-syndicalistes à la petite semaine, le moindre signe d’inquiétude. Des fois, on aurait presque dit qu’il s’emmerdait, n’éprouvait plus aucune joie à tuer, massacrer ses adversaires, aucune espèce d’intérêt à les réduire en purée, en faire de la pâtée pour chien. Accessoirement, il servait aussi de garde du corps à Aymeric Jégou, lequel, même s’il était particulièrement vicieux et ne s’embarrassait d’aucun scrupule dès lors qu’il s’agissait d’arracher des cris de souffrance à son prochain (et on pouvait alors lire sur son visage toute l’étendue de la satisfaction de ces cris lui apportaient), pouvait se montrer parfois physiquement un peu léger face à des montagnes de muscles qui faisaient deux fois sa taille et trois fois son poids, même s’il parvenait la plupart du temps à en venir à bout. Dans le cas contraire, Tacito intervenait pour remettre un peu d’ordre dans la mêlée, briser quelques mâchoires et fracasser quelques membres pour inciter les belligérants à plus de retenue.

Un soir qu’ils étaient en train d’écluser des chopes de Karlsbrau au Bouclier, bar tenu par un ancien skinhead du nom de Ronny Bell et connu pour être un repaire de fachos, Jégou, qui commençait à voir tout en double, avait dit à Cerqueira : Faut que je te dise un truc…

Cerqueira, qui pouvait avaler trois tonneaux de bière sans sourciller : Ouais, quoi ?

\textsc{Jégou} : On a décidé de casser du pédé.

\textsc{Cerqueira} : Qui ça, on ?

\textsc{Jégou} : Milo, Noé et moi.

Pour info, Milo Monteil et Noé Desmarais étaient deux abrutis de la même trempe que Jégou et Cerqueira.

\textsc{Cerqueira} : C’est pas ce qu’on fait déjà ?

\textsc{Jégou} : Si, mais on a décidé de passer à la vitesse supérieure.

\textsc{Cerqueira} : Ah ouais ?

\textsc{Jégou} : Ouais. On a formé un petit groupe d’action, les Disciples de la Colère. Parce que nous on est vachement en colère, on a carrément la haine. T’es pas en colère, toi ?

\textsc{Cerqueira} : Si, si, bien sûr, je suis grave en colère.

\textsc{Jégou} : Non, t’es pas vraiment en colère parce que t’es qu’un gros pédé de suceur de queues !

\textsc{Cerqueira} : Non, je suis pas pédé !

\textsc{Jégou} : Si, t’es pédé, tu le sais, je le sais, mais t’es mon pote et tout le monde n’est pas obligé de le savoir. Vaudrait mieux pas, d’ailleurs, sinon je suis pas sûr qu’on pourrait te sortir le cul des ronces. Mais c’est pas grave, je t’en veux pas, t’es l’exception qui confirme la règle, la preuve que le Mouvement est ouvert à tout et qu’on n’est pas les enfoirés sectaires que tout le monde raconte.

\textsc{Cerqueira} : Et vous allez faire quoi ?

\textsc{Jégou} : Rien, ma grosse. On en a juste plein le cul de voir toutes ces tafioles se bécoter en pleine rue et se balader en se tenant par la main. On est en France, merde, on a des valeurs !

\textsc{Cerqueira} : Oui, mais moi aussi je suis pédé.

\textsc{Jégou} : Toi, c’est pas pareil. T’es pédé, okay, mais c’est la société qui t’a obligé à le devenir.

\textsc{Cerqueira} : C’est vrai que je me dégoûte moi-même.

\textsc{Jégou} : Bien sûr, t’as honte de toi, t’es la victime de ce qu’on t’a obligé à devenir. Si tu fais tout pour arrêter d’être pédé, redevenir un mec normal, tu te sentiras beaucoup mieux et tout rentrera dans l’ordre.

\textsc{Cerqueira} : Tu crois ?

\textsc{Jégou} : J’en suis sûr, mon pote.

\textsc{Cerqueira} : Ouais, t’as sans doute raison. Comment t’as dit que ça s’appelait, votre truc ?

\textsc{Jégou} : Les Disciples de la Colère.

\textsc{Cerqueira} : Ça sonne bien.

\textsc{Jégou} : Grave ! C’est moi qui l’ai trouvé. Tu veux en être ?

\textsc{Cerqueira} : Ça dépend. Vous faites quoi, au juste ?

\textsc{Jégou} : On a décidé de s’attaquer aux trans, de leur faire bouffer leurs prothèses mammaires !

\textsc{Cerqueira} : J’ai rien contre les trans.

\textsc{Jégou} : T’as rien contre les trans ?

\textsc{Cerqueira} : Non. Un jour, mon père m’a dit qu’il ne pouvait plus supporter d’être un homme et qu’il avait décidé de changer de sexe. C’était pas du vice ou quoi, c’était juste qu’il se sentait comme une femme à l’intérieur.

Jégou, en faisant rapidement le signe de croix comme s’il avait vu le diable en personne : Seigneur Jésus !

Cerqueira, fixant sa chope avec son regard vide de ruminant dépressif : Le pire c’est que ma mère avait toujours rêvé d’être un homme. Ils s’aimaient toujours, mais ils voulaient inverser les rôles. Mon père est devenu ma mère et ma mère mon père.

\textsc{Jégou} : Quelle horreur !

\textsc{Cerqueira} : Pas tant que ça, on est resté une famille unie. Au début j’ai été un peu perturbé, j’ai fait quelques tentatives de suicide, et puis j’ai lu Mein Kampf et j’ai vu la lumière au bout du long tunnel noir dans lequel j’étais en train de ramper et étouffer.

\textsc{Jégou} : Et t’as pas envie de te venger ?

\textsc{Cerqueira} : Me venger de quoi ?

\textsc{Jégou} : De ce que t’ont fait subir tes parents. C’est à cause d’eux si t’es devenu pédé. C’est à cause d’eux que tu t’aimes pas et que tu te traites comme de la merde. Tu te rabaisses plus bas que terre.

\textsc{Cerqueira} : Ouais, je sais.

\textsc{Jégou} : Bon écoute, si t’as pas envie de casser du trans, c’est pas grave on s’en chargera Milo et moi. Je comprends que t’aies pas envie d’y toucher, les parents, c'est sacré. Sauf que nous, on est quand même obligés de s’en occuper, parce que si t’as bien lu Mein Kampf, la bible du national socialisme, tu sais comme moi que les trans sont pas des gens comme tout le monde, et que tous les gens qui sont pas comme tout le monde sont des dégénérés. Dieu a fait les hommes et les femmes avec une bite, des nichons et tout ce qui s’ensuit, et vouloir changer de sexe, ça revient à dire qu’il a mal fait son boulot. Et les gens qui osent dire que Dieu a mal fait son boulot sont forcément des dégénérés, des agents des forces du mal. Tiens, vendredi soir, on fait une petite virée avec les Disciples de la Colère. T’es le bienvenu si ça te dit.

\textsc{Cerqueira} : Okay, mais je touche pas aux trans.

\textsc{Jégou} : Je te ferai un mot, tu seras exempté pour raisons familiales.

\textsc{Cerqueira} : Je veux bien tabasser tout ce que vous voulez, les juifs, les cocos, les arabes et les pédés, mais pas les trans.

\textsc{Jégou} : Tu sais ce qu’on leur fait, nous, aux pervers et aux drogués qui salissent nos rues et violent nos gosses ?

\textsc{Cerqueira} : Non ?

Jégou, à voix basse : Tu le répéteras pas ?

\textsc{Cerqueira} : Dis pas de conneries !

\textsc{Jégou} : On les crame.

Une nuit, vers trois heures du matin, on a chopé Cerqueira en train de tapiner rue Armand Brunelle, habillé en femme.

Il y avait de la matière mentale qui pourrissait dans les plis de son cerveau et le poussait à faire des choses que sa propre morale réprouvait à cor et à cri.

Histoire de marquer le coup, j’ai fait quelques photos de notre homme en pleine action, la robe relevée jusqu’au nombril, le trou de balle bien en évidence, le genre de documents que les gens n’aiment pas trop savoir en liberté dans la nature.

Naturellement, on savait que quand il ne racolait pas le micheton dans les pissotières, il jouait les gros durs dans les services d’ordre d’extrême-droite avec une bande de détraqués qu’on avait depuis longtemps dans le collimateur. Jusque-là on laissait faire sans trop d’appréhension, dans le souci du respect des traditions et la liberté d’expression, mais on avait eu vent de certaines choses pas très reluisantes concernant leurs activités subalternes, la façon dont ils occupaient leur temps libre, et on avait, après mûre réflexion, décidé que le moment était venu de hausser le ton. Aussi lui ai-je clairement laissé entendre que ses petits copains ne seraient peut-être pas ravis d’apprendre que le meilleur d’entre eux menait ce qu’il est convenu d’appeler une double vie, nazi le jour, tapette la nuit.

Tout ça fleurait bon la schizo des familles, et quand je lui ai demandé s’il avait l’habitude de se flageller pour expier ses fautes, il a dégrafé sa robe à fleurs et m’a montré les cicatrices qui zébraient son dos. Je me suis dit (mon bon cœur me perdra) que ce type au physique d’ours des cavernes, de plantigrade prématurément arraché à la préhistoire, n’était peut-être pas totalement irrécupérable. Traduit dans le langage commun de la police nationale, cela signifiait qu’il n’était pas inévitable que ses petites combines sexuelles s’ébruitent à tout va, au risque de semer un vent de panique générale de nature à lui porter fortement préjudice. Je pouvais rester discret, aussi muet qu’une limace aphone sur une feuille de salade verte en plein soleil, si en échange il me concédait quelques informations utiles. Par exemple, on savait que son collègue de travail, Aymeric Jégou, n’était pas franc du collier, et le voir s’acoquiner avec deux ordures de compète comme Milo Monteil et Noé Desmarais n’était pas fait pour rassurer le staff technique de la kommandantur.

Avait-il, lui qui était dans le cénacle des chemises brunes et casques à pointe, des antisémites racistes, homophobes et fiers de l’être, entendu bruisser quelque rumeur suspecte à son sujet, reniflé l’odeur fétide de quelque flatulence le concernant ?

Il a hésité, minaudé, joué les pucelles effarouchées (ce qui était un peu fort de café pour un garage à bites de son espèce), mais quand je lui ai annoncé qu’un travelo brésilien avait été retrouvé cramé au milieu des bois, une vague lueur d’intérêt s’est allumée dans le fond de son regard de méduse anorexique.

Je précise que Zaahid, après avoir effectué la synthèse experte des éléments en sa possession, en était arrivé à l’irréfutable conclusion que le grand brûlé de la forêt de Pleimelding n’était autre que Tiago Alvarez. De plus, quelques jours plus tôt, on avait retrouvé un corps dans une fosse à purin au cours d’une intervention chez un éleveur de chevaux pédophile et cannibale. Titus et moi avions copieusement tabassé le pédophile, une espèce de connard à particule qui ne se prenait pas pour la moitié d’une merde, pour lui faire cracher le morceau. Cet enculé abusait sexuellement d’enfants du voisinage, avant de les tuer à coups de fourche et les découper en morceaux. Il avait trois congélateurs dans son sous-sol, tous pleins à ras bord de barbaque (sanglier, chevreuil, marcassin, perdrix, lièvre, bécasse et compagnie, le gars était chasseur et gastronome), dont des morceaux de gosses soigneusement emballés et étiquetés qui attendaient de passer à la casserole (les pauvres y passaient deux fois, en somme). Une fois qu’on lui a eu explosé la gueule dans les grandes largeurs (même sa marie-couche-toi-là de mère, complètement folle sur ses vieux jours, si elle n’était pas morte noyée quelques années plus tôt dans cette même fosse à purin, aurait été incapable de le reconnaître), de la Maîtraie (c’était son nom, Louis-Marie de la Maîtraie, du genre je te la mettrais bien où je pense, mignon petit page au sourire d’ange) a fini par reconnaître une certaine attirance pour la chair fraîche, mais a juré ses grands dieux qu’il n’avait rien à voir dans la mort du type qui se trouvait dans sa putain de fosse à purin. Depuis le décès de sa mère, il n’y mettait jamais plus les pieds, et jamais il ne lui viendrait à l’idée de mettre quelqu’un d’autre à sa place. Depuis le décès de sa très sainte mère, cette fosse à purin était un mausolée, un lieu de culte, un endroit sacré qu’il aurait mille fois préféré mourir que profaner. Apparemment, des ordures sans foi ni loi étaient venues noyer ce type dans sa fosse à purin à son insu, sans doute la nuit pendant qu’il dormait à poings fermés, l’estomac bien rempli après un bon repas de chair humaine. Car sans se vanter, il était un cuisinier hors pair. Une fois, il avait fait goûter sa terrine à des voisins dans l’affliction suite à la disparition d’une personne proche, des gens qu’il assistait dans leur détresse, sans leur dire que leur gamin de huit ans était l’ingrédient principal de la terrine en question, et ils avaient été unanimes pour dire que de toutes les terrines qu’ils avaient eu l’occasion d’ingurgiter au cours de leur vie, et dieu sait qu’ils en avaient ingurgité des tonnes et non des moindres, celle-ci était la meilleure et de loin. Après l’avoir croisé vingt secondes dans les couloirs de la PJ, passablement esquinté mais avec toujours cette même lueur de démence dans le regard, le psy avait conclu qu’il était dingue et qu’il n’y avait à priori aucune raison de ne pas le croire s’il disait n’être pour rien dans le cadavre de la fosse. Placement d’office en HP de haute sécurité avec camisole de force et chambre capitonnée, plus neuroleptiques à haute dose et séance quotidienne d’électrochocs, affaire classée.

Pour en revenir au cadavre en question, on n’a pas eu beaucoup de mal à l’identifier. Et pour cause, il avait encore ses papiers sur lui. Certains avaient souffert de leur séjour prolongé dans le purin, mais les cartes en plastique, après un brin de toilette, étaient pleinement exploitables. Elles appartenaient à un certain Abraham Botrel, cinquante-trois ans, folle notoire disparue deux mois plus tôt sans laisser d’adresse. Le mode opératoire n’était pas le même que pour Alvarez, mais le traitement peu gratifiant qui lui avait été réservé semblait indiquer qu’une même équipe de détraqués était à l’œuvre.

Moi, assis en face de Cerqueira dans le bureau réservé aux interrogatoires musclés (celui où les caméras de surveillance tombent régulièrement en panne), Titus debout près de la porte : Va falloir nous en dire un peu plus, mon petit Tacito.

\textsc{Cerqueira} : Pas devant le nègre.

\textsc{Moi} : Pardon ?

\textsc{Lui} : Depuis quand ils engagent des nègres dans la police ?

\textsc{Moi} : Je serais de toi, j’éviterais de la ramener. Mon collègue a parfois des réactions imprévisibles, surtout en présence de têtes de cons dans ton genre.

\textsc{Lui} : M’en fous ! Je veux bien parler, mais pas devant le nègre.

\textsc{Titus} : Qu’est-ce qu’elle dit, la pédale ?

\textsc{Cerqueira} : Je t’emmerde, négro.

Titus a fait un pas en avant, l’œil chargé de mauvaises intentions, et il a dit : Comment on t’appelle, chez les tapettes ? La velue ?

Cerqueira a poussé un grognement de sanglier auquel on vient de planter une fourchette dans le cul sans lui demander son avis, après quoi il a tenté de se jeter sur Titus qui le toisait avec un petit sourire narquois, l’air content de lui. Dans sa précipitation, Cerqueira avait zappé un menu détail : il avait les mains attachées dans le dos, derrière le dossier de sa chaise. Si vous avez des doutes, faites l’expérience chez vous : asseyez-vous sur une chaise, joignez vos mains derrière le dossier et essayez de vous lever, c’est sans danger mais instructif sur les lois de la physique et les limites du corps humain.

Il a sifflé entre ses dents : Tu me paieras ça, négro.

Je me suis dit que le moment était venu de détendre un peu l’atmosphère : T’as entendu parler du 27 avril 1848 ?

\textsc{Lui} : Je dirai rien tant que cette saleté de gorille sera dans la pièce !

L’instant d’après, la main droite de Titus se dirigeait à vive allure vers la joue gauche du prisonnier, tel un missile à tête chercheuse. Une fraction de seconde plus tard, elle s’écrasait sur sa cible avec une violence telle que l’ensemble du dispositif (je parle de la joue, son propriétaire et la chaise sur laquelle il était assis) effectuait un tour complet sur lui-même dans le sens des aiguilles d’une montre.

\textsc{Titus} : Je t’ai dit de la fermer, grosse fiotte.

Cerqueira, secouant la tête (on entendait tinter les glaçons) pour se remettre les idées en place après son voyage improvisé : Bande de fumiers, vous avez pas le droit de faire ça ! Je veux parler à mon avocat, tout de suite !

\textsc{Titus} : T’as pas d’avocat, pauvre con.

\textsc{Moi} : Eh bien, le 27 avril 1848, c’est le jour où le gouvernement provisoire de la République française, qui compte parmi ses membres des personnalités aussi éminentes que Louis Blanc, Ledru-Rollin, Arago et Lamartine, signe le second décret d’abolition de l’esclavage, celui de 1794 n’ayant pas donné les résultats escomptés.

\textsc{Cerqueira} : Et alors, qu’est-ce que tu veux que ça me foute !

\textsc{Moi} : Et alors, depuis cette date, traiter un Noir de nègre est un crime passible de prendre des grandes claques dans sa sale gueule d’enfoiré de suceur de queues !

Cerqueira, massant sa joue endolorie, sur un ton geignard : Pas la peine de continuer à me parler, je dirai plus rien.

Je lui ai collé une photo sous le nez : Tu vois, ça ?

Il a jeté un œil sur le document et détourné rapidement le regard, manifestement peu enthousiasmé par ce qu’il venait de voir.

\textsc{Moi} : T’es plutôt mignonne, en jupe, à condition d’avoir un penchant pour les femmes qui ont de la moustache et du poil aux pattes. Perso, je suis pas fan. J’en ai tout un tas du même genre, dont certaines en gros plan qui font leur petit effet. Je vais peut-être envoyer ça à la téloche, je sais qu’ils adorent les docus animaliers pour meubler les longues soirées d’hiver. Tu vas faire sensation ! Je verrais bien un titre un peu accrocheur du style «~Rue Armand Brunelle, le ballet des phoques, plongée dans les chiottes de l’extrême-droite~», un reportage-choc avec une musique dramatique qui scotche le spectateur devant son écran ou le réveille quand il pique du nez dans son paquet de chips. Non, sans blague, je peux viser le Pulitzer avec un truc comme ça. Par contre, je suis pas certain que tes petits copains nazis vont trouver ça à leur goût.

Cerqueira, une larmichette au coin de l’œil : C’est pas bien de se moquer des gens, commissaire.

\textsc{Moi} : Je me moque pas, je dis juste que t’es quand même une belle grosse salope comme on n’en fait plus !

\textsc{Lui} : Vous avez pas le droit.

Moi, lui soufflant la fumée de mon Rocky Patel Disciple dans les naseaux (on n’était pas censé fumer dans les locaux de la police nationale, sauf chez le préfet, mais je m’autorisais quelques libertés dans l’enceinte de ce que j’appelais mon «~cabinet privé~», à savoir la salle d’interrogatoire spéciale sujets récalcitrants, celle où on transformait le loup en agneau, l’aigle en taupe, le grand requin blanc en sardine à l’huile) : Excuse-moi, mon chaton, mais quand des individus se permettent d’en cramer d’autres au lance-flammes, j’ai tendance à oublier la Déclaration des droits de l’homme et du citoyen. D’ailleurs, la Déclaration parle des droits de l’homme, pas des droits de l’enculé. Je me doute qu’un gros nounours comme toi n’a rien à voir dans des saloperies de ce genre, raison pour laquelle j’aimerais t’éviter des ennuis. On t’a vu au Bouclier avec Jégou, Monteil et Desmarais. Ces bouffons sont des ordures de la pire espèce et je suis certain que tu sais des choses à leur sujet. Et toutes ces vilaines choses, je veux les savoir aussi, sinon je te colle en taule pour exhibitionnisme et racolage sur la voie publique, plus deux ou trois bricoles que je n’aurai aucun mal à dénicher dans ton CV. Le tarif officiel, c’est deux mois derrière les barreaux et trois mille cinq cent balles d’amende. Le fric, passe encore, mais les deux mois vont te paraître une éternité. Tes photos vont faire le tour du Net en trois secondes et demie, tu vas devenir une star du X et les détenus vont faire la queue pour te rendre visite.

\textsc{Lui} : Vous n’avez pas le droit de faire ça.

\textsc{Moi} : Je sais, tu l’as déjà dit. En même temps, je vais pas prendre de pincettes avec un vieux kleenex comme toi, Tacito. J’ai envie qu’on se parle d’homme à homme, comme des adultes responsables.

\textsc{Lui} : J’ai droit à un avocat.

\textsc{Titus} : Ici t’as droit à rien du tout, sac à merde ! Je peux te couper les couilles et te les faire bouffer, personne ne viendra à ton secours.

Cerqueira disposait de tout un catalogue de grossièretés dans lequel il pouvait puiser à volonté pour insulter son interlocuteur en fonction de ses origines raciales. J’ai lu dans ses yeux qu’il était en train de le feuilleter pour faire son choix. Mais au moment de passer à l’offensive, alors même que sa bouche d’égout venait de s’ouvrir pour déverser le raz de marée d’immondices qu’il destinait à Titus, de quoi ensevelir à tout jamais sa dignité et le pousser dans ses ultimes retranchements, il s’est ravisé et contenté de serrer les dents à s’en faire exploser les mâchoires, tout en le fusillant du regard et soufflant bruyamment tel un taureau sur le point de charger. Bien que d’une stupidité à toute épreuve, il venait quand même de prendre conscience que chacune de ses interventions ordurières ne faisait que l’enfoncer un peu plus dans la mouise.

Moi, d’une voix aussi suave qu’un coulis de framboise sur une boule de glace à la vanille : Allons allons, je suis certain que monsieur sait parfaitement où se trouve son intérêt et ne fera aucune difficulté pour nous dire tout ce qu’il sait. N’est-ce pas, mon cher Tacito ?

Pas de réponse.

Votre serviteur, qui n’est pas du genre à se laisser décontenancer par si peu : Dans l’Antiquité, Tacite était connu pour sa grande sagesse, comme en témoignent ses écrits et sa correspondance avec Pline le Jeune. Tu connais Tacite, Tacito ?

Toujours pas de réponse, mais bref regard en coin de l’intéressé dans ma direction, un regard chargé de méfiance dans lequel j’ai cru déceler également, contre toute attente (intervention divine, Allahu akbar ! comme diraient nos amis musulmans), une vague lueur d’intelligence, au sens d’intelligence avec l’ennemi, c'est-à-dire la volonté de nouer des relations circonstanciées avec l’adversaire dans le but de satisfaire un intérêt commun.

D’où la réplique suivante : Tacite est un historien de la Rome antique. Tu devrais lire les Annales, je suis sûr que ça te plairait.

\textsc{Cerqueira} : Très drôle !

\textsc{Moi} : Bon, pour en revenir à notre petite affaire, je te cache pas que j’ai un peu de mal à te croire quand tu me dis que t’es au courant de rien.

\textsc{Lui} : C’est pourtant vrai.

\textsc{Moi} : L’ennui, vois-tu, c’est qu’on t’a vu traîner au Sugar \& Spice.

\textsc{Lui} : Au quoi ?

\textsc{Moi} : Au Sugar \& Spice, rue Théo Cazenave.

\textsc{Lui} : Ah bon ?

\textsc{Moi} : Mais oui. J’ai des petites vidéos très sympas sur lesquelles on te reconnaît très bien.

\textsc{Lui} : Jamais mis les pieds.

\textsc{Moi} : Bien sûr que si. Mais on est dans un pays libre, tu as parfaitement le droit de préférer les garçon aux filles et de fréquenter les établissements de ton choix, y compris les plus extravagants. Non, vois-tu, ce qui m’embête le plus, dans cette histoire, c’est qu’on t’a vu à plusieurs reprises en grande conversation avec un certain Tiago Alvarez, lequel nous intéresse tout particulièrement. Tu vois de qui je parle ?

Pour la première fois, j’ai senti que la carapace de Cerqueira était en train de se fissurer.

Lui, à voix basse : Oui.

\textsc{Moi} : Pardon, j’ai pas bien entendu ?

Lui, nettement plus fort : Oui !

\textsc{Moi} : Excellent, on avance à grands pas. Et tu sais où il se trouve, en ce moment-même ?

\textsc{Lui} : Non.

\textsc{Moi} : Dans un tiroir, à la morgue. Et tu sais pourquoi il est à la morgue ?

\textsc{Lui} : Non.

\textsc{Moi} : Eh bien je vais te le dire. Il se trouve à la morgue parce que des gens très mal intentionnés l’ont fait griller comme une vulgaire saucisse. Et on a toutes les raisons de penser que ces gens très mal intentionnés font partie de la bande de nazillons avec laquelle tu traînes habituellement. Je me trompe ?

\textsc{Lui} : J’ai rien fait.

\textsc{Moi} : Raison de plus pour dire la vérité.

Lui, transpirant à grosses gouttes : Ils vont savoir que c’est moi.

\textsc{Moi} : Qui ça, «~ils~» ?

\textsc{Lui} : Ben eux !

\textsc{Moi} : Jégou et les autres ?

\textsc{Lui} : Oui, Aymeric, Monteil et Desmarais !

\textsc{Moi} : Ils ont buté Alvarez, pas vrai ?

\textsc{Lui} : Il ne m’arrivera rien ?

Moi, un sourire carnassier aux lèvres : Pas si tu dis la vérité. Et pas si tu dis à Titus que tu regrettes de l’avoir traité de nègre.

Lui, jetant un regard torve à Titus, qui se tenait debout à quelques centimètres de lui, le surplombant de toute la puissance musculeuse de son anatomie hors norme (et pour tout dire assez flippante quand on la voyait débouler par une nuit sans lune au détour d’une ruelle mal éclairée) : Quoi ?

\textsc{Moi} : Dis à Titus que tu regrettes de l’avoir traité de nègre.

Titus, d’une voix faussement douce : Tu sais, les Noirs sont des gens comme les autres, avec une tête, des bras, des jambes et un cœur qui bat. C’est juste que les gens naissent dans des endroits différents, avec des couleurs différentes, adaptées à leur environnement.

Cerqueira, la larme à l’œil, le nez reniflant : Oui, je sais bien.

\textsc{Titus} : T’es pas complètement con, quand même ?

\textsc{Lui} : Non.

\textsc{Moi} : Bien sûr qu’il y a encore de l’espoir de le sauver. Certains sont irrécupérables, comme Jégou, mais d’autres peuvent encore rejoindre les rangs de l’humanité. Et je suis persuadé que notre ami Tacito fait partie de ceux-là. Hein Tacito ?

Tacito, reniflant de plus en plus fort, tel un sanglier qui a flairé une truffe bien mûre et juteuse sous le tapis de feuilles mortes de l’automne : Oui. J’en ai marre de toutes ces conneries.

\textsc{Titus} : Alors dis que t’aimes les Noirs et que tu regrettes de m’avoir traité de nègre.

\textsc{Moi} : Tu veux un kleenex, Tacito ?

Lui, les yeux remplis de reconnaissance : Oui, je veux bien.

\textsc{Titus} : Dis que tu regrettes.

\textsc{Lui} : Je suis désolé, monsieur, je recommencerai plus.

\textsc{Titus} : Dis «~je regrette, monsieur Titus, de vous avoir traité de sale nègre~».

\textsc{Lui} : Je regrette, monsieur Titus, de vous avoir traité de sale nègre.

\textsc{Moi} : C’est vrai, ça, Tacito, je ne comprends pas comment on peut en vouloir à quelqu’un parce qu’il est physiquement différent. Tu aimerais, toi, qu’on te traite de sanglier parce que tu es couvert de poils de la tête aux pieds ?

\textsc{Lui} : Non, m’sieur. Je peux avoir mon mouchoir, maintenant ?

\textsc{Moi} : Alors écoute-moi bien, mon petit Tacito : est-ce que, si je te détache les mains pour te prouver ma confiance, tu seras respectueux de l’honneur que je te fais et ne tenteras pas de te servir de tes mains pour autre chose que moucher ce vilain nez qui pisse comme une fontaine ?

Lui, ravalant un filet de morve : Ça veut dire quoi, m’sieur ?

\textsc{Moi} : Ça veut dire : est-ce que tu promets de pas faire le con si je te détache ? Je peux te faire confiance ?

\textsc{Lui} : Oh oui, m’sieur le commissaire, bien sûr !

\textsc{Moi} : Je suis pas commissaire mais c’est pas grave, l’important c’est de savoir que je peux te faire confiance.

\textsc{Lui} : Mais alors vous êtes quoi, au juste ?

\textsc{Moi} : Titus, détache-le, s’il te plaît. Eh bien disons, pour répondre à ta question, que j’occupe une place un peu particulière au sein de la police nationale. Titus et moi opérons dans des conditions particulières, un peu hors-champ, si tu vois ce que je veux dire. Nous sommes, en quelque sorte, des hommes de l’ombre. Tu as vu le film de Lee Tamahori ? Titre original : Mulholland Falls, parce que ça se passe à Los Angeles, entre Hollywood et la vallée de San Fernando. T’es déjà allé à Los Angeles, Tacito ?

Tacito, soufflant comme un phoque (sans mauvais jeu de mots) dans la poignée de mouchoirs que je venais de lui refiler : Non. Et vous ?

\textsc{Moi} : J’essaie d’y aller au moins une fois par an. Non, je rigole ! Comment veux-tu qu’un pauvre flic comme moi ait les moyens d’aller à Los Angeles. Les Hommes de l’ombre, de Lee Tamahori, parle d’une unité spéciale de la police de Los Angeles dont la fonction est de nettoyer les rues de la ville en toute discrétion. Je fais un peu la même chose. À mon modeste niveau, bien sûr ! Ça y est, t’as fini ? Je ne pensais pas qu’un nez humain, même gros, pouvait contenir autant de morve !

Lui, s’essuyant le nez et les régions alentours, touchées elles aussi par le sinistre : J’ai fini, inspecteur.

\textsc{Moi} : Je suis pas vraiment inspecteur, mais c’est pas grave. Moins t’en sais, mieux ça vaut pour toi.

Titus, refilant une autre poignée de kleenex à Tacito : Tiens, essuie-toi, t’en as partout. T’es vraiment un gros porc !

\textsc{Tacito} : Dites pas ça, m’sieur le Noir. C’est juste que mon père était au chômage et que ma mère faisait des ménages pour gagner notre pain. J’ai pas été dans les grandes écoles avec les gens de la haute.

\textsc{Titus} : Tu peux m’appeler Inspecteur, si ça te défrise pas trop les bigoudis. Tacito, par contre, c’est pas facile à dire. Trop long. Je vais t’appeler Tata, si ça te dérange pas trop.

\textsc{Tacito} : C’était mon surnom à l’école communale.

\textsc{Titus} : T’aimes bien qu’on t’appelle Tata, alors. Et puis ça correspond bien à ce que t’es : une grosse tata.

Tacito, s’épongeant le pif : J’ai pas fait exprès, chef.

\textsc{Titus} : Pas grave, je t’aime bien quand même. On a tous nos petites différences, c’est pas une raison pour se faire la gueule, tu crois pas ?

\textsc{Tata} : Bien sûr, inspecteur-chef.

Titus, lui tendant la poubelle : Tiens, mets tes détritus là-dedans. Tu me fais de la peine avec ton gros nez tout mouillé et tes doigts pleins de morve. Nous, en Afrique, on est tellement pauvres qu’on n’a pas de kleenex pour s’essuyer. On fait ça avec des touffes d’herbe sèche ramassées dans la savane. Tu connais la Sierra Leone ?

\textsc{Tata} : Non, m’sieur. C’est où ? En Espagne ?

\textsc{Titus} : En Afrique, mon pote ! Qu’est-ce que tu veux que j’aille foutre en Espagne ! Aussi bizarre que ça puisse paraître, c’est de là que viennent mes ancêtres, mon petit Tata. C’est comme qui dirait le berceau de l’humanité. Même toi, qui détestait les Noirs il n’y a encore pas cinq minutes, eh ben il se trouve que t’as du sang noir dans les veines. Peut-être pas beaucoup, c’est vrai, mais assez quand même pour que t’en aies un peu. Qu’est-ce que tu dis de ça ?

\textsc{Tata}, écarquillant des yeux larges comme des roues de vélo : Je savais pas !

\textsc{Titus}, tout sourire : Ben tu le sais, maintenant.

Je vous l’avoue franchement, j’avais profité de l’intermède pour fumer quelques taffes de mon Rocky Patel Disciple (pas de la colère) en toute tranquillité, après quoi j’ai estimé que leur petite conversation, objectivement nulle à chier, n’avait aucune raison de s’éterniser.

C’est donc avec tact mais fermeté, toujours parfait dans le rôle de l’animateur charismatique, du meneur d’hommes déterminé mais sensible, d’une telle intelligence supérieure, certes, mais toujours au service de l’intérêt collectif, le mieux-être de la communauté (c’est fou ce que j’adore parler de moi, surtout en bien, sachant que je suis tenu par le secret professionnel sur bien des sujets que j’aimerais aborder de façon plus substantielle si la société n’était pas ce qu’elle est, avec ses grandes oreilles qui trainent partout et ses censeurs toujours prêts à bondir sur le dissident) : Tu vois, Tacito, je pourrais te rattacher les mains dans le dos et reprendre l’interrogatoire à l’endroit où on en était resté, comme si de rien n’était. Mais je ne vais pas le faire, parce que j’ai décidé de te faire confiance. Oui, je pense qu’on a fait un bon bout de chemin ensemble et que tu seras plus à l’aise pour parler si tu conserves ta liberté de mouvement. Naturellement, s’il te venait à l’idée de faire des bêtises, tu comprendras que je pourrai difficilement empêcher Titus, mon fidèle adjoint ici présent, de faire usage de son arme.

Plus rapide que le faucon qui s’abat sur le mulot insouciant, Titus, sourire de tueur aux lèvres, a sorti le Glock qu’il planquait dans son dos, lui a fait faire une série de moulinets façon Jamie Foxx dans Django Unchained, puis l’a remis à sa place.

\textsc{Re-moi} : Mais avoue que ce serait dommage, parce que vous êtes maintenant devenus des amis, des gens qui se comprennent et se respectent, et que je sais que ce n’est pas sans un pincement au cœur que Titus se verrait dans l’obligation de faire sauter ta cervelle de moineau.

\textsc{Tata} : Je serai sage comme une image, chef.

\textsc{Votre serviteur} : Bien. On en était où, déjà ?

\textsc{Titus} : Aux Disciples de la Colère, si ma mémoire est bonne.

\textsc{Moi} : Et elle est bonne, mon cher Titus, et je dirais même excellente ! Oui, c’est bien ça, les Disciples de la Colère de mes deux, autrement dit Jégou, Monteil et Desmarais ! Tu me corriges si je me trompe, Tata.

\textsc{Tacito} : Non, c’est bien ça.

\textsc{Moi} : Tu veux un cigare ? Non, parce que si tu veux un cigare, c’est avec plaisir que je t’en offre un. Je suis comme ça, moi, j’aime que mes amis soient bien traités.

\textsc{Tata} : J’sais pas, m’sieur, j’ai pas l’habitude de fumer.

\textsc{Moi} : Okay, comme tu veux ! T’en veux pas, t’en veux pas, je vais pas te forcer ! Je veux juste que tu te sentes à l’aise, détendu pour répondre à mes questions. Tu as soif, tu veux peut-être un verre d’eau ?

\textsc{Lui} : Oui, j’veux bien.

\textsc{Moi} : C’est con, on n’en a pas. Par contre, j’ai ça.

J’ai sorti une flasque de la poche de ma veste : C’est peut-être pas l’idéal pour étancher la soif, mais ça va te donner un bon coup de fouet si tu te sens un peu mou du genou. T’en veux ?

\textsc{Lui} : C’est quoi, chef ?

Titus, d’humeur badine : Du schnaps.

\textsc{Moi} : Jamais de la vie ! Non, du whisky, bien sûr, du vrai, pas du whisky de tap… de ped… enfin… du vrai whisky, quoi, en provenance directe des Highlands. Aberfeldy 12 ans d’âge, pas piqué des hannetons !

Je m’en suis jeté une rasade et j’ai demandé à Titus, qui avait l’air de bien se marrer : T’en veux ?

\textsc{Lui} : Non, jamais pendant le sévice.

Puis j’ai tendu la flasque à notre ami Tata, recroquevillé sur sa chaise comme une crotte de nez dans le fond d’un mouchoir : Une petite goutte ?

\textsc{Lui} : C’est pas de refus.

Je pensais qu’il allait se contenter d’un petit coup comme ça, vite fait, en passant, une petite gorgée pour la route, mais eu lieu de ça, cette espèce de pithécanthrope en phase terminale de putréfaction mentale n’a rien trouvé de mieux à faire que de lessiver la quasi-entièreté de ma flasque en une seule aspiration, comme s’il avait une pompe à whisky à la place du tube digestif.

Votre humble serviteur, après avoir récupéré et remis ses petites affaires en place : Bon, Tacitouille, maintenant que tu t’es bien rincé la dalle, j’espère que t’es prêt à répondre à mes questions.

\textsc{Tacito}, après un claquement de langue significatif : Fin prêt, commissaire.

\textsc{Moi} : À dire la vérité, toute la vérité, rien que la vérité ?

\textsc{Tacito}, main droite levée comme pour un petit salut nazi mi-coude à la Hitler : \textit{\foreignlanguage{portuguese}{Nada além da verdade, eu juro!}}

\textsc{Titus} : C’est quoi, ce charabia ?

\textsc{Tacito} : Du portugais, chef.

\textsc{Moi} : Mouais. Bien, donc, pour en revenir à tes petits camarades Jégou, Monteil et Desmarais, on est bien d’accord que ces enculés ont buté Botrel et Alvarez.

\textsc{Tata} : Botrel, je sais pas, mais Alvarez oui.

\textsc{Moi} : Botrel tu sais pas ?

\textsc{Tata} : Non, je sais même pas qui c’est.

\textsc{Moi} : Abraham Botrel, antiquaire, grosse fortune, folle notoire qui se prenait pour la réincarnation de Charles de Beaumont, alias le chevalier d’Éon, retrouvé noyé dans la fosse à purin d’un aristo pédophile et cannibale, ça te dit rien ?

\textsc{Lui} : Rien du tout.

\textsc{Moi} : Faut que je te le dise en portugais, ou quoi ?

\textsc{Lui} : J’ai juré de dire toute la vérité, rien que la vérité, chef.

Titus, faisant craquer les jointures de ses doigts : Je sais pas toi, mais moi j’ai comme l’impression qui se fout de notre gueule.

\textsc{Tacito} : Je jure sur la tête de ma mère que je connais pas de Botrel !

\textsc{Titus} : Parce que t’as une mère, toi ?

\textsc{Tacito} : Ben oui, comme tout le monde. Sauf que ma mère à moi, c’est mon père.

\textsc{Titus} : Quand je te disais qu’il se fout de notre gueule !

\textsc{Moi} : Oui, bon, on s’en fout de sa mère ! Okay, Tacito, tu ne connais pas de Botrel, mais Tiago Alvarez, tu le connais lui ?

\textsc{Lui}, sa bouche se mettant à trembloter comme s’il allait éclater en sanglots : Oui. Pour tout vous dire, j’étais raide-dingue de lui ! Je l’ai rencontré au Sugar \& Spice pendant une soirée déguisée, on a couché ensemble une ou deux fois, et puis après je me suis mis à flipper et j’ai voulu tout arrêter. J’aime les garçons, je l’avoue, mais j’ai pas intérêt à ce que ça se sache si je veux pas finir dans une benne à ordures avec le signe d’infamie tatoué sur le front.

\textsc{Moi} : Le signe d’infamie ?

\textsc{Lui} : Ouais, une croix dans un cercle, la marque de Caïn.

\textsc{Moi} : Parce qu’une croix dans un cercle, c’est la marque de Caïn ?

\textsc{Lui} : J’en sais rien, moi !

\textsc{Moi} : T’en sais rien, t’en sais rien ! Faudrait peut-être voir à savoir un peu quelque chose, de temps en temps !

\textsc{Lui} : Ce que je sais, c’est que j’ai annoncé à Tiago que tout était fini entre nous et qu’il l’a très mal pris. Il a commencé à me harceler, venir faire du grabuge en bas de chez moi, rappliquer sans prévenir sur mon lieu de travail avec des fringues provocantes sur le dos. Aymeric lui a dit de se barrer, a menacé de lui casser la gueule, mais Tiago n’était pas du genre à se laisser impressionner. À l’entendre, il était champion de capoeira et n’avait pas peur de se battre. Un jour, Aymeric a décrété que, champion de capoeira ou pas, le moment était venu de lui donner une bonne leçon, et que j’avais intérêt à filer doux si je voulais pas qu’il m’arrive des bricoles. Je devais lui tendre un piège, sous un prétexte quelconque, et les autres, les Disciples de la Colère, devaient lui tomber dessus et lui flanquer la frousse de sa vie. Je voulais pas qu’ils lui fassent de mal, pas trop en tout cas, juste lui foutre la trouille pour qu’il arrête ses conneries. Donc je l’ai appelé, disant que j’étais toujours amoureux de lui, que j’avais changé d’avis et voulais continuer à sortir avec lui. Ce qui n’était pas totalement faux, d’ailleurs, mais je pouvais difficilement le dire aux autres. Je lui ai donné rendez-vous chez moi vers minuit, et on lui est tombé dessus au moment où il s’apprêtait à sonner à l’interphone. Aymeric lui a braqué un flingue sous le nez, un Luger P08 hérité de son arrière-grand-père qui avait bossé pour la Gestapo, au 11 rue des Saussaies, et on l’a obligé à monter dans le Sprinter (utilitaire léger de chez Mercedes, ndlr) de Noé.

\textsc{Moi} : Donc t’étais présent au moment des faits.

\textsc{Lui} : Ouais, pas chez moi mais dans le Sprinter avec Aymeric, Milo et Noé au volant. On guettait son arrivée, et dès qu’il a pointé le bout de nez, Aymeric et moi on est sorti et on l’a fait monter de force dans le Sprinter. Noé a démarré en trombe et on a foutu le camp. Pendant que Milo le tenait en joue, je lui ai attaché les mains dans le dos avec du ruban adhésif.

\textsc{Moi} : Quel genre de ruban adhésif ?

\textsc{Lui} : Le genre dont on se sert pour masquer les murs quand on fait des travaux de peinture. Noé en a toujours dans son fourgon, en plus de ses accessoires habituels.

\textsc{Moi}, pianotant sur le bureau, signe que je commençais à trouver le temps long (je suppose d’ailleurs que vous aussi) : À savoir ? Dis donc, mon vieux, va falloir être un peu plus loquace. Je vais pas te sortir les vers du nez un par un !

\textsc{Lui}, aussi à l’aise que s’il barbotait les fesses à l’air dans un marigot infesté de sangsues, alligators et piranhas, le combo parfait pour des vacances réussies : Marteau, perceuse, barre de fer, manche de pioche, des outils de ce genre. Rien d’étonnant à ça, il travaille dans le bâtiment.

\textsc{Moi} : Je t’en foutrai, moi, du bâtiment !

\textsc{Lui}, essayant vainement de remonter sur la berge : Bon, c’est vrai qu’il s’en sert aussi pour autre chose que monter des murs ou ravaler des façades.

\textsc{Moi}, jetant un œil sur ma Rousselot P06 Ultramatic 720 (c’est une montre, si vous voulez tout savoir, que la somptueuse Zarina Brizzi, mystérieuse orientale au regard de braise~-- la Toscane est bien en Orient, non~-- somptueuse, disais-je, Zarina Brizzi, m’avait offerte pour mon anniversaire, lequel je ne vous le dirai pas, ça ne vous regarde pas, sachez seulement que c’était la première fois que quelqu’un, une femme en particulier, somptueuse qui plus est ce qui ne gâte rien (c’est quand même pas pareil si une femme moche, un infâme boudin dont on ne voudrait même pas pour récurer ses chiottes, vous refourgue une tocante toute pourrie achetée une bouchée de pain à un vendeur à la sauvette), m’offrait une montre, raison pour laquelle la chose m’était allée d’autant plus droit au cœur, telle une flèche décochée par quelque angelot grassouillet, me laissant pour ainsi dire sans voix, dans un état d’ébriété émotionnelle que je n’aurais jamais cru pouvoir atteindre un jour, et je ne parle pas seulement de la qualité de l’objet, indéniable, mais aussi des circonstances et la façon dont il m’avait été offert) : À Cordoue, on ne mange jamais de queue de taureau de combat. Il n’est pas rare que les gens me posent la question, aussi tenais-je à vous le dire avant de passer à autre chose.

Les autres m’ont regardé bizarrement, comme si j’étais de couleur verte, ou bleue, avec des yeux de lémurien, une trompe à la place du nez, des antennes télescopiques sur le front, des tentacules à la place des bras et des turboréacteurs à la place des jambes.

\textsc{Titus} : Je sais pas ce qu’ils mettent dans le whisky, mais je crois que tu devrais te reposer un peu.

\textsc{Tacito} : Il est toujours comme ça ?

\textsc{Titus} : T’emballe pas, Toto, il est juste un peu surmené en ce moment.

\textsc{Moi} : Ça va très bien, persil. On en était où ?

\textsc{Titus} : Nos quatre joyeux drilles venaient d’embarquer Tiago Alvarez sous la menace d’une arme.

\textsc{Moi} : Ah oui, c’est ça. Et ensuite ?

\textsc{Toto} : Ensuite on a roulé, on est sortis de la ville, et on s’est retrouvés quelque part dans la forêt, vers une heure du matin. Je pensais qu’ils allaient juste le tabasser un peu, mais quand j’ai vu Noé sortir du Sprinter avec un lance-flammes à la main, j’ai vu le bad trip arriver à plein nez !

\textsc{Moi} : Ça veut rien dire, ça, «~j’ai vu le bad trip arriver à plein nez~» !

\textsc{Lui} : Ah bon ?

\textsc{Moi} : Non, rien du tout.

\textsc{Lui} : Ben je me suis dit comme ça, à l’intérieur de moi-même, que les choses risquaient de tourner au vinaigre.

\textsc{Moi} : Et alors, qu’est-ce que t’as fait ?

\textsc{Lui} : J’ai refusé d’aller plus loin, les suivre dans les profondeurs de la nuit et sacrifier un innocent pour des conneries de runes nordiques et autres légendes celtiques tatouées sur de la peau humaine dans des vieux grimoires poussiéreux. J’ai dit que s’ils faisaient du mal à Tiago, j’arrêtais tout, je plaquais définitivement le Mouvement. Mais ils en avaient rien à foutre, ces cons ! On s’est pris le chou, une bagarre a éclaté, je me suis pris un coup de matraque télescopique derrière les oreilles et je me suis retrouvé le nez dans les feuilles mortes à ronfler comme un bienheureux. Et je peux dire que j’ai eu de la chance, en effet, parce que grâce à ça j’ai échappé au carnage qui s’en est suivi. Noé est un grand malade qui prend son pied à foutre le feu et regarder les poubelles et les bagnoles brûler, mais jamais j’aurais pensé qu’il était capable de foutre le feu à quelqu’un ! Franchement, c’est l’horreur ! Quand je me suis réveillé, une heure plus tard, les autres n’étaient toujours pas là. J’ai attendu, rongé par l’anxiété comme vous pouvez l’imaginer, en lançant des appels pour essayer de savoir où ils étaient. Je les ai vus revenir une heure plus tard, mais ils n’étaient plus que trois. Quand je leur ai demandé où était Tiago, ils m’ont répondu que c’était pas la peine de l’attendre, qu’il rentrerait plus tard par ses propres moyens. J’ai bien vu qu’Aymeric était tout retourné, mais il n’a rien voulu me dire de plus. Pas pour l’instant, en tout cas. Il était incapable de parler, totalement tétanisé par ce qu’il venait de subir. Aymeric est un nazi de la première heure, c’est vrai, un nostalgique du grand Reich aryen qui devait dominer le monde et régner en maître sur la Terre, mais c’est pas un assassin. Je dis pas que c’est un saint, loin de là, mais il a des valeurs et pense que tout être humain a droit à une chance de rédemption.

\textsc{Moi} : Arrête, tu vas me faire chialer. Donc, si je comprends bien, tu dormais tranquillement le nez dans la mousse et tu ne sais pas ce qui s’est passé dans la forêt ?

\textsc{Lui} : Non, rien sur le moment. Je vous le jure, commissaire ! Je l’ai su par la suite, quand Aymeric, qui en avait gros sur la patate, m’a tout raconté. Franchement, commissaire, je suis dégoûté !

\textsc{Titus} : Et dégoûtant, surtout.

L’entretien s’était poursuivi de la sorte pendant un certain temps, après quoi, estimant qu’on avait pressé le citron jusqu’à la dernière goutte de jus aigre et indigeste (et bu le calice de la misère humaine jusqu’à la lie, une vraie purge), on avait décidé de le foutre en cabane sous un certain nombre de chefs d’accusation, tous plus d’hiver et avariés les uns que les autres, parmi lesquels association de malfaiteurs, non-assistance à personne en danger et incitation à la haine raciale. En fait, il s’agissait surtout de s’assurer que ce sinistre crétin n’entrerait pas en contact avec ses ex-comparses pour les avertir que le bras séculier de la justice était sur le point de s’abattre sur eux. Ce bras séculier, vêtu d’un long manteau noir et armé d’une faux au tranchant ébréché à force de s’échiner sur des nuques rétives, se ferait une joie de leur séparer la tête du tronc et les envoyer croupir à tout jamais dans les culs-de-basse-fosse de la connerie humaine, précipice insondable dont on voyait chaque surgir des créatures plus monstrueuses les unes que les autres. Car l’espèce humaine, nous le savons, en dépit de certaines qualités qu’on ne saurait lui dénier (phrase toute faite, j’en conviens, parce que j’ai beau chercher, avec toute la force de mes petites cellules grises embuées par les vapeurs de Barolo, je ne suis pas certain de voir de quelles qualités on parle), reste quand même, comment dirais-je… une entité problématique dont on ne sait pas très bien quoi faire à court ou moyen terme, sachant qu’elle n’est pas là depuis très longtemps mais a déjà accompli un certain nombre de performances dont on ne sait trop quoi penser, sinon qu’elles convergent toutes dans un seul et unique sens, un peu embarrassant il faut bien le dire, à savoir assurer à ladite engeance une espèce de suprématie naturelle censée légitimer, envers et contre tout, la consubstance quasi divine de sa nature intrinsèque, son approche mystique et dévoyée de l’existence, sa vision totalement égocentrée du monde, de l’univers et tout ce qui s’ensuit, et, au-delà même de l’univers et tout ce qui se trouverait en excéder, par quelque malversation cosmique d’obscure nature, les sublimes contours, l’épicentre tant géostratégique que métaphysique de sa petite personne.

Vous avez compris quelque chose ? Tant mieux pour vous, parce que moi je vous avouerai humblement que je nage en plein brouillard. Il y avait finalement assez peu de chances pour que, dans l’infinie solitude de l’univers, éclose sur une planète perdue au fin fond de l’espace une forme de matière inhabituelle, interactive et redondante, susceptible de s’envisager elle-même, et capable d’incarner à elle seule toutes les tensions et forces contraires qui s’agitent au sein du creuset originel.

Par exemple, ce soir-là, alors même que les combats fratricides faisaient rage aux quatre coins de la planète, laissant dans leur sillage des femmes en pleurs serrant dans leurs bras la dépouille ensanglantée de leurs enfants, j’avais convié un modeste couple de mes plus proches amis, en l’occurrence Zaahid Shirani et Tosca Brizzi, à une petite collation sans prétention à mon humble domicile, au 157 rue des Anus en fleurs (adresse fantaisiste, vous l’aurez compris et je vous remercie d’avance pour votre sollicitude, mais je ne puis, pour des raisons de sécurité évidentes, dévoiler ici l’adresse exacte dudit domicile), domicile, vous ne l’ignorez plus maintenant, que j’avais l’honneur et le privilège de partager avec Zarina Brizzi, la propre sœur jumelle de cette chère Tosca. Je précise à ce sujet que le fait d’avoir sensiblement la même femme avait considérablement renforcé les liens qui nous unissaient déjà, Zaahid et moi, au point, sans aller tout à fait jusque-là, que nous nous considérions maintenant presque comme des frères, unis par une forme de gémellité profane, certes, et dizygote à souhait, mais néanmoins bien réelle.

Zarina, pour l’occasion, avait préparé avec amour un \textit{\foreignlanguage{italian}{bollito misto}} puissamment goûtu, sorte de pot-au-feu à base de bœuf, poule et \textit{\foreignlanguage{italian}{cotechino di puledro della provincia di Belluno}} (viande de poulain et couenne de porc, plus mélange d’épices en proportions variables), le tout servi avec une sauce verte (cresson, cerfeuil, corne de cerf et pimprenelle) du plus bel effet, mais suffisamment douce pour ne pas interférer avec la finesse extrême du breuvage élégiaque que j’avais choisi pour l’accompagner.

Ce breuvage, dont tous les mots, y compris ceux du plus brillant poète ou œnologue le plus averti, seraient bien impuissants à exprimer la nature profonde, n’était autre qu’un Volnay Santenots Cuvée Gauvain 1984 des Hospices de Beaune qui s’est révélé boxer très au-dessus de sa catégorie. Pour info, il provenait de la petite ponction que j’avais eu le bon goût d’effectuer sur la cave de feu Mathéo Riqueti, évêque du Sanctuaire de Ddarr, ami personnel de Prospero Cangelosi (le chef de chœur du Vatican) et ordure patentée avec laquelle, on s’en souvient, j’avais entretenu des relations pour le moins électriques.

S’il peut sembler assez inhabituel de servir ce genre de chose avec un bollito misto, un vin de moindre pédigrée (qu’on peut aussi écrire pedigree, puisqu’il vient de la déformation de «~pied de grue~» par les Anglais, «~pied de grue~» désignant à l’origine un arbre généalogique), plus jeune et fruité, lui étant généralement préféré, j’insiste pour dire et répéter que personne, autour de la table, n’a eu la mauvaise grâce de s’en plaindre. Je pense, moi, que servir un grand vin, même avec de la cervelle de singe flambée au calva ou des couilles de taureau farcies à la confiture de fraise, n’est jamais une perte de temps, même s’il convient bien évidemment de se rincer abondamment la bouche à l’eau tiède entre chaque gorgée.

Pour la conversation qui va suivre, je tiens à préciser que Tosca, au même titre que sa sœur, s’exprimait avec un accent italien particulièrement charmant et délicieux, une merveille de suavité que je ne tenterai en aucune façon de retranscrire par le truchement de quelque prose lourde et disgracieuse, en flagrante contradiction avec la nature même du modèle. Ne comptez pas sur moi pour commettre une telle hérésie, bande de lapereaux maléfiques !

\textsc{Zaahid} : Je reprendrais bien un peu de bollito misto, moi !

Il n’en fallait pas davantage pour que Zarina le resserve copieusement, avec toute l’élégance qu’elle savait mettre dans le moindre de ses gestes, comme si une force supérieure de beauté et d’intelligence l’animait en permanence.

\textsc{Moi} : Tu es sûr que tu seras des nôtres, ce soir ?

On avait prévu, Titus, Greg et moi, de mettre un terme définitif aux agissements des Disciples de la Colère, cette bande de fanatiques d’extrême-droite qui semaient la mort et la destruction sur leur passage. On savait qu’ils devaient se réunir à minuit dans un endroit secret, secret que Cerqueira avait eu la bonne idée d’éventer pour nous. Il n’avait pas trop le choix, c’est vrai, mais sachant qu’il devait lui-même assister à cette réunion, il prenait un risque non négligeable en crachant le morceau.

Zaahid, la bouche pleine : Et comment !

\textsc{Tosca} : Non, c’est bien trop dangereux.

\textsc{Zaahid} : Mais j’ai besoin d’action, moi ! J’en ai marre de passer mon temps à tripatouiller des cadavres.

Son verre était vide. Et pour cause, il ne cessait de le vider.

Moi, toujours au petit soin avec les amis : Un peu de Volnay ?

\textsc{Lui} : Oui, volontiers.

Tosca, qui était assise en face de moi (le plan de table, susceptible de varier au gré des circonstances, était ainsi composé ce soir-là, de Tosca en face de moi, Zarina à ma gauche, et Zaahid en face de Zarina), m’a refilé un coup de pied sous la table, non parce qu’elle me détestait et prenait un plaisir sadique à me faire souffrir, mais parce qu’elle entendait avec ce geste aussi discret que percutant me rallier subrepticement à sa cause, qui était de tout tenter pour décourager Zaahid de prendre part à notre petite virée digestive qui risquait de tourner à Règlements de comptes à OK Corral, avec votre serviteur dans le rôle de Wyatt Earp et l’honorable Greg Lussier dans celui de Doc Holliday.

Moi, le resservant : Mon cher Zaahid, tu sais que je t’aime beaucoup et que je ferais tout pour t’être agréable, mais je ne suis vraiment pas certain que ce soit une bonne idée de t’emmener avec nous ce soir.

Zarina, abondant dans notre sens : Il a raison, tu es complètement ivre.

Techniquement parlant, on s’était enfilé une bouteille de Corton Charlemagne à l’apéro (cuvée François de Salins 2014 des Hospices de Beaune, même provenance que le Volnay, une bouteille tout à fait remarquable dont le niveau avait baissé tellement vite qu’on s’était demandé si on n’avait pas ouvert une bouteille vide), plus une bouteille et demie de Volnay puisqu’on était en train de lessiver la deuxième, ce qui, pour quatre adultes en pleine possession de leurs moyens, ne représentait pas une charge d’alcool invraisemblable dans le sang, même si ça commençait tout de même à devenir légèrement problématique pour des choses à priori aussi anodines que prendre le volant ou se tenir en équilibre sur une jambe.

\textsc{Zaahid} : Non, je ne suis pas ivre. D’ailleurs, j’ai amené ça.

On s’était demandé pourquoi il était venu avec un sac de sport et ce qu’il pouvait bien y avoir à l’intérieur. Vous connaissez beaucoup de gens que vous invitez à dîner et qui débarquent avec un sac de sport ? Moi pas, et c’était d’autant plus interpellant que Zaahid, intellectuel exotique à la sensibilité réelle même si pas toujours explicite, avait toujours témoigné de la plus océanique répulsion pour tout ce qui touchait de près ou de loin à la cause sportive, son intérêt pour la discipline se limitant accessoirement à l’observation attentive des musculatures en action pendant l’effort, féminines notamment. Par exemple, et je peux en témoigner parce qu’il m’a obligé à me farcir des meetings en sa compagnie à de nombreuses reprises, faisant fi de mes dénégations les plus énergiques, l’athlétisme avait ses faveurs, surtout quand il s’agissait de voir courir ou encore sauter en hauteur ou à la perche des jeunes filles (souvent de couleur) à la plastique impressionnante moulées dans des tenues quasi inexistantes.

C’est ainsi, sous les yeux ébahis d’une assistance au bord de la congestion cérébrale, qu’il a ouvert son sac et, après avoir fouillé dedans pendant quelques instants, en a sorti un jouet en plastique qui ressemblait vaguement à une arme de poing.

\textsc{Moi} : C’est quoi, ça ?

\textsc{Lui} : Un pistolet 9 mm, que j’ai fabriqué moi-même avec mon imprimante 3D.

\textsc{Moi} : Tu m’avais caché ça. Et tu comptes t’en servir pour quoi ? Aller à la chasse aux papillons ?

\textsc{Lui} : Tu peux toujours rigoler. Il se trouve que cet engin fonctionne parfaitement, et offre une puissance de feu identique à celle d’un pistolet conventionnel tout en étant beaucoup plus léger et pratiquement indétectable.

\textsc{Moi} : T’as pas peur qu’il t’explose entre les pattes ?

\textsc{Zarina}, d’une voix fraîche et chantante comme les eaux claires du Tronto à Ascoli Piceno : Quelqu’un veut encore du bollito misto ?

\textsc{Moi}, d’une voix aussi douce et satinée qu’une comptine pour enfant en bas âge : Non merci, ma chérie.

\textsc{Tosca} : Même chose pour moi, je suis prête à exploser.

\textsc{Zarina} : Zaahid ?

\textsc{Zaahid} : Ce serait avec plaisir, mais je ne tiens pas à sortir de table sur une civière.

\textsc{Moi} : Rassure-moi, tu n’as quand même pas sérieusement l’intention de venir chasser le nazi avec cet équipement dérisoire ?

\textsc{Lui} : Dérisoire mon cul ! On fait des trucs très bien, de nos jours, avec une imprimante 3D. Est-ce que tu sais, par exemple, que les Sentinelles de la Révolution du Pakistan sont équipées de HGGP-9 ?

\textsc{Moi} : De quoi ?

\textsc{Lui} : Des HGGP-9, pour Homemade Ghost Gun Pistol 9 mm, un pistolet 3D dont le modèle a été publié en open source sur le Dark Net par Colton Murray, un crypto-anarchiste d’extrême-droite, fondateur de la Division du Chaos, récemment condamné pour trafic d’armes, détention d’images pédopornographiques et agression sexuelle sur mineure.

\textsc{Moi} : Et tu as imprimé cette merde avec une imprimante 3D ?

\textsc{Lui}, arborant un visage rayonnant de joie et de fierté : Oui, et elle marche du feu de Dieu !

\textsc{Zarina}, dont je dois reconnaître qu’à ma demande elle était quasiment nue, au même titre que sa sœur, et ce, conformément aux usages en vigueur au sein de notre petite communauté : Dans ce cas, je débarrasse et on passe au fromage.

\textsc{Tosca} : Je viens avec toi.

Et pourquoi, me direz-vous, parce que j’entends déjà des voix s’élever dans la pénombre pour crier au loup et exiger que toute la lumière soit fait sur de tels agissements, pourquoi obliger ces malheureuses créatures à évoluer en tenue d’Ève alors que leurs conjoints ne sont en aucun cas soumis à une telle injonction ?

Eh bien, mes chers amis, permettez-moi de vous détromper amplement à ce sujet, car il se trouve que nous-mêmes ici présents, à savoir le docteur Zaahid Shirani et votre humble serviteur, étions tous deux vêtus avec une légèreté comparable, c'est-à-dire, en tout et pour tout, un modeste slip de bain qui nous moulait le paquet avec insistance. Ben quoi ? Vous trouvez ça ridicule, pour ne pas dire plus ? Eh bien sachez, quand la température s’y prête, que manger dans de telles conditions, déchargé du poids inique de la contrainte vestimentaire, et ce sans pour autant se revendiquer de la mouvance naturiste stricto sensu, est une des choses les plus agréables qui soient. Et puis, il y a quand même une sacrée différence entre manger du bollito misto et boire du Volnay 84 des Hospices en petite tenue, activité d’une noblesse évidente, et se balader les fesses à l’air sur une plage du Cap d’Agde, entouré de viande périmée qui marine dans l’huile solaire, activité d’une désolante vulgarité. Car même si ça vous semble difficile à avaler, sachez que nous étions tous d’une infinie pudeur. Aucun d’entre nous, même en échange d’une forte somme d’argent (jusqu’à un certain point tout de même, c’est bien d’avoir des convictions, des règles de conduite, mais il faut quand même savoir lâcher un peu de lest de temps à autre), n’aurait accepté d’aller se balader les fesses à l’air sur une plage du Cap d’Agde, alors que dîner tous les quatre dans la tenue la plus naturelle qui soit ne nous semblait en aucun cas contraire à nos principes. Il s’agissait, comme cela se pratique dans certes entreprises, de renforcer la cohésion du groupe en le plaçant dans des situations extrêmes, même s’il devenait évident que le fond de la deuxième bouteille de Volnay ne serait en aucun cas suffisant pour nous permettre de mener l’expérience à son terme.

À toute cause il faut un carburant digne de ce nom, et mon choix pour remplacer le défunt Volnay s’était porté sur un Beaune 1er Cru Cuvée Rousseau Deslandes 69 des Hospices, lui aussi prélevé par mes soins dans la cave de Riqueti.

\textsc{Moi} : Mesdemoiselles, s’il vous plaît, ne bougez plus !

Tosca et Zarina nous tournaient le dos, prêtes à repartir à la cuisine.

Elles, stoppées net dans leur élan : Qu’est-ce qui se passe ?

\textsc{Moi} : Ne bougez surtout pas, je vous en conjure !

\textsc{Zaahid} : Quoi, qu’est-ce qui se passe ?

\textsc{Moi} : Tu ne remarques rien ?

\textsc{Lui} : Non, quoi ?

\textsc{Moi} : Regarde mieux.

\textsc{Lui} : Désolé, je ne vois rien.

\textsc{Moi} : C’est extraordinaire !

\textsc{Lui} : Quoi, qu’est-ce qui est extraordinaire ?

\textsc{Moi} : Je suppose que tu as remarqué que Tosca a un grain de beauté sur la fesse gauche ?

\textsc{Lui} : Évidemment, pour qui me prends-tu !

\textsc{Moi} : Et maintenant si tu observes attentivement la fesse droite de Zarina, chose que je t’autorise exceptionnellement à faire sans la moindre retenue, qu’est-ce que tu vois ?

\textsc{Lui} : Nom de Dieu !

\textsc{Elles} : Ah, c’est ça !

Que dire, sinon que, sur la fesse droite de Zarina, par un de ces effets de symétrie reproductive qui caractérise l’essentiel des créations de la nature (et, par extension, celles de l’homme dont l’inspiration est la même, n’en doutons pas, même s’il a souvent l’impression de s’affranchir des règles qui le gouvernent à son insu), se trouvait un grain en tout point similaire à celui qui ornait la fesse gauche de sa jumelle !

\textsc{Zarina} : Oui, Tosca a un grain de beauté sur la fesse gauche et moi sur la droite. On peut y aller, maintenant ?

\textsc{Moi} : Je suis sincèrement désolé, mais vous comprendrez que je ne pouvais décemment pas laisser Zaahid dans l’ignorance d’un tel miracle de la nature. Il était de mon devoir de lui ouvrir les yeux, je l’ai fait et je ne regrette rien.

\textsc{Zarina}, soulevant légèrement l’organe en question : J’ai aussi un grain de beauté sous le sein gauche.

\textsc{Tosca}, de la même façon : Et moi sous le sein droit.

\textsc{Zaahid} : Et moi j’en ai un sur la couille gauche, mais je pense que tout le monde s’en fout !

\textsc{Moi} : Non ?

\textsc{Lui} : Si, pourquoi ? Attends, ne me dis pas que…

\textsc{Moi}, titubant d’émotion : Si, j’en ai un sur la couille droite !

\textsc{Zaahid}, jaillissant de sa chaise pour bondir dans mes bras : Embrasse-moi, vieux frère !

\textsc{Moi} : Oui, oui. Bon, va te rassoir, maintenant, il faut que j’aille chercher une bouteille de vin.

\textsc{Lui} : Je viens avec toi !

\textsc{Moi} : Sûrement pas, tu tiens à peine debout. Je te ramène jusqu’à ta chaise.

\textsc{Lui} : Je tiens parfaitement debout, et j’aimerais assez que tu arrêtes de me traiter comme un adolescent boutonneux !

\textsc{Moi}, le forçant à s’assoir : Pose tes fesses là-dessus et tâche de rester tranquille !

\textsc{Lui} : Je suis un adulte parfaitement responsable, une personnalité reconnue du monde de la science !

\textsc{Moi} : Oui, enfin, reconnu surtout par moi, ce qui n’est pas un mince honneur, je te l’accord. Bon allez, presque tranquille pendant que je vais chercher du vin.

\textsc{Lui} : C’est inadmissible !

\textsc{Moi} : Quoi, encore ?

\textsc{Lui} : La façon dont on me traite dans cette maison. Je ne suis quand même pas n’importe qui, merde !

\textsc{Moi} : Mais bien sûr que non. Sans toi, on n’aurait jamais mis la main sur ces enfoirés de Disciples de la Colère.

\textsc{Lui} : Exact ! Voilà pourquoi j’exige non seulement de la considération mais aussi de participer à leur extermination !

\textsc{Moi} : Je te rappelle qu’il y a des lois dans ce pays, et qu’il n’est en principe pas prévu de procéder à leur extermination. Ou alors très peu. Non, le but du jeu est de leur passer les bracelets pour les empêcher de nuire et les traduire devant une cour de justice qui décidera de leur sort.

Tosca et Zarina, de retour de cuisine avec les bras chargés de claquos (fromage à pâte molle et croûte fleurie, typiquement associé au camembert, dans le langage populaire, dixit lalanguefrançaise.com qui n’est pas la moitié d’un repaire de branleurs analphabètes) : Vous êtes vraiment des fachos !

\textsc{Zaahid} : Un de ces bons vieux jurys populaires qui font la grandeur de la démocratie !

\textsc{Moi} : Oui, cette démocratie que le monde entier nous envie. Bon, je vais chercher du vin.

Le temps de :

1. enfiler mon peignoir Aescwig Paige collection printemps-hiver 2017 en laine de soie peignée et microfibre de bambou 100\% bio du Suriname (cadeau de Zaahid qui possédait exactement le même),

2. croiser la mère Ouvrard qui, comme par hasard, était en train d’errer tel un ectoplasme maléfique dans les couloirs à la recherche de cette pourriture rousse de Korax (un de ces trois enfoirés de chats qui se faisaient un malin plaisir de chier et pisser partout, en particulier sur mon paillasson en fibre de coco WELCOME TO THE JUNGLE, triple référence au chef-d’œuvre de Peter Berg avec Dwayne Johnson, alias The Rock, à son remake tout aussi inoubliable de Rob Meltzer avec le grand poète et penseur belge Jean-Claude Van Damme, et bien entendu au film de Jonathan Hensleigh (historien, avocat, scénariste et réalisateur méconnu) qui narre les aventures cannibalocaustiques de quatre sympathiques jeunes gens partis enquêter sur la disparition~-- histoire vraie~-- de Michael Rockefeller, richissime héritier de 23 ans, en Papouasie-Nouvelle-Guinée),

3. échanger quelques rapides paroles avec cette même mère Ouvrard qui n’a pas manqué de s’étonner de me voir en robe de chambre sur la palier,

4. sauter dans l’ascenseur (je rappelle aux moins attentifs d’entre vous que je logeais au sixième et dernier étage d’un immeuble tout confort situé dans les beaux quartiers de la toute proche périphérie urbaine dans ce qu’elle avait de plus humain et progressiste, même si le bâtiment n’était pas de la toute première fraîcheur, chose qui, je parle du fait d’habiter au sixième étage, ne facilite pas vraiment l’approvisionnement quand on se trouve subitement à court de vin et qu’il faut descendre au sous-sol en quatrième vitesse),

5. fouiller un peu et écarquiller des yeux d’enfant émerveillé devant les casiers en fer forgé sur lesquels j’avais empilé les innombrables bouteilles volées à Riqueti (paix à son âme, il n’aurait plus jamais soif, maintenant, et j’avais sauvé d’un avenir incertain le meilleur de son héritage),

6. trouver le Rousseau Deslandes 69, superbe flacon dont chaque grain de poussière et tache de moisi sur l’étiquette avait été religieusement conservés par mes soins, au point que j’osais à peine le prendre en main de peur de l’altérer,

7. méticuleusement refermer à quintuple tour la porte blindée de ma cave adorée qui ressemblait davantage à Fort Knox ou l’ancien bunker du Gothard qu’à un simple lieu de stockage et affinage domestique de produits de bouche et autres denrées de première nécessité,

8. reprendre l’ascenseur en sens inverse,

9. ré-échanger quelques paroles insipides avec cette vieille harpie de mère Ouvrard qui était toujours sur le pied de guerre dans le couloir, la bave aux lèvres, avec son crâne déplumé et son épiderme fripé et parcheminé maculé de taches suspectes, sa moustache et son poil au menton, éternellement chaussée de ses abominables pantoufles à pompon, en robe de chambre élimée de jour comme de nuit, mue par une irrésistible et démoniaque envie de pourrir la vie des gens au point de les pousser à déménager ou se lancer dans une longue et coûteuse psychanalyse, comme c’était le cas de ce pauvre Marc-Antoine Jacquinot, prof de philo au bord du suicide par pendaison dans la cage d’escalier, homme d’une solitude extrême qui faisait quotidiennement les frais de son manque de charisme et d’autorité naturelle (les gens sont cruels),

10. parvenir enfin à échapper à ses griffes et me précipiter chez moi le cœur battant,

et je pouvais de nouveau envisager l’existence non pas comme un long chemin de croix pavé des mauvaises intentions d’une cohorte de nuisibles perpétuellement à l’affût (ne me demandez pas pourquoi, d’où vient cette étrange répulsion, mais il y a des mots sur lesquels je n’ai pas la moindre envie de mettre un accent circonflexe, et affût en fait partie) d’un mauvais coup, mais de longues vacances enchanteresses dans des contrées merveilleuses peuplées de gens éminemment sympathiques et accueillants n’ayant qu’une seule idée en tête : se couper en quatre pour faire votre bonheur.

Pendant mon absence, Tosca et Zarina avaient fait le ménage et déposé au centre de la table un plateau de fromages digne des Mille et Une Nuits, lequel exerçait sur un Zaahid pourtant accoutumé une fascination sans cesse renouvelée, tant il contrastait avec l’indigence fromagère de son Bangladesh natal (même s’il n’y était pas né personnellement, il le considérait comme la terre de ses ancêtres, et, par voie de conséquence, sa terre natale par procuration), réduite à la production plus élémentaire qu’alimentaire de quelques vagues tonnes de fromages de yak (essentiellement pour les chiens, qui sont les seuls à pouvoir le mastiquer), vache maigre et buffle d’eau, plus ou moins frais, quelquefois fumés ou frits, d’une comestibilité douteuse quoi qu’il en soit.

Les deux créatures, qui je le rappelle évoluaient seins nus telles d’innocentes naïades entourées de requins affamés, avaient, en plus des couverts, changé les assiettes, chose qui, je ne vous le cache pas, avait le don de m’exaspérer au plus haut point, et ce d’autant plus qu’il fallait à nouveau les changer, de même que les couverts, pour manger le dessert, terminaison sucrière qui n’était heureusement pas prévue ce soir, que j’estimais redondante, inutilement alourdissante, et retardant d’autant le plaisir ultime et aboutissement dînatoire d’allumer des barreaux de chaise et s’envoyer des digestifs dans une ambiance à mi-chemin entre le gentlemen’s club et le tripot. Je reconnais que bien sûr, de nos jours, les données techniques concernant la vaisselle ont changé du tout au tout. Avant la femme s’y collait avec véhémence, ou les domestiques dans les familles les plus aisées, mais l’exercice était à la fois ingrat et hautement préjudiciable à la douceur des mains, ces mêmes mains qui devaient par la suite servir à des pratiques plus intimes nécessitant, précisément, un épiderme à l’opposé du papier de verre ou du gant de crin. Après la somptueuse Josephine Cochrane et son lave-vaisselle à manivelle en 1886, les frères Walkers (ils s’y sont mis à deux) ont inventé en 1911 le premier lave-vaisselle entièrement automatique, d’abord équipé d’un moteur à gasoil moyennement agréable en espace clos, avant de passer au confort de l’électrique quelques années plus tard. C’est bien évidemment très émouvant, et je ne saurais remettre en question l’utilité de cet appareil, mais je constate qu’aujourd’hui les gens s’en servent à tort et à travers et l’utilisent comme prétexte pour changer les couverts à tout bout de champ, ce qui est éthiquement irresponsable et intellectuellement discutable, même si je n’ignore pas que les tâches ménagères ne sont plus en odeur de sainteté, en admettant qu’elles l’aient jamais été. Il est clair que le fait de pouvoir laver sans effort entraîne ce que j’appellerai une sorte de frénésie dépensière énergétique, d’illusion hygiéniste du luxe comme si on pouvait changer la vaisselle entre chaque bouchée sous prétexte qu’on dispose d’une armée de domestiques pour la faire incessamment. Déjà qu’on ne mange plus avec ses doigts, ce qui me paraît hautement dommageable dans la plupart des cas, je trouve regrettable qu’on ne fasse même plus la vaisselle, même si, je l’admets, j’ai moi-même une sainte horreur de la faire, et plus particulièrement de la rincer et l’essuyer.

Voilà, c’était juste un petit aparté sans grand intérêt ni conséquence, je vous le concède, mais je tenais tout de même à soulever, ne serait-ce qu’un court instant, le sujet de cette épineuse question.

Revenons-en maintenant, si vous le voulez bien, à notre plantureux plateau de fromages.

Il y avait là de la Couronne Lochoise, merveille au lait de chèvre (cru, bien sûr) en provenance de la ferme de La Biche, à Betz-le-Château ; du neufchâtel du pays de Bray, adorable petite chose en forme de cœur fondant ; deux saint-marcellin de toute beauté, débordants d’amour ; un brie de Nangis résolument fermier dans le même état de coulaison superlative, exhalant un fumet d’une puissance à couper le souffle ; et enfin une portion conséquente de l’incomparable tomme de vache de l’abbaye Notre-Dame de Donezan, perchée à près de mille cinq cent mètres dans les Pyrénées ariégeoises.

Tu la sens (je te tutoie, nous sommes tous frères) la douceur de vivre, tu l’entends le petit ruisseau qui coule, fleurant bon le lait cru et l’herbe tendre des vertes prairies hexagonales ?

Vous savez, sur une bouteille de vin de 69, il n’est pas rare que le bouchon, imbibé sur la totalité de sa longueur et rongé par l’acidité, se montre excessivement friable. On ne saurait lui en vouloir, bien sûr, mais cela impose de le manipuler avec des précautions certaines. Pas question de se jeter sur lui et le trousser à la hussarde, l’empaler à grands coups de tire-bouchon comme si on ouvrait une bouteille de beaujolais nouveau (je ne conseille de toute façon à personne de faire, laissons cela à nos amis Japonais qui raffolent de cette infâme piquette, comme d’ailleurs à peu près tout ce qui vient de France, avec une nette prédilection pour les produits les plus ringards et franchouillards, beaujolais nouveau, béret basque, saucisson sec et quiche lorraine, Mireille Mathieu, David Guetta et Joe Dassin, bref tout ce qui leur rappelle de près ou de loin l’odeur de petite culotte de leurs lolitas gothico-victoriennes aux yeux bridés).

Je conseille, dans ces cas-là, d’utiliser un tire-bouchon au pas de vis suffisamment important pour assurer une bonne prise (évitez le Charles-de-Gaulle à ailettes et le tire-bouchon à levier, tout ce qui peut exercer une quelconque forme de torsion sur le bouchon, et veillez à rester bien droit pendant toute la durée de l’opération), de traverser le bouchon de part en part avec ladite vis (avec une extrême délicatesse pour ne pas risquer d’enfoncer l’obturateur dans le goulot), puis, toujours avec la même plus extrême délicatesse, d’extraire l’objet liégeux de son logement millimètre par millimètre, sans jamais le brusquer. Si vous appliquez scrupuleusement cette méthode, vous avez une bonne chance de réussir, sachant que malgré tout le bouchon peut à tout instant se casser en deux ou tomber en miettes. C’est ainsi, on n’y peut rien, mais quand on fait les choses dans les règles de l’art au moins on n’a rien à se reprocher. Selon le millésime, le vin peut concentrer plus ou moins de dépôt dans le fond de la bouteille. Dans le cas d’un dépôt important, qui n’est pas nécessairement mauvais signe, laissez reposer la bouteille à la verticale pendant au moins vingt-quatre heures avant de procéder à l’ouverture, que vous effectuerez dans le même esprit de verticalité et en limitant autant que possible la redispersion des particules dans le breuvage, celles-ci étant généralement chargées d’une amertume tout à fait dispensable à la dégustation.

Sachant qu’il avait affaire à des amateurs de premier choix, et un déboucheur hors pair, rompu à toutes les techniques d’extraction, le bouchon de notre Beaune 69, d’une qualité de liège exceptionnelle, n’a fait aucune difficulté pour quitter le logement qu’il occupait depuis tant d’années. On aurait dit qu’il n’attendait que ça, estimant qu’il avait accompli sa tâche à la perfection et pouvait maintenant rejoindre en toute sérénité le paradis des bouchons, en l’occurrence la boîte dans laquelle je remisais les bouchons les plus mémorables dont j’avais eu le privilège de croiser la route, tous du meilleur liège, et pour la plupart estampillés des millésimes, crus, appellations et châteaux parmi les plus respectables.

Bref, ma joie était intense et sans limite, et celle-ci est encore montée d’un cran quand les premières effluves du vin sont parvenues à mes narines, ouvertes à tous les vents telles des antennes paraboliques pour en capter les plus infimes fréquences.

Souvent, à l’ouverture, un vin de cet âge, qui a passé des décennies à méditer dans le fond d’une cave, tel un moine dans sa cellule, se montre aussi mutique et fermé à double tour que le cul d’une nonne endurcie par la prière et l’abstinence, un trou de balle que seul le doigt de Dieu a eu l’insigne honneur de visiter. Mais cette fois, tel Peter Pan et son Petit Oiseau Blanc (cui-cui), je me suis retrouvé catapulté sans préavis dans les jardins de l’hôpital pour enfants malades de Great Ormond Street, à Londres, peuplé de créatures magiques, d’arbres vivants et de champignons hallucinogènes. J’avais été, moi aussi, cet enfant malade qui pleurait seul dans son coin en attendant la Mort. Celle-ci, sans doute trop occupée par ailleurs, n’était jamais venue, mais elle avait laissé une empreinte indélébile dans le fond de mon cœur, un cœur brisé auquel seule l’absorption d’un authentique philtre d’amour comme le Beaune 1er Cru Cuvée Rousseau Deslandes 69 des Hospices (je pense que c’est de là que venait la référence à Peter Pan et aux enfants malades de Great Ormond Street) pouvait rendre le sourire.

L’heure était venue, pour moi, de passer d’un état de quasi-nudité à quelque chose de plus habillé. Pour servir le vin d’une part, le Rousseau Deslandes 69 (bon, alors, pour information et parce que j’en ai marre qu’on me pose sans cesse la question, qu’on m’abreuve de lettres recommandées et d’injures à caractère raciste sur les réseaux sociaux parce que je n’ai pas tout dit à son sujet, sachez que la cuvée en question, mélange de Cent Vignes, Montrevenots et Clos de la Mignotte, trois premiers crus de Beaune, porte les noms des bienheureux Antoine Rousseau et Barbe Deslandes, sa charmante épouse, fondateurs et généreux donateurs, en 1645, de l’Hôpital de la Sainte-Trinité destiné à accueillir les orphelins de la peste et la Guerre de Trente Ans, rattaché aux Hospices de Beaune sous la Révolution et aujourd’hui dévolu aux personnes âgées) exigeant tout de même une certaine élégance pour être servi avec la déférence due à son rang, et ensuite parce que j’avais rendez-vous dans une heure avec mes potes chasseurs de nazis pour aller nettoyer un nid.

Et moi, quand je nettoie des nids, j’aime assez être physiquement à mon avantage. Certains préfèrent la tenue de camouflage style raid dans la jungle, d’autres l’ensemble tactique façon GIGN avec gilet pare-balles et pantalon cargo, personnellement mon choix se porte plutôt sur le costume trois pièces nœud pap’ genre James Bond, le costume colonial pour les interventions par temps chaud ou encore, même si je la trouve un peu moule-burnes à mon goût, la combinaison de plongée tropicale avec kit anti-requin, fusée de détresse et micropropulseurs intégrés.

Après avoir rempli les verres (du genre aquarium, spécialement conçus pour permettre aux grands vins de Bourgogne et d’ailleurs de barboter en toute liberté et livrer aux narines exigeantes la quintessence de leur palette olfactive) de Tosca et Zarina, j’ai rempli le mien et reposé la bouteille.

\textsc{Zaahid} : Et moi, je n’y ai pas droit ?

\textsc{Moi} : Tu as assez bu comme ça.

\textsc{Zaahid} : C’est une blague !

Il a tenté d’attraper la bouteille, mais j’ai été plus rapide.

\textsc{Lui} : Donne-moi immédiatement cette bouteille !

\textsc{Moi} : Désolé, mais je pense que tu n’es plus en état d’apprécier un tel nectar.

\textsc{Tosca} : On ne va pas donner de la confiture à un cochon.

\textsc{Zaahid}, essayant péniblement de se lever : Donne-moi cette bouteille !

\textsc{Zarina} : Arrêtez, il va faire un malaise !

\textsc{Lui} : Je suis en parfaite santé, et j’exige de goûter ce vin !

\textsc{Tosca} : Tu veux du fromage, mon chéri ?

\textsc{Lui}, furax : Oui, je veux du fromage. Et du vin !

\textsc{Tosca} : Tu veux un peu de ça ?

\textsc{Lui} : C’est quoi ?

\textsc{Elle} : De la Couronne Lochoise.

\textsc{Moi} : Du fromage de chèvre.

\textsc{Lui} : Jamais de chèvre avec le Beaune.

\textsc{Moi} : Ah bon ?

\textsc{Lui} : Avec le bourgogne en général.

\textsc{Tosca}, désignant le brie de Nangis : Alors ça.

\textsc{Lui} : C’est quoi ?

\textsc{Elle} : De la vache.

\textsc{Moi} : Excellent avec le Beaune. Tu veux un peu de vin ?

\textsc{Lui} : Tu te fous de moi ?

\textsc{Moi} : Non. Je te demande juste si tu veux un peu de vin.

\textsc{Lui} : Bien sûr, que j’en veux. Je pense même être le seul expert digne de ce nom autour de cette table !

\textsc{Moi} : Le seul alcoolique, tu veux dire.

\textsc{Lui} : Je ne comprends pas comment on peut boire du bourgogne rouge avec du fromage de chèvre. C’est contraire à tous les principes.

\textsc{Moi}, tapant violemment du poing sur la table, chose qui n’est pas dans mes habitudes tant j’ai toujours prôné la non-violence et la résolution des conflits par le dialogue et l’ouverture d’esprit : Et moi j’emmerde les principes, et je n’autorise personne à me donner des leçons en matière d’accord vin et fromage. Je prétends, non, je ne prétends pas, j’affirme que le Beaune Rousseau Deslandes 69, à égalité peut-être avec le Nuits Les Lavières et Bas De Combe des Hospices du même nom, est le meilleur vin qui existe sur cette putain de terre ravagée par les guerres et le mildiou pour accompagner la Couronne Lochoise, surtout quand elle commence à héberger toute une colonie d’asticots bien vifs et grassouillets, alors que le persillé de Tignes, par exemple, sera plus à l’aise avec un Pommard cuvée Suzanne Chaudron ou même, pourquoi pas, un Corton Charlemagne cuvée François de Salins dans la force de l’âge.

\textsc{Lui} : N’importe quoi ! Du Corton Charlemagne sur du persillé de Tignes ! Pourquoi pas du Chambolle Musigny avec de la chevrette de Novel ou du bleu de Termignon, tant qu’on y est ! Non, si c’est pour entendre ça, je préfère encore aller me coucher sans manger.

\textsc{Moi} : C’est très bon, le Charlemagne avec le persillé.

\textsc{Lui} : Le jambon persillé, à la rigueur.

Et puis soudain, sans crier gare, alors qu’il avait le visage profondément enfoncé dans son aquarium de Rousseau Deslandes et semblait vivre une expérience fascinante de plongée viticole au cœur même de la quintessence du grain, une grosse larme a roulé le long de sa joue avant de tomber dans le liquide, sans doute pour lui apporter une petite touche de salinité supplémentaire.

\textsc{Moi} : Allons bon, qu’est-ce qui se passe, encore ?

\textsc{Lui}, reniflant comme un gamin de cinq ans qui vient de se prendre la fessée de sa vie (pratique aujourd’hui interdite, car on s’est rendu compte que ça pouvait faire naître des idées bizarres dans l’esprit des jeunes enfants, voire déclencher des vocations sado-masos préjudiciables au bon déroulement de leur future vie sexuelle) : Rien.

Tosca l’a pris dans ses bras et lui a caressé légèrement les boules pour tenter de le détendre. Cette technique, bien connue des chamanes bouriates de Mongolie intérieure, consiste à masser doucement les testicules du patient en grande détresse psychologique, généralement avec les mains enduites de lait de renne mélangé à des excréments d’ours polaire, le tout accompagné de chants traditionnels rythmés ou non par des tambours. Tosca, qui ne connaissait pas de chants traditionnels bouriates, les remplaçait avantageusement par des airs du grand poète Livournais Piero Ciampi.

\textsc{Moi} : Ne me raconte pas de conneries, je vois bien que tu fais une gueule de cent pieds de long.

\textsc{Zarina} : C’est l’alcool.

\textsc{Moi} : Sans doute, mais pas que.

\textsc{Tosca} : Il ne tient pas l’alcool. Hein, mon chéri, que tu ne tiens pas l’alcool ?

\textsc{Lui} : Je tiens très bien l’alcool, merci. Aussi bien que n’importe lequel d’entre vous !

\textsc{Moi} : Il y a des gens que ça rend gais, toi ça te rend triste comme un vieux pot de chambre abandonné avec plein de merde au fond.

\textsc{Lui} : Je suis pas triste. C’est juste que… que…

\textsc{Moi} : Que quoi ? Tu peux arrêter de chanter une seconde, Tosca ?

\textsc{Elle} : Non.

\textsc{Moi} : Comment ça, non ?

\textsc{Elle} : Non, je ne peux pas arrêter de chanter. Zaza (c’était le surnom ridicule qu’elle avait donné à Zaahid) ne va pas bien, et je sais mieux que personne ce qui peut l’aider à aller mieux.

\textsc{Zarina} : Je suis d’accord avec elle.

\textsc{Moi} : T’es toujours d’accord avec elle.

\textsc{Zarina} : C’est ma sœur et on est toujours d’accord sur tout.

\textsc{Moi} : N’empêche que si elle pouvait arrêter de couiner et lui masser les couilles pendant que je lui parle, ça pourrait peut-être nous permettre d’avancer un peu dans le débat.

\textsc{Zaahid} : C’est juste que je pensais à Jaya, voilà ! Pas la peine de s’engueuler pour si peu.

Dès qu’il en avait un coup dans le nez, le sujet de Jaya, sa fille adorée, revenait sur le tapis. Et pour cause, il ne la voyait plus. Elle l’avait désavoué, rejeté, le tenant pour responsable de la mort de sa mère, Faustina Barreira, dont elle avait retrouvé le corps sans vie en rentrant du collège. Expérience ô combien traumatisante pour une ado de douze ans, qui avait coïncidé avec le début d’une interminable descente aux enfers, un spectacle d’une désolation telle que même les plus charitables avaient fini par détourner pudiquement les yeux et prendre discrètement la tangente. Après un parcours exemplaire dans l’autodestruction et le cumul frénétique de tout ce qu’il ne faut pas faire si on veut avoir une chance de réussir dans l’existence, elle avait croisé la route de Simon Keskula, le fondateur de l’Alliance de la Révélation, une communauté survivaliste installée au pied de la Montagne de Lure, dans les Alpes-de-Haute-Provence, dont la spécialité était précisément de recueillir les âmes perdues pour les remettre dans le droit chemin et redonner un sens à leur vie. Le sens en question passait bien évidemment par la conscience aiguë du fait que les problèmes ne viennent jamais de vous, innocente créature qui n’avez rien demandé à personne et ne demandez qu’à vivre en paix avec elle-même et les autres, mais des autres, et tout particulièrement d’une bande de politiciens véreux qui vous imposent des règles absurdes dans le seul but de se remplir les poches à vos dépens. Et si, par malheur, vous vous retrouvez dans une situation telle que vous ne leur êtes plus d’aucune utilité, ils s’empressent de vous laisser tomber et vous désigner aux yeux de la société comme un résidu de fausse couche indigne de toute espèce de considération (sachant qu’un con, sidéré ou pas, reste un con quoi qu’il arrive), un déchet de l’humanité tout juste bon à crever la gueule ouverte sans que personne ne lève le petit doigt pour lui venir en aide (ou alors si, l’abbé Pierre, mais si c’est pour se retrouver avec le petit doigt où je pense, je ne suis pas certain que le jeu en vaille la chandelle).

Aujourd’hui âgée de vingt-deux ans, enfermée à double tour dans la communauté de Keskula (frère Simon, comme l’appelaient ses disciples avec des yeux remplis d’un amour et d’une reconnaissance éternelle qui faisaient plaisir à voir, et aussi un peu froid dans le dos), Jaya avait coupé tout lien non seulement avec son père, qui ne parvenait toujours pas à recoller les mille et un morceaux de son cœur éparpillés sur le froid carrelage de l’existence, mais aussi avec tous les gens qu’elle avait pu connaître de près ou de loin par le passé. Keskula prônait une vie en autarcie, dans le genre amish ou mennonite, totalement autosuffisante et sans contact avec le monde extérieur, considéré comme éminemment malsain et corrupteur, source de tous les maux et fossoyeur d’une humanité moribonde au bord de l’extinction. À l’Alliance de la Révélation, on cultivait son potager, on élevait ses poules, on gobait ses œufs, on se laissait pousser les cheveux, on ne se rasait plus les poils nulle part, on se baladait les fesses à l’air et on faisait caca dans les toilettes sèches prévues à cet effet. C’était comme une giclée de fraîcheur (même si ça ne sentait pas toujours le muguet) dans le gosier malodorant de la civilisation, un doigt dans le cul du consumérisme, un bras d’honneur à la science et la technologie. Pour ses disciples, encore assez peu nombreux mais d’une dévotion sans faille, proche du fanatisme religieux, frère Simon était un visionnaire, un nouveau Noé envoyé sur Terre pour construire une nouvelle arche et guider les heureux élus vers un monde meilleur, à l’image d’un Elon Musk qui construit des arches supersoniques pour guider les heureux élus milliardaires vers la planète rouge, terre promise des survivants de la guerre atomique qui se profile à l’horizon. La déflagration sera totale et tous les pauvres seront exterminés comme des cafards, rayés de la carte et condamnés à servir d’engrais radioactif pour les mutants nécrophages qui vivent dans les galeries souterraines et se nourrissent de déchets organiques. Mais ceci n’est qu’une théorie, bien sûr, car il n’existe à ce jour aucune preuve tangible de l’existence de ces créatures, créatures dont la forme exacte (quelque part entre le ver de terre, la sangsue et le dragon de Komodo, charognard et cannibale de sinistre réputation), reste purement spéculative, même si quelques traces suspectes semblent avoir été aperçues ici et là dans certaines zones parmi les plus désertiques et inhospitalières de la planète. Comme Musk, Keskula incarnait une nouvelle forme de messie 2.0 à équidistance entre le national socialisme, la scolastique médiévale et le néo-darwinisme cyberpunk, survivaliste et homophobe de l’incroyable HOC, l’Homme à l’Oreille Coupée, surnommé, en raison de la couleur pour le moins inhabituelle de son épiderme et la toxicité de sa personne, l’Agent Orange par Busta Rhymes, rappeur East Coast sujet à l’embonpoint, et même, au choix, la Limace ou l’Anus Orange par Rosie O’Donnell, actrice elle-même traitée de «~grosse plouc moche~» par l’intéressé, piètre ornithologue mais grand amateur de noms d’oiseaux. Selon eux (et il m’arrivait parfois, quand le long manteau noir du découragement s’étendait tel un linceul sur le peu d’optimisme qui restait en moi, d’être tenté de souscrire à leurs thèses frelatées), il fallait se rendre à l’évidence : les bons sentiments ne menaient nulle part et seul le pragmatisme le plus échevelé pourrait permettre à l’espèce humaine de survivre à sa mort annoncée depuis des lustres par des gens ayant le pouvoir, sinon de lire l’avenir, au moins d’en évoquer les contours sinueux à travers d’obscures métaphores semblables à des carreaux poussiéreux ne laissant transparaître qu’une infime partie de la lumière du jour. Ainsi, le mode de vie prôné par frère Simon privilégiait la reconnexion au monde et la fusion dans un espace-temps autorisant les projections les plus folles, optimistes et ambitieuses. Un avenir serein se dessinait sous les yeux ébahis des fidèles, représenté au cours des ateliers d’expression plastique dirigés par le Maître sous la forme de mandalas fortement inspirés par les kalams anthropomorphes en poudre de riz du Kérala.

Vous allez me dire : tout cela est ridicule, et je vous répondrai : oui, sans aucun doute, mais le respect des croyances n’est-elle pas la garantie d’une vie harmonieuse en société.

Aussi, ami lecteur dont je salue une fois de plus la résilience et la fidélité, il se peut fort bien que ton cerveau de qualité supérieure (même si vraisemblablement pollué aux pesticides) ait le plus grand mal à digérer favorablement le flot d’informations que je viens de déverser dans tes synapses surexcitées. Par contre, ce que tu pourras comprendre sans difficulté, c’est que mon ami Zaahid, en dépit d’une sensibilité élevée, une appétence toute particulière pour la spiritualité et une tendance naturelle à la bienveillance, ne pouvait se résoudre à voir la chair de sa chair s’évaporer dans la nature, aussi idyllique soit-elle.

Raison pour laquelle je lui ai dit, tandis que son gros nez plein de morve s’épanchait joyeusement dans le liquide prestigieux que j’avais commis l’erreur de lui servir (je me consolais en me disant que je ne l’avais pas payé cher, même si, en y réfléchissant, j’aurais aussi bien pu le payer de ma vie) : T’inquiète, je vais aller te la chercher, ta fille.

\textsc{Lui}, des sanglots dans sa voix qui n’était plus qu’une éponge imbibée de sonorités humides et incertaines : Mais elle ne veut plus me voir !

\textsc{Moi} : La première chose à faire est de la sortir des griffes de ce taré. J’ai fait ma petite enquête, Greg aussi, et il est arrivé aux mêmes conclusions que moi, à savoir que Keskula n’est pas à proprement parler un perdreau ou plutôt un vautour de l’année. Il partage de nombreux points communs avec Matthias Schuster, le curé de Montaulogne, ou encore Charles Manson, même s’il n’envoie pas encore ses disciples défoncés massacrer des stars de cinoche dans leurs villas de luxe de Benedict Canyon. Pas encore, en tout cas. Il faut dire que le Ravin des Avelines, le trou à rats où ils ont élu domicile, n’a pas grand-chose à voir avec Beverly Hills. Par contre, ça ressemble un peu au Barker Ranch de Manson, dans la vallée de la Mort. Comme lui, et Schuster le faux pasteur, il est né d’une mère alcoolique et prostituée, avec un père absent qu’il n’a fait qu’entrevoir au gré de ses rares visites au domicile conjugal, bref un grand classique de la misère humaine, fleuron de la tragédie populaire qui fait pleurer dans les chaumières et émeut même les cœurs les plus endurcis. Quand sa pute de mère s’est retrouvée en taule pour avoir assassiné de cent cinquante coups de couteau, déchaînement de violence, qu’ils ont dit, un client un peu plus entreprenant que les autres, il a été ballotté de famille d’accueil en famille d’accueil, toutes plus pourries les unes que les autres, où il a subi son lot de violences morales, physiques et sexuelles, de quoi bien lui faire comprendre que la vie n’était sans doute pas le camp de vacances qu’on lui avait vendu à la naissance. Du coup grosse déception, tristesse, dépression, et surtout forte tendance, comme il est d’usage dans ce cas de figure, à collectionner les actes de délinquance, allant du vol à l’étalage au viol en réunion, avant de réaliser, au détour d’un séjour en prison pendant lequel il a eu tout le loisir de méditer sur ses péchés et faire le point sur sa carrière professionnelle, qu’il était la réincarnation de Jésus et que sa mission, si toutefois il l’acceptait, était de fonder sa propre église 
% LTeX: language=off
e que s'apelerio
% LTeX: language=fr
l’Alliance de la Révélation, havre de paix pour les âmes en peine et terreau de renaissance glorieuse d’une civilisation qui courait à toutes jambes à sa perte. Une bien belle histoire, n’est-ce pas, pleine de fraîcheur et d’eau de source, parfaite carte postale qui cache, j’en ai bien peur, une tout autre réalité.

\textsc{Zaahid}, qui n’en démordait pas, au point qu’il en devenait parfois assez pénible à supporter : N’empêche que Jaya ne veut plus entendre parler de moi.

\textsc{Moi} : Chaque chose en son temps. D’abord on la sort des griffes de Keskula, après on l’envoie en cure de désintox avec suivi psychologique et tout le tintouin. Ça prendra le temps que ça prendra, mais elle finira par te revenir, et ensemble, main dans la main, vous pourrez faire le deuil de Faustina.

\textsc{Lui} : Si seulement !

\textsc{Moi} : Pas question qu’on laisse ta fille pourrir dans ce cloaque ! T’as vu Expendables, avec Sylvester Stallone, Jason Statham, Mickey Rourke, Dolph Lundgren et Jet Li ?

\textsc{Lui} : Non. Excuse-moi, mais j’ai autre chose à faire que regarder ce genre de conneries.

\textsc{Moi} : Ben t’aurais dû, au lieu de faire ton gros intello qui se la pète, parce qu’il se trouve que Titus, Greg, Sam, Maël et moi-même, on forme une équipe tout à fait comparable, un peu de muscle en moins mais beaucoup de cervelle en plus. S’il faut extraire Jaya manu militari de ce repaire de troglodytes d’extrême-droite, on n’hésitera pas à mouiller la chemise et je te garantis que tu reverras ta fille en un seul et unique morceau.

\textsc{Zarina} : Et c’est qui, Stallone ?

\textsc{Moi} : Stallone, ma petite chérie d’amour que j’aime plus que le chianti, le jambon de Parme, Mozart et la mozzarella réunis, c'est-à-dire globalement plus que tout au monde y compris moi-même qui ne suis qu’un pâle trublion de l’existence, un saltimbanque sans consistance, c’est un petit gars d’origine italo-bretonne qui a mijoté dans les casseroles de la Cuisine de l’Enfer, à Manhattan, passablement handicapé après une naissance au forceps, cette pince à bébé qui avait un peu trop tendance à le réduire en bouillie ou achever la mère une fois sur deux. Mais c’est d’abord et avant tout Rocky Balboa, alias l’Étalon italien, perso je dirais plutôt l’Étalon sur les talons (NDLR : bide total, le jeu de mots n’ayant été capté par personne sauf Zaahid, lequel, déjà difficile à décoincer en temps normal, s’est contenté d’un vague ricanement teinté de condescendance affligée), boxeur amateur dur au mal, con comme un balai et à peu près aussi séduisant qu’un rat crevé dans le caniveau, qui tente de gravir un par un et sur les genoux les échelons de la gloire et conjurer le sort qui s’acharne sur lui.

\textsc{Elle} : Je ne suis pas une spécialiste du cinéma comme toi et encore moins des films d’action, mais je connais Stallone, merci. Non, ce que je te demandais, c’est qui joue le rôle de Stallone dans votre petite équipe ?

\textsc{Moi} : Je sais pas, mais en tout cas c’est pas moi.

\textsc{Zaahid}, jamais à court de méchanceté : Je dirais le plus con et le plus costaud de la bande.

\textsc{Zarina} : Titus ?

\textsc{Moi}, révolté : Stallone n’est pas noir ! Et puis il est costaud, c’est vrai, mais loin d’être con. Je me demande s’il n’y a pas un petit fond de racisme refoulé, chez vous.

\textsc{Tosca} : Il plaisantait. Hein, Zaza, que tu plaisantais ?

\textsc{Zaza}, du fromage plein la bouche : Bien sûr, que je plaisantais. J’adore ce garçon.

\textsc{Moi}, d’une voix légèrement ébréchée par l’émotion que le seul fait d’évoquer de nom de mon fidèle lieutenant suscitait en moi : C’est le type le plus droit et honnête que je connaisse. Tu peux l’appeler au milieu de la nuit, il bondira de son lit pour voler à ton secours. Il a un sens de l’amitié qui défie l’entendement, et ça, pour moi, c’est une des choses les plus précieuses qui soient.

\textsc{Zaahid} : On se demande pourquoi.

\textsc{Moi} : Pourquoi quoi ?

\textsc{Lui} : Pourquoi c’est une des choses les plus précieuses qui soient.

\textsc{Moi} : Tu ne trouves pas que c’est agréable et rassurant de pouvoir compter sur quelqu’un en toute circonstance ?

\textsc{Lui} : Si, c’est agréable et rassurant. Enfin, surtout pour toi. Parce que pour lui, je ne sais pas si c’est agréable et rassurant de savoir que tu peux l’appeler à tout moment pour lui demander de faire des choses pas forcément très agréables ni rassurantes. Tu vois ce que je veux dire ?

\textsc{Moi}, serrant les poings en signe de victoire : Mes amis, mes amours, mes emmerdes, pour citer le grand poète franco-arménien Charles Aznavourian, je crois que notre Zaahid va mieux ! Quand il commence à couper les cheveux en quatre pour chercher la petite bête, c’est que la vie reprend ses droits et qu’on ne va pas tarder à le retrouver au meilleur de sa forme.

\textsc{Lui}, définitivement bougon : Non, je ne vais pas mieux, je ne vais pas bien et je me sens horriblement mal. J’en ai marre de tout, de Stallone, des Gardiens de la Révolution, du fromage et du reste ! Sans Jaya, ma vie n’a plus aucun sens.

\textsc{Moi} : C’est pas les Gardiens de la Révolution, c’est l’Alliance de la Révélation. Keskula est une ordure, je te l’accorde, mais je le préfère encore à Salami.

\textsc{Tosca} : AU salami.

\textsc{Moi} : Non, à Salami. Hussein Salami, le chef des Gardiens de la Révolution.

\textsc{Lui} : Jambon, salami, pâté de foie, tête de veau, Gardiens de la Révélation, Alliance de la Révolution, gardiens de mes couilles, alliance de mes deux, qu’est-ce que j’en ai à foutre ! Tout ce que je veux, c’est que ma fille revienne à la maison !

\textsc{Moi} : Et elle va revenir, je t’en donne ma parole. Mais d’abord, si ça tu n’y vois pas d’inconvénient, je vais aller rendre une petite visite aux Disciples de la Colère et leur faire passer l’envie de défiler au pas de l’oie. Le Stechschritt comme disent nos amis allemands, dans cette langue si douce et mélodieuse qui est la leur. Je me demande s’ils vont arrêter de nous faire chier un jour, ceux-là. Quand je pense qu’il n’y a même pas cent ans ils étaient encore en train de gazer des juifs par millions, et qu’ils ont déjà le culot de nous ressortir un putain de parti nazi en bonne et due forme, comme si de rien n’était, en plus de leurs bagnoles hors de prix et leur bouffe de merde ! Sans parler des Ritals, des Hongrois, des Belges qui suivent le même chemin, comme un seul homme, tous unis dans la connerie et marchant d’un pas décidé vers le néant. La France tente de résister, tant bien que mal, au souffle nationaliste qui fait tanguer l’Europe, mais la gangrène s’installe et la mort fera son œuvre si on ne se décide pas à trancher dans le vif. L’Homme est un cancre, le dernier de la classe. Non seulement il n’apprend rien, mais il prend un malin plaisir à répéter ses erreurs, encore et encore, à faire ses gammes sur le piano de l’horreur, écrire la symphonie du chaos, le concerto pour le pied gauche, celui qui porte bonheur quand on marche dans la merde. On nous bassine avec les images de camps de concentration, les SS qui font sauter des cervelles pharisiennes en rigolant, cadavres ambulants en pyjama rayé, spectres buréniens (relatif à Buren, l’artiste conceptuel, et accessoirement anagramme de rubénien, relatif à Rubens, auteur des Trois Grâces, souvent parodié en Trois Grosses, ou Trois Grasses, qui ne donnent pas forcément une image très flatteuse de la femme du XVIIe siècle) aux yeux hagards, entassés comme des carcasses de bestiaux dans des charniers à ciel ouvert, montagne de cheveux et de dents en or, le trésor de guerre de la glorieuse armée allemande, le devoir de mémoire par-ci, plus jamais ça par-là, et dans le même temps les mêmes relents nauséabonds reviennent au pas de charge empuantir l’atmosphère, avec la complaisance et la complicité active de tous les profiteurs de la Cinquième République, cette bande de tiques accrochées au cul du Pouvoir qui se pavanent sur les plateaux de télévision et jouissent impunément d’une telle pléthore d’avantages et privilèges que le seul fait d’en parler me donne envie de gerber ! Mais bon, je m’emporte, je m’emporte, alors que je ferais mieux de rester focus sur l’objet de la mission. Donc, voilà ce que je vous propose : on va tranquillement se siffler un ballon de fine au salon, musique douce, lumière tamisée et plus si affinités, fumer un bon cigare en parlant de tout et n’importe quoi, après quoi je prends mes cliques et mes claques et vais faire ce que j’ai à faire, même si je vous avouerai franchement que j’aimerais autant rester ici avec vous.

Un jour, ma chère et tendre Zarina, splendeur maléfique qui avait incarcéré mon cœur dans la prison de l’amour, avait décidé comme ça, l’air de rien, que mon salon était indigne de sa présence. Pas spécialement la déco, réduite à quelques toiles de maîtres volées ici et là, ni la pièce en elle-même, d’une surface certes modeste mais amplement suffisante dès lors qu’il s’agissait de passer l’aspi et qu’on n’avait pas les moyens de s’offrir une femme de ménage (fonction avec laquelle, comme nombre de mes semblables qui n’avaient pas été prévenus que les femmes étaient des hommes comme les autres et inversement, je n’avais strictement aucune affinité, mais que Zarina, jusqu’à présent et mon petit doigt me murmurait dans le creux de l’oreille que ça n’allait pas durer éternellement, assumait avec une bonne humeur relative, sachant qu’elle n’avait de toute façon pas le choix si elle ne voulait pas cohabiter avec des tonnes de poussière abritant des légions d’acariens tous plus moches les uns que les autres), mais plutôt le mobilier dans son ensemble, et tout spécialement les fauteuils et le canapé censés permettre aux invités de poser agréablement leurs fesses. Certes, je les avais depuis des lustres et le risque de passer à travers devenait chaque jour plus menaçant, telle une araignée grossissant à vue d’œil jusqu’à atteindre des dimensions éléphantesques, mais j’avoue que le célibataire endurci que j’étais et pensais rester (jusqu’à la date fatidique de notre rencontre, raz de marée qui avait totalement bouleversé le paysage de mon existence) n’y prêtait pas la moindre attention. Depuis qu’elle s’était installée chez moi, elle avait bien évidemment entrepris de refaire l’appartement à son goût, sans se soucier le moins du monde de mon opinion, au motif que les femmes qui, je le rappelle, sont des hommes comme les autres, sont néanmoins bien plus compétentes en la matière, et au point qu’il m’arrivait parfois de penser m’être gouré d’appartement en poussant la porte de ce qui avait jadis été chez moi. J’avais reconnu (il aurait fallu être complètement con et de mauvaise foi pour prétendre le contraire) qu’ils n’étaient pas de la toute première fraîcheur, faisant valoir dans le même temps qu’il n’était cependant pas dans mes intentions de les changer, pour des raisons diverses, plus ou moins complexes et avant tout financières, argumentaire qui avait été jugé irrecevable par mon interlocutrice, laquelle avait immédiatement décidé de prendre l’ensemble des transformations à sa charge, ses moyens le lui permettant aisément d’une part, le plaisir de le faire n’ayant pas de prix de l’autre. Difficile de refuser dans ces conditions atmosphériques.

Son choix s’était porté, et le mien aussi je ne vous le cache pas, sur un ensemble chesterfield de haute lignée, fabriqué à partir des meilleurs cuirs, d’occasion mais en parfait état, vintage comme on dit, issu d’un club de gentlemen anglais qui avait récemment mis la clé sous la porte. Autrement dit, fauteuils et canapé avaient été patinés par des générations de culs prestigieux, de sorte qu’on ne pouvait pas ne pas se sentir investi du poids écrasant de l’Histoire quand on avait le privilège de s’y enfoncer jusqu’au nombril.

C’est dans ces monuments historiques que nous avons pris place, Tosca et Zarina dans le canapé, Zaahid et moi-même dans les deux somptueux fauteuils qui se trouvaient de chaque côté.

J’ai sorti quatre énormes verres ventrus et courts sur patte, une bouteille de vieille fine Napoléon des années 50 (provenance Riqueti) et une boîte de Fuente Hemingway Work Of Art, considéré par nombre d’amateurs comme un des meilleurs de la gamme (qui comprend aussi les vitoles Short Story, Best Seller, Masterpiece, Between the Lines et Untold Story) crée par la maison Fuente (sise comme chacun sait à Santiago de los Caballeros, en République dominicaine) en hommage à Ernest Hemingway, écrivain bipolaire, paranoïaque et alcoolique, grand amateur de cigares et de chats polydactyles (il en avait près d’une centaine dans sa propriété de Key West, en Floride, une bien étrange passion dont on a toujours le plus grand mal à percevoir la finalité initiale).

La plupart des femmes détestent le cigare, et il faut bien reconnaître que son odeur, comme celle de l’époisses ou du munster, ces deux fromages à pâte molle bien connu des tables françaises, ou encore celle de la merde ou l’andouillette, et des abats en général (pour lesquels je n’ai jamais fait mystère de ma passion dévorante), n’est pas toujours des plus agréables, ce qui bien évidemment n’altère en aucune façon, bien au contraire serais-je tenté de dire, la valeur gustative de l’objet du délit. De nombreuses personnes y sont sensibles (y compris des hommes, soyons honnête), plus ou moins, mais parmi celles-ci, tout en haut du podium, se trouve cette créature à la fois mythique et bien réelle qu’on appelle une femme, pour qui toute odeur suspecte représente une agression caractérisée passible des pires représailles, à commencer par un tirage de gueule en bonne et due forme pouvant s’étaler sur plusieurs semaines (je conseille, pour rentrer en grâce, une attitude de soumission absolue accompagnée d’offrandes somptueuses humblement déposées à ses pieds). Je suppose que le fait de s’asperger de parfum et se tartiner de cosmétiques à longueur de journée n’est pas pour rien dans cette hypersensibilité olfactive, raison pour laquelle nous autres hommes, si nous voulons un jour (ou peut-être une nuit) pouvoir lui souffler l’épaisse fumée de nos cigares dans les naseaux sans l’entendre hurler à la mort et nous vouer aux gémonies, ferions sans doute bien de lui rappeler que nous n’aimons rien tant que la nature dans son expression la plus pure et dépourvue d’artifice. Non, nous n’avons pas besoin qu’elle se barbouille le visage de couleurs criardes pour la trouver belle, ni qu’elle s’impose diverses tortures physiques pour continuer à susciter notre intérêt. Nous l’aimons telle qu’elle est, jeune de préférence, c’est vrai, plutôt que molle et fripée, imago dépliant ses ailes au sortir de la chrysalide plutôt que vieille larve desséchée, mais sommes tout disposés à contempler le spectacle de sa décrépitude avec d’autant plus de bienveillance que nous conservons précieusement le souvenir de sa splendeur passée.

Par chance mon adorée, en plus des innombrables qualités qu’elle ne cessait de révéler avec une régularité constante qui ne tarissait guère de m’émerveiller, était grande amatrice de cigares, en particulier cubains, qu’elle (ainsi que sa sœur, qui ne crachait pas sur un barreau de chaise de temps à autre) avait commencé à fumer dès son plus jeune âge sous l’impulsion de son père, le très charismatique comte Stefano Fragale Di Brizzi, hélas tragiquement décédé trois ans plus tôt après avoir percuté un sanglier sur la route de Scarperia, non loin de Florence. Le comte, pourtant fin pilote, n’avait rien pu faire pour l’éviter. Sa 250 GT California Spyder, somptueux cabriolet estimé à plusieurs millions d’euros, avait été pulvérisée sur le coup, ce qui représentait quand même une grosse perte pour l’histoire de la carrosserie automobile et aussi pour les collectionneurs de Ferrari qui n’auraient pas hésité à s’entretuer à grands coups de club de golf, épingle à cravate ou pic à glace pour l’acquérir, sachant qu’il n’en restait guère plus d’une cinquantaine dans le monde. Le comte, qui comptait de nombreux amis parmi les moines et les ecclésiastiques en général, venait d’effectuer une rapide retraite au fond des bois de chênes chevelus, bien au frais dans le couvent de Bosco ai Frati, dans le but avoué de se recentrer, reconnecter ses chakras avec les forces cosmiques et énergies positives de l’univers, retrouver le chemin d’une foi qu’il avait parfois tendance à égarer dans les excès en tous genres, à commencer par l’alcool, les femmes et les mondanités, et accessoirement, ainsi que les réflexes acquis au long de sa longue et brillante carrière d’antiquaire lui commandaient instamment de le faire, tenter de s’approprier par tous les moyens, y compris les plus vils, certains tableaux et objets d’art présent dans l’enceinte de l’établissement. Ce soir-là, le comte avait méchamment fêté son départ avec ses potes moines qui n’étaient pas les derniers à rigoler, en dehors des heures de service bien entendu, et on peut avancer qu’il était sérieusement éméché au moment de prendre le volant, fait dont il était malheureusement coutumier. Le sanglier, lui, est un animal qui ne s’embarrasse pas de formules de politesse. Il fonce tête baissée, et malheur à celui qui à l’imprudence ou la malchance de se trouver sur son chemin. Sauf que le sanglier, il a beau être très très costaud, quand il se mange une 250 GT lancée à vive allure, avec au volant un type en état d’ébriété qui mâchouille un havane et écoute Lucia di Lammermoor de Donizetti à fond sur son lecteur CD dernier cri (pas d’origine, c’est vrai, mais en 1961 il n’y avait pas de lecteur CD), eh ben le sanglier, tout costaud qu’il est, il explose comme un fruit pourri et n’a plus qu’à aller ramasser ses défenses sur le bitume. Les secours ont retrouvé le sanglier les quatre fers en l’air sur le bord de la chaussée, la Ferrari en vrac dans le fossé, et le comte en vrac dans la Ferrari. Ils ont mis des heures à l’extraire de la tôle froissée, rassembler tous les morceaux et les remettre dans le bon ordre. Le comte ne voyageait pas seul : dans le coffre de sa voiture, ils ont trouvé, outre quelques objets ayant appartenu à des résidents célèbres du couvent comme Jean de Pérouse et Giuliano della Cavallina, une petite huile sur bois de très belle facture signée Vittorio Brancaccio, bras droit de Fra Angelico et assurément son élève le plus doué. Sur le marché noir américain, ce genre de pièce pouvait se négocier facilement dans les quarante ou cinquante mille dollars. Les Ricains, dans leurs villas ultramodernes de New York, Los Angeles, Dallas, San Francisco, Houston, Miami, Boston ou Chicago, aiment s’entourer de reliques en provenance de la vieille Europe, cet endroit bizarre que leurs ancêtres ont décidé, quatre siècles plus tôt, de quitter pour venir s’installer sur une terre vierge peuplée de sauvages (qui, soit dit en passant, ne l’entendaient pas du tout de cette oreille, et qu’il a fallu par conséquent méthodiquement massacrer pour arrêter de se faire scalper et planter des flèches dans le cul). Cette nostalgie du vieux continent, somme toute bien compréhensible de la part de gens en quête de leurs racines véritables, autres que la très sainte bible et les 400 millions d’armes à feu présentes sur le territoire (et je parle des civils, pas des militaires lourdement armés), soit au moins une par habitant (même si certains n’en possèdent pas alors que d’autres en ont trois ou quatre, en fonction de la passion qui les anime et du sentiment d’insécurité qui les habite), encourage l’OPA des Américains sur le patrimoine européen. La France et l’Italie, qui jouissent d’une réputation particulière en matière de richesse culturelle, sont des cibles privilégiées, et il se peut fort bien qu’on assiste un jour ou l’autre au spectacle hallucinant de châteaux, manoirs et cathédrales emportés d’une seule pièce par la voie des airs d’un continent à l’autre. On en est encore, pour l’instant, au stade de la reconstruction le plus souvent grotesque, caricaturale et kitschissime des édifices en question, du genre gros gâteau à la crème posé au milieu d’un écrin de verdure aseptisée, mais il faut s’attendre à tout, surtout si les Ricains s’allient avec la Corée du Nord, les Russes, les Indiens (le premier sinistre Modi aimerait bien jouer dans la cour des grands, lui aussi, et commence à s’agacer de voir qu’on ne lui prête pas plus d’attention qu’à un intouchable dans un bidonville de New Delhi) et les Chinois pour nous réduire à néant, raser le Moyen-Orient dans la foulée, faire de la bande de Gaza la plaque tournante du speed, de la cocaïne et des amines sympathicomimétiques, du Japon un centre d’essais nucléaires et du sanctuaire shinto d’Izumo un centre d’amaigrissement et de remise en forme pour oligarques dépressifs, du Canada et du Groenland des stations de ski trois étoiles avec tire-fesses en or et chalets tout confort, et de l’Afrique (enfin débarrassée de ses dictateurs ridicules boudinés dans leurs uniformes de carnaval et coiffés de bérets en peau de léopard) une mine à ciel ouvert, avant bien sûr de s’entretuer joyeusement entre amis jusqu’à ce qu’il n’en reste qu’un (à noter que les Américains, au cas où les choses tourneraient au vinaigre, ont prévu de se replier sur Mars, nouvel eldorado de l’espace et terre promise pour fonder la colonie de l’avenir).

