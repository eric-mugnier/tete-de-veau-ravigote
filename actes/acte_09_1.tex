
\noindent Cette nuit-là, j’ai fait des rêves étranges dont seuls les épisodes les plus marquants surnagent encore dans ma mémoire.

Nathan, Sam, Greg, Titus et moi, accompagnés de Molawa VIII (voir note en bas de page) et ce bon vieux Jim Bowie, ancien trafiquant d’esclaves en Louisiane, caracolions dans la vallée de San Fernando à la recherche d’un temple maléfique dédié à Atiena, la Gardienne de la Nuit au corps de rêve et aux yeux vert céladon. Sous la forme d’un centaure entouré d’un harem de juments de Thessalie, le général Antonio Lopez de Santa Anna y Perez de Lebron, alias l’Aigle du Mexique, le Napoléon du Nouveau Monde ou encore le Héros Immortel de Cempoala, leur filait le train en tirant des coups de feu dans tous les sens avec ses Colt 1860 Army calibre 44 PN (il en avait un dans chaque main et chevauchait à toute berzingue en s’agrippant à son canasson à la seule force de ses cuisses trapues couverte d’une pilosité abondante).

En équilibre instable sur le dos d’une mule pas très accommodante, les couilles en vrac à force de rebondir sur le dos de la bête, je ne vous cache pas que j’avais toutes les peines du monde à suivre le rythme endiablé de cette folle cavalcade.

À noter aussi, chose qui ne va pas nécessairement de soi compte tenu du contexte susdécrit, que je ne portais ni cache-poussière, pantalon de cuir, santiags ni sombrero, mais un peignoir Aescwig Paige collection printemps-hiver 2017 en laine de soie peignée et microfibre de bambou 100\% bio du Suriname (cadeau de mon ami Zaahid Shirani, légiste de génie et beau-frère potentiel qui possédait exactement le même et prenait le plus grand plaisir à l’enfiler~-- entre autre~-- sitôt rentré chez lui après une dure journée de labeur) et des pantoufles en peau de Cottontail du désert (Sylvilagus auduboni, une espèce qu’on ne rencontre guère que dans les steppes semi-arides du désert de Sonora).

Pas non plus de six-coups tonitruant pour moi, mais juste Manu, mon fidèle 6.35 Manufrance, une arme de collection à laquelle j’étais attaché comme à la prunelle de mes yeux, vous le savez maintenant (il avait appartenu à mon grand-père Philibert, résistant de la première heure dont l’esprit taquin veillait en permanence sur moi depuis les hautes sphères où il résidait, en paix avec lui-même, sans doute occupé à tremper ses vers dans l’eau claire de quelque céleste ruisseau éternellement poissonneux, entouré de naïades au physique de reine de beauté attentives à satisfaire ses moindres désirs). À côté de la mitraille ambiante, les détonations de Manu ressemblaient aux jappements d’un de ces foutus roquets qui ne perdent jamais une occasion de brailler même lorsqu’ils se retrouvent face à un molosse qui fait cent fois leur taille, lequel molosse les regarde généralement de haut, l’œil morne, un filet de bave au coin du bec, sans y prêter plus d’attention qu’à une feuille morte balayée par le vent, dédain qui ne fait que pousser lesdits roquets à hurler de plus belle jusqu’à l’extinction de voix.

Note de bas de page, toujours pas en bas de page pour les mêmes raisons pratiques et esthétiques que précédemment : Molawa VIII, chef hottentot de la région du Cap, s’est éteint en 1830. Récupéré puis empaillé comme une vulgaire charogne exotique par les frères Verreaux, antiquaires peu portés sur le respect des morts mais beaucoup sur l’argent, il finit, début vingtième, par échouer entre les mains de Francesc d’Asis Darder et Llimona, vétérinaire catalan secoué du bocal et naturaliste pervers officiant en tant que directeur du zoo de Barcelone. Au crépuscule de sa vie, passée à amonceler jalousement tout ce que la nature a produit de plus insolite et terrifiant depuis les origines du monde, Darder lègue sa collection de reliques horrifiques au musée municipal de Banyoles, sa ville natale. Lance à la main et revêtu de son seul pagne bouffé aux mites, sa Majesté MOLAWA trône en bonne place dans la salle principale, au milieu des momies, têtes réduites, peaux de serpents, monstres en tous genres et autres joyeusetés zoologiques à l’avenant. Au début des années 1990, quand le fait d’exposer des pygmées dans des vitrines ne fait plus rigoler grand monde, hormis quelques nostalgiques du colonialisme et autres suprémacistes blancs, El Negro de Banyoles fait l’objet de ce que n’hésiterai pas à appeler une vive polémique. Les visiteurs de couleur se sentent quelque peu mal à l’aise face à ce compatriote empaillé comme un vulgaire babouin, témoin d’un passé pas si lointain, voire toujours d’actualité, où l’Homme Blanc régnait en maître absolu sur la Terre, traitant tout ce qui divergeait peu ou prou de son idéal occidental en bête de somme corvéable à merci. Alphonse Arcelin, médecin haïtien et néanmoins socialiste installé à Cambrils, sympathique station balnéaire du Baix Camp, près de Tarragone, entend parler de l’affaire, envoie une missive retentissante au maire de Banyoles, exigeant le retrait immédiat de cette horreur et sa restitution à qui de droit dans les meilleurs délais. Mais le maire en question, qu’une telle infamie n’émeut nullement, n’entend pas se laisser déposséder aussi aisément de l’une des principales sources de revenus de sa commune. Grâce à El Negro, son musée des horreurs ne désemplit pas. Il a augmenté les tarifs et aimerait bien continuer à se remplir les poches aussi longtemps que possible. Après tout, il ne s’agit que d’une enveloppe vide avec de la paille à l’intérieur, une enveloppe en forme de personne réelle, certes, mais qui permet aussi, au-delà de son aspect mortuaire discutable, de prendre la pleine mesure des fondements historiques de notre civilisation. Quelques siècles plus tôt, qu’on le veuille ou non, nos ancêtres découvraient avec stupeur qu’il existait des créatures plus proches d’eux qu’ils ne l’avaient imaginé jusqu’alors, d’apparence bien plus humaine que le singe, capables de communiquer entre elles avec un langage rudimentaire et d’effectuer des actions non dépourvues de logique et d’efficacité. Si la chose était avérée, l’histoire de notre filiation en serait grandement bouleversée. Alphonse Arcelin, fort de ses origines caribéennes, affronte alors une cabale néo-colonialiste qui nie haut et fort l’aspect éthique de la chose au profit de l’histoire et la science, seules valeurs acceptables dans ce cas de figure frappé de péremption. On ne va quand même pas se laisser casser les burnes par une poignée de primates endimanchés ! Si les descendants de ce glorieux personnage souhaitent se recueillir sur sa dépouille (comme neuve ou presque grâce aux bons soins des conservateurs attentifs qui se sont relayés à son chevet, soit dit en passant), ils devront faire le déplacement jusqu’à Banyoles, où l’entrée du musée leur sera bien évidemment gracieusement offerte, contrairement aux frais du voyage qui resteront à leur charge, faut pas non plus pousser mémé dans les orties. Après une lutte acharnée qui causera la ruine du valeureux médecin et son exil à Cuba, Molawa sera rapatrié en grande pompe sur ses terres natales et se verra, en présence des édiles locaux en tenue d’apparat frétillant d’aise sous l’œil embué des caméras internationales, offrir enfin une sépulture digne de sa condition.

C’est donc passablement ensuqué que j’ai ouvert les yeux sur un monde dans lequel je ne tenais pas plus que ça à me retrouver, un univers dont l’étroitesse m’apparaissait chaque jour un peu plus manifeste, odieuse et insoutenable, une cellule dont les murs se rapprochaient chaque jour davantage, réduisant mon espace vital à une bulle inconfortable, un trou à rat dont l’atmosphère confinée devenait chaque jour un peu plus suffocante. J’avais même, l’espace d’un instant, espéré qu’il avait enfin disparu pour laisser la place à une île perdue au fin fond d’une mer imaginaire, une île paradisiaque bénéficiant d’un climat idéal, de paysages enchanteurs sans cesse renouvelés, d’une beauté inépuisable, psychédélique, un coin de paradis entièrement peuplé de créatures de rêve totalement dépourvues d’inhibition, comme quand Fletcher Christian débarque à Tahiti après avoir jeté cette pourriture de capitaine Bligh à la mer, des créatures de rêve fascinées par ma chevelure dorée et l’azur magnétique de mon regard de grand fauve mélancolique. J’avais espéré, même si je n’étais pas exactement le sosie de Marlon Brando (un Marlon tellement investi dans son rôle d’aventurier du Nouveau Monde qu’il repartira du tournage avec Tarita Teriipaia, l’interprète de la belle et sauvage Maïmiti, avec laquelle il aura deux gosses dont une fille, Cheyenne, qui se suicidera vingt-cinq ans plus tard après que son demi-frère Christian aura abattu Dag Drollet, son compagnon violent, de deux balles dans le dos, sachant que Brando père, l’inoubliable interprète, entre autres, du Bal des maudits, Reflets dans un œil d’or, Missouri Breaks, la Comtesse de Hong-Kong et la Poursuite impitoyable, était aussi un bel enfoiré prétentieux et Cheyenne une gamine pourrie-gâtée toxico qui souffrait de troubles mentaux, comme quoi tout ne va pas forcément pour le mieux dans l’existence, même à Mulholland Drive, chose qui n’avait sans doute pas échappé à Lee Tamahori et David Lynch), tirer un trait sinon sur le fardeau de mon ancienne vie, au moins sur le fiasco de la soirée précédente, et repartir sur de meilleures bases vers un avenir souriant au ciel dégagé des noirs nuages du passé.

Et au lieu de ça, de cette perspective gorgée du nectar sucré de l’espérance, je me retrouvais une fois de plus dans mon lit, sans la moindre envie d’en sortir, avec la sensation toujours aussi désagréable d’avoir passé la nuit à mastiquer des copeaux de chêne du Limousin au fond d’une cave sombre et humide (ce qui est le cas de la plupart des caves quand elles sont bonnes).

Au prix d’un effort surhumain, du type de ceux que fournissaient les héros de l’Antiquité pour déplacer des montagnes ou terrasser des dragons, j’ai réussi à me laisser tomber du lit en douceur et sortir de la chambre sans réveiller Zarina. En même temps, réussir à réveiller Zarina quand elle dormait à poings fermés constituait un exploit autrement majeur que tous ceux accomplis par les héros et autres demi-dieux de la Mythologie. Vous avez déjà vu un troupeau d’éléphants traverser la jungle au pas de charge, ou assisté à l’effondrement d’une tour de cent cinquante étages ? Eh bien, si ce même troupeau avait traversé la chambre au pas de charge ou cette même tour s’était effondrée au pied du lit, Zarina n’aurait même pas soulevé une paupière pour s’enquérir de ce qui se passait. Qu’un séisme, un ouragan s’abatte sur elle, que des rafales de vent d’une violence extrême arrachent les fenêtres de la chambre et que des pluies diluviennes la remplissent d’eau jusqu’au plafond ? Peu importe, elle aurait continué à dormir comme si de rien n’était. Vous pensez peut-être qu’une guerre atomique, déclenchée par les milliardaires caractériels qui dirigent ce monde, aurait pu venir à bout de sa résistance. Eh bien je vous fiche mon billet qu’on l’aurait retrouvée endormie au milieu des décombres fumants de l’immeuble, sans une égratignure, aussi fraîche que la rosée du matin. Sauf qu’il n’y aurait plus de rosée du matin, ni de matin tout court, du reste, mais seulement la nuit sans fin de l’Apocalypse, le silence assourdissant de la mort et les vapeurs toxiques qui recouvrent tout d’une épaisse couche de moisissure radioactive. Et quand elle ouvrirait enfin les yeux, ce serait pour se rendre compte que tout avait disparu. Enfin, presque tout, parce que j’aurais survécu moi aussi, bien évidemment. ADAM \& EVE 2, l’éternel retour, mais sans le jardin d’Eden, les arbres chargés de fruits défendus et les serpents maléfiques qui sifflent sur nos têtes pour nous pousser à croquer dedans. Tel le Phénix d’Héliopolis, le Fenghuang de la dynastie Han, le Minka aborigène ou encore le Wakinyan Tanka des Sioux, je renaîtrais de mes cendres, nu comme un asticot blafard et grassouillet, aussi intact et glorieux qu’au premier jour, après quoi je m’empresserais d’aller réveiller ma princesse endormie d’un baiser langoureux sur ses lèvres brûlantes de désirs inassouvis. C’est à nous que reviendrait le privilège de repeupler le monde de petits condensés de nous-mêmes et relancer la folle histoire de l’espèce humaine, indigne s’il en est, aussi médiocre et imparfaite que terriblement prétentieuse, imbue d’elle-même jusqu’à l’overdose existentielle, le trauma crypto-sensoriel de forte magnitude, mais exceptionnellement autorisée, à travers nous, à tenter une nouvelle fois sa chance dans ce coin perdu au fin fond de l’univers qu’on appelle la Terre.

Et d’ailleurs, à propos de repeupler l’univers, je ne sais pas si vous avez remarqué, mais les lendemains de cuite, outre un solide état de déprime arrimé au cuir chevelu, se signalent souvent par un niveau d’excitation sexuelle anormalement élevé. La plupart du temps, il faut recourir à une activité masturbatoire proche de la frénésie pour y remédier. Je suppose que le sexe est un antidépresseur naturel, comme le safran, la camomille, l’orpin et le millepertuis, ce qui est encore une façon très habile (et habituelle de la part d’une entité aussi sournoise et rouée que dame Nature, rompue à tous les artifices pour arnaquer son monde), compte tenu du marasme existentiel qui le nôtre, d’encourager l’espèce à se reproduire malgré tout, ce qui, sur le plan de la pure objectivité scientifique, ferait clairement de l’alcoolisme un rouage essentiel de l’évolution des espèces, la nôtre en l’occurrence, sachant que notre avenir est de toute façon largement compromis. Un peu plus un peu moins, on n’en est pas à une cirrhose près. Dans le cas présent, j’aurais pu me soulager dans l’arrière-train de Zarina sans qu’elle y trouve à redire, d’autant moins, comme je viens de vous l’expliquer, que rien ne pouvait la réveiller quand elle dormait à poings fermés. Entendez par là que le troupeau d’éléphants (pachydermes dont on imagine sans peine la taille colossale du membre et le poids abyssal des burnes) susévoqué aurait pu lui passer dessus à plusieurs reprises, un par un ou tous ensemble, et je parle cette fois au sens de l’expression dans ce qu’elle a de plus bestial et ordurier, sans susciter la moindre réaction de sa part. J’ai néanmoins, dans ma grande sagesse, jugé préférable de n’en rien faire. Quant à l’érection monumentale (tout est relatif, bien sûr, rien à voir avec celles d’un Jonah Falcon ou un Roberto Cabrera) qui me précédait dans tous mes déplacements (fidèle animal de compagnie tenu fermement en laisse sans quoi il n’aurait pas manqué de se ruer sur tout ce qui bouge), je lui ferais rendre gorge dès que mon emploi du temps éminemment chargé me le permettrait. Il était de toute façon hors de question de me présenter au bureau dans cet état, le personnel féminin n’ayant alors pas manqué de me traîner en justice sous les chefs d’inculpation les plus divers, tels que inculpation ou inculpalpation, pourquoi pas, tandis que le personnel masculin, moins procédurier, n’en aurait pas moins conçu à mon égard une rancune tenace pour avoir osé exhiber en toute impudeur des attributs d’une telle magnificence.

Mon périple matinal, loin d’être terminé, m’a ensuite conduit à la cuisine, où j’ai entrepris de mettre en branle la machine à café.

Pendant qu’il coulait, j’avais tout le loisir de me traîner jusqu’à la salle de bain afin de prendre une bonne douche et me laver les dents, autant d’activités d’un ennui mortel auxquelles je me devais néanmoins de sacrifier avant de rejoindre les rangs de la civilisation. Il se trouve que moi, Djeferson Beauvais avec un D et un seul F, commandant de police judiciaire de Saint-Nom-la-Branlette, j’étais soumis à quelques obligations difficilement dispensables, même s’il m’arrivait parfois d’officier dans un état de semi-ébriété coupable, avec des trous dans les chaussettes au niveau du gros orteil et un slip d’une propreté douteuse (le modèle Brando en jersey de coton de chez Dolce \& Gabbana, d’accord, mais tout de même).

Un quart d’heure plus tard, après m’être consciencieusement récuré la couenne, généreusement vidé les bourses et dentifricé le ratichier, j’étais, sinon l’éphèbe éminemment désirable de ma prime jeunesse, au moins en état de faire illusion auprès des gens que je serais amené à rencontrer pendant la journée.

C’est donc plein d’espoir que je me suis rendu à la cuisine pour siroter mon demi-litre de café matinal, dégusté seul, pour lui-même, sans l’adjonction de quelque autre denrée alimentaire que ce soit.

J’ai jeté un œil à mon téléphone et constaté que Bérénice avait essayé de m’appeler plusieurs fois aux premières lueurs de l’aube, alors que j’en écrasais encore sévèrement sous les quelques centimètres carrés de drap que Zarina avait consenti à me laisser, sachant que chaque dixième de millimètre de chacun d’eux avait été conquis de haute lutte, à la force du poignet, au cours de la nuit assez peu réparatrice qui avait été la mienne (Zarina, quoique dépourvue d’origines asiatiques à ma connaissance, avait en effet la fâcheuse manie de s’enrouler dans les draps comme une farce au porc, champignons noirs, vermicelle et pousses de soja dans une fine galette de riz).

Je m’apprêtais à la rappeler, aussi rapidement que possible mais sans excès de vitesse inconsidéré (cet appel ne me disait rien qui vaille et je n’étais par conséquent pas particulièrement pressé de donner suite), quand la sonnerie de mon téléphone (une version électro de la Chevauchée des Walkyries de Richard W, suffisamment immonde et irritante pour forcer le plus réfractaire à la téléphonie mobile et au progrès en général à décrocher dans la seconde) s’est fait entendre.

J’ai pris une grande respiration, décroché et tenté de présenter mes salutations respectueuses à l’intéressée.

Elle ne m’en a pas laissé le temps : Je peux savoir ce que vous avez fait hier soir ?

\textsc{Moi} : Hein ? Quoi ? Qui ça ?

\textsc{Elle} : Titus et toi.

\textsc{Moi}, l’air aussi dégagé que possible : Eh bien… pas grand-chose, à vrai dire… Pourquoi tu me demandes ça ?

\textsc{Elle}, hurlant presque : TITUS N’EST PAS RENTRÉ DE LA NUIT !

\textsc{Moi}, qui avais refusé de croire à cette éventualité et devais me rendre à l’évidence que j’avais sans doute fait preuve d’un peu trop de légèreté dans cette histoire : Ah !… hum hum… oui, en effet, c’est assez embêtant… Alors comme ça, tu dis que… qu’il…

\textsc{Elle} : N’est pas rentré de la nuit, oui, c’est ce que je viens de dire !

\textsc{Moi} : C’est bizarre, quand même.

\textsc{Bérénice} : Oui, je trouve aussi. Est-ce que, par le plus grand des hasards, tu aurais des informations à ce sujet ?

\textsc{Moi} : Eh bien, écoute, euh, là j’avoue que tu me prends un peu de court… Il est tôt, je viens de sortir du lit, je ne suis pas encore très bien réveillé… Alors… oui… effectivement… maintenant que tu le dis… je crois me souvenir qu’il nous a quittés à une heure assez avancée de la nuit… aux alentours de… enfin… je ne me rappelle plus exactement… mais… assez tard, en effet… ou tôt, suivant la façon dont on… enfin, tu vois ce que je veux dire…

\textsc{Elle} : Et ?

\textsc{Moi} : Eh bien je me suis dit qu’il rentrait chez lui, voilà tout.

\textsc{Elle} : Il a dit qu’il rentrait ?

Votre serviteur, au comble de l’inconfort : Euh… non, il n’a rien dit de particulier à ce sujet… mais je ne vois pas ce que je pouvais penser d’autre à ce moment-là.

\textsc{Elle} : Eh bien, il se trouve qu’il n’est pas rentré.

\textsc{Moi} : T’es sûre ?

\textsc{Elle} : Certaine, oui.

\textsc{Moi} : Peut-être qu’il est rentré et… et qu’il est ressorti…

\textsc{Elle} : J’ai le sommeil assez léger, je crois que je m’en serais rendu compte. Vous faisiez quoi, quand il est parti ?

\textsc{Moi} : Mon Dieu… pas grand-chose, comme je te l’ai dit… On était en opération spéciale… strictement confidentielle… je ne peux malheureusement pas t’en dire davantage à ce sujet, question de sécurité, la tienne et celle de tes proches… mais il se trouve que… comment dire… eh bien il se trouve que les choses ne se sont pas exactement déroulées comme prévu… Les impondérables, tu sais ce que c’est.

\textsc{Elle} : Non, pas trop. JE CROIS SURTOUT QUE T’ES EN TRAIN DE BIEN TE FOUTRE DE MA GUEULE !!!!!!!!!

\textsc{Moi}, écarquillant des yeux ronds comme des culs de babouins : COMMM-MMMENT ???????????!!!!!! Mais enfin comment oses-tu imaginer une chose pareille ! Non, je suis offusqué, là, limite vexé. L’affaire n’avançait pas, s’étirait inutilement, en un mot comme en cent on se trainait comme des merdes dans le purin, des limaces dans la bave, des…

\textsc{Elle} : Oui, bon, ça va, épargne-moi tes comparaisons douteuses !

\textsc{Moi} : Bref, on s’est dit que ça ne servait à rien de continuer à patauger dans la semoule comme des mamelouks en détresse. Quand Titus a décidé de partir, j’en ai conclu tout naturellement qu’il rentrait chez lui.

\textsc{Elle} : Ben voyons. Je l’ai appelé plusieurs fois, j’ai laissé des messages, et il ne répond toujours pas ! Tu peux essayer de faire quelque chose, s’il te plaît ?

\textsc{Moi} : Mais bien sûr, Bérénice, tu sais aussi bien que moi que tu peux compter sur… enfin sur moi, quoi… Je vais faire tout ce qui est en mon pouvoir (autant dire pas grand-chose, note de la Rédaction) pour le retrouver. Cela dit, je ne suis pas sûr qu’il réponde davantage si c’est moi qui appelle…

\textsc{Elle} : Mais tu peux localiser son téléphone, non ?

\textsc{Moi} : Oui, en principe.

\textsc{Elle} : Comment ça, en principe ?

\textsc{Moi} : Eh bien… il y a des fois où on peut, et d’autres fois où on ne peut pas…

\textsc{Elle} : Ah bon ? Vous n’avez pas des spécialistes pour faire ce genre de chose ?

\textsc{Moi} : Si, mais peut-être qu’il n’a plus de batterie.

\textsc{Elle} : On ne peut pas le localiser s’il n’a plus de batterie ?

\textsc{Moi} : C’est plus difficile. Peut-être aussi que son téléphone a été détruit pour une raison ou pour une autre…

\textsc{Elle} : Et alors ? Il pourrait… je ne sais pas, moi, m’appeler d’une cabine ! Il sait pertinemment que je me fais un sang d’encre quand je suis sans nouvelles ! Tu n’as vraiment aucune idée de l’endroit où il peut être ?

\textsc{Moi}, tel Judas Iscariote (le Judas «~sikariot~» de la Peshitta, traduction syriaque de la Bible chère aux Maronites) claquant la bise au Roi des Juifs dans les jardins de Gethsémani : Ma foi… comme ça… non, je ne vois pas… Je vais me renseigner pour savoir s’il ne lui est pas arrivé quelque chose… Je ne pense pas, mais on ne sait jamais… un accident… que sais-je…

\textsc{Elle} : Je ne sais pas pourquoi, mais j’ai la désagréable impression que tu me caches quelque chose.

\textsc{Moi}, pour une poignée de deniers : Hein ??? Quoi ??? Comment ???

\textsc{Elle} : Tu as très bien entendu.

Et pour quelques deniers de plus : Mais jamais de la vie, voyons ! Qu’est-ce que tu vas imaginer !

\textsc{Elle} : Je connais les hommes.

\textsc{Moi}, plus roublard que Tuco : Je ne vois pas de quoi tu veux parler.

\textsc{Elle} : Ils sont prêts à se couvrir les uns les autres jusqu’à la mort, au nom de je ne sais quel sens débile de l’amitié virile et l’esprit de corps !

\textsc{Moi} : Ce n’est pas pire que les femmes qui passent leur temps à se tirer dans les pattes !

\textsc{Elle} : Tu me jures que tu ne sais pas où est Titus ?

\textsc{Moi}, plus répugnant qu’une punaise de lit sortant furtivement de sa tanière pour sucer le sang de sa victime endormie : Tu penses bien que je te le dirais si je savais quelque chose ! Titus est mon meilleur ami, c’est vrai, mais toi et les enfants comptez aussi beaucoup pour moi.

\textsc{Elle} : N’empêche que tu n’as pas répondu à la question.

\textsc{Moi} : Quelle question ?

\textsc{Elle} : Est-ce que tu me jures que tu ne sais pas où est Titus ?

\textsc{Moi}, après un raclement de gorge suffisamment prononcé pour laisser planer le vautour déplumé du doute au-dessus de la charogne faisandée de ma duplicité grouillante des asticots replets de la bêtise crasse et la lâcheté triomphante (oui, bon, ça va, personne n’est parfait) : Bien sûr, que je te le jure, mon amour !

\textsc{Bérénice} : Quoi ???!!!!

\textsc{Moi}, confus : Excuse-moi, ça m’a échappé dans le feu de l’action.

\textsc{Elle} : Tu m’aimes ?

\textsc{Moi} : Bien sûr, que je t’aime, comme la femme de mon meilleur ami, comme une sœur, une personne chère, hors de prix, à laquelle il ne me viendrait jamais à l’idée de mentir.

\textsc{Elle} : Ah bon, tu m’as fait peur !

\textsc{Moi} : Tu veux dire que si je te disais que je t’aime, non pas seulement comme la femme de mon meilleur ami, mais comme une femme qui fait naître en moi les désirs les plus… comment dire… intenses, au sens érotique du terme, tu en concevrais une certaine… rancœur, amertume ?

\textsc{Elle} : Disons que je serais un peu perturbée, oui.

\textsc{Moi} : Eh bien rassure-toi, il n’en est rien.

\textsc{Elle} : Encore heureux !

\textsc{Moi}, évoluant avec une grâce discutable sur le fil d’un rasoir qui commençait à m’entamer sérieusement la voûte plantaire : Écoute, Bérénice, ma douce et tendre Bérénice, Bébé mon bébé, je ne sais pas au juste ce que tu vas imaginer, mais dis-toi bien que…

\textsc{Elle}, d’une voix dont la surface se voulait d’un calme olympien mais n’en laissait pas moins transparaître une agitation comparable à celle d’un groupe d’orques en train de mettre en pièces un requin-baleine dans les eaux troubles du golfe du Mexique : Je n’imagine rien du tout. Je note juste que mon mari a découché, ce qui ne lui était encore jamais arrivé sans un mot du médecin, qu’il ne répond pas au téléphone quand je l’appelle, et que son meilleur ami, un type plutôt bizarre avec lequel il a passé la nuit, prétend ne rien savoir de ce qui lui est arrivé. Quant aux enfants, qui ont l’habitude de déjeuner avec leur père avant de partir à l’école, ils me posent des questions auxquelles je suis incapable de répondre.

\textsc{Moi} : Dis-leur qu’il est parti au boulot un peu plus tôt que prévu.

\textsc{Elle} : Certainement pas ! Je ne leur ai jamais menti, ce n’est pas maintenant que je vais commencer.

\textsc{Moi} : C’est tout à ton honneur.

\textsc{Elle} : C’est jamais bon de mentir aux enfants.

\textsc{Moi} : C’est évident. Ils comptent sur nous, nous font une totale confiance. Si on commence à leur raconter n’importe quoi, ils perdent leurs repères et partent en vrille.

\textsc{Elle} : Quand la confiance est altérée, tout s’écroule.

\textsc{Moi} : Ouais. Cette connerie de Père Noël, par exemple, ce vieux barbu en pyjama rouge, qui se la joue papy sympa mais sent le pédophile à plein nez, le «~bad Santa~» obsédé du cul, se trimbalant en permanence une érection digne d’un cheval de trait, une gaule à casser des noix, censé se balader en traineau dans le ciel et passer par les cheminées avec sa hotte pleine de jouets… Ben voyons ! Vu son tour de taille, ce bouffon n’arriverait même pas à passer dans le tunnel sous la Manche sans toucher les bords ! C’est l’exemple-type des conneries qu’on raconte aux gosses pendant des années, jusqu’au jour où ils apprennent brutalement qu’on leur a menti soi-disant pour la bonne cause, essayer de mettre un peu de rêve dans la bouillie insipide de la vie quotidienne, fêter dignement le jour béni où Jésus a ouvert les yeux dans une étable de Bethléem, le plus simplement du monde, sans se soucier du fait que l’étable en question, remplie de paille et éclairée à la bougie, pouvait prendre feu à tout moment.

\textsc{Elle}, très remontée, au point d’utiliser une formulation qui n’était pas habituellement la sienne (je n’irais pas jusqu’à dire que j’en étais choqué, mais tout de même, je découvrais une facette de sa personnalité dont j’avais à peine soupçonné l’existence jusqu’ici) : Rien à secouer de l’âne, du bœuf et des Rois mages !

\textsc{Moi} : Exact, qu’ils aillent tous se faire foutre ! À commencer par ce prétendu Messie qui aurait mieux fait de nous prévenir gentiment qu’on allait passer le restant de nos jours à mentir, voler, violer et assassiner son prochain ! Lui aussi nous a bien roulés dans la farine comme des boulettes de viande avant de passer à la casserole ! En même temps, comment faire confiance à un type qui prétend être le fils de Dieu, rien que ça, débarque pour sauver le monde et n’est même pas fichu de se sauver lui-même ? Mais excuse-moi, je crois que je m’égare un peu.

\textsc{Elle} : Non, t’as raison. Je ne comprends toujours pas pourquoi on continue à fêter la naissance de ce type deux mille ans après. Regarde, moi, par exemple, on n’a pas fait tout ce ramdam le jour où je suis venue au monde !

\textsc{Moi} : Pareil pour moi, à commencer par mes parents qui s’en foutaient comme de l’an quarante !

\textsc{Elle} : Ah bon ?

\textsc{Moi} : Bien sûr. Je te l’ai déjà raconté, non ?

\textsc{Elle} : Je crois pas, non.

\textsc{Moi}, pas avare de confidences aux premières heures du jour, quand les derniers fêtards rentrent chez eux ivres morts, que les travailleuses du sexe sucent leurs derniers clients, que les agents de propreté urbaine collectent leurs dernières ordures sur les trottoirs de la ville, que les rats d’égout regagnent furtivement leur tanière la dent jaune et le ventre plein, que les collégiens matent fébrilement une dernière vidéo de boule sur les réseaux sociaux avant d’aller en classe, que le soleil se lève paresseusement entre deux nuages endormis, que les dealers goûtent quelques heures de repos bien mérité avant de retourner au taf, que les politiciens véreux, hommes d’affaires corrompus, chanteurs de charme affiliés à la mafia et autres marchands de mort spécialisés dans la vente d’armes aux pays sous embargo sillonnent le ciel dans leurs jets privés pour aller planquer leur pognon dans les paradis fiscaux, que les milliardaires paranoïaques et mégalos, accros aux drogues dures, aux partouzes et à la pêche au gros, lancent leurs dernières OPA pour devenir enfin les maîtres du monde, se prendre des cuites au Clos D’Ambonnay, se rouler dans les œufs d’esturgeon et se taper des putes mineures sur des yachts de 50 mètres de long au large des côtes de Tanzanie : Ma mère ne voulait pas d’enfant, mais elle était super canon et mon père ne pensait qu’à la niquer par tous les orifices. La plupart du temps, il se contentait de l’enculer vite fait sur la table de la cuisine ou dans le local à poubelles, chose qui, je ne m’explique toujours pas pourquoi, avait le don de le mettre dans un état de transe sexuelle irrépressible, et comme ma mère oubliait régulièrement de prendre la pilule et que mon père n’était pas un adepte du retrait d’urgence au moment fatidique, ce qui ne devait pas arriver arriva néanmoins. Je naquis donc, beau bébé replet promis au plus brillant avenir, à ceci près que mon père voulait une fille et que le service trois pièces de toute beauté que j’affichais au sortir du ventre maternel ne laissait planer aucune équivoque sur la nature de mon sexe. Sa tronche s’est décomposée quand il a posé les yeux dessus, il a claqué les talons, comme tout bon militaire qui se respecte, s’est fendu d’un demi-tour-droite irréprochable techniquement, a quitté la chambre sans dire un mot, droit comme un I, puis est allé se saouler la gueule avec ses potes de régiment pendant une durée indéterminée qui a duré exactement trois jours et trois nuits. Ensuite, il s’est repointé à la maison comme si de rien n’était, et ne s’est pas gêné pour m’en faire baver des ronds de chapeaux jusqu’à ce que je sois enfin en âge de me barrer sans demander mon reste.

\textsc{Elle} : Ton père était dans l’armée ?

\textsc{Moi} : Oui, un de ces tueurs professionnels qui sillonnent la planète pour faire un peu de ménage. Sur ce, je te souhaite une bonne journée.

\textsc{Elle} : Tu te fous de ma gueule ?

\textsc{Moi} : Non… excuse-moi… ce n’est pas ce que je voulais dire… enfin si… passe une bonne journée dans le sens où ne t’en fais pas, je vais retrouver Titus et tout va rentrer dans l’ordre.

J’ai prévenu le bureau que je risquais d’être en retard, mais comme j’étais le chef je pouvais faire à peu près ce que je voulais et me permettre d’être en retard quand j’avais des affaires plus importantes à traiter que ce qui constituait l’ordinaire des activités d’un commissariat de quartier, à savoir faire des rondes dans les quartiers chauds, contrôler des gens de couleur qui n’ont rien demandé à personne, faire souffler des automobilistes dans des éthylotests, verbaliser des pervers à la sortie des écoles, et enregistrer des plaintes de vieilles dames qui se sont fait braquer leurs économies par des escrocs souriants et bien habillés (il avait l’air si gentil).

Je me suis habillé en vitesse, et j’ai ouvert délicatement ma porte pour jeter un œil sur le palier. Pas de mère Ouvrard ni de Marc-Antoine Jacquinot en vue, tapis dans un angle mort pour fondre sur moi comme l’aigle sur sa proie. Juste Korax avachi sur le paillasson de sa porte d’entrée, plus sournois et dédaigneux que jamais, bien entendu, mais apparemment peu disposé à tenter une de ces attaques surprises dont il avait le secret. C’est donc l’esprit relativement dégagé que j’ai pu prendre l’ascenseur, dévaler (à vitesse réduite, l’engin n’étant pas de la toute première jeunesse, ni même de la seconde) les six étages qui me séparaient de la terre ferme, sortir de l’immeuble et respirer enfin les fraîches senteurs de l’aube si nécessaires à mon équilibre mental, même si raisonnablement polluées par le monoxyde de carbone, le dioxyde d’azote et les particules fines chargées d’hydrocarbures polycycliques hautement cancérigènes contenus dans les gaz d’échappement.

Contrairement à son habitude, le 1,6 16v de ma Kangoo, boosté à près de deux cent cinquante chevaux de course par les bons soins de Nathan, le frère de Greg, a consenti à s’ébrouer après seulement deux tentatives infructueuses, ce qui n’était pas loin de constituer un record personnel. Le modeste quatre cylindres produisait maintenant un feulement rauque de grand fauve en rut, subtilement mis en valeur par une ligne d’échappement optimisée qui m’avait coûté la peau des fesses. La prochaine mission de Nathan, si toutefois il l’acceptait, consisterait à doter le paisible utilitaire de tous les aménagements nécessaires pour se transformer à loisir en voiture-bélier, aéronef ou sous-marin de poche. Enfin, un exosquelette à cristaux liquides permettrait de rendre le véhicule invisible à la demande, par simple action sur un commutateur dédié. On n’en était pas encore là, mais je disposais tout de même d’un certain nombre de gadgets intéressants, tels que blindage intégral, quatre roues motrices avec pneus increvables, autoradio à écran tactile Bluetooth, système audio haute performance avec ampli de 900 watts, mais aussi et je dirai même surtout boîte à gants transformée en humidor pour une parfaite conservation de mes cigares préférés.

Et justement, à propos de cigares, j’ai ouvert la boîte à gants, me suis glissé un Fuente Short Story dans le bec, et l’ai allumé à la flamme de mon Guevara Vintage double jet anti-tempête (utile en mer par gros temps quand on souhaite s’en griller un sous des trombes d’eau, ce qui n’est pas chose facile, ou encore quand on se retrouve pris au piège par une tornade qui avance inexorablement en ravageant tout sur son passage et qu’on s’apprête à allumer ce qui sera sans doute son dernier cigare).

Un peu tôt pour commencer à fumer, je vous l’accorde, mais les mauvaises habitudes sont les plus difficiles à perdre et la journée s’annonçait aussi chargée que l’haleine d’une hyène qui vient d’arracher ses derniers lambeaux de chair à une carcasse de gnou faisandée depuis des lustres sous le soleil impitoyable de Namibie. Le Kalahari, par exemple, en dépit de l’aridité redoutable qu’on lui connaît, n’en offre pas moins des paysages d’une beauté à couper le souffle, et ce ne sont certainement pas les amateurs de safari en Land Rover et de bivouac sous les étoiles qui me contrediront. Car tous savent bien qu’il n’est pas donné à tout le monde de connaître ces moments inoubliables où le lion rugit à quelques mètres de vous, le vautour vous tourne autour dans le ciel chauffé à blanc, le guépard vous regarde droit dans les yeux et le suricate vous mange dans le creux de la main. Tout comme il n’est pas donné à tout le monde~-- et heureusement, du reste, car ils n’hésiteront pas à vous tirer une flèche empoisonnée dans le cul si vous vous avisez de leur manquer de respect~-- de croiser la route des fiers Bochimans qui arpentent les pistes du désert depuis des dizaines de milliers d’années, parlent à la Lune comme à une vieille amie et pratiquent aujourd’hui encore les étranges rituels hérités de leurs plus lointains ancêtres. Bien sûr, ils ne fument pas la pipe ou le cigare, comme vous et moi, et n’imaginent même pas qu’on puisse résider à plein temps dans un manoir des Highlands, mettre des glaçons dans un verre de Bruichladdich de 30 ans d’âge, porter des costumes sur mesure, piloter des avions de chasse et tirer des missiles balistiques sur des sites stratégiques d’Europe de l’Est ou du Moyen-Orient. Non, tout ça leur passe largement au-dessus de la tête. Mais ils n’en ont pas moins, dans le dénuement le plus total, vêtus de simples peaux de bêtes et exposés sans cesse à la cuisante morsure du feu solaire, réussi à survivre et s’intégrer dans l’environnement hostile qui a toujours été le leur. Jamais, certes, vous ne les verrez défiler en kilt dans les rues de Fort William, s’époumoner dans des cornemuses ni admirer les moutons de Rosa Bonheur dans les musées de Londres, Hambourg ou Washington, mais soyez certain qu’ils sauront vous défendre au péril de leur vie si le babouin vous menace, le phacochère vous charge ou l’éléphant furibard tente de vous écrabouiller comme la grosse merde de capitaliste nuisible et envahissant que vous êtes (le pachyderme en question étant tout particulièrement motivé du fait que vous n’avez cessé de le persécuter pour lui voler ses défenses et lui couper les pattes pour en faire des cendriers, ce qui montre bien l’extrême perversité et la frivolité inqualifiable de vos agissements).

Greg, réveillé aux aurores, m’attendait sur le pas de sa porte, rasé de près, l’œil vif et le jarret fringant, impatient de prendre part à ce qui s’annonçait sinon comme l’aventure la plus palpitante de sa vie, au moins de ces dix dernières années. C’était, à priori, le genre de chose qu’on prend plaisir à raconter à ses petits-enfants au coin du feu quand on est vieux, malade et sur le point de crever, à condition bien sûr d’avoir des petits-enfants et une cheminée à portée de main, et que le fait d’être sur le point de crever nous permette encore de prononcer quelques paroles vaguement intelligibles sans foutre la trouilles aux gosses.

À vrai dire, n’ayant ni enfants et encore moins petits-enfants, je crois surtout qu’il était impatient de prendre la fuite, Lou (ou Loulou suivant les cas, rarement Louloulou ou Loulouloulou, de son vrai nom Louise De La Croix, créature pour tout dire assez fascinante issue d’une vieille famille d’aristos désargentés, dont le père était un escroc notoire ayant fait plusieurs fois le tour de la planète pour échapper à ses créanciers, tant et si bien que plus personne~-- et sans doute pas même lui~-- ne savait précisément où il était ni même s’il était encore en vie, cette dernière option paraissant hautement improbable vu le nombre de gens prêts à sacrifier tout ou partie de leur fortune pour avoir sa peau), sa nouvelle copine nymphomane, ne lui laissant pas une seconde de répit. C’était une fille dont il avait fait la connaissance du temps où il bossait quinze heures par jour comme analyste financier chez Reckless \& Knot, avec laquelle il avait… comment dire… partagé quelques affinités électives dans les endroits et positions les plus divers et incongrus, avant de lui faire part de sa décision longuement mûrie en ses âme et conscience de mettre un terme à toute activité autre que strictement professionnelle les concernant, considérant que cette incursion dans le domaine du privé, en dépit de son caractère physiquement satisfaisant, là n’était pas la question, s’avérait néanmoins incompatible avec la bonne marche de l’entreprise en général et de sa santé mentale en particulier. Mais ne voilà-t-il pas, au moment où il s’y attendait le moins (il ne faut jamais dire jamais, c’est bien connu, et surtout se tenir prêt en permanence à toute éventualité), que l’imprévisible Lou venait de refaire surface et lui mettre à nouveau le grappin dessus (assez facilement, il faut bien le dire), plus chaude qu’un brasier dévastant des centaines d’hectares de garrigue dans l’Hérault. Fragile psychologiquement, autant que sexuellement fortement attiré par cet incendie que nulle compagnie de soldats du feu n’était jamais parvenue à maîtriser, Greg avait commis l’erreur de remettre un doigt dedans et s’était aussitôt retrouvé pris au piège (je pense alors au fameux fingertrap de la famille Addams, cadeau de dixième anniversaire de Fester, le frère de Gomez, qui a dû apprendre à manger avec les pieds parce qu’il est resté coincé dedans pendant deux ans).

Et pour vous dire toute la vérité, eh bien sachez que le Greg que j’évoquais précédemment, rasé de près, à l’œil vif et au jarret fringant, tel un Bilbon Sacquet prêt à prendre part aux aventures les plus rocambolesques avec une bande de Nains sortis de nulle part, des magiciens et des Elfes sylvains pas toujours très bien disposés, affronter des hordes d’Orques hideux et terrasser des dragons gardiens de trésors sous des montagnes solitaires, ce Greg-là n’était hélas rien d’autre qu’une vue de l’esprit, un mirage, un fol espoir, un fantasme à mille lieues de la réalité affligeante qui s’offrait à ma vue dépitée.

Quand je l’ai vu arriver en traînant les pieds, tel un petit vieux accablé par le poids des ans et pressé d’en finir avec une existence qui lui a procuré en gros quatre-vingt-dix et quelques pour cent d’emmerdes pour dix tout petits pour cent de moments de vague satisfaction aussi rares que fugaces (autrement dit pas du tout assez pour faire pencher favorablement la balance), monter dans la bagnole avec l’entrain d’une vache qu’on fait entrer à coups de pied dans le cul dans une bétaillère pour la conduire à l’abattoir, me tendre une main tellement molle que j’ai eu l’impression de serrer la pince à une flaque de vomi, attacher sa ceinture avec résignation, poser les mains bien à plat sur ses genoux et se mettre à regarder fixement devant lui sans desserrer les dents, j’ai compris que je n’étais pas au bout de mes peines.

J’ai cherché son regard pendant quelques instants, sans la moindre réaction de sa part, et lui ai posé la première question qui m’est venue à l’esprit, assez basique il est vrai, et d’autant plus superflue que j’en connaissais déjà la réponse : Ça va ?

\textsc{Greg}, sans tourner la tête : Non, pas vraiment.

\textsc{Moi} : Mal dormi, peut-être ?

\textsc{Lui} : Pas fermé l’œil de la nuit.

\textsc{Moi} : Lou ?

\textsc{Lui} : Lou.

\textsc{Moi} : Elle t’attendait ?

\textsc{Lui} : Derrière la porte, dans sa tenue la plus suggestive.

\textsc{Moi}, la tête noyée dans un nuage de fumée : Lumière tamisée, senteurs orientales, lingerie fine, résille et balconnet, difficile de résister.

\textsc{Lui} : Pas exactement, mais elle a des moyens de persuasion très efficaces. Même un eunuque n’y résisterait pas.

\textsc{Moi} : Un moine bénédictin, peut-être ?

\textsc{Lui} : Pas davantage.

\textsc{Moi} : Tu veux dire que saint Benoît de Nursie lui-même n’aurait pas résisté longtemps à ses avances ?

\textsc{Lui} : Le pauvre vieux n’aurait pas tenu trente secondes.

\textsc{Moi} : Que si Augustin d’Hippone l’avait croisée dans les rues de Carthage ou les travées de la basilique Saint-Pierre de Rome, ses Confessions compteraient bon nombre de pages en plus ?

\textsc{Lui} : Bon nombre, et je ne suis pas certain qu’il aurait osé les écrire toutes.

\textsc{Moi} : J’en conclus qu’elle ferait bander un mort.

\textsc{Lui} : Un cimetière tout entier !

\textsc{Moi} : Avec elle, les cadavres sortent de terre avec une trique d’enfer ! Lève-toi et gicle !

\textsc{Lui} : Tout à fait. Tel Lazare recroquevillé dans le fond de sa tombe, ma bite croupissait dans le fond de mon slip, aussi inerte et engluée dans sa propre bave qu’une limace à l’agonie.

\textsc{Moi} : Quelle vision lugubre et déprimante. Ce membre éminent du genre humain à jamais perdu pour la France, le monde, l’univers tout entier ! La théorie de l’évolution bouleversée dans ses fondements mêmes, ébranlée dans ses fondations, durement secouée.

\textsc{Lui} : Une tragédie, oui, on peut dire ça.

\textsc{Moi} : Digne des riches heures de l’Antiquité, les champs de bataille dévastés, les ruines jonchées d’ossements, les rats courant ici et là, à la recherche d’un lambeau de chair à grignoter.

\textsc{Lui} : Oui, abominable. Je ne sais pas comment elle s’y est prise, quel stratagème elle a utilisé, mais toujours est-il qu’elle a réussi à la ressusciter.

\textsc{Moi} : Un vrai miracle !

\textsc{Lui} : Et ça a duré des heures et des heures, tel un torrent de sexe qui a déferlé sur moi sans que je puisse rien faire pour échapper à son emprise ! Aucun homme ne devrait avoir à endurer pareille torture.

\textsc{Moi} : Toutes mes condoléances, vieux. Malheureusement, tu sais comme moi qu’on a du pain sur la planche. Titus a disparu, Bérénice m’est tombée dessus aux premières lueurs du jour, alors que j’ai moi-même passé une très mauvaise nuit et que si ça ne tenait qu’à moi je retournerais me coucher jusqu’à après-demain soir, et dans un moment d’aberration, de pure folie, j’ai juré de tout mettre en œuvre pour le ramener vivant à la maison.

\textsc{Lui} : Tu ne crois pas que tu en fais un peu trop ?

\textsc{Moi}, tirant nerveusement sur mon cigare : Je crois pas, non.

\textsc{Lui} : Et puis c’est nouveau, ça ?

\textsc{Moi} : Quoi ?

\textsc{Lui} : Tu fumes le matin, maintenant ?

\textsc{Moi} : C’est dire dans quel état de nerfs je suis ! Tendu comme un string, mon vieux, prêt à exploser à la moindre pression sur la ficelle. Je comptais sur toi pour me remonter le moral, j’ai bien peur de m’être trompé.

\textsc{Lui} : Désolé, mais j’ai les couilles en purée, comme si un rouleau-compresseur avait passé la nuit à passer et repasser dessus. Je ne sais même pas comment je fais pour tenir encore debout. J’ai eu droit à tout : le flipper, le papillon, la balançoire, le soixante-neuf, le cavalier pendu, la liane ensorcelée, le triangle lumineux, l’étoile mystérieuse, le papillon…

\textsc{Moi} : Tu l’as déjà dit.

\textsc{Lui} : Ah bon ?

\textsc{Moi} : Oui.

\textsc{Lui} : J’ai eu droit à plusieurs espèces de papillons, en fait. Sans oublier les ciseaux, le cadenas, la brouette enchantée, la bête à deux dos, l’aurore boréale, le cheval d’Hector, la cravate de notaire, le bateau ivre, j’en passe et des meilleurs. Tout, je te dis !

\textsc{Moi} : La mouche à merde, aussi ?

\textsc{Lui} : Non, pas la mouche à merde.

\textsc{Moi} : Un vrai cauchemar, en tout cas.

\textsc{Lui} : Tu l’as dit ! Il faut une condition physique de sportif de haut niveau pour tenir le coup.

\textsc{Moi} : Et c’est loin d’être ton cas.

\textsc{Lui} : J’ai passé l’âge.

\textsc{Moi} : Pour info, c’est quoi le triangle lumineux ?

\textsc{Lui} : Sensiblement la même chose que le missionnaire, sauf que la femme est un peu plus active.

\textsc{Moi} : Tu veux mon avis ?

\textsc{Lui}, dans un soupir : Non.

\textsc{Moi} : Tant pis, je te le donne quand même : tu ferais bien de te débarrasser de cette pute avant qu’il soit trop tard. À ce rythme-là, elle aura ta peau dans pas longtemps.

\textsc{Lui} : Je sais.

\textsc{Moi} : C’est du suicide, reconnais-le.

\textsc{Lui} : Je le reconnais, mais c’est pas une raison pour la traiter de pute. Elle est malade, tu comprends ? Malade !

\textsc{Moi} : Je suis sûr qu’elle a tué plein de mecs en les obligeant à baiser comme des dingues jusqu’au bout de la nuit.

\textsc{Lui} : Elle te fais des trucs que t’as même pas idée ! C’est une sorte de génie du sexe, dont la créativité semble inépuisable. Je pense qu’elle a signé un pacte avec le diable !

\textsc{Moi} : C’est une tueuse en série, mon pote, un vampire qui vide les hommes de leur substance jusqu’à la dernière goutte. Tu ne t’en rends pas compte, mais tu viens de prendre un aller simple pour l’enfer. T’en as marre de la vie, ou quoi ?

\textsc{Lui} : Je me pose parfois la question.

\textsc{Moi} : Fous-la dehors et reprends le cours normal de ton existence, ça vaudra mieux. Sauve ce qui peut encore être sauvé. Je suis là, moi. Si tu as besoin de quelque chose, n’hésite pas à me demander.

\textsc{Lui} : Je sais, vieux, je sais.

\textsc{Moi} : Cette femme est un démon, un succube de la pire espèce, une sorcière qui erre nue dans la forêt et s’accouple avec les bêtes sauvages.

\textsc{Lui} : Faut peut-être pas exagérer non plus. C’est une grosse chaudasse, il n’y a aucun doute là-dessus, mais ça reste un être humain fait de chair et de sang, avec un cœur qui bat sous sa poitrine avantageuse.

\textsc{Moi} : Et une chatte qui miaule en permanence, toujours affamée, jamais repue, les crocs affûtés, prêts à se planter dans le moindre morceau de viande qui passe à leur portée. Je me fais du souci pour toi, voilà tout. Je n’ai pas envie qu’on retrouve ton cadavre méconnaissable au fond de l’océan ou dans le lit d’une rivière à sec, à moitié dévoré par les sangsues.

\textsc{Lui} : Honnêtement, je ne vois pas très bien ce que mon cadavre ferait dans le lit d’une rivière à sec.

\textsc{Moi}, appuyant délicatement (je rappelle qu’avec le troupeau d’étalons survitaminés qui piaffaient sous le capot, tout excès d’enthousiasme se traduisait aussitôt par la sensation de décoller aux commandes d’un avion de chasse, ce qui, à moins d’être titulaire d’un brevet de pilote en bonne et due forme, pouvait avoir des conséquences désastreuses pour l’environnement et la sécurité des passagers) sur le champignon après avoir enclenché la première et desserré le frein à main : Moi non plus, mais on ne sait jamais. Je tenais à te mettre en garde, te rappeler une dernière fois que si l’activité physique est bonne pour la santé, largement recommandée par la sphère médicale dans sa plus grande majorité, il n’en reste pas moins que certaines choses sont réservées à des professionnels aguerris et ne devraient en aucun cas être reproduites sans assistance par des amateurs présomptueux. C’est tout ce que j’avais à dire. Maintenant, en voiture Simone et en route pour de nouvelles aventures !

\textsc{Lui}, manifestement vexé : Merci pour ta sollicitude. Mais sache, pour ta gouverne, que je ne manque jamais de m’échauffer avant de passer à l’acte.

\textsc{Moi}, évitant de justesse une joggeuse moulée dans une tenue rose fluo de nature à distraire dangereusement l’attention de tout conducteur un tant soit peu sensible aux attraits de la plastique féminine (et amateur de belles choses, vins fins, rivières de diamants, tonnes d’or, voitures de sport, croisières sur le Nil, couchers de soleil sur le Bosphore, plaisirs simples d’une bonne bouillabaisse partagée entre amis, gambas, naturisme, poules de luxe, bel canto et œuvres d’art en général) : C’est tout à ton honneur. Il n’empêche, et je me permets de te le dire en toute gentillesse, que tu arrives tous les matins sur ton lieu de travail dans un état physique déplorable. J’ajoute que ta santé mentale semble également affectée par les excès de ta vie sexuelle débridée, raison pour laquelle je te demande de lever le pied dans les plus brefs délais. Je crains, si tu ne suis pas mon conseil, que l’érotomanie te guette.

\textsc{Lui} : Et moi, je me demande comment tu peux savoir dans quel état je me trouve en arrivant sur mon lieu de travail puisque tu n’y es pas. Ce n’est pas parce que je suis un peu fatigué ce matin que c’est tous les jours pareil.

\textsc{Moi} : Désolé de te le dire, mais tu n’es plus le même depuis que tu es retombé entre les griffes de cette nymphe des temps modernes, ce suppôt de Lilith.

\textsc{Lui}, perfide : Je ne suis pas certain que tu sois en mesure de me donner des leçons sur le sujet.

\textsc{Moi}, écrasant la pédale de frein pour éviter de griller un feu rouge : Ah oui ? Je peux savoir ce que tu entends par là ?

\textsc{Lui}, avec un petit sourire méchant esquissé au coin des lèvres : J’ai bien vu ton petit manège, hier soir.

\textsc{Moi} : Pardon ?

\textsc{Lui} : Ton petit manège avec la Gardienne de la Nuit. Tu m’excuseras, mais dans le genre succube, elle se pose un peu là !

\textsc{Moi}, faisant ronfler le moteur pour démarrer sur les chapeaux de roues sitôt le feu passé au vert : Sois à tout jamais maudit jusqu’à la trente-sixième génération ! Que les vers te picorent, les asticots te butinent, la moisissure recouvre entièrement ta peau vérolée et les légions de l’enfer viennent danser la salsa sur le tas de terre retournée de ta sépulture anonyme ! Comment toi, mon ami, fidèle parmi les fidèles, oses-tu proférer de pareilles inepties !!!!!!! Jamais, tu m’entends, jamais je n’ai considéré Atiena comme autre chose qu’une créature magique sortie tout droit d’un recueil de contes pour enfants !

\textsc{Lui}, ricanant presque : Oui, enfin, ta façon de la regarder s’apparentait davantage à celle du grand méchant loup en train de reluquer le Petit Chaperon rouge et sa petite motte de beurre frais, ou encore d’Émile Louis en train de jeter un coup d’œil dans le rétro de son bus rempli de gamines handicapées de la DDASS. Tu te pourléchais clairement les babines, n’en déplaise à ta susceptibilité outragée !

\textsc{Moi}, écrasant la pédale d’accélérateur avec une violence telle qu’on s’est retrouvés instantanément de l’autre côté de la route : C’est honteux ! Je ne sais pas ce qui me retient de m’arrêter et te jeter hors de la voiture !

\textsc{Lui} : Ta réaction prouve que j’ai mis dans le mille.

\textsc{Moi} : T’as rien mis dans quoi que ce soit ! À part ta bite dans le cul de cette pute, peut-être !

\textsc{Lui} : Tu vois, tu deviens grossier quand t’es énervé. Et puis c’est pas une pute, je te l’ai déjà dit.

\textsc{Moi} : Je suis pas énervé, je suis offusqué !

\textsc{Lui} : Reconnais que tu la trouves à ton goût.

\textsc{Moi} : Là n’est pas la question.

\textsc{Lui} : Ben si, un peu quand même.

\textsc{Moi} : Je ne mélange pas le travail et les sentiments, moi.

\textsc{Lui} : Moi non plus.

\textsc{Moi} : Tes sentiments interfèrent avec ton travail, ça revient au même.

\textsc{Lui} : C’est pas mes sentiments, c’est ma libido.

\textsc{Moi} : Oui ben ta libido, tu vas la laisser un peu de côté et te concentrer sur le boulot.

\textsc{Lui} : Métro, boulot, libidodo.

\textsc{Moi} : Très drôle !

\textsc{Lui} : Tu ne vas pas faire la gueule pendant tout le trajet, j’espère.

\textsc{Moi} : Je fais pas la gueule, je suis fatigué d’entendre des conneries à longueur de journée. Ça commence tôt le matin et ça ne s’arrête plus jusqu’au milieu de la nuit. J’essaie de t’aider, te sortir un peu de l’ornière dans laquelle tu te trouves, et toi tu ne trouves rien de mieux à faire que d’insinuer des choses à mon sujet, comme si je n’étais pas le preux chevalier blanc épris de justice et d’équité que je prétends être.

\textsc{Lui} : D’équitation, tu veux dire.

\textsc{Moi} : Pardon ?

\textsc{Lui} : Le preux chevalier blanc épris d’équitation.

\textsc{Moi} : Et voilà, je te parle de choses sérieuses, et toi tu continues à me bassiner avec tes blagues à deux balles ! Comment veux-tu que je ne sois pas au bout du rouleau avec des gens comme toi.

\textsc{Lui} : Moi aussi, je suis au bout du rouleau.

\textsc{Moi}, évitant de justesse une collision après avoir refusé une priorité à droite : Pas pour les mêmes raisons que moi.

\textsc{Lui} : Ralentis, tu veux.

\textsc{Moi} : Je ne roule pas vite.

\textsc{Lui}, s’accrochant tant bien que mal à tout ce qui lui tombait sous la main, y compris mon bras ou ma jambe à l’occasion, renforçant au passage les risques de perte de contrôle inhérent à la conduite pour le moins sportive que j’avais choisi d’adopter : Si. Tu as déjà failli provoquer au moins une demi-douzaine d’accidents.

\textsc{Moi} : Je suis un peu tendu, excuse-moi. Et enlève ta main de ma cuisse, tu veux bien.

\textsc{Lui} : Ralentis ou je descends de la voiture.

\textsc{Moi} : Je ne vois pas très bien comment.

\textsc{Lui} : Je vais sauter en marche, au risque de me casser une jambe ou atterrir sur un piéton qui n’a rien demandé à personne et va finir à l’hôpital avec une double fracture du crâne et des lésions irréversibles au niveau de la moelle épinière.

\textsc{Moi} : Je consens à lever le pied si tu retires ce que tu as dit à propos de moi et la Gardienne de la Nuit.

\textsc{Lui} : ATTENTION !!!!!!!

\textsc{Moi} : Quoi encore ?

\textsc{Lui} : Il y a une femme enceinte qui vient de s’engager sur le trottoir !!!!!!

\textsc{Moi} : Elle n’est pas enceinte, elle est juste grosse. Énorme, même.

\textsc{Lui} : Ce n’est une raison pour l’écraser.

J’ai pilé juste à temps, sous le regard horrifié de la grosse bonne femme qui s’est arrêtée net dans sa trajectoire.

Elle s’est mise à gesticuler des bras et des jambes, surtout des bras parce que ses jambes lourdes et cylindriques ne lui permettaient guère de se livrer à des facéties musculaires et autres prouesses articulaires, tout en m’insultant copieusement et me menaçant de représailles judiciaires dignes d’un caïd de la pègre ou un tueur cannibale. Ça a duré un certain temps, pendant lequel je me suis prudemment abstenu de toute déclaration intempestive ou objection inappropriée, après quoi elle a paru soulagée et s’est enfin décidée à reprendre sa route vers la destination qui était la sienne (qui n’était pas celle du cimetière, manifestement, en tout cas pas encore, car compte tenu de sa surcharge pondérale et des difficultés circulatoires afférentes, il y avait gros parier qu’elle ne tarderait pas à s’y retrouver).

C’est le moment choisi par Greg pour détacher sa ceinture et ouvrir la portière dans l’intention manifeste de s’extraire du véhicule.

\textsc{Moi} : Qu’est-ce que tu fais ?

\textsc{Lui} : Je te l’ai dit, je sors de la voiture.

\textsc{Moi} : C’est bon, je vais ralentir.

\textsc{Lui}, qui n’avait aucune envie de parcourir in pede (traduction latine de «~à pied~», laquelle ne fait aucunement référence à quelque orientation sexuelle réelle ou supposée concernant l’individu en question, lequel, par le plus grand des hasards, se trouvait également figurer en bonne place sur la liste~-- certes succincte mais tout de même pas totalement insignifiante~-- de mes meilleurs amis) les deux ou trois bons kilomètres qui nous séparaient encore de la rue des Maléfices, en plein cœur d’un quartier sensible où il ne faisait pas nécessairement bon se balader les mains dans les poches en sifflotant Dixie de Dan Emmett : Tu promets ?

\textsc{Moi} : Je promets. Mais toi, tu promets de ne plus faire d’insinuations gratuites à mon sujet.

\textsc{Lui} : J’ai bien vu comment tu la regardais.

\textsc{Moi} : Tu recommences ?

\textsc{Lui}, une fesse dans la voiture : Non non, j’arrête. Mais reconnais que tu n’aurais rien contre lui faire un brin de causette au bord de l’eau.

\textsc{Moi} : Au bord de l’eau ?

\textsc{Lui}, deux fesses dans la voiture, refermant la portière : Ou ailleurs.

\textsc{Moi} : Non. Au bord de l’eau, c’est bien.

\textsc{Lui} : De l’eau de mer, par exemple.

\textsc{Moi} : Sur la plage, quoi.

\textsc{Lui} : Au coucher du soleil.

\textsc{Moi} : Je ne suis pas fan des clichés, genre compter fleurette à une fille sublime sur la plage au coucher du soleil alors que le temps est d’une douceur élégiaque et que la fille porte une robe blanche translucide avec rien en dessous à part son corps de rêve tapissé d’un épiderme aussi fin et soyeux que la peau d’une pêche de vigne gorgée de nectar sucré. C’est un peu de la merde, tout ça, de la carte postale pour blaireau élevé à la malbouffe et la musique d’ascenseur.

\textsc{Lui}, attachant sa ceinture : Un peu, oui. Mais pas totalement.

\textsc{Moi} : Tu me fatigues, tu sais.

\textsc{Lui} : Je sais.

Je suis reparti, mais comme j’avais oublié de faire deux choses que tout conducteur se doit absolument de faire quand il redémarre après s’être arrêté sur le bas-côté de façon plus ou moins intempestive, à savoir mettre son clignotant et regarder son rétro pour s’assurer que personne n’arrive en trombe au même moment, j’ai évité de justesse une Mini Cooper qui précisément n’avait rien trouvé de mieux à faire que d’arriver en trombe à ce moment-là. Sinon en trombe, en tout cas à une vitesse (pour autant que je puisse en juger, et sans me vanter je suis assez fort pour déterminer avec une marge d’erreur quasi insignifiante la vitesse des véhicules qui se déplacent autour de moi) largement supérieure aux trente kilomètres-heure autorisés. Cela dit, en cas de choc, même si la conductrice en question (il s’agissait d’une femme, je le précise sans la moindre arrière-pensée sexiste concernant la prétendue dangerosité des femmes au volant, lesquelles, si l’on en croit certains individus dont la misogynie avérée et les blagues graveleuses ne plaident guère en faveur de l’intelligence, seraient moins occupées à conduire qu’à bavasser au téléphone ou se repoudrer le nez dans le rétro de courtoisie, quand elles ne seraient pas en train de fouiller dans la boîte à gants ou vagabonder dans leurs pensées au point de perdre tout contact avec la réalité) n’était pas totalement en règle avec la limitation de vitesse en vigueur dans le quartier, j’aurais été tenu pour seul responsable du sinistre, ayant déboîté sans mettre mon clignotant ni m’assurer que la voie était libre. J’aurais bien sûr prétendu le contraire, comme toute ordure qui se respecte, avec un aplomb sans faille, une morgue abjecte digne du politicien le plus corrompu, et Greg se serait fait un devoir d’abonder dans mon sens, en toute mauvaise foi, répugnant de duplicité servile, mais il m’aurait fallu un certain temps (au moins deux ou trois heures) avant de pouvoir à nouveau contempler mon doux visage dans la glace sans être aussitôt pris d’une violente envie de gerber. Comme vous n’êtes pas sans le savoir, du moins je l’espère, l’être humain est ainsi fait qu’il s’accommode assez facilement des bassesses ordinaires de sa moralité douteuse, trouvant sans cesse des arrangements avec sa conscience, laquelle semble ne lui avoir été donnée que pour servir de caution à ses exactions (en plus, tout de même, d’être un rempart naturel à la frénésie destructrice qui l’agite habituellement, à mon sens largement supérieure aux capacités créatrices qui lui ont été dévolues, sans cesse dévoyées par la purulence égotique qui préside à l’essentiel de ses activités).

Surprise, et heureuse de s’en tirer à bon compte, la conductrice s’est éloignée en vociférant tant et plus, jetant des coup d’œil furieux dans son rétroviseur et agitant les bras de façon désordonnée, au risque de perdre le contrôle de son moyen de locomotion, griller le stop qui l’attendait au bout de la rue et emplafonner le bus qui arrivait à vive (très, sans doute même trop, le chauffeur, en plein divorce et affligé de problèmes de santé aussi divers que douloureux, affichant une nervosité assez préjudiciable à la souplesse de sa conduite) allure sur sa droite.

Greg, qui, sans doute par manque de vigilance au moment des faits, ne semblait pas avoir pris la pleine mesure de la catastrophe que nous venions de frôler : Tu crois qu’il lui est arrivé quoi, à Titus ?

\textsc{Moi} : Je pense que la Gardienne de la Nuit l’a entraîné dans son repaire pour l’initier à des activités sexuelles dont nous n’avons même pas idée.

\textsc{Lui} : Dans ce cas, on ferait mieux de le laisser où il est.

\textsc{Moi} : L’ennui, c’est qu’on ne sait pas où il est.

\textsc{Lui} : Au Caribbean Hôtel, non ?

\textsc{Moi} : J’ai tout lieu de le penser, en effet. Je pense qu’elle en a fait son jouet sexuel et le séquestre dans une chambre sans numéro située quelque part dans les bas-fonds de l’établissement, une sorte de nid d’amour, d’alcôve secrète réservée aux proies de cette harpie en jupon.

\textsc{Lui} : Oui, enfin, je crois surtout qu’elle lui a tapé dans l’œil et qu’il a complètement oublié qu’il avait une femme, des gosses et un boulot. Il va réapparaître en fin de matinée, le bec enfariné, et prétendre qu’il ne se souvient de rien.

\textsc{Moi} : Peut-être qu’elle lui a fait boire un philtre d’amour à base de belladone, jusquiame noire, miel vert et vin de Pramnos, telle Circé à Ulysse.

\textsc{Lui} : On appelle ça du GHB de nos jours. C’est moins glamour mais tout aussi efficace.

\textsc{Moi} : Je préfère miel vert et vin de Pramnos. Toujours est-il que Bérénice va être folle de rage si elle apprend qu’il a passé la nuit avec une femme. Elle saura aussi qu’on a essayé d’étouffer l’affaire, et notre belle amitié finira dans la benne à ordures.

\textsc{Lui} : Si on lui sort que Titus a été l’innocente victime d’une créature de rêve dotée de pouvoirs surnaturels, elle nous clouera au pilori sans l’ombre d’une hésitation.

\textsc{Moi} : Oui, à grands coups de marteau. Après quoi elle nous crèvera les yeux et nous arrachera les entrailles pour les donner à bouffer aux corbeaux !

\textsc{Lui} : Elle retrouvera la fille, l’arrosera d’essence et se fera un plaisir de la réduire en cendres.

\textsc{Moi} : Comme les sorcières au Moyen Âge.

\textsc{Lui} : Quant à Titus, elle lui coupera les couilles avec des ciseaux rouillés et l’obligera à les bouffer pour qu’il s’étouffe avec.

\textsc{Moi} : Une mort atroce, et on ne pourra rien faire pour l’empêcher.

\textsc{Lui} : Rien.

\textsc{Moi}, appuyant sur le champignon : C’est pour ça qu’il faut se dépêcher si on veut éviter le pire.

\textsc{Greg} : Je te signale que tu viens de griller un feu rouge.

\textsc{Moi} : Non, je suis passé à l’orange.

\textsc{Greg} : Un orange très très mûr, alors.

\textsc{Moi} : Tu veux sauver Titus, oui ou merde ?

\textsc{Lui} : Bien sûr que je veux sauver Titus. Je donnerais tout pour sauver Titus. Tout sauf ma vie, parce que même si j’adore Titus, j’ai quand même une certaine affection pour ma propre existence. Je dirais même une certaine addiction, contractée au fil des années passées à rouler ma bosse sur cette terre. C’est sans doute de la faiblesse de ma part, mais si quelqu’un devait mourir, j’aimerais autant que ce soit lui. J’aurais du mal à m’en remettre, bien sûr, et son souvenir resterait gravé dans ma mémoire jusqu’à la fin de mes jours, mais je pense que j’aurais assez de force de caractère pour arriver à vivre avec.

\textsc{Moi} : Tu devrais avoir honte de dire des choses pareilles !

\textsc{Lui} : J’ai honte, mais j’ai assez de force de caractère pour arriver à vivre avec.

\textsc{Moi} : Et moi ?

\textsc{Lui}, Quoi, toi ?

\textsc{Moi} : Si ma vie était en jeu ?

\textsc{Lui} : Tu veux savoir si je te laisserais crever comme un chien ?

\textsc{Moi} : Je ne me fais aucune illusion.

\textsc{Lui} : Toi c’est pas pareil, t’es comme un frère pour moi. Je ne dis pas que j’irais jusqu’à donner ma vie pour toi, mais je serais prêt à sacrifier quelques morceaux de mon anatomie.

\textsc{Moi} : Non ?

\textsc{Lui}, posant une main sur ma cuisse : Si.

\textsc{Moi} : Okay, je te remercie. Tu peux enlever ta main, s’il te plaît ?

\textsc{Lui} : Ah oui, pardon.

\textsc{Moi} : On arrive bientôt. Je vais tâcher de trouver une place stratégique pour me garer, histoire qu’on puisse se barrer en vitesse si les choses tournent mal.

Personne n’ayant eu l’idée saugrenue de le déplacer pendant la nuit, le Caribbean Hôtel se trouvait exactement au même endroit que la veille.
