
\noindent Il y a ce qu’on appelle la «~loi des séries~», comme si le destin vous imposait une thématique à laquelle il semble impossible de déroger.

Assez loin de ces considérations, tous les dimanches matin, aux premières lueurs de l’aube, Robert Pleimelding allait à la chasse. Avec sa fidèle Greta, un braque allemand à poil dur qu’il appelait «~ma fille~» ou «~fifille~», alors même qu’il avait déjà une fille, une vraie, avec laquelle il n’échangeait guère plus de trois ou quatre mots par an, et encore seulement les bonnes années.

Le rituel était immuable, et rien, pas même un cyclone déferlant sur le secteur, n’aurait pu altérer son déroulement.

Tous les dimanches matin, tandis que Claire, sa femme, ronflait encore à poings fermés dans le creux de son lit (ils faisaient chambre à part depuis de nombreuses années, leurs horaires et habitudes nocturnes ayant atteint un seuil d’incompatibilité irréductible), Robert, après avoir passé un bon quart d’heure aux toilettes et avalé un demi-litre de café noir, enfilait sa tenue de chasse et sortait sur la terrasse pour voir le soleil se lever (quand il y en avait, bien sûr, car l’astre luminescent ne parvenait pas toujours à transpercer l’épaisse couche de brouillard qui tapissait la vallée) et respirer à pleins poumons les toutes nouvelles senteurs du jour.

Accessoirement, Robert était le genre de type qui, en plus d’une réussite familiale discutable, n’avait aucun ami, pas même au bistrot du village où il se gardait bien de mettre les pieds pour éviter les ennuis. Il avait, toute sa vie durant, travaillé pour le comte Léopold Chiasson de Bellisle, propriétaire du château voisin, en tant que garde-chasse. Il avait vingt-cinq ans quand il était entré au service de Bellisle, lequel en avait une bonne vingtaine de plus. Le comte, qui vivait seul sur ses terres avec quelques domestiques, une armada de bestioles en tout genre (ça allait de la poule et la pintade au lapin en passant par les chevaux, bien sûr, comme la plupart des rupins qui se la jouent, plus un certain nombre d’espèces exotiques rapportées de ses nombreuses expéditions à l’étranger) et une meute de chiens de race, affichait un très net penchant pour les gens de son sexe. Bon, il est certain que quand on voyait la tronche de Robert quarante ans plus tard, il était difficile d’imaginer quelle petite gueule d’amour il avait été dans sa jeunesse. Non seulement il avait une bouche spécialement conçue par le designer en chef de la Création pour enfourner de la saucisse au kilomètre, mais il était équipé d’un petit cul trois étoiles dont il aurait été criminel de ne se servir que pour s’assoir ou faire ses besoins. Homme de goût, amateur d’art et de belles choses en général, Chiasson de Bellisle ne pouvait qu’être émerveillé à la vue d’un tel chef-d’œuvre. Convaincu, il avait engagé sur le champ Robert et sa jeune épouse, Robert en tant que garde-chasse et Claire, qui avait de réels talents en la matière, en tant que cuisinière.

Rapidement, les relations entre Robert et son employeur avaient pris une tournure d’un genre un peu particulier.

D’abord, Robert était anormalement bien payé pour un garde-chasse, avec un salaire correspondant davantage à celui d’un ministre que d’un type qui passe son temps à se balader en forêt avec un fusil sur l’épaule, veillant à l’équilibre de la faune et la flore, et s’assurant qu’aucun contrevenant ne passait outre les panneaux PROPRIÉTÉ PRIVÉE affichés aux quatre coins du domaine. Dans le cas contraire, il avait ordre de faire feu sans sommation et enterrer le corps sur place, à la merci des bêtes sauvages, le comte étant un fervent défenseur des méthodes à l’ancienne, le bon vieux temps où le châtelain avait toute latitude pour faire régner l’ordre sur ses terres. Autant dire que si vous vous faisiez piquer avec un panier de cèpes ou un lapin dans la musette, vous aviez intérêt à courir très vite pour passer entre les plombs. À moins, bien sûr, de présenter, comment dire, certaines caractéristiques physiques susceptibles d’amener la partie civile à envisager l’éventualité d’un règlement à l’amiable du litige. Dans ce cas, pourquoi ne pas rester à quatre pattes le nez dans la mousse, à cueillir tranquillement des champignons, pendant que s’exerce avec une saine vigueur le droit séculier de la justice ?

Même chose pour Claire, dont la rémunération ne devait pas être très éloignée de celle d’un chef trois étoiles dans un palace méditerranéen. Il faut dire que sa cuisine était assez remarquable, et qu’elle était tout à fait capable de rassasier des tablées de quinze ou vingt personnes.

Si vous ajoutiez à cela que les tourtereaux étaient logés gratos dans le pavillon de gardien, disposant, en plus d’une terrasse et d’un jardin très convenables, de trois chambres, une belle cuisine parfaitement équipée, un bureau-bibliothèque (même si Robert n’avait pas lu plus de trois ou quatre livres dans toute son existence, et encore des livres avec des images et des grosses lettres) et un grand salon cossu avec poutres apparentes, baie vitrée et cheminée assez vaste pour faire cuire un obèse du Mississippi (c’est là qu’il y en a le plus) imbibé de sirop de maïs trafiqué et huile de palme hydrogénée, il n’y avait à priori aucune raison de se plaindre, même s’il leur était demandé une totale disponibilité en retour.

Robert aimait Claire (et les roberts de Claire, si ronds, fermes et débordants d’affection qu’ils faisaient, je ne crains pas de l’affirmer au risque de passer pour un immonde pourceau, l’envie et l’admiration de tous) et Claire aimait Robert, autrement dit Claire et Robert et inversement s’aimaient d’un amour si beau et pur qu’il brillait telle une pépite dans le lit boueux de la rivière infestée de sangsues de l’existence impitoyable qui est notre lot commun. Cet édifice affectif inébranlable reposait sur des fondations solides dont le ciment principal n’était autre qu’un mot de neuf lettres commençant par un c et se terminant par un e, je veux bien sûr parler de la confiance, laquelle se rapproche davantage de la confidence que du confit d’oie (même si rien n’empêche de se confier à quelqu’un en mangeant du confit d’oie, surtout si on arrose le tout de quelques bonnes rasades de Cahors ou Madiran). À ce titre, Robert n’avait jamais caché à Claire qu’il avait, depuis sa prime enfance, des hésitations sur la nature exacte de sa sexualité. Après mûre réflexion, il en était arrivé à la conclusion qu’il aimait les femmes, très certainement, mais n’était pas pour autant insensible aux appas de son sexe. Pas assez pour se lancer à corps perdu dans une homosexualité débridée, certes, mais tout de même suffisamment pour céder de temps à autre aux sirènes de l’inversion. Il lui arrivait, à l’occasion, de fréquenter, dans une tenue qui n’était pas la sienne habituellement, certains établissements dont la nature douteuse était de notoriété publique. Certains vont jouer au casino, d’autres vont se faire défoncer le cul dans des endroits fleurant bon la sueur, le cuir et le foutre.

Un soir, sous un prétexte aussi quelconque que fallacieux, Bellisle avait invité Robert à passer le voir.

Robert avait trouvé son employeur au coin du feu, nu sous une robe de chambre élimée dans le genre de celles que portaient les aristos des temps jadis, luth en main. Quand je parle de luth, je ne parle pas de lutte gréco-romaine mais de vrai luth, cette sorte de guitare du Moyen Âge qui a plus ou moins la forme d’une larme ou une goutte d’eau, l’instrument dont un Bernard de Ventadour\nf{Bernard de Ventadour (v.~1130--v.~1200), troubadour limousin considéré comme l’un des plus grands poètes occitans du \textsc{xii}\textsuperscript{e}~siècle, célèbre pour ses chansons d’amour courtois adressées à des dames de haute noblesse. \textit{Source :} \textup{fr.wikipedia.org/wiki/Bernard\_de\_Ventadour}} se servait pour enchanter la belle Aliénor d’Aquitaine\nf{Aliénor d’Aquitaine (v.~1122--1204), duchesse d’Aquitaine et reine successivement de France (épouse de Louis~VII) puis d’Angleterre (épouse d’Henri~II Plantagenêt). Grande mécène des troubadours, elle est l’une des figures politiques les plus puissantes du Moyen Âge occidental. \textit{Source :} \textup{fr.wikipedia.org/wiki/Ali\%C3\%A9nor\_d\%27Aquitaine}} à la cour du roi de France.

Invité à s’assoir, Robert s’était aussitôt retrouvé englouti dans un fauteuil qui avait dû voir défiler des millions de culs de toutes formes et tailles à travers les âges. À peine avait-il eu le temps d’ouvrir le bec pour formuler quelques vagues remerciements qu’il s’était vu proposer un havane et sommé d’avaler un verre de cognac de la taille d’un ballon de foot. Pendant ce temps, lui-même passablement éméché, le comte, sur des accords hésitants, poussait la chansonnette d’une voix plus proche de la toile émeri que du velours. La scène frisait le ridicule, admettons-le, mais il aurait fallu être foutrement ingrat pour ne pas rendre les armes devant une telle parade nuptiale, si bien arrosée et finement orchestrée, et le fait est que si on pouvait reprocher bien des choses à Robert, comme à tout le monde du reste, l’ingratitude ne faisait certainement pas partie de ses défauts.

C’est ainsi, après avoir rapidement épuisé son répertoire, que le comte a eu beau jeu de se jeter, telle une hyène en chaleur aux babines ruisselantes de salive nauséabonde, sur la proie en partie liquéfiée qui s’étalait devant lui. Consciente que sa fin était proche, que toute résistance était inutile, celle-ci venait d’ailleurs d’écluser son troisième (on aurait dit que le comte avait des mains partout, comme Shiva, tant il était capable de continuer à jouer et chanter~-- mal, certes, mais quand même~-- tout en remplissant les verres à la vitesse de l’éclair) ballon de cognac (pour info, sachant que non seulement il ne fumait pas mais allait bientôt avoir quelque chose d’autrement consistant à se fourrer sous la dent, Robert avait décliné l’offre du havane).

La suite (ce déchaînement de frénésie sexuelle qui fait irrésistiblement penser à une meute de hyènes affamées se disputant la carcasse d’un phacochère tombé sous les crocs d’un lion) se passe de commentaires. C’est moche, la vie est moche, on n’a pas d’autre choix que faire avec (raison pour laquelle la plupart d’entre nous ont une telle soif de liberté, qu’ils n’obtiendront bien entendu jamais, même si l’argent, qui précisément achète, corrompt et dénature tout, peut leur donner l’illusion du contraire), à moins d’éviter de naître ou se faire sauter le caisson à la première occasion, mais la nature est toujours là, tapie, embusquée derrière les écrans de fumée de la raison, l’intelligence et la conscience, et quand elle réclame son dû, ses esclaves se voient dans l’obligation de satisfaire au plus vite à ses exigences. C’est d’ailleurs ce qui pousse certains à commettre des exactions que la morale réprouve, et avec elle la société dans son ensemble, qui tente désespérément de sauver les meubles.

Hélas, ou tant mieux, tout a une fin, à commencer par les meilleures choses dont la durée de vie est d’autant plus courte qu’elles se rapprochent de l’excellence.

Un jour, alors qu’il était en train de fumer un Cohiba (un Esplendido Gran Reserva, son module préféré et grand classique de la marque) en contemplant les plus belles fleurs de sa roseraie, Bellisle avait rendu son âme à Lidos, empereur des trous de balle et dieu des sodomites. La chose s’était produite en une fraction de seconde : d’abord une violente douleur dans la poitrine, aussi vive que l’éclair, suivie de quelques spasmes désordonnés, doigts crispés cherchant vainement à se raccrocher à quelque chose, et notre homme s’affaissait tel un vieux slip plein de merde dans un massif de ces fabuleuses roses noires (ça ne s’invente pas) en provenance des jardins luxuriants d’Anatolie du sud-est.

C’est le jardinier, digne héritier de la statuaire grecque mais solidement équipé sur le plan génital (contrairement aux Grecs qui ont souvent des petites bites, vous l’aurez sans doute remarqué si vous vous intéressez un tant soit peu à la question), qui avait découvert le corps. On se doute que le comte ne l’avait pas engagé pour ses seules compétences en matière de taille, tonte et bouturage. Non, le vieux salopard avait tenté à maintes reprises de le détourner du droit chemin de la reproduction, des vertes allées de l’amour courtois, allant jusqu’à lui proposer des sommes déraisonnables en échange de l’anneau de jouissance qu’il dissimulait jalousement entre ses petites fesses potelées (une version moderne du Seigneur des anneaux, si vous préférez, sous sa forme sainte trilogie de l’arrière-train : La Communauté de l’anus, Les Deux Trous et Le Retour du doigt). Mais l’animal (à poil la plupart du temps, hormis un micro-short en jean déchiré~-- tout juste si on ne lui voyait pas les boules de chaque côté~-- et une paire de bottes en caoutchouc), aussi difficile à croire que cela puisse sembler, était d’une hétérosexualité à toute épreuve, un parangon de vertu insensible à toute espèce de corruption, une sorte de réincarnation sexy du Christ. Il aurait pu le virer et le remplacer par un autre, plus coopératif, mais se plaisait énormément à la dégustation platonique de ce fruit défendu, rêvant à la jutosité de sa chair et la saveur sucrée de ses sécrétions intimes. Il existait, chez Bellisle, une force qui le poussait à toutes les perversions, et s’infliger le supplice de Tantale était sans doute pour lui le moyen d’expier une partie de ses fautes.

Prévoyant, et très épris, Bellisle, qui n’avait aucun héritier direct, avait couché (avec, mais pas que) Robert sur son testament, lui léguant une somme confortable grâce à laquelle ce dernier avait pu acquérir l’agréable demeure qu’il habitait aujourd’hui.

Et donc, comme je vous le disais, cette agréable demeure, il la quittait tous les dimanches matin pour aller à la chasse avec Greta, sa chienne, acquise elle aussi avec cet héritage ; tout comme, du reste, le juxtaposé Chapuis Progress Grand Luxe calibre 12/70 (crosse anglaise en noyer premier choix, plaque de couche en bois de rose et sujets animaliers gravés en taille douce sur les contre-platines et le dessous de bascule, grosse émotion) qu’il chérissait comme la prunelle de ses yeux et passait des heures à lustrer en écoutant des opérettes de Jacques Offenbach\nf{Jacques Offenbach (1819--1880), compositeur et violoncelliste franco-allemand, créateur de l'opéra-bouffe français. Auteur de plus de cent opérettes, dont \textit{La Belle Hélène} (1864), \textit{Orphée aux Enfers} (1858) et \textit{Les Contes d'Hoffmann} (posth., 1881). \textit{Source :} \textup{fr.wikipedia.org/wiki/Jacques\_Offenbach}}, Reynaldo Hahn\nf{Reynaldo Hahn (1874--1947), compositeur, chef d'orchestre et critique musical vénézuélo-français, ami intime de Marcel Proust. Connu pour ses mélodies et ses opérettes, notamment \textit{Ciboulette} (1923) et \textit{Mozart} (1925). \textit{Source :} \textup{fr.wikipedia.org/wiki/Reynaldo\_Hahn}} et Francis Lopez\nf{Francis Lopez (1916--1995), compositeur français spécialiste de l'opérette populaire, auteur de \textit{La Belle de Cadix} (1945) et \textit{Le Chanteur de Mexico} (1951), qui connurent un succès considérable dans l'après-guerre. \textit{Source :} \textup{fr.wikipedia.org/wiki/Francis\_Lopez}}.

Solitaire de nature, c’est seul avec Greta qu’il aimait s’adonner à sa passion. Il abandonnait volontiers le gros gibier aux viandards pour se consacrer à des cibles plus modestes, selon lui autrement intéressantes gustativement et passionnantes à traquer, comme le lièvre et la perdrix, mais surtout le plus noble, intense et raffiné d’entre tous, je veux bien sûr parler de Scolopax rusticola, migrateur à long bec et plumage mimétique plus connu sous le nom de bécasse. Naturellement, à l’occasion, il ne crachait pas sur la grive, le garenne ou le colvert. Du moment que la bête tenait dans la marmite, il n’y avait aucune raison de s’en priver. En ce qui concerne la bécasse, Greta n’avait pas son pareil pour la débusquer (au même titre que la truffe), et Claire pour la cuisiner, au four, à la ficelle, en cocotte au vin blanc ou rouge, à la truffe ou au foie gras, flambée à l’Armagnac ou avec ce très vieux Calva d’une richesse aromatique sans précédent que Robert se procurait au compte-goutte chez un producteur confidentiel (Raoul Gounelle, je ne devrais pas vous le dire mais je le fais quand même parce que vous êtes de bons clients). Quand le comte (de son vivant bien sûr, il n’allait quand même pas revenir de l’au-delà exprès pour ça) leur faisait l’honneur de venir dîner chez eux, elle ne manquait jamais de le régaler avec une bonne plâtrée de bécasse, surtout les recettes avec beaucoup de truffe et de foie gras, ou à défaut de bécasse un de ces civets de lièvre ou rôti de colvert aux châtaignes et champignons des bois (le comte en raffolait et s’en foutait plein les moustaches en poussant des grognements de satisfaction) dont elle avait le secret, après quoi leur hôte s’allumait un de ses havanes favoris (il demandait toujours la permission et personne n’osait la lui refuser) et s’endormait lourdement sur le Calva avant de rentrer à quatre pattes chez lui, Robert dans son sillage pour vérifier qu’il ne se trompait pas de direction.

Ce matin-là, donc (je ne sais pas si je vais y arriver), aux aurores, Greta (folle de joie à l’idée d’aller se dégourdir les papattes en forêt), Robert, sa gibecière à rabat en cuir pleine fleur Lazzaro Bernardini (dans les deux mille euros sur plaisirsdelachiasse.com), son fusil Chapuis de collection, un thermos de café, des vivres pour midi (dont une belle grosse gourdasse d’un Hautes-Côtes de Nuits somme toute assez gouleyant), et assez de cartouches pour soutenir un siège face au GIGN, sont tous montés dans le Land Cruiser que ce même Robert, toujours lui, s’était offert avec le pognon de Bellisle. Robert s’est installé au volant, Greta à ses côtés (elle préférait monter devant), a tourné la clé de contact, le 4x4 a hoqueté quelques instants avant de faire entendre la douce musique de son six cylindres en ligne, puis tout ce joli petit monde a pris le chemin des quelques cinquante hectares de forêt que cet enfoiré de Robert, encore et toujours lui, avait également hérité de son bienfaiteur. Une belle journée s’annonçait, pleine de soleil et de ciel bleu, et avec elle la promesse d’une partie de chasse réussie quel que soit le résultat des courses. Même s’il fallait se contenter d’un lapin ou deux, avec en prime une perdrix et quelques cèpes, la joie de se balader au grand air, en toute liberté, avec Greta gambadant à ses côtés, l’œil vif et la truffe au ras du sol, attentive au moindre bruit, au moindre frémissement, et surtout personne pour vous poser des questions idiotes et vous débiter des inepties dans les oreilles, suffirait amplement à son bonheur.

Seulement voilà, dans l’existence, tout ne se passe pas nécessairement comme prévu, raison pour laquelle il est plus prudent de ne rien prévoir du tout.

Quelques heures plus tard, il avait tiré un lièvre et ramassé quelques belles poignées de girolles qui, préparées au beurre, à l’ail et au persil, constitueraient un accompagnement idéal pour le civet. Quoi de plus merveilleux que de pouvoir se goinfrer des bienfaits que dame Nature met généreusement à la disposition de ceux qui savent où les trouver, quand d’autres, ignorant tout des mystères de la truffe et la perdrix, doivent se satisfaire de poulet de batterie et autres champignons de couche insipides.

Nous le retrouvons adossé à un chêne centenaire, en train de siroter un café en écoutant chanter les oiseaux et regardant les écureuils sauter de branche en branche, tout en caressant tendrement la tête du lièvre qu’il venait d’abattre et humant à pleins poumons une poignée des girolles qu’il venait de collecter.

Il avait, en cet instant béni des dieux, le sentiment d’appartenir à un club très fermé de privilégiés. Certes, en plus d’un Chapuis de collection et une gibecière Lazzaro Bernardini entièrement fabriquée à la main dans un cuir de très haute qualité, il ne s’agissait somme toute que de quelques hectares (une cinquantaine tout de même) de forêt giboyeuse hérités d’un généreux donateur auquel il avait lui-même fait don d’une bonne partie de sa personne et son existence. Mais tout de même, quelle joie d’être le seul à en profiter, d’en avoir la jouissance exclusive, pouvoir s’y promener en toute tranquillité de corps et d’esprit, même s’il lui arrivait parfois de croiser tel ou tel bipède humain qui s’était indûment introduit dans la place (il aurait pu l’abattre et faire disparaître le corps, comme Bellisle en son temps, mais, toujours magnanime et didactique, près du peuple auquel il avait toujours appartenu, il se contentait généralement de quelques remontrances et un simple avertissement). Certains n’auraient sans doute pas manqué de prétendre qu’une telle attitude était injuste et égoïste, un tel privilège inacceptable, surtout dans le contexte actuel où tant de gens vivaient les uns sur les autres, dans un état de promiscuité qui les poussait à la violence et l’incompréhension, ce à quoi il ne se serait pas gêné pour rétorquer que la vie était une guerre sans merci qui obligeait chacun, même le mieux intentionné d’entre nous, à défendre âprement le moindre centimètre-carré de terrain arraché à l’adversité. Ces quelques hectares de forêt étaient comme une île déserte au milieu d’un océan de béton charriant la fureur mécanique et les miasmes de la modernité. La pollution gagnait du terrain, bientôt lièvres et champignons seraient à tel point chargés d’immondices qu’ils pourriraient sur pied et deviendraient impropres à la consommation. Ensuite, quand toute forme de vie serait à l’agonie, viendraient les bulldozers pour transformer ce havre de paix en terrain vague et laisser le champ libre aux promoteurs et autres marchands de mort par correspondance. Et un jour, après avoir tout détruit, comme c’était déjà le cas dans certaines métropoles saturées de gaz toxiques et d’agressivité, aussi hostiles que les jungles d’antan, l’être humain serait obligé de faire pousser des arbres et du gazon sur le toit des immeubles, à des centaines de mètres de la terre ferme, et une fois de plus il se féliciterait ouvertement de son ingéniosité et sa capacité à survivre, tout contrôler et dominer le monde. En prévision de la disparition des espèces animales, il aura pensé, sur le modèle d’une arche de Noé de laboratoire, à conserver, soigneusement numérotées et étiquetées, les cellules souches qui lui permettront de reconstituer artificiellement un simulacre de vie, cette vie qu’il s’acharne à détruire faute de pouvoir la soumettre entièrement à sa volonté, cette garce impudente et d’un naturel désarmant qui a toujours une longueur d’avance sur lui, se permet de lui damer le pion au moment où il s’y attend le moins, à lui qui est au sommet de la chaîne alimentaire, le prédateur ultime auquel nul ne peut se soustraire, l’égal des dieux dont le souvenir s’efface peu à peu dans les méandres de la mémoire collective en déshérence. Sur ce champ de bataille et de dévastation, jonché des cadavres du vieux monde, des vestiges du passé soigneusement rassemblés dans des sanctuaires payants que les gens font la queue pour aller visiter, se dresse, de toute la hauteur de sa supériorité intellectuelle et son omnipotence technologique, revêtu de l’armure étincelante de son génie créatif, l’Homme victorieux et tout-puissant qui, à l’instar des plus hautes instances de l’univers désormais en perdition, est seul en mesure de définir les contours de l’avenir et assurer brillamment la survie d’une espèce en voie de disparition, la sienne en l’occurrence.

Tandis que, adossé à son chêne, Robert laissait son esprit vagabonder au fil de réflexions dont la hauteur de vue le laissait lui-même pantois, tant il savait son intelligence limitée d’ordinaire, Greta, qui furetait alentour à la recherche d’une piste, avait disparu au détour d’un fourré.

Sachant qu’elle ne tarderait pas à revenir, Robert a continué à siroter son café comme si de rien n’était, l’appelant de temps à autre pour s’assurer qu’elle restait bien dans le périmètre.

Et en effet, quelques instants plus tard, elle a refait surface avec un objet non identifié entre les dents, objet qu’elle s’est empressée de déposer fièrement aux pieds de son maître.

Lequel a dit, en tapotant doucement la tête de l’animal : Bonne fille !

Greta a poussé un petit jappement de satisfaction en sautillant sur place et remuant frénétiquement la queue.

Puis il a ajouté, plus pour lui-même que quelqu’un d’autre puisqu’il n’y avait personne d’autre que lui-même pour l’entendre, à part Greta qui n’avait pas les accréditations nécessaires pour répondre à ce genre de question (il arrive toutefois que certaines personnes s’adressent à leur chien comme à un être humain, avec le sentiment qu’il existe entre eux un lien si fort, intense et mystérieux que même le plus brillant poète, le plus au fait des arcanes du langage et des subtilités littéraires, serait bien incapable de traduire avec des mots) : Qu’est-ce que c’est que ça ?

Greta, si elle avait pu faire usage de la parole, n’aurait pas manqué de répondre que, pour autant qu’on puisse en juger, il s’agissait vraisemblablement d’une vieille basket pourrie calcinée jusqu’à l’os, autrement dit un objet qu’on s’attendait assez peu à découvrir en pleine forêt au milieu des ronces.

Interloqué, Robert s’est levé, a récupéré son barda, et dit à Greta en lui agitant la godasse sous le nez : T’as trouvé ça où, ma fille ?

Greta, même si elle ne parlait pas le français, savait que quand son maître se levait, récupérait son barda et lui agitait quelque chose sous le nez, cela signifiait qu’il voulait en savoir un peu plus sur l’objet en question, à commencer par l’endroit où elle l’avait trouvé.

C’est donc sans hésiter qu’elle s’est mise à trottiner devant lui pour le conduire audit endroit.

L’endroit en question était une sorte de taillis dont une bonne partie avait été réduite en cendres. Au milieu de ce tas de cendres, particulièrement nauséabond et hideux, se trouvait une horreur sans nom qui ressemblait vaguement à un corps humain, recroquevillé sur lui-même dans une expression d’indicible souffrance. Quelques touffes de cheveux et lambeaux de chair calcinée adhéraient encore au crâne noirci. Il en allait de même pour le reste du squelette, lequel présentait ici et là des morceaux de viande semblable à du charbon de bois.

Sous le choc, Robert prenait lentement conscience que sa partie de chasse venait de prendre un tour tragique. Car ici même, dans son bois à lui, sa PROPRIÉTÉ PRIVÉE dûment signalée par de nombreux panneaux d’avertissement de couleur vive, une personne avait été réduite en cendres par une ou plusieurs autres, le tout sans aucune autorisation, dans le plus total mépris de la législation en vigueur et la propriété d’autrui.

Dans le cas présent, faire le 15 était sans objet, la victime ayant peu de chances d’être ranimée.

Même chose pour le 18, les pompiers n’étant plus d’aucune utilité pour circonscrire l’incendie.

Idem pour l’alerte attentat, l’enfance en danger et le sauvetage en mer, autant de situations délicates qui ne cadraient pas avec la configuration présente.

Restait le 17, que Robert s’est empressé de composer sitôt son téléphone en main.

Shirani et moi-même, qui n’avions rien de mieux à foutre à ce moment-là (ou peut-être que si, mais la description du sinistre nous avait paru intéressante sur le coup), sommes arrivés sur le terrain quelques heures plus tard (il n’y avait pas le feu, si j’ose dire, la victime n’en étant plus à quelques minutes près), avec la PTS, bien sûr, ses gants en latex, ses masques chirurgicaux, ses pinces à épiler et ses écouvillons, plus un nombre assez considérable d’agents en uniforme censés passer les alentours au peigne fin (même si, en réalité, on est plus près du troupeau de sangliers qui ravage tout sur son passage).

À propos de Shirani, je dois vous dire que nos rapports dépassaient maintenant très largement le cadre de la simple relation professionnelle. Comme Riggs et Murtaugh dans L’Arme fatale, Mills et Somerset dans Seven, Rizzoli et Isles (parité oblige) ou même Turner et Hooch dans un style plus canin, on était, au fil du temps, devenus des vrais potes qui partageaient des vraies valeurs et se tapaient la cloche dès qu’ils en avaient l’occasion. Un certain nombre de choses nous avaient rapprochés, notamment un goût commun pour la viande froide (avec ou sans mayonnaise) et, pourquoi ne pas le dire au risque de passer pour des cons obsolètes, une appétence marquée pour les belles Italiennes carrossées comme des voitures de sport. Je sais que ce genre de comparaison mécanique ne se fait plus trop aujourd’hui, et on ne peut que se féliciter du fait que les femmes revendiquent haut et fort le droit d’être considérées comme des êtres humains à part entière, et non plus seulement des appareils ménagers ou des objets destinés à satisfaire les pulsions sexuelles de mâles sujets à la violence et la vulgarité. Bon, en même temps, je note quand même que les sexbots et autres love dolls plus ou moins perfectionnés font fureur chez nos amis Nippons, lesquels, on le sait, ont toujours été des pervers de premier plan, des prédateurs faussement timides chez qui l’obsession de la chair fraîche le dispute à un sens de l’honneur et la politesse frisant le ridicule. Cela dit, avec l’Intelligence Artificielle, il faut s’attendre à voir ce genre de compagnon de route, tant féminin que masculin, accéder à un niveau de sophistication et de réalisme de plus en plus élevé, au point de représenter une réelle menace pour l’avenir de l’humanité sur le plan relationnel et affectif. Déjà que c’était pas brillant, ça risque de devenir franchement catastrophique. Imaginez un homme (ou une femme, bien sûr, vu qu’elles sont à présent sensiblement logées à la même enseigne) qui a un four micro-ondes, un lave-vaisselle, un aspirateur robot laveur et une machine à laver : quelque chose me dit qu’il ou elle pourrait bien être tenté de s’offrir un conjoint full option, personnalisable (ou fait main sur mesure pour les plus fortunés, avec des vrais cheveux et poils humains, modèle chauffant pour donner l’illusion de la vie, hyperréaliste, tel qu’il serait pratiquement impossible de différencier le vrai du faux, hormis peut-être un usage limité de la parole et une mobilité un peu raide, comme si le sujet avait un balai dans le cul l’empêchant de se déplacer en toute fluidité), garanti deux ans constructeur, en leasing pour celles et ceux qui aiment changer régulièrement de partenaire, ou achat comptant pour celles et ceux désireux de construire une relation sur le long terme, le modèle pouvant même être équipé d’un logiciel de vieillissement pour ne pas rester éternellement jeune pendant que l’autre vieillit comme un con (contrairement à celles et ceux qui préfèrent se taper des petits jeunes plutôt que des vieux rogatons à leur image). Pour ce qui est des gosses, il est également tout à fait possible de concevoir des copies numériques, permettant de jouir de tous les avantages d’une vie de famille épanouie (rires et cris d’enfants dans une grande maison joyeuse et lumineuse pleine de charme et dotée de tout le confort moderne, grandes tablées les dimanches et jours de fête, jeux à n’en plus finir, histoires pour s’endormir, éventuellement activités pédophiles pour les proches, ces dernières étant difficilement contestables en l’absence de législation morale ou psychologique appropriée, etc) sans avoir à en supporter les inconvénients (enfants malades, chiants, ingrats et mal élevés, capables de se faire la malle à l’improviste, accumuler les conneries à la puberté, voire se retourner violemment contre leurs parents dans les cas les plus extrêmes). On peut aussi, si on tient absolument à en avoir des vrais, se procurer assez aisément des orphelins et autres enfants des rues issus des pays pauvres, et parfois même les acheter (à des tarifs d’autant plus attractifs que le vendeur est aux abois) directement à leurs parents qui en ont des tonnes et ne savent plus quoi en foutre (la contraception n’est pas le fort de ces populations autochtones), d’autant qu’ils n’ont pas les moyens de les nourrir et qu’il est assez pénible pour une mère ou un père, même alcoolique au dernier degré, de voir ses enfants crever de faim sous ses yeux. Bref, c’est un modèle de vie autrement confortable, juste et humain qui se dessine sous nos yeux ébahis dans un horizon pas si lointain, et, serais-je tenté de dire si je ne craignais pas d’être mal interprété, que s’octroyer l’assurance d’une vie sexuelle et conjugale réussie mérite bien quelques coups de canif dans la couche d’ozone et l’apparition de plus en plus fréquente de quelques séismes dévastateurs aux quatre coins de la planète, y compris dans des zones jusqu’ici relativement épargnées comme Malibu, Pacific Palisades et les hauteurs de Beverly Hills. Qu’est-ce qui est préférable ? Éviter qu’une star hollywoodienne voie sa baraque à 100 millions de dollars partir en flammes ou assurer au plus grand nombre une vie sexuelle heureuse et épanouie ?

De toute façon, les stars ont les moyens de s’offrir des assurances à la hauteur, alors que la misère sexuelle n’est couverte par rien (le pauvre bougre hagard qui se branle vingt heures par jour au péril de sa vie, devient aveugle à force de s’esquinter les yeux sur son écran d’ordinateur, accumule les tendinites et finit avec la teub en sang et un rythme cardiaque considérablement dégradé, peut toujours se gratter les couilles pour obtenir le moindre dédommagement de quelque organisme que ce soit, y compris le sien qui n’en peut plus) et conduit le plus souvent à des drames d’une violence inimaginable (qui, en plus, donnent des idées à des scénaristes peu scrupuleux, de véritables rats qui épluchent les tabloïds à la recherche des faits divers les plus sordides) et permettent à ces mêmes stars hollywoodiennes de s’illustrer dans des reconstitutions criantes de vérité, rafler des Oscars à la pelle et engranger toujours plus de pognon pour se taper des escorts à vingt mille dollars la nuit, se faire construire des villas dans des endroits paradisiaques et faire leurs courses en jet privé. Donc oui, je crois, j’affirme, même, qu’il est grand temps de se préoccuper un peu du petit peuple qui souffre en silence dans des bidonvilles voués à la mort et la destruction, tas de tôles infects prenant l’eau de toute part (heureusement qu’il ne pleut pas souvent dans ces coins-là), se nourrissant des rats, cafards et autres punaises qui pullulent entre les matelas éventrés et les amoncellements de déchets en tout genre. Qu’ils aient, au moins, le plaisir de couler des jours paisibles avec une femme aimante et des enfants obéissants, dans un confort relatif, certes, rustique pour ne pas dire préhistorique, mais dans une béatitude sexuelle et affective de tous les instants, un état d’extase conjugale permanent.

La première chose que j’ai fait, en débarquant sur les lieux du drame avec Zaahid Shirani, légiste et néanmoins ami, a été de m’adresser au propriétaire des lieux, Robert Pleimelding, lequel nous attendait à la sortie du bois. L’endroit exact du sinistre n’étant pas facile à trouver, on avait jugé préférable de se donner rendez-vous en un point facilement accessible par voie de locomotion motorisée, en l’occurrence ma somptueuse Kangoo de dix ans d’âge (j’étais, dans un élan de générosité inhabituel, et aussi parce que je n’aime pas spécialement voyager seul, passé prendre Zaahid à la morgue).

\textsc{Moi} : Vous êtes qui, au juste ?

Je le savais déjà mais j’aime bien reposer la question pour faire chier le monde. Le métier de flic n’est pas toujours rigolo, on a les distractions qu’on peut.

\textsc{Robert Pleimelding} : Robert Pleimelding. Je suis le propriétaire des lieux.

\textsc{Moi} : Vous voulez dire que ce bois vous appartient ?

Là encore j’avais parfaitement compris, mais j’appliquais la méthode Columbo (l’imper cradingue, le basset pétomane et le strabisme en moins), qui consiste à se faire passer pour une bille afin d’endormir la méfiance du suspect.

\textsc{Lui} : C’est ça.

\textsc{Moi} : Et c’est vous qui avez découvert le corps ?

\textsc{Lui}, essayant de retenir Greta qui s’était prise de sympathie pour Zaahid et tenait absolument à lui renifler les bas de pantalon (ou alors c’était la vieille odeur de cadavre rance et produits chirurgicaux qui émanait de sa personne) : Oui, c’est moi. Greta, laisse le monsieur tranquille !

\textsc{Zaahid}, qui n’était pas très à l’aise avec les animaux, en particulier les chiens : C’est quoi, comme race ?

\textsc{Robert} : Un drahthaar.

\textsc{Moi} : Un quoi ?

\textsc{Lui} : Un drahthaar, mot allemand qui signifie «~avoir des cheveux en fil de fer~». Un croisement de griffon-kortals et de braque allemand à poil court, si vous préférez.

\textsc{Moi} : Je préfère.

\textsc{Zahhid} : Un très bel animal, en tout cas.

\textsc{Moi} : Ce serait mieux si vous arriviez à le tenir.

\textsc{Robert} : C’est le cas en général. Je ne sais pas pourquoi, mais elle semble très attirée par monsieur.

\textsc{Moi} : Oui, eh bien il se trouve que «~monsieur~» n’aime pas trop les chiens, raison pour laquelle je vous serais reconnaissant de tenir cet animal à distance.

\textsc{Lui} : Au pied, Greta ! Pas bouger !

Le chien est allé s’assoir aux pieds de son maître, tout en continuant à reluquer Zaahid avec la langue pendante et des étoiles plein les yeux.

\textsc{Moi} : Et ça, c’est quoi ?

\textsc{Lui} : Un fusil de chasse.

\textsc{Moi} : Oui, je vois bien que c’est un fusil de chasse. Je vais être obligé de vous le confisquer pour expertise. Vous ne voyez pas que ce soit l’arme du crime !

\textsc{Lui} : Je suis parfaitement en règle !

\textsc{Moi} : Je n’en doute pas. N’empêche qu’il va quand falloir que j’examine votre pétoire, des fois qu’on retrouve des plombs dans le périmètre. Vous avez quoi dans cette sacoche ?

Il a ouvert la sacoche et me l’a collée sous le nez : Voyez vous-même !

\textsc{Moi} : Hun hun. C’est quoi, ça, comme oiseau ?

\textsc{Lui} : C’est pas un oiseau, c’est un lièvre.

\textsc{Moi} : Ah oui, je suis bête ! Et ça ?

\textsc{Lui} : Vous n’en avez jamais vu ?

\textsc{Moi} : Excusez-moi, mais je ne passe pas vie à glandouiller dans la forêt. On dirait des champignons.

Naturellement, je savais parfaitement qu’il s’agissait de girolles. Là encore, je travaillais mon personnage d’idiot de service pour mettre le suspect en confiance et l’inciter à se trahir. Suspect qui n’avait d’ailleurs à priori rien de suspect, si ce n’est qu’il avait soi-disant découvert le corps sur un terrain lui appartenant. Je suis d’accord que ce n’était pas comme s’il l’avait trouvé dans son salon en rentrant chez lui après une dure journée de labeur. Les traces d’effraction qu’on n’aurait pas manqué de remarquer auraient joué en sa faveur (même s’il aurait très bien pu les fabriquer lui-même), et les relevés de scène de crime auraient été autrement plus aisés qu’en pleine forêt. Ici, en dépit des interdictions placardées un peu partout, n’importe qui pouvait introduire à sa guise, en particulier la nuit, et s’adonner en toute tranquillité à des activités peu recommandables. Cela dit, quand quelqu’un appelle les flics pour leur signaler un meurtre, il y a toujours ce vieux réflexe conditionné qui les incite à placer l’intéressé en tête de liste des suspects. On ne manque pas d’exemple où l’assassin, non content d’avoir lui-même attiré l’attention de la police sur l’événement, se montre exagérément coopératif auprès des forces de l’ordre, allant parfois jusqu’à mener sa propre enquête. Cela traduit non seulement une faille psychologique importante, sinon une «~catastrophe psychique~» telle que définie par Racamier dans son article intitulé «~Entre agonie psychique, déni psychotique et perversion narcissique~» publié en 1986 dans la Revue Française de Psychanalyse, mais aussi la volonté de coller au plus près de l’enquête pour tenter de la manipuler à son profit.

\textsc{Pleimelding} : Oui, ce sont des champignons. Des girolles, pour être plus précis. Vous n’allez pas me dire que vous n’avez jamais mangé de girolles ?

\textsc{Moi} : Je ne vous dis rien du tout. C’est vous qui allez me dire des choses, à commencer par l’endroit où se trouve le corps. C’est loin ?

\textsc{Lui} : Une trentaine de minutes, en marchant d’un bon pas.

\textsc{Zaahid}, jetant un œil énamouré sur ses mocassins Guido Cattani en alligator de Louisiane (élevage, en accord avec la convention de Washington sur le commerce international des espèces menacées d’extinction) à cinq mille balles la paire : Personne n’a une paire de bottes à me prêter ?

Perso, je n’avais pas ce genre de problème avec ma vieille paire de pompes columbesques dont la semelle collée par-dessus la jambe par des asiatiques sous-payés se décollait dangereusement sur les bords. Le seul risque, c’était qu’elles lâchent en cours de route et que je me retrouve obligé de gambader en chaussette au milieu des bois, chose qui peut s’avérer romanesque à première vue, mais surtout se révéler extrêmement préjudiciable pour l’intégrité physique de ses pieds.

J’ai dit, à l’intention de Pleimelding qui semblait trouver la situation très à son goût : Laissez votre fusil à l’agent Bescond, ici présent. Thomas, s’il te plaît, tu prends le fusil de monsieur et tu le ranges en lieu sûr. Désolé, monsieur Pleimelding, mais je ne peux pas laisser un civil se balader avec calibre 12 en pleine enquête policière.

\textsc{Pleimelding} : Et si on croise un lièvre ou une perdrix, on fait quoi ?

\textsc{Moi}, lui laissant entrevoir la crosse de mon arme de service rangée dans son holster : Ne vous en faites pas, j’ai ce qu’il faut.

\textsc{Lui} : Oui, sauf que vous n’avez aucune chance de toucher quoi que ce soit avec un engin dans ce genre.

\textsc{Moi} : Eh bien ce sera pour une autre fois, voilà tout. Au cas où vous l’auriez oublié, je vous rappelle qu’on est de la police, pas un groupe de touristes allemands en villégiature sur la Côte. On n’est pas venu chasser la perdrix mais voir ce qui s’est passé dans votre bois, savoir qui a tué qui, quand et comment, et surtout mettre la main sur le responsable de ce carnage. Veuillez remettre votre arme à l’agent Bescond, s’il vous plaît.

\textsc{Pleimelding} : Soit. Mais je vous préviens tout de suite que s’il lui arrive la moindre bricole, éraflure ou quoi que ce soit, je vous traîne devant les tribunaux !

\textsc{Moi} : Pourquoi, il n’est pas assuré ?

\textsc{Lui}, manifestement outré : Bien sûr que si, qu’il l’est ! Mais au cas où vous ne le sauriez pas, l’argent ne fait pas tout, dans la vie.

\textsc{Moi} : Non, je ne savais pas.

\textsc{Lui} : Certaines choses, même si elles sont dépourvues de valeur marchande~-- ce qui n’est pas le cas de ce fusil, je vous le dis tout de suite, peuvent avoir une grande valeur sentimentale. Ça peut être n’importe quoi : une pince à cheveux ayant appartenu à votre mère, les lunettes de votre père…

\textsc{Moi} : Le stérilet de votre grand-mère ou la vieille paire de pantoufles que portait votre grand-père pour fumer sa pipe au coin feu en lâchant des caisses interminables, oui, en effet, il me semble que j’ai déjà vaguement entendu parler de ça. Sinon, sans indiscrétion, ça vaut combien, un joujou de ce genre ?

\textsc{Lui} : Dans les trente mille euros.

\textsc{Moi} : Ah oui, quand même !

Puis, à l’agent Bescond qui n’était pas spécialement connu pour être un modèle de douceur et de délicatesse, ayant toujours préféré le football américain au patinage artistique et le bûcheronnage sportif à la natation synchronisée : Thomas, tu prends le plus grand soin du fusil de monsieur, c’est bien compris ?

\textsc{Bescond} : Oui, chef.

\textsc{Moi} : Veille bien à ce qu’il ne lui arrive rien. Dans le cas contraire, c’est toi qui paieras les pots cassés. Et assure-toi qu’il n’est pas chargé, je n’ai pas envie que quelqu’un se fasse sauter le caisson avec !

\textsc{Lui}, au garde-à-vous : Chef, oui chef !

\textsc{Moi} : Mon cul, je n’entends rien ! Montrez-moi que vous en avez une paire !

\textsc{Lui}, à tue-tête : Chef, oui chef !!!

\textsc{Moi} : Je suis vache, mais je suis réglo. Aucun sectarisme racial ici. Je n’ai rien contre les négros, ritals, rouquins ou métèques ! Est-ce que c’est clair ?

\textsc{Lui} : Chef, oui chef !

\textsc{Lui} : Dites, chef…

\textsc{Moi} : Quoi encore ?

\textsc{Lui} : Vous connaissez la différence entre un rappeur et un campeur ?

\textsc{Moi} : Non, et je m’en fous !

\textsc{Lui}, hilare : C’est pourtant simple, chef : le rappeur nique ta mère et le campeur monte ta tente !

\textsc{Moi} : On se tutoie, maintenant ?

\textsc{Lui} : Chef, non chef ! C’était juste pour…

\textsc{Moi} : Mon cul ! Rompez, agent Bescond. Serez de corvée de chiottes la semaine prochaine !

\textsc{Lui} : Chef, oui chef !

Bescond, non content d’être roux, gras du bide et globalement assez désagréable à regarder (notamment en raison de ce début de calvitie et ce teint rougeaud qui le faisait ressembler étrangement à un ouakari chauve des forêts marécageuses d’Amérique du Sud, un des singes les plus laids du monde), était d’un naturel volontiers frondeur. Ses plaisanteries vaseuses et autres blagues salaces ne faisaient rire personne, ou alors seulement quelques abrutis en phase terminale de décérébration dans son genre, mais je ne pouvais m’empêcher d’éprouver certaine tendresse pour lui, ne serait-ce que par égard pour les efforts considérables qu’il fournissait pour attirer l’attention sur lui. Son répertoire de blagues était impressionnant, surtout pour quelqu’un comme moi qui n’ai jamais été foutu d’en retenir une seule. Je me demandais bien où il allait fourrer tout ça dans sa cervelle de lilliputien. D’autre part, force était de reconnaître qu’il disposait d’une culture cinématographique hors du commun, totalement inattendue de la part d’un type dont le QI ne devait pas dépasser celui du raton laveur. Le cinéma, américain notamment, n’avait aucun secret pour lui, et il connaissait par cœur des répliques entières d’un nombre considérable de films (dont certains, comme Full Metal Jacket, sur lesquels je pouvais lui donner la réplique, étant moi-même assez friand de ce genre de performance). Aussi, au lieu de lui coller des blâmes à répétition pour insulte permanente envers les grands du rire et la vis comica, je me contentais de produire une expression faciale à équidistance entre la grimace et le sourire, au mieux de le remettre gentiment à sa place en le priant d’aller voir ailleurs si j’y étais, chose qu’il s’empressait de faire en courant.

Zaahid, dont les relations avec Greta semblaient être en bonne voie puisqu’il en était maintenant à lui tapoter prudemment (gentiment à la rigueur, amoureusement serait prématuré) le haut du crâne, chose qu’elle semblait apprécier au plus haut point (dès qu’il arrêtait, elle le sollicitait en le poussant du museau pour qu’il remette ça) : Il serait peut-être temps d’aller le voir, ce corps, vous ne croyez pas ?

Pour veiller au grain et s’assurer que personne ne viendrait fourrer son groin dans nos affaires, j’avais laissé deux hommes en faction, dont l’agent Bescond, mon ouakari favori. À voir la tête de cent pieds de long qu’il affichait, ce dernier n’était manifestement pas très content d’être tenu à l’écart des festivités. Il ne s’agissait en aucun cas d’une mesure de rétorsion, mais de précaution. En effet, la dernière fois qu’il avait vu un macchabée (on avait retrouvé un clodo les tripes à l’air dans une benne à ordures, je vous prie de croire que ça ne sentait pas le muguet), il avait tourné de l’œil et mis trois semaines à s’en remettre (il avait fallu l’assistance d’une psychologue particulièrement gironde pour qu’il sorte enfin de sa torpeur). J’estime qu’un chef digne de ce nom se doit de préserver la santé mentale de ses hommes. Celle de l’agent Bescond étant déjà sérieusement entamée de naissance, je me devais de le préserver d’autant plus des chocs traumatiques qui auraient pu le faire basculer définitivement dans la démence.

Après quoi, telle une joyeuse bande de scouts en goguette, notre fine équipe s’est courageusement engagée dans la forêt.

Celle-ci était dense et touffue, à la limite de la pénétrabilité.

Au bout de cinq minutes, j’étais (et nous l’étions tous) déjà complètement paumé.

Heureusement pour nous, notre guide, ex-garde-chasse du comte de Bellisle, connaissait le coin comme sa poche. Il empruntait des chemins de traverse avec une aisance déconcertante, tournant à gauche ou à droite comme s’il avait un GPS à la place du cerveau.

Zaahid et moi marchions devant avec lui, sans oublier Greta qui elle aussi connaissait parfaitement le chemin. Les chiens ont une méthode très efficace pour s’orienter : ils pissent tous les cinq ou dix mètres. Cette méthode, soit dit en passant, est beaucoup plus efficace que celle du Petit Poucet, pourtant le plus malin des sept fils du bûcheron. En effet, je ne sais pas si vous avez déjà essayé de semer des cailloux dans la forêt, mais je peux vous dire qu’il faudrait en semer une sacrée quantité (et je parle de cailloux de grosse taille, plus proches du rocher que du grain de sable) pour espérer retrouver son chemin. C’est là, si je puis me permettre, que le conte de Perrault manque un peu de réalisme, car il aurait fallu, pour que son plan ait une chance de fonctionner, que le Petit Poucet emmène avec lui une pleine remorque de briques. D’ailleurs, là où on voit qu’il n’est pas si malin que ça, c’est quand il tente de rééditer l’expérience avec des petits morceaux de pain. Faut être con pour laisser des petits morceaux de pain dans la forêt, alors que celle-ci regorge d’oiseaux et que tout le monde sait que les oiseaux adorent les petits morceaux de pain (pas seulement les oiseaux, du reste, mais aussi les fourmis et toutes sortes de bestioles dont on ne soupçonne même pas l’existence).

Ce con de Pleimelding avait parlé d’une demi-heure de route, mais compte tenu de la populace qu’il fallait éviter d’égarer à tous les coins de rue, ou de sentier, trois bons quarts d’heure nous ont été nécessaires pour arriver à destination. Inutile de préciser que Zaahid et moi-même, peu habitués à ce genre de pérégrinations champêtres, en avions plein les pattes. La PTS et les agents en uniforme, qui avaient tout donné pour garder le rythme infernal imposé par l’ex-garde-chasse, se trouvaient dans un état similaire, en sueur et perclus de crampes.

Le spectacle n’était pas joli à regarder.

J’ai dit à Zaahid, qui reluquait la scène de crime avec un intérêt certain : Si je te dis «~Tu vois, Julien, c’est ça la chevrotine~», tu me réponds quoi ?

\textsc{Zaahid} : Que je ne m’appelle pas Julien.

\textsc{Moi} : Logique.

\textsc{Lui} : Ouais.

\textsc{Moi} : Et à part ça ?

\textsc{Lui} : À part ça quoi ?

\textsc{Moi} : Cette phrase ne te dit rien ?

\textsc{Lui} : Rien du tout.

\textsc{Pleimelding} : Je suppose que vous voulez parler de Julien Dandieu, dans Le vieux fusil.

\textsc{Moi} : Exact.

\textsc{Lui} : Le petit Julien, âgé d’une dizaine d’années, est parti chasser avec son père.

\textsc{Moi} : Il s’agit d’un flashback, bien évidemment.

\textsc{Lui} : Bien évidemment. Soudain, ils se retrouvent nez à nez avec un sanglier…

\textsc{Moi} : Un énorme sanglier.

\textsc{Lui} : Un sanglier adulte, de taille normale.

\textsc{Moi} : Non, un sanglier énorme, sans doute un de ces vieux mâles acariâtres qui chargent tout ce qui bouge.

\textsc{Lui} : Ils ne sont pas spécialement acariâtres, c’est juste qu’ils ont la vue qui baisse.

\textsc{Moi} : Peut-être, n’empêche qu’ils chargent tout ce qui bouge.

\textsc{Lui} : Oui, bon, si vous voulez. Le petit Julien et son père, donc, se retrouvent nez à nez avec un énorme sanglier…

\textsc{Moi} : Qui, comme je le disais, se met aussitôt à les charger, la bave aux lèvres. Le père prend son fusil, épaule tranquillement, tire et stoppe net le sanglier qui s’écroule le groin dans les feuilles mortes. C’est mieux quand c’est moi qui raconte, vous ne trouvez pas ?

Grimace de Pleimelding qui lui aussi aimait bien raconter les histoires et n’aimait pas que quelqu’un dise que c’était mieux quand quelqu’un d’autre, lui en l’occurrence, le faisait à sa place.

\textsc{Zaahid} : Et c’est que là que le père dit à son fils : «~Tu vois, Julien, c’est ça la chevrotine~».

\textsc{Moi} : C’est ça.

\textsc{Pleimelding}, qui ratait rarement une occasion de ramener sa science : La chevrotine, je le rappelle, est une charge de plombs de gros diamètre destinée à la chasse au gros gibier, le chevreuil en particulier. C’est dévastateur à bout portant, mais dès que la distance augmente l’efficacité diminue. À une époque, pour éviter la dispersion, on liait la charge avec du fil en laiton. Dans un cas comme dans l’autre, les animaux mouraient rarement du premier coup. Il fallait les traquer et s’y reprendre à plusieurs fois pour les exterminer. Ce n’était plus de la chasse mais de la boucherie, raison pour laquelle la chevrotine a progressivement disparu au profit de cartouches à balle unique. Longtemps interdite, elle est de nouveau autorisée dans certains départements, comme la Corse, pour essayer de contenir la prolifération des sangliers. Il y en a partout et ils n’hésitent pas à s’introduire dans les habitation pour faire les poubelles et ravager les potagers. Pour le tir à courte distance, à moins de quinze mètres, la chevrotine est tout indiquée. Il s’agit de régions où tout le monde ou presque a un fusil de chasse. Non seulement la chevrotine est bon marché, mais il n’est pas nécessaire d’être un bon tireur pour faire mouche.

\textsc{Moi} : Bon, ça y est ?

\textsc{Lui} : Excusez-moi. Je voulais juste…

\textsc{Moi} : On n’en a rien à faire, de vos histoires de chevrotine !

\textsc{Lui} : C’est vous qui en avez parlé le premier.

\textsc{Moi} : Je demandais au docteur Shirani, ici présent et manifestement atterré par la nullité de la discussion, si la phrase «~Tu vois, Julien, c’est ça la chevrotine~» lui disait quelque chose. Je me fous de la Corse et des sangliers !

\textsc{Zaahid} : Ne t’énerve pas, Djef.

\textsc{Moi} : Je ne m’énerve pas. C’est juste que cet abruti commence à me taper sur le système à ramener sa fraise à tout bout de champ !

\textsc{Zaahid} : Laisse tomber, et dis-moi plutôt ce que tu voulais me dire avec ton histoire de chevrotine.

\textsc{Pleimelding} : Il voulait parler du \textit{Vieux fusil}, un film de Robert Enrico\nf{Robert Enrico (1931--2001), réalisateur français, auteur notamment du \textit{Vieux fusil} (1975), film sur la vengeance d'un chirurgien dont la femme a été massacrée par les SS en 1944. Cinq César, dont meilleur film, meilleur acteur (Noiret) et meilleur second rôle. \textit{Source :} \textup{fr.wikipedia.org/wiki/Robert\_Enrico}} avec Romy Schneider\nf{Romy Schneider (1938--1982), actrice austro-française, icône du cinéma européen des années 1960--1970. Elle incarne Clara Dandieu dans \textit{Le Vieux fusil} (1975) et reste célèbre pour la trilogie \textit{Sissi} et sa collaboration avec Claude Sautet. \textit{Source :} \textup{fr.wikipedia.org/wiki/Romy\_Schneider}} et Philippe Noiret\nf{Philippe Noiret (1930--2006), acteur français majeur, interprète de Julien Dandieu dans \textit{Le Vieux fusil} (1975, César du meilleur acteur). Également connu pour \textit{Cinema Paradiso} (1988) et \textit{Il Postino} (1994). \textit{Source :} \textup{fr.wikipedia.org/wiki/Philippe\_Noiret}}.

\textsc{Moi} : Vous allez la fermer, oui ou merde ? Prenez un peu de distance, s’il vous plaît. Il s’agit d’une enquête policière et vous n’avez rien à faire là.

\textsc{Lui} : Je croyais que j’étais le principal suspect.

\textsc{Moi} : Pour l’instant vous n’êtes rien du tout. Continuez comme ça et je vous colle les bracelets pour entrave à la justice.

\textsc{Zaahid} : Et alors, c’est quoi cette histoire de vieux fusil ?

\textsc{Moi} : Eh bien des années plus tard, en juin 44, Clara, la femme de Julien Dandieu, alias Romy Schneider et Philippe Noiret comme a eu la délicatesse de nous le rappeler notre hôte, est assassinée par Nazis en déroute, dans le genre Oradour-sur-Glane\nf{Oradour-sur-Glane, village de Haute-Vienne où, le 10~juin 1944, des soldats de la division SS \textit{Das Reich} massacrent 643~habitants, dont des femmes et des enfants enfermés dans l'église et brûlés vifs. Le village martyr est conservé en l'état et classé monument historique. \textit{Source :} \textup{fr.wikipedia.org/wiki/Massacre\_d\%27Oradour-sur-Glane}}.

\textsc{Zaahid} : C’est moche.

\textsc{Moi} : Très moche, oui, d’autant plus que Clara…

\textsc{Pleimelding}, qui avait manifestement décidé d’en finir avec la vie : La femme de Julien Dandieu, n’est pas seulement assassinée mais littéralement massacrée par les SS, qui non contents de la violer sauvagement, finissent par la carboniser au lance-flammes.

J’ai souvent envie de tuer des gens, je l’avoue, mais lui j’avais envie de le découper très lentement avec un couteau émoussé.

Deux agents glandaient dans les parages, que j’ai apostrophés en ces termes choisis : Vous deux, là, venez voir un peu ici.

J’aimerais dire qu’ils ont rappliqué au pas de gymnastique, mais en fait non : ils ont mis trois plombes pour arriver en traînant la patte comme des grands blessés.

\textsc{Moi} : Vous voyez ce monsieur et son animal de compagnie ?

\textsc{Eux} : Oui.

\textsc{Moi} : Vous me les éjectez de la scène de crime, et au trot ! Je ne veux pas les revoir dans le périmètre jusqu’à nouvel ordre, sinon je ne réponds plus de rien !

Aussitôt dit aussitôt fait, les flics ont embarqué Greta et son maître, et j’ai enfin pu continuer à vivre ma vie sans personne pour me chier dans les bottes.

\textsc{Moi}, à Zaahid qui se trouvait présentement à quatre pattes en train de renifler le cadavre : C’est là où je voulais en venir, justement.

\textsc{Lui} : Au lance-flammes ?

\textsc{Moi} : Oui. J’ai déjà vu des gens auxquels on avait foutu le feu, mais ça ne ressemblait pas vraiment ça. T’en penses quoi ?

\textsc{Lui} : Faut voir. Ce qui est certain, c’est que les traces de crémation ne sont pas les mêmes selon la méthodologie adoptée. Si on arrose un cadavre d’essence pour y mettre le feu, il y a de fortes chances qu’on trouve des traces de crémation en dehors de la sphère anatomique stricto sensu.

\textsc{Moi} : Ce qui veut dire, en français ?

\textsc{Lui} : Qu’il y aura des projections de combustible aux alentours, alors que ce ne sera pas le cas si on utilise un lance-flammes. Les choses seront aussi très différentes selon que la personne est brûlée vive ou non.

\textsc{Moi} : C'est-à-dire ?

\textsc{Lui} : Un mort ne bouge pas, alors qu’un vivant aura tendance à se débattre, même s’il est solidement attaché. En conséquence, les traces seront différentes. Et si les poumons sont encore utilisables, ce qui ne semble pas vraiment être le cas ici, on trouvera des traces de suie à l’intérieur.

\textsc{Moi} : Et là, qu’est-ce que tu dirais, comme ça, à vue de nez ?

\textsc{Lui} : Eh bien, après un rapide examen des lieux, j’aurais tendance à dire que la personne était en vie et se tenait debout quand elle est passée à la rôtissoire.

\textsc{Moi} : Putain, comme Romy Schneider dans Le vieux fusil !

\textsc{Lui} : Si tu le dis.

\textsc{Moi} : Sauf que cette fois c’est pas du cinéma !

\textsc{Lui} : Ne t’emballe pas, c’est juste une première impression.

\textsc{Moi} : Et si c’est comme dans Le vieux fusil, peut-être qu’il faut se pencher sur la piste d’un sympathisant d’extrême-droite, un nostalgique du IIIe Reich, de la Schutzstaffel, Das Reich et la division Totenkopf, membre d’un de ces quelconques groupuscules néo-nazis qui poussent comme des champignons vénéneux sur le terreau de la connerie humaine ! Peut-être qu’il a un portrait de Hitler dans sa chambre, fréquente les sites de militaria sur le dark web et écoute des chants nazis dans son sous-sol de sa maison transformé en QG de la Kommandantur. Et dans ce cas, la victime était peut-être juive, communiste, gay, handicapée mentale, ou les quatre à la fois !

\textsc{Lui} : Oui, enfin, je trouve quand même que tu fais une fixette sur les Nazis. Ils ne sont pas les seuls à avoir utilisé le lance-flammes, arme de guerre aujourd’hui passée de mode mais très répandue à l’époque. Les Américains, par exemple, s’en sont servi en Corée et au Vietnam.

\textsc{Moi} : N’empêche que si ça s’est vraiment passé comme je le pense, on aurait tout intérêt à aller fouiner un peu du côté de l’extrême-droite. L’autre jour, j’ai vu un film de Steve Olson avec Logan Price, Finn Bowman et Savannah Ramirez. Dans un futur proche, à Brooklyn, des vampires nazis s’attaquent aux Juifs pour les vider de leur sang une bonne fois pour toutes. Sauf que les Juifs ultra-orthodoxes Borough Park ont retenu la leçon de 45 et ne sont pas du genre à se laisser génocider sans réagir. Or, aux États-Unis, terre promise de la liberté de vendre chèrement sa peau au plus offrant, il se trouve que le lance-flammes est en vente libre, comme d’ailleurs à peu près tout ce qui peut servir à se débarrasser efficacement de son prochain, à part les armes chimiques et les mines antipersonnel, en principe réservées à un usage strictement militaire. Même Elon Musk\nf{Elon Musk (né en 1971), entrepreneur sud-africano-américain, fondateur de SpaceX, Tesla et co-fondateur de PayPal. En 2018, sa société The Boring Company a commercialisé environ 20~000~unités d'un lance-flammes surnommé «~Not a Flamethrower~» à 500~dollars pièce. \textit{Source :} \textup{fr.wikipedia.org/wiki/Elon\_Musk}}, en son temps, vendait des lance-flammes pour le compte de la Boring Company, sa société de construction de tunnels. Il s’agissait plus de briquets géants que de véritables armes de guerre, mais tout de même on était assez proche de l’esprit néo-nazi high-tech dont il semble aujourd’hui se revendiquer, sachant qu’il ferait un SS tout à fait convaincant avec un uniforme sur le dos. Je me souviens qu’à l’époque, la rumeur circulait qu’il préparait une attaque massive de zombies pour booster les ventes de son gadget crématoire. Donc, les Juifs ultra-orthodoxes de Brooklyn se précipitent dans les armureries et font une razzia sur les lance-flammes, la seule arme véritablement efficace pour renvoyer les vampires en enfer. La plupart des vampires sont détruits, mais leur chef, l’ignoble Frozzan, alias le Porteur de la Mort, échappe au carnage et tombe amoureux de la belle Rachel, une jeune juive sublime aux yeux verts qui est le portrait craché de sa femme, la somptueuse Katharina, morte dans des circonstances on ne peut plus dramatiques. Nazi de la première heure et membre de la garde rapprochée du Führer, Frozzan, de son vrai nom Armin Böhmer, assiste en direct à la mort du dictateur sanguinaire, antisémite, raciste et homophobe, et sort très éprouvé de cette terrible épreuve. Il se débarrasse ensuite de son uniforme, enfile des vêtements civils empruntés à Hitler qui n’en a plus besoin et fait sensiblement la même taille que lui, et fonce retrouver sa femme dans leur appartement de la Friedrichstrasse, se frayant tant bien que mal un passage au milieu des cadavres et des gravats. Mais quand il arrive, il trouve la porte défoncée et l’appartement dévasté. Le corps de Katharina, entièrement dénudé et couvert de contusions horribles, repose au milieu du salon, sans vie. Il ne fait aucun doute qu’elle a été sauvagement violée et assassinée par les soudards de l’Armée rouge, horde barbare dont on connaît la violence extrême et l’absence totale de moralité. Fou de rage et de douleur, Böhmer renie Dieu, allant même jusqu’à uriner sur un crucifix jeté au sol dans la tourmente, et jure de revenir d’entre les morts pour se venger et poursuivre l’œuvre du Führer.

\textsc{Zaahid} : Vraiment très intéressant.

\textsc{Moi}, sachant pertinemment qu’il était en train de se foutre de ma gueule : Ouais, et encore plus quand on sait que c’est Logan Price et Savannah Ramirez qui jouent Frozzan et Rachel. Tu connais Savannah Ramirez ?

\textsc{Zaahid}, examinant la scène de crime à travers une loupe à très fort grossissement, d’une voix de fausset censée exprimer toute l’étendue de son désintérêt pour la question, en plus d’une ironie cruelle à mon encontre : Non, je n’ai pas ce plaisir.

Il y a des gens qui vont très souvent au cinéma, d’autres une fois de temps en temps, d’autres jamais. Zaahid appartenait à la troisième catégorie, laquelle regroupe des gens qui, sans être à proprement parler des hégéliens pur sucre de betterave, estiment toutefois que le cinéma, au même titre que d’autres disciplines telles que le macramé, la poterie, l’origami, Photoshop, les châteaux de sable, la cuisine de bonne femme, l’opérette, la charcuterie fine et la natation synchronisée (pour n’en citer que quelques-unes parmi les plus populaires), n’appartiennent pas à la catégorie des beaux-arts au sens noble du terme. Zaahid, par exemple, n’avait même jugé utile de se pourvoir d’un poste de télévision. Il considérait ce genre de lucarne sur le monde comme une calamité pour la paix intérieure et la connaissance profonde de soi, seul véritable chemin d’accès à la connaissance d’autrui et l’espérance d’une vie harmonieuse en société. Pour beaucoup de gens, aux ambitions plus modestes, Zaahid n’était qu’un crétin prétentieux, un de ces intellectuels autoproclamés et suffisants qui traitent les autres comme de la merde et considèrent qu’il n’y a aucune voie salutaire en dehors de celle qu’ils ont choisi de suivre.

\textsc{Moi} : Tu as tort, c’est une très belle femme.

\textsc{Lui} : Il y en a plein les rues, des très belles femmes. Et je les préfère en chair et en os.

\textsc{Moi} : Donc tu te fiches complètement de ce que je te raconte.

\textsc{Lui} : On peut dire ça.

\textsc{Moi} : Et tu ne veux surtout pas connaître la fin du Vampire de Borough Park.

\textsc{Lui}, évoluant comme un poisson dans l’eau au milieu des agents de la PTS affairés à récolter des indices : Au cas où tu ne l’aurais pas remarqué, je suis un peu occupé.

\textsc{Moi} : À ne surtout pas confondre avec Un vampire à Brooklyn de Wes Craven, avec Eddie Murphy dans le rôle d’un vampire caribéen qui débarque à Brooklyn pour assurer sa descendance. Mais je suppose que tu n’as jamais entendu parler de Wes Craven.

\textsc{Lui} : Bingo !

\textsc{Moi} : Ni de Beverly Hills Vamp, un navet qui met aux prises trois crétins patentés avec des putes vampires dans un bordel de la ville.

\textsc{Lui} : Ça me ferait mal !

\textsc{Moi} : Oui, eh bien quoi qu’il en soit, la fin du Vampire de Borough Park est assez inattendue, pour ne pas dire totalement inhabituelle. Comme je te l’ai dit, Frozzan et Rachel tombent amoureux l’un de l’autre, et il décide de faire d’elle sa compagne pour l’éternité, perspective que l’élue de son cœur accueille avec une joie sans limite mêlée quand même d’un soupçon d’appréhension. D’autant qu’il leur faut en permanence déjouer les pièges d’Obadiah McClelland, le chasseur de vampires de service, interprété par un Finn Bowman en grande forme, à mon humble avis bien meilleur que dans Les Barons de l’Aube de Trevor Wilson ou La Forêt maudite de Cameron Delgado, même si ça reste un acteur de série B.

Je me suis arrêté un instant de parler afin de mesurer l’impact de mes propos (j’estimais avoir fait un étalage assez encyclopédique de mes connaissances sur le sujet) sur l’homme qui se disait mon ami et que je considérais comme tel, mais, à en juger par l’attention qu’il me portait, cet impact n’excédait pas celui d’un moucheron percuté de plein fouet par le pare-brise d’un véhicule à l’arrêt.

Un constat d’une grande tristesse pour moi, mais il en fallait heureusement (pour vous surtout, qui rêvez de connaître la fin de l’histoire) un peu plus pour me faire taire.

C’est donc comme si de rien n’était que j’ai repris le cours de mon récit, faisant fi de l’indifférence de mon ami et néanmoins légiste fort peu cinéphile, présentement occupé à exhorter ses troupes (les agents de la PTS, ndlr) à récolter un maximum d’indices sur et autour du cadavre calciné : Bref, je te passe les détails, mais, pendant une nuit d’amour torride, Frozzan plante ses crocs acérés dans la gorge frémissante de Rachel qui exulte de bonheur. Je précise que cet enfoiré de McClelland, adepte de pratiques sexuelles déviantes, assiste à une bonne partie de la scène en regardant à travers le trou de la serrure de la chambre nuptiale, et que c’est seulement quand il a satisfait ses coupables penchants qu’il se décide à enfoncer la porte. Oui, je ne te l’ai peut-être pas dit, mais il y a une bonne dose d’érotisme dans ce film. Par exemple, quand Frozzan plante ses crocs dans la gorge de Rachel, elle est entièrement nue et il en profite pour lui malaxer copieusement la poitrine et la pénétrer sauvagement avec sa vieille queue toute rongée par les asticots qui manque de tomber en miettes.

Regard appuyé en direction de Zaahid pour mesurer l’effet produit par mes déclarations, lequel Zaahid, sentant le poids de mon regard sur sa nuque indifférente, tourne la tête et pose sur moi un œil rendu quasi vitreux par l’ennui et la condescendance. Je pensais qu’il allait au moins dire un mot, me faire part sans ambages de ses sentiments sur l’univers passionnant que je m’efforçais patiemment de lui faire entrevoir, mais sa tête est repartie dans l’autre sens et retournée à ses occupations sans prononcer la moindre syllabe, à tel point que j’ai, l’espace d’un instant, douté de l’opportunité de poursuivre mon récit.

\textsc{Moi} : Donc, comme je te le disais, McClelland enfonce la porte et se rue dans la chambre, le lance-flammes à la main, le visage atrocement déformé par la haine et la concupiscence, bientôt rejoint par un groupe de Juifs de Borough Park parmi les plus remontés, eux aussi armés de lance-flammes et bien décidés à réduire Frozzan en cendres. Mais en découvrant Rachel dans ses bras, entièrement nue, ils marquent un temps d’arrêt bien compréhensible. Du coup, Frozzan et Rachel, qui est devenue elle aussi un vampire, en profitent pour se transformer en chauves-souris et s’enfuir à tire-d’aile par la fenêtre de la chambre.

\textsc{Zaahid}, qui était d’ordinaire quelqu’un de plutôt calme et réservé : Tu commences à m’emmerder, avec tes histoires de chauves-souris !

\textsc{Moi} : Ah bon ?

\textsc{Lui} : Oui.

\textsc{Moi} : C’était juste pour détendre l’atmosphère.

\textsc{Lui} : Eh bien va la détendre ailleurs. Je comprends que tu sois stressé, ce n’est pas tous les jours qu’on assiste à un spectacle de ce genre, mais j’ai besoin de calme pour travailler.

\textsc{Moi} : Excuse-moi, je pensais bien faire.

Je mentais, bien sûr, je n’en avais rien à cirer de faire bien ou mal, et la vue du cadavre n’exerçait sur moi aucune pression particulière. C’est juste que je m’emmerdais et cherchais un moyen de faire passer le temps. Mais je voyais bien que Zaahid était concentré sur sa tâche, et je pouvais aisément comprendre que le bruit de fond incessant de mes litanies vampiriques pouvait lui taper sur le système, comme d’ailleurs celui de toute personne normalement constituée qui aurait dû les supporter sans piper mot.

Du coup, je me suis rabattu sur Pleimelding, que je savais cinéphile, et l’ai trouvé en pleine conversation avec le flic chargé de le surveiller. Quand je dis «~conversation~», il s’agissait plutôt d’un long monologue évoquant par le menu quelques-unes de ses plus mémorables parties de chasse. Le malheureux fonctionnaire, réduit à l’impuissance par le flot ininterrompu de paroles qui s’abattait sur lui, ne pouvait pas en placer une, et j’ai lu dans son regard éploré qu’il espérait que ma venue allait enfin mettre un terme à son calvaire. Au lieu de ça, il a dû rapidement se faire à l’évidence que ce qu’il venait d’endurer n’était qu’une pâle mise en bouche par rapport au plat de résistance que je m’apprêtais à lui servir.

Pleimelding, que j’ai coupé sèchement alors qu’il rabattait (il revivait la scène avec une intensité confondante) un douze cors vers son employeur, le comte Léopold Chiasson de Bellisle, propriétaire du château du même nom et de la majeure partie des terres environnantes, y compris la forêt où nous nous trouvions en ce moment même, s’est au contraire montré très intéressé par mon histoire de guerre des gangs entre vampires nazis et Juifs orthodoxes de Brooklyn. Poli, je lui ai d’abord demandé s’il s’intéressait au cinéma. Il m’a répondu que oui, preuve en était sa parfaite connaissance du Vieux fusil de Robert Enrico. Je lui ai fait remarquer qu’il s’entendrait bien avec l’agent Bescond, le rouquin gras du bide que j’avais laissé en faction à l’orée du bois et chargé de veiller à ce que son précieux fusil ne s’évapore pas dans la nature. Ce à quoi il a rétorqué que si tout le monde connaissait Jean Bouise, Romy Schneider et Philippe Noiret, tous d’excellents acteurs au demeurant, ces mêmes soi-disant cinéphiles se fichaient comme d’une guigne de savoir qui était Christian Teyras, second rôle à la notoriété confidentielle, certes, mais au talent bien réel. Car c’est lui, n’en déplaise à ces pisse-froid de la pellicule, qui joue le père de Julien Dandieu et prononce la phrase restée dans toutes les mémoires ou presque : «~Tu vois, Julien, c’est ça la chevrotine~». Je l’ai félicité, et avant qu’il ait eu le temps de me donner des détails sur la vie et l’œuvre de Christian Teyras (dont la carrière se résumait à une poignée de films tombés dans l’oubli), j’ai enchaîné avec mon histoire de guerre des gangs new-yorkais, avec d’un côté les vampires nazis de l’ancien SS Armin Böhmer, alias Frozzan, le Porteur de la Mort, et de l’autre les Juifs orthodoxes de Borough Park conduits par Zachariah Matusevitch, le chef de la communauté, assisté dans cette dure épreuve par Obadiah McClelland, spécialiste du paranormal, expert en sciences occultes et chasseur de vampires unanimement respecté.

Depuis que mon ami Grégoire Lussier, ancien analyste M\&A chez Reckless \& Knot (un scandale financier retentissant avait contraint la banque à fermer la majeure partie de ses succursales avant d’aller se refaire une santé aux Bahamas), avait ouvert son cabinet de détectives privés, les affaires marchaient plutôt bien.

Il s’était, pour faire plus sérieux (et accessoirement se sentir moins seul), adjoint les services de deux hommes de valeur, eux aussi anciens de chez Reckless \& Knot, qui partageaient la même passion que lui pour les affaires douteuses, les histoires glauques et les polars old school : Romuald Pueyrredon, dit Le Grand Acquisiteur (rapport à la fusion-acquisition, la synergie de croissance et les magouilles financières dont il maîtrisait parfaitement l’architecture sophistiquée), et Mathéo Baleya, surnommé La Balayette car il n’avait pas son pareil pour faire le ménage sur le marché, en d’autres termes se débarrasser de la concurrence.

Rien de plus normal, quand on est curieux de nature et un peu voyeur sur les bords, d’aimer fourrer son nez dans les affaires des autres, remuer la merde pour faire remonter les mauvaises odeurs, mais il arrive parfois, pour des raisons que j’ignore et sur lesquelles je continue à m’interroger avec toute la puissance cérébrale qui me caractérise, que certains, sans doute plus sanguins et moins ouverts d’esprit que d’autres, dans l’impossibilité de prendre du recul et contrôler leurs émotions, en prennent ombrage au point de se livrer à des gestes regrettables. C’était, par exemple, le cas de Yiorgos Panayiotou, qui prenait facilement ombrage de tout et n’importe quoi, y compris les choses les plus minimes, et avait tendance à accumuler des gestes dont le niveau de regrettabilité se situait toujours très au-dessus de la moyenne. En effet, non content de s’adonner au trafic de stupéfiants et autres activités peu recommandables, ce ressortissant chypriote affichait un penchant maladif pour les femmes des autres, lesquels autres n’avaient pas toujours les ressources nécessaires pour prendre la chose avec philosophie. Beau parleur, séducteur, toujours propre sur lui, bien mis de sa personne, le sourire facile, il emballait tout ce qui bouge à la vitesse d’une araignée. En quelques heures, la victime se retrouvait solidement emmaillotée dans un cocon affectif duquel il lui était impossible de s’extraire. C’est dans le cadre d’une enquête sur ce Yiorgos Panayiotou que Greg avait fait l’objet d’une blessure par balle, tirée par ce même Yiorgos, par chance aussi mauvais tireur que citoyen. La balle, en dépit des allégations de Greg qui prétendait avoir vu la Mort en face, croisé son regard de glace dissimulé sous un grand capuchon noir, n’avait fait qu’effleurer le bras, entraînant une ITT inférieure à huit jours.

Ce jour-là, à quinze heures, Greg avait rendez-vous avec une certaine Sally Robinson.

Qui était cette Sally Robinson ?

Aucune idée.

Pourquoi avait-il rendez-vous avec elle ?

Parce qu’elle l’avait appelé et qu’elle avait insisté pour prendre rendez-vous.

Pourquoi ?

Il s’agissait apparemment d’une affaire urgente de la plus extrême gravité.

Et il avait accepté ?

Oui, il avait fini par accepter de la rencontrer, même si ça ne l’arrangeait pas car il avait déjà assez de boulot comme ça, surtout que le boulot et lui ça faisait deux et qu’il s’en tenait généralement au strict nécessaire pour ne pas crever de faim.

Et qu’est-ce qu’il bouffait généralement ?

Caviar, foie gras, truffe, des choses de ce genre, copieusement arrosées des meilleurs vins et alcools.

Pas donné, donc.

Non, pas donné, raison pour laquelle il devait quand même bosser un minimum pour assurer ce train de vie de ministre.

Et pour laquelle il avait accepté de recevoir cette Sally Robinson, même s’il trouvait ce nom bizarre et n’avait à priori aucune envie de se retrouver assis face à elle dans son bureau.

En effet.

Est-ce qu’elle était à l’heure, au moins ?

Oui, avec un sens de l’exactitude quasi helvétique, et je ne vous cache pas que c’était là un excellent point pour elle, car même si le détective a l’habitude de passer des heures à poireauter, et pas toujours dans les meilleures conditions (souvent sous la pluie avec de fortes rafales de vent, pas tellement parce qu’il ne pourrait pas trouver un endroit pour attendre au sec, mais parce que la tension dramatique est d’autant plus intense que les conditions météorologiques sont désastreuses, que les éléments eux-mêmes semblent se déchaîner au rythme de l’action et traduire par leur mauvaise humeur les tourments intérieurs du héros et le destin tragique qui lui pend au nez), Greg avait horreur des gens qui ne respectent pas scrupuleusement les horaires établis. Lui-même était toujours excessivement à l’heure, à la seconde près. Il ne le faisait pas spécialement par respect pour l’autre, dont il n’avait la plupart du temps pas grand-chose à secouer, mais parce que l’espèce d’angoisse sourde et obsédante qui l’habitait en permanence lui interdisait formellement de prendre la moindre liberté avec une chose aussi fluide, insaisissable et implacable que le temps. Le temps ne se voit pas mais ses effets sont bien réels. Ils se nomment décrépitude, mort, décomposition, disparition.

Heureusement, Sally Robinson était à l’heure et Greg a pu échapper au supplice infernal des idées noires qui, dans le cas où elle serait arrivée avec ne serait-ce que deux ou trois minutes de retard, n’auraient pas manqué de tourner dans son crâne tel un essaim de mouches à merde autour d’une bouse fraîchement pondue. Naturellement, il n’aurait pas fallu non plus qu’elle arrive en avance, ce qui aurait, de fait, placé Greg dans une situation de retardataire d’une flagrante injustice. Toute sa construction mentale de la journée en cours en aurait été irrémédiablement perturbée, mise à mal, et finalement détruite, tant il est impossible, même en courant vite, de rattraper la moindre seconde de temps perdu. Tout comme il est impossible d’en gagner, du reste, d’où la précision extrême que l’on se doit d’adopter si on tient à rester dans le jeu (de dupes, bien évidemment, car on finit toujours par perdre, quels que soient les trésor d’ingéniosité que l’on développe pour s’en sortir).

Sally Robinson n’était pas à proprement parler ce qu’il est convenu d’appeler une belle femme, au grand dam de Greg qui s’attendait à voir débarquer Rhonda Fleming, Lana Turner, Gene Tierney, Lauren Bacall ou Sharon Stone.

Ne vous en déplaise, Sally Robinson n’avait rien de commun avec ces femmes fatales qui ondulent dans les romans de Chandler, Hammett, Goodis et les autres, chantent comme des sirènes et entraînent les marins d’eau douce dans les profondeurs glacées du désespoir et la décrépitude la plus totale, tant physique que morale. Si tous les gens qui croisaient le chemin de Sally Robinson avaient envie de l’éradiquer, à commencer par les femmes dont elle dégradait sérieusement l’image, c’est parce que la société se présente sous la forme d’un organisme savamment structuré et se comporte comme tel, à savoir qu’il cherche à éliminer ou expulser tout corps étranger dont il détecte la présence en son sein, expurger sans ménagement de la surface de son épiderme satiné toute excroissance disgracieuse ou furoncle gorgé de pus. Et je vous prie de croire que la présence de Sally Robinson avait été détectée, depuis longtemps, et que tous les anticorps disponibles avaient été mobilisés pour tenter de venir à bout d’un des agents pathogènes les plus redoutables de toute l’histoire de l’humanité, au moins sur le plan de l’esthétique et du bon goût, de la joie de vivre et la culture générale.

Vous vous dites : bon sang, mais où va-t-il chercher tout ça !

On a vu des gens moches, certes, mais quand même pas au point de se poser sérieusement la question de savoir si le seul fait de leur présence sur terre représente une réelle menace pour la survie de l’espèce qui nous tient le plus particulièrement à cœur : la nôtre.

Oui, eh bien si vous pensez vraiment que je pousse le bouchon un peu loin, je vous suggère de procéder à l’expérience suivante : essayez d’imaginer Danny DeVito avec des oreilles de Mickey et les nichons de Pamela Anderson.

Si vous y parvenez (ce qui en soi représente déjà un exploit non négligeable), dites-moi franchement quelle est la conclusion qui vous vient le plus spontanément à l’esprit.

Alors ? Vous êtes d’accord avec moi qu’un tel niveau de laideur est difficilement acceptable dans le monde aseptisé qui le nôtre, et pourrait, si on le laissait aller et venir en toute liberté, représenter une source de traumatisme irréversible pour les plus fragiles d’entre nous, catégorie à laquelle, heureusement pour lui, Greg n’appartenait absolument pas, même si son premier réflexe avait été de prendre ses jambes à son cou et sauter dans le premier avion en partance pour nulle part.

Il s’est dit : pas de panique, mon garçon, on est au vingt-et-unième siècle ! Ce n’est pas le siècle des lumières, plutôt celui de l’extinction des feux, mais un vrai professionnel se doit de garder la tête froide en toute circonstance.

Aujourd’hui on change de sexe comme de chemise, ou de capote, et les enfants pourront bientôt porter plainte contre leurs parents pour les avoir fait naître avec des attributs inappropriés, obtenir une rente à vie en réparation des souffrances psychologiques endurées et passer le restant de leurs jours à se dorer la pilule sur une plage de sable fin. Depuis le temps qu’on raconte que les enfants sont des petites choses fragiles qu’il faut manier avec d’infinies précaution, il fallait bien s’attendre à ce qu’ils nous reprochent jusqu’à la taille de leur sexe ou la couleur de leurs yeux. Dieu merci, la science peut réparer quelques erreurs tragiques de ce genre, mais il s’agit la plupart du temps d’un aller simple et le résultat n’est pas garanti sur facture. L’être humain, je l’ai déjà dit, livre un combat sans merci contre la nature, et ne se gêne plus pour remettre ses décisions en cause et corriger le tir en cas de besoin. Sa volonté de puissance est telle qu’il ne tolère plus que qui ou quoi que ce soit décide à sa place de ce qui est bon ou pas pour lui. La nature, qui jusqu’ici faisait autorité, est aujourd’hui contestée sans ménagement. Contrairement à Dieu, qui par essence ne fait pas d’erreur et ne peut donc faire l’objet d’aucune remontrance, la nature, qui est en quelque sorte sa version laïque, encaisse son lot de critiques et agressions caractérisées. Il faut vraiment qu’elle se fâche tout rouge pour que l’être humain commence à se poser des questions sur le bien-fondé de ses agissements, entrevoir l’idée que quoi qu’il fasse, quel que soit le sentiment de supériorité et l’autosatisfaction qui gonfle sa poitrine d’orgueil, il aura toujours une longueur de retard face aux forces telluriques qui lui ont, dans un moment d’aberration, de bug cosmique, donné le jour.

Nul doute que pour certains, Sally Robinson représentait une forme d’évolution ultime de l’être humain, de prototype d’un nouveau genre, une avancée significative sur la voie de la perfection et la guérison de tous les maux.

Pour Greg, elle (ou il, tant il était difficile de qualifier avec certitude cette créature mutante, fruit pourri tombé de l’arbre de la folie) représentait surtout une potentielle source de revenu qui, au même titre que l’investissement locatif, l’hypnothérapie et les cryptomonnaies, méritait d’être considérée avec tout le respect dû à son rang. Certes, il était assez compliqué, en voyant entrer ce modèle réduit (sa taille ne devait pas excéder le mètre cinquante) surmonté de la tête d’un homme de quarante-cinq ans, laquelle tête affichait, outre une calvitie déjà bien avancée, un double menton prononcé, des oreilles de Mickey, des sourcils épilés et une bouche tartinée de lipstick rose et satiné (intense et hydratant, parfait pour un look tendre et décontracté de tous les jours, les parties de pêche en mer, la cueillette des champignons et les soirées karaoké, alors que les rouges crémeux, plus mats et parfois collant au point de ne plus pouvoir décoller le bec de l’endroit où il se pose, sont à réserver pour les grandes occasions telles que mariage, barmitsva, première communion, enterrement de vie de garçon, proche décédé ou toute autre chose susceptible d’être enterrée, etc), il était assez compliqué, disais-je, en voyant entrer cette créature improbable et mal fagotée tout droit sortie de Queer Eye for the Straight Guy, de ne pas éclater de rire et se rouler par terre en se tordant douloureusement les côtes, surtout si on avait le malheur de poser les yeux sur les boucles d’oreille en forme de crucifix ou l’énorme paire de nichons que Sally Robinson trimballait devant elle avec une fierté manifeste. À ce stade, on frisait la provocation pure et simple.

Sally Robinson, drapée de tissu rose cochon en forme de costard trois fois trop grand et nimbée d’un parfum de tubéreuse à décimer un régiment de sapeurs-pompiers : Bonjour. Je suis Sally Robinson.

«~En v’là une nouvelle qu’elle est bonne~» s’est dit Greg en lui tendant la main : Grégoire Lussier. Entrez, je vous en prie.

La chose est entrée, et son parfum, plus efficace qu’une bonne rasade d’insecticide surpuissant, a laissé une traînée de mouches mortes dans son sillage.

Elle se déplaçait, avec une élégance toute relative, sur des chaussures à talons sensiblement de la même couleur charcutière que son costume ridicule.

\textsc{Greg}, souriant de ses plus belles dents, lui désignant un des deux fauteuils réservés à la clientèle : Asseyez-vous, je vous en prie.

\textsc{Sally}, prenant place dans le fauteuil en tortillant du cul tel un cycliste en pleine ascension du col du Galibier : Je vous remercie.

«~Mais de rien, ma bonne dame~» s’est dit Greg en allant s’assoir en face d’elle, devant le bureau sur lequel s’entassaient une foule de dossiers multicolores, vides pour la plupart, mais donnant l’impression qu’il passait ses journées et une bonne partie de ses nuits à faire don des plus belles années de sa vie à une clientèle toujours plus nombreuse et exigeante : Bien, madame Rob….

\textsc{Sally}, passant une main soigneusement manucurée sur la maigre touffe de cheveux qui garnissait encore (mais sans doute pas pour longtemps) l’arrière de son crâne : Mademoiselle.

\textsc{Greg}, exhibant à nouveau les plus beaux fleurons de sa dentition : Mademoiselle Robinson, pardon. Puis-je savoir ce qui vous amène ?

\textsc{Sally} : Eh bien voilà : j’ai un ami qui a disparu.

\textsc{Greg} : Un ami ?

\textsc{Sally} : Oui, un ami.

\textsc{Greg} : Quel genre d’ami ?

\textsc{Sally} : Un ami proche, suffisamment pour que je m’inquiète de sa disparition.

\textsc{Greg} : Je vois. Et je suppose que vous voudriez que je le retrouve ?

\textsc{Sally} : C’est ça.

\textsc{Greg} : Depuis combien de temps a-t-il disparu ?

\textsc{Sally}, des sanglots dans la voix, une voix de fausset qui sonnait comme celle d’un enfant dans le corps d’un adulte : Trois semaines jour pour jour.

\textsc{Greg} : Vous voulez boire quelque chose ?

\textsc{Sally} : Oui, je veux bien.

Greg s’est levé, dirigé vers le bar, est revenu avec une bouteille de Jack Daniel’s de deux verres : Je n’ai que ça. J’espère que ça fera l’affaire.

\textsc{Sally}, mouchant son gros nez avec des grâces de pucelle effarouchée : Ce sera parfait, merci. Excusez-moi, je suis ridicule.

«~Ça tu peux le dire, ma grosse ! Ton rimmel va couler si tu continues à chialer comme une midinette~» s’est dit Greg en remplissant les verres : Je vous prie, je vois bien que la situation n’est pas facile pour vous. Tenez, buvez ça, vous vous sentirez mieux après.

\textsc{Sally} : Merci, vous êtes vraiment très aimable.

\textsc{Greg}, intérieurement : «~Tu parles, Charles !~»

\textsc{Extérieurement} : Donc, vous dites que cette personne a disparu il y a trois semaines jour pour jour, c’est bien ça ?

\textsc{Sally}, après avoir englouti cul sec son verre de Jack Daniel’s sans sourciller : Oui. J’avais rendez-vous avec Tiago au Sugar \& Spice, mais il n’est jamais venu.

\textsc{Greg} : Tiago ?

\textsc{Sally} : Tiago Alvarez, la personne qui a disparu.

\textsc{Greg} : D’origine espagnole, je suppose.

\textsc{Sally} : Brésilienne. Vous connaissez le Sugar \& Spice ?

\textsc{Greg} : Non, ça ne me dit rien.

\textsc{Sally} : C’est un cabaret, au 127 rue Théo Cazenave.

\textsc{Greg}, sirotant son Jack Daniel’s en affichant un air profond de type chez qui la moindre syllabe prononcée déclenche un tsunami de pensées complexes qui s’entrechoquent comme des électrons dans un accélérateur de particules : Je vois.

Pour l’instant, tout ce qu’il voyait, c’était qu’une tante n’était pas venue à son rendez-vous avec une autre tante dans un repaire de tantes de la rue Théo Cazenave.

\textsc{Sally}, contemplant d’un œil morne le verre vide qu’elle tenait toujours à la main, un filet de morve au coin du nez : Cela se passait il y a trois semaines jour pour jour, et depuis je n’ai plus aucune nouvelle.

\textsc{Greg} : Je vous en sers un autre ?

\textsc{Sally} : Oui, c’est pas de refus.

À ce train-là, la bouteille ne ferait pas de vieux os.

Greg s’est exécuté, puis il est allé remettre la bouteille dans le placard pour bien signifier à mademoiselle Sally Robinson qu’il ne fallait pas confondre son officine avec un débit de boissons.

\textsc{Greg} : Il ne répond pas au téléphone ?

\textsc{Sally}, le nez dans le Jack Daniel’s : Non. J’ai laissé des tonnes de messages, en vain. Aujourd’hui son téléphone ne sonne même plus. Je vous l’ai dit : nous sommes très proches. Jamais il ne resterait trois semaines sans donner de nouvelles. J’ai la certitude qu’il lui est arrivé quelque chose, quelque chose de grave.

\textsc{Greg}, toujours compatissant : Vous ne pensez pas qu’il aurait pu… comment dire… s’évaporer dans la nature ?

\textsc{Sally} : Tiago n’avait aucune raison de s’évaporer, comme vous dites. Tout allait bien pour lui. J’aurais été le premier à le savoir s’il avait eu le moindre problème.

\textsc{Greg} : Vous êtes allé trouver la police ?

\textsc{Sally} : Non, je sais comment les flics traitent les gens dans mon genre. Je n’ai pas envie qu’ils se foutent de ma gueule, et je sais très bien qu’ils ne lèveront pas le petit doigt pour retrouver Tiago. Un pédé de plus ou de moins, c’est le cadet de leurs soucis !

Greg s’est renversé dans son fauteuil, paupières mi-closes, a hoché la tête, affiché son sourire numéro 12 (réservé aux situations délicates exigeant une grande finesse d’esprit, des nerfs d’acier et une joie de vivre à toute épreuve), puis il a plongé ses yeux (bleus, Greg était blond, grand, mince, tout le contraire de Sally) droit dans ceux de son interlocuteur, l’a fixé quelques instants sans desserrer les dents, avant de déclarer, de cette voix grave et onctueuse qui avait le pouvoir de placer celui ou celle qui l’entendait dans un état de bien-être proche de l’extase : Je comprends. Et il fait quoi, ce Tiago, dans la vie ?

\textsc{Sally} : Ambulancier.

\textsc{Greg} : Oui, ce n’est pas comme si il travaillait à la DGSE. Des problèmes de drogue, alcool ou autre ?

\textsc{Sally} : Il sniffe un peu de temps à autre, comme tout le monde. Rien de bien méchant.

\textsc{Greg} : Dois-je comprendre que vous-même…

\textsc{Sally} : Comme tout le monde.

\textsc{Greg} : Tout le monde ne sniffe pas un peu de temps à autre.

\textsc{Sally} : Vous devriez essayer. Passez donc me voir au Sugar \& Spice, j’y suis presque tous les soirs.

\textsc{Greg} (sourire numéro 3, crispé pour exprimer une gêne certaine, mais sans excès pour ne pas froisser le vis-à-vis) : J’y penserai à l’occasion. Pour en revenir à Tiago, vous ne lui connaissez pas d’ennemi, de gens qui pourraient avoir des raisons de lui en vouloir.

\textsc{Sally} : Non. C’est un garçon charmant, très ouvert…

\textsc{Greg} : Je n’en doute pas. Et vous-même ?

\textsc{Sally} : Moi aussi je suis très ouverte…

\textsc{Greg} : Non, je veux dire vous-même, vous faites quoi dans la vie ?

\textsc{Sally} : Je suis artiste de music-hall, je travaille au Sugar \& Spice.

\textsc{Greg}, après s’être raclé le fond de la gorge : Ah très bien ! Vous chantez ?

\textsc{Sally} : Oui, et je danse, aussi.

\textsc{Greg} : Bien, parfait. Au sujet de Tiago, je suppose que vous vous êtes renseignée auprès de son employeur.

\textsc{Sally} : Bien sûr, mais il n’en sait pas plus que moi. Personne ne sait rien, et tout le monde est très inquiet parce que Tiago n’est pas du genre à disparaître comme ça du jour au lendemain, sans donner de nouvelles à qui que ce soit. Il est très proche de sa mère, par exemple, et elle non plus n’a pas de nouvelles. Elle voulait appeler la police mais je l’ai persuadée de n’en rien faire. Je lui ai dit que ça risquait de coûter un peu d’argent, mais que j’allais mettre un vrai professionnel sur le coup.

\textsc{Greg} : Je crois qu’il serait quand même souhaitable de le faire.

\textsc{Sally} : Quoi ?

\textsc{Greg} : Appeler la police.

\textsc{Sally} : Vous refusez le job ?

\textsc{Greg} : Je n’ai pas dit ça. Je pense comme vous qu’ils ne feront rien, ou pas grand-chose, mais autant mettre toutes les chances de notre côté.

\textsc{Sally} : Non. Je veux, nous voulons que ce soit vous qui meniez l’enquête.

\textsc{Greg} : Je peux savoir comment vous êtes arrivée jusqu’ici ?

\textsc{Sally} : En voiture, j’essaie d’éviter les transports en commun. Pourquoi ?

\textsc{Greg} : Pour rien. Non, ce que je veux dire, c’est pourquoi moi ? Pourquoi êtes-vous venue me trouver moi et pas un autre ?

\textsc{Sally} : Vous avez été chaudement recommandé.

\textsc{Greg} : Je peux savoir par qui ?

\textsc{Sally} : Une personne pour qui vous avez travaillé et qui a été pleinement satisfaite de vos services. Je ne peux pas vous en dire plus, elle tient à garder l’anonymat.

Greg s’est creusé le chou quelques instants pour essayer de voir qui pouvait correspondre à la personne en question. En vain, d’autant que personne, jusqu’à présent, n’avait eu à se plaindre de ses services. Quand un client vient vous trouver pour enquêter sur ceci ou cela, résoudre telle ou telle affaire, on n’est pas censé enquêter sur lui, en tout cas pas officieusement, et pas davantage que ne l’exigent les données du problème, sauf bien sûr si on flaire une embrouille et a le sentiment de se faire manipuler. Il pouvait donc s’agir d’à peu près n’importe qui, en rapport direct ou indirect avec les protagonistes de l’affaire et le Sugar \& Spice.

\textsc{Greg} : C’est bon, je n’insiste pas. De toute façon, ça n’a pas grande importance.

\textsc{Sally} : Non, en effet. Alors vous acceptez ou pas ?

\textsc{Greg} : Vous connaissez mes tarifs ?

\textsc{Sally} : Oui. C’est cher mais votre réputation n’est plus à faire.

\textsc{Greg} : Nous sommes dans la moyenne de la profession. J’ajoute que nous avons aussi des forfaits très intéressants, surtout pour les enquêtes qui risquent de prendre un certain temps.

\textsc{Sally} : Nous avons constitué une petite cagnotte qui devrait nous permettre de faire face aux dépenses, même les plus imprévues. Je ne sais pas si vous êtes au courant, mais notre petite communauté LGBT, QIA+ si affinités, traverse une mauvaise passe, si j’ose m’exprimer ainsi. Durant les six derniers mois, plusieurs d’entre nous ont disparu sans laisser de trace, se sont comme qui dirait volatilisés dans la nature. Il y a peut-être dans le secteur un dingue qui s’est donné pour mission d’exterminer tous les queers, les transgenres, les non-binaires et les pédés. La police est au courant mais se fiche comme d’une guigne de ce qui peut bien nous arriver. Si un tel individu existe, il se pourrait que Tiago soit tombé entre ses mains. Alors, c’est oui ou c’est non ?

\textsc{Greg} : C’est oui.

Par le plus grand des hasards, et sans doute aussi parce qu’il avait fait venir le major Sandor Balint et son fidèle Zoltan, un malinois de cinq ans d’âge, excessivement racé et incontestablement le plus grand renifleur de sperme et sécrétions intimes en tout genre de sa génération, fleuron de la brigade cynophile, récemment élu Truffe de Platine et meilleur chien policier de tous les temps (que son maître ne manquait jamais de récompenser avec une poignée de croquettes Waterflox au sanglier, baies sauvages et légumes de saison, je suppose que le nom vous rappelle quelque chose, une affaire de sinistre mémoire qui prouve une fois encore que les tréfonds de l’âme humaine sont loin d’avoir été atteints), Zaahid avait identifié, sur un tronc d’arbre avoisinant la scène de crime, des traces de liquide séminal pour le moins suspectes. D’après lui, ces traces attestaient qu’on avait affaire à un grand malade qui s’était astiqué le poireau en regardant le corps brûler, péché mignon de certains pyromanes qui éprouvent une vive excitation sexuelle à la vue des flammes. À ce sujet, j’ai une pensée émue pour Ottis Toole\nf{Ottis Toole (1947--1996), serial killer américain originaire de Jacksonville (Floride), condamné pour plusieurs meurtres commis notamment en compagnie de Henry Lee Lucas. Il est également suspecté du meurtre d’Adam Walsh (1981), dont la disparition conduisit à la création du \textit{National Center for Missing and Exploited Children}. \textit{Source :} \textup{fr.wikipedia.org/wiki/Ottis\_Toole}}, alias le cannibale de Jacksonville, qui s’est livré à de nombreuses exactions en compagnie de son petit copain Henry Lucas\nf{Henry Lee Lucas (1936--2001), serial killer américain condamné pour onze meurtres. Il avait initialement avoué plusieurs centaines de crimes à travers les États-Unis, aveux largement rétractés par la suite et controversés. \textit{Source :} \textup{fr.wikipedia.org/wiki/Henry\_Lee\_Lucas}}, lui-même sérieusement dérangé du bocal. Il faut dire que le pauvre Ottis avait commencé à morfler dès sa plus tendre et juteuse enfance, son âge le plus fondant, étant issu d’une famille de bouseux analphabètes qui pratiquaient inceste, torture et consanguinité depuis des générations. Dès l’âge de cinq ou six ans, parfois dès le berceau, les enfants étaient quotidiennement abusés et soumis à de cruels châtiments (tant corporels que psychologiques) pour les endurcir, les préparer à la dure réalité de l’existence misérable qui les attendait, mais aussi s’assurer qu’ils seraient aux premières loges pour perpétuer la tradition, les valeurs familiales. Sa grand-mère, au lieu de faire des cookies, lui chanter de comptines pour l’endormir et lui remonter le moral quand il est triste, comme le font toutes les gentilles grands-mères, est une vieille folle qui se lave tous les trente-six du mois, picole comme un trou et pratique le satanisme à haute dose. La nuit, ils font la tournée des cimetières pour déterrer des cadavres. Quand il n’est pas sage, rechigne à démembrer un corps ou avaler une potion qu’elle a fabriquée spécialement pour lui, elle l’arrose de seaux de pisse et le barbouille d’excréments. Toutes ces activités qui, en dépit de leur caractère inhabituel, pourraient sembler ludiques à première vue, sont en réalité très traumatisantes pour un gamin de huit ans, heureusement pour lui déjà accro à l’alcool et aux drogues qui lui permettent d’envisager l’existence avec davantage, sinon de sérénité, au moins de détachement. À quatorze ans, sur les conseils avisés de sa grande sœur qui picole, se drogue et se prostitue (le tiercé gagnant pour finir en pièces détachées dans une benne à ordures), il fréquente les bars gay et loue ses services pour quelques cents. Rapidement, il se retrouve avec le fion tellement large qu’un troupeau d’éléphants aurait pu le traverser au pas de charge sans toucher les bords, et comme il a un petit pois à la place de son cerveau imbibé d’alcool et des images horrifiques accumulées pendant son enfance, il se lance dans une carrière de serial killer cannibale qui lui vaudra une certaine notoriété auprès de tous les cinglés amateurs de faits divers sordides, autant dire une bonne partie de la population mondiale. À ce sujet, je rappelle quand même que l’espèce humaine se trimballe un sérieux problème d’addiction à la violence, l’horreur et la tragédie, données stylistiques qui semblent indissociables de son activité et consubstantielles à sa nature profonde, laquelle serait donc, par voie de conséquence, fondamentalement violente, horrible et tragique, ce qui, reconnaissons-le, ne présage rien de bon pour l’avenir.

Un beau jour, parce que la vie est un merveilleux conte de fées, un monde où tout est possible, dans lequel les rêves les plus improbables se réalisent au moment où on s’y attend le moins, Ottis rencontre un beau jeune homme borgne qui répond au doux nom d’Henry Lucas. Le petit Henry a été élevé dans une cabane en rondins au milieu des bois, avec les bêtes sauvages, les tapis de mousse sous les pieds et les champignons, mais sans eau ni électricité. Viola, sa mère, alcoolique au dernier degré, l’oblige à garder les cheveux longs et aller à l’école habillé en fille, ce qui, de l’avis de tous les psychiatres consultés, n’est pas l’idéal sur le plan de la construction personnelle, la mise en place de structures psychologiques stables et socialement compatibles. Elle le tabasse quotidiennement et se prostitue sous ses yeux avec une clientèle dont je vous laisse imaginer le niveau d’éducation. En fait de poètes et hommes du monde, il s’agit surtout de brutes épaisses dont la pudeur et la délicatesse ne font pas partie des qualités premières. Il n’y a pas que du bon dans le mauvais, et inversement. Un jour, un ami de sa mère qui s’est pris d’affection pour lui, apprend au petit Henry, alors âgé de dix ans, à égorger les animaux avant de leur faire l’amour. Non seulement les dernières convulsions sont source de plaisir, mais ça permet d’éviter de prendre un mauvais coup au cas où l’animal récalcitrant se débattrait plus que de raison. C’est ce genre de petits conseils qui forment la jeunesse et l’aident à mettre un pied serein dans l’âge adulte. À vingt-quatre ans, Henry, qui a déjà passé une bonne partie de sa vie en prison pour vol, vol en récidive et tentatives d’évasion, se dispute avec sa mère, se retrouve sans trop savoir comment avec un couteau à la main, lequel couteau se retrouve sans trop savoir comment dans le ventre de sa mère, laquelle mère se retrouve sans trop savoir comment refroidie pour de bon. Il fallait bien que ça arrive un jour, elle ne l’avait pas volé, diront certains auxquels je laisse l’entière responsabilité de leurs propos. Henry plaide la légitime défense, écope de trente ans de prison, est déclaré trop con et secoué du bulbe pour effectuer sa peine, atterrit à l’hôpital psychiatrique du coin, tente de mettre fin à ses jours à plusieurs reprises, suit un traitement de choc à base de courant électrique et antidépresseurs surpuissants, est libéré dix ans plus tard pour bonne conduite (il est tellement shooté qu’il tient à peine sur ses jambes et ne risque pas de faire chier grand monde), peine quelque peu à se réadapter à la vie normale, retourne en taule un an après pour avoir tenté d’enlever deux écolières dans l’intention manifeste de leur faire subir des sévices sexuels inappropriés, et en ressort quatre ans plus tard à condition de se tenir à carreau, arrêter de faire la sortie des écoles et cesser immédiatement de tenter de s’approprier sous la menace des choses qui ne lui appartiennent pas. C’est alors qu’il fait la connaissance d’Ottis Toole, charmant garçon avec lequel, même s’il conserve une certaine attirance pour les adolescentes, il ne s’interdit pas d’avoir des relations dépassant de strict cadre de l’amitié consensuelle. Toole est con comme un balai, c’est vrai, mais aussi d’une touchante naïveté qui l’émeut au plus profond de lui-même. Il est comme le petit frère (ils ont dix ans d’écart) qu’il n’a jamais eu et pourra enculer all night long sans avoir de comptes à rendre à la justice divine et encore moins celle des hommes, même s’il ne faisait pas toujours bon être gay dans l’Amérique profonde des années 80. Lui-même, d’ailleurs, n’était pas vraiment gay, mais plutôt bisexuel, à voile et à vapeur, comme on disait à l’époque en étouffant un ricanement gêné. Quand on a appris à enculer des chèvres, qu’on a des besoins sexuels importants et qu’on n’a rien d’autre sous la main que son petit frère pour les soulager, pourquoi ne pas en profiter, même si le petit frère en question affiche des pratiques douteuses avec lesquelles on n’est pas toujours en accord, par exemple manger de la chair humaine cuite au barbecue. Mais bon, on sait que la vie de couple n’est pas toujours parfaite ni idéale. Il y a des hauts et des bas, des points d’achoppement sur lesquels on doit éviter de se focaliser si on veut avoir une chance de s’en sortir. Mais cette belle amitié prend fin le jour où Lucas enlève Becky Powell, la nièce de Toole dont il est secrètement amoureux. Putain de destinée, qui finit toujours par triompher quels que soient les trésors d’énergie qu’on dilapide à la combattre. Quelle perte de temps ! Ce qui devait arriver arriva, les deux hommes se séparent : Toole rentre à Jacksonville, la queue entre les jambes et le cœur meurtri (la queue aussi, du reste, à force de la fourrer n’importe où), tandis que Lucas et Powell décident d’aller couler le parfait amour au Texas (où ils vont couler, en effet, mais pas le parfait amour).

Fin d’une des plus belles, tragiques et subversives histoires d’amour du vingtième siècle (bon évidemment pas très glamour du fait qu’elle ne se passe pas chez les riches, dans un château avec des toiles de maîtres, des meubles de prestige et les portraits des ancêtres accrochés aux murs, mais met~-- mémé était une sorcière~-- en scène des personnages issus de milieux modestes, sinon franchement défavorisés, d’immondes cloaques infestés de vermine, autrement dit des gens qui n’ont jamais mastiqué un toast au caviar, chié une langouste flambée au rhum ni bu du Cristal au goulot, des cons arriérés, avec des QI de mouche de merde, formatés dans le seul et unique but de faire le mal, nuire à leurs concitoyens, mais animés malgré tout d’une farouche volonté de s’en sortir, du vibrant désir d’échapper à une condition qu’ils n’ont pas choisie et rejettent de toute la force de leurs âmes corrompues, flétries par vice), dont les deux protagonistes écoperont de peines de prison à vie et finiront leurs jours derrière les barreaux.

D’après Zaahid, qui connaissait évidemment le cas de Toole, on pouvait se trouver en présence d’un gros tas de merde du même genre, un attardé sexuellement perturbé qui prenait son pied en foutant le feu à des gens et regardant les flammes danser sous la lune. Ou alors, il l’avait fait griller comme un méchoui dans l’idée de se tailler quelques côtelettes pour son repas du soir, hypothèse cependant peu probable compte tenu de l’état de calcination avancée du cadavre, même pour un amateur de viande bien cuite. L’analyse détaillée de la scène de crime, notamment les débris organiques et végétaux situés sous la dépouille, avait révélé la présence de trois prothèses dentaires, une boucle d’oreille sertie de diamant et deux piercings de tétons en forme de serpent qui se mord la queue. D’autre part, il était établi avec certitude que la victime était un individu de sexe mâle âgé d’une trentaine d’années, descendant très probable des Nambiquara\nf{Les Nambiquara sont un peuple autochtone du Brésil établi sur les hauts plateaux du Mato Grosso, étudiés par Claude Lévi-Strauss lors de son expédition de 1938. Leur population a été drastiquement réduite par la colonisation et les épidémies ; ils ne sont plus aujourd’hui que quelques milliers. \textit{Source :} \textup{fr.wikipedia.org/wiki/Nambikwara}}, une tribu amérindienne des hauts plateaux du Mato Grosso. Comme partout en Amérique du Sud, la colonisation a signé leur perte, et ils ne sont plus aujourd’hui que quelques milliers disséminés dans des villages le long des fleuves, menacés en permanence par la cupidité des chercheurs d’or et la rapacité des promoteurs. Naturellement, les missionnaires portugais ne faisaient pas qu’apporter la bonne parole à des sauvages qui se baladaient à poil dans la forêt. Force était de reconnaître que pour des sauvages, à mi-chemin entre la femme et la guenon, les femelles, surtout les plus jeunes, étaient quand même très attirantes, même si elles dégageaient des odeurs pas toujours très catholiques, et les missionnaires, au-delà de la parole, se devaient de vaincre leurs appréhensions afin d’introduire en elles les gènes salvateurs de la civilisation. Quelques siècles plus tard, on ne peut que rendre hommage à leur clairvoyance et leur dévouement : les femmes portent des robes à fleurs et ont troqué leurs colliers de perles, bracelets en cul de tatou et autres plumes dans le nez contre des bijoux made in China à trois euros le kilo qui leur irritent la peau. Quant aux hommes, ils ont cessé de faire joujou avec leurs flèches empoisonnées pour bosser dans les plantations ou à la centrale hydroélectrique du coin. Et pour ce qui est de l’or présent un peu partout sur le territoire, qu’ils ne s’inquiètent de rien, des gens très compétents venus d’ailleurs se chargent d’en tirer le meilleur parti. Car enfin, si on veut que la civilisation continue son irrésistible progression à travers les contrées désolées de la barbarie et l’absence totale de culture occidentale de gens qui n’ont pas le permis de conduire et n’ont même, pour la plupart, jamais vu de voiture, il faut bien que tout le monde y mette un peu du sien.

Un beau soir, Greg m’a appelé pour savoir si j’avais entendu parler de certaines choses concernant la communauté LGBT.

J’ai répondu : Non, pas vraiment.

Le fait est que les retours concernant les difficultés de la communauté LGBT n’étaient pas monnaie courante dans la police.

\textsc{Lui} : Apparemment, il y en a plein qui disparaissent.

\textsc{Moi} : Ah bon ?

\textsc{Lui} : Oui. Crois-le ou non, mais je travaille en ce moment pour le sosie de Danny DeVito.

\textsc{Moi} : Sans blague ?

\textsc{Lui} : Oui. Sauf que c’est une femme avec une paire de nichons digne d’un film de Russ Meyer !

Russ Meyer\nf{Russ Meyer (1922--2004), réalisateur américain, pionnier du film érotique indépendant dit «~sexploitation~». Auteur de \textit{Faster, Pussycat! Kill! Kill!} (1965) et d'une œuvre caractérisée par l'humour burlesque et la mise en scène de femmes à la poitrine opulente. \textit{Source :} \textup{fr.wikipedia.org/wiki/Russ\_Meyer}}, le roi de la sexploitation (avec Tinto Brass\nf{Tinto Brass (né en 1933), réalisateur et monteur italien, connu pour des films érotiques provocateurs dont \textit{Caligula} (1979) et \textit{Salon Kitty} (1976). \textit{Source :} \textup{fr.wikipedia.org/wiki/Tinto\_Brass}} et Jess Franco\nf{Jess Franco (1930--2013), réalisateur espagnol prolifique, auteur de plus de deux cents films d'exploitation mêlant horreur et érotisme. Figure culte du cinéma de genre européen des années 1960--1980. \textit{Source :} \textup{fr.wikipedia.org/wiki/Jess\_Franco}}), ennemi juré de la morale judéo-chrétienne et du code Hays\nf{Le code Hays, ou Code de production, est un ensemble de règles de censure morale imposées à Hollywood de 1934 à 1968. Il interdisait notamment la nudité, les relations sexuelles explicites et toute représentation jugée contraire aux bonnes mœurs. \textit{Source :} \textup{fr.wikipedia.org/wiki/Code\_Hays}}, à qui l’on doit une foultitude de chefs-d’œuvre presque tous aussi inoubliables les uns que les autres, parmi lesquels Le Désir dans les tripes, Mondo Topless, La Vallée des Plaisirs et la série des Vixens. Spéciale dédicace à Kitten Natividad, alias Lola Langusta.

\textsc{Moi} : On n’arrête pas le progrès. Si demain des gens ont envie de se faire greffer un troisième sein, une langue de serpent ou un deuxième trou de balle, il y aura toujours quelqu’un pour leur donner satisfaction. Ça leur coûtera la peau des fesses, mais ils pourront toujours, moyennant une petite rallonge, se la faire remplacer par du cuir d’hippopotame, nettement plus résistant quand on passe sa vie assis devant un ordinateur ou une console de jeu.

\textsc{Lui} : T’as pas tort.

\textsc{Moi} : Rarement, en effet.

\textsc{Lui} : T’exagères un peu, comme d’habitude, mais t’as pas tort. Donc, si je comprends, t’as rien d’intéressant à me dire sur les LGBT.

\textsc{Moi} : Non, pas vraiment. Je sais qu’ils se crêpent régulièrement le chignon, se tirent les cheveux et se donnent des coups de pieds dans les tibias, mais je t’avouerai franchement que ça ne fait les gros titres de la gazette de la police. Après tout, jusqu’à preuve du contraire, ce sont des adultes responsables et on n’a pas à se mêler de leurs affaires. S’ils veulent disparaître, s’évaporer dans la nature comme des pets, c’est leur problème. Par contre, j’ai un cadavre sur le dos qui pèse une tonne et dont j’aimerais assez me débarrasser au plus vite. On l’a retrouvé dans la forêt, carbonisé jusqu’à l’os. Il ressemble à quoi, ton mec qui a disparu ?

\textsc{Lui} : Brun, la trentaine, d’origine brésilienne, plutôt mignon.

Une petite musique s’est fait entendre dans le fond de mon crâne, une petite musique genre flûte de pan dans la pampa, comme dans \textit{Mission} de Joffé\nf{Roland Joffé (né en 1945), réalisateur britannique, auteur de \textit{Mission} (1986), film sur les missions jésuites guaranis en Amérique du Sud au \textsc{xviii}\textsuperscript{e}~siècle, Palme d'or à Cannes. La bande originale, signée Ennio Morricone, inclut le célèbre \textit{Gabriel's Oboe}. \textit{Source :} \textup{fr.wikipedia.org/wiki/Mission\_(film,\_1986)}} ou \textit{Apocalypto} de Mel Gibson\nf{Mel Gibson (né en 1956), acteur et réalisateur américano-australien. Récompensé de l'Oscar du meilleur film pour \textit{Braveheart} (1995), il réalise également \textit{La Passion du Christ} (2004) et \textit{Apocalypto} (2006). Ses déclarations antisémites (2006) et ses soutiens politiques controversés ont alimenté les polémiques. \textit{Source :} \textup{fr.wikipedia.org/wiki/Mel\_Gibson}}, ce crétin réactionnaire qui vote Trump au même titre que Kanye West\nf{Kanye West (né en 1977), rappeur, producteur et entrepreneur américain, auteur de \textit{The College Dropout} (2004) et \textit{My Beautiful Dark Twisted Fantasy} (2010). Connu pour ses prises de position erratiques, ses déclarations antisémites (2022) et son soutien affiché à Donald Trump. \textit{Source :} \textup{fr.wikipedia.org/wiki/Kanye\_West}}, Jon Voight\nf{Jon Voight (né en 1938), acteur américain, Oscar du meilleur acteur pour \textit{Le Retour} (1978) et père d'Angelina Jolie. Républicain militant, il soutient Donald Trump depuis 2016. \textit{Source :} \textup{fr.wikipedia.org/wiki/Jon\_Voight}}, Dana White et Buzz Aldrin\nf{Buzz Aldrin (né en 1930), astronaute et pilote militaire américain. Le 21~juillet 1969, il est le deuxième homme à marcher sur la Lune lors de la mission \textit{Apollo~11}, après Neil Armstrong. \textit{Source :} \textup{fr.wikipedia.org/wiki/Buzz\_Aldrin}}, calviniste de mes deux qui aurait mieux fait de rester sur la Lune le jour où il y a posé le pied, le 21 juillet 1969. Make America Beauf Again, les rednecks ont encore de beaux jours devant eux.

\textsc{Moi} : Il n’avait pas des piercings aux tétons, par hasard ?

\textsc{Lui} : Si, des anneaux.

\textsc{Moi} : En forme de serpent qui se mord la queue ?

\textsc{Lui} : Tout juste, Auguste.

\textsc{Moi} : On les a retrouvés au milieu des cendres. Je suppose qu’il y a des tas gens qui ont des piercings de ce genre, mais il y a quand même de fortes chances pour ton client et le mien soient une seule et même personne. D’autant que si on en croit Zaahid, qui se trompe rarement sur le sujet, le mien descend en droite ligne des Nambiquara, une tribu amérindienne du Mato Grosso dont seuls quelques rares spécimens s’accrochent encore aux branches de la forêt amazonienne. Je pense que ça commence à faire beaucoup, tu ne crois pas.

\textsc{Lui} : C’est clair, Albert. Il s’appelle Tiago Alvarez et bosse comme ambulancier SMUR au CHU Désiré Trudeau.

\textsc{Moi} : C’est énorme !

\textsc{Lui} : Oui, c’est super excitant !

\textsc{Moi} : T’as son adresse ?

\textsc{Lui} : Bien sûr, que je l’ai ! Tu me prends pour qui ? Un amateur ?

\textsc{Moi} : Tu peux me la donner, s’il te plaît. On va en avoir besoin pour récupérer de l’ADN et le comparer à celui de notre grand brûlé.

\textsc{Lui} : 73 rue Valentin Abou, dans le 14\ieme{}. Je te cache pas que je suis déjà allé y faire un tour, histoire de m’assurer qu’il n’y était pas. En même temps, s’il était mort depuis trois semaines, l’odeur de charogne aurait alerté le voisinage.

\textsc{Moi}, en train de siroter tendrement un verre de Chevalières (une dizaine d’hectares de chardonnay entre les Rougeots et Meix-Chavaux, à Meursault) dans ma cuisine en compagnie de Zarina Brizzi, une fille dont j’ai déjà eu l’occasion de vous parler et vous laisser entendre à quel point sa présence était loin de me laisser dans l’état d’indifférence totale que générait ordinairement chez moi la présence de ses congénères : Ça dépend du voisinage. On a bien retrouvé une petite vieille morte depuis deux ans dans son appartement de Saint-Brieuc. Elle ne devait pas recevoir souvent de la visite.

\textsc{Lui} : Pauvre vieille.

Petite mise au point en passant : c’est au Narcisse Rose que j’avais, quelque temps plus tôt, fait la connaissance de Zarina Brizzi, trente-deux ans et toutes ses dents, archétype du missile supersonique made in Italy (comme le balistique Alfa au propergol solide développé par Leonardo dans les années 70, soi-disant resté à quai après l’adoption du traité rédigé par l’ONU dans le but de contrôler et limiter au maximum la prolifération des armes nucléaires à travers le monde), aussi bouleversante qu’une assiette de pappardelle au sanglier ou de lapin au chou et raisins secs, longue chevelure brune ramassée en un savant entrelacs (ne me demandez pas pourquoi il y a un S à la fin, ça fait partie du charme de la langue française, comme contrebasse, architecture, daim, ornithorynque, Seldjoukides, groupe sanguin, orang-outan et Vincent van Gogh pour n’en citer que quelques-uns) de mèches tressées avec une maestria confondante, regard profond comme les eaux de l’Arno quand le soleil se couche sur Florence et que vous êtes en train de vous goinfrer de crostini neri aux foie de volaille, câpres et anchois dans une antique trattoria de la via della Scala, etc., etc. … et tout cela, oui tout cela est très émotionnant, c’est certain, et je pourrais continuer pendant des centaines et des centaines de milliers de pages à vous décrire au millimètre près le moindre centimètre-carré de son incomparable épiderme à la coloration parfaite, mais l’essentiel de ce que vous avez à savoir tient en quelques mots : oui, elle et moi formions maintenant ce qu’il est convenu d’appeler un couple, et j’avais, à mon grand regret, dû renoncer au vœu de chasteté qui m’avait jusqu’alors tenu éloigné de la promiscuité charnelle inhérente à ce genre d’activité. Autrement dit il nous arrivait, Zarina et moi, de niquer comme des bêtes jusque tard dans la nuit. Et vous devez savoir aussi, par la même occasion, que Zarina avait une sœur, Tosca, copie conforme d’elle-même à quelques détails près, et que cette même sœur formait avec mon ami Zaahid Shirani ce qu’il est également convenu d’appeler un couple, autrement qu’il arrivait aussi à Zaahid et Tosca de niquer jusque tard dans la nuit, et je dirais même, les concernant, que cette activité avait pris une place prépondérante dans leur existence. J’ajoute que toutes deux avaient rendu les clés de leur suite du Jade Mountain Hôtel pour venir s’installer chez l’un et l’autre de leurs conjoints respectifs, ce qui ne m’arrangeait qu’à moitié dans la mesure où j’avais pris des habitudes de vieux garçon auxquelles je n’entendais pas renoncer aussi facilement.

\textsc{Moi} : Tu sais qu’on n’a pas le droit d’aller fouiller comme ça chez les gens ?

\textsc{Lui} : Ah bon ?

\textsc{Moi} : C’est de la violation de domicile. Pour faire ce genre de choses, il faut avoir une autorisation. T’as touché à rien, j’espère ?

\textsc{Lui} : Non, je voulais juste m’assurer qu’il n’était pas en train de pourrir dans un coin.

\textsc{Moi} : Quoi d’autre ?

\textsc{Lui} : Rien de spécial. Ah si, Alvarez fréquente le Sugar \& Spice, un cabaret de la rue Théo Cazenave. Je te conseille d’aller y faire un tour, ça vaut le coup d’œil !

\textsc{Moi} : Je n’y manquerai pas, dès que j’aurai la preuve qu’on parle bien du même type.

\textsc{Lui} : Si c’est le cas, je vais perdre mon job.

\textsc{Moi} : Pourquoi ? T’es pas obligé de le dire à ta cliente. Tu continues à palper tes honoraires comme si de rien n’était, et un beau jour, quand tu penses que la comédie a assez duré, tu lui balances l’info. De mon côté, je me suis arrangé pour que l’affaire ne fuite pas dans les journaux. Tu peux dormir sur tes deux oreilles, jamais ta cliente ne fera le lien entre les deux affaires. Elle s’appelle comment, d’ailleurs ?

\textsc{Lui} : Et le secret professionnel, t’en fais quoi ?

\textsc{Moi} : Je m’assois dessus. Et l’amitié, t’en fais quoi ?

\textsc{Lui} : Je m’assois dessus. Non, sans blague, c’est pas que je veux pas te le dire, mais je ne vois pas très bien à quoi ça te servirait dans l’état actuel des choses. Il sera toujours temps de voir quand tu auras les résultats de l’analyse ADN. J’ajoute que ma cliente, du haut de son mètre cinquante-sept, n’a pas vraiment le physique de l’emploi.

\textsc{Moi} : C’est vrai que c’est pas souvent que des nains assassinent des gens. Je sais même pas s’il existe un seul cas de tueur nain dans les annales du crime. Le nain, bizarrement, tue assez peu, alors qu’il aurait toutes les raisons d’en vouloir à l’existence et la société qui ne cesse de lui renvoyer son infirmité au visage. Cela dit, il ne faut pas se fier aux apparences. Il n’est pas rare, par exemple, que l’assassin vienne signaler lui-même la disparition de sa victime. Si ça se trouve, c’est ton sosie de Danny DeVito qui a refroidi son petit copain Alvarez. Enfin, refroidi, c’est une façon de parler. Je dirais plutôt un peu trop réchauffé. Certains criminels se croient intouchables, invisibles, ça les fait bander de venir se fourrer dans la gueule du loup pour jouer avec ses dents. Mais ils finissent toujours par se faire croquer.

\textsc{Lui} : Je crois que je t’en ai assez dit comme ça. Tu sais que ma cliente mesure moins d’un mètre soixante, qu’elle ressemble à Danny DeVito avec des nichons et fréquente le Sugar \& Spice. Je pense que même un flic aussi peu doué que toi ne devrait pas avoir trop de mal à l’identifier. Mais je suis certain qu’elle n’a rien à voir là-dedans. Je pense juste qu’elle est raide-dingue d’Alvarez et aimerait bien savoir ce qui lui est arrivé. Je peux même te dire qu’elle et ses potes du Sugar \& Spice ont fait une cagnotte pour retrouver leur mascotte. Par contre, s’il est arrivé quelque chose à Alvarez et si tu retrouves le tueur, on pourrait le livrer à la communauté LGBT qui se fera une joie d’organiser un simulacre de procès et le condamner à la peine capitale. J’ai évoqué le sujet avec ma cliente, très à cheval sur les principes, et elle ne m’a pas caché que le cas échéant, elle serait ravie de procéder elle-même à l’exécution.

\textsc{Moi} : On vit quand même dans un drôle de monde.

\textsc{Lui} : C’est bien vrai, ma bonne dame.

\textsc{Moi} : Je fais ma petite enquête et je te tiens au courant. De ton côté, préviens-moi si t’as du nouveau.

\textsc{Lui} : Je n’y manquerai pas.

J’ai raccroché et me suis précipité comme un sauvage sur la bouteille de Meursault qui attendait sagement dans le réfrigérateur, toute perlante de gouttelettes de buée rafraîchissante (même si, techniquement parlant, ce n’est la buée qui rafraîchit, on est bien d’accord, sachant que c’est seulement la vision de celle-ci qui présage de la fraîcheur du liquide qui se trouve dans la bouteille, en l’occurrence un Meursault Chevalières signé Jean-Charles Pichard, considéré par beaucoup et non des moindres comme le pape de l’appellation, le très saint père de la Côte de Beaune devant lequel les apôtres du chardonnay venaient se prosterner et lécher la terre sacrée du vignoble en poussant des petits couinements de plaisir). Nos verres étaient vides mais nos cœurs pleins d’amour et nos sous-vêtements débordants de victuailles (du boudin, de la saucisse, des noix, de la moule, du poireau, de la courgette, de la frisée et j’en passe) comme des caddies de supermarché en période de fêtes. Enfin, surtout Zarina, parce que moi, comme je vous l’ai dit, j’étais plus proche du moine cistercien que de la star du X dopée à la testostérone. Mon truc, c’était plutôt la contemplation, la méditation, au fond d’une cellule ou assis sur un rocher surplombant le vide, la bite à l’air de préférence, dressée comme une antenne pour capter la tiédeur buccale du soir et les fréquences spirituelles de l’univers. J’allais par monts et par vaux, simplement vêtu, la bourse vide et la queue entre les jambes, prêcher la bonne parole et secourir la veuve et l’orpheline. J’avais fait don de mon existence au service de l’autre, renonçant à tous les plaisirs, honneurs et privilèges dus à mon rang. Néanmoins, en prenant sur moi et avec l’aide de notre seigneur tout-puissant, grâce lui soit rendue, j’arrivais tant bien que mal à un niveau de performance qui n’était pas sans rappeler les riches heures de mon étincelante jeunesse, les plus brillants faits d’armes du séducteur impénitent que j’étais alors, ce prédateur à la démarche souple et féline et au sourire carnassier auquel aucune adolescente boutonneuse et dentairement appareillée ne pouvait résister. Cela dit, j’avais trouvé auprès de Zarina, devant laquelle je me sentais tel un enfant qui s’émerveille chaque jour un peu plus de la beauté du monde, une oreille bienveillante, attentive à mes tourments et sensibles à mes exigences parfois retorses. Bon, je ne veux pas entrer dans les détails, mais le fait est que mon approche des plaisirs physiques n’était pas toujours très orthodoxe. Le mot «~chair~», pour moi, avait une consonance qui résonnait parfois bien au-delà de ce que des oreilles dites normales sont en capacité de percevoir, et encore moins, si elles y parviennent, en capacité d’accepter. Oui, je crois pouvoir dire que j’avais trouvé en Zarina une âme sœur des plus compatissantes, et les retours que j’avais de Zaahid sur sa liaison avec Tosca me confortaient dans l’idée que nous avions l’un et l’autre mis la main sur des perles d’une grande rareté, des joyaux d’autant plus précieux qu’ils appartenaient à une vieille lignée florentine (comment ne pas imaginer les palais somptueux, les jardins merveilleux, les fontaines jaillissantes et les villas de rêve sur la Riviera) dont l’immense fortune faisait l’admiration et l’envie de bien des pèlerins désargentés. Je m’empresse (je sais à quel point les gens sont méchants) de préciser que tel n’était pas exactement notre cas à Zaahid et moi-même, puisque le fric que je détournais copieusement dès que j’en avais l’occasion, les perquisitions-acquisitions-disparitions inexpliquées de liasses de billets de banque, ainsi que les saisies de drogue que je me permettais de réinjecter discrètement sur le marché afin de satisfaire une demande en constante expansion, me garantissaient un train de vie plus que raisonnable, train dans lequel je me gardais bien de voyager en dehors de ma plus stricte intimité, heureusement constituée de personnages aussi peu recommandables que moi. C’est ainsi que je persistais à rouler ostensiblement dans ma vieille Kangoo déglinguée et porter des fringues de confection d’un goût douteux, lesquelles me valaient régulièrement les sarcasmes de la profession. Zaahid, de son côté, pouvait se prévaloir de son salaire de légiste, un salaire plutôt modeste qu’il optimisait officieusement avec un certain nombre de prestations médicales réservées à ce qu’il appelait lui-même une «~clientèle particulière~», c’est-à-dire des gens (des grands timides pour la plupart, des mélancoliques ou des agoraphobes) qui préféraient rester en dehors de la filière hospitalière classique. Cette clientèle particulière, qui débarquait souvent avec des trous de balles situés à des endroits inhabituels, savait se montrer reconnaissante.

