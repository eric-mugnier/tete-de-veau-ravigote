\chapter{Acte 7}

\noindent Et on y est allé, figurez-vous, car il nous arrivait parfois, quand on n’avait rien de mieux ni de plus pressant à faire, de faire ce qu’on avait dit qu’on allait faire, chose qui, dans le cas présent, était loin de tomber sous le sens, car j’avais personnellement beaucoup mieux et plus pressant à faire que courir à la recherche d’un ami d’enfance enlevé sous mes yeux par une mystérieuse créature, très belle au demeurant, pour ne pas dire fascinante, mais dont les intentions véritables étaient loin d’être d’une clarté aveuglante. Ce que j’avais de mieux et autrement plus pressant à faire, c’était, vous l’aurez compris, de foncer chez moi ventre à terre pour retrouver enfin Zarina, belle Italienne aux contours immortels et avec des braises à la place des yeux qui avait saisi mon cœur — et le reste — entre ses griffes puissantes et ne semblait aucunement décidée à les lâcher. D’autant moins, je l’avoue à ma grande honte, que je ne faisais pas le moindre effort pour tenter de m’en affranchir. Bien au contraire, c’était avec une lascivité répugnante que je m’abandonnais totalement à l’emprise sexuelle et affective qu’elle exerçait sur mon humble et pathétique personne. Sa longue chevelure moirée exerçait sur moi un attrait puissant, et c’est telle une limace ruisselante de bave que je rampais à ses pieds menus, léchant goulûment ses chevilles d’une finesse exquise, lapant jalousement la moindre de ses sécrétions, reniflant désespérément l’air parfumé de ses fragrances enchanteresses, ronronnant tel un chaton orphelin à la moindre de ses caresses, étalon en rut hennissant dès que le bout de ses doigts effleurait mon épiderme hypersensible, agneau tondu de frais offrant sa gorge au sacrifice en bêlant de joie comme le dernier des imbéciles, mortifiant mes chairs luxurieuses pour tenter d’en extirper le vice et sauver mon âme faisandée des barbecues de l’enfer. J’étais, en un mot comme en cent, réduit à un état de loque humaine ayant sciemment renoncé à toute espèce de respect de soi-même, sens critique et semblant de dignité. Songez que je pouvais, à des lieues à la ronde, tel un limier surentraîné, sentir son odeur et percevoir le souffle régulier de sa respiration nocturne, deviner, moulées par le fin tissu des draps, les courbes affolantes de son corps de rêve, fraîche oasis au sein de laquelle il me tardait d’aller me ressourcer après une longue et quotidienne traversée du désert de l’existence.

Oui, au cas où vous ne l’auriez pas remarqué, il se trouve que je suis aussi un très grand écrivain de la passion amoureuse, descripteur minutieux des suggestions voluptueuses qui vous sucent les neurones à la paille et pondent leurs œufs dans le cerveau ramolli des amants telles des mouches à viande dans la dépouille d’un rat crevé ou un chat écrasé, et je tenais à le préciser, en toute modestie. Voilà, c’est fait. Et rassurez-vous, je ne compte pas m’étendre sur le sujet plus que nécessaire.

Le Caribbean Hôtel ne se trouvait pas à plus de cinq cents mètres de l’endroit où nous étions. On aurait pu y aller à pied, bien sûr, comme des écologistes responsables alliant saine pratique sportive et conscience aigüe de l’environnement, mais non seulement on n’en avait strictement rien à foutre (la majeure partie d’entre nous estimant que la présence de l’être humain sur terre touchait à son terme et que rien ne pourrait plus empêcher l’inéluctable, même si on décidait de faire machine arrière et sautait à pieds joints sur la pédale de frein), mais surtout on en avait plein les pattes d’une journée qui s’éternisait au-delà du raisonnable, raison pour laquelle nous avons jugé que le mieux était encore de remonter dans le G30 pour nous y transporter.

L’établissement, niché au creux d’un quartier connu pour abriter une forte population noire (mais je parle ici d’une population noire de qualité supérieure, pas de zombies accros à la xylazine, non, je parle de gens éduqués à forte valeur ajoutée, anesthésistes-réanimateurs, profs d’université, cadres supérieurs, assistants de production, directeurs financiers, experts-comptables, chefs de produit, community managers, web designers, traders, architectes, plus évidemment quelques dealers en costard-cravate et putes haut de gamme pour permettre à tout ce petit monde de s’encanailler un peu), était la réplique à quelques détails près du couvent des Ursulines de la Nouvelle-Orléans, œuvre du capitaine Broutin, ingénieur militaire, cartographe et architecte officiel de Louisiane dans les années 1700. À l’origine destiné à l’éducation des jeunes filles de bonnes familles, lesquelles étaient pour la plupart des petites garces échevelées qui ne pensaient qu’à jouer à touche-pipi, il s’est peu à peu transformé en lupanar fréquenté par les esclavagistes du coin, aussi nombreux que les morpions dans le slip d’un légionnaire.

Le Caribbean, d’inspiration latino-créole revue et corrigée par les châteaux de la Loire, le gothique flamand et Ludwig Mies van der Rohe, ne s’étalait pas aussi largement dans l’espace que son glorieux modèle de Pelican State (le pélican brun, je le rappelle au passage pour celles et ceux qui s’attendraient à trouver des pélicans en train de barboter dans la piscine de l’Eden Roc, est un volatile endémique de cette région du globe). En compensation, il s’était vu adjoindre quelques étages de plus, ce qui lui conférait une forme de majesté supplémentaire. Cela dit, une bonne séance de réfection n’aurait pas été superflue pour lui redonner son lustre d’antan, même si les petites imperfections et lézardes qui se manifestaient ici et là contribuaient indéniablement à son charme suranné.

En tant que chef d’expédition, c’était à moi que revenait la charge de pénétrer en premier dans la place.

Il fallait appuyer longuement sur un bouton en forme de furoncle et se positionner idéalement sous l’œil de verre de la caméra de surveillance pour espérer obtenir une réponse, l’hôtel étant à priori fermé à cette heure avancée de la nuit. Seule la clientèle, dûment équipée pour aller et venir à sa guise, pouvait s’abstenir de ces contraintes sécuritaires.

Quelques longues minutes plus tard, une voix s’est fait entendre dans l’interphone : Qui êtes-vous ? Vous avez perdu votre clé ?

À laquelle j’ai répondu, en exhibant ma carte : Commissaire Beauvais, police judiciaire.

\textsc{La voix} : Désolé, nous n’avons plus de chambre.

\textsc{Moi} : J’ai de bonnes raisons de penser qu’il se passe des choses bizarres dans votre hôtel.

\textsc{La voix} : Qui sont ces gens qui vous accompagnent ?

\textsc{Moi} : Mes assistants. Je vous demanderai d’ouvrir cette porte et de répondre aux quelques questions que j’aimerais vous poser.

\textsc{La voix} : Il est tard. Revenez demain.

\textsc{Moi} : Dois-je en conclure que vous faites volontairement entrave à l’action de la justice ?

\textsc{La voix} : Vous avez un mandat, monsieur le policier ?

\textsc{Sam} : Ouvre la porte, connard ! Sinon je la défonce, et je te défonce la gueule après !

\textsc{La voix} : Il n’est pas question que ce grossier personnage mette un pied dans mon établissement.

\textsc{Moi} : Sam, s’il te plaît, ferme-la. Je suis désolé, cher monsieur, mais nous avons eu une journée extrêmement fatigante, pour ne pas dire harassante et pleine de rebondissements inattendus. Je vous demande juste d’ouvrir cette porte et de bien vouloir répondre à quelques-unes de mes questions, rien de plus.

\textsc{La voix} : Je comprends. Mais nous avons une procédure très stricte, ici. Il nous est interdit de laisser entrer n’importe qui à une heure aussi avancée de la nuit. Avec les agressions, sexuelles et autres, qui se multiplient dans le quartier depuis un certain temps, nous sommes obligés de nous montrer extrêmement prudents. Nous avons la responsabilité de nos clients, nous nous devons d’en assurer la sécurité. C’est aussi pour ce genre de service qu’ils acceptent de payer un prix aussi élevé pour avoir le privilège de résider dans nos murs. J’ajoute, au cas où vous ne le sauriez pas, que cet établissement est interdit aux Blancs.

\textsc{Moi} : C’est une blague ?

\textsc{La voix} : Pas du tout. Nous sommes officiellement rattachés à l’ambassade de Namibie, dont nous sommes en quelque sorte une annexe résidentielle. Nous jouissons d’un certain nombre de privilèges parmi lesquels celui de choisir notre clientèle. Cet hôtel est interdit aux Blancs.

\textsc{Sam} : Interdit aux Blancs ! On aura tout vu !

\textsc{La voix} : Exclusivement réservé aux gens de couleur, si vous préférez. Voici les faits : un jour, en 1884, un explorateur prussien du nom de Gustav Nachtigal, représentant de Bismarck, débarque dans le golfe de Guinée pour annexer des territoires. Un an plus tard, son successeur, Heinrich Göring, le père d’Hermann, arrive en Namibie avec la ferme intention de faire main basse sur ses richesses et réduire ses habitants à l’esclavage. Depuis, vos semblables, car cela vaut aussi pour les Français et les Européens en général, n’ont cessé de nous persécuter, nous dépouiller, nous humilier. Vous comprendrez que nous ayons aujourd’hui, en dépit de vos efforts pour faire table rase du passé et des quelques vagues compensations que vous daignez nous accorder de temps à autre, quelques réticences à votre égard. Je n’irais pas jusqu’à dire que nous vous détestons profondément, mais nous ne vous aimons pas beaucoup.

\textsc{Moi} : J’entends bien, cher monsieur, et vous présente mes plus plates excuses concernant les erreurs de mes ancêtres.

\textsc{La voix} : Les horreurs, vous voulez dire !

\textsc{Moi} : Oui, si vous voulez. Reste que nous sommes ici dans le cadre d’une enquête policière de la toute première importance, en lien étroit avec la lutte contre le racisme et l’homophobie.

\textsc{La voix} : Vraiment ?

\textsc{Moi} : Mais oui.

\textsc{La voix} : Dans ce cas, revenez avec un mandat en bonne et due forme. Je verrai alors ce que je peux faire, même si légalement rien ne m’oblige à donner suite à vos exigences. Ici, vous êtes officiellement sur le territoire de Namibie et vos lois ne s’appliquent pas.

\textsc{Moi} : Il se trouve que nous sommes à la recherche d’un homme de couleur, originaire de Sierra Leone, descendant d’un loyaliste des King’s Dragoons de Freetown, dont nous pensons qu’il pourrait avoir été enlevé et se trouver à l’instant même retenu contre son gré dans votre établissement.

\textsc{La voix} : Rien que ça ?

\textsc{Moi} : C’est déjà pas mal.

\textsc{La voix} : Disons que c’est un bon début. Et qu’est-ce qui vous fait croire que ce personnage serait retenu contre son gré au Caribbean Hôtel, un des établissements les plus sélects de la ville, réputé pour sa clientèle triée sur le volet et la qualité de ses prestations ?

\textsc{Moi} : Le ravisseur est une femme.

\textsc{La voix} : Vraiment ?

\textsc{Moi} : Oui. Une femme de couleur, qui plus est, d’une très grande beauté.

\textsc{La voix} : Tout s’explique, en effet.

\textsc{Moi} : Atiena, Gardienne de la Nuit, ça vous dit quelque chose ?

\textsc{La voix} : Rien du tout.

J’ai sorti mon smartphone et affiché une photo de l’intéressée que j’avais prise discrètement à son insu, dans le plus strict irrespect de son prétendu droit à l’image (ça me fait d’autant plus rigoler que les gens passent leur temps à se prendre en photo sous toutes les coutures), photo que j’ai ensuite exposée à l’œil inquisiteur de la caméra de surveillance : Tenez, c’est elle.

\textsc{La voix} : Très jolie femme, en effet.

\textsc{Moi} : N’est-ce pas.

\textsc{La voix} : Tout à fait remarquable.

\textsc{Moi} : Oui, je dois bien reconnaître que j’ai rarement vu une femme aussi belle. Une femme ou quoi que ce soit d’autre, d’ailleurs.

\textsc{La voix} : Nous autres, hommes de couleur, sommes particulièrement sensibles à la beauté des femmes.

\textsc{Moi} : Je pense que nous sommes particulièrement sensibles à la beauté des choses que nous désirons le plus.

\textsc{La voix} : Vous voulez dire que plus nous les désirons et plus elles sont belles, et non l’inverse ?

\textsc{Moi} : Oui. Le désir rend aveugle, ou sinon aveugle au moins fortement myope, ce qui fausse complètement la perception des choses.

\textsc{La voix} : C’est une théorie comme une autre.

\textsc{Moi} : En tout cas, c’est cette créature qui a enlevé Titus Beaugendre, lequel se trouve être un collègue de travail et un ami d’enfance. Bien qu’il soit de couleur, je le considère comme mon propre frère. Je ne sais pas où j’en serais sans lui. Sans doute au même point, mais la vie n’aurait pas du tout la même saveur. Ensemble, nous luttons contre le vice et la corruption.

\textsc{La voix} : Je vois. Malheureusement, je crains fort de ne pas pouvoir faire grand-chose pour vous.

\textsc{Moi} : Vous n’avez jamais vu cette femme traîner dans les couloirs de l’hôtel ?

\textsc{La voix} : Comment ça, traîner dans les couloirs de l’hôtel ? Vous n’êtes tout de même pas en train d’insinuer que…

\textsc{Moi} : Je n’insinue rien du tout. Elle a parlé d’un hôtel dans lequel elle résidait, et j’ai tout lieu de penser qu’il s’agit du Caribbean.

\textsc{La voix} : Qu’est-ce qui vous fait croire ça ?

Moi, hésitant : Un faisceau d’éléments concordants.

\textsc{La voix} : Mais naturellement vous n’avez aucune preuve de ce que vous avancez ?

\textsc{Moi} : Naturellement. Si j’en avais, je serais venu avec un mandat de perquisition en bonne et due forme. Alors que là j’arrive les mains dans les poches, en comptant sur votre bonne volonté pour m’aider à résoudre cette affaire.

\textsc{La voix} : Désolé, je ne peux rien vous dire de plus.

\textsc{Moi} : Vous ne connaissez pas cette femme ?

\textsc{La voix} : Bien sûr, que je la connais. Tout le monde la connaît dans notre petite communauté.

Sam, incapable de tenir sa langue : Qui, la clocharde ?

\textsc{Moi} : La ferme, Sam !

\textsc{La voix} : Ne vous en faites pas, je ne prête pas plus d’attention à cet individu que s’il s’agissait d’un moucheron écrasé sur la pare-brise de ma voiture.

\textsc{Sam} : Vous avez une voiture ?

\textsc{La voix} : Eh oui. Ça vous étonne, pas vrai ?

\textsc{Sam} : Un peu, oui. Je ne savais même pas que les Noirs avaient le droit de conduire.

\textsc{La voix} : Mais je ne suis pas Noir.

\textsc{Sam} : Je croyais que c’était interdit aux Blancs, ici ?

\textsc{La voix} : Je n’ai pas dit que j’étais Blanc. Si vous voulez tout savoir, j’appartiens au groupe ethnique des métis du Cap. Ma mère était française et mon père hottentot, descendant d’un héros de la bataille de Salt River, durant laquelle nombre de nos frères sont tombés pour rejeter les Portugais à la mer. Mais je ne vois pas pourquoi je perds mon temps à vous parler de ça. Pour en revenir à la Gardienne de la Nuit, celle que vous traitez de clocharde, il se trouve qu’il s’agit d’une personne extrêmement influente, très connue dans notre communauté, appréciée, respectée et crainte, car elle dispose de pouvoirs surnaturels hérités de ses ancêtres. Mais au lieu d’utiliser ces pouvoirs dans le seul but de s’enrichir et manipuler son prochain, elle s’efforce de les mettre au service du Bien. C’est une référence pour nous, un bien précieux que nous vénérons et protégeons en permanence. Quoi qu’il en soit, monsieur le commissaire, si comme vous le prétendez elle a «enlevé» votre ami, je peux vous assurer que vous ne tarderez pas à le retrouver en meilleure forme que jamais. Je suppose qu’elle s’est efforcée de le mettre à l’abri face à un grave danger qui le menaçait, voilà tout. Vous vous faites du souci pour rien, il ne peut être en de meilleures mains. Sur ce, je vous souhaite une excellente nuit.

\textsc{Moi} : Soit. Mais je vous préviens que si Titus n’est pas rentré chez lui à la première heure, je reviendrai avec un mandat et passerai votre coupe-gorge au peigne fin.

\textsc{La voix} : Il vous faudrait pour ça obtenir une autorisation préalable et je doute fort que vous y arriviez, tout policier que vous soyez. Nous avons des relations dans les plus hautes sphères du Pouvoir, et il nous suffirait d’un claquement de doigts pour mettre un terme à votre carrière pour une raison quelconque, tracasseries policières, par exemple. Je vous déconseille tout excès de zèle si vous ne voulez pas vous retrouver à faire la sortie des écoles avec un uniforme de gardien de la paix sur le dos. Cela dit, notre politique est de ne pas faire de vagues et nous œuvrons autant que possible dans la discrétion. Soyez tranquille, votre ami sera rentré chez lui à la première heure, en pleine forme, et tout cela ne sera plus qu’un mauvais souvenir que vous oublierez au plus vite. Bonne nuit, commissaire.

Il y a eu un déclic suivi d’un grésillement dont il n’était pas nécessaire d’être doctorant à Polytechnique pour comprendre le sens : foutez le camp en vitesse si vous ne souhaitez pas vous exposer à de gaves sanctions et réduire drastiquement votre espérance de vie.

J’aurais pu résonner, appuyer comme un taré sur le bouton pendant des heures afin de tarauder la cervelle de mon adversaire jusqu’à la folie et le pousser à une reddition sans condition, mais ma petite voix intérieure, celle qui me renseignait habituellement sur les choses à faire ou ne pas faire en cas de difficulté inattendue, qui me disait, par exemple, de détaler comme un lapin si un cordon de CRS arrivait au pas charge dans ma direction au cours d’une manif non déclarée place de la Nation, ma petite voix intérieure m’a vivement conseillé de n’en rien faire.

Il fallait se rendre à l’évidence : le Caribbean Hôtel était un bunker bien gardé, les négociations étaient terminées et il ne servait à rien d’insister, sauf si je tenais absolument à me retrouver à la circulation et passer le restant de mes jours à aider des petites vieilles à traverser la rue, dresser des PV pour stationnement gênant et faire souffler des gens dans des éthylomètres.

Autant dire que c’est le pas traînant et la mine déconfite qu’on est tous gentiment remontés dans le van, lequel, sans doute lui-même dans un état de dépression mécanique avancée, bielles et pistons baignant dans l’huile rance de la déconfiture, a finalement, après une longue série de rots, pets, râles déchirants, éructations sinistres et autres manifestations sonores aussi intempestives que déplaisantes, consenti à démarrer.

