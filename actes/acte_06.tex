\chapter{Acte 6}

\noindent J’ai tout juste eu le temps de siffler deux verres de fine (une merveille dont je pourrais vous parler des heures durant sans faire la moindre pause, mais malheureusement le temps presse), fumer la moitié de mon cigare, enfiler un sweat à capuche, une paire de baskets et un gilet pare-balles (j’avais finalement renoncé au costume trois pièces et à la chemise à jabot) avant que mon téléphone se mette à sonner furieusement pour attirer mon attention sur le fait suivant : Titus m’attendait en bas, en compagnie d’un certain nombre de personnes mal intentionnées et armées jusqu’aux dents, et le contrat moral passé avec eux stipulait que je devais descendre les rejoindre au plus vite, sachant qu’il me faudrait des heures pour arriver jusqu’à l’ascenseur si je tombais sur la mère Ouvrard et son âme damnée en forme de chat.

C’est donc la mort dans l’âme que je me suis arraché au confort domestique qui m’enlaçait tendrement et me serrait contre lui avec toute la force de ses petits bras potelés, non sans avoir auparavant jeté un coup d’œil à ma Rousselot P06 Ultramatic 720 (un très bel objet, vraiment, la montre du mec qui a réussi dans la vie, tout à fait le genre de gadget qu’il m’aurait fallu économiser pendant plusieurs générations avant de parvenir à m’offrir, pour info il était zéro heure et vingt-sept minutes) et déposé sur les lèvres charnues de l’étoile de mes nuits (lady Zarina Fragale Di Brizzi, joyau de l’aristocratie italienne, par ailleurs excellente cuisinière et pianiste de grand talent qui aurait pu, avec un peu de travail, prétendre à une carrière de concertiste internationale) un long baiser mouillé censé exprimer toute l’affection que je lui portais d’une part (c’était elle qui m’avait fait cadeau de la Rousselot P06, ça crée des liens, et, comme je vous l’ai dit, elle était parée de toutes les qualités et beautés du monde), et d’autre part la crainte que je nourrissais tel un rat affamé planqué dans le fond de mes tripes de ne peut-être jamais revoir son visage angélique, rayon de soleil qui enchantait ma misérable existence, laquelle misérable existence pouvait fort bien s’achever brutalement durant les heures fatidiques qui allaient suivre, fauchée par une décharge de fusil de chasse en pleine poitrine ou une attaque sauvage à la clé de 12 ou la tronçonneuse.

Il m’a fallu un certain temps pour me débarrasser de Zarina qui refusait obstinément de me laisser partir, usant honteusement de toutes les armes en sa possession, charme, chantage et tentative de corruption de la plus extrême bassesse, arguant que les autres étaient bien assez nombreux pour faire le boulot sans moi, qu’elle ne s’en remettrait jamais s’il devait m’arriver quelque chose, que j’aurais sa mort sur la conscience, ce à quoi j’ai répondu qu’en ma qualité de guide suprême, fin stratège et chef d’unité, il était bien évidemment hors de question que je plante mes hommes pour me réfugier dans les jupons d’une femme, aussi belle et désirable soit-elle. Même si ma pauvre mère, paix à ses cendres (elles se trouvaient ici même, chez moi, bien en évidence sur une copie de buffet Louis XV made in Taïwan, à l’intérieur d’un vase canope à tête de faucon, l’effigie de Kébehsénouf, la divinité chargée de veiller sur les intestins du défunt, touchante attention car le voyage vers l’au-delà peut s’avérer plus long que prévu et on n’est jamais à l’abri de choper une bonne chiasse pendant le trajet), était encore de ce monde et m’implorait sur son lit de mort de renoncer à cette folie suicidaire, je devrais détacher l’un après l’autre ses doigts crochus profondément implantés dans mes chairs afin de prendre le large et voguer toutes voiles dehors vers mon rendez-vous avec l’Histoire.

Morceaux choisis de la joute verbale nullissime qui nous a opposés quelques instants (elle n’a pas dépassé ce stade de joute verbale nullissime, même si j’ai senti plusieurs fois des fourmis dans les doigts et rêver de serrer son petit cou jusqu’à le faire craquer comme du bois mort) :

\textsc{Elle} : Reste ici !

\textsc{Moi} : C’est hélas hors de question.

\textsc{Elle} : Reste ici ou tu le paieras très cher !

\textsc{Moi} : Je casserai mon PEL.

\textsc{Elle} : Je te préviens, je ne serais plus là quand tu rentreras. Si tu rentres.

\textsc{Moi} : Je mettrai tout en œuvre pour y parvenir, et je t’en fais le serment ici même, d’une voix ferme dans laquelle il est impossible de déceler la moindre trace de tremblement ni hésitation : je rentrerai, plus fort et couvert de gloire que jamais !

\textsc{Elle} : Je m’en fiche, je ne serai plus là pour voir ça.

\textsc{Moi} : Dans ce cas, je passerai le restant de ma vie seul comme un chien à pleurer jour et nuit ton absence, noyer mon chagrin dans l’alcool.

\textsc{Elle} : Oui, dans la fine Napoélon de 50 ans d’âge.

\textsc{Moi} : On n’est pas obligé de noyer son chagrin dans de la piquette.

\textsc{Elle} : Je te fais confiance pour picoler, mais je doute fort que tu passes le restant de ta vie à pleurer jour et nuit.

\textsc{Moi} : Je ferai une grève de la faim pour t’obliger à revenir.

\textsc{Elle} : Je ne reviendrai pas.

\textsc{Moi} : Tu me laisserais crever de faim ?

Elle, s’agrippant à mon bras avec l’énergie du désespoir (elle aurait mérité un prix de comédie) : Avec la plus grande joie !

Moi (je n’étais pas mal non plus, entre la tragédie grecque et le théâtre de boulevard) : Dans ce cas, je vais me jeter à corps perdu dans la bataille, et tant pis si je vole au devant d’une mort certaine !

\textsc{Elle} : Non, ne fais pas ça !

\textsc{Moi} : Si, mon amour, il le faut, car sans toi je ne vois plus l’intérêt de continuer à respirer.

\textsc{Elle} : Je ne le permettrai pas !

\textsc{Moi} : C’est gentil, merci.

\textsc{Elle} : Quoi, qu’est-ce qui est gentil ?

\textsc{Moi} : De te préoccuper de moi.

\textsc{Elle} : Tu sais que je t’aime.

\textsc{Moi} : Et moi aussi, je t’aime, plus que tout au monde. Mais je suis un homme, un homme d’honneur, et je me dois d’accompagner mes hommes au combat. Il n’en est pas un, parmi eux, qui ne donnerait sa vie pour moi.

\textsc{Elle} : Et toi ?

\textsc{Moi} : Quoi, moi ? Si je donnerais ma vie pour eux ?

\textsc{Elle} : Oui…

\textsc{Moi} : Euh… Oui, enfin, moi c’est différent. Je suis le chef, et un chef se doit de rester en vie quoi qu’il arrive, ne serait-ce que pour témoigner et présenter ses plus sincères condoléances aux veuves et aux orphelins. Je me présenterai tête basse devant mes juges et assumerai courageusement toute la responsabilité du carnage. Je suppose que tu n’aimerais pas épouser un lâche ?

\textsc{Elle} : C’est une demande en mariage ?

\textsc{Moi} : C’est une façon de parler. Cela dit, si je m’en sors, et je dois m’en sortir pour les raisons que je viens de t’expliquer, il n’est pas exclu que j’envisage la question avec un certain intérêt.

\textsc{Elle} : Parce que pour l’instant, ça ne t’intéresse pas.

\textsc{Moi} : Si, bien sûr, mais il se trouve que j’ai d’autres chats à fouetter. Tu sais ce que c’est, on est pris dans les flots tumultueux de l’existence, et dans le feu de l’action on ne sait plus trop où donner de la tête.

\textsc{Elle} : Mais tu vas en rester en vie.

\textsc{Moi} : Je ne te cache pas que c’est assez dans mes intentions. Oui, je vais rester en vie, et aussi longtemps que possible, parce que je veux continuer à te serrer dans mes bras et m’abreuver à la source fraîche de tes lèvres purpurines jusqu’à la fin des temps.

\textsc{Elle} : Donc tu me promets de revenir sain et sauf ?

\textsc{Moi} : Je te le promets sur la tête de ce que j’ai de plus cher au monde, toi en l’occurrence. Et tu me promets que tu seras là pour m’accueillir ?

\textsc{Elle} : Bien sûr, que je serai là. Les traits tirés, rongée par l’inquiétude, mais je serai là, fidèle au poste !

Moi, lui dévorant goulûment les lèvres : Mon amour !

\textsc{Elle} : Allez, fiche le camp, maintenant !

J’ai entrouvert la porte et embrassé le périmètre du regard pour m’assurer que la voie était libre.

Entendons-nous bien : j’étais prêt à affronter les légions de l’enfer, accepter mes super-pouvoirs et intégrer l’équipe des Secret Warriors de Daisy Johnson pour sauver le monde des métamorphes, filer des baffes à cette andouille de Kim Jong-un jusqu’à ce qu’il ait l’empreinte de mes doigts tatouée à tout jamais dans la peau de ses foutues joues de poupon mégalo, siroter à la paille des hectolitres de CAJOHN’S Sauce Extrême Black Mamba (2,5 millions sur l’échelle de Scoville), me coltiner l’intégrale des 25 saisons de l’Inspecteur Derrick, démembrer des T.rex à mains nues pour les beaux yeux d’une actrice de seconde zone (vous aurez deviné que je parle d’Ann Darrow dans King Kong, film qui, au-delà d’une imagerie coloniale bon enfant à la limite du ridicule, explore sans ambiguïté les pulsions zoophiles de son héroïne, à l’instar d’une Charlotte Rampling dans Max mon amour de Nagisa Oshima, réalisateur obsédé par les déviance sexuelles à qui l’on doigt, entre autres productions tout aussi sulfureuses, les Plaisirs de la chair et l’Empire des sens), TOUT MAIS PAS LA MERE OUVRARD, SA ROBE DE CHAMBRE ELIMEE ET SES PANTOUFLES DE LA MORT !!!

Je me suis permis de glisser une oreille inquiète dans l’espace avoisinant, aussi tendue que la peau sur le visage de cette pauvre (mais très riche) Donatella Versace : rien à signaler, j’avais peut-être des chances de m’en sortir sans dommage. La vie est une lutte sans merci où la plus infime fraction de seconde de relâchement peut infléchir le destin dans un sens ou dans l’autre. Pile vous êtes encore en vie, face la mort s’abat sur vous comme une mouche à merde sur un étron fraîchement pondu. La très vieille et quasi momifiée Maria Ouvrard, et avec elle Korax, le démon à quatre pattes qui la suivait comme son ombre, deux entités auxquelles le premier crétin venu n’aurait pas hésité à donner le bon dieu sans confession, pouvaient à tout moment faire basculer votre petite vie paisible de justicier masqué dans un cauchemar existentiel tel qu’il semblait que les Forces du Mal au grand complet avaient conjugué leurs efforts pour le concevoir. Heureusement pour moi, la vieille taupe avalait des cargaisons de médocs, et parmi eux des doses conséquentes de somnifères ultra puissants qui semblaient bien, pour une fois, avoir produit leurs effets. J’implorais quotidiennement le Malin pour qu’il les rappelle à lui, elle et sa pourriture de chat qui passait son temps à mitrailler mon paillasson de déjections malodorantes, mais le Seigneur des Ténèbres, ancien ange déchu je le rappelle (et même ange de qualité supérieure au rayonnement suprême, à tel point que certaines mauvaises langues prétendent que Dieu n’aurait pas hésité à le virer du Paradis parce qu’il lui faisait de l’ombre, hypothèse hideuse et blasphématoire que j’aime autant ne pas envisager), mais le Seigneur des ténèbres, disais-je, n’avait aucune envie de se les coltiner jusqu’à la fin des Temps, raison pour laquelle il restait sourd à mes supplications (en dépit des messes noires que j’organisais deux ou trois fois par semaine dans les sous-sols de l’immeuble et des nourrissons innocents volés dans les maternités que je sacrifiais à tour de bras pour lui être agréable). J’étais même persuadé que le jour où ses deux fléaux casseraient enfin leurs pipes, chose qu’ils feraient probablement de concert, le Diable les renverrait aussitôt sur Terre, à l’endroit précis d’où ils venaient, pour continuer à faire chier le monde jusqu’à ce que l’immeuble soit officiellement déclaré hanté, avant d’être rasé et transformé en espace vert. Et même alors, il y avait de fortes chances pour que cet espace vert, victimes de rumeurs insolites concernant certaines disparitions ou apparitions inquiétantes, se vide peu à peu de ses visiteurs et finisse par être purement et simplement interdit au public, puis laissé à l’abandon jusqu’à devenir un terrain vague réputé pour ses mauvaises ondes et son atmosphère délétère, un repaire de satanistes, zombies toxicomanes et adolescents en mal de sensations fortes.

Bon, je ne crois pas inutile de rappeler que si je n’étais pas rond comme une queue de pelle, peu s’en fallait. À vrai dire, il est souhaitable de s’alcooliser un tantinet quand on se prépare pour une baston qui risque de s’avérer un poil plus saignante que prévu. Pas trop, bien sûr, sinon on risque de viser à côté et se faire descendre sur un malentendu, ce qui serait quand même de la dernière incivilité, mais assez pour se sentir pousser des ailes et courir au casse-pipe le sourire aux lèvres (plutôt que la peur au ventre, laquelle, comme le soulignait une Elisabeth Badinter au sommet de son art, est mauvaise conseillère).

J’ai le plaisir de vous annoncer que je suis arrivé sans encombre sur le trottoir devant chez moi.

Titus m’attendait, s’est autorisé quelques réflexions dispensables sur mon état général, puis il m’a demandé, s’étonnant sans doute de me voir aussi peu chargé pour une mission d’une telle importance (c’était un peu comme se pointer en maillot de bain avec des claquettes aux pieds pour traverser la Sibérie) : T’as pensé à prendre l’artillerie lourde ?

\textsc{Moi} : T’inquiète, j’ai Manu, mon fidèle 6.35.

Il m’a regardé comme s’il me voyait pour la première fois : Tu te fiches de moi, là ?

\textsc{Moi} : Non, pourquoi ?

\textsc{Lui} : Manu ne ferait pas de mal à une mouche.

\textsc{Moi} : Détrompe-toi, il fait des ravages à bout portant.

\textsc{Lui} : Je voudrais pas te vexer, mais le principe d’une arme à feu c’est pas de faire des ravages à bout portant, mais à distance.

\textsc{Moi} : Ouais, n’empêche que t’as peu de chance de rater ta cible à bout portant.

Lui, découragé : Vu comme ça. Bon, c’est pas grave, de toute façon Sam a pris du rab.

\textsc{Moi} : Sam est là ?

\textsc{Lui} : Ouais, il vient de rentrer de vacances.

\textsc{Moi} : À Zanzibar, oui, je sais.

\textsc{Lui} : Oui. La pêche sous-marine, c’est bien, mais ça ne remplace pas la chasse au gros gibier. Il avait besoin de se dégourdir un peu les jambes, retrouver le feu de l’action, l’odeur de la poudre et les montées d’adrénaline.

\textsc{Moi} : Ça ne m’étonne pas de lui. Et il y a qui d’autre, si ce n’est pas indiscret ?

\textsc{Lui} : Tu verras bien.

\textsc{Moi} : Où est ta voiture ?

\textsc{Lui} : C’est pas ma voiture.

Moi (dans le genre mec chiant qui répète tout ce qu’on dit, pose un milliard de fois la même question et ne comprend rien à rien) : T’as pas pris ta voiture ?

\textsc{Lui} : Non, trop petite.

\textsc{Moi} : Trop petit, ton break 307 ?

Lui, d’une patience d’ange : Oui, trop petit.

\textsc{Moi} : Vous auriez pu me le dire, quand même.

\textsc{Lui} : Pourquoi faire ?

\textsc{Moi} : J’aimerais bien qu’on me tienne un peu un courant, c’est tout. Je te rappelle que c’est quand même un peu moi le chef des opérations spéciales. C’est encore loin ?

\textsc{Lui} : Là-bas, au coin de la rue.

On est arrivés là-bas, au coin de la rue, et mon regard a été aussitôt aimanté par une espèce de gros truc noir (version automobile du monolithe de 2001), le genre de monstre à quatre roues motrices blindé jusqu’aux ouïes utilisé par les forces de l’ordre pour trimballer un certain nombre de choses ou personnalités dites sensibles telles que :

sacs de sport remplis de fric piqué aux trafiquants et autres honnêtes pères de famille partis planquer leurs économies en Suisse ;

tonnes de coke promises à la destruction (après quelques modestes retenues pour le petit personnel et plus si affinités, le reliquat étant éventuellement discrètement repositionné sur le marché pour alimenter l’effort de guerre, on ne peut pas se permettre de gaspiller quand la patrie est au bord de l’invasion, et j’ajoute à titre personnel que je ne vois pas ce qu’il y a de mal à dépouiller des méchants qui utilisent leurs revenus illégaux pour financer des activités criminelles hautement toxiques pour l’humanité, en plus de s’acheter des voitures de sport et des montres de luxe par treize à la douzaine);

repentis de la mafia sous très haute protection qui ont accepté de témoigner contre leur ancien employeur avant de changer de nom, se faire refaire le portrait et disparaître sans laisser de traces ;

tueurs en série cannibales équipés de muselières pour les empêcher de bouffer tout le monde ;

dictateurs sud-américains moustachus ou nord-coréens maniaco-dépressifs qui ont plus d’ennemis que de poils sur le corps (ça vaut surtout pour les nord-coréens maniaco-dépressifs);

ces mêmes véhicules pouvant également être utilisés pour : procéder à des interventions sous testostérone dans des zones de non-droit infestées de junkies, dealers, forcenés et repris de justice en cavale ;

défoncer des repaires de terroristes eux-mêmes défoncés à la fénétylline, armés de fusils d’assaut communistes et ceinturés d’explosif artisanal à base de peroxyde d’acétone ;

traverser des champs de mines en essuyant des tirs nourris d’armes de gros calibre (genre lance-roquettes antichar, mitrailleuse lourde et obus de mortier);

s’inscrire aux Cup Series de la National Association for Stock Car Auto Racing pour s’offrir quelques tours de pistes sur le speedway de Daytona ;

et enfin, aussi étrange que cela puisse paraître le cas a déjà été observé ici et là, aux États-Unis notamment, mais aussi dans la banlieue de Moscou, faire ses courses (à vitesse réduite) au supermarché si on n’a vraiment rien d’autre à se mettre sous la main ni de mieux à faire.

Contre toute attente, le véhicule en question n’était autre qu’un van Chevy G30 de 1992 entièrement reconditionné par les bons soins de Nathan, le frère de Greg, lequel Nathan vivait sous l’emprise d’un démon de la mécanique qui le poussait à s’attaquer aux tas de ferraille les plus insignifiants pour les transformer en machines de guerre crachant le feu et inspirant la terreur sur leur passage.

Dans le cas présent, le pacifique G30 n’avait plus rien à voir avec le véhicule sympathique et débonnaire du bon père de famille qui baise (missionnaire, levrette les jours de fête) sa femme tous les week-end, baise ses go…. euh… non… passe du temps avec ses gosses tous les week-end, fait cramer de la barbaque au barbecue avec ses potes bourrés qui rotent, pètent et fument comme des pompiers pyromanes en lâchant des blagues salaces, du jogging en rentrant du boulot pour ne pas ressembler à un gros tas de graisse ambulant, taille les haies, bine les patates et bouture les bégonias, passe l’aspi une fois par an pour se donner bonne conscience, boursicote avec l’argent du ménage, se fait construire (ou la construit lui-même avec des copains bricolos) une piscine même s’il neige 364 jours par an dans le secteur, se paluche comme un voleur sur des sites de boules quand tout le monde dort à poings fermés, vote à droite parce qu’il y en a ras le bol de tous ces assistés qui n’en foutent pas la rame (quand ils ne violent pas nos femmes et ne nous tabassent pas pour nous piquer notre carte bleue), paie ses impôts rubis sur l’ongle, s’insurge contre la rapacité fiscale qui tue les petits et protège les gros, peste contre le laxisme étatique qui favorise l’invasion migratoire de parasites multicolores et la dégradation islamo-gauchiste du tissu républicain, se lave consciencieusement les mains après avoir pissé, et veille à ce que ses enfants n’aient rien à débourser le jour où le moment sera venu de le mettre en terre ou le carboniser.

Fidèle à son habitude, Nathan avait mis le paquet.

En toute modestie, il considérait ce G30 comme un des chefs-d’œuvre de sa carrière de magicien spécialiste de la transformation auto. Selon lui, plutôt que de se retrouver sur la route à rouler bêtement comme n’importe quel tas de boue grossièrement monté sur quatre roues sans intérêt, l’objet, ou plutôt ce qu’il convenait d’appeler la sculpture mobile, entre art conceptuel et abstraction pure, aurait dû être exposé dans un musée, en bonne place au milieu des fleurons de la créativité humaine. Il suffisait, pour se rendre compte du chemin parcouru entre sa sortie d’usine et le stade ultime de perfection qu’elle affichait aujourd’hui, d’écouter la symphonie fantastique émise par son moteur au premier tour de clé. Ce moteur, un V8 de six litres deux d’origine, avait été entièrement remanié et magnifié dans le sens bien évidemment d’une puissance accrue au-delà du raisonnable, autorisant accélérations décoiffantes et pointes de vitesse dignes d’un avion de chasse, mais aussi pour qu’on ait le sentiment, confortablement installé dans les banquettes en cuir de buffle qui garnissaient le véhicule, d’être à la Scala de Milan en 1950, en train d’assister à l’enregistrement en live de la Tétralogie de Wagner sous la direction incomparable (quoiqu’un peu rigide et exagérément théâtrale, aux dires de certains dont l’opinion m’importe finalement assez peu) du très germanique Wilhelm Furtwängler, directeur de génie que sa relative indolence (euphémisme) pendant la guerre ne saurait en aucun cas désigner comme un sympathisant nazi, j’en veux pour preuves les tentatives restées vaines d’Hitler et Goebbels pour le corrompre, ainsi que son soutien avéré pour la communauté juive. Je rappelle que grâce à lui, de nombreux musiciens et chefs allemands comme Josef Krips, Otto Klemperer, Bruno Walter, Hans Knappertsbusch et même l’étrange Arnold Schönberg, peintre assez médiocre il faut bien le dire mais compositeur de génie, inventeur du dodécaphonisme et cofondateur, avec Berg et Webern, de la Seconde école de Vienne (la Première regroupant des gens comme Haydn, Mozart et ce bon vieux Ludwig van), ont été sauvés de la déportation, ainsi qu’en témoignent certains documents transmis par Curt Riess au général Robert McClure, l’officier américain chargé d’instruire le procès de Furtwängler dans le cadre du projet de dénazification tel que ratifié par les accords de Potsdam en juillet 45. Cette année-là, il faisait un temps superbe à Postdam, un ciel plus bleu que bleu irradiant une luminosité intense, et Staline, Truman et Churchill semblaient très à l’aise dans les jardins du château de Cecilienhof, ancienne résidence de Guillaume de Prusse et sa charmante épouse Cécile de Mecklembourg-Schwerin. Tellement à l’aise que c’est depuis le fond de son transat avec vue imprenable sur la Havel, un cigare offert par Churchill au bec, que Truman donna l’ordre au général Spaatz, futur chef d’état-major de l’US Air Force, de larguer Fat Man et Little Boy sur la gueule des Japs : plus de deux cent mille morts et des milliers de grands brûlés condamnés à endurer les pires souffrances pendant des décennies. On a parfois tendance à l’oublier, mais c’est dans un déluge de feu que la «guerre froide» (d’après George Orwell dans You and the Atomic Bomb) a commencé.

Mais quel que soit mon amour pour la musique sérielle et mon ardent désir que toute la lumière soit faite sur les activités de Furtwängler pendant la période nazie, au-delà des zones d’ombre liées à certaines déclarations pas toujours très heureuses du maître berlinois, je préfère en rester là. Je me contenterai de rappeler que Berta Geissmar, sa fidèle secrétaire de confession juive, n’a jamais cessé d’affirmer et continue à clamer depuis sa tombe que Furtwängler, en dépit de certaines allégations proférées par des esprits chagrins, n’a jamais été l’ami du gros Göring, médiocre mélomane mais excellent mégalomane et toxicomane notoire, lequel semble lui avoir toujours préféré le fils de notables salzbourgeois Herbert von Karajan. Ce dernier, qui ne crachait pas sur le fric, les honneurs et les joies de la vie en (bonne) société, avait eu la présence d’esprit, après des épousailles peu fructueuses - erreur de jeunesse - avec Elmy Holgerloef, une chanteuse d’opérette proche de Goebbels, de se remarier avec Anita Gütermann, riche héritière dont il divorcera à son tour, une fois au sommet de sa gloire, pour convoler avec la succulente Éliette Mouret, mannequin de 17 ans croisé à Saint-Tropez, la Côte d’Azur étant une des autres grandes passions de Karajan. Tel était bien évidemment son droit le plus strict, raison pour laquelle je ne me permettrai en aucune manière d’émettre ne serait-ce que la moindre critique sur sa vie et ses motivations amoureuses. Par contre, si l’on en croit certains rapports diversement circonstanciés, il semblerait que notre homme ait été nettement plus conciliant envers le régime nazi, en tout cas moins regardant, mais en conservant toutefois suffisamment de recul pour ne jamais se commettre totalement. Mais cette fois encore, j’espère que vous ne m’en voudrez pas, le moment me semble assez mal choisi pour rouvrir une polémique sur ce descendant d’un Valaque de Kozani émigré en Saxe et anobli par le prince-électeur Frédéric-Auguste le Juste pour services rendus à la Nation.

En clair, je ne vais pas vous refaire toute l’histoire de la seconde guerre mondiale, ni remonter aux Croisades et encore moins à l’extinction du crétacé ou au Big Bang, pour tenter de vous faire comprendre à quel point Nathan était un génie de la mécanique (à l’égal d’un Furtwängler pour la direction d’orchestre ou un Harmony Korine pour le cinéma d’auteur américain), un alchimiste capable de transmuter la pire poubelle asthmatique perdant ses boulons en chemin en missile de croisière inaltérable.

Il m’a fallu quelques instants pour encaisser le choc de cette rencontre inattendue, puis je me suis retrouvé dans le van en compagnie de Titus, qui m’avait escorté jusqu’ici, de Greg bien sûr, de son frère Nathan, guest-star dont personne n’avait jugé utile de m’avertir de la présence, de mon cher Manu, qu’on ne présente plus, et du capitaine Samuel Girard, ancien des Forces Spéciales reconverti dans la sécurité civile, le gardiennage, le recouvrement de dette et accessoirement l’extorsion de fonds au profit de particuliers ou sociétés aux activités pour le moins douteuses, cuisinier hors pair capable de vous tirer des larmes avec quelques ingrédients savammment disposés dans le fond d’une assiette (cette capacité à émouvoir, en complète contradiction avec l’essence même de son être, restait pour moi un des plus grands mystères de l’univers, à l’égal de l’existence d’un système suffisamment foireux pour avoir conduit à l’émergence d’une espèce aussi nuisible que la nôtre), phalériste détenteur d’une remarquable collection de décorations anciennes (chose qui, en dépit des explications savantes qu’il manquait rarement de déverser à chaque présentation, n’éveillait pas en moi le moindre embryon des prémices du plus infime début de commencement d’intérêt), ceinture noire cinquième dan d’aïkido (il avait suivi l’enseignement de Yoshimura Masayoshi, grand maître de l’école Katsuhiko à Kobe, sur l’île de Honshu, connu pour sa capacité à projeter ou tétaniser ses adversaires par la seule force de la pensée, décédé dans sa cent-onzième année d’une chute malencontreuse dans les escaliers de son dojo), et enfin tireur d’élite capable de défoncer le trou de balle d’un oiseau-mouche à cinq cents mètres de distance.

Il régnait dans l’habitacle une douce ambiance de fin du monde, en plus de quelque chose de parfaitement ridicule et dérisoire que je ne parvenais pas à m’expliquer réellement, si ce n’était qu’il fallait être complètement ravagé pour monter dans un van transformé en char d’assaut avec une poignée de tarés pour aller investir en pleine nuit un repaire de nazillons nazes connus dans le milieu sous le vocable fleuri (des fleurs vénéneuses et malodorantes) de Disciples de la Colère. Et comme si ça ne suffisait pas, Nathan, dont les goût musicaux contrastaient assez violemment avec le profil général, avait choisi, pour nous accompagner dans cette aventure, l’Air de la folie de Lucia de Lammermoor (aria de la scène 1 de l’acte III, quand Lucia perd la raison et poignarde son mari Arturo Bucklaw pendant leur nuit de noces avant de se donner la mort), opéra seria de Donizetti inspiré de La Fiancée de Lammermoor de Walter Scott et créé le 26 septembre 1835 sur la grande scène du théâtre San Carlo de Naples. Je n’ai rien contre l’auteur de Lucrèce Borgia, Don Pasquale et La Fille du régiment, même si je ne suis pas spécialement fan de bel canto, un peu primaire à mon goût, pour ne pas dire chiant, tout comme je trouve très chiantes ces pénibles histoire d’amour impossible sur fond tempétueux de grands espaces battus par les vents, châteaux hantés, marais insalubres, vieilles rivalités familiales et envolées lyriques semblables à des vols de corbeaux planant sur les destinées d’innocentes victimes de malédictions ancestrales, mais un tel choix musical, qui plus est dans une interprétation grésillante de Maria Callas datée de 1953 (entendons-nous bien, ce n’était pas la voix de Maria qui était grésillante mais l’enregistrement, même s’il a pu arriver à la voix de Maria de grésiller un peu en fin de carrière, ce qui n’enlève bien entendu rien à l’immensité de son talent, génie, même, dans certaines interprétations restées légendaires, de Puccini à Delibes en passant par Gounod et Verdi), un tel choix musical, disais-je, que d’aucuns auraient pu juger audacieux (voire disruptif, pour reprendre un concept cher à Jean-Marie Dru et popularisé par un transfuge prépubère et chronophile de la banque Rothschild infiltré au plus haut niveau de l’État), me paraissait à moi totalement incongru. D’autant que ce même air tournait en boucle avec une coupable insistance, chose qui à la fois vous mettait les nerfs en pelote, indiquait on ne plus clairement que Nathan souffrait d’une pathologie mentale sur laquelle il faudrait bien se résoudre à statuer un jour ou l’autre, et in fine vous plaçait dans une espèce d’état de transe suspect dont j’ai vite compris que le but ultime n’était pas de chanter les louanges de l’harmonie mais d’exalter jusqu’à l’euphorie la pulsion de mort qui habite chacun de nous, et ce faisant le pousser à commettre les exactions les plus répréhensibles sans se soucier le moins du monde des conséquences de ses actes, prendre conscience ne serait-ce qu’un seul instant de leurs implications désastreuses dans le cadre de la vie en société et des valeurs fondamentales de la République.

Oui, c’est exactement ce qui se produit si vous écoutez l’Air de la folie de Lucia de Lammermoor en boucle, et tout spécialement dans la version grésillante de Maria Callas enregistrée en 1953 à la Scala de Milan sous la direction de Tullio Serafin, excellent violoniste au demeurant, ancien assistant de Toscanini et grand amateur de (tournedos) Rossini, qui, outre la Callas, a eu le privilège de diriger au cours de sa longue et prestigieuse carrière (le bougre est mort en pleine possession de ses moyens ou presque à quatre-vingt-dix ans, cf la biographie Tullio Serafin, le patriarche du mélodrame par Teodoro Celli \& Giuseppe Pugliese) des cantatrices aussi admirables, inoubliables et bouleversantes que la sulfureuse Elisabeth Schwarzkopf, la langoureuse Victoria de los Angeles et la sémillante Christa Ludwig.

Bref, comme on dit chez nous, ça craignait méchamment du boudin, pour reprendre une expression populaire qui fait clairement référence au caca (et son odeur déplaisante, étant entendu que les expressions populaires, parfois non dénuées de bon sens, sont rarement d’une finesse exemplaire), matérialisé ici par le boudin noir, et par extension à l’anus et la sodomie, le boudin faisant alors office de représentation du pénis qui s’introduit dans le fion de la victime pour y effectuer un certain nombre d’allées et venues plus ou moins dévastatrices, d’où cette fois encore la notion de désagrément afférente à ladite expression. Oui, ça boudinait grave, à tel point que je me suis demandé un instant ce que je foutais dans ce van, par quelle aberration je m’étais retrouvé à monter dedans, dans un état d’ébriété avancée qui plus est (ou était, suivant le niveau de cohérence temporelle que l’on choisit d’adopter), et si je n’allais pas tout bonnement sauter en marche et me précipiter dans le premier taxi venu pour regagner mes pénates. Et tenez, à propos de Pénates, avez-vous seulement la moindre idée de qui ils sont, ou étaient, suivant le niveau de cohé… etc… etc… vertige de la syntaxe (rien à voir avec la sainte taxe de Trump, fallait oser la faire, celle-là) ? D’abord, ce sont des hommes et non des femmes, contrairement à ce qu’on pourrait penser à l’ouïe de la consonance du mot (petit NB vite fait en passant : à propos de la locution «à l’ouïe de», le Littré, hélas tombé en désuétude, nous confirme qu’elle «est bonne, qu’on la dit à Genève et qu’elle appartient au style réfugié, autrement dit celui des protestants français chassés par la révocation de l’édit de Nantes», voilà qui ne manque pas de charme ni de piquant), chose étrange, du reste, car on ne dit pas une mainate, par exemple, mais bel et bien UN mainate, religieux ou autre (il en existe de nombreuses espèces comme le Martin triste, le Mino de Dumont, le Scissirostre des Célèbes ou mainate dubitatif, le Streptocitte à cou blanc et le Basilorne de Céram ou mainate des Moluques, sans oublier l’étourneau de Rothschild et le mainate des mangroves de Floride), sturnidé du genre Gracula qui en plus de bouffer comme un cochon a la capacité inattendue, au même titre que le perroquet sinon mieux, de reproduire à la perfection les sons qui l’entourent, à commencer par les intonations de la voix humaine. Ensuite, figurez-vous que les Pénates n’étaient autres que les dieux romains, probables descendants des Dioscures de Zeus et Léda, chargés de veiller sur le confort du foyer et le bien-être de ses occupants. Oui, à ceci près que mon bien-être à moi tout le monde s’en foutait comme de l’an quarante, à commencer par le ou les dieux censés résider dans les plus hautes sphères de l’atmosphère, et que dans le cas présent ce n’était pas Léda mais Laideron, sa sœur jumelle moche, qui avait enfanté de la situation nauséabonde dans laquelle j’étais allé me fourrer avec la naïveté d’un enfant de chœur tournant le dos au curé de la paroisse.

Au bout d’un quart d’heure à tourner dans la ville, j’ai commencé à me poser des questions sur la nature exacte de notre destination. En d’autres termes, les Disciples de la Colère se réunissaient habituellement dans un pavillon de la banlieue nord, rue Jordan Peshkov (un agent du KGB qui bossait pour la France pendant la guerre froide, exfiltré de justesse avant de tomber aux mains de l’ennemi, liquidé des années plus tard dans cette même rue alors qu’il coulait des jours paisibles dans notre beau pays et pensait bien en avoir fini avec cette période difficile de son existence, mais c’était compter sans la rancune tenace de certains de ses ex-employeurs), et sans avoir un GPS dans le cul ni être doté d’un sens de l’orientation particulièrement performant, j’avais quand même le très nette impression que l’itinéraire choisi n’était pas le plus court chemin pour s’y rendre, même si, j’en conviens, ma perception des choses était peut-être très légèrement altérée par les quelques dizaines d’hectolitres de boissons alcoolisées ingurgités pendant la soirée.

D’où ma question : On va où, au juste ? Allo, y a quelqu’un, une présence, une âme qui vive, une loupiote dans la nuit ? ou dois-je, tel le poète, errer sans fin dans les grottes du destin à la lueur d’une bougie au bord de l’extinction ?

Greg, manifestement agacé par le fait que je n’avais pas respecté le deal, lequel était de se préparer physiquement et mentalement à affronter une bande de tarés qui n’en avaient plus rien à foutre de quoi que ce soit, de véritables boules de haine, bonbonnes de gaz remplies de clous qui n’attendaient qu’une occasion de vous exploser à la gueule : On passe prendre quelqu’un.

\textsc{Moi} : Ah bon ? Qui ça ?

\textsc{Lui} : Sally Robinson, au Sugar \& Spice.

Votre serviteur, cueilli à froid par une déferlante d’interrogation : C’est une blague ?

\textsc{Lui} : Non.

\textsc{Sam} : C’est qui, ce type ?

\textsc{Greg} : Un client à moi.

\textsc{Sam} : T’emmènes tes clients en vadrouille, maintenant ?

\textsc{Greg} : Il est personnellement impliqué dans une sale affaire avec ces types. Pout tout te dire, le gars est gay et ils ont carbonisé au lance-flammes une personne qui lui était chère.

\textsc{Sam} : J’étais pas au courant, j’aime pas ça. Je pars pas en expé avec n’importe qui. Tu le savais, Nath ?

\textsc{Nathan} : J’étais vaguement au courant, oui.

\textsc{Sam} : Vaguement ?

\textsc{Nath} : Greg m’a juste dit qu’on passerait récupérer quelqu’un en route.

\textsc{Sam} : Et toi, Titi ?

\textsc{Titus} : Si on avait cramé ta femme au lance-flammes, je crois que toi aussi tu aurais à cœur de te venger.

\textsc{Sam} : Sauf que j’ai pas de femme, et qu’en l’occurrence, si j’ai tout bien compris, la femme est un homme.

\textsc{Titus} : Et alors ?

\textsc{Greg} : Alors il faudrait voir à ne pas se tromper de camp. Les homophobes c’est eux, pas nous.

\textsc{Sam} : Je suis pas homophobe.

\textsc{Nat} : Allons donc. Tout le monde sait qu’il y a plein d’homophobes dans l’armée.

\textsc{Sam} : Je suis plus dans l’armée. Et puis, si tu vas par là, il y en a plein aussi chez les garagistes.

\textsc{Titus} : Vous disputez pas, les gars. C’est comme les racistes, il y en a plein partout.

\textsc{Nat} : Oui, mais moi je suis un garagiste progressiste qui vit avec son temps. C’est pas parce qu’on passe sa vie les mains dans le cambouis qu’on est une brute pour autant.

\textsc{Sam} : Et moi je suis ni raciste ni homophobe, je vous rappelle que j’ai quand même passé quelques années dans la légion. C’est formateur, la légion, on apprend à vivre avec plein de gens d’origines et de confessions différentes. Par contre, comme tous les pros, j’aime bien savoir avec qui je pars en opé. Risquer ma peau avec un trave qui se trémousse sur une scène de music-hall, très peu pour moi, même si je reconnais qu’il peut y avoir des traves très costauds. J’en ai connu un en Afghanistan, d’origine néerlandaise, qui se produisait dans un clandé de Kandahar, eh bien je peux vous garantir que ce type était une véritable boule de muscles. Et quand je dis boule, c’est pas seulement une image : il mesurait moins d’un mètre soixante et était aussi large que haut, de sorte qu’on pouvait le faire rouler dans les lignes ennemies comme une boule dans un jeu de quilles.

\textsc{Greg} : Ben c’est marrant que tu dises ça, parce Sally Robinson, la femme dont je te parle, enfin le mec dont je te parle, a exactement la même taille et le même profil sphérique que ton pote de Kandahar.

\textsc{Moi} : Sauf que lui, si on le fait rouler, j’ai peur que ses cinquante kilos de nichon l’empêchent d’aller bien loin !

\textsc{Nathan} : Perso, je suis un garagiste ouvert d’esprit et ça me dérange pas d’aller au casse-pipe avec un travelo.

\textsc{Titus} : Et avec un Black ?

\textsc{Nathan} : Avec un Black non plus. Y a des garagistes travelos, vous savez.

\textsc{Titus} : Et black aussi.

\textsc{Nathan} : Oui, et aussi des travelos black qui bossent dans des garages.

\textsc{Moi} : Oui enfin, je sais pas pourquoi, mais j’ai le plus grand mal à imaginer un type passer ses journées les mains dans la graisse et s’habiller en femme à la nuit tombée pour aller faire le guignol sur une scène de music-hall.

\textsc{Greg} : Tout est possible, en ce bas monde.

\textsc{Moi} : Très bas, même, et plus on descend plus on tombe sur des trucs pas très catholiques.

\textsc{Titus} : Catholiques ou musulmans, on s’en fout, du moment qu’ils ne viennent pas nous chier dans les pattes.

\textsc{Greg} : Tous ces trucs bizarres qui vivent dans les abysses.

\textsc{Nathan} : Qui ? les musulmans ?

\textsc{Sam} : Qui vivent dans les abysses de la civilisation.

\textsc{Moi} : Faut pas généraliser. Il y a des musulmans très bien, qui mangent avec une fourchette et un couteau.

\textsc{Greg} : Ouais, à Dubaï.

\textsc{Titus} : Ils ont même du pétrole, une montre suisse et des voitures de sport avec des sièges en cuir de dromadaire.

\textsc{Greg} : Et des autoroutes flambant neuves pour rouler dessus avec leurs voitures de sport.

\textsc{Moi} : Et une cravate en soie sous la djellaba. Si ça c’est pas un signe de civilisation, alors je serais curieux qu’on me dise ce que c’est.

\textsc{Sam} : N’empêche, cravate ou pas, je maintiens que les trucs bizarres qui vivent dans les abysses ne sont pas censés remonter à la surface.

\textsc{Titus} : Peut-être pas, mais ça leur arrive quand même de remonter.

\textsc{Sam} : Le fait est qu’on retrouve souvent des créatures venues du fond des âges venues s’échouer ici et là.

\textsc{Moi} : Vomies par les entrailles de l’océan.

\textsc{Greg} : Ouais, comme ce calmar de quinze mètres de long retrouvé une plage de Californie la semaine dernière.

\textsc{Moi} : Un calmar avec une cravate ?

\textsc{Greg} : Ouais, et une djellaba.

\textsc{Titus} : Un calmar musulman, sans doute.

Nathan, l’œil vif du conducteur toujours sur le qui-vive, au ralenti dans une ruelle à sens unique : Et qu’est-ce qu’il est devenu, ton calmar ?

\textsc{Greg} : Je sais pas. Ils ont dû le mettre dans un grand bocal rempli de formol pour l’exposer au musée d’histoire naturelle de Santa Barbara.

\textsc{Nathan} : On arrive bientôt, les gars.

\textsc{Sam} : Vous êtes toujours sûrs qu’on embarque ce clown avec nous ? Il ne faut pas confondre chasse aux nazis et vacances au club Med. En tout cas, je vous préviens tout de suite qu’il ne faudra pas compter sur moi pour assurer ses arrières. Il vient, il se démerde, et il ne faudra pas qu’il vienne se plaindre s’il se prend une balle dans le cul.

\textsc{Moi} : On n’a qu’à mettre ça au vote. Qui est pour que Sally Robinson vienne avec nous ?

Un doigt s’est levé, celui de Greg, suivi de celui de Nathan, plus hésitant mais se sentant obligé de suivre son grand frère, et enfin celui de Titus, qui en tant que représentant officiel des minorités opprimées et autres espèces en voie disparition (contrairement au garagiste qu’on ne peut décemment pas classer dans cette catégorie, tant les problèmes récurrents qui affectent nos véhicules, sa rapacité notoire et sa déontologie approximative le rangent sans ambiguïté du côté des oppresseurs, voire des presseurs tout court, pressurateurs, pressoirs ou presses hydrauliques qui extraient jusqu’à la dernière goutte de suc de leurs victimes) pouvait difficilement faire autrement, même si on sentait bien qu’il ne barbotait pas dans un lagon d’eau turquoise d’enthousiasme à toute épreuve.

\textsc{Moi} : Proposition retenue à trois voix contre deux. Sally sera des nôtres ce soir, et Greg, qui a eu la bonne idée de nous la fourrer dans les pattes, sera personnellement chargé d’assurer sa protection.

Greg, jetant sur moi un œil chargé de ce qu’il faut bien appeler un doux mélange de réprobation certaine et d’acrimonie à peine dissimulée : Pas de problème.

\textsc{Nathan} : En parlant de calmar….

\textsc{Greg} : On ne parle pas de calmar…

\textsc{Nat} : Non, mais on en parlait y a pas longtemps.

\textsc{Moi} : On a parlé de calmar, c’est vrai.

\textsc{Greg} : Peut-être, mais on ne va pas en parler toute la soirée.

\textsc{Nathan} : Je voulais juste préciser que justement, j’en ai mangé ce soir, des calmars.

\textsc{Moi} : T’as mangé des calmars ?

\textsc{Lui} : Ouais, en boîte.

\textsc{Sam} : Je suis pas fan de calmar.

\textsc{Greg} : Surtout en boîte, c’est sec et ça n’a pas de goût.

\textsc{Moi} : Par contre, frits à la romaine, c’est pas dégueulasse. Surtout avec de la mayonnaise.

\textsc{Titus} : Vous avez vraiment des conversations de merde, les gars.

\textsc{Moi} : Pourquoi, on ne mange pas de calmar en Sierra Leone ?

\textsc{Titus} : Non. On mange des plantains frits, de la soupe de gombo et de la sauce palabre aux feuilles de taro. Quand on mange, parce que la plupart du temps on ne mange pas. La plupart des gens crèvent de faim dans l’indifférence quasi générale, alors tu penses bien qu’on a autre chose à foutre que de s’amuser à découper des calmars en rondelles.

\textsc{Sam} : C’est quoi la soupe de gombo ?

Nathan, passant un coup d’essuie-glace sur le pare-brise crasseux du G30 (dont l’éclairage n’était pas la principale qualité) pour s’assurer de la réalité de sa vision : C’est quoi, ça ?

Greg, qui se trouvait à ses côtés (à la place dite «du mort», du temps où la ceinture de sécurité n’avait pas encore été inventée et où la personne assise à cet endroit était celle qui avait le plus de chances de passer comme une balle à travers le pare-brise), n’avait manifestement aucune réponse significative à apporter à la question : Aucune idée.

Quant à nous, Sam et Titus derrière et votre serviteur relégué tout au fond tel un pestiféré, notre visibilité était loin d’être optimale et ne nous permettait par conséquent pas d’apporter des informations susceptibles d’éclairer le problème d’un jour nouveau.

Globalement, la situation était la suivante : au détour d’une ruelle, aussi sombre et étroite que l’anus d’un ver de terre, une forme venait d’apparaître dans les phares du van.

Cette forme, qui se dressait au milieu de la chaussée tel un spectre décharné de retour sur terre après un long séjour six feet under, avançait lentement vers nous, ayant manifestement du mal à tenir sur ses jambes ramollies par l’inactivité, mais ne semblait pas, en dépit d’une infériorité numérique évidente à tout point de vue, manifester la moindre appréhension concernant le sort funeste qui l’attendait si elle venait à entrer en collision avec le monstre de puissance et d’acier dont Nathan s’efforçait tant bien que mal de conserver le contrôle. Car n’en doutons pas, si une telle chose venait à se produire, ce pantin désarticulé serait renvoyé en pièces détachées dans les flammes de l’enfer et n’aurait aucune chance de survivre une seconde fois au traumatisme de sa destruction, intégrale cette fois, définitive et sans appel, et tous les tubes de colle et tours de passe-passe des magiciens de l’au-delà n’y pourraient rien changer.

Court vêtue d’oripeaux malodorants qu’elle semblait avoir empruntés à un épouvantail et entassés les uns sur les autres sans se soucier du résultat, la chose avançait vers nous en se dedandinant. Je dois maintenant vous révéler qu’il s’agissait d’une femme d’origine manifestement africaine, d’une beauté assez confondante quand on se donnait la peine de l’examiner un peu mieux sous toutes les coutures. La peau très noire tendue sous les muscles saillants, les membres longs et fins, la poitrine ferme et menue, la taille de guêpe, le cul rebondi, le cou interminable et les dents d’une blancheur éclatante dissimulées derrière le rideau pulpeux de lèvres écarlates, tout semblait indiquer, au-delà des apparences qui ne plaidaient pas en sa faveur, qu’elle était le rejeton de quelque prestigieuse lignée. Je pourrais aussi vous parler de ses yeux, son regard, mais des centaines de pages seraient nécessaires pour en faire le tour. Une telle interruption, pour intéressante qu’elle soit, ne manquerait pas de plomber le rythme de l’action en cours, particulièrement dynamique en cet instant vous n’êtes pas sans l’avoir remarqué. Je pourrais aussi, durant quelques centaines de pages supplémentaires, lui tartiner le cul des éloges les plus dithyrambiques qui soient, tant cette partie de son anatomie était ronde et charnue, moulée à la louche dans le lait entier du désir et la sexualité la plus débridée, sculptée et polie de main de maître dans le diamant brut de l’extase et la sensualité, mais le résultat, pour terriblement excitant qu’il soit, serait tout aussi préjudiciable au bon déroulement de l’action.

Greg, à son frère : Qu’est-ce que tu fabriques ?

\textsc{Nathan} : Ben rien, je roule.

\textsc{Greg} : Justement, il faudrait peut-être songer à t’arrêter.

\textsc{Nathan} : Je la connais pas, moi, cette bonne femme.

\textsc{Greg} : C’est pas une raison pour l’écraser.

\textsc{Nathan} : J’aime pas ça.

\textsc{Greg} : Arrête-toi, je te dis !

Nathan, sautant à contrecœur sur la pédale de frein : Ça va, c’est bon, je m’arrête.

\textsc{Sam} : Elle est canon !

\textsc{Moi} : Je confirme : bizarrement fringuée, mais canon. T’en penses quoi, Titi ?

\textsc{Titus} : Bof.

\textsc{Moi} : Comment ça, bof ?

\textsc{Lui} : Je te rappelle que j’ai une femme et des gosses, Djef.

\textsc{Moi} : Et alors ? Moi aussi j’ai une femme, ça ne m’empêche pas d’apprécier la beauté des autres.

\textsc{Lui} : C’est pas vraiment une femme.

\textsc{Moi} : Je te demande pardon ?

\textsc{Lui} : C’est une femme, bien sûr, mais c’est pas comme si vous étiez mariés et aviez des gosses.

\textsc{Sam} : Je sais pas d’où elle sort, mais j’avoue que ça ne me déplairait pas de faire plus ample connaissance, apprendre à mieux se connaître.

\textsc{Titus} : Vous êtes vraiment des porcs, les gars. Surtout toi, Djef.

\textsc{Moi} : On n’est pas des porcs, on est des amateurs de belles choses, et contrairement à toi on est particulièrement sensibles au charme des articles exotiques.

\textsc{Sam} : Exact.

\textsc{Titus} : Je vous rappelle qu’on a une mission en cours, on n’est pas là pour faire du tourisme sexuel.

\textsc{Nathan} : Je suis d’accord que c’est pas forcément une bonne idée de s’arrêter.

\textsc{Greg} : Oui ben arrête-toi quand même !

\textsc{Nathan} : C’est ce que je suis en train de faire, figure-toi.

\textsc{Greg} : On s’en débarrasse en vitesse et on file rue Théo Cazenave récupérer mon client.

\textsc{Sam} : Ça non plus c’est pas une bonne idée.

On n’arrête pas aussi facilement un char d’assaut qu’une planche à roulettes. Avec le G30 de Nathan, il fallait commencer à freiner la veille pour espérer s’arrêter le lendemain. Pour un type comme lui, qui aimait pulvériser les chronos, s’arrêter n’était pas la préoccupation première. Il voyait l’existence comme une longue ligne droite sur laquelle on aurait pu accélérer sans fin jusqu’à atteindre le septième ciel. En fait, il aurait dû être pilote de chasse. Autrement dit, le V8 du G30 avait été gonflé au point de frôler la désintégration à chaque instant, mais il fallait prendre de l’élan et sauter à pieds joints sur la pédale de frein pour réussir son coup.

Par miracle, le van s’est arrêté à temps, à quelques centimètres de sa proie, en l’occurrence la sublime princesse-zombie court vêtue d’oripeaux malodorants qui déambulait en pleine nuit au milieu de la rue sans se soucier du qu’en-dira-t-on, et après tout on ne pouvait pas lui en vouloir car quand on est aussi magiquement belle il n’y a aucune raison de se faire chier à se soucier de ce que pensent les autres, ce ramassis de minables qui ne vous arrivent pas à la cheville et n’ont rien de mieux à faire que de vous cracher à la gueule toute la bile accumulée au cours de leur misérable existence. Bon, elle semblait quand même avoir pris quelques substances qui ne justifiaient peut-être pas d’un état sanitaire irréprochable sur le plan mental, mais malgré tout ça, en dépit de ces approximations difficilement compréhensibles pour le commun des mortels, il existait au plus profond d’elle-même des choses que rien ni personne ne pourrait jamais atteindre ni altérer.

Nathan, qui détestait s’arrêter, a aussitôt descendu sa vitre pour balancer une bordée d’injures à l’intéressée (bordée dont je préfère, afin de ménager la sensibilité du lecteur, m’abstenir de répéter les termes exacts, particulièrement grossiers il faut bien le dire, Nathan, je le rappelle à sa décharge, étant garagiste, et le garagiste dans son ensemble n’étant pas spécialement réputé pour la tenue de son langage). Eh bien vous allez rire, ou pas je n’en sais rien, mais les imprécations en question n’ont pas eu l’heur de l’émouvoir le moins du monde.

Strictement rien à foutre, à peine un regard pour ce pauvre Nathan qui s’époumonait la tête à la fenêtre.

Tant et si bien qu’il a fini par sortir de la voiture, suivi de Greg, Sam et moi-même, qui tous (à part Titus qui, respectueux des valeurs et traditions en vigueur dans son pays, savait rester digne en toute circonstance) souhaitions nous assurer que la mystérieuse apparition n’était pas le fruit d’une hallucination collective (bon j’avoue qu’en ce qui me concerne je faisais surtout une petite fixette sur sa plastique enchanteresse, et caressais secrètement le projet de l’enlever pour l’attacher nue au fond d’une cave et abuser d’elle à longueur de journée jusqu’à ce que mort s’ensuive, projet ignoble s’il en est, et rêve inaccessible condamné à rester enchaîné à tout jamais dans les bas fonds de mes plus viles pulsions, la boue fantasmatique d’étreintes telles que l’humanité n’en a plus connu depuis que le dernier rhinocéros laineux s’est éteint dans les steppes de Mongolie).

La mystérieuse apparition n’était pas le fruit d’une hallucination collective, je vous le confirme avec émotion, mais ne semblait pas excessivement perméable aux informations que nous tentions de lui faire passer aussi élégamment que possible, à savoir qu’il aurait très aimable de sa part de dégager le passage afin que nous puissions continuer notre route sans avoir à faire usage de la force, chose qui, contrairement à ce que l’arsenal que nous trimballions pouvait laisser penser, aurait été pour nous (au moins pour Greg et moi-même, qui étions d’une nature plutôt affectueuse, Nathan et surtout Sam pouvant se montrer assez belliqueux à l’occasion) la source de la plus vive contrariété.

Aussi lui ai-je dit, de ma voix la plus douce, en la prenant par le bras avec une avidité certaine : Venez, mon enfant, je vais vous reconduire sur le trottoir.

Sa peau était douce et soyeuse comme de la mousse, et la sensation de son contact si délicieuse que mon cœur a bien failli s’arrêter de battre. Cela dit, j’avais beau la pousser fermement, tout en faisant preuve de la plus extrême délicatesse bien entendu, du plus grand soin pour éviter d’infliger ne serait-ce que la plus légère flétrissure à son épiderme, l’étrange créature ne bougeait pas d’un millimètre.

\textsc{Greg} : Il ne faut pas rester là, mademoiselle, c’est dangereux.

Sam, dans un autre registre : Bouge ton cul de là, si tu ne veux pas qu’il t’arrive des bricoles.

\textsc{Nathan} : Laisse tomber, tu vois bien qu’elle est complètement défoncée.

\textsc{Moi} : Messieurs, s’il vous plaît ! Essayez de faire preuve d’un peu de compassion, au moins une fois dans votre vie.

Finalement, la fille s’est remise en marche et dirigée droit vers l’arrière du véhicule, à l’endroit où se trouvait Titus, lequel était en train de se curer les ongles avec un couteau de chasse.

Il a levé la tête et la fille a planté ses yeux droit dans les siens, accrochant son regard pour ne plus le lâcher.

Quelques minutes plus tard, alors qu’ils étaient toujours en train de se fixer intensément sans dire un mot, j’ai commencé à trouver le temps long et me suis dit que le moment était venu de rompre le charme. Il s’agissait peut-être d’un de ces coups de foudre dont les gens parlent avec admiration et que tout un chacun rêve de connaître au moins une fois dans sa vie, cette espèce d’éclair qui fait que deux êtres qui se rencontrent pour la première fois deviennent inséparables en une fraction de seconde, irrésistiblement attirés l’un vers l’autre comme les deux moitiés d’un même corps qui se retrouveraient enfin après s’être cherchés durant des siècles aux quatre coins de la planète. Et dans ce cas, même quelqu’un comme Titus, qui était la droiture même, n’hésiterait pas à bazarder la femme et les enfants qu’il adorait par dessus tout pour partir à l’aventure avec cette inconnue. Oui, l’être humain fait parfois preuve de comportements irrationnels, en totale contradiction avec ses convictions les plus affirmées. Par quelle mystérieuse alchimie deux êtres qui ne se connaissent ni d’Eve ni d’Adam ont soudain le sentiment d’être une seule et même personne, éprouvent l’un pour l’autre une attraction aussi instantanée que fusionnelle, au point de perdre tout repère et se laisser aller corps et âme dans la tourmente ? S’agit-il de la rencontre inopinée de deux profils fantasmatiques en parfaite adéquation, lesquels, relégués depuis des lustres dans les oubliettes du refoulement, explosent littéralement au contact l’un de l’autre, produisant un feu d’artifice d’étincelles contradictoires qui cloue le sujet sur place et le déstabilise au point de tout laisser tomber pour s’y consacrer sans réserve ? C’est complexe, et il se peut aussi que les phéromones copieusement émises par les protagonistes sous le coup de l’émotion jouent un rôle non négligeable dans cette affaire. Ou alors, il s’agit de quelque chose qui échappe totalement à notre entendement, auquel cas il n’est d’aucune utilité de continuer à se gratter les méninges jusqu’au sang. On n’y comprendra jamais rien, et le mieux, dans ce cas-là, est de prendre le parti de s’en foutre et se laisser porter sans résistance par les flots écumeux du destin.

Quoi qu’il en soit, ne serait-ce que par égard pour Bérénice et les enfants, que je connaissais depuis toujours et considérais un peu comme les miens, je ne pouvais décemment pas laisser Titus s’abîmer dans les affres de l’amour fou, assister sans réaction au spectacle désolant de la passion dévastant ce qu’il avait mis tant d’années à construire, je veux bien sûr parler de cette réussite familiale exemplaire qui constituait pour nous une inépuisable source d’admiration et d’inspiration.

\textsc{Moi} : Titus ?

Lui, sans quitter la fille des yeux : Quoi ?

\textsc{Moi} : Tu fais quoi, là ?

\textsc{Lui} : Rien.

\textsc{Moi} : Okay. Faut y aller, maintenant.

\textsc{La fille} : Non, lui pas aller.

\textsc{Sam} : Comment ça, pas lui ?

\textsc{La fille} : Lui pas aller.

\textsc{Nathan} : C’est quoi ces conneries ?

\textsc{La fille} : Lui grand danger pour.

\textsc{Sam} : Elle est complètement cinglée ! Fichons le camp d’ici, les gars.

La fille, à Titus : Toi pas aller.

\textsc{Titus} : Pas aller où ?

\textsc{La fille} : Pas aller, courir grand danger.

\textsc{Titus} : Je peux savoir qui vous êtes ?

\textsc{La fille} : Toi pas aller, grand danger.

\textsc{Greg} : Elle commence vraiment à me faire flipper.

\textsc{Sam} : Oui, moi aussi.

\textsc{Nathan} : Même chose. Je serais assez d’avis qu’on se barre d’ici vite fait. Je vous rappelle qu’on doit encore aller chercher l’autre folle.

\textsc{Moi} : On y va, on y va. T’en penses quoi, Titus ?

\textsc{La fille} : Titus pas aller.

\textsc{Titus} : Pas aller où ?

\textsc{La fille} : Grand danger !

\textsc{Greg} : Je crois qu’on n’en tirera rien de plus.

\textsc{Titus} : Où est-ce que je ne dois pas aller ?

\textsc{La fille} : Toi sortir voiture, maintenant, tout de suite, et rentrer chez toi !

\textsc{Sam} : Pas aller, pas aller, commence à me les briser, celle-là, avec ses pas aller, pas aller !

La fille, en agitant ses ongles de trente centimètres de long sous le nez de Titus : Toi sortir voiture et rentrer chez toi. Pas aller avec les autres, ou toi mourir ce soir.

\textsc{Nathan} : Laisse tomber, elle est folle !

\textsc{Titus} : N’empêche que j’ai pas envie de mourir ce soir, moi. J’ai une femme et des gosses.

\textsc{Sam} : Tu ne vas pas me dire que tu crois à ces conneries !

\textsc{La fille} : Pas conneries ! Si lui aller, lui mourir. Grand danger !

\textsc{Moi} : Tu t’appelles comment, ma chérie ?

\textsc{La fille} : Moi pas chérie ! Moi faire rêves, voir et entendre choses. Moi voir lui dans rêve. Lui pas aller, ou mourir ce soir.

\textsc{Moi} : Et tu as vu quoi d’autre, mon petit lapin ?

\textsc{La fille} : Moi pas lapin. Moi voir gens avec armes tirer partout et lui mourir.

Moi, en montrant Sam : Et lui, tu l’as vu ?

\textsc{La fille} : Pas vu lui. Juste Titus, tête exploser et cervelle gicler partout. Lui pas aller, ou jamais plus revoir femme et enfants.

\textsc{Moi} : Euh… bon, d’accord. Tu fais quoi, sinon ?

\textsc{La fille} : Moi rien faire. Juste voir et entendre choses.

Puis, s’agrippant à Titus pour essayer de le faire sortir de la voiture : Toi sortir voiture, maintenant. Toi pas aller ou mourir ce soir.

Nathan, envoyant des coups d’accélérateur pour signifier son impatience : Bon, on y va. J’ai pas que ça à faire, moi.

\textsc{Greg} : Qu’est-ce que t’as à faire ?

\textsc{Nathan} : Ben… rien. Enfin si, j’ai mes plantes à arroser.

\textsc{Greg} : Tu te fous de ma gueule ?

\textsc{Nathan} : Non, c’est des plantes spéciales que j’ai fait venir d’Amérique du Sud. On ne peut les arroser qu’à une heure précise de la nuit, sans quoi elles dépérissent et dégagent des vapeurs toxiques.

\textsc{Sam} : N’importe quoi ! Je crois que j’ai eu ma dose de conneries pour ce soir !

\textsc{Nathan} : Et puis ils passent Bagne de femmes avec Zarah Leander à trois heures du mat.

\textsc{Sam} : Bagne de femmes ? c’est quoi, ce truc ?

\textsc{Nathan} : Un vieux film de Douglas Sirk. À l’époque, en 36, Sirk ne s’appelait pas encore Sirk mais Sierck. Il bossait pour la principale société de films allemande mais s’est retrouvé embringué dans une sale histoire. Quand sa première femme a appris qu’il s’était remarié avec une Juive, elle l’a balancé aux Boches et il s’est retrouvé obligé de quitter le pays en quatrième vitesse, sans avoir pu revoir son fils de dix ans, Klaus, qu’il avait eu avec la précédente femme en question dont j’ai oublié le nom.

\textsc{Sam} : Ça tombe bien on s’en fout.

\textsc{Nathan} : N’empêche que ce type a eu une vie extraordinaire. Il a quitté l’Allemagne nazie, s’est retrouvé à élever des poulets dans la banlieue de San Francisco, avant de se remettre au cinoche et tourner les films qui l’ont rendu célèbre, comme Désir de femme, Le Secret magnifique, Tout ce que le ciel permet, ou encore Écrit sur du vent avec Rock Hudson, Lauren Bacall et Dorothy Malone. Excusez du peu.

\textsc{Sam} : Super ! Bon, on y va, maintenant ?

\textsc{Nathan} : Mais le drame de sa vie, c’est que…

\textsc{Sam} : Le drame de ma vie, c’est de bosser avec des amateurs qui s’intéressent à tout sauf la mission en cours. Je comprends les mecs qui préférèrent bosser en solo, au moins ils ne sont pas obligés de composer avec une bande de bras cassés dont on ne sait jamais ce qu’ils vont faire ou pas.

La fille, toujours en train de tirer sur Titus qui refusait de sortir de la voiture : Toi venir, venir avec moi.

\textsc{Moi} : Mais enfin, vous ne m’enlèverez pas de l’idée qu’il se passe quand même des trucs bizarres sur cette planète ! Allez, ma petite caille, il faut le laisser tranquille, maintenant.

\textsc{La fille} : Moi pas petite caille. Toi laisser lui partir ou lui mourir ce soir.

\textsc{Nathan} : Je sais que tout le monde s’en fout, mais ça doit quand même être terrible d’apprendre que son fils est mort sur le front russe sans qu’on n’ait jamais pu le revoir ni le serrer une dernière fois dans ses bras. Tout ça à cause de sa salope d’ex-femme nazie !

\textsc{Greg} : Les drames de la séparation.

\textsc{La fille} : Gentil Titus. Toi laisser autres et venir avec moi.

\textsc{Moi} : Je crois que t’as un ticket, Titus. Pense à ta femme et tes gosses, tu ne vas pas tout envoyer balader sur un coup de tête.

\textsc{Titus} : Arrête tes conneries ! Non, je voudrais juste savoir qui est cette fille et ce qu’elle attend de moi.

\textsc{La fille} : Toi sortir voiture et venir avec moi.

\textsc{Titus} : Je ne sais même pas comment tu t’appelles.

\textsc{La fille} : Moi Atiena, gardienne de la nuit. Toi venir.

\textsc{Titus} : Venir ? Mais où ça ?

\textsc{Atiena} : Suivre moi dans chambre.

\textsc{Titus} : Quelle chambre ?

\textsc{Atiena} : Chambre hôtel, tout près d’ici.

\textsc{Sam} : Et voilà, j’en étais sûr !

\textsc{Titus} : Quoi ?

\textsc{Sam} : C’est une pute, j’en étais sûr !

\textsc{Atiena} : Moi pas pute. Moi Atiena, gardienne de la nuit.

\textsc{Nathan} : Gardienne de mon cul, oui !

\textsc{Greg} : Nathan, s’il te plaît, tu sais que j’ai horreur de la grossièreté.

\textsc{Atiena} : Moi Atiena, gardienne de la nuit.

\textsc{Moi} : Oui oui. Alors écoute, Atiena, ma petite caille en sucre…

\textsc{Atiena} : Moi pas caille sucre. Moi Atiena, gardienne de la nuit.

\textsc{Moi} : Oui, bien sûr, il n’y a aucun doute là-dessus et tu ne trouveras personne ici pour prétendre le contraire. N’est-ce pas, les gars ?

\textsc{Greg} : Personne.

\textsc{Sam} : Absolument personne.

\textsc{Moi} : Non, vois-tu, ce que je voudrais te faire comprendre, ô Atiena gardienne de la nuit, c’est que nous avons des choses à faire et que Titus ne va pas pouvoir aller avec toi dans chambre hôtel tout près d’ici. Toi comprendre ou moi pas parler français ?

\textsc{Atiena} : Toi méchant. Titus venir avec moi dans chambre et lui pas mourir ce soir.

\textsc{Moi} : Et il en pense quoi, Titus ? Lui vouloir venir dans chambre hôtel tout près d’ici ?

\textsc{Titus} : Lui pas vouloir mourir ce soir.

\textsc{Sam} : Je le crois pas ! Tu ne vas quand même pas te laisser influencer par cet épouvantail sortir de nulle part !

\textsc{La fille} : Moi pas épouvantail. Moi Atiena, gar…

\textsc{Sam} : … dienne de la nuit, oui, je sais !

\textsc{Nathan} : Il a raison, Titus. Tu ne vas tout de même pas croire les balivernes de cette… cette…

\textsc{Sam} : Pute, tu peux le dire.

\textsc{Atiena} : Moi pas pute, moi gardienne de la nuit. Marcher dans rue pour sauver gens qui vont mourir.

On en était là de nos aventures fascinante avec Atiena, la gardienne de la nuit, qui était d’ailleurs elle-même une créature fascinante dont les yeux étrangement bridés (caractéristique des Khoïsan du Kalahari, proches des San de Namibie, longtemps persécutés tant par ces enfoirés de Bantous, leurs ennemis naturels, que par ces pourritures d’Afrikaners, à tel point qu’ils mériteraient aujourd’hui d’être classés dans la catégorie des espèces en voie de disparition) exerçaient sur nous, et Titus en particulier, une emprise irrésistible, et dont la plastique assez sidérante (désolé de vous le dire mais il fallait bien que quelqu’un le fasse, et ce quelqu’un, en tant qu’auteur responsable, ne pouvait être que moi) faisait éclore en nous un florilège de sentiments dont le désir n’était pas totalement absent, et je dirais même, si je devais être parfaitement honnête (et, sur la vie de ma mère, Dieu m’est témoin que je m’y efforce autant que faire se peut), occupait une place de plus en plus prépondérante parmi nos préoccupations du moment (au moins les miennes, et, je pense pouvoir l’affirmer sans trop m’avancer, celles de ce cher Titus).

On en était donc là, disais-je, passablement englués dans les miasmes de l’expectative, quand le téléphone de Greg s’est mis à retentir avec une violence inouïe, nous faisant tous sursauter comme des lapereaux effarouchés (sauf Atiena, bien sûr, la gardienne de la nuit, entièrement focalisée sur l’objectif qu’elle s’était fixé de faire sortir Titus de la voiture pour l’emporter dans sa tanière et le sauver d’une mort certaine, que j’imaginais sans peine longue et douloureuse).

Greg, parti à la recherche de l’objet qui éructait dans le fond d’une de ses nombreuses poches (Greg adorait les poches, il en avait toujours plein sur lui, à commencer par un de ces affreux gilets multipoches style grand reporter ou aventurier qu’il portait jour et nuit hiver comme été, dans lequel on trouvait toujours au moins une brosse à dents, un paquet de chewing-gums sans sucre, quelques élastiques et bouts de ficelle, des capotes, de la menue monnaie, une boîte d’allumettes et un couteau suisse alors qu’il ne fumait pas et n’avait jamais mis les pieds à Lausanne, même pour aller planquer du fric à la Banque Cantonale Vaudoise) : Qu’est-ce que c’est encore que ça ?

Ça, c’était Sally Robinson, qui commençait à en avoir marre de faire le pied de grue sur le trottoir comme une vulgaire tapineuse devant le Sugar \& Spice.

\textsc{Greg} : Salut Sally. Ça va ?

\textsc{Sally} : Non, ça ne va pas. Qu’est-ce que vous foutez, bordel ? Ça fait une demi-heure que je vous attends !

\textsc{Greg} : Oui, je sais, je suis vraiment désolé.

\textsc{Sally} : J’en ai marre, moi ! Je n’arrête pas de faire aborder par des pervers qui essaient de m’embarquer.

\textsc{Greg} : Oui, en effet.

\textsc{Sally} : Quoi, en effet ?

\textsc{Greg} : En effet, ce sont vraiment des pervers.

\textsc{Sally} : Ça veut dire quoi, ça ?

\textsc{Greg} : Rien, juste que ça ne doit pas être marrant de se faire aborder sans arrêt par des pervers qui essaient de vous embarquer.

\textsc{Sally} : Non, en effet, c’est vraiment pas marrant. Vous faites quoi, en ce moment ?

\textsc{Greg} : On discute avec la gardienne de la nuit.

\textsc{Sally} : La quoi ?

\textsc{Greg} : La gardienne de la nuit, une fille qu’on a rencontré en chemin et dont on a un peu de mal à se débarrasser.

\textsc{Sally} : Vous vous foutez de moi ?

\textsc{Greg} : Pas du tout. Croyez bien que je suis le premier surpris de me retrouver dans une telle situation.

\textsc{Sally} : C’est qui, cette gardienne de la nuit ?

\textsc{Greg} : Une Black dans les vingt-trente ans, superbe, qui parle français comme une vache espagnole. On n’en sait pas plus, sinon qu’elle refuse de laisser partir Titus.

Atiena, gardienne de la nuit : TITUS PAS ALLER !!!

\textsc{Sally} : C’est qui, Titus ?

\textsc{Greg} : Un collègue, black lui aussi, costaud, venu nous filer un coup de main pour nettoyer le trou à rats que vous savez.

\textsc{Sally} : Vous êtes où ?

\textsc{Greg} : Pas très loin, rue des Nénuphars.

Ça faisait des lustres qu’on n’avait pas vu un nénuphar pousser dans le secteur, mais sans doute qu’il y en avait eu du temps où le quartier n’était qu’un vaste marécage infesté de malandrins et femmes de petite vertu. Aujourd’hui la rue n’était guère plus avenante mais nettement moins humide, même si les chats, les chiens et les noctambules avinés ne se privaient pas de pisser sous les portes cochères. Peut-être que la gardienne de la nuit était déjà là des siècles auparavant, se dressant devant le voyageur égaré pour lui interdire le passage et le sauver ainsi d’une mort certaine. Sa peau noire, ses yeux bridés et ses cheveux hirsutes devaient leur flanquer la trouille de leur vie, à une époque où on buvait de l’eau croupie et brûlait des sorcières à tous les coins de rues. La pauvre aurait fini sur le bûcher, sans aucun doute, pour apparence délictuelle et commerce avec le Diable. Comment avoir la peau aussi noire si on n’avait pas passé une bonne partie de sa vie dans les flammes de l’enfer ? Cette carbonisation et ces tatouages rituels qu’elle arborait un peu partout sur son corps magnifique, sans parler de ces breloques et autres amulettes suspectes qui tintaient à chacun de ses pas, étaient la preuve irréfutable d’une existence vouée au blasphème et la dépravation. Et si elle avait survécu jusqu’ici, renaissant sans cesse de ses cendres, c’était bien la preuve supplémentaire que Lucifer la considérait comme un de ses émissaires les plus efficaces sur terre et veillait sur elle comme sur la prunelle de ses yeux maléfiques.

\textsc{Sally} : Je connais, c’est à dix minutes à pied. Je viens vous rejoindre, j’en ai marre de faire du sur place.

\textsc{Greg} : Vous êtes sûr ?

\textsc{Sally} : Oui. Attendez-moi là-bas, j’arrive.

Greg, avant de raccrocher : Bon, comme vous voudrez.

\textsc{Moi} : Qu’est-ce qui se passe ?

\textsc{Greg} : Sally Robinson en a marre de poireauter, elle vient nous rejoindre ici.

Atiena, toujours en train de tirer sur Titus qui avait déjà un pied hors du van : TOI PAS ALLER, VENIR AVEC MOI !

J’avais le sentiment diffus, mais quand même assez preignant, que la situation était en train de se barrer en couille. Vous savez, et je dis ça en toute modestie, quand on a une certaine expérience du terrain comme c’est mon cas, on finit par sentir ces choses-là. Il y a une petite sonnette d’alarme qui est toujours à l’affût dans votre cerveau et se met en branle à la moindre anomalie. Et, depuis quelque temps, même si elles n’étaient pas aussi énormes qu’un troupeau de brontosaures en train de brouter (broutosaures) dans les vastes prairies du Crétacé, les anomalies semblaient s’accumuler avec un regain de vigueur inquiétant. Certains appellent ça l’instinct. Pour expliciter le concept, je vais reprendre l’exemple du brontosaure sus-cité. L’animal, énorme, donc, pait (et pète aussi, ce qui produit à chaque fois une déflagration de tremblement de terre) tranquillement dans les vastes prairies du Jurassique supérieur, indifférent à ce qui se passe autour de lui tellement son énormité lui tient lieu de protection, de garantie que nul ne viendra lui casser les couilles sous quelque prétexte aussi fallacieux que dérisoire que ce soit. Il pait, pète, et le temps s’écoule ainsi avec une lenteur toute préhistorique, tandis qu’il arrache des grosses touffes d’herbe bien grasse et juteuse avec sa minuscule tête accrochée au bout d’un cou interminable. On voit bien qu’il n’est pas d’une intelligence folle, qu’il ne sort pas de Cambridge, Oxford ou Harvard (on ne voit d’ailleurs pas très bien comment il aurait réussi à franchir la porte sans faire s’écrouler le bâtiment), mais il s’en branle, le brontosaure, il s’en secoue la nouille énergiquement, parce que quand on est aussi énorme que lui on n’a pas besoin d’être un fils de bonne famille ni de faire de longues études pour s’en sortir. Non, tout ce qu’on a à faire, c’est brouter et péter tranquillement dans les vastes prairies du Tithonien (d’après Tithon, le fils de Laomédon, enlevé par la déesse de l’Aurore avant de finir transformé en cigale), en se déplaçant le moins possible pour éviter de se fatiguer (parce que c’est pas forcément facile, même quand on est très très costaud, de véhiculer quinze tonnes et vingt mètres de barbaque sur quatre courtes pattes de tabouret), brouter encore et encore jusqu’à ce qu’il ne reste plus un seul brin d’herbe dans les vastes prairies du Jurassique supérieur, je vous parle de ça il y a au moins cent cinquante millions d’années. Donc le brontosaure il est là, en train de brouter tranquillement, se servant de son très long cou pour aller arracher des grosses touffes d’herbe au sommet des arbres ou au fond des précipices, lâchant des caisses monumentales qui font des trous énormes dans la couche d’ozone, et pis vla ti pas que soudain une petite cloche se met à tinter au fond de sa minuscule cervelle de sauropode de la formation de Morrison, bien avant Little Big Horn, le siège de Fort Alamo et la ruée vers l’or. À l’époque, il n’y avait pas l’ombre d’une plume de Sioux ou de Cheyenne dans les Grandes Plaines de l’Ouest, qui n’étaient d’ailleurs encore pas les Grandes Plaines de l’Ouest mais juste un trou à rats infesté d’une flopée de dinosaures dont certains avaient des dents aussi longues et tranchantes que des couteaux de chasse. Et parmi eux, il y avait toute une tripotée de grands théropodes carnivores, comme l’Allosaure ou le Mégalosaure, qui n’étaient pas des plus commodes, même s’ils n’étaient pas aussi énormes que notre pote brontosaure qui passait le plus clair de son temps à s’empiffrer dans ce qui deviendrait un jour, un bon paquet de millions d’années plus tard, le théâtre de l’une des plus fantastiques épopées de l’histoire coloniale du XIXe siècle, je veux bien sûr parler de la légendaire Conquête de l’Ouest. Alors certes, il était loin de se douter qu’un bipède du nom de Steven Spielberg tournerait un jour Jurassic Park dans ce qui n’était pas encore la Californie mais juste un trou à rats… enfin bref vous connaissez la suite, mais il était tout à fait capable de sentir que quelque chose ne tournait pas rond quand un de ces putains de grands théropodes carnivores, genre T-Rex, se pointait dans le secteur avec la ferme intention de se tailler un bon steak dans un gros cul de bronto. À ce moment-là, une petite sonnette se mettait à tinter dans son crâne de piaf, et il s’arrêtait aussitôt de brouter pour jeter un coup d’œil aux alentours, se servant de son cou immense comme d’une perche pour effectuer des moulinets avec sa tête et ratisser la zone à trois cent quatre-vingts degrés. Bon, évidemment, sa vitesse de déplacement ne lui permettait pas de prendre la fuite efficacement, d’autant que le Rex, tout en restant lui-même assez lourdingue, était capable d’accélérations non négligeables à défaut d’être foudroyantes. Pourquoi ? Eh bien mais tout simplement parce que dame Nature, toujours pétrie de bonnes intentions, avait jugé cocasse de le doter, en sa qualité de prédateur ultime, de pattes arrière extrêmement puissantes lui autorisant une certaine approche de la bipédie, en même temps que ses pattes avant, ainsi dégagées des contraintes de la quadrupédie, s’étaient vues recyclées en organes préhensiles armés de griffes tranchantes comme des lames de rasoir. Enfin, comme si ça ne suffisait pas, lorsqu’il s’agissait de s’attaquer à des proies disproportionnées, type bronto, Rex savait faire preuve d’un esprit de corps inhabituel, autrement dit faire appel à quelques uns de ses congénères les plus déterminés pour ne laisser aucune chance à la victime, quitte à devoir ensuite partager le butin à peu près équitablement (chose qui pouvait à l’occasion, ne nous voilons pas la face, occasionner quelques légères prises de bec). Bronto, face à une telle adversité, était condamné d’avance, sauf s’il parvenait, sur un malentendu, à transformer son agresseur en amas de viande sanguinolente en le piétinant sans ménagement. Mais la nature, même si elle le destinait à servir de pâture à ses plus sanguinaires rejetons (d’authentiques machines de guerre dépourvue de la moindre once de compassion), l’avait hypocritement équipé d’un avertisseur interne censé le prévenir du danger et lui offrir une chance infime de sauver sa peau.

Cet avertisseur, c’était l’instinct, et tous les animaux, y compris l’animal humain, en étaient pourvus. Chez nous, je pense qu’il est légèrement dévoyé, comme tout le reste, au point qu’il n’est pas rare que la victime imaginaire se transforme, à titre préventif dirons-nous, en agresseur bien réel. Le revers de la médaille, en quelque sorte.

Toujours est-il que quelques cent cinquante millions d’années plus tard, j’étais là, digne et alcoolisé (sinon alcoolique) représentant d’une espèce certes controversée mais ayant tout de même, au fil des millénaires, accompli quelques prouesses techniques qui laissaient loin derrière la concurrence, réduite à un rôle de figuration sur la scène du grand théâtre de l’évolution. J’étais là, et mon avertisseur interne me hurlait dans le creux de l’oreille que le costume que je croyais tiré à quatre épingle de la situation était en train de se détricoter lentement sous mes yeux incrédules.

Un des principes mêmes de la notion de naissance, voyez-vous, et c’est particulièrement vrai pour l’être humain (d’autres s’en sortent beaucoup mieux à ce niveau-là, même s’ils restent passablement démunis face à l’adversité), c’est de venir au monde dans un état de fragilité et de dépendance absolues. La nature, qui ne laisse au hasard que le strict nécessaire, et encore n’est-on pas sûr qu’il s’agisse réellement de hasard, a mis sur point un système de filiation qui protège le nouveau-né des appétits de sa mère, laquelle pourrait fort bien l’ingurgiter sur le champ (ce qu’elle s’abstient généralement de faire, à quelques rares exceptions près chez les coraux, les coquillages et les rongeurs), comme elle le fait, de nombreux cas en témoignent (mantes, araignées, scorpions, grenouilles, crabes, poissons, insectes), du mâle qui vient d’accomplir périlleuse besogne. En fait, au sein de cette noble maison, tout le monde est plus ou moins suceptible de dévorer tout le monde, qu’il s’agisse de son mari, ses parents ou ses enfants (et, pourquoi pas, se dévorer soi-même dans les cas les plus extrêmes, on parle alors de tendance suicidaires observées, comme par hasard, exclusivement chez l’être humain, en dehors de quelques fourmis et autres pucerons qui explosent spontanément dans le but de protéger la communauté, ce qui n’est pas le cas de l’être humain qui cherche le plus souvent à entraîner le plus de monde possible dans se chute). Certes, il existe des raisons de nature alimentaire ou sanitaire censée légitimer ces pratiques barbares, mais il apparaît clairement que la nature ne connaît aucune limite en matière de reproduction et n’hésite pas à ses livrer aux pires expérimentations (à l’image des «médecins» nazis de Dachau, Auschwitz, Ravensbrück et Buchenwald) pour optimiser son fonctionnement. En l’occurrence, l’expression aristotélicienne «la nature a horreur du vide» se vérifie une fois de plus par son absence totale de compassion envers ses sujets. Un petit animal aussi charmant que le Hamster doré, par exemple, met volontairement bas des portées excessives de rejetons à seule fin qu’une partie d’entre eux serve de casse-croûte à la mère, laquelle peut ainsi se nourrir tranquillement d’une partie de sa descendance tout en allaitant l’autre, ce qui tout de même assez abject quand on y pense (et je pense globalement que la nature est une des entreprises les plus abjectes qui aient jamais été conçues, laquelle, si elle n’est pas la sinistre conséquence de quelque fâcheux concours de circonstances, ne peut être que l’œuvre d’un dément, l’ultime éructation d’un cerveau malade dont on ne peut imaginer un seul instant qu’il appartienne à un quelconque parangon d’amour et de vertu). Comment, devant le spectacle pitoyable de ces mâles minuscules condamnés à s’approcher en catimini de l’objet de leur désir (en espérant qu’il dorme et prenant toutes les précautions pour ne surtout pas le réveiller), faire leur petite affaire en quatrième vitesse et le plus discrètement que possible (pas question de se mettre à hurler des insanités du genre «oh oui c’est bon», «t’aimes ça, salope» ou encore «tu la sens ma grosse queue» pour s’encourager à la manœuvre), avant de se barrer à toutes jambes (pattes) la goutte au gland et le slip sur les talons, comment, disais-je, devant un spectacle d’une telle affliction, ne pas évoquer les techniques de soumission chimique utilisées par les violeurs patentés. Ce n’est que face à une femme inconsciente (voire morte pour les plus méfiants, qui sacrifient du même coup tout espoir de descendance), qu’ils se sentent suffisamment à l’aise pour s’accoupler. Nul doute qu’ils appartiennent à cette catégorie de mâles minuscules (ici au sens psychologique du terme) contraints d’user des plus vils expédients pour arriver à leurs fins. On se doit bien évidemment, sur un plan juridique, de leur faire porter l’entière responsabilité de leurs actes (il s’agit de fixer des limites pour éviter au monde des perspectives post-apocalyptique peu réjouissantes, d’établir des règles pour cartographier les vastes territoires de la perversité humaine), mais il ne trompe personne qu’ils ne sont que des marionnettes dont la nature, confortablement installée au plus profond du génome, tire sournoisement les ficelles.

Naturellement, je vous livre toutes ces pensées après coup, car vous pensez bien que dans le feu de l’action j’avais d’autres chattes à fouetter que me livrer à ce genre de considérations sur la triste réalité de la condition humaine.

En tant qu’organisateur de l’événement, je me devais de prendre de toute urgence la décision qui s’imposait, décision qui, je dois bien le dire, ne s’imposait pas clairement à mon esprit, au point que j’envisageais de plus en plus de rentrer chez moi et déléguer mes pouvoirs à quelqu’un de mieux informé.

«TITUS PAS ALLER, VENIR AVEC MOI !!!!!!!!!»

Ces paroles, répétées à l’envi par la sexuellement très pertinente Atiena, sorte de clocharde céleste surgie des profondeurs de la nuit, érodaient inexorablement le peu de volonté et les maigres ressources intellectuelles qui s’accrochaient encore à mes neurones en perdition.

Pendant ce temps, Sam et Nathan commençaient à s’exciter. Ces deux va-t-en-guerre, pressés d’en découdre, trouvaient que la plaisanterie n’avait que trop duré. J’étais, en mon for intérieur, tenté de leur donner raison, tant la situation était en train de tourner au vaudeville le plus ahurissant. Je sentais également, en ce même for intérieur, que mon collègue et indéfectible ami Titus Beaugendre, ébranlé par l’intervention de cette sorcière aux yeux vert céladon (nom du berger forézien héros de L’Astrée d’Honoré d’Urfé, comte de Châteauneuf, marquis du Valromey et seigneur de Virieu-le-Grand, écrivain à ses heures à qui l’on doit également quelques épîtres morales et paraphrases sur les cantiques de Salomon de moindre intérêt), ne trouvait plus aucun charme à notre compagnie. Greg, quant à lui, n’était pas du genre à foncer tête baissée dans les emmerdements. Il prenait tout son temps pour observer les choses à distance, et ne s’engageait pleinement dans la bataille que lorsqu’il avait acquis la certitude d’obtenir gain de cause. L’épisode Yiorgos Panayiotou (la balle dans le bras, le bijoutier reconnaissant et les vacances à Zanzibar) n’avait fait que le renforcer dans l’idée qu’il valait mieux tourner dix fois sa langue dans sa bouche avant de l’ouvrir pour raconter n’importe quoi. Contrairement à la plupart d’entre nous, il ne parlait que quand il avait quelque chose à dire, c’est à dire pratiquement jamais, ce qui était très reposant pour tout le monde.

Cent cinquante millions d’années plus tard, donc, soit bien après la grande extinction du Crétacé, laquelle, en plus d’avoir exterminé la quasi totalité des dinosaures (il semble que seules quelques espèces de théropodes à plumes aient survécu), avait également causé la perte des ammonites après trois cent cinquante millions d’années de bons et loyaux services au sein des océans (les pauvres, victimes elles aussi des expérimentations stylistiques de dame Nature, sont passées par toutes les configurations possibles au niveau de leur coquille, la spirale classique restant quand même la plus stable et populaire), je me trouvais là, au beau milieu de la rue, en pleine nuit, sans trop savoir sur quel pied danser (je danse très mal, du reste, ce qui fait que je ne danse pas du tout parce que je n’ai pas envie de passer pour un clown).

Depuis un bon quart d’heure maintenant, la Gardienne de la Nuit essayait vainement de tirer Titus hors du van, mais je ne doutais pas un instant qu’elle finirait par arriver à ses fins. C’était écrit dans le Grand Livre Noir de la Nuit éternelle et autres étendues infinies du Cosmos. Titus était mon bras droit, ma couille gauche, mon troisième œil ou tout ce que vous voudrez, de sorte que, rempli d’une détresse poisseuse qui m’engluait jusqu’aux ouïes, je ne me voyais aucunement aller au combat sans son appui technique et logistique. Non seulement je ne me voyais pas y aller sans lui, mais je me voyais de moins en moins y aller tout court, tant la soirée avait pris un tour inhabituel qui ne présageait rien de bon.

Je vous passe les détails, mais Titus, comme je m’y attendais avec son cerveau de limace et sa volonté de poisson rouge, a fini par répondre aux injonctions de la Gardienne de la Nuit.

Le malheureux avait les yeux exorbités et ne parvenait plus à détacher son regard de celui d’Atiena, particulièrement fascinant il faut bien le dire. J’avais moi-même, à titre d’indication, les plus grandes difficultés à détacher mon regard de son anatomie callypige de Vénus hottentote, sans commune mesure toutefois avec le postérieur monumental de Saartje «Swatchie» Baartman, cette Khoïsan exhibée sur toutes les scènes du monde à l’époque des grandes expositions coloniales, assimilée par Cuvier et ses contemporains à un singe humain proche de la monstruosité, un phénomène de foire qui attisait le sentiment de supériorité et la sexualité trouble des visiteurs. Ce brave Cuvier, disciple de la malédiction de Canaan et la Table des peuples (Genèse 9:18-29) mais guidé par un indomptable esprit scientifique et un sens de l’éthique inébranlable, avait même jugé pertinent de conserver dans le formol quelques morceaux choisis de son anatomie, notamment son cerveau et ses organes génitaux. Deux siècles plus tard, au début des années 2000 et sous la pression de Mendela, ces «pièces de musée» et le squelette de Swatchie seront restituées à l’Afrique du Sud dans le cadre de la vaste campagne de repentir colonial orchestrée par l’UNESCO, au grand dam des pilleurs de tombes et autres receleurs étatiques du butin esclavagiste. Depuis, tout un tas de reliques, crânes, momies et organes en tout genre ont repris le chemin de leurs pays d’origine, à commencer par El Negro de Banyoles, ce guerrier du Botswana empaillé en toute discrétion par les frères Verreaux (et véreux) en 1831 et revendu quelques années plus tard à un certain Francesc Darder i Llimona, vétérinaire et médecin catalan grand amateur d’objets scabreux et auteur d’un très recherché Manuel pratique pour l’élevage des oies (à noter qu’il s’intéressait aussi à la reproduction des truites, j’en veux pour preuve sa conférence lors de la fête du poisson de Ripoll restée dans toutes les mémoires), et longtemps exposé au Museu Darder de Banyoles en compagnie d’autres articles de choix tels que fœtus, têtes réduites et peaux humaines.

Dieu sait comment, à moins que ce ne soit le diable en personne, Atiena avait réussi à prendre le contrôle de l’esprit de Titus (ce qui, je vous le concède, était plus ou moins à la portée du premier venu). Certains parasites microscopiques, des vers notamment, qui se développent à l’intérieur d’un hôte spécifique, sont capables de ce genre de prouesse. Ainsi le spinochorde, qui ne peut se reproduire que dans l’eau, prend le contrôle du cerveau de la sauterelle et la force à se noyer le moment venu. De façon tout aussi sournoise, le toxoplasme obéit à un cycle reproductif qui nécessite la mobilisation d’un hôte intermédiaire, tel que l’oiseau ou le rongeur, avant de passer à l’hôte définitif qui est le chat. C’est ainsi, le moment venu, qu’il prend le contrôle du cerveau de son hôte intermédiaire pour le pousser à finir entre les griffes du chat, au lieu de le fuir comme son instinct lui commande de le faire. Des chimpanzés, infectés par ce même cheval de Troie cellulaire, ont vu leur libido perturbée au point d’être attirés sexuellement par les léopards, chose qui réduit évidemment drastiquement leur espérance de vie, le léopard ne mangeant pas de ce pain-là (mais ayant par ailleurs un goût prononcé pour la viande de chimpanzé). On n’a, à ma connaissance, pas encore observé de cas probant de membre d’une quelconque tribu africaine qui serait allé volontairement à la rencontre d’un lion pour lui rouler une pelle. Cela dit, les contaminations humaines ne sont pas rares et peuvent conduire à de sérieuses complications telles que malaises, céphalées, myalgies, désordre mental, hallucinations, convulsions et parfois même coma. Pour la petite histoire, il semblerait également qu’un accroissement substantiel des sécrétions hormonales rendent les sujets infectés nettement plus attirants que les autres, mais les études conduites en ce sens, de nature à renvoyer définitivement les inhibiteurs de la phosphodiestérase de type 5 au rang de remède de bonne femme, en sont encore au stade embryonnaire).

De toute évidence, Atiena appartenait à cette catégorie de parasite capable de manipuler mentalement sa proie.

J’ai pensé un instant à lui tirer une ou deux balles dans la tête, je ne vous le cache pas, et puis je me suis dit que ce serait peut-être un peu disproportionné, un manque évident de savoir-vivre (mais le savoir-vivre des uns passe par le savoir-crever-la-gueule-ouverte des autres, non, vous ne croyez pas ?), et surtout qu’il serait quand même dommage de faire exploser bêtement la cervelle d’une créature en laquelle la nature avait manifestement investi sans compter une bonne partie de ses ressources.

\textsc{Sam} : Il faut faire quelque chose !

\textsc{Moi} : Je suis bien d’accord. Mais quoi ? Titus ?

Pas de réponse.

\textsc{Greg} : Je sais pas vous, mais moi je la sens de moins en moins, cette petite virée.

\textsc{Nathan} : On n’a qu’à le laisser là et continuer sans lui.

\textsc{Moi} : Sans Titus ? Hors de question !

\textsc{Sam} : M’est avis que cette chose le tient en son pouvoir.

\textsc{Atiena} : Moi pas chose. Moi Gardienne de la Nuit.

Je suppose que rares sont celles et ceux parmi vous qui ont entendu parler de ce bon vieux Jim Bowie, un type auquel il valait mieux éviter de chercher des crosses. Après une brillante carrière de trafiquant d’esclaves en Louisiane, il a rencontré Dieu au détour d’un verre de bourbon et décidé de se racheter une conduite en s’engageant dans l’armée pour lutter contre le général Antonio Lopez de Santa Anna y Perez de Lebron, le Napoléon du Nouveau Monde, héros immortel de Zempoala et gouverneur de Veracruz. Courageuse attitude qui lui a valu de trouver la mort pendant le siège de Fort Alamo, en compagnie d’autres valeureux combattants comme Bill Travis et surtout un certain Davy Crockett, ancien trappeur du Tennessee connu pour être un type cool épris de justice et proche des Amérindiens. Mais ce qui a fait la réputation de Jim Bowie, c’est qu’il adorait jouer du couteau. Et pas n’importe lequel. Lui, ce qu’il appréciait particulièrement, c’était le genre de bibelot avec lequel vous pouvez facilement décapiter un sanglier d’une main tout en vous grattant les couilles de l’autre. Il n’est pas, techniquement, l’inventeur du couteau qui porte son nom, le couteau Bowie, mais s’est découvert une telle passion pour lui, un amour à ce point fusionnel qu’il est désormais impossible de les dissocier l’un de l’autre. Jim en avait toujours un sur lui, et s’en servait aussi bien pour couper sa viande, vider un ours que découper ses ennemis en rondelles. C’est d’ailleurs aussi parce qu’il avait l’habitude de se curer les dents avec que l’objet a hérité de son surnom chantant de «cure-dent de l’Arkansas», même s’il fallait un sacré coup de main pour ne pas se couper la langue par la même occasion. Le Bowie est ce qu’on appelle un couteau de survie, la bonne à tout faire des objets tranchants, le compagnon idéal des virées en solitaire dans la jungle ou les montagnes infestées de Peaux-rouges (je parle d’avant les guerres indiennes, le génocide et les réserves, bien sûr, parce que maintenant, à part Donald Trump, je ne pense pas qu’il reste beaucoup de Peaux-rouges réellement dangereux aux USA), le chaînon manquant entre le couteau de poche et le sabre d’abordage. Il suffit qu’il voie sa lame affûtée étinceler sous le chaud soleil des Tropiques pour que le jaguar qui s’apprêtait à vous sauter dessus regagne sa tanière la queue entre les pattes en poussant des petits couinements de chaton effarouché. Si tu aimes tuer des gens et adores par dessus tout les voir se tordre de douleur pendant que tu leur tournes et leur retournes la lame dans le bide, alors c’est l’engin qu’il te faut. Il se glisse dans la main avec facilité, je dirais presque gourmandise, et ne fait aucune difficulté pour traverser le cuir épais d’un tapir ou un caïman. L’essayer c’est l’adopter, et une fois que vous aurez passé votre pouce sur le tranchant de sa lame, taillé votre premier morceau de bois et éviscéré votre premier arapaïma avec lui, après avoir pris un bain dans les eaux troubles de l’Amazone et échappé de justesse aux dents acérées du poisson vampire et du piranha à ventre rouge, alors vous ne pourrez plus jamais vous en passer. Vivre sans lui n’aura plus de sens, et toutes vos facultés intellectuelles seront mobilisées par un seul et unique objectif : vous en servir sans cesse pour couper, trancher, élaguer, tailler, sectionner, tronçonner, débiter, amputer, ouvrir, inciser, mutiler tout ce qui vous passe à portée de main.

C’est exactement ce qui arrivé à Bobby Beausoleil, en ce dimanche 27 juillet 1969, quand il a étripé Gary Hinman à Old Topanga Canyon, dans le comté de Los Angeles, avec son bowie mexicain. L’endroit, à l’époque, grouillait de chevelus défoncés à la mescaline et au LSD. Les produits naturels, issus de pratiques ancestrales basées sur une connaissance millénaire de l’environnement, étaient privilégiés. Les hippies n’avaient pas envie de bosser. Ils se foutaient du capitalisme qui avilissait l’homme et le rendait hermétique à sa propre existence. Ils étaient jeunes, beaux (pas tous, mais on s’en foutait parce que la beauté n’était pas un critère de sélection, même les moches avaient le droit de vivre), ils avaient envie de se balader à poil, sans entrave, parce que toutes ces conneries de fringues c’était bon pour les bourgeois qui passaient leur vie à grelotter de trouille et se planquer derrière des écrans de fumée. Les bourgeois corrompus vivaient dans le mensonge et n’hésitaient pas à offrir leurs propres enfants en sacrifice à Mammon. Les hippies avaient envie de profiter de la vie, cramer leur jeunesse par les deux bouts, mouiller, bander sans arrêt, éjaculer à tout-va, danser sous la lune, s’accoupler et faire leurs besoins sans honte ni retenue devant les autres. Ils n’avaient pas envie de vieillir pour ressembler à ces vieux cons pétris de certitudes et rongés par l’amertume qui se claquemuraient derrière leurs portes fermées à double tour et dormaient avec un flingue sous l’oreiller. Ces vieux singes vérolés avaient tout donné au Fric et le Fric ne leur avait rendu que des miettes de bonheur tiédasse noyées au sein d’un océan de servitude et de désillusions. Les hippies voulaient vivre aux quatre vents, sans barrières ni limites, revendiquaient le libre usufruit d’une terre qui appartenait à tout le monde mais n’était la propriété de personne, et surtout pas d’une poignée d’autocrates névrosés qui s’arrogeaient le droit de vie et de mort sur des populations entières. Les gens n’avaient pas de comptes à rendre, de papiers d’identité à présenter, de garanties à fournir pour être acceptés à la table commune. Ils n’avaient pas besoin de s’excuser d’être en vie, prêter allégeance à un quelconque pouvoir, être validés par un comité de censeurs grisonnants au cœur sec.

Bobby Beausoleil était un de ces hippies défoncés à la mescaline qui se baladaient à moitié à poil, une guitare en bandoulière, dans les montagnes de Santa Monica. Sans domicile fixe, il avait trouvé refuge chez Gary Hinman, un prof de musique à la cool qui aimait dépanner les petits jeunes dans le besoin. Bobby était du genre beau gosse pas farouche, on pouvait lui mettre la main au paquet sans qu’il fonce porter plainte au commissariat du coin, qui l’aurait de toute façon envoyé bouler comme une merde. Le soleil tapait fort, ce jour-là, sur sa petite tête de piaf, et son Bowie chicano commençait à se sentir à l’étroit dans son étui en cuir de buffle. Quelque temps plus tôt, alors qu’il se baladait pieds nus dans les montagnes de Santa Monica (ce qui n’est pas très malin parce que les crotales ne sont pas rares dans le secteur), sa guitare en bandoulière et son petit paquet de couilles moulé à la louche dans son microshort en lycra vert fluo, Bobby s’était retrouvé du côté de Santa Susana Pass Road, au-dessus de Chatsworth, dans ce qu’on a coutume d’appeler la Vallée de San Fernando depuis que les Espagnols l’ont piquée aux indiens. C’est là que se trouvait le Spahn Movie Ranch, un site de tournage désaffecté squatté par un certain Charles Manson et sa petite famille, des filles surtout, dont la plupart se prostituaient pour faire tourner la boutique, quand elles n’étaient pas occupées à voler ou faire les poubelles de supermarchés. Les bourgeois étaient tous des porcs qui puaient la pisse et ne pensaient qu’à se taper des petits culs roses et frais. Ils profitaient de la misère humaine pour assouvir leurs plus bas instincts, avec la bénédiction des autorités qui leur mangeaient dans la main. Mais un jour ou l’autre, peut-être plus proche qu’il y paraissait, le moment serait venu de régler la note, et celle-ci risquait d’être un peu plus salée que ce qu’ils avaient imaginé. Comme Bobby, Charlie jouait de la gratte et poussait la chansonnette pour dénoncer les dérives de la société et mettre en garde l’humanité - et notamment ces enfoirés de rednecks consanguins qui tiraient sur tout ce qui bouge - sur les conséquences de ses actes. Un jour ça allait péter, de la viande froide allait être expédiée aux quatre coins de l’univers, les flammes allaient ravager le monde, les oiseaux prendre feu dans le ciel, et seuls une poignée de disciples triés sur le volet survivraient pour reconstruire un monde nouveau, meilleur et plus juste. Charlie avait un regard intense, avec des yeux comme des braises qui vous foutaient la cervelle en ébullition. Quand il parlait, les gens l’écoutaient religieusement et se sentaient aussitôt transportés vers des horizons insoupçonnés. Il a proposé à Bobby de venir s’installer chez lui, Bobby a accepté. Bobby «belle gueule» Beausoleil s’est tout de suite très bien entendu avec tout le monde, les filles en particulier, Mary, Linda, Gipsy, Brenda, Susan, Squeaky (ancienne du Girls Athletic Club de Santa Monica et chasse gardée de Charlie, c’est elle qui avait le privilège de bichonner le vieux George Spahn, le propriétaire du ranch, alors âgé de quatre-vints piges, pratiquement aveugle et physiquement en état de décomposition avancée, mais toujours prêt à accorder des réductions de crédit quand on lui tripotait la nouille, autant dire que vivre entouré de ces petites chattes étaient pour lui une véritable bénédiction), Leslie, Snake, Sandy, Sherry, Pat et les autres. C’est qu’il y en avait du monde, là-dedans, et c’était que des gens jeunes et beaux animés des meilleurs intentions, une vraie petite famille baignant dans l’amour et la compassion comme on aimerait en voir plus souvent. De l’avis général, Bobby n’était pas seulement beau gosse, c’était aussi un super musicien, un authentique artiste, et tous les producteurs auraient dû se bousculer au portillon pour lui faire signer des contrats mirobolants. Même chose pour Charlie, bien sûr, dont les textes étaient d’une profondeur inégalée, disant tout le mal-être d’une jeunesse en perdition dans un monde gangrené par l’injustice, la bêtise et la cupidité. Et justement, en parlant de fric, il se trouve que Gary Hinman, le prof de musique à la cool qui dealait de la drogue chez qui Bobby avait séjourné un certain temps, avait semble-t-il un joli bas de laine planqué sous son matelas. En plus, Bobby n’était pas content parce que Gary l’avait arnaqué en lui vendant ce qui était soi-disant de la mescaline de première bourre et n’était en fait que de la strychnine bidouillée, le genre de cocktail qui pouvait vous exploser à la gueule à tout moment. Il l’a dit à Charlie, et Charlie, qui détestait se faire arnaquer, et ce d’autant plus qu’il avait l’impression de se faire arnaquer depuis son plus jeune âge (fils d’un père qu’il n’a jamais connu et d’une mère alcoolique et prostituée qui passe la moitié de son temps en taule, il est élevé par un oncle et une tante qui le tabassent et abusent de lui sexuellement, alors que d’autres sont choyés par des parents aimants, pètent dans la soie et bouffent avec des couverts en argent), est parti en sucette sur les chapeaux de roues. Bobby prêchait l’amour et la non-violence, comme tout bon hippie qui se respecte. Il voulait juste qu’on le laisse faire de la musique, vivre d’amour et d’eau fraîche, de chasse et de cueillette, comme ses ancêtres avant lui, et leurs ancêtres avant eux aussi loin qu’on puisse remonter dans les méandres de l’histoire et la préhistoire. Une vie simple, basique, axée principalement sur les ressources naturelles et l’exercice de ses droits les plus élémentaires dans le strict respect de ceux d’autrui. Charlie aussi, mais il prêchait surtout (à son corps défendant, car il n’était au fond de lui-même qu’amour et bonté, tendresse filiale détournée du droit chemin par des années de mauvais traitements) la violence et la haine comme des maux nécessaires pour se frayer un chemin jusqu’à Zion, la terre promise, pendant lumineux de la sombre Babylone. Ils sont allés voir Gary, qui jouait de la flûte dans son salon en fumant un joint (pas en même temps), et Charlie, après avoir menacé de lui enfoncer sa putain de flûte dans le fion jusqu’à la luette, lui a flanqué un coup de sabre de samouraï qui lui a arraché la moitié d’une oreille. Gary s’est mis à couiner comme un goret, disant qu’il n’était pour rien dans cette histoire de came, et Charlie lui a dit qu’il ferait bien de les rembourser vite fait s’il ne tenait pas à se faire hacher menu. Il avait appris de source sûre qu’il planquait un petit héritage d’au moins vingt mille dollars chez lui, et Charlie exigeait le pactole en échange du préjudice subi, tant sur le plan moral que physique, cette came frelatée ayant très bien pu les envoyer ad patres, chose qui ne s’était heureusement pas produite mais laissait quand même des traces dans le corps et l’esprit de celles et ceux qui étaient passés aussi près de l’annihilation pure et simple de leur identité, la négation de leur être. Bobby, qui n’était pas le mauvais bougre, a suggéré à Charlie de rentrer au ranch pour essayer de se calmer un peu, la colère n’étant pas bonne conseillère, tandis que lui-même, Susan (alias Sadie Mae Glutz, ancienne danseuse nue à San Francisco, future meurtrière de Sharon Tate enceinte de huit mois et demi) et Mary (Brunner, alias Mother Mary, la taulière, la première victime de Charlie, son âme damnée, la pierre fondatrice de son édifice criminel qui n’aura de cesse, par la suite, de l’aider à agrandir sa «famille» et pousser des jeunes femmes dans son lit, l’équivalent, si vous voulez, d’une Monique Olivier pour un Michel Fourniret), resteraient avec Gary pour discuter tranquillement le coup et tenter de trouver une issue à cette affaire qui soit honorable et satisfaisante pour tout un chacun. En définitive, la chose ne serait pas si grave si Bobby n’avait revendu une partie de la dope frelatée de Gary aux Straight Satans, un gang de bikers déjantés avec lesquels il venait mieux entretenir des relations de bon voisinage. Maintenant ils en avaient après lui et il allait passer un sale quart d’heure s’il ne les remboursait pas illico de l’argent indûment investi. Les Straight Satans avaient le crime dans le sang et ils en avaient dessoudés pour moins que ça. En plus ils travaillaient salement, à l’ancienne, et les souffrances endurées étaient proportionnelles à l’embonpoint de la créance.

Finalement, la discussion a tourné au vinaigre. Pendant trois jours, Gary a été torturé par la joyeuse bande pour savoir où il cachait son fric. Mais, d’après ses dires, il était fauché comme les blés. Gary pissait le sang et voulait voir un médecin. Quand il a vu que Gary ne craquerait pas, Bobby a changé son fusil d’épaule. Il avait remarqué les deux bagnoles garées dans la cour, et il a exigé que Gary les lui donne en échange de sa dette. Quand l’affaire a été réglée, sachant pertinemment que Gary irait le balancer à la première occasion, Bobby l’a planté deux fois dans le thorax avec son Bowie mexicain. Après quoi la petite bande l’a regardé agoniser en sirotant des bières avant d’écrire des cochoncetés sur les murs avec son sang, du genre MORT AUX PORCS, et de laisser des indices foireux pour faire croire que les Black Panthers étaient dans le coup.

Bon, eh bien quelques années plus tard, on retrouve le même genre de coupe-chou entre les mains de John Rambo (interprété par Stallone après avoir été refusé par Dustin Hoffman, Al Pacino, Robert de Niro, Steve McQueen et Clint Eastwood, entre autres), ancien béret vert et héros de la guerre du Vietnam qui éprouve quelques difficultés à se réinsérer dans la vie civile. Non de son fait, car il est plein de bonne volonté, mais à cause de son regard de tueur et son allure générale qui ne plaident pas en sa faveur. Il suffit qu’il débarque dans un trou perdu au milieu des montagnes pour qu’un gros con de shérif redneck lui tombe dessus et le prie de quitter les lieux sans demander son reste. Rambo estime qu’on n’a pas à le traiter de cette façon, surtout après qu’il ait risqué sa vie dans les rizières pour sauver le pays des griffes du Vietcong et porter au plus haut les valeurs du capitalisme et de l’impérialisme américain. Rambo a vu ses compagnons d’armes mourir les uns après les autres, dans des conditions atroces, rongés par la vermine et la pourriture dans la chaleur étouffante de la forêt primaire, et s’il s’en est sorti, c’est grâce à son courage, bien sûr, mais aussi aux techniques de survie et de combat que lui a enseignées le colonel Trautman, son mentor à Fort Bragg, en Caroline du Nord (ainsi nommé en l’honneur du général de division Braxton Bragg, brute esclavagiste et commandant de l’armée du Tennessee, défaite par Ulysses Grant lors de la troisième bataille de Chattanooga). Bref, Rambo fait de la résistance, le shérif le boucle pour vagabondage et détention d’une arme de catégorie D de grande taille, et les flics beaufs et racistes du coin redoublent d’efforts pour lui faire subir les pires humiliations. Sauf qu’ils ne savent pas qui est Rambo, et surtout ils ne savent pas ce qui se passe quand Rambo pète un plomb. Parce que Rambo, c’est pas qu’il est complètement taré ou quoi, mais quand même, après ce qu’il a enduré au Vietnam, les horreurs qu’il a vues (ses potes les tripes à l’air, la langue arrachée, le visage en bouillie, les vers qui grouillent dans les plaies, les araignées et les serpents qui tombent des arbres, la puanteur et la crasse, les Viets qui s’amusent à vous couper les couilles, vous arracher les ongles et vous crever les yeux, les corps déchiquetés par les mines), disons que tout ne tourne pas toujours très rond dans sa petite tête et qu’il vaut mieux éviter de le pousser dans ses derniers retranchements. Sinon, l’animal traqué se transforme en prédateur impitoyable, prêt à tout pour sauver sa peau, le chasseur devient la proie et ses chances de survie en milieu hostile sont pratiquement inexistantes. C’est ce que le shérif Will Teasle (excellent Brian Dennehy) et sa bande de ploucs en uniforme ne vont pas tarder à apprendre à leurs dépens. Ils vont se lancer à la poursuite de Rambo dans la forêt et en prendre plein la gueule pour pas un rond. Le sergent Arthur Galt, à bord d’un hélico qui survole la montagne, entreprend de dézinguer Rambo accroché à la paroi à coups de carabine, ignorant les ordres de Teasle qui exige qu’on le prenne vivant. Il le rate à plusieurs reprises, mais Rambo finit par dégringoler dans le ravin et s’ouvre le bras en tentant de se raccrocher à un sapin. Déjà qu’il n’était pas très satisfait de la tournure prise par les événements, ce fâcheux contretemps est l’élément déclencheur d’un accès de fureur difficilement contrôlable. N’oublions pas que Rambo est un être frustre (aux antipodes de l’intellectuel de gauche qui écrit des essais, donne des conférences dans les grandes universités et passe à la télé pour débattre sur les grands sujets de société de son époque, sous la houlette de journalistes onctueux confortablement installés dans des fauteuils de velours) dont les nerfs ont été mis à rude épreuve par les années passées sous les drapeaux. Je ne dis pas qu’il est totalement stupide, non, loin de là, mais simplement que ses réactions ne sont pas celles d’un homme cultivé qui pèse longuement le pour et le contre avant de se lancer dans la destruction méthodique de son prochain. Sous le coup d’une émotion qu’il n’arrive pas à contenir, Rambo se saisit d’une grosse pierre et la balance sur l’hélico qui continue à lui tourner autour comme une saleté de moustique géant. La pierre atterrit dans le pare-brise de l’hélico qui fait une embardée. Galt, qui tirait depuis le bord, perd l’équilibre, tombe à son tour dans le ravin et se fracasse mortellement la tronche sur les rochers tranchants comme des éclats de verre. Teasle, du haut de la falaise, assiste impuissant à la mort de son collègue et néanmoins ami, même si Galt était quand même une sacrée tête de nœud (vous savez ce que c’est, les gens ont beau être des abrutis de la pire espèce, il reste toujours une part d’humanité en eux, même factice, qui fait qu’on hésite à les exterminer comme des cafards). Suite à cet incident désastreux, une chasse à l’homme sans merci s’engage dans la montagne. Rambo, habitué à évoluer dans la jungle et tirer parti des éléments, n’éprouve aucune difficulté à se débarrasser un par un des hommes du shérif. Désireux de ne pas envenimer la situation, il ne les tue pas mais se contente de leur infliger quelques blessures et les avertir qu’ils s’exposent au pire s’ils continuent à lui coller au train. Lui, tout ce qu’il veut, c’est qu’on lui foute la paix. Mais les flics ne l’entendent pas de cette oreille, même quand le colonel Trautman débarque pour leur dire que c’est lui, Samuel Trautman, colonel de son état et formateur de bérets verts à Fort Bragg, qui a formé, conçu Rambo, qu’il en a fait une véritable machine de guerre, une arme de destruction massive rompue aux techniques les plus extrêmes en matière de guérilla et de combat rapproché, et qu’ils n’ont aucune chance de s’en sortir en l’affrontant sur son terrain. Le mieux est de faire semblant de lâcher le morceau et lui tomber dessus en douceur quand il se décidera à sortir de son trou. Naturellement, Teasle ne veut rien entendre. Il continue à s’acharner sur Rambo, lequel entre dans une fureur noire et met la ville à feu et à sang. À la fin, Trautman récupère son poulain et ils partent ensemble sous le soleil couchant pour aller effectuer d’autres missions périlleuses au service de la nation. Rambo, avec son regard de chien mouillé, encore traumatisé par les horreurs de la guerre, pleure sur l’épaule paternelle de Trautman comme un gros bébé musclé avec un cerveau de poule, d’une naïveté déconcertante qui émeut même les cœurs les plus secs. Le spectateur médusé comprend alors que Rambo n’est qu’un gosse des rues perdu dans un monde trop grand pour lui, exploité par un gouvernement cruel et sans pitié, et finalement rendu à ses semblables qui le recrachent comme un vieux bout de viande périmée coincé entre deux chicots pourris. Ses seuls amis s’appellent Randall 18, Hoyt Spectra, Martin Cougar, Saco M60, Heckler \& Koch MP5A3, Dragunov SVD-63, Gil Hibben IV, Browning M2 et Winchester modèle 1892, une référence absolue dans le monde merveilleux des armes à feu qui fascine petits et grands depuis toujours (à titre d’exemple, John Wayne, membre de la Marion McDaniel Lodge 56 de Tucson, médaille d’or du Congrès et autre référence absolue de l’american way of life dans tout ce qu’elle a de plus frétillant, l’utilise abondamment et avec une jouissance manifeste dans La Prisonnière du désert, notamment pour tirer Debbie Edwards - Natalie Wood dans un de ses rôles les plus sexy - des griffes de Relampago, le chef des Comanches qui l’a enlevée alors qu’elle n’était encore qu’une enfant et élevée à la sauce indienne).

Si je prends la peine de vous raconter tout ça, c’est parce que c’est précisément une réplique exacte du Randall 18 de Rambo que Sam trimballait dans un étui en cuir de buffle accroché à sa ceinture, réplique qu’il a sortie d’un geste brusque et pointée en direction d’Atiena en prononçant ces paroles dont l’extrême indignité me révulse aujourd’hui encore, bien longtemps après les faits, au plus profond de moi-même : GARDIENNE DE MON CUL, OUI !!!!!!

C’était d’une vulgarité insupportable, laquelle, je dois le dire, m’a profondément choqué sur le coup. Et puis, finalement, après quelques semaines de soins intensifs dans une unité psychiatrique spécialisée dans la remise à niveau des militaires polytraumatisés par les horreurs de la guerre (je parle de gars des forces spéciales ayant écumé les pires zones de combat de la planète et affronté des terroristes prêts à toutes les ignominies pour se débarrasser d’eux), j’ai réussi à sortir du marasme et me réadapter peu ou prou à la vie civile, même s’il m’arrivait encore de sentir monter en moi des pulsions destructrices difficilement contrôlables.

Moi, sur un ton extrêmement réprobatif : Sam, non !

\textsc{Lui} : Quoi, non ?

\textsc{Moi} : Range-ça tout de suite, tu veux !

Lui, les yeux injectés de sang, les traits déformés par la haine au point que je ne suis même pas certain que je l’aurais reconnu si je l’avais croisé par hasard dans la rue : M’en vais lui tailler les oreilles en pointe, moi, à cette salope !

\textsc{Atiena} : Moi pas salope. Toi gros porc.

Greg, s’allumant une Alain Delon (The Taste of France, il s’en faisait régulièrement expédier de Phnom Penh par un membre de sa famille qui tenait un restaurant huppé en plein cœur de celle qu’on appelait autrefois La Perle de l’Asie, chose que je désapprouvais au plus haut point pour au moins deux raisons : d’une part je trouve que la clope c’est de la merde, d’autre part j’ai toujours pensé qu’Alain Delon était un acteur médiocre qui ne devait sa carrière qu’à son physique avantageux) à la flamme d’un briquet jetable d’une marque bien connue que je ne citerai pas (sauf, bien sûr, en échange d’une rétribution conséquente) : Quelque chose me dit qu’on n’est pas sorti de l’auberge.

Sam (qui, maintenant que j’y pense, avait à peu près la même tête de basset que Stallone dans First Blood, cette même tête ayant, avec l’âge et le botox, changé radicalement de physionomie pour ressembler de plus en plus à celle d’un bulldog amateur de cigares) : Vous avez entendu ça, les gars ?

\textsc{Greg} : C’est vrai que t’es un gros porc.

\textsc{Sam} : Moi, je suis un gros porc ?

\textsc{Greg} : Oui. Tu passes ton temps à roter, péter, dire des gros mots, et en plus tu manges comme un cochon.

Sam, se tournant vers moi : Djef, je te rappelle que c’est toi le chef de cette opération.

\textsc{Moi} : Et alors ?

\textsc{Sam} : Alors c’est à toi de remettre de l’ordre dans la discussion.

\textsc{Moi} : Quelle discussion ? Je t’ai demandé de ranger ce couteau, je constate que tu l’as toujours dans la main.

\textsc{Atiena} : Lui gros porc.

\textsc{Moi} : N’en rajoute pas, mon poussin. Je ne suis pas certain de pouvoir le tenir en laisse bien longtemps.

Sam, brandissant son couteau : Je vais la tuer !

\textsc{Atiena} : Moi pas poussin.

\textsc{Nathan} : Non, toi casse-couilles !

\textsc{Elle} : Moi pas casse-couilles. Moi Atiena, Gardienne de la Nuit.

Greg, étonnamment en verve à une heure aussi avancée de la nuit : Et des ennuis.

Vous me connaissez assez pour savoir que je ne suis pas spécialement adepte de tous ces trucs mystiques qu’on nous balance à travers la gueule à longueur de journée. Je sais qu’il arrive parfois qu’une vieille bique à l’article de la mort débarque en fauteuil roulant à Lourdes et en reparte en trottinant sur ses jambes tel un lapereau découvrant la vie avec émerveillement, mais je ne crois pas pour autant au miracle de la grotte enchantée (au propre comme au figuré). Même chose pour les senteurs idylliques qui envahissent les naseaux frémissants du pèlerin tout pantelant de ferveur religieuse, les gouttes de sang qui suintent des statues de plâtre et les apparitions mariales à Medjugorje. Non, la Vierge (dont on attend toujours les résultats de l’examen gynécologique censé confirmer ses dires) a certainement d’autres chattes à fouetter que se fendre d’apparitions quotidiennes aux yeux de trois adolescentes analphabètes (Ivanka, Vicka et Mirjana, pour ne pas les nommer) de Bosnie-Herzégovine (à moins de bosser pour l’office de tourisme du coin), ou se pointer de la même façon à Fatima pour s’entretenir d’eschatologie cosmique avec des bergers prépubères pour qui jouer à touche-pipi constitue l’essentiel des préoccupations existentielles. Je suis prêt à entendre beaucoup de choses, y compris les plus débiles qui soient, mais il ne faut pas non plus me prendre pour un perdreau de l’année. Pour être tout à fait franc, je ne crois pas non plus une seule seconde que Jésus soit sorti de sa tombe par ses propres moyens, la bouche en cœur et frais comme un gardon, après avoir été cloué sur une croix. Je sais qu’il est de bon ton de dire que des tas de choses échappent à notre entendement, et je ne doute pas que ce soit le cas. Pour moi, si miracle il y a, la Nature seule en est responsable, et elle fait tellement chier le monde par ailleurs que c’est bien la moindre des choses qu’elle se rende un peu utile de temps à autre. Si des gens développent des maladies rares et meurent du jour au lendemain, je ne vois pas pourquoi d’autres ne guériraient pas de la même façon. En un mot comme en cent, il faut s’attendre à tout dans l’existence, à commencer par le pire dont on peut être certain qu’il finira tôt ou tard par arriver (la technique principale de survie, ou culture du déni, est de tout mettre en œuvre pour ne surtout jamais y penser). Bien sûr, il y aura toujours des gens pour me dire que la Nature n’est pas venue au monde toute seule, n’a pas surgi de nulle part, sous l’effet de quelque génération spontanée, mais qu’elle est le fruit d’une intelligence supérieure qui, tout en lui laissant l’illusion d’une certaine liberté, continue à contrôler discrètement ses moindres faits et gestes. Tout notre modèle humain, social et sociétal, est construit sur cette idée de manipulation occulte, laquelle, sous éteignoir la plupart du temps, refait parfois surface avec une vigueur accrue, quand elle n’alimente pas les délires paranoïaques et complotistes d’une certaine catégorie de la population.

À la lueur chevrotante des lignes qui précèdent, je gage que vous n’aurez aucun mal à imaginer ma surprise quand j’ai été témoin de la scène suivante : Sam, qui brandissait son couteau sous le nez d’Atiena en exprimant clairement son intention de la désosser comme un vulgaire jambon, s’est soudain immobilisé dans sa gestuelle belliqueuse tel un Troll des montagnes changé en statue de pierre par les premiers rayons du soleil (CF Le Hobbit, de Tolkien, quand Bilbon et les treize Nains de la Compagnie de Thorin sont faits prisonniers par Tom, Bert et Bill, trois Trolls des montagnes bien décidés à les bouffer mais incapables de se mettre d’accord sur la façon de les accommoder, Tom préférant les consommer crus après s’être assis dessus pour en faire de la gelée, Bill au barbecue avec juste un peu de sel et de poivre, et Bert, seul véritable gastronome et cuisinier de la bande, longuement mijotés avec des aromates).

Titus, quant à lui, s’il n’était pas à proprement parler plongé dans un état de minéralisation avancée, n’en subissait pas moins les effets d’une sorte de transe hypnotique interdisant tout contact avec le néocortex qui présidait habituellement à se destinée (à noter que son système limbique présentait lui aussi des signes d’indisponibilité temporaire).

\textsc{Nathan} : Sam ?

\textsc{Sam} : …

\textsc{Nathan} : Tu m’entends, Sam ?

\textsc{Sam} : …

Nathan, hésitant à le toucher comme s’il craignait de le voir tomber en poussière : Sam ?

\textsc{Sam} : …

\textsc{Atiena} : Lui pas entendre.

\textsc{Nathan} : Qu’est-ce que tu lui as fait, sorcière ?

\textsc{Atiena} : Moi pas sorcière. Moi Atiena, gar…..

\textsc{Nathan} : … dienne de la Nuit, oui, je sais. Je te préviens que Sam est ceinture noire cinquième dan d’aïkido. Il a été l’élève de maître Yoshimura Masayoshi, à Kobe, qui était lui aussi capable de paralyser …

\textsc{Atiena} : Adversaire par pensée, oui, moi savoir. Connaître lui.

\textsc{Nathan} : Toi connaître euh… tu connais maître Yoshimura Masayoshi ?

\textsc{Atiena} : Moi voir lui dans télévision.

\textsc{Nathan} : Ah bon.

\textsc{Atiena} : Moi venir très loin, Afrique. Faire partie société secrète capable de paralyser gens par pensée, comme maître Machin Chose.

Nathan, consultant sa Casio G-Shock Master of G Rangeman GW-9500-3ER à boussole numérique et altimètre barométrique : Tiens, ma montre est en panne. C’est bizarre.

\textsc{Greg} : Faut penser à changer les piles.

\textsc{Nathan} : Quelle heure est-il, por favor ?

Greg, consultant sa Baume \& Mercier Classima 10415, un modèle classique et discret en acier inoxydable étanche à cinquante mètres et bénéficiant de deux ans de garantie : Une heure du mat. Pourquoi ?

\textsc{Lui} : J’en ai ma claque. Je crois que je vais rentrer arroser mes plantes et regarder Bagne de femmes de Douglas Sirk.

\textsc{Greg} : T’es pas bien, avec nous ?

\textsc{Nathan} : Si, frangin. C’est juste que si Sam et Titus sont hors jeu, je crois qu’on n’a aucune chance de mener cette mission à bien.

\textsc{Greg} : Aucune, en effet.

\textsc{Atiena} : Titus pas aller.

\textsc{Moi} : Si Titus pas aller, nous pas aller non plus.

\textsc{Atiena} : Titus pas aller.

\textsc{Nathan} : Tu l’as déjà dit, poupée.

\textsc{Atiena} : Moi pas poupée.

\textsc{Greg} : Qu’est-ce qu’on fait ?

\textsc{Moi} : J’avoue que je suis légèrement dépassé par la situation.

\textsc{Nathan} : Elle est chouette, ta montre.

\textsc{Greg} : Merci.

\textsc{Moi} : T’as pas vu la mienne.

\textsc{Nathan} : Non. C’est quoi ?

\textsc{Moi} : Une Rousselot P06 Ultramatic 720, avec bracelet en cuir d’alligator fumé au bois de hêtre.

\textsc{Nathan} : Pas dégueu.

\textsc{Greg} : Juste un léger parfum de scandale.

\textsc{Moi} : C’est de l’alligator d’élevage, je précise.

\textsc{Nathan} : C’est qui, celui-là ?

Il faisait allusion au type qui venait d’apparaître dans la lumière des phares du van.

\textsc{Greg} : Celle-là.

\textsc{Nathan} : Pardon ?

\textsc{Greg} : Celle-là. Elle s’appelle Sally Robinson.

\textsc{Sally Robinson} : Bonjour, je suis Sally Robinson, celle par qui tous les scandales arrivent. Je peux savoir ce qui se passe, ici ?

\textsc{Moi} : Il se passe que rien ne se passe comme il faut.

\textsc{Sally} : Ah bon ? Et c’est qui, cette charmante jeune personne ?

\textsc{Atiena} : Moi Atiena, Gardienne de la Nuit.

\textsc{Sally} : Bonjour, charmante gardienne de la nuit. Dois-je comprendre que vous serez des nôtres, ce soir ?

\textsc{Moi} : Peau de balle ! Je suis désolé de vous décevoir, ma vieille Sally, mais je crois qu’on va être obligé de reporter l’opération aux calendes grecques.

\textsc{Sally} : Vous ne feriez pas ça ?

\textsc{Moi} : Si c’était nécessaire.

\textsc{Greg} : Nous avons eu quelques petits imprévus.

Sally, avançant la main pour toucher les dreads de la Gardienne : Très chouettes, vos dreadlocks, chère amie.

Atiena, avec un geste de recul : Toi pas toucher ! Toi homme ou femme ?

\textsc{Sally} : Les deux à la fois, mon petit chat.

\textsc{Atiena} : Moi pas petit chat.

\textsc{Sally} : Je suis meneuse de revue au Sugar \& Spice. Tu connais le Sugar \& Spice ?

\textsc{Atiena} : Pas connaître.

\textsc{Sally} : Tu devrais faire du music-hall, chérie. Je connais plein de monde dans le milieu. Je peux te présenter, si tu veux.

\textsc{Atiena} : Toi pervers sexuel ?

\textsc{Sally} : Mais non, voyons, pas du tout. C’est juste que tu as un physique à monter sur une scène. Je m’y connais, tu sais, en matière de spectacle vivant.

\textsc{Nathan} : Elle est folle, celle-là !

\textsc{Greg} : Tiago Rodriguez, lâchement assassiné par une bande de nostalgiques du troisième reich, était un excellent ami de Sally. N’est-ce pas, Sally ?

\textsc{Sally} : Alvarez, pas Rodriguez. Oui, c’était un excellent ami à moi, raison pour laquelle je tiens tout particulièrement à venger sa mort. Je me suis acheté une tenue de combat spécialement pour l’occasion.

\textsc{Moi} : Eh bien il faudra la plier et la ranger dans un coin en attendant de la ressortir. Je suis désolé de vous le dire, ma chère Sally, mais Atiena, Gardienne de la Nuit ici présente, affirme que les planètes ne sont pas idéalement alignées pour garantir le plein succès de l’opération.

\textsc{Greg} : Pire : elle affirme que nous entêter dans cette voie serait courir à une mort certaine.

\textsc{Atiena} : Titus pas aller.

\textsc{Greg} : Vous voyez.

Sally, sortant un bout de carton de carton imprimé d’une des très nombreuses poches de son gilet de camouflage : Tenez, ma chère, voici ma carte. N’hésitez pas à m’appeler en cas de besoin.

Atiena, dédaignant le bout de carton : Moi pas besoin.

\textsc{Sally} : On ne sait jamais. J’ai le bras long, vous savez.

\textsc{Atiena} : Pas besoin !

Sally, remballant son bout de carton : Comme vous voudrez. De toute façon, en cas d’absolue nécessité, vous pourrez toujours passer par monsieur (il parlait de Greg). Il a mes coordonnées. Ce n’est pas à proprement parler mon agent, mais il travaille pour moi.

\textsc{Greg} : Oui, d’ailleurs, à ce propos, j’attends toujours le règlement de mes honoraires.

\textsc{Sally} : Nous étions convenu d’une opération de nettoyage ethnique, ce soir.

\textsc{Greg} : Ethnique ?

\textsc{Sally} : Nazique, si vous préférez.

\textsc{Moi} : Chère madame, je suis le commandant Beauvais, de la police judiciaire. Je suis aussi le chef de cette petite assemblée, et si je décide que l’opération est annulée, c’est que j’ai de bonnes raisons de le faire. Je me dois bien évidemment de faire régner l’ordre et mettre les méchants face à leurs responsabilités, mais je dois aussi garantir la sécurité de mes hommes. Vous, par exemple, n’êtes absolument pas une spécialiste de la question. J’ai accepté de vous recevoir parce que Greg, qui est un ami, a lourdement insisté, mais plus je vous regarde et plus j’ai la désagréable sensation d’avoir commis une grave erreur.

\textsc{Sally} : Oui, eh bien dites-vous bien que j’ai exactement la même désagréable sensation. D’abord on me fait poireauter des heures, et ensuite quand j’arrive ici qu’est-ce que je trouve, une équipe totalement désorganisée, en proie au doute, et manifestement sous l’emprise d’une tierce personne dont personne ne m’avait averti de la présence. Il me semble que je paye tout de même assez cher pour qu’on me traite avec un minimum de déférence.

\textsc{Greg} : Justement, puisqu’on en parle, j’attends toujours de palper le solde de mes émoluments.

\textsc{Nathan} : Je peux dire quelque chose ?

\textsc{Moi} : Quoi ?

\textsc{Nathan} : Je sais pas si vous l’avez remarqué, mais la situation est totalement ridicule.

\textsc{Moi} : Et alors ?

\textsc{Nathan} : Alors je propose d’y mettre un terme. Pour commencer, Atiena pourrait rendre sa liberté à Sam.

\textsc{Moi} : Oui, c’est vrai. Atiena ?

\textsc{Atiena} : Quoi ?

Moi (je rappelle que Sam était toujours figé en statue de pierre, le bras levé, son Randall 18 Attack Survival à la main) : Pourriez-vous, s’il vous plaît, rendre sa liberté à Sam ?

\textsc{Atiena} : Lui gros porc.

\textsc{Moi} : Oui, lui gros porc, mais je pense qu’il a compris la leçon, maintenant.

\textsc{Atiena} : Lui menacer moi avec couteau.

\textsc{Moi} : C’est juste, mais je pense qu’il ne le refera plus.

Nathan, essayant sans succès de récupérer le couteau dans la main de Sam : Je n’arrive pas à le retirer.

\textsc{Moi} : Nathan, que vous connaissez bien maintenant, aimerait récupérer le couteau dans la main de Sam. Peut-être que vous pourriez l’aider à le faire.

Atiena, après un temps de réflexion : D’accord.

Aussitôt, les doigts de Sam se sont relâchés et Nathan a pu récupérer le couteau sans effort.

\textsc{Moi} : Très bien. Nathan, range ce couteau, tu veux.

\textsc{Nathan} : Tout de suite, chef.

\textsc{Moi} : Merveilleux. Maintenant que tout danger semble écarté, voulez-vous bien rendre sa liberté à Sam, ô somptueuse et vénérable déesse de la nuit ?

\textsc{Atiena} : Moi pas déesse, moi Gardienne de la Nuit.

\textsc{Moi} : Oui, bien sûr, c’est ce que je voulais dire.

\textsc{Atiena} : D’accord.

Quelques instants plus tard, Sam était de retour parmi nous. Quand je dis de retour, c’est une façon de parler, car j’avais la très nette et subtilement désagréable impression qu’une partie de lui-même était restée en rade pendant l’éclipse qu’il venait de traverser. Son regard, aussi vague qu’un paysage de neige vu à travers une épaisse couche de brume, semblait glisser sur les êtres et les choses comme la rosée du matin sur les plumes d’un canard endormi. Un sourire béat illuminait son visage, à peine reconnaissable tant l’expression de brutalité viscérale qui l’animait ordinairement avait disparu au profit de quelque chose qui s’apparentait étrangement à de la douceur ou de la bienveillance, deux mots qui en temps normal auraient suffi à le plonger dans un état de fureur incontrôlable.

\textsc{Nathan} : Mon dieu, qu’est-ce que vous lui avez fait !

\textsc{Atiena} : Sam revenu.

\textsc{Nathan} : Sam ?

\textsc{Sam} : Oui ?

\textsc{Nathan} : Ça va ?

\textsc{Sam} : Très bien, merci.

\textsc{Nathan} : C’est toi, Sam ?

\textsc{Sam} : Ben oui, qui veux-tu que ce soit.

\textsc{Nathan} : Désolé, mais cette chose n’est pas Sam.

\textsc{Atiena} : Lui Sam.

\textsc{Nathan} : Greg, tu ne vas quand même pas me dire que cette chose est Sam !

\textsc{Greg} : Physiquement parlant, c’est Sam, il n’y a aucun doute là-dessus. Par contre, je te concède qu’il semble avoir changé moralement.

\textsc{Nathan} : Merde !

\textsc{Moi} : Oui, merde.

\textsc{Lui} : Non, merde ! Remballez les couteaux, l’artillerie et tout le toutim, les gars, on a de la visite !

Les phares d’une voiture venaient d’apparaître au coin de la rue.

\textsc{Greg} : On dirait…

\textsc{Nathan} : Oui, c’est une voiture de flics !

La voiture s’est approchée, au ralenti, et arrêtée à quelques centimètres de nous.

Deux flics en uniforme en sont sortis : un petit, râblé, costaud, ramassé sur lui-même comme un pitbull prêt à passer à l’attaque, qui devait avoir dans les vingt-cinq/trente ans, et un autre nettement plus grand, osseux, froid comme un bac à glace, les traits taillés à la hache, d’une cinquantaine d’années.

Tous deux avaient des parfaites têtes d’abrutis comme on les aime, des lunettes de soleil sur le nez alors qu’on était en pleine nuit, la main sur le revolver et un sourire mauvais au coin du bec. C’était le genre sadique, cowboy mâchouilleur de chewin-gum comme on en croise sur les routes désertes du Texas, au volant d’une Dodge Charger gonflée à bloc, et qui n’hésitent pas à vous farcir de pruneaux au moindre pet de travers. Il faut dire qu’ils s’emmerdent tellement que le fait de tomber sur un contrevenant représente pour eux une véritable aubaine. Ils seraient prêts à flinguer un coyote en train de traverser hors des clous.

Le petit trapu a dit : Messieurs-dames bonsoir. Contrôle des papiers, s’il vous plaît.

J’ai dit : Je suis de la maison, les gars.

Le grand noueux a dit : Dans ce cas, vous n’êtes pas sans savoir que les rassemblements sur la voie publique de nature à troubler l’ordre public sont interdits.

\textsc{Moi} : On ne trouble rien du tout. Et oui, merci, je suis au courant, article 431 du code de procédure pénale.

J’ai fait le geste de glisser ma main dans ma poche pour sortir ma plaque, ce qui a eu pour effet de faire apparaître comme par magie un flingue dans la main des deux flics qui nous faisaient face.

\textsc{Le grand noueux} : J’éviterais de faire ça, si j’étais vous.

\textsc{Moi} : Commandant Beauvais, de la PJ. Je veux juste vous montrer ma carte.

Le petit trapu a dit, le flingue braqué sur moi : Alors allez-y. Mais en douceur, s’il vous plaît.

\textsc{Le grand noueux} : Lentement, très lentement.

\textsc{Le petit trapu} : J’ai des fourmis dans la gâchette, surtout à une heure aussi avancée de la nuit.

J’ai sorti ma carte.

Le grand noueux a dit, après avoir examiné le document sous toutes les coutures : Toutes mes excuses, mon commandant. Mais avec les trucs bizarres qu’on voit en ce moment, on est obligé de prendre toutes les précautions.

\textsc{Le petit trapu} : Les gens n’ont plus aucun respect pour l’uniforme, vous savez.

Ce que je savais, c’était que ces deux bouffons tombaient comme des poils de cul sur la soupe. Déjà qu’en temps normal personne n’avait envie de voir leurs tronches, en temps pas normal, comme c’était le cas maintenant, leur présence constituait à elle seule une insulte au bon goût passible des pires sanctions administratives et judiciaires. Cela dit, vu qu’il était difficile de les traduire devant la cour martiale pour un motif aussi inconsistant, monstrueux, certes, mais assez peu susceptible d’intéresser les autorités concernées dont on connaît, soit dit en passant, le laxisme systémique, j’ai préféré jouer la carte de l’apaisement.

\textsc{Moi} : Pas de problème, les gars. Ce n’est pas moi qui vous reprocherai de faire votre boulot.

\textsc{Le grand noueux} : Vous comprendrez que quand on voit des gens s’agiter au milieu de la rue, on ne peut pas laisser passer ça.

\textsc{Le petit trapu} : Sinon c’est tout le système qui s’écroule.

\textsc{Le grand noueux} : C’est la porte ouverte à l’anarchie, les hordes sauvages qui déferlent dans les cités et cassent tout sur leur passage.

\textsc{Le petit trapu} : Le chaos s’installe, la guerre civile.

\textsc{Moi} : Vous avez mille fois raison, les gars. On ne serait pas dans cette merde s’il y avait plus de jeunes comme vous. Hélas, la plupart ne pense plus qu’à faire la fête, boire de l’alcool et consommer de la drogue. Les zones de non-droit se multiplient et les honnêtes gens n’osent plus sortir de chez eux de peur de se faire agresser ou prendre une balle perdue.

\textsc{Le grand noueux} : Ouais, c’est la guerre des gangs, comme aux States. On trouve des armes à tous les coins de rues, et les gamins de douze ans n’hésitent plus à vous tirer dessus.

\textsc{Le petit trapu} : Des siècles de civilisation pour en arriver là, avouez que c’est bien triste. Perso, j’ai jamais fumé un joint de ma vie. Mon père était flic, il m’a transmis des valeurs que j’essaie de transmettre à mon tour, à coups de pompe dans le cul si nécessaire. Vous pensez que j’en fais trop ?

\textsc{Moi} : Pas du tout, mon garçon. Je suis content de voir qu’on peut s’appuyer sur des gars comme vous pour faire régner l’ordre dans cette putain de ville. On n’est pas encore à Gotham City, mais on ne va pas tarder à y arriver si personne ne prend les choses en main.

\textsc{Le grand noueux} : Pour sûr. On peut vous demander ce que vous faites là à une heure aussi avancée de la nuit, mon commandant ?

\textsc{Moi} : Il s’agit d’une petite réunion privée avec quelques amis. Il faut bien se détendre un peu de temps en temps.

\textsc{Le petit trapu} : Pour sûr. Surtout qu’on ne fait pas des métiers faciles, hein, mon commandant.

\textsc{Moi} : Et comment ! Des fois je me dis que j’aurais mieux fait d’être vétérinaire, ingénieur ou photographe de mode.

Greg, qui avait toujours un peu de mal à fermer sa gueule : Ou professeur. Même cuisinier ou chauffeur routier. Ou écrivain.

Le grand, soupçonneux en plus d’être noueux : Vous êtes flic aussi ?

\textsc{Greg} : Non, enquêteur privé.

\textsc{Le petit trapu} : Ah ouais ?

\textsc{Le grand noueux} : C’est marrant, ça.

\textsc{Le petit trapu} : Vous devez aussi en voir des vertes et des pas mûres, dans votre profession.

\textsc{Greg} : Le fait est que c’est une plongée vertigineuse dans la noirceur de l’âme humaine.

Le grand, toujours aussi noueux et soupçonneux : Pardon ?

\textsc{Le petit trapu} : Il cause bien, le monsieur.

\textsc{Le grand noueux} : Ce serait ti pas que vous auriez fait des études, par hasard ?

\textsc{Greg} : Pas vraiment. En vérité, je ne connais pas de meilleure école que celle de la vie. Je me suis formé sur le tas, comme on dit.

\textsc{Le petit trapu} : Et ça gagne bien, comme boulot ?

\textsc{Greg} : Il y a des hauts et des bas, comme partout.

\textsc{Le grand noueux} : Peut-être bien que vous avez des clients pleins aux as qui vous demandent d’enquêter sur leur femme ou d’espionner leur voisin ?

\textsc{Greg} : Ça peut arriver, s’ils pensent que leur voisin se tape leur femme. Ou alors s’ils pensent que leur voisin est un type bizarre et qu’ils veulent en savoir un peu plus à son sujet. On est là pour rendre service avant tout.

\textsc{Le petit trapu} : N’empêche qu’il faut quand même avoir les moyens pour se payer les services d’un privé. C’est comme qui dirait pas à la portée de toutes les bourses.

\textsc{Le grand noueux} : Ils vont où, les deux, là ?

Il parlait de Titus et Atiena qui étaient en train de se faire la malle discrètement, bras dessus bras dessous, manifestement assez peu désireux de participer à une discussion dont je me serais volontiers dispensé moi aussi, je dois bien le reconnaître, même si j’ai toujours plaisir à échanger avec mes concitoyens, aussi stupides, bornés et atteints de crétinerie aiguë qu’ils soient, ne serait-ce que parce que la présence de l’être humain, y compris la mienne, reste pour moi une source d’interrogation constante (et de consternation par la même occasion, moins en ce qui me concerne, mais quand même).

\textsc{Moi} : Ils rentrent chez eux, je pense.

\textsc{Le grand noueux} : Ils habitent ensemble ?

\textsc{Moi} : Non, ils rentrent chez eux chacun de leur côté.

\textsc{Lui} : Oui, mais ils sont ensemble, là.

\textsc{Moi} : Ils marchent côte à côte, ça ne veut pas dire qu’ils sont ensemble.

\textsc{Lui} : Je n’entends rien, mon commandant. C’est juste que j’ai l’habitude de me poser des questions quand j’assiste à des trucs bizarres.

\textsc{Moi} : Je ne vois pas ce qu’il y a de bizarre là-dedans.

\textsc{Le petit trapu} : Vous les connaissez bien ?

\textsc{Moi} : Très bien.

\textsc{Lui} : C’est qui, le grand black ?

\textsc{Moi} : Titus Beaugendre, un collègue de la PJ.

\textsc{Lui} : Et la fille ?

\textsc{Moi} : Une amie à lui.

\textsc{Le grand noueux} : Elle est pas un peu bizarre, la fille ?

\textsc{Moi} : Comment ça, bizarre ?

\textsc{Lui} : Le genre qui prendrait un peu de drogue, voyez, des trucs de ce genre.

\textsc{Moi} : Donc, si je comprends bien, vous insinuez que moi, Djeferson Beauvais de la PJ, je fréquente des drogués ?

\textsc{Lui} : J’ai pas dit ça.

\textsc{Moi} : Mais vous l’avez pensé très fort.

\textsc{Lui} : Vous savez ce que c’est, mon commandant. Avec tout le respect que je vous dois, on croit connaître les gens et il arrive parfois qu’on découvre des choses pas très jolies à leur sujet. Je dis pas que c’est systématique, je dis que ça arrive parfois. Et en ce qui concerne votre amie, disons que j’ai trouvé qu’elle avait une façon assez bizarre de nous regarder.

Moi, l’air détaché du type qui connaît la musique et n’apprécie pas plus que ça qu’on vienne le bassiner avec des histoires futiles de gens qui auraient soi-disant une façon assez bizarre de regarder les autres : Vous faites erreur sur toute la ligne, agent… ?

\textsc{Lui} : Brigadier-chef Darian Lisnic, mon commandant. À vos ordres !

\textsc{Moi} : C’est pas français, ça.

\textsc{Lui} : Moldave, mon commandant.

\textsc{Moi} : Ah ! la Moldavie ! ses paysages vallonnés, ses forêts luxuriantes, ses ruisseaux poissonneux, ses vignobles ancestraux, ses tumulus de l’âge du bronze, un bien beau pays s’il en est ! Eh bien sachez qu’elle non plus, cette admirable créature dont vous avez l’outrecuidance de prétendre qu’elle regarde les gens de façon assez bizarre, n’est pas française, comme vous l’aurez sans doute remarqué à son allure générale, la finesse de ses traits et la couleur de son épiderme. Sachez, mon cher Darian… vous permettez que je vous appelle Darian ?

\textsc{Lui} : À vos ordres, chef !

\textsc{Moi} : C’est pas un ordre, c’est une question.

\textsc{Lui} : Vous faites comme vous voulez, c’est vous le chef.

\textsc{Moi} : Mais ça ne te dérange pas ?

\textsc{Lui} : Quoi, que vous soyez le chef ?

\textsc{Moi} : Non, que je t’appelle par ton prénom.

\textsc{Lui} : Vous faites comme vous voulez, chef. Si vous avez envie de m’appeler par mon prénom, vous m’appelez par mon prénom. Je n’y vois aucun inconvénient. D’ailleurs, même si j’en voyais un, ça ne servirait à rien que je le dise, puisque vous êtes le chef et que vous pouvez m’appeler comme vous voulez.

\textsc{Moi} : Mais vous n’en voyez pas.

\textsc{Lui} : D’inconvénient ?

\textsc{Moi} : Oui.

\textsc{Lui} : Aucun, mon commandant.

\textsc{Moi} : Dans ce cas, mon cher Darian, sachez que cette jeune personne appartient à un peuple très ancien, les Khoïsan, qui chassent l’antilope et font griller des sauterelles sur des barbecue de fortune depuis des temps immémoriaux pour nourrir leur très nombreuse famille, absence de contraception oblige. Depuis le Paléolithique supérieur, pour être exact. Ça vous dit quelque chose, le Paléolithique supérieur ?

Il a entrouvert la bouche dans l’intention manifeste de s’exprimer sur le sujet, mais je ne lui ai pas laissé le temps de le faire, certain que la réponse, constituée principalement de balbutiements inintelligibles, ne présenterait pas grand intérêt : Je suppose que non, à en juger par le regard bovin et la lippe tombante que vous affichez en ce moment-même.

Il s’est rétracté dans le fond de sa coquille comme un escargot ébouillanté.

J’ai enchaîné, sans me soucier des bâillements et autres signes d’ennui dispensés par mon auditoire : Quoi qu’il en soit, mon cher Darian, Darwin lui-même, dans sa Filiation de l’homme et la sélection liée au sexe, en parle de façon très élogieuse. Les San, noble peuple s’il en est, descendent d’une population fantôme vieille de plusieurs centaines de milliers d’années. Leur ADN contient des haplogroupes mitochondriaux appartenant à la Mère de toutes les mères, celle dont le ventre primordial a enfanté l’espèce humaine au sens noble du terme. Le généticien japonais Kudo Naonori, professeur à l’université de Sendai, membre honoraire de la Royal Society de Londres et spécialiste de l’évolution des espèces, auteur d’ouvrages de référence sur la théorie de la coalescence et la dérive génétique, a dressé un arbre généalogique complet de tous les locus polymorphes connus. Ce dernier indique on ne peut plus clairement que les origines de l’Homme moderne ne remontent pas à quelques deux cent mille ans, comme on le pensait depuis la découverte des squelettes de la basse vallée de l’Omo, mais à plus de trois cent mille. Quelques rares ossements, dont un crâne quasiment complet, ont été retrouvés dans une mine de barytine de la province de Youssoufia, au Maroc, en compagnie de nombreux outils en pierre taillée de facture très avancée. Si vous aviez traversé autant d’épreuves au cours de votre existence, peut-être que vous aussi auriez une façon un peu «bizarre» de considérer vos semblables.

Lisnic, à la fois ému par le fait que pour une des rares fois de sa vie quelqu’un s’adressait à lui comme à autre chose que l’abruti de première qu’il n’avait jamais cessé d’être, allant même jusqu’à pousser le vice d’en concevoir une certaine fierté de classe, et agacé par celui d’être une fois de plus confronté à l’immensité océanique du désastre culturel dont il était l’un des plus fidèles représentants : Z’êtes un vrai puits de science, mon commandant.

\textsc{Moi} : Tu l’as dit, bouffi.

Lui, tête baissée tel un collégien qui vient de se faire méchamment remonter les bretelles par son paternel après avoir récolté un zéro en maths assorti de commentaires peu élogieux sur sa conduite au sein de l’établissement : Désolé, je voulais pas vous froisser. C’est juste que quand je les ai vu se barrer en loucedé, elle et son compagnon, je me suis dit qu’ils avaient peut-être des choses à cacher.

\textsc{Moi} : Comme je vous l’ai déjà dit, c’est juste un ami, pas son compagnon.

\textsc{Le petit trapu} : N’empêche que j’ai trouvé que lui aussi avait l’air un peu bizarre, si je puis me permettre.

Greg, qui rongeait son frein depuis un moment et n’attendait qu’une occasion de monter au créneau : Dites-moi, mon petit vieux, tout le monde a l’air bizarre, avec vous.

L’autre l’a regardé de travers, comme s’il s’agissait d’un chien qui venait de pisser sur ses bas de pantalon, et je me suis dit que la conversation, qui se déroulait jusqu’ici sur un ton relativement badin, pouvait dégénérer rapidement si je n’intervenais pas pour calmer le jeu.

\textsc{Moi} : Je peux connaître vos nom, adresse et matricule, soldat ?

\textsc{Lui} : Vous voulez aussi mon numéro de sécurité sociale ?

\textsc{Moi} : Votre nom suffira.

\textsc{Lui} : Barkad Achaari, chef.

\textsc{Moi} : Pas très français non plus, tout ça.

\textsc{Lui} : Marocain, chef. Tout à l’heure, vous parliez d’un crâne retrouvé dans une mine de barytine de la province de Youssoufia. On trouve toutes sortes de minerais, dans la région, argent, cuivre, cobalt et manganèse, mais elle est surtout connue pour ses gisements de phosphates, qui sont parmi les plus riches du monde. Il y a aussi, à Oued Zem et Khourigba, des gisements de fossiles de reptiles marins du Mésostoïque qui attirent les amateurs du monde entier.

\textsc{Moi} : Du Mésostoïque ?

\textsc{Achaari} : Oui, chef, du Mésostoïque.

\textsc{Moi} : Tu veux dire du Mésozoïque.

\textsc{Lui} : Oui, peut-être. Toujours est-il que mon père est originaire du coin. Après avoir été longtemps chercheur d’or à Tichla, dans le Sahara occidental, et vendu des dents des fossiles aux touristes pour arrondir ses fins de mois, il a fini agent de sécurité à l’OCP, l’Office Chérifien des Phosphates de Safi, plus gros producteur de phosphates du monde.

Moi, de la docte voix de l’enseignant qui s’adresse à un amphithéâtre plein à craquer d’étudiants avides de boire son jus vocal jusqu’à la dernière gorgée syllabique : Ton histoire ému aux larmes, mon petit Barkad, et je crois que je ne suis pas le seul. Maintenant, si tu veux mon sentiment sur le sujet qui nous occupe, à savoir est-ce que l’homme de loi jouit d’une hypersensibilité particulière pour tout ce qui est bizarre, insolite, inhabituel, eh bien sache que ma réponse est oui, sans aucun doute. C’est dans sa nature. Dès qu’un truc sort de l’ordinaire, il le renifle à des lieues à la ronde, comme un requin qui a flairé l’odeur du sang.

\textsc{Achaari} : Ouais, c’est comme un sixième sens.

\textsc{Lisnic} : On anticipe les situations délicates, toujours prêt à réagir au quart de tour.

\textsc{Achaari} : On sent tout de suite quand quelque chose cloche, même le plus infime détail. C’est un truc de dingue !

\textsc{Lisnic} : Ouais, et on fait en sorte que le citoyen lambda continue à vivre sa petite vie tranquille sans jamais se douter qu’il risque sa peau à tous les coins de rues. C’est ça, notre job, et c’est pour ça qu’on passe notre temps à arpenter les rues de la cité au lieu de s’occuper de notre petite famille. Pas vrai, mon commandant ?

\textsc{Moi} : Pour sûr.

\textsc{Lisnic} : Et lui, c’est qui ? Un ami à vous, aussi ?

Il parlait de Sam, toujours en train de bâiller aux corneilles avec un air bienheureux d’idiot du village pour qui chaque brin de muguet, oisillon tombé du nid ou hérisson écrasé au bord de la route représente une source d’émerveillement digne des Mille et une nuits, Vingt mille lieues sous ta mère et tous les autres trucs avec mille dans le titre.

\textsc{Moi} : Affirmatif. C’est le capitaine Samuel Girard, un ancien des Forces Spéciales.

\textsc{Achaari} : Il n’a pas l’air au mieux de sa forme, lui non plus.

Sam, d’une voix monocorde de robot humanoïde : Je suis le capitaine Samuel Girard, ancien des Forces Spéciales, et je vais très bien, merci.

\textsc{Lisnic} : Dites, il aurait pas un peu sauté sur une mine ou un truc dans le genre, le capitaine. Il a l’air un peu… comment dire… diminué intellectuellement.

\textsc{Moi} : Il a eu un petit accident du travail.

\textsc{Sam} : Accident du travail.

Achaari, s’adressant cette fois à Sally Robinson : Et vous, vous êtes qui ?

\textsc{Sally Robinson} : Sally Robinson, meneuse de revue au Sugar \& Spice, à quelques rues d’ici.

\textsc{Achaari} : Je connais, mon cousin y a travaillé comme serveur. Sally : Et il s’appelle comment, votre cousin ?

\textsc{Achaari} : Moustapha Nedali. C’est le fils de ma tante Rachida.

Sally, une lueur égrillarde dans le fond de l’œil : Oui, Moustapha, je me rappelle très bien. Beau comme un dieu, le gamin. Mousse, comme on l’appelait, le petit Moumousse. Mousse la peau douce comme de la mousse, qui sentait bon le houmous. Qu’est-ce qu’il devient, le petit chéri ?

\textsc{Achaari} : Aux dernières nouvelles, il est coach sportif au Marouazi Palace de Casablanca, dans le Triangle d’or. Il s’occupe des riches clientes qui ont besoin de se remettre en forme entre deux séances de shopping.

\textsc{Sally} : Je vois. Il faut dire qu’il sait se servir de ses mains, le petit coquinou !

\textsc{Achaari} : Je vous en prie.

\textsc{Sally} : Et pas que de ses mains, vous pouvez me croire !

Achaari, affichant les signes d’une certaine nervosité : Attention, vous parlez de mon cousin, là !

\textsc{Sally} : Oui, votre cousin, le petit Moumousse. On était toutes amoureuses de lui.

\textsc{Achaari} : Quoi ???!!!! Vous n’êtes tout de même en train de me dire que…

\textsc{Sally} : Que quoi ?

\textsc{Achaari} : Que Moustapha, le fils de ma tante Rachida, est…

Sally, le sourire d’une oreille à l’autre : Gay ? Non, je n’ai pas dit ça. Plutôt à voile et à vapeur, si vous voyez ce que je veux dire. Il faut dire qu’un corps pareil, des yeux de braise et des tablettes de chocolat de ce calibre, ce serait dommage de ne pas en faire profiter le plus grand monde. C’est une attitude responsable qui force l’admiration. Non, vous ne croyez pas ?

\textsc{Achaari} : Vous n’avez pas à parler de mon cousin comme ça.

\textsc{Sally} : Et je ne parle du reste.

\textsc{Achaari} : Quoi ? Quel reste ?

Sally, se pourléchant ostensiblement les babines tartinées de gloss repulpant à l’acide hyaluronique : Eh bien, son petit équipement personnel. Son petit héritage familial, quoi.

Achaari, grattant du pied tel un taureau prêt à charger : Vous vous foutez de moi ?

\textsc{Sally} : Pas du tout, mon mignon. Je dis juste que votre cousin trimballe dans son cabas un service trois pièces de star du X ! Je ne vois pas ce qu’il y a de mal à ça.

Achaari, le souffle coupé par l’indignation : Non mais… vous l’entendez, mon commandant ?

Moi, sentant venir le drame : Oui, je me paluche pas mal mais je ne suis pas encore sourd. Le problème, voyez-vous, c’est qu’on est dans un pays libre. Chacun à le droit de s’exprimer comme il l’entend, avec les mots qui sont les siens.

\textsc{Achaari} : Désolé, mais je ne peux pas le laisser dire ça.

Sally, remontée à bloc : LA laisser. Je suis une femme, au cas où vous ne l’auriez pas remarqué.

\textsc{Achaari} : Non, effectivement, je n’avais pas remarqué. Vous, une femme ? C’est une blague !

\textsc{Sally} : Pas du tout. Ça ne se voit peut-être pas extérieurement, autant qu’il faudrait en tout cas, mais intérieurement je suis une femme de toute la force de mon âme.

\textsc{Achaari} : Homme, femme, chèvre ou table basse, je me fiche de savoir ce que vous êtes ! Tout ce que je sais, c’est que ça ne vous donne pas le droit de manquer de respect à ma famille !

\textsc{Sally} : Je ne manque de respect à personne. Au contraire, je dis que votre cousin Moustapha est un des plus beaux gosses qu’il m’ait été donné de sucer… croiser, pardon…

Achaari, au bord de l’éruption : Quoi, qu’est-ce que vous venez de dire ?

\textsc{Moi} : Calmez-vous, mon vieux. Greg, s’il te plaît, tu ne veux pas dire à ta cliente de fermer un peu sa grande gueule ?

\textsc{Greg} : Oui, en effet, je pense que certaines limites ne sont pas loin d’avoir été franchies. Essayons de rester digne, je vous en prie.

\textsc{Nathan} : Quelle soirée de merde !

\textsc{Moi} : Je crois qu’on a tous besoin de se détendre un peu.

\textsc{Sally} : Je suis parfaitement détendue. Ou plutôt je l’étais, avant que tout le monde décide de me casser les couilles !

\textsc{Greg} : C’est quoi, votre problème ? Vous en voulez à la terre entière, c’est ça ?

\textsc{Sally} : Pas le moins du monde. Mais comme l’a très bien dit votre ami, on est dans un pays libre et j’ai le droit de m’exprimer comme n’importe qui d’autre. Et, n’en déplaise à monsieur, je ne pense manquer de respect à sa famille en disant que Moustapha est un des plus beaux garçons qu’il m’ait été donné de rencontrer.

\textsc{Achaari} : Vous n’avez pas dit ça.

\textsc{Sally} : Si, je l’ai dit.

Lisnic, écartant son subordonné d’un revers de la main afin de prendre les rênes (et pas les rennes, comme j’ai pu le voir écrit ici et là, on n’est pas dans le Grand Nord) de l’interrogatoire : Non. Vous avez prétendu vous êtes livré à des actes de nature sexuelle avec le cousin de mon collègue. De tels propos, adressés à un agent dans l’exercice de ses fonctions, constituent un outrage caractérisé.

\textsc{Achaari} : C’est de la diffamation pure et simple !

\textsc{Sally} : Ma langue a fourché, voilà tout.

\textsc{Lisnic} : Vous reconnaissez donc n’avoir entretenu aucune relation de nature sexuelle, orale ou autre, avec monsieur Moustapha Nedali, le cousin de monsieur Barkad Achaari, mon collègue ici présent ?

\textsc{Sally} : Oui, si vous voulez.

\textsc{Lisnic} : Ce n’est pas si je veux, c’est oui ou c’est non.

\textsc{Sally} : Oui, je le reconnais, et croyez bien que je le déplore.

\textsc{Lisnic} : Vous avez également affirmé, en parlant de ce que vous avez qualifié assez légèrement de «petit héritage familial» de monsieur Moustapha Nedali, qu’il, je cite «trimballe dans son cabas un service trois pièces de star du X». Vous comprenez bien qu’une telle affirmation, hors cadre, constitue une grave atteinte à la dignité de l’intéressé.

Sally, tirant nerveusement sur une Vogue Superslims Bleue qu’on voyait trembloter entre ses doigts aux ongles interminables recouverts d’une épaisse couche de vernis rose fuchsia : Je ne vois pas en quoi.

\textsc{Lisnic} : Donc vous maintenez vos propos ?

\textsc{Sally} : Et comment, que je les maintiens !

\textsc{Greg} : Vous feriez mieux de lâcher l’affaire.

Nathan, entre ses dents : Commence à me courir sur le haricot, la grosse.

Lisnic, une vague lueur de plaisir sadique déformant ses lèvres minces et cruelles de jeune loup aux dents longues : Dans ce cas, Mme Robinson, vous me permettrez de vous poser la question suivante : Qu’est-ce que vous permet d’affirmer que, je cite, «monsieur Moustapha Nedali trimballe dans son cabas un service trois pièces de star du X» ? Vous l’avez déjà vu ?

\textsc{Sally} : Mousse se baladait souvent en slip dans les loges.

\textsc{Lisnic} : En slip ?

\textsc{Sally} : En caleçon, si vous préférez.

\textsc{Lisnic} : Quel genre de caleçon ?

\textsc{Sally} : Je ne vais pas citer de marque, mais c’est le genre boxer qui moule bien le paquet. Poutre apparente, comme on dit.

\textsc{Lisnic} : Je vois. D’où ma question suivante : Pourquoi monsieur Moustapha Nedali évoluait-il dans cette tenue dans les loges ?

\textsc{Sally} : Pour une raison très simple, monsieur l’agent…

\textsc{Lisnic} : Brigadier-chef, s’il vous plaît.

\textsc{Sally} : … monsieur le brigadier-chef, pardon : parce qui’il avait besoin d’aller dans les loges pour se changer, comme tout le monde.

\textsc{Lisnic} : Se changer ?

\textsc{Sally} : Oui. Je vous rappelle que le Sugar \& Spice est ce qu’on appelle un cabaret, autrement dit un établissement dans lequel se produisent des artistes de music-hall qui sont amenés à changer régulièrement de tenue.

\textsc{Lisnic} : J’avais cru comprendre que monsieur Nedali travaillait comme serveur dans cet établissement ?

Sally, d’une voix exprimant clairement son exaspération : Oui, mais dans notre établissement même les serveurs ont droit à une tenue particulière. C’est comme dans les clubs Playboy : les Bunnies portent un costume de lapin sexy avec corset, nœud pap, oreilles et queue de lapin. Eh bien chez nous c’est pareil : les serveurs portent une tenue de service.

\textsc{Lisnic} : Quel genre de tenue ?

\textsc{Sally} : Légère.

\textsc{Lisnic} : Vous voulez dire que Mr Nedali travaillait en slip ?

\textsc{Sally} : Les employés portent une tenue de service adaptée à leurs besoins, spécialement conçue pour ne pas entraver leur liberté de mouvement tout en étant agréable à regarder.

\textsc{Lisnic} : C’est à dire ?

\textsc{Sally} : Un boxer en latex avec étui pénien.

\textsc{Lisnic} : C’est une plaisanterie ?

\textsc{Sally} : Pas du tout, non. Tous nos serveurs portent un boxer en latex avec étui pénien, très apprécié de la clientèle. Pour répondre à votre question, encore que rien ne m’oblige à le faire, c’est grâce à ça, et aussi au fait que j’ai à de très nombreuses reprises eu le privilège d’entrevoir Moumousse sous la douche, que je suis en mesure d’affirmer qu’il était plutôt bien équipé de ce côté-là.

\textsc{Lisnic} : Donc, vous reconnaissez à demi-mot qu’il vous arrive de mater les gens sous la douche.

\textsc{Sally} : Ecoutez, monsieur l’agent brigadier-machin-chose, les mœurs sont très libres au Sugar \& Spice, et il n’est pas rare que nous prenions nos douches en commun. Les sportifs font ça régulièrement, on n’en fait pas tout un plat.

\textsc{Lisnic} : Mais les sous-entendus vont bon train.

\textsc{Sally} : Il y a longtemps que je ne me soucie plus de ce que racontent ou pensent les gens. Quand on a, comme c’est mon cas, une personnalité que je qualifierai de clivante, on apprend à passer outre ce genre de considérations. Si on n’en est pas capable, mieux vaut rester terré chez soi.

\textsc{Moi} : Je ne voudrais surtout pas me mêler de ce qui ne me regarde pas, chacun, je le répète, étant libre de se balader dans la tenue de son choix et se tripoter sous la douche avec qui bon lui semble, mais il commence à se faire tard.

\textsc{Sally} : Je ne vois effectivement pas où est le problème entre personnes majeures et consentantes.

\textsc{Achaari} : C’est pas la question !

\textsc{Sally} : Si. Et j’ajoute que rien ne m’oblige à répondre à vos questions débiles concernant ma vie privée. Maintenant, si vous voulez me menotter et me placer en garde à vue parce que votre cousin se balade à moitié à poil sur son lieu de travail, je vous en prie, ne vous gênez surtout pas !

\textsc{Greg} : Si vous pouviez éviter d’en rajouter.

\textsc{Nathan} : Je crois en effet qu’on a déjà perdu assez de temps comme ça.

\textsc{Sally} : Désolé, mais je n’apprécie pas tellement d’être stigmatisé en raison de mes orientations sexuelles.

\textsc{Achaari} : Il ne s’agit pas de ça !

\textsc{Sally} : Bien sûr que si. Vous vous croyez autorisé à me traiter comme une moins que rien parce que je suis différente des autres. Si quelqu’un peut se sentir blessé ici, c’est moi et pas vous. J’ai des témoins, je pourrais fort bien porter plainte.

\textsc{Lisnic} : Je ne suis certain que votre stratégie de défense soit la bonne.

\textsc{Sally} : Quelle stratégie de défense ? Je ne vois pas pourquoi je devrais me défendre de quoi que ce soit ! C’est plutôt vous qui aurez à répondre des accusations indignes que vous portez contre moi.

\textsc{Greg} : Vous ne pouvez pas la boucler un peu ?

\textsc{Sally} : Oui, bien sûr, on m’insulte ouvertement, tout juste si on ne me traite pas de pervers, de délinquant sexuel, et je devrais me taire, tendre gentiment la joue gauche ! Eh bien non, monsieur le détective, personne ne fait taire Sally Robinson !

\textsc{Nathan} : Quelle soirée de merde !

\textsc{Moi} : Messieurs, je vous en prie !

\textsc{Nathan} : Quoi, c’est vrai, on n’a rien fait de mal ! Pourquoi est-ce que ces soi-disant gardiens de la paix nous traitent comme de vulgaires malfrats ? C’est de l’abus de pouvoir, ni plus ni moins ! Vous n’avez pas mieux à foutre que d’emmerder les honnêtes gens ?

\textsc{Lisnic} : On ne fait que notre travail, monsieur. La loi nous oblige à intervenir quand on repère un attroupement au milieu de la chaussée. C’est ce que nous avons fait, et tout se passait pour mieux avant que cet individu ne se mette à proférer des insanités au sujet du cousin de mon collègue ici présent.

\textsc{Sally} : Je n’ai fait que rapporter l’exacte vérité des faits.

\textsc{Lisnic} : Reconnaissez que vous vous êtes livré à des actes de provocation gratuite.

\textsc{Achaari} : Dont je ne serais pas étonné qu’ils soient directement liés aux origines ethniques de mon cousin et moi-même.

\textsc{Sally} : Si je comprends bien, vous me traitez de raciste, maintenant, alors que c’est vous qui avez fait preuve de discrimination à mon encontre.

\textsc{Lisnic} : Pas du tout.

\textsc{Achaari} : J’ai rarement vu une telle mauvaise foi !

\textsc{Moi} : Messieurs, je ne voudrais pas abuser de mes prérogatives, mais en temps que votre supérieur hiérarchique, je pense en effet que la comédie a assez duré. Vous admettrez que vos agissements sont assez peu en rapport avec la procédure habituelle. En clair, si vous continuez à nous casser les couilles au lieu de faire le boulot pour lequel le contribuable vous rémunère, je me verrai dans l’obligation d’en référer à qui de droit. Suis-je clair ?

\textsc{Lisnic} : Très clair, mon commandant.

\textsc{Moi} : J’ai été ravi de faire votre connaissance.

\textsc{Lisnic} : Et réciproquement, mon commandant.

\textsc{Moi} : Hélas, il n’est de bonne compagnie qui ne se quitte.

\textsc{Greg} : Eh oui, toutes les bonnes choses ont une fin.

\textsc{Sally} : Contrairement aux mauvaises qui ne s’arrêtent jamais.

\textsc{Lisnic} : Mes respects, mon commandant. J’espère que vous ne garderez pas un trop mauvais souvenir de notre rencontre.

\textsc{Moi} : Nullement. Je n’irais peut-être pas jusqu’à dire que vous faites honneur à la police française, mais vous êtes sur la bonne voie. Si l’occasion se présente, je rendrai compte à vos supérieurs de l’excellente qualité de votre intervention.

\textsc{Lisnic} : Merci, mon commandant. Et veuillez excuser mon subordonné, l’agent Achaari. Il fait parfois preuve d’excès de zèle, mais dans le fond c’est quelqu’un de tout à fait fiable et généreux.

\textsc{Sally} : Fiable et généreux ?

\textsc{Lisnic} : Fiable et généreux. Il ne laisse jamais un collègue en plan et privilégie le plus souvent la prévention à la répression.

\textsc{Achaari} : Toutes mes excuses, mon commandant, si je vous ai paru un peu excessif. Je reconnais que j’ai tendance à m’emporter dès qu’on touche à la famille. Je suppose que ça ne me regarde pas, mais le capitaine n’a pas l’air bien du tout.

Sam, en effet, était allongé sur le trottoir, les bras en croix.

\textsc{Moi} : Ne vous en faites pas, il fait ça tout le temps. C’est sa façon à lui de se reposer après une dure journée de travail.

\textsc{Nathan} : Ça va, Sam ?

\textsc{Sam} : Ça va, Sam ?

\textsc{Nathan} : Je te demande si ça va, Sam ?

\textsc{Sam} : Je te demande si ça va, Sam ?

\textsc{Nathan} : Je suis embêté, quand même. Je me demande si on ne devrait pas l’emmener faire un petit tour aux urgences.

\textsc{Lisnic} : C’est vrai qu’il n’a pas l’air bien.

\textsc{Achaari} : C’est bizarre cette façon de répéter ce qu’on lui dit.

\textsc{Moi} : Messieurs, votre sollicitude me touche. Mais je connais Sam depuis des années, je sais parfaitement comment m’y prendre avec lui. Il souffre d’une forme assez rare de tétanie passagère. La crise dure quelques minutes, puis il revient à lui comme si de rien n’était.

\textsc{Greg} : Il a longtemps travaillé pour les forces spéciales, en Syrie et en Afghanistan notamment, et a reçu de nombreuses décorations pour services rendus à la Nation. Un jour, près de Kandahar, il participait à une opération de nettoyage d’un trou à rats infesté de terroristes. Il s’est pris une balle dans la tête, et c’est un miracle qu’il soit encore en vie. Wilfrid Chauveau, professeur de neurochirurgie à la Pitié-Salpêtrière, n’en revient toujours pas. Il a fallu toute une série d’interventions extrêmement pointues, dont certaines ont duré plusieurs heures, pour le sortir de là. Seulement, depuis, il est sujet à des accès de mélancolie et des troubles du comportement plus ou moins spectaculaires. De temps en temps, comme c’est le cas ce soir, il s’allonge sur le sol, les bras en croix, la bouche ouverte et les yeux révulsés, et séjourne dans cette position un temps indéterminé.

Un gargouillis, style cuvette de chiottes qui se débouche d’un seul coup, est sorti de la gorge de Sam, après quoi ses yeux ont effectué quelques tours complets dans leurs orbites, il a remué un bras, puis l’autre, et posé sur ce qui l’entourait un regard empreint d’une certaine forme de curiosité brumeuse.

Greg s’est aussitôt porté à son chevet : Sam ? Ça va ?

\textsc{Sam} : Euh… oui, je crois… On est où, là ?

\textsc{Greg} : Rue des Nénuphars.

\textsc{Sam} : Merde, alors ! Qu’est-ce qu’on fout là ?

\textsc{Greg} : Tu ne te rappelles pas ?

\textsc{Sam} : Non.

\textsc{Greg} : Mais lui (il parlait de moi), tu le reconnais, quand même ?

\textsc{Sam} : Ben oui, c’est Djef.

\textsc{Greg} : Et moi, je suis qui ?

\textsc{Sam} : Greg. Et lui c’est Nathan. Mais ça ne dit toujours pas ce qu’on fait là.

\textsc{Moi} : Tu ne te rappelles de rien, donc ?

\textsc{Lui} : De rien du tout. Qu’est-ce que je fais allongé par terre ?

\textsc{Greg} : C’est rien, juste un petit malaise. Suite à ton opération du cerveau.

\textsc{Sam} : J’ai été opéré du cerveau ?

\textsc{Moi} : Oui. Même que tu as bien failli ne jamais récupérer la totalité de tes facultés mentales.

Greg, entre ses dents : Je ne suis pas certain qu’il les ait totalement récupérées.

\textsc{Sam} : Merde, alors !

\textsc{Greg} : Oui, hein. Mais tout ça c’est du passé, rassure-toi. Tout va bien, maintenant.

\textsc{Sam} : Je peux savoir ce que la police fait là ?

\textsc{Greg} : Rien. Elle allait partir, justement. N’est-ce pas, messieurs ?

\textsc{Lisnic} : Oui, tout à fait.

\textsc{Achaari} : On ne faisait que passer.

\textsc{Lisnic} : On a vu de la lumière, on s’est arrêté pour jeter un œil.

\textsc{Achaari} : Réflexe professionnel.

\textsc{Lisnic} : C’est notre devoir de prêter assistance aux gens dans la détresse. Content de voir que tout va bien, mon capitaine.

\textsc{Sam} : Et la grosse, là, c’est qui ?

\textsc{Sally} : Il parle de moi, là ?

\textsc{Moi} : Je crois bien, oui.

\textsc{Greg} : C’est Sally Robinson, une cliente à moi.

Sam, avec une moue de dégoût : Rarement vu un boudin pareil !

\textsc{Sally} : Non mais je vous en prie ! Pour qui il se prend, cet abruti !

\textsc{Moi} : Vous voyez bien qu’il n’est pas dans son état normal.

\textsc{Sally} : Tout de même, ce n’est pas une façon de parler aux gens !

Nathan, tout en aidant Sam à se relever : Ça va, tu tiens debout ?

\textsc{Sam} : Ça va, merci.

\textsc{Nathan} : Viens, on aller s’assoir un peu dans le van.

\textsc{Sam} : Pas envie de m’assoir.

\textsc{Nathan} : Faut te reposer.

\textsc{Sam} : Pourquoi faire ? Je me sens tout à fait bien, maintenant. Une absence passagère, rien de plus.

\textsc{Greg} : Tu ne te rappelles vraiment de rien ?

\textsc{Sam} : De quoi je devrais me rappeler ?

\textsc{Greg} : Tu ne te rappelles pas qu’il y avait d’autres gens avec nous ?

\textsc{Sam} : Des gens ? Non. Quels gens ? Et elle, c’est qui ?

\textsc{Greg} : Je viens de te le dire : c’est Sally Robinson, une cliente à moi.

\textsc{Sam} : Ah bon. Tu pourrais les choisir un peu mieux, tes clientes. J’ai rarement vu une bonne femme aussi moche. On dirait un homme.

\textsc{Sally} : Je vais lui en coller une !

\textsc{Greg} : Ne faites pas attention à lui.

\textsc{Sally} : Je ne suis pas venue ici pour me faire insulter !

\textsc{Sam} : Et qu’est-ce qu’elle fait là, cette mocheté ?

\textsc{Sally} : J’exige que cet imbécile me fasse des excuses. Tout de suite !

\textsc{Greg} : Calmez-vous, je vous prie.

\textsc{Sally} : Je me calmerai si je veux ! Vous me décevez beaucoup, monsieur Lussier. Beaucoup ! Je pensais que vous étiez quelqu’un de sérieux et raisonnable, et je me rends compte que vous n’êtes qu’une ordure comme les autres !

\textsc{Greg} : Excusez-moi, mais je ne vois pas très bien le rapport.

\textsc{Sally} : Tous ces gens que vous présentez comme étant vos amis sont des beaufs racistes et homophobes !

\textsc{Moi} : Pardon ?

\textsc{Nathan} : Qu’est-ce qu’elle dit, cette morue ?

\textsc{Sally} : Là, vous voyez !

\textsc{Greg} : Nathan, s’il te plaît !

\textsc{Nathan} : Désolé, mais je n’ai pas l’intention de me laisser insulter par ce monstre de foire !

\textsc{Sam} : Haha ! Elle fait moins sa maline, la grosse !

\textsc{Sally} : C’est inadmissible, insupportable !

\textsc{Greg} : Ne nous emballons pas.

\textsc{Sally} : Je ne resterai pas une minute de plus avec des gens aussi grossiers ! Je ne vous félicite pas, monsieur Lussier.

\textsc{Greg} : Je suis désolé. La fatigue, le stress, les événements qui s’entrechoquent…

\textsc{Sam} : Dans ton cul, oui.

\textsc{Sally} : Ça suffit, je m’en vais !

\textsc{Nathan} : C’est ça, barre-toi, Elephant Man !

\textsc{Sam} : Retourne dans ta cage, King Kong !

\textsc{Sally} : Tu peux parler, toi, le débile !

\textsc{Sam} : C’est à moi que tu parles, Moby Dick ?

\textsc{Sally} : Demeuré !

\textsc{Sam} : Je vais te défoncer, le trave !

\textsc{Lisnic} : Allons, allons, je vous en prie ! Tout se passait bien jusqu’à présent, ne m’obligez à intervenir.

\textsc{Nathan} : On t’a rien demandé, à toi, le Moldave pourrave !

\textsc{Greg} : Nathan, s’il te plaît.

\textsc{Lisnic} : Je vais faire comme si je n’avais rien entendu.

\textsc{Nathan} : Bien sûr, que t’as rien entendu. De toute façon, t’es sourd comme un pot. Un pot de chambre moldave !

\textsc{Achaari} : Chef !

\textsc{Lisnic} : Quoi ?

\textsc{Achaari} : On nous signale une tentative de viol à quelques rues d’ici. Apparemment la fille rentrait chez elle en maillot de bain, après une soirée bien arrosée, et elle s’est fait agresser par un type déguisé en sanglier.

\textsc{Lisnic} : Comment ça ? Avec les sabots et tout et tout ?

\textsc{Achaari} : Non, juste la tête. Un masque de sanglier, si vous préférez.

Sally, à Sam : HOMOPHOBE !

\textsc{Sam} : TA GUEULE, LE TRAVELO !

\textsc{Sally} : TRISOMIQUE !

\textsc{Greg} : Tout ça va trop loin. Djef, fais quelque chose !

\textsc{Lisnic} : Oui, faites quelque chose, mon commandant. La situation est en train de déraper.

\textsc{Achaari} : On y va, chef ? On a assez perdu de temps avec cette bande de cinglés.

\textsc{Sam} : Dis donc, toi, le bouffeur de couscous, remballe ta merguez et va faire joujou dans la semoule !

Moi, avec la voix de Pierre Thau, inoubliable interprète des plus grands rôles de basse du répertoire (Méphisto, Don Quichiotte, la statue du Commandeur, le Comte des Grieux et Don Pedro, pour ne citer que les plus fameux) : VOUS ALLEZ LES FERMER VOS GRANDES GUEULES, OUI OU MERDE !!!!!!!!!!!!!! NOM DE DIEU DE BORDEL DE MERDE !!!!!!!!!! JE COMMENCE VRAIMENT À EN AVOIR PLEIN LE CUL DE VOS CONNERIES !!!!!!!!!!!!!!

\textsc{Greg} : Il a les raison, les gars, vous faites chier !

Moi, à Sally : Et vous, allez-vous-en, votre présence ne fait qu’envenimer la situation. (Aux flics :) Ça vaut aussi pour vous, bande de nazes, avec tout le respect que je vous dois.

\textsc{Lisnic} : Ne vous en faites pas, mon commandant, on est sur le départ.

\textsc{Achaari} : On a cette affaire de fille en maillot de bain violée par un sanglier rue Timothée Carbonneau qui requiert toute notre attention.

\textsc{Moi} : C’est tout à votre honneur !

\textsc{Lisnic} : Oui. On serait bien resté à bavarder avec vous, mais le devoir nous appelle. Un viol en soi c’est pas si grave, surtout quand les risques de grossesse sont limités par l’hybridation, mais le problème c’est qu’on observe souvent un effet «trainée de poudre» où tout le monde se met à violer tout le monde dans une espèce de frénésie sexuelle incontrôlable. On a vu des ménagères de moins de cinquante ans se faire violer par des chihuahuas, des majorettes par des étalons en rut, des écolières par des éléphants de mer échappés du zoo, heureusement la plupart du temps trop lents pour les poursuivre efficacement, et même des macaques roux abusés sexuellement par des langurs de Hanuman - à Shimla, notamment, où cette vermine pullule -, des racailles de banlieue par des fils de bonnes familles aux yeux injectés de sang et aux bourses anormalement dilatées, j’en passe et des meilleurs, les champs du possible étant quasiment illimités dans ce domaine de compétence. L’amour seul peut nous sauver, j’en sais quelque chose en tant qu’oracolophile distingué.

\textsc{Achaari} : Hamdullah !

\textsc{Lisnic} : J’ajoute que l’animal en question, décrit par les témoins comme une bête de forte corpulence aux réactions imprévisibles, apparemment dotée d’un membre énorme, surdimensionné tant par la longueur que le diamètre, capable d’assommer un bœuf aussi sûrement d’une barre de fer, est en fuite, en train d’errer quelque part dans le quartier en laissant derrière lui une traînée de liquide spermatique, chose qui ne va pas sans poser problème pour la sécurité de nos concitoyens.

\textsc{Moi} : Je dois reconnaître que vous avez idéalement narré la chose, mon ami.

\textsc{Greg} : Quel talent !

\textsc{Sarah} : C’est pas tous les jours qu’on croise un flic moldave qui maîtrise aussi bien la langue de Molière.

\textsc{Lisnic} : Je vous remercie tous du fond du cul.

\textsc{Achaari} : Hamdullah !

\textsc{Lisnic} : Mais le temps presse. J’ai peur, si on tarde à intervenir, qu’une milice populaire armée jusqu’aux dents ne tente de se substituer aux forces de l’ordre.

\textsc{Moi} : Assurément.

\textsc{Lisnic} : Quelle joie de se lever aux aurores pour exercer un aussi beau métier que le nôtre, et se coucher à point d’heure avec la satisfaction du devoir accompli.

\textsc{Moi} : Bien peu de gens, dans cette société qui a perdu ses valeurs et se délite à vue d’œil, peuvent se vanter d’avoir une existence aussi palpitante que celle du fonctionnaire de police.

\textsc{Achaari} : Protéger et servir, telle est notre mission.

\textsc{Greg} : Je reconnais, en tant que privé, qu’il m’arrive parfois d’envier la fonction publique dans ce qu’elle a de plus noble et désintéressé.

\textsc{Moi} : Au revoir, messieurs.

Lisnic, au garde à vous : Force et honneur, mon commandant.

Achaari, dans la même position : Protéger et servir, telle est notre devise.

\textsc{Moi} : Amen.

Après avoir claqué des talons et exécuté un demi-tour quasi parfait, les deux guignols sont remontés dans leur caisse et partis sur les chapeaux de roues vers de nouvelles aventures. Quelle vie palpitante que celle du gardien de la paix, toujours sur la brèche pour secourir la veuve et l’orphelin, l’honnête homme persécuté par une vermine tenace, la joggeuse importunée par des pervers sexuels et autres sociopathes avides de sexe et de pouvoir, les deux étant souvent les pis d’une même mamelle tuméfiée d’affirmation obsessionnelle de sa personnalité inexistante, de revanche sur une vie ingrate qui traite ses rejetons comme de la merde et les expose sans cesse aux pires vicissitudes, poussant le vice jusqu’à leur faire miroiter une récompense bien méritée s’ils souffrent en silence jusqu’à leur dernier souffle.

Sally, remettant ses nichons en place dans son soutien-gorge débordé : Eh bien puisque c’est comme ça, que tout le monde se fiche comme d’une guigne du respect de mes droits les plus élémentaires à l’égalité et la différence, je m’en vais moi aussi. Mais soyez certain, Mr Lussier, que je suis extrêmement déçue par votre comportement et surtout celui de vos amis. Je pensais avoir affaire à des gentlemen, je me suis bien trompée.

\textsc{Greg} : Je vous rappelle, chère madame, que j’attends toujours le règlement du solde de mes honoraires.

\textsc{Sally} : Tu peux t’assoir dessus, mon grand.

\textsc{Greg} : Pardon ?

\textsc{Sally} : C’est comme Tiago Alvarez. Il s’agissait soi-disant de venger sa mort, mais ça aussi tout le monde s’en fout !

Greg, perdant quelque peu le contrôle de ses nerfs d’acier, au point d’employer des mots d’une certaine grossièreté, ordinairement absents de son vocabulaire : Tu vas me payer ce que tu me dois, morue !

Moi, émergeant peu à peu des vapeurs brumeuses de l’alcool : Greg, s’il te plaît. Tu ne vas pas t’y mettre, toi aussi.

Greg, manifestement en proie à cette forte excitation qui le dominait dès lors qu’il s’agissait d’évoquer l’aspect financier de ses activités : Je n’ai pas pour habitude de bosser pour des prunes.

Sally, après un ricanement d’une méchanceté absolue, ce genre de ricanement sardonique d’où toute humanité semble avoir effacée par la main même de Satan : Bosser !!! Laissez-moi rire !

Greg, prêt à fondre sur sa proie telle la harpie féroce sur le paresseux tridactyle assoupi dans la jungle guyanaise (ou - pour reprendre une expression chère à Simone de Bavoir, alors qu’elle n’était pas encore le castor de Jean-Paul Tartre - le jaguar richement orné sur le stylistiquement insignifiant pécari à lèvres blanches, bien trop occupé à retourner bruyamment le terre avec son groin humide pour se rendre compte du danger qui le menace) : Je vais buter cette salope !

Nathan, peinant à cacher sa joie de voir la situation se dégrader encore un peu plus : C’est ce qu’on aurait dû faire dès le début.

Sam, se frottant les mains et se pourléchant les babines, des éclairs de haine joyeuse plein les yeux (j’emploie à dessein le mot «joyeuse» car la perspective de faire subir à son prochain les pires atrocités était synonyme pour lui d’ambiance festive et hautement récréative, tendance qui préexistait chez lui depuis la tendre enfance, s’était très largement accrue au cours des longues années en territoire hostile, et pouvait à présent s’épanouir en toute liberté dans ses activités sécuritaires à caractère privé) : Laissez, je m’en charge !

Votre serviteur, qui, après avoir un temps formé le projet d’éradiquer les Disciples de la Colère (Jégou, Monteil et Desmarais) de la surface de la terre, ne rêvait plus à présent que d’une seule chose : regagner prestement ses pénates, se glisser subrepticement sous la couette - tel un serpent à sonnette ayant passé la journée à ramper sous le soleil brûlant de l’Arizona, serrer son corps recru (adjectif peu usité auquel j’aimerais redonner un semblant de vie, comme ça, même si je m’en fous royalement par ailleurs, des choses autrement plus graves ayant lieu actuellement dans le monde, ce que je veux dire par là c’est qu’on n’en est pas forcément à un adjectif près dans la langue française, riche par nature, peut-être trop, même si, on le sait, des choses en apparence infimes, des détails de l’Histoire, peuvent engendrer de durables turbulences, lesquelles gagnent en intensité au fil du temps, des siècles si nécessaire, voire des millénaires, et finissent par conduire à des catastrophes majeures qui elles-mêmes, par une étrange distorsion de la perception humaine, peuvent apparaître comme des déflagrations épiphénoménologiques sans réelle consistance, des flatulences civilisationnelles) contre celui de sa dulcinée, et se laisser lentement glisser dans la tiédeur réparatrice d’un sommeil bien mérité : Non, Sam, personne ne va se charger de personne ! Pas pour l’instant, en tout cas.

\textsc{Sam} : Faut que je tue quelqu’un, tout de suite !

\textsc{Moi} : Un peu de patience, mon garçon. Chère madame Robinson, je comprends votre déception. Moi-même, si j’étais à votre place, je crois que je serais extrêmement déçue. Enfin, déçu. Tiago Alvarez, votre amant…

\textsc{Sally} : L’amour de ma vie, vous voulez dire !

\textsc{Moi} : L’amour de votre vie, si vous voulez, a été refroidi par une bande de brutes épaisses qui ne méritent aucune clémence.

\textsc{Greg} : Refroidi, c’est une façon de parler.

\textsc{Moi} : Okay. Je retire refroidi, et je remplace par euh… réchauffé…

\textsc{Greg} : Je dirais plutôt carbonisé.

\textsc{Moi} : Réchauffé à mort, si tu préfères. Excuse-moi, mais on ne va pas passer la soirée à jouer sur les mots !

\textsc{Greg} : Assurément. Ne serait-ce que par égard pour Sally qui est toujours sous le coup de la plus vive émotion.

\textsc{Moi} : La plus vive.

\textsc{Sally} : Je vous déteste !

\textsc{Moi} : Désolé, Sally, mais je crois que vous avez compris le fond de ma pensée. Au cas où ce ne serait pas le cas, je vous donne ma parole que Tiago Alvarez sera vengé comme il se doit. À ce propos, j’aimerais, si vous le permettez, faire part à l’assemblée d’une petite idée qui m’est venue pour solutionner le problème.

\textsc{Sam} : Moi aussi !

\textsc{Moi} : Quoi, toi aussi ?

\textsc{Sam} : J’ai une idée pour solutionner le problème.

\textsc{Moi} : Génial. Et quelle est-elle ?

\textsc{Sam} : On bute la grosse et on rentre chez nous !

\textsc{Moi} : Négatif. Ton idée c’est de la merde, et tu commences à me casser sérieusement les burnes ! Je ne te cache pas que je suis assez déçu par ton comportement.

\textsc{Nathan} : Il a été malade quand il était petit.

\textsc{Moi} : Genre les oreillons ?

\textsc{Nathan} : Pire. Son père le battait et sa mère faisait des passes pour arrondir les fins de mois.

\textsc{Sam} : Qu’est-ce que tu racontes ?

\textsc{Nathan} : Quoi, c’est pas vrai ?

Sam, après qu’un vol de corbeaux ait traversé le ciel nuageux de son cerveau : Euh… si, plus ou moins, mais je tiens pas forcément à ce que tout le monde le sache.

\textsc{Greg} : Il faut nommer les choses pour avoir une idée claire de ce qu’elles sont.

\textsc{Sam} : Quoi ?

\textsc{Greg} : Rien, laisse tomber.

\textsc{Nathan} : Sam, je voulais juste dire que t’es un mec pas toujours facile à gérer.

\textsc{Moi} : C’est sans doute doute pour ça qu’il n’a pas fait carrière dans l’armée.

\textsc{Sam} : Quoi ? J’ai pas fait carrière dans l’armée ?

\textsc{Moi} : Pas plus que ça, non.

\textsc{Sam} : Je suis capitaine, je te rappelle.

\textsc{Greg} : Croquemitaine, à la rigueur.

\textsc{Sam} : Très drôle ! Si je suis parti de l’armée, c’est uniquement parce que j’estimais que mes qualités n’étaient pas récompensées à leur juste valeur.

\textsc{Greg} : Ben voyons !

\textsc{Moi} : Non, Sam. Si t’es parti de l’armée, comme tu dis, c’est parce qu’on t’a foutu dehors pour alcoolisme et comportement déplacé envers les jeunes recrues.

\textsc{Nathan} : Et en plus, tu t’amusais à clouer des animaux morts sur des troncs d’arbre !

\textsc{Greg} : Ouais, comme Jeffrey Dahmer !

\textsc{Sam} : Vous insinuez quoi, là ?

\textsc{Greg} : Rien.

Nathan, remonté : T’as jamais tué des chats avec la 22 de ton père quand t’étais petit ?

Sam, gêné : Peut-être une fois ou deux, je me rappelle plus.

\textsc{Nathan} : Après tu les ramenais discrètement dans ta chambre et tu les disséquais.

\textsc{Sam} : Oui, peut-être. C’est loin, tout ça.

\textsc{Nathan} : Avant de les plonger dans l’acide sulfurique pour voir combien de temps il fallait pour les dissoudre complètement.

\textsc{Sam} : Tous les gosses font ça. On leur apprend à disséquer des rats et des grenouilles en cours de sciences naturelles.

\textsc{Greg} : Non, tous les gosses ne butent pas des chats à la 22 pour les dissoudre dans de l’acide.

\textsc{Sam} : Les gosses sont cruels avec les animaux. T’as jamais arraché les ailes des mouches, toi ?

\textsc{Greg} : Arracher les ailes des mouches, passe encore, même s’il faut être franchement débile pour s’amuser à ça. Mais tuer des chats à la 22 pour les disséquer et les dissoudre dans de l’acide, on change clairement de catégorie !

\textsc{Moi} : Le bien et le mal sont des concepts qui n’existent pas dans la nature. La seule chose qui compte, c’est d’assurer la survie et le développement de l’espèce.

\textsc{Greg} : D’accord, mais l’espèce humaine est un cas particulier. Il a fallu créer des garde-fous pour limiter la casse. Et même avec ça, on a du mal à empêcher les gens de s’entretuer.

\textsc{Moi} : L’homme n’est pas prêt pour la liberté. Trop dangereux. Ce serait comme laisser un loup affamé errer au milieu d’un troupeau de moutons bien gras et dodus. Il faut lui mettre une muselière, le trimballer en laisse et passer derrière lui pour ramasser son caca.

Tout en parlant, et m’écoutant parler avec une certaine délectation (la teneur de mes propos me ravissait autant que le son de ma voix, aussi chaude et enveloppante qu’une couche de pâte feuilletée autour d’un rôti de bœuf), j’ai avisé Sally qui tentait de profiter de l’occasion pour se faire discrètement la malle.

J’aurais pu la laisser partir, trop content de m’en débarrasser, mais le sens de l’autre et l’amour du travail bien fait qui m’habitent en permanence me permettent rarement de céder à la facilité.

Je l’ai donc stoppée dans son élan : Vous partez déjà ?

Elle s’est arrêtée, retournée, et a rétorqué d’une voix étouffée par la rage et la déception : Je crois que je n’ai plus rien à faire ici.

J’ai dit : Je comprends que vous soyez dépitée. Mais je vous en prie, revenez ici, j’ai encore des choses à vous dire.

\textsc{Sally} : Je ne vois pas quoi. Mais si vous insistez.

\textsc{Moi} : Non seulement j’insiste, mais je vous donne ma parole que votre ami sera vengé.

Sally, au bord des larmes : C’était bien plus qu’un simple ami.

\textsc{Moi} : Je le sais, et vous présente une fois de plus, en mon nom et ceux de mes camarades ici présents, mes plus sincères condoléances. Mais j’aimerais maintenant, si vous le permettez, revenir à l’idée que je souhaitais soumettre à l’assistance. Comme vous n’êtes pas sans le savoir, les principaux suspects dans cette sombre affaire sont les sieurs Jégou, Monteil et Desmarais, tristes sires s’il en est, qui forment le noyau dur d’un groupuscule d’extrême-droite connu sous le nom de Disciples de la Colère. Je dirais plutôt Disciples de la Connerie. Quand ils ne sont pas en train de se saouler la gueule au Bouclier, un bar facho de la rue de Gobineau, ils se retrouvent rue Jordan Peshkov, chez Jégou qui fait office de cerveau de la bande, ou plutôt chez sa grand-mère puisque c’est là qu’il habite, dans le sous-sol transformé en bunker nazi, pour préparer la troisième guerre mondiale et le retour triomphal du führer qui attendait patiemment son heure dans un caisson de cryogénisation. En attendant ce jour de gloire, ils bricolent des complots racistes, antisémites et homophobes pour déstabiliser la société et saper les fondements de la démocratie. La vieille, sourde comme un pot et toute acquise aux thèses négationnistes de son petit-fils adoré, qu’elle a élevé depuis sa plus tendre enfance, ses parents s’étant barrés en vitesse dès qu’ils ont compris qu’ils avaient engendré un monstre, leur mitonne des petits plats pour qu’ils ne crévent pas de faim pendant les longues soirées d’hiver passées à refaire le monde, ou plutôt imaginer le plus sûr moyen de le réduire en cendres. Pour en revenir à l’affaire qui nous occupe, Cerqueira, qui était présent au moment des faits, affirme avoir tout tenté pour empêcher Desmarais de commettre le pire. Mais autant essayer de retirer un os de la gueule d’un dogue allemand ! Plombier de métier, cette sous-merde est aussi pompier volontaire. Pas pour sauver des gens, je vous rassure tout de suite, mais pour assouvir sa fascination pour le feu. Voir cramer des choses le met en joie, lui déclenche des érections d’anthologie, lui procure des orgasmes apocalyptiques, choses d’autant plus rares, exceptionnelles et profitables qu’il souffre d’impuissance chronique. C’est au contact des flammes qu’il s’épanouit, et sa grande passion consiste à faire semblant d’éteindre des incendies qu’il a lui-même allumés. Si on en croit Cerqueira, et je suis assez enclin à le faire, c’est Desmarais qui a carbonisé Alvarez au lance-flammes. Ils ont tous, de près ou de loin, participé à cette abomination, mais c’est à Desmarais que revient la palme du pire taré de service. Je vais donc m’en occuper personnellement, à mon rythme, et corriger le tir de dame Nature qui s’est une fois de plus fourvoyée sur la longue route semée d’embûches de l’évolution. Je propose que Sam s’occupe de Jégou et Greg et Nathan de Monteil.

\textsc{Nathan} : Je vais me le faire au Cougar MT-6, avec des pointes de flèches explosives.

\textsc{Greg} : Je ramasserai les morceaux.

\textsc{Sam} : Moi je vais me le faire à l’ancienne, à mains nues. Je vais lui déboiter les articulations une par une, et ensuite je lui briserai gentiment la nuque pour mettre un terme à ses souffrances.

\textsc{Moi} : Excellent. Pour ma part, je n’ai pas encore de modus operandi précis en tête, mais je vais essayer de trouver quelque chose de sympa, si possible en rapport avec sa passion pour Hitler, le suprémacisme et la destruction méthodique de son prochain.

\textsc{Greg} : Et Titus ?

\textsc{Moi} : Ça fait déjà trois fois que j’essaie de l’appeler.

\textsc{Greg} : Il fait quoi, à ton avis.

\textsc{Moi} : Il est prisonnier de la Gardienne de la Nuit, une déesse surgie des entrailles de la terre pour lui sauver la vie.

\textsc{Greg} : Tu crois ?

\textsc{Moi} : C’est comme ça que je vois les choses. En attendant, s’il n’est rentré à la maison à la première heure, Bérénice va me tomber dessus et je vais avoir toutes les peines du monde à lui expliquer ce qui s’est réellement passé.

\textsc{Greg} : Tu sais ce qui s’est réellement passé ?

\textsc{Moi} : Pas vraiment, non.

\textsc{Sam} : Si on allait le chercher ?

\textsc{Moi} : Où ça ?

\textsc{Nathan} : La fille a parlé d’un hôtel tout près d’ici.

En tant qu’homme moderne parfaitement connecté au monde réel et toujours au fait des dernières avancées en matière de technologies de pointe, Greg a aussitôt sorti son smartphone, appareil sur lequel il disposait de toutes les applis nécessaires pour quadriller le périmètre au dixième de millimètre près : Attends, je regarde s’il y en a un dans le coin.

Moi, estimant que le moment d’allumer un Gurkha Ghost Spook était amplement arrivé : Alors ?

\textsc{Greg} : Il y en a un à deux pas d’ici, le Caribbean Hôtel, rue des Maléfices.

Moi, savourant avec une délectation indécente l’épaisse fumée de mon cigare : Ça ne me dit rien qui vaille.

\textsc{Sam} : Sans doute un hôtel de passe.

\textsc{Nathan} : On devrait aller y faire un tour.

\textsc{Moi} : Je ne suis pas chaud. Après tout, Titus est majeur et vacciné. Il a parfaitement le droit de faire des folies de son corps si bon lui semble.

\textsc{Greg} : Je te rappelle qu’il a une femme et des enfants.

\textsc{Moi} : Merci, je suis au courant.

\textsc{Greg} : Et qu’il n’avait l’air dans son état normal quand il s’est fait embarquer par cette fille.

\textsc{Nathan} : Très bizarre, cette fille.

\textsc{Sam} : Ça ne coûte rien d’aller y jeter un œil.

\textsc{Moi} : Vite fait, alors.

\textsc{Sally} : Ce sera sans moi. Bonne nuit. Et n’oubliez pas, monsieur Lussier, que nous avons une affaire en cours.

\textsc{Greg} : Vous vouliez savoir qui a tué Alvarez, vous le savez. Le reste ne me concerne en aucune manière.

\textsc{Sally} : Vous n’aurez pas un centime de plus tant que cette ordure sera en vie. Je suis même prête à doubler la mise si vous me donnez satisfaction.

\textsc{Greg} : C’est du chantage pur et simple.

\textsc{Sally} : Appelez ça comme vous voudrez. Bonne nuit, messieurs.

\textsc{Sam} : Bon, on fait quoi ? On va à l’hôtel ou pas ?

Moi, après un temps d’hésitation : Okay, on y va !

