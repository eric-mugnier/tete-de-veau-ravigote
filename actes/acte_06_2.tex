Les phares d’une voiture venaient d’apparaître au coin de la rue.

\textsc{Greg} : On dirait…

\textsc{Nathan} : Oui, c’est une voiture de flics !

La voiture s’est approchée, au ralenti, et arrêtée à quelques centimètres de nous.

Deux flics en uniforme en sont sortis : un petit, râblé, costaud, ramassé sur lui-même comme un pitbull prêt à passer à l’attaque, qui devait avoir dans les vingt-cinq/trente ans, et un autre nettement plus grand, osseux, froid comme un bac à glace, les traits taillés à la hache, d’une cinquantaine d’années.

Tous deux avaient des parfaites têtes d’abrutis comme on les aime, des lunettes de soleil sur le nez alors qu’on était en pleine nuit, la main sur le revolver et un sourire mauvais au coin du bec. C’était le genre sadique, cowboy mâchouilleur de chewin-gum comme on en croise sur les routes désertes du Texas, au volant d’une Dodge Charger gonflée à bloc, et qui n’hésitent pas à vous farcir de pruneaux au moindre pet de travers. Il faut dire qu’ils s’emmerdent tellement que le fait de tomber sur un contrevenant représente pour eux une véritable aubaine. Ils seraient prêts à flinguer un coyote en train de traverser hors des clous.

Le petit trapu a dit : Messieurs-dames bonsoir. Contrôle des papiers, s’il vous plaît.

J’ai dit : Je suis de la maison, les gars.

Le grand noueux a dit : Dans ce cas, vous n’êtes pas sans savoir que les rassemblements sur la voie publique de nature à troubler l’ordre public sont interdits.

\textsc{Moi} : On ne trouble rien du tout. Et oui, merci, je suis au courant, article 431 du code de procédure pénale.

J’ai fait le geste de glisser ma main dans ma poche pour sortir ma plaque, ce qui a eu pour effet de faire apparaître comme par magie un flingue dans la main des deux flics qui nous faisaient face.

\textsc{Le grand noueux} : J’éviterais de faire ça, si j’étais vous.

\textsc{Moi} : Commandant Beauvais, de la PJ. Je veux juste vous montrer ma carte.

Le petit trapu a dit, le flingue braqué sur moi : Alors allez-y. Mais en douceur, s’il vous plaît.

\textsc{Le grand noueux} : Lentement, très lentement.

\textsc{Le petit trapu} : J’ai des fourmis dans la gâchette, surtout à une heure aussi avancée de la nuit.

J’ai sorti ma carte.

Le grand noueux a dit, après avoir examiné le document sous toutes les coutures : Toutes mes excuses, mon commandant. Mais avec les trucs bizarres qu’on voit en ce moment, on est obligé de prendre toutes les précautions.

\textsc{Le petit trapu} : Les gens n’ont plus aucun respect pour l’uniforme, vous savez.

Ce que je savais, c’était que ces deux bouffons tombaient comme des poils de cul sur la soupe. Déjà qu’en temps normal personne n’avait envie de voir leurs tronches, en temps pas normal, comme c’était le cas maintenant, leur présence constituait à elle seule une insulte au bon goût passible des pires sanctions administratives et judiciaires. Cela dit, vu qu’il était difficile de les traduire devant la cour martiale pour un motif aussi inconsistant, monstrueux, certes, mais assez peu susceptible d’intéresser les autorités concernées dont on connaît, soit dit en passant, le laxisme systémique, j’ai préféré jouer la carte de l’apaisement.

\textsc{Moi} : Pas de problème, les gars. Ce n’est pas moi qui vous reprocherai de faire votre boulot.

\textsc{Le grand noueux} : Vous comprendrez que quand on voit des gens s’agiter au milieu de la rue, on ne peut pas laisser passer ça.

\textsc{Le petit trapu} : Sinon c’est tout le système qui s’écroule.

\textsc{Le grand noueux} : C’est la porte ouverte à l’anarchie, les hordes sauvages qui déferlent dans les cités et cassent tout sur leur passage.

\textsc{Le petit trapu} : Le chaos s’installe, la guerre civile.

\textsc{Moi} : Vous avez mille fois raison, les gars. On ne serait pas dans cette merde s’il y avait plus de jeunes comme vous. Hélas, la plupart ne pense plus qu’à faire la fête, boire de l’alcool et consommer de la drogue. Les zones de non-droit se multiplient et les honnêtes gens n’osent plus sortir de chez eux de peur de se faire agresser ou prendre une balle perdue.

\textsc{Le grand noueux} : Ouais, c’est la guerre des gangs, comme aux States. On trouve des armes à tous les coins de rues, et les gamins de douze ans n’hésitent plus à vous tirer dessus.

\textsc{Le petit trapu} : Des siècles de civilisation pour en arriver là, avouez que c’est bien triste. Perso, j’ai jamais fumé un joint de ma vie. Mon père était flic, il m’a transmis des valeurs que j’essaie de transmettre à mon tour, à coups de pompe dans le cul si nécessaire. Vous pensez que j’en fais trop ?

\textsc{Moi} : Pas du tout, mon garçon. Je suis content de voir qu’on peut s’appuyer sur des gars comme vous pour faire régner l’ordre dans cette putain de ville. On n’est pas encore à Gotham City, mais on ne va pas tarder à y arriver si personne ne prend les choses en main.

\textsc{Le grand noueux} : Pour sûr. On peut vous demander ce que vous faites là à une heure aussi avancée de la nuit, mon commandant ?

\textsc{Moi} : Il s’agit d’une petite réunion privée avec quelques amis. Il faut bien se détendre un peu de temps en temps.

\textsc{Le petit trapu} : Pour sûr. Surtout qu’on ne fait pas des métiers faciles, hein, mon commandant.

\textsc{Moi} : Et comment ! Des fois je me dis que j’aurais mieux fait d’être vétérinaire, ingénieur ou photographe de mode.

\textsc{Greg}, qui avait toujours un peu de mal à fermer sa gueule : Ou professeur. Même cuisinier ou chauffeur routier. Ou écrivain.

\textsc{Le grand}, soupçonneux en plus d’être noueux : Vous êtes flic aussi ?

\textsc{Greg} : Non, enquêteur privé.

\textsc{Le petit trapu} : Ah ouais ?

\textsc{Le grand noueux} : C’est marrant, ça.

\textsc{Le petit trapu} : Vous devez aussi en voir des vertes et des pas mûres, dans votre profession.

\textsc{Greg} : Le fait est que c’est une plongée vertigineuse dans la noirceur de l’âme humaine.

\textsc{Le grand}, toujours aussi noueux et soupçonneux : Pardon ?

\textsc{Le petit trapu} : Il cause bien, le monsieur.

\textsc{Le grand noueux} : Ce serait ti pas que vous auriez fait des études, par hasard ?

\textsc{Greg} : Pas vraiment. En vérité, je ne connais pas de meilleure école que celle de la vie. Je me suis formé sur le tas, comme on dit.

\textsc{Le petit trapu} : Et ça gagne bien, comme boulot ?

\textsc{Greg} : Il y a des hauts et des bas, comme partout.

\textsc{Le grand noueux} : Peut-être bien que vous avez des clients pleins aux as qui vous demandent d’enquêter sur leur femme ou d’espionner leur voisin ?

\textsc{Greg} : Ça peut arriver, s’ils pensent que leur voisin se tape leur femme. Ou alors s’ils pensent que leur voisin est un type bizarre et qu’ils veulent en savoir un peu plus à son sujet. On est là pour rendre service avant tout.

\textsc{Le petit trapu} : N’empêche qu’il faut quand même avoir les moyens pour se payer les services d’un privé. C’est comme qui dirait pas à la portée de toutes les bourses.

\textsc{Le grand noueux} : Ils vont où, les deux, là ?

Il parlait de Titus et Atiena qui étaient en train de se faire la malle discrètement, bras dessus bras dessous, manifestement assez peu désireux de participer à une discussion dont je me serais volontiers dispensé moi aussi, je dois bien le reconnaître, même si j’ai toujours plaisir à échanger avec mes concitoyens, aussi stupides, bornés et atteints de crétinerie aiguë qu’ils soient, ne serait-ce que parce que la présence de l’être humain, y compris la mienne, reste pour moi une source d’interrogation constante (et de consternation par la même occasion, moins en ce qui me concerne, mais quand même).

\textsc{Moi} : Ils rentrent chez eux, je pense.

\textsc{Le grand noueux} : Ils habitent ensemble ?

\textsc{Moi} : Non, ils rentrent chez eux chacun de leur côté.

\textsc{Lui} : Oui, mais ils sont ensemble, là.

\textsc{Moi} : Ils marchent côte à côte, ça ne veut pas dire qu’ils sont ensemble.

\textsc{Lui} : Je n’entends rien, mon commandant. C’est juste que j’ai l’habitude de me poser des questions quand j’assiste à des trucs bizarres.

\textsc{Moi} : Je ne vois pas ce qu’il y a de bizarre là-dedans.

\textsc{Le petit trapu} : Vous les connaissez bien ?

\textsc{Moi} : Très bien.

\textsc{Lui} : C’est qui, le grand black ?

\textsc{Moi} : Titus Beaugendre, un collègue de la PJ.

\textsc{Lui} : Et la fille ?

\textsc{Moi} : Une amie à lui.

\textsc{Le grand noueux} : Elle est pas un peu bizarre, la fille ?

\textsc{Moi} : Comment ça, bizarre ?

\textsc{Lui} : Le genre qui prendrait un peu de drogue, voyez, des trucs de ce genre.

\textsc{Moi} : Donc, si je comprends bien, vous insinuez que moi, Djeferson Beauvais de la PJ, je fréquente des drogués ?

\textsc{Lui} : J’ai pas dit ça.

\textsc{Moi} : Mais vous l’avez pensé très fort.

\textsc{Lui} : Vous savez ce que c’est, mon commandant. Avec tout le respect que je vous dois, on croit connaître les gens et il arrive parfois qu’on découvre des choses pas très jolies à leur sujet. Je dis pas que c’est systématique, je dis que ça arrive parfois. Et en ce qui concerne votre amie, disons que j’ai trouvé qu’elle avait une façon assez bizarre de nous regarder.

\textsc{Moi}, l’air détaché du type qui connaît la musique et n’apprécie pas plus que ça qu’on vienne le bassiner avec des histoires futiles de gens qui auraient soi-disant une façon assez bizarre de regarder les autres : Vous faites erreur sur toute la ligne, agent… ?

\textsc{Lui} : Brigadier-chef Darian Lisnic, mon commandant. À vos ordres !

\textsc{Moi} : C’est pas français, ça.

\textsc{Lui} : Moldave, mon commandant.

\textsc{Moi} : Ah ! la Moldavie ! ses paysages vallonnés, ses forêts luxuriantes, ses ruisseaux poissonneux, ses vignobles ancestraux, ses tumulus de l’âge du bronze, un bien beau pays s’il en est ! Eh bien sachez qu’elle non plus, cette admirable créature dont vous avez l’outrecuidance de prétendre qu’elle regarde les gens de façon assez bizarre, n’est pas française, comme vous l’aurez sans doute remarqué à son allure générale, la finesse de ses traits et la couleur de son épiderme. Sachez, mon cher Darian… vous permettez que je vous appelle Darian ?

\textsc{Lui} : À vos ordres, chef !

\textsc{Moi} : C’est pas un ordre, c’est une question.

\textsc{Lui} : Vous faites comme vous voulez, c’est vous le chef.

\textsc{Moi} : Mais ça ne te dérange pas ?

\textsc{Lui} : Quoi, que vous soyez le chef ?

\textsc{Moi} : Non, que je t’appelle par ton prénom.

\textsc{Lui} : Vous faites comme vous voulez, chef. Si vous avez envie de m’appeler par mon prénom, vous m’appelez par mon prénom. Je n’y vois aucun inconvénient. D’ailleurs, même si j’en voyais un, ça ne servirait à rien que je le dise, puisque vous êtes le chef et que vous pouvez m’appeler comme vous voulez.

\textsc{Moi} : Mais vous n’en voyez pas.

\textsc{Lui} : D’inconvénient ?

\textsc{Moi} : Oui.

\textsc{Lui} : Aucun, mon commandant.

\textsc{Moi} : Dans ce cas, mon cher Darian, sachez que cette jeune personne appartient à un peuple très ancien, les Khoïsan, qui chassent l’antilope et font griller des sauterelles sur des barbecue de fortune depuis des temps immémoriaux pour nourrir leur très nombreuse famille, absence de contraception oblige. Depuis le Paléolithique supérieur, pour être exact. Ça vous dit quelque chose, le Paléolithique supérieur ?

Il a entrouvert la bouche dans l’intention manifeste de s’exprimer sur le sujet, mais je ne lui ai pas laissé le temps de le faire, certain que la réponse, constituée principalement de balbutiements inintelligibles, ne présenterait pas grand intérêt : Je suppose que non, à en juger par le regard bovin et la lippe tombante que vous affichez en ce moment-même.

Il s’est rétracté dans le fond de sa coquille comme un escargot ébouillanté.

J’ai enchaîné, sans me soucier des bâillements et autres signes d’ennui dispensés par mon auditoire : Quoi qu’il en soit, mon cher Darian, Darwin lui-même, dans sa Filiation de l’homme et la sélection liée au sexe, en parle de façon très élogieuse. Les San, noble peuple s’il en est, descendent d’une population fantôme vieille de plusieurs centaines de milliers d’années. Leur ADN contient des haplogroupes mitochondriaux appartenant à la Mère de toutes les mères, celle dont le ventre primordial a enfanté l’espèce humaine au sens noble du terme. Le généticien japonais Kudo Naonori, professeur à l’université de Sendai, membre honoraire de la Royal Society de Londres et spécialiste de l’évolution des espèces, auteur d’ouvrages de référence sur la théorie de la coalescence et la dérive génétique, a dressé un arbre généalogique complet de tous les locus polymorphes connus. Ce dernier indique on ne peut plus clairement que les origines de l’Homme moderne ne remontent pas à quelques deux cent mille ans, comme on le pensait depuis la découverte des squelettes de la basse vallée de l’Omo, mais à plus de trois cent mille. Quelques rares ossements, dont un crâne quasiment complet, ont été retrouvés dans une mine de barytine de la province de Youssoufia, au Maroc, en compagnie de nombreux outils en pierre taillée de facture très avancée. Si vous aviez traversé autant d’épreuves au cours de votre existence, peut-être que vous aussi auriez une façon un peu «~bizarre~» de considérer vos semblables.

Lisnic, à la fois ému par le fait que pour une des rares fois de sa vie quelqu’un s’adressait à lui comme à autre chose que l’abruti de première qu’il n’avait jamais cessé d’être, allant même jusqu’à pousser le vice d’en concevoir une certaine fierté de classe, et agacé par celui d’être une fois de plus confronté à l’immensité océanique du désastre culturel dont il était l’un des plus fidèles représentants : Z’êtes un vrai puits de science, mon commandant.

\textsc{Moi} : Tu l’as dit, bouffi.

\textsc{Lui}, tête baissée tel un collégien qui vient de se faire méchamment remonter les bretelles par son paternel après avoir récolté un zéro en maths assorti de commentaires peu élogieux sur sa conduite au sein de l’établissement : Désolé, je voulais pas vous froisser. C’est juste que quand je les ai vu se barrer en loucedé, elle et son compagnon, je me suis dit qu’ils avaient peut-être des choses à cacher.

\textsc{Moi} : Comme je vous l’ai déjà dit, c’est juste un ami, pas son compagnon.

\textsc{Le petit trapu} : N’empêche que j’ai trouvé que lui aussi avait l’air un peu bizarre, si je puis me permettre.

\textsc{Greg}, qui rongeait son frein depuis un moment et n’attendait qu’une occasion de monter au créneau : Dites-moi, mon petit vieux, tout le monde a l’air bizarre, avec vous.

L’autre l’a regardé de travers, comme s’il s’agissait d’un chien qui venait de pisser sur ses bas de pantalon, et je me suis dit que la conversation, qui se déroulait jusqu’ici sur un ton relativement badin, pouvait dégénérer rapidement si je n’intervenais pas pour calmer le jeu.

\textsc{Moi} : Je peux connaître vos nom, adresse et matricule, soldat ?

\textsc{Lui} : Vous voulez aussi mon numéro de sécurité sociale ?

\textsc{Moi} : Votre nom suffira.

\textsc{Lui} : Barkad Achaari, chef.

\textsc{Moi} : Pas très français non plus, tout ça.

\textsc{Lui} : Marocain, chef. Tout à l’heure, vous parliez d’un crâne retrouvé dans une mine de barytine de la province de Youssoufia. On trouve toutes sortes de minerais, dans la région, argent, cuivre, cobalt et manganèse, mais elle est surtout connue pour ses gisements de phosphates, qui sont parmi les plus riches du monde. Il y a aussi, à Oued Zem et Khourigba, des gisements de fossiles de reptiles marins du Mésostoïque qui attirent les amateurs du monde entier.

\textsc{Moi} : Du Mésostoïque ?

\textsc{Achaari} : Oui, chef, du Mésostoïque.

\textsc{Moi} : Tu veux dire du Mésozoïque.

\textsc{Lui} : Oui, peut-être. Toujours est-il que mon père est originaire du coin. Après avoir été longtemps chercheur d’or à Tichla, dans le Sahara occidental, et vendu des dents des fossiles aux touristes pour arrondir ses fins de mois, il a fini agent de sécurité à l’OCP, l’Office Chérifien des Phosphates de Safi, plus gros producteur de phosphates du monde.

\textsc{Moi}, de la docte voix de l’enseignant qui s’adresse à un amphithéâtre plein à craquer d’étudiants avides de boire son jus vocal jusqu’à la dernière gorgée syllabique : Ton histoire ému aux larmes, mon petit Barkad, et je crois que je ne suis pas le seul. Maintenant, si tu veux mon sentiment sur le sujet qui nous occupe, à savoir est-ce que l’homme de loi jouit d’une hypersensibilité particulière pour tout ce qui est bizarre, insolite, inhabituel, eh bien sache que ma réponse est oui, sans aucun doute. C’est dans sa nature. Dès qu’un truc sort de l’ordinaire, il le renifle à des lieues à la ronde, comme un requin qui a flairé l’odeur du sang.

\textsc{Achaari} : Ouais, c’est comme un sixième sens.

\textsc{Lisnic} : On anticipe les situations délicates, toujours prêt à réagir au quart de tour.

\textsc{Achaari} : On sent tout de suite quand quelque chose cloche, même le plus infime détail. C’est un truc de dingue !

\textsc{Lisnic} : Ouais, et on fait en sorte que le citoyen lambda continue à vivre sa petite vie tranquille sans jamais se douter qu’il risque sa peau à tous les coins de rues. C’est ça, notre job, et c’est pour ça qu’on passe notre temps à arpenter les rues de la cité au lieu de s’occuper de notre petite famille. Pas vrai, mon commandant ?

\textsc{Moi} : Pour sûr.

\textsc{Lisnic} : Et lui, c’est qui ? Un ami à vous, aussi ?

Il parlait de Sam, toujours en train de bâiller aux corneilles avec un air bienheureux d’idiot du village pour qui chaque brin de muguet, oisillon tombé du nid ou hérisson écrasé au bord de la route représente une source d’émerveillement digne des Mille et une nuits, Vingt mille lieues sous ta mère et tous les autres trucs avec mille dans le titre.

\textsc{Moi} : Affirmatif. C’est le capitaine Samuel Girard, un ancien des Forces Spéciales.

\textsc{Achaari} : Il n’a pas l’air au mieux de sa forme, lui non plus.

\textsc{Sam}, d’une voix monocorde de robot humanoïde : Je suis le capitaine Samuel Girard, ancien des Forces Spéciales, et je vais très bien, merci.

\textsc{Lisnic} : Dites, il aurait pas un peu sauté sur une mine ou un truc dans le genre, le capitaine. Il a l’air un peu… comment dire… diminué intellectuellement.

\textsc{Moi} : Il a eu un petit accident du travail.

\textsc{Sam} : Accident du travail.

\textsc{Achaari}, s’adressant cette fois à Sally Robinson : Et vous, vous êtes qui ?

\textsc{Sally Robinson} : Sally Robinson, meneuse de revue au Sugar \& Spice, à quelques rues d’ici.

\textsc{Achaari} : Je connais, mon cousin y a travaillé comme serveur. Sally : Et il s’appelle comment, votre cousin ?

\textsc{Achaari} : Moustapha Nedali. C’est le fils de ma tante Rachida.

\textsc{Sally}, une lueur égrillarde dans le fond de l’œil : Oui, Moustapha, je me rappelle très bien. Beau comme un dieu, le gamin. Mousse, comme on l’appelait, le petit Moumousse. Mousse la peau douce comme de la mousse, qui sentait bon le houmous. Qu’est-ce qu’il devient, le petit chéri ?

\textsc{Achaari} : Aux dernières nouvelles, il est coach sportif au Marouazi Palace de Casablanca, dans le Triangle d’or. Il s’occupe des riches clientes qui ont besoin de se remettre en forme entre deux séances de shopping.

\textsc{Sally} : Je vois. Il faut dire qu’il sait se servir de ses mains, le petit coquinou !

\textsc{Achaari} : Je vous en prie.

\textsc{Sally} : Et pas que de ses mains, vous pouvez me croire !

\textsc{Achaari}, affichant les signes d’une certaine nervosité : Attention, vous parlez de mon cousin, là !

\textsc{Sally} : Oui, votre cousin, le petit Moumousse. On était toutes amoureuses de lui.

\textsc{Achaari} : Quoi ???!!!! Vous n’êtes tout de même en train de me dire que…

\textsc{Sally} : Que quoi ?

\textsc{Achaari} : Que Moustapha, le fils de ma tante Rachida, est…

\textsc{Sally}, le sourire d’une oreille à l’autre : Gay ? Non, je n’ai pas dit ça. Plutôt à voile et à vapeur, si vous voyez ce que je veux dire. Il faut dire qu’un corps pareil, des yeux de braise et des tablettes de chocolat de ce calibre, ce serait dommage de ne pas en faire profiter le plus grand monde. C’est une attitude responsable qui force l’admiration. Non, vous ne croyez pas ?

\textsc{Achaari} : Vous n’avez pas à parler de mon cousin comme ça.

\textsc{Sally} : Et je ne parle du reste.

\textsc{Achaari} : Quoi ? Quel reste ?

\textsc{Sally}, se pourléchant ostensiblement les babines tartinées de gloss repulpant à l’acide hyaluronique : Eh bien, son petit équipement personnel. Son petit héritage familial, quoi.

\textsc{Achaari}, grattant du pied tel un taureau prêt à charger : Vous vous foutez de moi ?

\textsc{Sally} : Pas du tout, mon mignon. Je dis juste que votre cousin trimballe dans son cabas un service trois pièces de star du X ! Je ne vois pas ce qu’il y a de mal à ça.

\textsc{Achaari}, le souffle coupé par l’indignation : Non mais… vous l’entendez, mon commandant ?

\textsc{Moi}, sentant venir le drame : Oui, je me paluche pas mal mais je ne suis pas encore sourd. Le problème, voyez-vous, c’est qu’on est dans un pays libre. Chacun à le droit de s’exprimer comme il l’entend, avec les mots qui sont les siens.

\textsc{Achaari} : Désolé, mais je ne peux pas le laisser dire ça.

\textsc{Sally}, remontée à bloc : LA laisser. Je suis une femme, au cas où vous ne l’auriez pas remarqué.

\textsc{Achaari} : Non, effectivement, je n’avais pas remarqué. Vous, une femme ? C’est une blague !

\textsc{Sally} : Pas du tout. Ça ne se voit peut-être pas extérieurement, autant qu’il faudrait en tout cas, mais intérieurement je suis une femme de toute la force de mon âme.

\textsc{Achaari} : Homme, femme, chèvre ou table basse, je me fiche de savoir ce que vous êtes ! Tout ce que je sais, c’est que ça ne vous donne pas le droit de manquer de respect à ma famille !

\textsc{Sally} : Je ne manque de respect à personne. Au contraire, je dis que votre cousin Moustapha est un des plus beaux gosses qu’il m’ait été donné de sucer… croiser, pardon…

\textsc{Achaari}, au bord de l’éruption : Quoi, qu’est-ce que vous venez de dire ?

\textsc{Moi} : Calmez-vous, mon vieux. Greg, s’il te plaît, tu ne veux pas dire à ta cliente de fermer un peu sa grande gueule ?

\textsc{Greg} : Oui, en effet, je pense que certaines limites ne sont pas loin d’avoir été franchies. Essayons de rester digne, je vous en prie.

\textsc{Nathan} : Quelle soirée de merde !

\textsc{Moi} : Je crois qu’on a tous besoin de se détendre un peu.

\textsc{Sally} : Je suis parfaitement détendue. Ou plutôt je l’étais, avant que tout le monde décide de me casser les couilles !

\textsc{Greg} : C’est quoi, votre problème ? Vous en voulez à la terre entière, c’est ça ?

\textsc{Sally} : Pas le moins du monde. Mais comme l’a très bien dit votre ami, on est dans un pays libre et j’ai le droit de m’exprimer comme n’importe qui d’autre. Et, n’en déplaise à monsieur, je ne pense manquer de respect à sa famille en disant que Moustapha est un des plus beaux garçons qu’il m’ait été donné de rencontrer.

\textsc{Achaari} : Vous n’avez pas dit ça.

\textsc{Sally} : Si, je l’ai dit.

\textsc{Lisnic}, écartant son subordonné d’un revers de la main afin de prendre les rênes (et pas les rennes, comme j’ai pu le voir écrit ici et là, on n’est pas dans le Grand Nord) de l’interrogatoire : Non. Vous avez prétendu vous êtes livré à des actes de nature sexuelle avec le cousin de mon collègue. De tels propos, adressés à un agent dans l’exercice de ses fonctions, constituent un outrage caractérisé.

\textsc{Achaari} : C’est de la diffamation pure et simple !

\textsc{Sally} : Ma langue a fourché, voilà tout.

\textsc{Lisnic} : Vous reconnaissez donc n’avoir entretenu aucune relation de nature sexuelle, orale ou autre, avec monsieur Moustapha Nedali, le cousin de monsieur Barkad Achaari, mon collègue ici présent ?

\textsc{Sally} : Oui, si vous voulez.

\textsc{Lisnic} : Ce n’est pas si je veux, c’est oui ou c’est non.

\textsc{Sally} : Oui, je le reconnais, et croyez bien que je le déplore.

\textsc{Lisnic} : Vous avez également affirmé, en parlant de ce que vous avez qualifié assez légèrement de «~petit héritage familial~» de monsieur Moustapha Nedali, qu’il, je cite «~trimballe dans son cabas un service trois pièces de star du X~». Vous comprenez bien qu’une telle affirmation, hors cadre, constitue une grave atteinte à la dignité de l’intéressé.

\textsc{Sally}, tirant nerveusement sur une Vogue Superslims Bleue qu’on voyait trembloter entre ses doigts aux ongles interminables recouverts d’une épaisse couche de vernis rose fuchsia : Je ne vois pas en quoi.

\textsc{Lisnic} : Donc vous maintenez vos propos ?

\textsc{Sally} : Et comment, que je les maintiens !

\textsc{Greg} : Vous feriez mieux de lâcher l’affaire.

\textsc{Nathan}, entre ses dents : Commence à me courir sur le haricot, la grosse.

\textsc{Lisnic}, une vague lueur de plaisir sadique déformant ses lèvres minces et cruelles de jeune loup aux dents longues : Dans ce cas, Mme Robinson, vous me permettrez de vous poser la question suivante : Qu’est-ce que vous permet d’affirmer que, je cite, «~monsieur Moustapha Nedali trimballe dans son cabas un service trois pièces de star du X~» ? Vous l’avez déjà vu ?

\textsc{Sally} : Mousse se baladait souvent en slip dans les loges.

\textsc{Lisnic} : En slip ?

\textsc{Sally} : En caleçon, si vous préférez.

\textsc{Lisnic} : Quel genre de caleçon ?

\textsc{Sally} : Je ne vais pas citer de marque, mais c’est le genre boxer qui moule bien le paquet. Poutre apparente, comme on dit.

\textsc{Lisnic} : Je vois. D’où ma question suivante : Pourquoi monsieur Moustapha Nedali évoluait-il dans cette tenue dans les loges ?

\textsc{Sally} : Pour une raison très simple, monsieur l’agent…

\textsc{Lisnic} : Brigadier-chef, s’il vous plaît.

\textsc{Sally} : … monsieur le brigadier-chef, pardon : parce qui’il avait besoin d’aller dans les loges pour se changer, comme tout le monde.

\textsc{Lisnic} : Se changer ?

\textsc{Sally} : Oui. Je vous rappelle que le Sugar \& Spice est ce qu’on appelle un cabaret, autrement dit un établissement dans lequel se produisent des artistes de music-hall qui sont amenés à changer régulièrement de tenue.

\textsc{Lisnic} : J’avais cru comprendre que monsieur Nedali travaillait comme serveur dans cet établissement ?

\textsc{Sally}, d’une voix exprimant clairement son exaspération : Oui, mais dans notre établissement même les serveurs ont droit à une tenue particulière. C’est comme dans les clubs Playboy : les Bunnies portent un costume de lapin sexy avec corset, nœud pap, oreilles et queue de lapin. Eh bien chez nous c’est pareil : les serveurs portent une tenue de service.

\textsc{Lisnic} : Quel genre de tenue ?

\textsc{Sally} : Légère.

\textsc{Lisnic} : Vous voulez dire que Mr Nedali travaillait en slip ?

\textsc{Sally} : Les employés portent une tenue de service adaptée à leurs besoins, spécialement conçue pour ne pas entraver leur liberté de mouvement tout en étant agréable à regarder.

\textsc{Lisnic} : C’est à dire ?

\textsc{Sally} : Un boxer en latex avec étui pénien.

\textsc{Lisnic} : C’est une plaisanterie ?

\textsc{Sally} : Pas du tout, non. Tous nos serveurs portent un boxer en latex avec étui pénien, très apprécié de la clientèle. Pour répondre à votre question, encore que rien ne m’oblige à le faire, c’est grâce à ça, et aussi au fait que j’ai à de très nombreuses reprises eu le privilège d’entrevoir Moumousse sous la douche, que je suis en mesure d’affirmer qu’il était plutôt bien équipé de ce côté-là.

\textsc{Lisnic} : Donc, vous reconnaissez à demi-mot qu’il vous arrive de mater les gens sous la douche.

\textsc{Sally} : Ecoutez, monsieur l’agent brigadier-machin-chose, les mœurs sont très libres au Sugar \& Spice, et il n’est pas rare que nous prenions nos douches en commun. Les sportifs font ça régulièrement, on n’en fait pas tout un plat.

\textsc{Lisnic} : Mais les sous-entendus vont bon train.

\textsc{Sally} : Il y a longtemps que je ne me soucie plus de ce que racontent ou pensent les gens. Quand on a, comme c’est mon cas, une personnalité que je qualifierai de clivante, on apprend à passer outre ce genre de considérations. Si on n’en est pas capable, mieux vaut rester terré chez soi.

\textsc{Moi} : Je ne voudrais surtout pas me mêler de ce qui ne me regarde pas, chacun, je le répète, étant libre de se balader dans la tenue de son choix et se tripoter sous la douche avec qui bon lui semble, mais il commence à se faire tard.

\textsc{Sally} : Je ne vois effectivement pas où est le problème entre personnes majeures et consentantes.

\textsc{Achaari} : C’est pas la question !

\textsc{Sally} : Si. Et j’ajoute que rien ne m’oblige à répondre à vos questions débiles concernant ma vie privée. Maintenant, si vous voulez me menotter et me placer en garde à vue parce que votre cousin se balade à moitié à poil sur son lieu de travail, je vous en prie, ne vous gênez surtout pas !

\textsc{Greg} : Si vous pouviez éviter d’en rajouter.

\textsc{Nathan} : Je crois en effet qu’on a déjà perdu assez de temps comme ça.

\textsc{Sally} : Désolé, mais je n’apprécie pas tellement d’être stigmatisé en raison de mes orientations sexuelles.

\textsc{Achaari} : Il ne s’agit pas de ça !

\textsc{Sally} : Bien sûr que si. Vous vous croyez autorisé à me traiter comme une moins que rien parce que je suis différente des autres. Si quelqu’un peut se sentir blessé ici, c’est moi et pas vous. J’ai des témoins, je pourrais fort bien porter plainte.

\textsc{Lisnic} : Je ne suis certain que votre stratégie de défense soit la bonne.

\textsc{Sally} : Quelle stratégie de défense ? Je ne vois pas pourquoi je devrais me défendre de quoi que ce soit ! C’est plutôt vous qui aurez à répondre des accusations indignes que vous portez contre moi.

\textsc{Greg} : Vous ne pouvez pas la boucler un peu ?

\textsc{Sally} : Oui, bien sûr, on m’insulte ouvertement, tout juste si on ne me traite pas de pervers, de délinquant sexuel, et je devrais me taire, tendre gentiment la joue gauche ! Eh bien non, monsieur le détective, personne ne fait taire Sally Robinson !

\textsc{Nathan} : Quelle soirée de merde !

\textsc{Moi} : Messieurs, je vous en prie !

\textsc{Nathan} : Quoi, c’est vrai, on n’a rien fait de mal ! Pourquoi est-ce que ces soi-disant gardiens de la paix nous traitent comme de vulgaires malfrats ? C’est de l’abus de pouvoir, ni plus ni moins ! Vous n’avez pas mieux à foutre que d’emmerder les honnêtes gens ?

\textsc{Lisnic} : On ne fait que notre travail, monsieur. La loi nous oblige à intervenir quand on repère un attroupement au milieu de la chaussée. C’est ce que nous avons fait, et tout se passait pour mieux avant que cet individu ne se mette à proférer des insanités au sujet du cousin de mon collègue ici présent.

\textsc{Sally} : Je n’ai fait que rapporter l’exacte vérité des faits.

\textsc{Lisnic} : Reconnaissez que vous vous êtes livré à des actes de provocation gratuite.

\textsc{Achaari} : Dont je ne serais pas étonné qu’ils soient directement liés aux origines ethniques de mon cousin et moi-même.

\textsc{Sally} : Si je comprends bien, vous me traitez de raciste, maintenant, alors que c’est vous qui avez fait preuve de discrimination à mon encontre.

\textsc{Lisnic} : Pas du tout.

\textsc{Achaari} : J’ai rarement vu une telle mauvaise foi !

\textsc{Moi} : Messieurs, je ne voudrais pas abuser de mes prérogatives, mais en temps que votre supérieur hiérarchique, je pense en effet que la comédie a assez duré. Vous admettrez que vos agissements sont assez peu en rapport avec la procédure habituelle. En clair, si vous continuez à nous casser les couilles au lieu de faire le boulot pour lequel le contribuable vous rémunère, je me verrai dans l’obligation d’en référer à qui de droit. Suis-je clair ?

\textsc{Lisnic} : Très clair, mon commandant.

\textsc{Moi} : J’ai été ravi de faire votre connaissance.

\textsc{Lisnic} : Et réciproquement, mon commandant.

\textsc{Moi} : Hélas, il n’est de bonne compagnie qui ne se quitte.

\textsc{Greg} : Eh oui, toutes les bonnes choses ont une fin.

\textsc{Sally} : Contrairement aux mauvaises qui ne s’arrêtent jamais.

\textsc{Lisnic} : Mes respects, mon commandant. J’espère que vous ne garderez pas un trop mauvais souvenir de notre rencontre.

\textsc{Moi} : Nullement. Je n’irais peut-être pas jusqu’à dire que vous faites honneur à la police française, mais vous êtes sur la bonne voie. Si l’occasion se présente, je rendrai compte à vos supérieurs de l’excellente qualité de votre intervention.

\textsc{Lisnic} : Merci, mon commandant. Et veuillez excuser mon subordonné, l’agent Achaari. Il fait parfois preuve d’excès de zèle, mais dans le fond c’est quelqu’un de tout à fait fiable et généreux.

\textsc{Sally} : Fiable et généreux ?

\textsc{Lisnic} : Fiable et généreux. Il ne laisse jamais un collègue en plan et privilégie le plus souvent la prévention à la répression.

\textsc{Achaari} : Toutes mes excuses, mon commandant, si je vous ai paru un peu excessif. Je reconnais que j’ai tendance à m’emporter dès qu’on touche à la famille. Je suppose que ça ne me regarde pas, mais le capitaine n’a pas l’air bien du tout.

Sam, en effet, était allongé sur le trottoir, les bras en croix.

\textsc{Moi} : Ne vous en faites pas, il fait ça tout le temps. C’est sa façon à lui de se reposer après une dure journée de travail.

\textsc{Nathan} : Ça va, Sam ?

\textsc{Sam} : Ça va, Sam ?

\textsc{Nathan} : Je te demande si ça va, Sam ?

\textsc{Sam} : Je te demande si ça va, Sam ?

\textsc{Nathan} : Je suis embêté, quand même. Je me demande si on ne devrait pas l’emmener faire un petit tour aux urgences.

\textsc{Lisnic} : C’est vrai qu’il n’a pas l’air bien.

\textsc{Achaari} : C’est bizarre cette façon de répéter ce qu’on lui dit.

\textsc{Moi} : Messieurs, votre sollicitude me touche. Mais je connais Sam depuis des années, je sais parfaitement comment m’y prendre avec lui. Il souffre d’une forme assez rare de tétanie passagère. La crise dure quelques minutes, puis il revient à lui comme si de rien n’était.

\textsc{Greg} : Il a longtemps travaillé pour les forces spéciales, en Syrie et en Afghanistan notamment, et a reçu de nombreuses décorations pour services rendus à la Nation. Un jour, près de Kandahar, il participait à une opération de nettoyage d’un trou à rats infesté de terroristes. Il s’est pris une balle dans la tête, et c’est un miracle qu’il soit encore en vie. Wilfrid Chauveau, professeur de neurochirurgie à la Pitié-Salpêtrière, n’en revient toujours pas. Il a fallu toute une série d’interventions extrêmement pointues, dont certaines ont duré plusieurs heures, pour le sortir de là. Seulement, depuis, il est sujet à des accès de mélancolie et des troubles du comportement plus ou moins spectaculaires. De temps en temps, comme c’est le cas ce soir, il s’allonge sur le sol, les bras en croix, la bouche ouverte et les yeux révulsés, et séjourne dans cette position un temps indéterminé.

Un gargouillis, style cuvette de chiottes qui se débouche d’un seul coup, est sorti de la gorge de Sam, après quoi ses yeux ont effectué quelques tours complets dans leurs orbites, il a remué un bras, puis l’autre, et posé sur ce qui l’entourait un regard empreint d’une certaine forme de curiosité brumeuse.

Greg s’est aussitôt porté à son chevet : Sam ? Ça va ?

\textsc{Sam} : Euh… oui, je crois… On est où, là ?

\textsc{Greg} : Rue des Nénuphars.

\textsc{Sam} : Merde, alors ! Qu’est-ce qu’on fout là ?

\textsc{Greg} : Tu ne te rappelles pas ?

\textsc{Sam} : Non.

\textsc{Greg} : Mais lui (il parlait de moi), tu le reconnais, quand même ?

\textsc{Sam} : Ben oui, c’est Djef.

\textsc{Greg} : Et moi, je suis qui ?

\textsc{Sam} : Greg. Et lui c’est Nathan. Mais ça ne dit toujours pas ce qu’on fait là.

\textsc{Moi} : Tu ne te rappelles de rien, donc ?

\textsc{Lui} : De rien du tout. Qu’est-ce que je fais allongé par terre ?

\textsc{Greg} : C’est rien, juste un petit malaise. Suite à ton opération du cerveau.

\textsc{Sam} : J’ai été opéré du cerveau ?

\textsc{Moi} : Oui. Même que tu as bien failli ne jamais récupérer la totalité de tes facultés mentales.

\textsc{Greg}, entre ses dents : Je ne suis pas certain qu’il les ait totalement récupérées.

\textsc{Sam} : Merde, alors !

\textsc{Greg} : Oui, hein. Mais tout ça c’est du passé, rassure-toi. Tout va bien, maintenant.

\textsc{Sam} : Je peux savoir ce que la police fait là ?

\textsc{Greg} : Rien. Elle allait partir, justement. N’est-ce pas, messieurs ?

\textsc{Lisnic} : Oui, tout à fait.

\textsc{Achaari} : On ne faisait que passer.

\textsc{Lisnic} : On a vu de la lumière, on s’est arrêté pour jeter un œil.

\textsc{Achaari} : Réflexe professionnel.

\textsc{Lisnic} : C’est notre devoir de prêter assistance aux gens dans la détresse. Content de voir que tout va bien, mon capitaine.

\textsc{Sam} : Et la grosse, là, c’est qui ?

\textsc{Sally} : Il parle de moi, là ?

\textsc{Moi} : Je crois bien, oui.

\textsc{Greg} : C’est Sally Robinson, une cliente à moi.

\textsc{Sam}, avec une moue de dégoût : Rarement vu un boudin pareil !

\textsc{Sally} : Non mais je vous en prie ! Pour qui il se prend, cet abruti !

\textsc{Moi} : Vous voyez bien qu’il n’est pas dans son état normal.

\textsc{Sally} : Tout de même, ce n’est pas une façon de parler aux gens !

\textsc{Nathan}, tout en aidant Sam à se relever : Ça va, tu tiens debout ?

\textsc{Sam} : Ça va, merci.

\textsc{Nathan} : Viens, on aller s’assoir un peu dans le van.

\textsc{Sam} : Pas envie de m’assoir.

\textsc{Nathan} : Faut te reposer.

\textsc{Sam} : Pourquoi faire ? Je me sens tout à fait bien, maintenant. Une absence passagère, rien de plus.

\textsc{Greg} : Tu ne te rappelles vraiment de rien ?

\textsc{Sam} : De quoi je devrais me rappeler ?

\textsc{Greg} : Tu ne te rappelles pas qu’il y avait d’autres gens avec nous ?

\textsc{Sam} : Des gens ? Non. Quels gens ? Et elle, c’est qui ?

\textsc{Greg} : Je viens de te le dire : c’est Sally Robinson, une cliente à moi.

\textsc{Sam} : Ah bon. Tu pourrais les choisir un peu mieux, tes clientes. J’ai rarement vu une bonne femme aussi moche. On dirait un homme.

\textsc{Sally} : Je vais lui en coller une !

\textsc{Greg} : Ne faites pas attention à lui.

\textsc{Sally} : Je ne suis pas venue ici pour me faire insulter !

\textsc{Sam} : Et qu’est-ce qu’elle fait là, cette mocheté ?

\textsc{Sally} : J’exige que cet imbécile me fasse des excuses. Tout de suite !

\textsc{Greg} : Calmez-vous, je vous prie.

\textsc{Sally} : Je me calmerai si je veux ! Vous me décevez beaucoup, monsieur Lussier. Beaucoup ! Je pensais que vous étiez quelqu’un de sérieux et raisonnable, et je me rends compte que vous n’êtes qu’une ordure comme les autres !

\textsc{Greg} : Excusez-moi, mais je ne vois pas très bien le rapport.

\textsc{Sally} : Tous ces gens que vous présentez comme étant vos amis sont des beaufs racistes et homophobes !

\textsc{Moi} : Pardon ?

\textsc{Nathan} : Qu’est-ce qu’elle dit, cette morue ?

\textsc{Sally} : Là, vous voyez !

\textsc{Greg} : Nathan, s’il te plaît !

\textsc{Nathan} : Désolé, mais je n’ai pas l’intention de me laisser insulter par ce monstre de foire !

\textsc{Sam} : Haha ! Elle fait moins sa maline, la grosse !

\textsc{Sally} : C’est inadmissible, insupportable !

\textsc{Greg} : Ne nous emballons pas.

\textsc{Sally} : Je ne resterai pas une minute de plus avec des gens aussi grossiers ! Je ne vous félicite pas, monsieur Lussier.

\textsc{Greg} : Je suis désolé. La fatigue, le stress, les événements qui s’entrechoquent…

\textsc{Sam} : Dans ton cul, oui.

\textsc{Sally} : Ça suffit, je m’en vais !

\textsc{Nathan} : C’est ça, barre-toi, Elephant Man !

\textsc{Sam} : Retourne dans ta cage, King Kong !

\textsc{Sally} : Tu peux parler, toi, le débile !

\textsc{Sam} : C’est à moi que tu parles, Moby Dick ?

\textsc{Sally} : Demeuré !

\textsc{Sam} : Je vais te défoncer, le trave !

\textsc{Lisnic} : Allons, allons, je vous en prie ! Tout se passait bien jusqu’à présent, ne m’obligez à intervenir.

\textsc{Nathan} : On t’a rien demandé, à toi, le Moldave pourrave !

\textsc{Greg} : Nathan, s’il te plaît.

\textsc{Lisnic} : Je vais faire comme si je n’avais rien entendu.

\textsc{Nathan} : Bien sûr, que t’as rien entendu. De toute façon, t’es sourd comme un pot. Un pot de chambre moldave !

\textsc{Achaari} : Chef !

\textsc{Lisnic} : Quoi ?

\textsc{Achaari} : On nous signale une tentative de viol à quelques rues d’ici. Apparemment la fille rentrait chez elle en maillot de bain, après une soirée bien arrosée, et elle s’est fait agresser par un type déguisé en sanglier.

\textsc{Lisnic} : Comment ça ? Avec les sabots et tout et tout ?

\textsc{Achaari} : Non, juste la tête. Un masque de sanglier, si vous préférez.

\textsc{Sally}, à Sam : HOMOPHOBE !

\textsc{Sam} : TA GUEULE, LE TRAVELO !

\textsc{Sally} : TRISOMIQUE !

\textsc{Greg} : Tout ça va trop loin. Djef, fais quelque chose !

\textsc{Lisnic} : Oui, faites quelque chose, mon commandant. La situation est en train de déraper.

\textsc{Achaari} : On y va, chef ? On a assez perdu de temps avec cette bande de cinglés.

\textsc{Sam} : Dis donc, toi, le bouffeur de couscous, remballe ta merguez et va faire joujou dans la semoule !

\textsc{Moi}, avec la voix de Pierre Thau, inoubliable interprète des plus grands rôles de basse du répertoire (Méphisto, Don Quichiotte, la statue du Commandeur, le Comte des Grieux et Don Pedro, pour ne citer que les plus fameux) : VOUS ALLEZ LES FERMER VOS GRANDES GUEULES, OUI OU MERDE !!!!!!!!!!!!!! NOM DE DIEU DE BORDEL DE MERDE !!!!!!!!!! JE COMMENCE VRAIMENT À EN AVOIR PLEIN LE CUL DE VOS CONNERIES !!!!!!!!!!!!!!

\textsc{Greg} : Il a les raison, les gars, vous faites chier !

\textsc{Moi}, à Sally : Et vous, allez-vous-en, votre présence ne fait qu’envenimer la situation. (Aux flics :) Ça vaut aussi pour vous, bande de nazes, avec tout le respect que je vous dois.

\textsc{Lisnic} : Ne vous en faites pas, mon commandant, on est sur le départ.

\textsc{Achaari} : On a cette affaire de fille en maillot de bain violée par un sanglier rue Timothée Carbonneau qui requiert toute notre attention.

\textsc{Moi} : C’est tout à votre honneur !

\textsc{Lisnic} : Oui. On serait bien resté à bavarder avec vous, mais le devoir nous appelle. Un viol en soi c’est pas si grave, surtout quand les risques de grossesse sont limités par l’hybridation, mais le problème c’est qu’on observe souvent un effet «~trainée de poudre~» où tout le monde se met à violer tout le monde dans une espèce de frénésie sexuelle incontrôlable. On a vu des ménagères de moins de cinquante ans se faire violer par des chihuahuas, des majorettes par des étalons en rut, des écolières par des éléphants de mer échappés du zoo, heureusement la plupart du temps trop lents pour les poursuivre efficacement, et même des macaques roux abusés sexuellement par des langurs de Hanuman~-- à Shimla, notamment, où cette vermine pullule -, des racailles de banlieue par des fils de bonnes familles aux yeux injectés de sang et aux bourses anormalement dilatées, j’en passe et des meilleurs, les champs du possible étant quasiment illimités dans ce domaine de compétence. L’amour seul peut nous sauver, j’en sais quelque chose en tant qu’oracolophile distingué.

\textsc{Achaari} : Hamdullah !

\textsc{Lisnic} : J’ajoute que l’animal en question, décrit par les témoins comme une bête de forte corpulence aux réactions imprévisibles, apparemment dotée d’un membre énorme, surdimensionné tant par la longueur que le diamètre, capable d’assommer un bœuf aussi sûrement d’une barre de fer, est en fuite, en train d’errer quelque part dans le quartier en laissant derrière lui une traînée de liquide spermatique, chose qui ne va pas sans poser problème pour la sécurité de nos concitoyens.

\textsc{Moi} : Je dois reconnaître que vous avez idéalement narré la chose, mon ami.

\textsc{Greg} : Quel talent !

\textsc{Sarah} : C’est pas tous les jours qu’on croise un flic moldave qui maîtrise aussi bien la langue de Molière.

\textsc{Lisnic} : Je vous remercie tous du fond du cul.

\textsc{Achaari} : Hamdullah !

\textsc{Lisnic} : Mais le temps presse. J’ai peur, si on tarde à intervenir, qu’une milice populaire armée jusqu’aux dents ne tente de se substituer aux forces de l’ordre.

\textsc{Moi} : Assurément.

\textsc{Lisnic} : Quelle joie de se lever aux aurores pour exercer un aussi beau métier que le nôtre, et se coucher à point d’heure avec la satisfaction du devoir accompli.

\textsc{Moi} : Bien peu de gens, dans cette société qui a perdu ses valeurs et se délite à vue d’œil, peuvent se vanter d’avoir une existence aussi palpitante que celle du fonctionnaire de police.

\textsc{Achaari} : Protéger et servir, telle est notre mission.

\textsc{Greg} : Je reconnais, en tant que privé, qu’il m’arrive parfois d’envier la fonction publique dans ce qu’elle a de plus noble et désintéressé.

\textsc{Moi} : Au revoir, messieurs.

\textsc{Lisnic}, au garde à vous : Force et honneur, mon commandant.

\textsc{Achaari}, dans la même position : Protéger et servir, telle est notre devise.

\textsc{Moi} : Amen.

Après avoir claqué des talons et exécuté un demi-tour quasi parfait, les deux guignols sont remontés dans leur caisse et partis sur les chapeaux de roues vers de nouvelles aventures. Quelle vie palpitante que celle du gardien de la paix, toujours sur la brèche pour secourir la veuve et l’orphelin, l’honnête homme persécuté par une vermine tenace, la joggeuse importunée par des pervers sexuels et autres sociopathes avides de sexe et de pouvoir, les deux étant souvent les pis d’une même mamelle tuméfiée d’affirmation obsessionnelle de sa personnalité inexistante, de revanche sur une vie ingrate qui traite ses rejetons comme de la merde et les expose sans cesse aux pires vicissitudes, poussant le vice jusqu’à leur faire miroiter une récompense bien méritée s’ils souffrent en silence jusqu’à leur dernier souffle.

\textsc{Sally}, remettant ses nichons en place dans son soutien-gorge débordé : Eh bien puisque c’est comme ça, que tout le monde se fiche comme d’une guigne du respect de mes droits les plus élémentaires à l’égalité et la différence, je m’en vais moi aussi. Mais soyez certain, Mr Lussier, que je suis extrêmement déçue par votre comportement et surtout celui de vos amis. Je pensais avoir affaire à des gentlemen, je me suis bien trompée.

\textsc{Greg} : Je vous rappelle, chère madame, que j’attends toujours le règlement du solde de mes honoraires.

\textsc{Sally} : Tu peux t’assoir dessus, mon grand.

\textsc{Greg} : Pardon ?

\textsc{Sally} : C’est comme Tiago Alvarez. Il s’agissait soi-disant de venger sa mort, mais ça aussi tout le monde s’en fout !

\textsc{Greg}, perdant quelque peu le contrôle de ses nerfs d’acier, au point d’employer des mots d’une certaine grossièreté, ordinairement absents de son vocabulaire : Tu vas me payer ce que tu me dois, morue !

\textsc{Moi}, émergeant peu à peu des vapeurs brumeuses de l’alcool : Greg, s’il te plaît. Tu ne vas pas t’y mettre, toi aussi.

\textsc{Greg}, manifestement en proie à cette forte excitation qui le dominait dès lors qu’il s’agissait d’évoquer l’aspect financier de ses activités : Je n’ai pas pour habitude de bosser pour des prunes.

\textsc{GSallyreg}, après un ricanement d’une méchanceté absolue, ce genre de ricanement sardonique d’où toute humanité semble avoir effacée par la main même de Satan : Bosser !!! Laissez-moi rire !

\textsc{Greg}, prêt à fondre sur sa proie telle la harpie féroce sur le paresseux tridactyle assoupi dans la jungle guyanaise (ou~-- pour reprendre une expression chère à Simone de Bavoir, alors qu’elle n’était pas encore le castor de Jean-Paul Tartre~-- le jaguar richement orné sur le stylistiquement insignifiant pécari à lèvres blanches, bien trop occupé à retourner bruyamment le terre avec son groin humide pour se rendre compte du danger qui le menace) : Je vais buter cette salope !

\textsc{Nathan}, peinant à cacher sa joie de voir la situation se dégrader encore un peu plus : C’est ce qu’on aurait dû faire dès le début.

\textsc{Sam}, se frottant les mains et se pourléchant les babines, des éclairs de haine joyeuse plein les yeux (j’emploie à dessein le mot «~joyeuse~» car la perspective de faire subir à son prochain les pires atrocités était synonyme pour lui d’ambiance festive et hautement récréative, tendance qui préexistait chez lui depuis la tendre enfance, s’était très largement accrue au cours des longues années en territoire hostile, et pouvait à présent s’épanouir en toute liberté dans ses activités sécuritaires à caractère privé) : Laissez, je m’en charge !

\textsc{Votre serviteur}, qui, après avoir un temps formé le projet d’éradiquer les Disciples de la Colère (Jégou, Monteil et Desmarais) de la surface de la terre, ne rêvait plus à présent que d’une seule chose : regagner prestement ses pénates, se glisser subrepticement sous la couette~-- tel un serpent à sonnette ayant passé la journée à ramper sous le soleil brûlant de l’Arizona, serrer son corps recru (adjectif peu usité auquel j’aimerais redonner un semblant de vie, comme ça, même si je m’en fous royalement par ailleurs, des choses autrement plus graves ayant lieu actuellement dans le monde, ce que je veux dire par là c’est qu’on n’en est pas forcément à un adjectif près dans la langue française, riche par nature, peut-être trop, même si, on le sait, des choses en apparence infimes, des détails de l’Histoire, peuvent engendrer de durables turbulences, lesquelles gagnent en intensité au fil du temps, des siècles si nécessaire, voire des millénaires, et finissent par conduire à des catastrophes majeures qui elles-mêmes, par une étrange distorsion de la perception humaine, peuvent apparaître comme des déflagrations épiphénoménologiques sans réelle consistance, des flatulences civilisationnelles) contre celui de sa dulcinée, et se laisser lentement glisser dans la tiédeur réparatrice d’un sommeil bien mérité : Non, Sam, personne ne va se charger de personne ! Pas pour l’instant, en tout cas.

\textsc{Sam} : Faut que je tue quelqu’un, tout de suite !

\textsc{Moi} : Un peu de patience, mon garçon. Chère madame Robinson, je comprends votre déception. Moi-même, si j’étais à votre place, je crois que je serais extrêmement déçue. Enfin, déçu. Tiago Alvarez, votre amant…

\textsc{Sally} : L’amour de ma vie, vous voulez dire !

\textsc{Moi} : L’amour de votre vie, si vous voulez, a été refroidi par une bande de brutes épaisses qui ne méritent aucune clémence.

\textsc{Greg} : Refroidi, c’est une façon de parler.

\textsc{Moi} : Okay. Je retire refroidi, et je remplace par euh… réchauffé…

\textsc{Greg} : Je dirais plutôt carbonisé.

\textsc{Moi} : Réchauffé à mort, si tu préfères. Excuse-moi, mais on ne va pas passer la soirée à jouer sur les mots !

\textsc{Greg} : Assurément. Ne serait-ce que par égard pour Sally qui est toujours sous le coup de la plus vive émotion.

\textsc{Moi} : La plus vive.

\textsc{Sally} : Je vous déteste !

\textsc{Moi} : Désolé, Sally, mais je crois que vous avez compris le fond de ma pensée. Au cas où ce ne serait pas le cas, je vous donne ma parole que Tiago Alvarez sera vengé comme il se doit. À ce propos, j’aimerais, si vous le permettez, faire part à l’assemblée d’une petite idée qui m’est venue pour solutionner le problème.

\textsc{Sam} : Moi aussi !

\textsc{Moi} : Quoi, toi aussi ?

\textsc{Sam} : J’ai une idée pour solutionner le problème.

\textsc{Moi} : Génial. Et quelle est-elle ?

\textsc{Sam} : On bute la grosse et on rentre chez nous !

\textsc{Moi} : Négatif. Ton idée c’est de la merde, et tu commences à me casser sérieusement les burnes ! Je ne te cache pas que je suis assez déçu par ton comportement.

\textsc{Nathan} : Il a été malade quand il était petit.

\textsc{Moi} : Genre les oreillons ?

\textsc{Nathan} : Pire. Son père le battait et sa mère faisait des passes pour arrondir les fins de mois.

\textsc{Sam} : Qu’est-ce que tu racontes ?

\textsc{Nathan} : Quoi, c’est pas vrai ?

\textsc{Sam}, après qu’un vol de corbeaux ait traversé le ciel nuageux de son cerveau : Euh… si, plus ou moins, mais je tiens pas forcément à ce que tout le monde le sache.

\textsc{Greg} : Il faut nommer les choses pour avoir une idée claire de ce qu’elles sont.

\textsc{Sam} : Quoi ?

\textsc{Greg} : Rien, laisse tomber.

\textsc{Nathan} : Sam, je voulais juste dire que t’es un mec pas toujours facile à gérer.

\textsc{Moi} : C’est sans doute doute pour ça qu’il n’a pas fait carrière dans l’armée.

\textsc{Sam} : Quoi ? J’ai pas fait carrière dans l’armée ?

\textsc{Moi} : Pas plus que ça, non.

\textsc{Sam} : Je suis capitaine, je te rappelle.

\textsc{Greg} : Croquemitaine, à la rigueur.

\textsc{Sam} : Très drôle ! Si je suis parti de l’armée, c’est uniquement parce que j’estimais que mes qualités n’étaient pas récompensées à leur juste valeur.

\textsc{Greg} : Ben voyons !

\textsc{Moi} : Non, Sam. Si t’es parti de l’armée, comme tu dis, c’est parce qu’on t’a foutu dehors pour alcoolisme et comportement déplacé envers les jeunes recrues.

\textsc{Nathan} : Et en plus, tu t’amusais à clouer des animaux morts sur des troncs d’arbre !

\textsc{Greg} : Ouais, comme Jeffrey Dahmer !

\textsc{Sam} : Vous insinuez quoi, là ?

\textsc{Greg} : Rien.

\textsc{Nathan}, remonté : T’as jamais tué des chats avec la 22 de ton père quand t’étais petit ?

\textsc{Sam}, gêné : Peut-être une fois ou deux, je me rappelle plus.

\textsc{Nathan} : Après tu les ramenais discrètement dans ta chambre et tu les disséquais.

\textsc{Sam} : Oui, peut-être. C’est loin, tout ça.

\textsc{Nathan} : Avant de les plonger dans l’acide sulfurique pour voir combien de temps il fallait pour les dissoudre complètement.

\textsc{Sam} : Tous les gosses font ça. On leur apprend à disséquer des rats et des grenouilles en cours de sciences naturelles.

\textsc{Greg} : Non, tous les gosses ne butent pas des chats à la 22 pour les dissoudre dans de l’acide.

\textsc{Sam} : Les gosses sont cruels avec les animaux. T’as jamais arraché les ailes des mouches, toi ?

\textsc{Greg} : Arracher les ailes des mouches, passe encore, même s’il faut être franchement débile pour s’amuser à ça. Mais tuer des chats à la 22 pour les disséquer et les dissoudre dans de l’acide, on change clairement de catégorie !

\textsc{Moi} : Le bien et le mal sont des concepts qui n’existent pas dans la nature. La seule chose qui compte, c’est d’assurer la survie et le développement de l’espèce.

\textsc{Greg} : D’accord, mais l’espèce humaine est un cas particulier. Il a fallu créer des garde-fous pour limiter la casse. Et même avec ça, on a du mal à empêcher les gens de s’entretuer.

\textsc{Moi} : L’homme n’est pas prêt pour la liberté. Trop dangereux. Ce serait comme laisser un loup affamé errer au milieu d’un troupeau de moutons bien gras et dodus. Il faut lui mettre une muselière, le trimballer en laisse et passer derrière lui pour ramasser son caca.

Tout en parlant, et m’écoutant parler avec une certaine délectation (la teneur de mes propos me ravissait autant que le son de ma voix, aussi chaude et enveloppante qu’une couche de pâte feuilletée autour d’un rôti de bœuf), j’ai avisé Sally qui tentait de profiter de l’occasion pour se faire discrètement la malle.

J’aurais pu la laisser partir, trop content de m’en débarrasser, mais le sens de l’autre et l’amour du travail bien fait qui m’habitent en permanence me permettent rarement de céder à la facilité.

Je l’ai donc stoppée dans son élan : Vous partez déjà ?

Elle s’est arrêtée, retournée, et a rétorqué d’une voix étouffée par la rage et la déception : Je crois que je n’ai plus rien à faire ici.

J’ai dit : Je comprends que vous soyez dépitée. Mais je vous en prie, revenez ici, j’ai encore des choses à vous dire.

\textsc{Sally} : Je ne vois pas quoi. Mais si vous insistez.

\textsc{Moi} : Non seulement j’insiste, mais je vous donne ma parole que votre ami sera vengé.

\textsc{Sally}, au bord des larmes : C’était bien plus qu’un simple ami.

\textsc{Moi} : Je le sais, et vous présente une fois de plus, en mon nom et ceux de mes camarades ici présents, mes plus sincères condoléances. Mais j’aimerais maintenant, si vous le permettez, revenir à l’idée que je souhaitais soumettre à l’assistance. Comme vous n’êtes pas sans le savoir, les principaux suspects dans cette sombre affaire sont les sieurs Jégou, Monteil et Desmarais, tristes sires s’il en est, qui forment le noyau dur d’un groupuscule d’extrême-droite connu sous le nom de Disciples de la Colère. Je dirais plutôt Disciples de la Connerie. Quand ils ne sont pas en train de se saouler la gueule au Bouclier, un bar facho de la rue de Gobineau, ils se retrouvent rue Jordan Peshkov, chez Jégou qui fait office de cerveau de la bande, ou plutôt chez sa grand-mère puisque c’est là qu’il habite, dans le sous-sol transformé en bunker nazi, pour préparer la troisième guerre mondiale et le retour triomphal du führer qui attendait patiemment son heure dans un caisson de cryogénisation. En attendant ce jour de gloire, ils bricolent des complots racistes, antisémites et homophobes pour déstabiliser la société et saper les fondements de la démocratie. La vieille, sourde comme un pot et toute acquise aux thèses négationnistes de son petit-fils adoré, qu’elle a élevé depuis sa plus tendre enfance, ses parents s’étant barrés en vitesse dès qu’ils ont compris qu’ils avaient engendré un monstre, leur mitonne des petits plats pour qu’ils ne crévent pas de faim pendant les longues soirées d’hiver passées à refaire le monde, ou plutôt imaginer le plus sûr moyen de le réduire en cendres. Pour en revenir à l’affaire qui nous occupe, Cerqueira, qui était présent au moment des faits, affirme avoir tout tenté pour empêcher Desmarais de commettre le pire. Mais autant essayer de retirer un os de la gueule d’un dogue allemand ! Plombier de métier, cette sous-merde est aussi pompier volontaire. Pas pour sauver des gens, je vous rassure tout de suite, mais pour assouvir sa fascination pour le feu. Voir cramer des choses le met en joie, lui déclenche des érections d’anthologie, lui procure des orgasmes apocalyptiques, choses d’autant plus rares, exceptionnelles et profitables qu’il souffre d’impuissance chronique. C’est au contact des flammes qu’il s’épanouit, et sa grande passion consiste à faire semblant d’éteindre des incendies qu’il a lui-même allumés. Si on en croit Cerqueira, et je suis assez enclin à le faire, c’est Desmarais qui a carbonisé Alvarez au lance-flammes. Ils ont tous, de près ou de loin, participé à cette abomination, mais c’est à Desmarais que revient la palme du pire taré de service. Je vais donc m’en occuper personnellement, à mon rythme, et corriger le tir de dame Nature qui s’est une fois de plus fourvoyée sur la longue route semée d’embûches de l’évolution. Je propose que Sam s’occupe de Jégou et Greg et Nathan de Monteil.

\textsc{Nathan} : Je vais me le faire au Cougar MT-6, avec des pointes de flèches explosives.

\textsc{Greg} : Je ramasserai les morceaux.

\textsc{Sam} : Moi je vais me le faire à l’ancienne, à mains nues. Je vais lui déboiter les articulations une par une, et ensuite je lui briserai gentiment la nuque pour mettre un terme à ses souffrances.

\textsc{Moi} : Excellent. Pour ma part, je n’ai pas encore de modus operandi précis en tête, mais je vais essayer de trouver quelque chose de sympa, si possible en rapport avec sa passion pour Hitler, le suprémacisme et la destruction méthodique de son prochain.

\textsc{Greg} : Et Titus ?

\textsc{Moi} : Ça fait déjà trois fois que j’essaie de l’appeler.

\textsc{Greg} : Il fait quoi, à ton avis.

\textsc{Moi} : Il est prisonnier de la Gardienne de la Nuit, une déesse surgie des entrailles de la terre pour lui sauver la vie.

\textsc{Greg} : Tu crois ?

\textsc{Moi} : C’est comme ça que je vois les choses. En attendant, s’il n’est rentré à la maison à la première heure, Bérénice va me tomber dessus et je vais avoir toutes les peines du monde à lui expliquer ce qui s’est réellement passé.

\textsc{Greg} : Tu sais ce qui s’est réellement passé ?

\textsc{Moi} : Pas vraiment, non.

\textsc{Sam} : Si on allait le chercher ?

\textsc{Moi} : Où ça ?

\textsc{Nathan} : La fille a parlé d’un hôtel tout près d’ici.

En tant qu’homme moderne parfaitement connecté au monde réel et toujours au fait des dernières avancées en matière de technologies de pointe, Greg a aussitôt sorti son smartphone, appareil sur lequel il disposait de toutes les applis nécessaires pour quadriller le périmètre au dixième de millimètre près : Attends, je regarde s’il y en a un dans le coin.

\textsc{Moi}, estimant que le moment d’allumer un Gurkha Ghost Spook était amplement arrivé : Alors ?

\textsc{Greg} : Il y en a un à deux pas d’ici, le Caribbean Hôtel, rue des Maléfices.

\textsc{Moi}, savourant avec une délectation indécente l’épaisse fumée de mon cigare : Ça ne me dit rien qui vaille.

\textsc{Sam} : Sans doute un hôtel de passe.

\textsc{Nathan} : On devrait aller y faire un tour.

\textsc{Moi} : Je ne suis pas chaud. Après tout, Titus est majeur et vacciné. Il a parfaitement le droit de faire des folies de son corps si bon lui semble.

\textsc{Greg} : Je te rappelle qu’il a une femme et des enfants.

\textsc{Moi} : Merci, je suis au courant.

\textsc{Greg} : Et qu’il n’avait l’air dans son état normal quand il s’est fait embarquer par cette fille.

\textsc{Nathan} : Très bizarre, cette fille.

\textsc{Sam} : Ça ne coûte rien d’aller y jeter un œil.

\textsc{Moi} : Vite fait, alors.

\textsc{Sally} : Ce sera sans moi. Bonne nuit. Et n’oubliez pas, monsieur Lussier, que nous avons une affaire en cours.

\textsc{Greg} : Vous vouliez savoir qui a tué Alvarez, vous le savez. Le reste ne me concerne en aucune manière.

\textsc{Sally} : Vous n’aurez pas un centime de plus tant que cette ordure sera en vie. Je suis même prête à doubler la mise si vous me donnez satisfaction.

\textsc{Greg} : C’est du chantage pur et simple.

\textsc{Sally} : Appelez ça comme vous voudrez. Bonne nuit, messieurs.

\textsc{Sam} : Bon, on fait quoi ? On va à l’hôtel ou pas ?

\textsc{Moi}, après un temps d’hésitation : Okay, on y va !

