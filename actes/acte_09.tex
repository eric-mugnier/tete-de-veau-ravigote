\chapter{Acte 9}

\noindent Cette nuit-là, j’ai fait des rêves étranges dont seuls les épisodes les plus marquants surnagent encore dans ma mémoire.

Nathan, Sam, Greg, Titus et moi, accompagnés de Molawa VIII (voir note en bas de page) et ce bon vieux Jim Bowie, ancien trafiquant d’esclaves en Louisiane, caracolions dans la vallée de San Fernando à la recherche d’un temple maléfique dédié à Atiena, la Gardienne de la Nuit au corps de rêve et aux yeux vert céladon. Sous la forme d’un centaure entouré d’un harem de juments de Thessalie, le général Antonio Lopez de Santa Anna y Perez de Lebron, alias l’Aigle du Mexique, le Napoléon du Nouveau Monde ou encore le Héros Immortel de Cempoala, leur filait le train en tirant des coups de feu dans tous les sens avec ses Colt 1860 Army calibre 44 PN (il en avait un dans chaque main et chevauchait à toute berzingue en s’agrippant à son canasson à la seule force de ses cuisses trapues couverte d’une pilosité abondante).

En équilibre instable sur le dos d’une mule pas très accommodante, les couilles en vrac à force de rebondir sur le dos de la bête, je ne vous cache pas que j’avais toutes les peines du monde à suivre le rythme endiablé de cette folle cavalcade.

À noter aussi, chose qui ne va pas nécessairement de soi compte tenu du contexte sus-décrit, que je ne portais ni cache-poussière, pantalon de cuir, santiags ni sombrero, mais un peignoir Aescwig Paige collection printemps-hiver 2017 en laine de soie peignée et microfibre de bambou 100\% bio du Suriname (cadeau de mon ami Zaahid Shirani, légiste de génie et beau-frère potentiel qui possédait exactement le même et prenait le plus grand plaisir à l’enfiler - entre autre - sitôt rentré chez lui après une dure journée de labeur) et des pantoufles en peau de Cottontail du désert (Sylvilagus auduboni, une espèce qu’on ne rencontre guère que dans les steppes semi-arides du désert de Sonora).

Pas non plus de six-coups tonitruant pour moi, mais juste Manu, mon fidèle 6.35 Manufrance, une arme de collection à laquelle j’étais attaché comme à la prunelle de mes yeux, vous le savez maintenant (il avait appartenu à mon grand-père Philibert, résistant de la première heure dont l’esprit taquin veillait en permanence sur moi depuis les hautes sphères où il résidait, en paix avec lui-même, sans doute occupé à tremper ses vers dans l’eau claire de quelque céleste ruisseau éternellement poissonneux, entouré de naïades au physique de reine de beauté attentives à satisfaire ses moindres désirs). À côté de la mitraille ambiante, les détonations de Manu ressemblaient aux jappements d’un de ces foutus roquets qui ne perdent jamais une occasion de brailler même lorsqu’ils se retrouvent face à un molosse qui fait cent fois leur taille, lequel molosse les regarde généralement de haut, l’œil morne, un filet de bave au coin du bec, sans y prêter plus d’attention qu’à une feuille morte balayée par le vent, dédain qui ne fait que pousser lesdits roquets à hurler de plus belle jusqu’à l’extinction de voix.

Note de bas de page, toujours pas en bas de page pour les mêmes raisons pratiques et esthétiques que précédemment : Molawa VIII, chef hottentot de la région du Cap, s’est éteint en 1830. Récupéré puis empaillé comme une vulgaire charogne exotique par les frères Verreaux, antiquaires peu portés sur le respect des morts mais beaucoup sur l’argent, il finit, début vingtième, par échouer entre les mains de Francesc d’Asis Darder et Llimona, vétérinaire catalan secoué du bocal et naturaliste pervers officiant en tant que directeur du zoo de Barcelone. Au crépuscule de sa vie, passée à amonceler jalousement tout ce que la nature a produit de plus insolite et terrifiant depuis les origines du monde, Darder lègue sa collection de reliques horrifiques au musée municipal de Banyoles, sa ville natale. Lance à la main et revêtu de son seul pagne bouffé aux mites, sa Majesté MOLAWA trône en bonne place dans la salle principale, au milieu des momies, têtes réduites, peaux de serpents, monstres en tous genres et autres joyeusetés zoologiques à l’avenant. Au début des années 1990, quand le fait d’exposer des pygmées dans des vitrines ne fait plus rigoler grand monde, hormis quelques nostalgiques du colonialisme et autres suprémacistes blancs, El Negro de Banyoles fait l’objet de ce que n’hésiterai pas à appeler une vive polémique. Les visiteurs de couleur se sentent quelque peu mal à l’aise face à ce compatriote empaillé comme un vulgaire babouin, témoin d’un passé pas si lointain, voire toujours d’actualité, où l’Homme Blanc régnait en maître absolu sur la Terre, traitant tout ce qui divergeait peu ou prou de son idéal occidental en bête de somme corvéable à merci. Alphonse Arcelin, médecin haïtien et néanmoins socialiste installé à Cambrils, sympathique station balnéaire du Baix Camp, près de Tarragone, entend parler de l’affaire, envoie une missive retentissante au maire de Banyoles, exigeant le retrait immédiat de cette horreur et sa restitution à qui de droit dans les meilleurs délais. Mais le maire en question, qu’une telle infamie n’émeut nullement, n’entend pas se laisser déposséder aussi aisément de l’une des principales sources de revenus de sa commune. Grâce à El Negro, son musée des horreurs ne désemplit pas. Il a augmenté les tarifs et aimerait bien continuer à se remplir les poches aussi longtemps que possible. Après tout, il ne s’agit que d’une enveloppe vide avec de la paille à l’intérieur, une enveloppe en forme de personne réelle, certes, mais qui permet aussi, au-delà de son aspect mortuaire discutable, de prendre la pleine mesure des fondements historiques de notre civilisation. Quelques siècles plus tôt, qu’on le veuille ou non, nos ancêtres découvraient avec stupeur qu’il existait des créatures plus proches d’eux qu’ils ne l’avaient imaginé jusqu’alors, d’apparence bien plus humaine que le singe, capables de communiquer entre elles avec un langage rudimentaire et d’effectuer des actions non dépourvues de logique et d’efficacité. Si la chose était avérée, l’histoire de notre filiation en serait grandement bouleversée. Alphonse Arcelin, fort de ses origines caribéennes, affronte alors une cabale néo-colonialiste qui nie haut et fort l’aspect éthique de la chose au profit de l’histoire et la science, seules valeurs acceptables dans ce cas de figure frappé de péremption. On ne va quand même pas se laisser casser les burnes par une poignée de primates endimanchés ! Si les descendants de ce glorieux personnage souhaitent se recueillir sur sa dépouille (comme neuve ou presque grâce aux bons soins des conservateurs attentifs qui se sont relayés à son chevet, soit dit en passant), ils devront faire le déplacement jusqu’à Banyoles, où l’entrée du musée leur sera bien évidemment gracieusement offerte, contrairement aux frais du voyage qui resteront à leur charge, faut pas non plus pousser mémé dans les orties.  Après une lutte acharnée qui causera la ruine du valeureux médecin et son exil à Cuba, Molawa sera rapatrié en grande pompe sur ses terres natales et se verra, en présence des édiles locaux en tenue d’apparat frétillant d’aise sous l’œil embué des caméras internationales, offrir enfin une sépulture digne de sa condition.

C’est donc passablement ensuqué que j’ai ouvert les yeux sur un monde dans lequel je ne tenais pas plus que ça à me retrouver, un univers dont l’étroitesse m’apparaissait chaque jour un peu plus manifeste, odieuse et insoutenable, une cellule dont les murs se rapprochaient chaque jour davantage, réduisant mon espace vital à une bulle inconfortable, un trou à rat dont l’atmosphère confinée devenait chaque jour un peu plus suffocante. J’avais même, l’espace d’un instant, espéré qu’il avait enfin disparu pour laisser la place à une île perdue au fin fond d’une mer imaginaire, une île paradisiaque bénéficiant d’un climat idéal, de paysages enchanteurs sans cesse renouvelés, d’une beauté inépuisable, psychédélique, un coin de paradis entièrement peuplé de créatures de rêve totalement dépourvues d’inhibition, comme quand Fletcher Christian débarque à Tahiti après avoir jeté cette pourriture de capitaine Bligh à la mer, des créatures de rêve fascinées par ma chevelure dorée et l’azur magnétique de mon regard de grand fauve mélancolique. J’avais espéré, même si je n’étais pas exactement le sosie de Marlon Brando (un Marlon tellement investi dans son rôle d’aventurier du Nouveau Monde qu’il repartira du tournage avec Tarita Teriipaia, l’interprète de la belle et sauvage Maïmiti, avec laquelle il aura deux gosses dont une fille, Cheyenne, qui se suicidera vingt-cinq ans plus tard après que son demi-frère Christian ait abattu Dag Drollet, son compagnon violent, de deux balles dans le dos, sachant que Brando père, l’inoubliable interprète, entre autres, du Bal des maudits, Reflets dans un œil d’or, Missouri Breaks, la Comtesse de Hong-Kong et la Poursuite impitoyable, était aussi un bel enfoiré prétentieux et Cheyenne une gamine pourrie-gâtée toxico qui souffrait de troubles mentaux, comme quoi tout ne va pas forcément pour le mieux dans l’existence, même à Mulholland Drive, chose qui n’avait sans doute pas échappé à Lee Tamahori et David Lynch), tirer un trait sinon sur le fardeau de mon ancienne vie, au moins sur le fiasco de la soirée précédente, et repartir sur de meilleures bases vers un avenir souriant au ciel dégagé des noirs nuages du passé.

Et au lieu de ça, de cette perspective gorgée du nectar sucré de l’espérance, je me retrouvais une fois de plus dans mon lit, sans la moindre envie d’en sortir, avec la sensation toujours aussi désagréable d’avoir passé la nuit à mastiquer des copeaux de chêne du Limousin au fond d’une cave sombre et humide (ce qui est le cas de la plupart des caves quand elles sont bonnes).

Au prix d’un effort surhumain, du type de ceux que fournissaient les héros de l’Antiquité pour déplacer des montagnes ou terrasser des dragons, j’ai réussi à me laisser tomber du lit en douceur et sortir de la chambre sans réveiller Zarina. En même temps, réussir à réveiller Zarina quand elle dormait à poings fermés constituait un exploit autrement majeur que tous ceux accomplis par les héros et autres demi-dieux de la Mythologie. Vous avez déjà vu un troupeau d’éléphants traverser la jungle au pas de charge, ou assisté à l’effondrement d’une tour de cent cinquante étages ? Eh bien, si ce même troupeau avait traversé la chambre au pas de charge ou cette même tour s’était effondrée au pied du lit, Zarina n’aurait même pas soulevé une paupière pour s’enquérir de ce qui se passait. Qu’un séisme, un ouragan s’abatte sur elle, que des rafales de vent d’une violence extrême arrachent les fenêtres de la chambre et que des pluies diluviennes la remplissent d’eau jusqu’au plafond ? Peu importe, elle aurait continué à dormir comme si de rien n’était. Vous pensez peut-être qu’une guerre atomique, déclenchée par les milliardaires caractériels qui dirigent ce monde, aurait pu venir à bout de sa résistance. Eh bien je vous fiche mon billet qu’on l’aurait retrouvée endormie au milieu des décombres fumants de l’immeuble, sans une égratignure, aussi fraîche que la rosée du matin. Sauf qu’il n’y aurait plus de rosée du matin, ni de matin tout court, du reste, mais seulement la nuit sans fin de l’Apocalypse, le silence assourdissant de la mort et les vapeurs toxiques qui recouvrent tout d’une épaisse couche de moisissure radioactive. Et quand elle ouvrirait enfin les yeux, ce serait pour se rendre compte que tout avait disparu. Enfin presque tout, parce que j’aurais survécu moi aussi, bien évidemment. ADAM \& EVE 2, l’éternel retour, mais sans le jardin d’Eden, les arbres chargés de fruits défendus et les serpents maléfiques qui sifflent sur nos têtes pour nous pousser à croquer dedans. Tel le Phénix d’Héliopolis, le Fenghuang de la dynastie Han, le Minka aborigène ou encore le Wakinyan Tanka des Sioux, je renaîtrais de mes cendres, nu comme un asticot blafard et grassouillet, aussi intact et glorieux qu’au premier jour, après quoi je m’empresserais d’aller réveiller ma princesse endormie d’un baiser langoureux sur ses lèvres brûlantes de désirs inassouvis. C’est à nous que reviendrait le privilège de repeupler le monde de petits condensés de nous-mêmes et relancer la folle histoire de l’espèce humaine, indigne s’il en est, aussi médiocre et imparfaite que terriblement prétentieuse, imbue d’elle-même jusqu’à l’overdose existentielle, le trauma crypto-sensoriel de forte magnitude, mais exceptionnellement autorisée, à travers nous, à tenter une nouvelle fois sa chance dans ce coin perdu au fin fond de l’univers qu’on appelle la Terre.

Et d’ailleurs, à propos de repeupler l’univers, je ne sais pas si vous avez remarqué, mais les lendemains de cuite, outre un solide état de déprime arrimé au cuir chevelu, se signalent souvent par un niveau d’excitation sexuelle anormalement élevé. La plupart du temps, il faut recourir à une activité masturbatoire proche de la frénésie pour y remédier. Je suppose que le sexe est un antidépresseur naturel, comme le safran, la camomille, l’orpin et le millepertuis, ce qui est encore une façon très habile (et habituelle de la part d’une entité aussi sournoise et rouée que dame Nature, rompue à tous les artifices pour arnaquer son monde), compte tenu du marasme existentiel qui le nôtre, d’encourager l’espèce à se reproduire malgré tout, ce qui, sur le plan de la pure objectivité scientifique, ferait clairement de l’alcoolisme un rouage essentiel de l’évolution des espèces, la nôtre en l’occurrence, sachant que notre avenir est de toute façon largement compromis. Un peu plus un peu moins, on n’en est pas à une cirrhose près. Dans le cas présent, j’aurais pu me soulager dans l’arrière-train de Zarina sans qu’elle y trouve à redire, d’autant moins, comme je viens de vous l’expliquer, que rien ne pouvait la réveiller quand elle dormait à poings fermés. Entendez par là que le troupeau d’éléphants (pachydermes dont on imagine sans peine la taille colossale du membre et le poids abyssal des burnes) sus-évoqué aurait pu lui passer dessus à plusieurs reprises, un par un ou tous ensemble, et je parle cette fois au sens de l’expression dans ce qu’elle a de plus bestial et ordurier, sans susciter la moindre réaction de sa part. J’ai néanmoins, dans ma grande sagesse, jugé préférable de n’en rien faire. Quant à l’érection monumentale (tout est relatif, bien sûr, rien à voir avec celles d’un Jonah Falcon ou un Roberto Cabrera) qui me précédait dans tous mes déplacements (fidèle animal de compagnie tenu fermement en laisse sans quoi il n’aurait pas manqué de se ruer sur tout ce qui bouge), je lui ferais rendre gorge dès que mon emploi du temps éminemment chargé me le permettrait. Il était de toute façon hors de question de me présenter au bureau dans cet état, le personnel féminin n’ayant alors pas manqué de me traîner en justice sous les chefs d’inculpation les plus divers, tels que inculpation ou inculpalpation, pourquoi pas, tandis que le personnel masculin, moins procédurier, n’en aurait pas moins conçu à mon égard une rancune tenace pour avoir osé exhiber en toute impudeur des attributs d’une telle magnificence.

Mon périple matinal, loin d’être terminé, m’a ensuite conduit à la cuisine, où j’ai entrepris de mettre en branle la machine à café.

Pendant qu’il coulait, j’avais tout le loisir de me traîner jusqu’à la salle de bain afin de prendre une bonne douche et me laver les dents, autant d’activités d’un ennui mortel auxquelles je me devais néanmoins de sacrifier avant de rejoindre les rangs de la civilisation. Il se trouve que moi, Djeferson Beauvais avec un D et un seul F, commandant de police judiciaire de Saint-Nom-la-Branlette, j’étais soumis à quelques obligations difficilement dispensables, même s’il m’arrivait parfois d’officier dans un état de semi-ébriété coupable, avec des trous dans les chaussettes au niveau du gros orteil et un slip d’une propreté douteuse (le modèle Brando en jersey de coton de chez Dolce \& Gabbana, d’accord, mais tout de même).

Un quart d’heure plus tard, après m’être conscieusement récuré la couenne, généreusement vidé les bourses et dentifricé le ratichier, j’étais, sinon l’éphèbe éminemment désirable de ma prime jeunesse, au moins en état de faire illusion auprès des gens que je serais amené à rencontrer pendant la journée.

C’est donc plein d’espoir que je me suis rendu à la cuisine pour siroter mon demi-litre de café matinal, dégusté seul, pour lui-même, sans l’adjonction de quelque autre denrée alimentaire que ce soit.

J’ai jeté un œil à mon téléphone et constaté que Bérénice avait essayé de m’appeler plusieurs fois aux premières lueurs de l’aube, alors que j’en écrasais encore sévèrement sous les quelques centimètres carrés de drap que Zarina avait consenti à me laisser, sachant que chaque dixième de millimètre de chacun d’eux avait été conquis de haute lutte, à la force du poignet, au cours de la nuit assez peu réparatrice qui avait été la mienne (Zarina, quoique dépourvue d’origines asiatiques à ma connaissance, avait en effet la fâcheuse manie de s’enrouler dans les draps comme une farce au porc, champignons noirs, vermicelle et pousses de soja dans une fine galette de riz).

Je m’apprêtais à la rappeler, aussi rapidement que possible mais sans excès de vitesse inconsidéré (cet appel ne me disait rien qui vaille et je n’étais par conséquent pas particulièrement pressé de donner suite), quand la sonnerie de mon téléphone (une version électro de la Chevauchée des Walkyries de Richard W, suffisamment immonde et irritante pour forcer le plus réfractaire à la téléphonie mobile et au progrès en général à décrocher dans la seconde) s’est fait entendre.

J’ai pris une grande respiration, décroché et tenté de présenter mes salutations respectueuses à l’intéressée.

Elle ne m’en a pas laissé le temps : Je peux savoir ce que vous avez fait hier soir ?

\textsc{Moi} : Hein ? Quoi ? Qui ça ?

\textsc{Elle} : Titus et toi.

Moi, l’air aussi dégagé que possible : Eh bien… pas grand chose, à vrai dire… Pourquoi tu me demandes ça ?

Elle, hurlant presque : TITUS N’EST PAS RENTRÉ DE LA NUIT !

Moi, qui avais refusé de croire à cette éventualité et devais me rendre à l’évidence que j’avais sans doute fait preuve d’un peu trop de légèreté dans cette histoire : Ah!… hum hum… oui, en effet, c’est assez embêtant… Alors comme ça, tu dis que… qu’il…

\textsc{Elle} : N’est pas rentré de la nuit, oui, c’est ce que je viens de dire !

\textsc{Moi} : C’est bizarre, quand même.

\textsc{Bérénice} : Oui, je trouve aussi. Est-ce que, par le plus grand des hasards, tu aurais des informations à ce sujet ?

\textsc{Moi} : Eh bien, écoute, euh, là j’avoue que tu me prends un peu de court… Il est tôt, je viens de sortir du lit, je ne suis pas encore très bien réveillé… Alors… oui… effectivement… maintenant que tu le dis… je crois me souvenir qu’il nous a quitté à une heure assez avancée de la nuit… aux alentours de… enfin… je ne me rappelle plus exactement… mais… assez tard, en effet… ou tôt, suivant la façon dont on… enfin, tu vois ce que je veux dire…

\textsc{Elle} : Et ?

\textsc{Moi} : Eh bien je me suis dit qu’il rentrait chez lui, voilà tout.

\textsc{Elle} : Il a dit qu’il rentrait ?

Votre serviteur, au comble de l’inconfort : Euh… non, il n’a rien dit de particulier à ce sujet… mais je ne vois pas ce que je pouvais penser d’autre à ce moment-là.

\textsc{Elle} : Eh bien, il se trouve qu’il n’est pas rentré.

\textsc{Moi} : T’es sûre ?

\textsc{Elle} : Certaine, oui.

\textsc{Moi} : Peut-être qu’il est rentré et… et qu’il est ressorti…

\textsc{Elle} : J’ai le sommeil assez léger, je crois que je m’en serais rendu compte. Vous faisiez quoi, quand il est parti ?

\textsc{Moi} : Mon dieu… pas grand chose, comme je te l’ai dit… On était en opération spéciale… strictement confidentielle… je ne peux malheureusement pas t’en dire davantage à ce sujet, question de sécurité, la tienne et celle de tes proches… mais il se trouve que… comment dire… eh bien il se trouve que les choses ne se sont pas exactement déroulées comme prévu… Les impondérables, tu sais ce que c’est.

\textsc{Elle} : Non, pas trop. JE CROIS SURTOUT QUE T’ES EN TRAIN DE BIEN TE FOUTRE DE MA GUEULE !!!!!!!!!

Moi, écarquillant des yeux ronds comme des culs de babouins : COMMM-MMMENT ???????????!!!!!! Mais enfin comment oses-tu imaginer une chose pareille ! Non, je suis offusqué, là, limite vexé. L’affaire n’avançait pas, s’étirait inutilement, en un mot comme en cent on se trainait comme des merdes dans le purin, des limaces dans la bave, des…

\textsc{Elle} : Oui, bon, ça va, épargne-moi tes comparaisons douteuses !

\textsc{Moi} : Bref, on s’est dit que ça ne servait à rien de continuer à patauger dans la semoule comme des mamelouks en détresse. Quand Titus a décidé de partir, j’en ai conclu tout naturellement qu’il rentrait chez lui.

\textsc{Elle} : Ben voyons. Je l’ai appelé plusieurs fois, j’ai laissé des messages, et il ne répond toujours pas ! Tu peux essayer de faire quelque chose, s’il te plaît ?

\textsc{Moi} : Mais bien sûr, Bérénice, tu sais aussi bien que moi que tu peux compter sur… enfin sur moi, quoi… Je vais faire tout ce qui est en mon pouvoir (autant dire pas grand chose, note de la Rédaction) pour le retrouver. Cela dit, je ne suis pas sûr qu’il réponde davantage si c’est moi qui appelle…

\textsc{Elle} : Mais tu peux localiser son téléphone, non ?

\textsc{Moi} : Oui, en principe.

\textsc{Elle} : Comment ça, en principe ?

\textsc{Moi} : Eh bien… il y a des fois où on peut, et d’autres fois où on ne peut pas…

\textsc{Elle} : Ah bon ? Vous n’avez pas des spécialistes pour faire ce genre de chose ?

\textsc{Moi} : Si, mais peut-être qu’il n’a plus de batterie.

\textsc{Elle} : On ne peut pas le localiser s’il n’a plus de batterie ?

\textsc{Moi} : C’est plus difficile. Peut-être aussi que son téléphone a été détruit pour une raison ou pour une autre…

\textsc{Elle} : Et alors ? Il pourrait… je ne sais pas, moi, m’appeler d’une cabine ! Il sait pertinemment que je me fais un sang d’encre quand je suis sans nouvelles ! Tu n’as vraiment aucune idée de l’endroit où il peut être ?

Moi, tel Judas Iscariote (le Judas «sikariot» de la Peshitta, traduction syriaque de la Bible chère aux Maronites) claquant la bise au Roi des Juifs dans les jardins de Gethsémani : Ma foi… comme ça… non, je ne vois pas… Je vais me renseigner pour savoir s’il ne lui est pas arrivé quelque chose… Je ne pense pas, mais on ne sait jamais… un accident… que sais-je…

\textsc{Elle} : Je ne sais pas pourquoi, mais j’ai la désagréable impression que tu me caches quelque chose.

Moi, pour une poignée de deniers : Hein ??? Quoi ??? Comment ???

\textsc{Elle} : Tu as très bien entendu.

Et pour quelques deniers de plus : Mais jamais de la vie, voyons ! Qu’est-ce que tu vas imaginer !

\textsc{Elle} : Je connais les hommes.

Moi, plus roublard que Tuco : Je ne vois pas de quoi tu veux parler.

\textsc{Elle} : Ils sont prêts à se couvrir les uns les autres jusqu’à la mort, au nom de je ne sais quel sens débile de l’amitié virile et l’esprit de corps !

\textsc{Moi} : Ce n’est pas pire que les femmes qui passent leur temps à se tirer dans les pattes !

\textsc{Elle} : Tu me jures que tu ne sais pas où est Titus ?

Moi, plus répugnant qu’une punaise de lit sortant furtivement de sa tanière pour sucer le sang de sa victime endormie : Tu penses bien que je te le dirais si je savais quelque chose ! Titus est mon meilleur ami, c’est vrai, mais toi et les enfants comptez aussi beaucoup pour moi.

\textsc{Elle} : N’empêche que tu n’as pas répondu à la question.

\textsc{Moi} : Quelle question ?

\textsc{Elle} : Est-ce que tu me jures que tu ne sais pas où est Titus ?

Moi, après un raclement de gorge suffisamment prononcé pour laisser planer le vautour déplumé du doute au-dessus de la charogne faisandée de ma duplicité grouillante des asticots replets de la bêtise crasse et la lâcheté triomphante (oui, bon, ça va, personne n’est parfait) : Bien sûr, que je te le jure, mon amour !

\textsc{Bérénice} : Quoi ???!!!!

Moi, confus : Excuse-moi, ça m’a échappé dans le feu de l’action.

\textsc{Elle} : Tu m’aimes ?

\textsc{Moi} : Bien sûr, que je t’aime, comme la femme de mon meilleur ami, comme une sœur, une personne chère, hors de prix, à laquelle il ne me viendrait jamais à l’idée de mentir.

\textsc{Elle} : Ah bon, tu m’as fait peur !

\textsc{Moi} : Tu veux dire que si je te disais que je t’aime, non pas seulement comme la femme de mon meilleur ami, mais comme une femme qui fait naître en moi les désirs les plus… comment dire… intenses, au sens érotique du terme, tu en concevrais une certaine… rancœur, amertume ?

\textsc{Elle} : Disons que je serais un peu perturbée, oui.

\textsc{Moi} : Eh bien rassure-toi, il n’en est rien.

\textsc{Elle} : Encore heureux !

Moi, évoluant avec une grâce discutable sur le fil d’un rasoir qui commençait à m’entamer sérieusement la voûte plantaire : Ecoute, Bérénice, ma douce et tendre Bérénice, Bébé mon bébé, je ne sais pas au juste ce que tu vas imaginer, mais dis-toi bien que …

Elle, d’une voix dont la surface se voulait d’un calme olympien mais n’en laissait pas moins transparaître une agitation comparable à celle d’un groupe d’orques en train de mettre en pièces un requin-baleine dans les eaux troubles du golfe du Mexique : Je n’imagine rien du tout. Je note juste que mon mari a découché, ce qui ne lui était encore jamais arrivé sans un mot du médecin, qu’il ne répond pas au téléphone quand je l’appelle, et que son meilleur ami, un type plutôt bizarre avec lequel il a passé la nuit, prétend ne rien savoir de ce qui lui est arrivé. Quant aux enfants, qui ont l’habitude de déjeuner avec leur père avant de partir à l’école, ils me posent des questions auxquelles je suis incapable de répondre.

\textsc{Moi} : Dis-leur qu’il est parti au boulot un peu plus tôt que prévu.

\textsc{Elle} : Certainement pas ! Je ne leur ai jamais menti, ce n’est pas maintenant que je vais commencer.

\textsc{Moi} : C’est tout à ton honneur.

\textsc{Elle} : C’est jamais bon de mentir aux enfants.

\textsc{Moi} : C’est évident. Ils comptent sur nous, nous font une totale confiance. Si on commence à leur raconter n’importe quoi, ils perdent leurs repères et partent en vrille.

\textsc{Elle} : Quand la confiance est altérée, tout s’écroule.

\textsc{Moi} : Ouais. Cette connerie de Père Noël, par exemple, ce vieux barbu en pyjama rouge, qui se la joue papy sympa mais sent le pédophile à plein nez, le «bad Santa» obsédé du cul, se trimbalant en permanence une érection digne d’un cheval de trait, une gaule à casser des noix, censé se balader en traineau dans le ciel et passer par les cheminées avec sa hotte pleine de jouets… Ben voyons ! Vu son tour de taille, ce bouffon n’arriverait même pas à passer dans le tunnel sous la Manche sans toucher les bords ! C’est l’exemple-type des conneries qu’on raconte aux gosses pendant des années, jusqu’au jour où ils apprennent brutalement qu’on leur a menti soi-disant pour la bonne cause, essayer de mettre un peu de rêve dans la bouillie insipide de la vie quotidienne, fêter dignement le jour béni où Jésus a ouvert les yeux dans une étable de Bethléem, le plus simplement du monde, sans se soucier du fait que l’étable en question, remplie de paille et éclairée à la bougie, pouvait prendre feu à tout moment.

Elle, très remontée, au point d’utiliser une formulation qui n’était pas habituellement la sienne (je n’irais pas jusqu’à dire que j’en étais choqué, mais tout de même, je découvrais une facette de sa personnalité dont j’avais à peine soupçonné l’existence jusqu’ici) : Rien à secouer de l’âne, du bœuf et des Rois mages !

\textsc{Moi} : Exact, qu’ils aillent tous se faire foutre ! À commencer par ce prétendu Messie qui aurait mieux fait de nous prévenir gentiment qu’on allait passer le restant de nos jours à mentir, voler, violer et assassiner son prochain ! Lui aussi nous a bien roulés dans la farine comme des boulettes de viande avant de passer à la casserole ! En même temps, comment faire confiance à un type qui prétend être le fils de Dieu, rien que ça, débarque pour sauver le monde et n’est même pas fichu de se sauver lui-même ? Mais excuse-moi, je crois que je m’égare un peu.

\textsc{Elle} : Non, t’as raison. Je ne comprends toujours pas pourquoi on continue à fêter la naissance de ce type deux mille ans après. Regarde, moi, par exemple, on n’a pas fait tout ce ramdam le jour où je suis venue au monde !

\textsc{Moi} : Pareil pour moi, à commencer par mes parents qui s’en foutaient comme de l’an quarante !

\textsc{Elle} : Ah bon ?

\textsc{Moi} : Bien sûr. Je te l’ai déjà raconté, non ?

\textsc{Elle} : Je crois pas, non.

Moi, pas avare de confidences aux premières heures du jour, quand les derniers fêtards rentrent chez eux ivres morts, que les travailleuses du sexe sucent leurs derniers clients, que les agents de propreté urbaine collectent leurs dernières ordures sur les trottoirs de la ville, que les rats d’égout regagnent furtivement leur tanière la dent jaune et le ventre plein, que les collégiens matent fébrilement une dernière vidéo de boule sur les réseaux sociaux avant d’aller en classe, que le soleil se lève paresseusement entre deux nuages endormis, que les dealers goûtent quelques heures de repos bien mérité avant de retourner au taf, que les politiciens véreux, hommes d’affaires corrompus, chanteurs de charme affiliés à la mafia et autres marchands de mort spécialisés dans la vente d’armes aux pays sous embargo sillonnent le ciel dans leurs jets privés pour aller planquer leur pognon dans les paradis fiscaux, que les milliardaires paranoïaques et mégalos, accros aux drogues dures, aux partouzes et à la pêche au gros, lancent leurs dernières OPA pour devenir enfin les maîtres du monde, se prendre des cuites au Clos D’Ambonnay, se rouler dans les œufs d’esturgeon et se taper des putes mineures sur des yachts de 50 mètres de long au large des côtes de Tanzanie : Ma mère ne voulait pas d’enfant, mais elle était super canon et mon père ne pensait qu’à la niquer par tous les orifices. La plupart du temps, il se contentait de l’enculer vite fait sur la table de la cuisine ou dans le local à poubelles, chose qui, je ne m’explique toujours pas pourquoi, avait le don de le mettre dans un état de transe sexuelle irrépressible, et comme ma mère oubliait régulièrement de prendre la pilule et que mon père n’était pas un adepte du retrait d’urgence au moment fatidique, ce qui ne devait pas arriver arriva néanmoins. Je naquis donc, beau bébé replet promis au plus brillant avenir, à ceci près que mon père voulait une fille et que le service trois pièces de toute beauté que j’affichais au sortir du ventre maternel ne laissait planer aucune équivoque sur la nature de mon sexe. Sa tronche s’est décomposée quand il a posé les yeux dessus, il a claqué les talons, comme tout bon militaire qui se respecte, s’est fendu d’un demi-tour-droite irréprochable techniquement, a quitté la chambre sans dire un mot, droit comme un I, puis est allé se saouler la gueule avec ses potes de régiment pendant une durée indéterminée qui a duré exactement trois jours et trois nuits. Ensuite, il s’est repointé à la maison comme si de rien n’était, et ne s’est pas gêné pour m’en faire baver des ronds de chapeaux jusqu’à ce que je sois enfin en âge de me barrer sans demander mon reste.

\textsc{Elle} : Ton père était dans l’armée ?

\textsc{Moi} : Oui, un de ces tueurs professionnels qui sillonnent la planète pour faire un peu de ménage. Sur ce, je te souhaite une bonne journée.

\textsc{Elle} : Tu te fous de ma gueule ?

\textsc{Moi} : Non… excuse-moi… ce n’est pas ce que je voulais dire… enfin si… passe une bonne journée dans le sens où ne t’en fais pas, je vais retrouver Titus et tout va rentrer dans l’ordre.

J’ai prévenu le bureau que je risquais d’être en retard, mais comme j’étais le chef je pouvais faire à peu près ce que je voulais et me permettre d’être en retard quand j’avais des affaires plus importantes à traiter que ce qui constituait l’ordinaire des activités d’un commissariat de quartier, à savoir faire des rondes dans les quartiers chauds, contrôler des gens de couleur qui n’ont rien demandé à personne, faire souffler des automobilistes dans des éthylotests, verbaliser des pervers à la sortie des écoles, et enregistrer des plaintes de vieilles dames qui se sont fait braquer leurs économies par des escrocs souriants et bien habillés (il avait l’air si gentil).

Je me suis habillé en vitesse, et j’ai ouvert délicatement ma porte pour jeter un œil sur le palier. Pas de mère Ouvrard ni de Marc-Antoine Jacquinot en vue, tapis dans un angle mort pour fondre sur moi comme l’aigle sur sa proie. Juste Korax avachi sur le paillasson de sa porte d’entrée, plus sournois et dédaigneux que jamais, bien entendu, mais apparemment peu disposé à tenter une de ces attaques surprises dont il avait le secret. C’est donc l’esprit relativement dégagé que j’ai pu prendre l’ascenseur, dévaler (à vitesse réduite, l’engin n’étant pas de la toute première jeunesse, ni même de la seconde) les six étages qui me séparaient de la terre ferme, sortir de l’immeuble et respirer enfin les fraîches senteurs de l’aube si nécessaires à mon équilibre mental, même si raisonnablement polluées par le monoxyde de carbone, le dioxyde d’azote et les particules fines chargées d’hydrocarbures polycycliques hautement cancérigènes contenus dans les gaz d’échappement.

Contrairement à son habitude, le 1,6 16v de ma Kangoo, boosté à près de deux cent cinquante chevaux de course par les bons soins de Nathan, le frère de Greg, a consenti à s’ébrouer après seulement deux tentatives infructueuses, ce qui n’était pas loin de constituer un record personnel. Le modeste quatre cylindres produisait maintenant un feulement rauque de grand fauve en rut, subtilement mis en valeur par une ligne d’échappement optimisée qui m’avait coûté la peau des fesses. La prochaine mission de Nathan, si toutefois il l’acceptait, consisterait à doter le paisible utilitaire de tous les aménagements nécessaires pour se transformer à loisir en voiture-bélier, aéronef ou sous-marin de poche. Enfin, un exosquelette à cristaux liquides permettrait de rendre le véhicule invisible à la demande, par simple action sur un commutateur dédié. On n’en était pas encore là, mais je disposais tout de même d’un certain nombre de gadgets intéressants, tels que blindage intégral, quatre roues motrices avec pneus increvables, autoradio à écran tactile Bluetooth, système audio haute performance avec ampli de 900 watts, mais aussi et je dirai même surtout boîte à gants transformée en humidor pour une parfaite conservation de mes cigares préférés.

Et justement, à propos de cigares, j’ai ouvert la boîte à gants, me suis glissé un Fuente Short Story dans le bec, et l’ai allumé à la flamme de mon Guevara Vintage double jet anti-tempête (utile en mer par gros temps quand on souhaite s’en griller un sous des trombes d’eau, ce qui n’est pas chose facile, ou encore quand on se retrouve pris au piège par une tornade qui avance inexorablement en ravageant tout sur son passage et qu’on s’apprête à allumer ce qui sera sans doute son dernier cigare).

Un peu tôt pour commencer à fumer, je vous l’accorde, mais les mauvaises habitudes sont les plus difficiles à perdre et la journée s’annonçait aussi chargée que l’haleine d’une hyène qui vient d’arracher ses derniers lambeaux de chair à une carcasse de gnou faisandée depuis des lustres sous le soleil impitoyable de Namibie. Le Kalahari, par exemple, en dépit de l’aridité redoutable qu’on lui connaît, n’en offre pas moins des paysages d’une beauté à couper le souffle, et ce ne sont certainement pas les amateurs de safari en Land Rover et de bivouac sous les étoiles qui me contrediront. Car tous savent bien qu’il n’est pas donné à tout le monde de connaître ces moments inoubliables où le lion rugit à quelques mètres de vous, le vautour vous tourne autour dans le ciel chauffé à blanc, le guépard vous regarde droit dans les yeux et le suricate vous mange dans le creux de la main. Tout comme il n’est pas donné à tout le monde - et heureusement, du reste, car ils n’hésiteront pas à vous tirer une flèche empoisonnée dans le cul si vous vous avisez de leur manquer de respect - de croiser la route des fiers Bochimans qui arpentent les pistes du désert depuis des dizaines de milliers d’années, parlent à la Lune comme à une vieille amie et pratiquent aujourd’hui encore les étranges rituels hérités de leurs plus lointains ancêtres. Bien sûr, ils ne fument pas la pipe ou le cigare, comme vous et moi, et n’imaginent même pas qu’on puisse résider à plein temps dans un manoir des Highlands, mettre des glaçons dans un verre de Bruichladdich de 30 ans d’âge, porter des costumes sur mesure, piloter des avions de chasse et tirer des missiles balistiques sur des sites stratégiques d’Europe de l’Est ou du Moyen-Orient. Non, tout ça leur passe largement au-dessus de la tête. Mais ils n’en ont pas moins, dans le dénuement le plus total, vêtus de simples peaux de bêtes et exposés sans cesse à la cuisante morsure du feu solaire, réussi à survivre et s’intégrer dans l’environnement hostile qui a toujours été le leur. Jamais, certes, vous ne les verrez défiler en kilt dans les rues de Fort William, s’époumoner dans des cornemuses ni admirer les moutons de Rosa Bonheur dans les musées de Londres, Hambourg ou Washington, mais soyez certain qu’ils sauront vous défendre au péril de leur vie si le babouin vous menace, le phacochère vous charge ou l’éléphant furibard tente de vous écrabouiller comme la grosse merde de capitaliste nuisible et envahissant que vous êtes (le pachyderme en question étant tout particulièrement motivé du fait que vous n’avez cessé de le persécuter pour lui voler ses défenses et lui couper les pattes pour en faire des cendriers, ce qui montre bien l’extrême perversité et la frivolité inqualifiable de vos agissements).

Greg, réveillé aux aurores, m’attendait sur le pas de sa porte, rasé de près, l’œil vif et le jarret fringant, impatient de prendre part à ce qui s’annonçait sinon comme l’aventure la plus palpitante de sa vie, au moins de ces dix dernières années. C’était, à priori, le genre de chose qu’on prend plaisir à raconter à ses petits-enfants au coin du feu quand on est vieux, malade et sur le point de crever, à condition bien sûr d’avoir des petits-enfants et une cheminée à portée de main, et que le fait d’être sur le point de crever nous permette encore de prononcer quelques paroles vaguement intelligibles sans foutre la trouilles aux gosses.

À vrai dire, n’ayant ni enfants et encore moins petits-enfants, je crois surtout qu’il était impatient de prendre la fuite, Lou (ou Loulou suivant les cas, rarement Louloulou ou Loulouloulou, de son vrai nom Louise De La Croix, créature pour tout dire assez fascinante issue d’une vieille famille d’aristos désargentés, dont le père était un escroc notoire ayant fait plusieurs fois le tour de la planète pour échapper à ses créanciers, tant et si bien que plus personne - et sans doute pas même lui - ne savait précisément où il était ni même s’il était encore en vie, cette dernière option paraissant hautement improbable vu le nombre de gens prêts à sacrifier tout ou partie de leur fortune pour avoir sa peau), sa nouvelle copine nymphomane, ne lui laissant pas une seconde de répit. C’était une fille dont il avait fait la connaissance du temps où il bossait quinze heures par jour comme analyste financier chez Reckless \& Knot, avec laquelle il avait… comment dire… partagé quelques affinités électives dans les endroits et positions les plus divers et incongrus, avant de lui faire part de sa décision longuement mûrie en ses âme et conscience de mettre un terme à toute activité autre que strictement professionnelle les concernant, considérant que cette incursion dans le domaine du privé, en dépit de son caractère physiquement satisfaisant, là n’était pas la question, s’avérait néanmoins incompatible avec la bonne marche de l’entreprise en général et de sa santé mentale en particulier. Mais ne voilà-t-il pas, au moment où il s’y attendait le moins (il ne faut jamais dire jamais, c’est bien connu, et surtout se tenir prêt en permanence à toute éventualité), que l’imprévisible Lou venait de refaire surface et lui mettre à nouveau le grappin dessus (assez facilement, il faut bien  le dire), plus chaude qu’un brasier dévastant des centaines d’hectares de garrigue dans l’Hérault. Fragile psychologiquement, autant que sexuellement fortement attiré par cet incendie que nulle compagnie de soldats du feu n’était jamais parvenue à maîtriser, Greg avait commis l’erreur de remettre un doigt dedans et s’était aussitôt retrouvé pris au piège (je pense alors au fameux fingertrap de la famille Addams, cadeau de dixième anniversaire de Fester, le frère de Gomez, qui a dû apprendre à manger avec les pieds parce qu’il est resté coincé dedans pendant deux ans).

Et pour vous dire toute la vérité, eh bien sachez que le Greg que j’évoquais précédemment, rasé de près, à l’œil vif et au jarret fringant, tel un Bilbon Sacquet prêt à prendre part aux aventures les plus rocambolesques avec une bande de Nains sortis de nulle part, des magiciens et des Elfes sylvains pas toujours très bien disposés, affronter des hordes d’Orques hideux et terrasser des dragons gardiens de trésors sous des montagnes solitaires, ce Greg-là n’était hélas rien d’autre qu’une vue de l’esprit, un mirage, un fol espoir, un fantasme à mille lieues de la réalité affligeante qui s’offrait à ma vue dépitée.

Quand je l’ai vu arriver en traînant les pieds, tel un petit vieux accablé par le poids des ans et pressé d’en finir avec une existence qui lui a procuré en gros quatre-vingt-dix et quelques pour cent d’emmerdes pour dix tout petits pour cent de moments de vague satisfaction aussi rares que fugaces (autrement dit pas du tout assez pour faire pencher favorablement la balance), monter dans la bagnole avec l’entrain d’une vache qu’on fait entrer à coups de pied dans le cul dans une bétaillère pour la conduire à l’abattoir, me tendre une main tellement molle que j’ai eu l’impression de serrer la pince à une flaque de vomi, attacher sa ceinture avec résignation, poser les mains bien à plat sur ses genoux et se mettre à regarder fixement devant lui sans desserrer les dents, j’ai compris que je n’étais pas au bout de mes peines.

J’ai cherché son regard pendant quelques instants, sans la moindre réaction de sa part, et lui ai posé la première question qui m’est venue à l’esprit, assez basique il est vrai, et d’autant plus superflue que j’en connaissais déjà la réponse : Ça va ?

Greg, sans tourner la tête : Non, pas vraiment.

\textsc{Moi} : Mal dormi, peut-être ?

\textsc{Lui} : Pas fermé l’œil de la nuit.

\textsc{Moi} : Lou ?

\textsc{Lui} : Lou.

\textsc{Moi} : Elle t’attendait ?

\textsc{Lui} : Derrière la porte, dans sa tenue la plus suggestive.

Moi, la tête noyée dans un nuage de fumée : Lumière tamisée, senteurs orientales, lingerie fine, résille et balconnet, difficile de résister.

\textsc{Lui} : Pas exactement, mais elle a des moyens de persuasion très efficaces. Même un eunuque n’y résisterait pas.

\textsc{Moi} : Un moine bénédictin, peut-être ?

\textsc{Lui} : Pas davantage.

\textsc{Moi} : Tu veux dire que saint Benoît de Nursie lui-même n’aurait pas résisté longtemps à ses avances ?

\textsc{Lui} : Le pauvre vieux n’aurait pas tenu trente secondes.

\textsc{Moi} : Que si Augustin d’Hippone l’avait croisée dans les rues de Carthage ou les travées de la basilique Saint-Pierre de Rome, ses Confessions compteraient bon nombre de pages en plus ?

\textsc{Lui} : Bon nombre, et je ne suis pas certain qu’il aurait osé les écrire toutes.

\textsc{Moi} : J’en conclus qu’elle ferait bander un mort.

\textsc{Lui} : Un cimetière tout entier !

\textsc{Moi} : Avec elle, les cadavres sortent de terre avec une trique d’enfer ! Lève-toi et gicle !

\textsc{Lui} : Tout-à-fait. Tel Lazare recroquevillé dans le fond de sa tombe, ma bite croupissait dans le fond de mon slip, aussi inerte et engluée dans sa propre bave qu’une limace à l’agonie.

\textsc{Moi} : Quelle vision lugubre et déprimante. Ce membre éminent du genre humain à jamais perdu pour la France, le monde, l’univers tout entier ! La théorie de l’évolution bouleversée dans ses fondements mêmes, ébranlée dans ses fondations, durement secouée.

\textsc{Lui} : Une tragédie, oui, on peut dire ça.

\textsc{Moi} : Digne des riches heures de l’Antiquité, les champs de bataille dévastés, les ruines jonchées d’ossements, les rats courant ici et là, à la recherche d’un lambeau de chair à grignoter.

\textsc{Lui} : Oui, abominable. Je ne sais pas comment elle s’y est prise, quel stratagème elle a utilisé, mais toujours est-il qu’elle a réussi à la ressusciter.

\textsc{Moi} : Un vrai miracle !

\textsc{Lui} : Et ça a duré des heures et des heures, tel un torrent de sexe qui a déferlé sur moi sans que je puisse rien faire pour échapper à son emprise ! Aucun homme ne devrait avoir à endurer pareille torture.

\textsc{Moi} : Toutes mes condoléances, vieux. Malheureusement, tu sais comme moi qu’on a du pain sur la planche. Titus a disparu, Bérénice m’est tombée dessus aux premières lueurs du jour, alors que j’ai moi-même passé une très mauvaise nuit et que si ça ne tenait qu’à moi je retournerais me coucher jusqu’à après-demain soir, et dans un moment d’aberration, de pure folie, j’ai juré de tout mettre en œuvre pour le ramener vivant à la maison.

\textsc{Lui} : Tu ne crois pas que tu en fais un peu trop ?

Moi, tirant nerveusement sur mon cigare : Je crois pas, non.

\textsc{Lui} : Et puis c’est nouveau, ça ?

\textsc{Moi} : Quoi ?

\textsc{Lui} : Tu fumes le matin, maintenant ?

\textsc{Moi} : C’est dire dans quel état de nerfs je suis ! Tendu comme un string, mon vieux, prêt à exploser à la moindre pression sur la ficelle. Je comptais sur toi pour me remonter le moral, j’ai bien peur de m’être trompé.

\textsc{Lui} : Désolé, mais j’ai les couilles en purée, comme si un rouleau-compresseur avait passé la nuit à passer et repasser dessus. Je ne sais même pas comment je fais pour tenir encore debout. J’ai eu droit à tout : le flipper, le papillon, la balançoire, le soixante-neuf, le cavalier pendu, la liane ensorcelée, le triangle lumineux, l’étoile mystérieuse, le papillon…

\textsc{Moi} : Tu l’as déjà dit.

\textsc{Lui} : Ah bon ?

\textsc{Moi} : Oui.

\textsc{Lui} : J’ai eu droit à plusieurs espèces de papillons, en fait. Sans oublier les ciseaux, le cadenas, la brouette enchantée, la bête à deux dos, l’aurore boréale, le cheval d’Hector, la cravate de notaire, le bateau ivre, j’en passe et des meilleurs. Tout, je te dis !

\textsc{Moi} : La mouche à merde, aussi ?

\textsc{Lui} : Non, pas la mouche à merde.

\textsc{Moi} : Un vrai cauchemar, en tout cas.

\textsc{Lui} : Tu l’as dit ! Il faut une condition physique de sportif de haut niveau pour tenir le coup.

\textsc{Moi} : Et c’est loin d’être ton cas.

\textsc{Lui} : J’ai passé l’âge.

\textsc{Moi} : Pour info, c’est quoi le triangle lumineux ?

\textsc{Lui} : Sensiblement la même chose que le missionnaire, sauf que la femme est un peu plus active.

\textsc{Moi} : Tu veux mon avis ?

Lui, dans un soupir : Non.

\textsc{Moi} : Tant pis, je te le donne quand même : tu ferais bien de te débarrasser de cette pute avant qu’il soit trop tard. À ce rythme-là, elle aura ta peau dans pas longtemps.

\textsc{Lui} : Je sais.

\textsc{Moi} : C’est du suicide, reconnais-le.

\textsc{Lui} : Je le reconnais, mais c’est pas une raison pour la traiter de pute. Elle est malade, tu comprends ? Malade !

\textsc{Moi} : Je suis sûr qu’elle a tué plein de mecs en les obligeant à baiser comme des dingues jusqu’au bout de la nuit.

\textsc{Lui} : Elle te fais des trucs que t’as même pas idée ! C’est une sorte de génie du sexe, dont la créativité semble inépuisable. Je pense qu’elle a signé un pacte avec le diable !

\textsc{Moi} : C’est une tueuse en série, mon pote, un vampire qui vide les hommes de leur substance jusqu’à la dernière goutte. Tu ne t’en rends pas compte, mais tu viens de prendre un aller simple pour l’enfer. T’en as marre de la vie, ou quoi ?

\textsc{Lui} : Je me pose parfois la question.

\textsc{Moi} : Fous-la dehors et reprends le cours normal de ton existence, ça vaudra mieux. Sauve ce qui peut encore être sauvé. Je suis là, moi. Si tu as besoin de quelque chose, n’hésite pas à me demander.

\textsc{Lui} : Je sais, vieux, je sais.

\textsc{Moi} : Cette femme est un démon, un succube de la pire espèce, une sorcière qui erre nue dans la forêt et s’accouple avec les bêtes sauvages.

\textsc{Lui} : Faut peut-être pas exagérer non plus. C’est une grosse chaudasse, il n’y a aucun doute là-dessus, mais ça reste un être humain fait de chair et de sang, avec un cœur qui bat sous sa poitrine avantageuse.

\textsc{Moi} : Et une chatte qui miaule en permanence, toujours affamée, jamais repue, les crocs affûtés, prêts à se planter dans le moindre morceau de viande qui passe à leur portée. Je me fais du souci pour toi, voilà tout. Je n’ai pas envie qu’on retrouve ton cadavre méconnaissable au fond de l’océan ou dans le lit d’une rivière à sec, à moitié dévoré par les sangsues.

\textsc{Lui} : Honnêtement, je ne vois pas très bien ce que mon cadavre ferait dans le lit d’une rivière à sec.

Moi, appuyant délicatement (je rappelle qu’avec le troupeau d’étalons survitaminés qui piaffaient sous le capot, tout excès d’enthousiasme se traduisait aussitôt par la sensation de décoller aux commandes d’un avion de chasse, ce qui, à moins d’être titulaire d’un brevet de pilote en bonne et due forme, pouvait avoir des conséquences désastreuses pour l’environnement et la sécurité des passagers) sur le champignon après avoir enclenché la première et desserré le frein à main : Moi non plus, mais on ne sait jamais. Je tenais à te mettre en garde, te rappeler une dernière fois que si l’activité physique est bonne pour la santé, largement recommandée par la sphère médicale dans sa plus grande majorité, il n’en reste pas moins que certaines choses sont réservées à des professionnels aguerris et ne devraient en aucun cas être reproduites sans assistance par des amateurs présomptueux. C’est tout ce que j’avais à dire. Maintenant, en voiture Simone et en route pour de nouvelles aventures !

Lui, manifestement vexé : Merci pour ta sollicitude. Mais sache, pour ta gouverne, que je ne manque jamais de m’échauffer avant de passer à l’acte.

Moi, évitant de justesse une joggeuse moulée dans une tenue rose fluo de nature à distraire dangereusement l’attention de tout conducteur un tant soit peu sensible aux attraits de la plastique féminine (et amateur de belles choses, vins fins, rivières de diamants, tonnes d’or, voitures de sport, croisières sur le Nil, couchers de soleil sur le Bosphore, plaisirs simples d’une bonne bouillabaisse partagée entre amis, gambas, naturisme, poules de luxe, bel canto et œuvres d’art en général) : C’est tout à ton honneur. Il n’empêche, et je me permets de te le dire en toute gentillesse, que tu arrives tous les matins sur ton lieu de travail dans un état physique déplorable. J’ajoute que ta santé mentale semble également affectée par les excès de ta vie sexuelle débridée, raison pour laquelle je te demande de lever le pied dans les plus brefs délais. Je crains, si tu ne suis pas mon conseil, que l’érotomanie te guette.

\textsc{Lui} : Et moi, je me demande comment tu peux savoir dans quel état je me trouve en arrivant sur mon lieu de travail puisque tu n’y es pas. Ce n’est pas parce que je suis un peu fatigué ce matin que c’est tous les jours pareil.

\textsc{Moi} : Désolé de te le dire, mais tu n’es plus le même depuis que tu es retombé entre les griffes de cette nymphe des temps modernes, ce suppôt de Lilith.

Lui, perfide : Je ne suis pas certain que tu sois en mesure de me donner des leçons sur le sujet.

Moi, écrasant la pédale de frein pour éviter de griller un feu rouge : Ah oui ? Je peux savoir ce que tu entends par là ?

Lui, avec un petit sourire méchant esquissé au coin des lèvres : J’ai bien vu ton petit manège, hier soir.

\textsc{Moi} : Pardon ?

\textsc{Lui} : Ton petit manège avec la Gardienne de la Nuit. Tu m’excuseras, mais dans le genre succube, elle se pose un peu là !

Moi, faisant ronfler le moteur pour démarrer sur les chapeaux de roues sitôt le feu passé au vert : Sois à tout jamais maudit jusqu’à la trente-sixième génération ! Que les vers te picorent, les asticots te butinent, la moisissure recouvre entièrement ta peau vérolée et les légions de l’enfer viennent danser la salsa sur le tas de terre retournée de ta sépulture anonyme ! Comment toi, mon ami, fidèle parmi les fidèles, oses-tu proférer de pareilles inepties !!!!!!! Jamais, tu m’entends, jamais je n’ai considéré Atiena comme autre chose qu’une créature magique sortie tout droit d’un recueil de contes pour enfants !

Lui, ricanant presque : Oui, enfin, ta façon de la regarder s’apparentait davantage à celle du grand méchant loup en train de reluquer le Petit Chaperon rouge et sa petite motte de beurre frais, ou encore d’Emile Louis en train de jeter un coup d’œil dans le rétro de son bus rempli de gamines handicapées de la DDASS. Tu te pourléchais clairement les babines, n’en déplaise à ta susceptibilité outragée !

Moi, écrasant la pédale d’accélérateur avec une violence telle qu’on s’est retrouvés instantanément de l’autre côté de la route : C’est honteux ! Je ne sais pas ce qui me retient de m’arrêter et te jeter hors de la voiture !

\textsc{Lui} : Ta réaction prouve que j’ai mis dans le mille.

\textsc{Moi} : T’as rien mis dans quoi que ce soit ! À part ta bite dans le cul de cette pute, peut-être !

\textsc{Lui} : Tu vois, tu deviens grossier quand t’es énervé. Et puis c’est pas une pute, je te l’ai déjà dit.

\textsc{Moi} : Je suis pas énervé, je suis offusqué !

\textsc{Lui} : Reconnais que tu la trouves à ton goût.

\textsc{Moi} : Là n’est pas la question.

\textsc{Lui} : Ben si, un peu quand même.

\textsc{Moi} : Je ne mélange pas le travail et les sentiments, moi.

\textsc{Lui} : Moi non plus.

\textsc{Moi} : Tes sentiments interfèrent avec ton travail, ça revient au même.

\textsc{Lui} : C’est pas mes sentiments, c’est ma libido.

\textsc{Moi} : Oui ben ta libido, tu vas la laisser un peu de côté et te concentrer sur le boulot.

\textsc{Lui} : Métro, boulot, libidodo.

\textsc{Moi} : Très drôle !

\textsc{Lui} : Tu ne vas pas faire la gueule pendant tout le trajet, j’espère.

\textsc{Moi} : Je fais pas la gueule, je suis fatigué d’entendre des conneries à longueur de journée. Ça commence tôt le matin et ça ne s’arrête plus jusqu’au milieu de la nuit. J’essaie de t’aider, te sortir un peu de l’ornière dans laquelle tu te trouves, et toi tu ne trouves rien de mieux à faire que d’insinuer des choses à mon sujet, comme si je n’étais pas le preux chevalier blanc épris de justice et d’équité que je prétends être.

\textsc{Lui} : D’équitation, tu veux dire.

\textsc{Moi} : Pardon ?

\textsc{Lui} : Le preux chevalier blanc épris d’équitation.

\textsc{Moi} : Et voilà, je te parle de choses sérieuses, et toi tu continues à me bassiner avec tes blagues à deux balles ! Comment veux-tu que je ne sois pas au bout du rouleau avec des gens comme toi.

\textsc{Lui} : Moi aussi, je suis au bout du rouleau.

Moi, évitant de justesse une collision après avoir refusé une priorité à droite : Pas pour les mêmes raisons que moi.

\textsc{Lui} : Ralentis, tu veux.

\textsc{Moi} : Je ne roule pas vite.

Lui, s’accrochant tant bien que mal à tout ce qui lui tombait sous la main, y compris mon bras ou ma jambe à l’occasion, renforçant au passage les risques de perte de contrôle inhérent à la conduite pour le moins sportive que j’avais choisi d’adopter : Si. Tu as déjà failli provoquer au moins une demi-douzaine d’accidents.

\textsc{Moi} : Je suis un peu tendu, excuse-moi. Et enlève ta main de ma cuisse, tu veux bien.

\textsc{Lui} : Ralentis ou je descends de la voiture.

\textsc{Moi} : Je ne vois pas très bien comment.

\textsc{Lui} : Je vais sauter en marche, au risque de me casser une jambe ou atterrir sur un piéton qui n’a rien demandé à personne et va finir à l’hôpital avec une double fracture du crâne et des lésions irréversibles au niveau de la moelle épinière.

\textsc{Moi} : Je consens à lever le pied si tu retires ce que tu as dit à propos de moi et la Gardienne de la Nuit.

\textsc{Lui} : ATTENTION !!!!!!!

\textsc{Moi} : Quoi encore ?

\textsc{Lui} : Il y a une femme enceinte qui vient de s’engager sur le trottoir !!!!!!

\textsc{Moi} : Elle n’est pas enceinte, elle est juste grosse. Enorme, même.

\textsc{Lui} : Ce n’est une raison pour l’écraser.

J’ai pilé juste à temps, sous le regard horrifié de la grosse bonne femme qui s’est arrêtée net dans sa trajectoire.

Elle s’est mise à gesticuler des bras et des jambes, surtout des bras parce que ses jambes lourdes et cylindriques ne lui permettaient guère de se livrer à des facéties musculaires et autres prouesses articulaires, tout en m’insultant copieusement et me menaçant de représailles judiciaires dignes d’un caïd de la pègre ou un tueur cannibale. Ça  a duré un certain temps, pendant lequel je me suis prudemment abstenu de toute déclaration intempestive ou objection inappropriée, après quoi elle a paru soulagée et s’est enfin décidée à reprendre sa route vers la destination qui était la sienne (qui n’était pas celle du cimetière, manifestement, en tout cas pas encore, car compte tenu de sa surcharge pondérale et des difficultés circulatoires afférentes, il y avait gros parier qu’elle ne tarderait pas à s’y retrouver).

C’est le moment choisi par Greg pour détacher sa ceinture et ouvrir la portière dans l’intention manifeste de s’extraire du véhicule.

\textsc{Moi} : Qu’est-ce que tu fais ?

\textsc{Lui} : Je te l’ai dit, je sors de la voiture.

\textsc{Moi} : C’est bon, je vais ralentir.

Lui, qui n’avait aucune envie de parcourir in pede (traduction latine de «à pied», laquelle ne fait aucunement référence à quelque orientation sexuelle réelle ou supposée concernant l’individu en question, lequel, par le plus grand des hasards, se trouvait également figurer en bonne place sur la liste - certes succinte mais tout de même pas totalement insignifiante - de mes meilleurs amis) les deux ou trois bons kilomètres qui nous séparaient encore de la rue des Maléfices, en plein cœur d’un quartier sensible où il ne faisait pas nécessairement bon se balader les mains dans les poches en sifflotant Dixie de Dan Emmett : Tu promets ?

\textsc{Moi} : Je promets. Mais toi, tu promets de ne plus faire d’insinuations gratuites à mon sujet.

\textsc{Lui} : J’ai bien vu comment tu la regardais.

\textsc{Moi} : Tu recommences ?

Lui, une fesse dans la voiture : Non non, j’arrête. Mais reconnais que tu n’aurais rien contre lui faire un brin de causette au bord de l’eau.

\textsc{Moi} : Au bord de l’eau ?

Lui, deux fesses dans la voiture, refermant la portière : Ou ailleurs.

\textsc{Moi} : Non. Au bord de l’eau, c’est bien.

\textsc{Lui} : De l’eau de mer, par exemple.

\textsc{Moi} : Sur la plage, quoi.

\textsc{Lui} : Au coucher du soleil.

\textsc{Moi} : Je ne suis pas fan des clichés, genre compter fleurette à une fille sublime sur la plage au coucher du soleil alors que le temps est d’une douceur élégiaque et que la fille porte une robe blanche translucide avec rien en-dessous à part son corps de rêve tapissé d’un épiderme aussi fin et soyeux que la peau d’une pêche de vigne gorgée de nectar sucré. C’est un peu de la merde, tout ça, de la carte postale pour blaireau élevé à la malbouffe et la musique d’ascenseur.

Lui, attachant sa ceinture : Un peu, oui. Mais pas totalement.

\textsc{Moi} : Tu me fatigues, tu sais.

\textsc{Lui} : Je sais.

Je suis reparti, mais comme j’avais oublié de faire deux choses que tout conducteur se doit absolument de faire de faire quand il redémarre après s’être arrêté sur le bas-côté de façon plus ou moins intempestive, à savoir mettre son clignotant et regarder son rétro pour s’assurer que personne n’arrive en trombe au même moment, j’ai évité de justesse une Mini Cooper qui précisément n’avait rien trouvé de mieux à faire que d’arriver en trombe à ce moment-là. Sinon en trombe, en tout cas à une vitesse (pour autant que je puisse en juger, et sans me vanter je suis assez fort pour déterminer avec une marge d’erreur quasi insignifiante la vitesse des véhicules qui se déplacent autour de moi) largement supérieure aux trente kilomètres-heure autorisés. Cela dit, en cas de choc, même si la conductrice en question (il s’agissait d’une femme, je le précise sans la moindre arrière-pensée sexiste concernant la prétendue dangerosité des femmes au volant, lesquelles, si l’on en croit certains individus dont la misogynie avérée et les blagues graveleuses ne plaident guère en faveur de l’intelligence, seraient moins occupées à conduire qu’à bavasser au téléphone ou se repoudrer le nez dans le rétro de courtoisie, quand elles ne seraient pas en train de fouiller dans la boîte à gants ou vagabonder dans leurs pensées au point de perdre tout contact avec la réalité) n’était pas totalement en règle avec la limitation de vitesse en vigueur dans le quartier, j’aurais été tenu pour seul responsable du sinistre, ayant déboîté sans mettre mon clignotant ni m’assurer que la voie était libre. J’aurais bien sûr prétendu le contraire, comme toute ordure qui se respecte, avec un aplomb sans faille, une morgue abjecte digne du politicien le plus corrompu, et Greg se serait fait un devoir d’abonder dans mon sens, en toute mauvaise foi, répugnant de duplicité servile, mais il m’aurait fallu un certain temps (au moins deux ou trois heures) avant de pouvoir à nouveau contempler mon doux visage dans la glace sans être aussitôt pris d’une violente envie de gerber. Comme vous n’êtes pas sans le savoir, du moins je l’espère, l’être humain est ainsi fait qu’il s’accommode assez facilement des bassesses ordinaires de sa moralité douteuse, trouvant sans cesse des arrangements avec sa conscience, laquelle semble ne lui avoir été donnée que pour servir de caution à ses exactions (en plus, tout de même, d’être un rempart naturel à la frénésie destructrice qui l’agite habituellement, à mon sens largement supérieure aux capacités créatrices qui lui ont été dévolues, sans cesse dévoyées par la purulence égotique qui préside à l’essentiel de ses activités).

Surprise, et heureuse de s’en tirer à bon compte, la conductrice s’est éloignée en vociférant tant et plus, jetant des coup d’œil furieux dans son rétroviseur et agitant les bras de façon désordonnée, au risque de perdre le contrôle de son moyen de locomotion, griller le stop qui l’attendait au bout de la rue et emplafonner le bus qui arrivait à vive (très, sans doute même trop, le chauffeur, en plein divorce et affligé de problèmes de santé aussi divers que douloureux, affichant une nervosité assez préjudiciable à la souplesse de sa conduite) allure sur sa droite.

Greg, qui, sans doute par manque de vigilance au moment des faits, ne semblait pas avoir pris la pleine mesure de la catastrophe que nous venions de frôler : Tu crois qu’il lui est arrivé quoi, à Titus ?

\textsc{Moi} : Je pense que la Gardienne de la Nuit l’a entraîné dans son repaire pour l’initier à des activités sexuelles dont nous n’avons même pas idée.

\textsc{Lui} : Dans ce cas, on ferait mieux de le laisser où il est.

\textsc{Moi} : L’ennui, c’est qu’on ne sait pas où il est.

\textsc{Lui} : Au Caribbean Hôtel, non ?

\textsc{Moi} : J’ai tout lieu de le penser, en effet. Je pense qu’elle en a fait son jouet sexuel et le séquestre dans une chambre sans numéro située quelque part dans les bas-fonds de l’établissement, une sorte de nid d’amour, d’alcôve secrète réservée aux proies de cette harpie en jupon.

\textsc{Lui} : Oui, enfin, je crois surtout qu’elle lui a tapé dans l’œil et qu’il a complètement oublié qu’il avait une femme, des gosses et un boulot. Il va réapparaître en fin de matinée, le bec enfariné, et prétendre qu’il ne se souvient de rien.

\textsc{Moi} : Peut-être qu’elle lui a fait boire un philtre d’amour à base  de belladone, jusquiame noire, miel vert et vin de Pramnos, telle Circé à Ulysse.

\textsc{Lui} : On appelle ça du GHB de nos jours. C’est moins glamour mais tout aussi efficace.

\textsc{Moi} : Je préfère miel vert et vin de Pramnos. Toujours est-il que Bérénice va être folle de rage si elle apprend qu’il a passé la nuit avec une femme. Elle saura aussi qu’on a essayé d’étouffer l’affaire, et notre belle amitié finira dans la benne à ordures.

\textsc{Lui} : Si on lui sort que Titus a été l’innocente victime d’une créature de rêve dotée de pouvoirs surnaturels, elle nous clouera au pilori sans l’ombre d’une hésitation.

\textsc{Moi} : Oui, à grands coups de marteau. Après quoi elle nous crèvera les yeux et nous arrachera les entrailles pour les donner à bouffer aux corbeaux !

\textsc{Lui} : Elle retrouvera la fille, l’arrosera d’essence et se fera un plaisir de la réduire en cendres.

\textsc{Moi} : Comme les sorcières au Moyen Âge.

\textsc{Lui} : Quant à Titus, elle lui coupera les couilles avec des ciseaux rouillés et l’obligera à les bouffer pour qu’il s’étouffe avec.

\textsc{Moi} : Une mort atroce, et on ne pourra rien faire pour l’empêcher.

\textsc{Lui} : Rien.

Moi, appuyant sur le champignon : C’est pour ça qu’il faut se dépêcher si on veut éviter le pire.

\textsc{Greg} : Je te signale que tu viens de griller un feu rouge.

\textsc{Moi} : Non, je suis passé à l’orange.

\textsc{Greg} : Un orange très très mûr, alors.

\textsc{Moi} : Tu veux sauver Titus, oui ou merde ?

\textsc{Lui} : Bien sûr que je veux sauver Titus. Je donnerais tout pour sauver Titus. Tout sauf ma vie, parce que même si j’adore Titus, j’ai quand même une certaine affection pour ma propre existence. Je dirais même une certaine addiction, contractée au fil des années passées à rouler ma bosse sur cette terre. C’est sans doute de la faiblesse de ma part, mais si quelqu’un devait mourir, j’aimerais autant que ce soit lui. J’aurais du mal à m’en remettre, bien sûr, et son souvenir resterait gravé dans ma mémoire jusqu’à la fin de mes jours, mais je pense que j’aurais assez force de caractère pour arriver à vivre avec.

\textsc{Moi} : Tu devrais avoir honte de dire des choses pareilles !

\textsc{Lui} : J’ai honte, mais j’ai assez de force de caractère pour arriver à vivre avec.

\textsc{Moi} : Et moi ?

Lui, Quoi, toi ?

\textsc{Moi} : Si ma vie était en jeu ?

\textsc{Lui} : Tu veux savoir si je te laisserais crever comme un chien ?

\textsc{Moi} : Je ne me fais aucune illusion.

\textsc{Lui} : Toi c’est pas pareil, t’es comme un frère pour moi. Je ne dis pas que j’irais jusqu’à donner ma vie pour toi, mais je serais prêt à sacrifier quelques morceaux de mon anatomie.

\textsc{Moi} : Non ?

Lui, posant une main sur ma cuisse : Si.

\textsc{Moi} : Okay, je te remercie. Tu peux enlever ta main, s’il te plaît ?

\textsc{Lui} : Ah oui, pardon.

\textsc{Moi} : On arrive bientôt. Je vais tâcher de trouver une place stratégique pour me garer, histoire qu’on puisse se barrer en vitesse si les choses tournent mal.

Personne n’ayant eu l’idée saugrenue de le déplacer pendant la nuit, le Caribbean Hôtel se trouvait exactement au même endroit que la veille.

Vous rigolez, mais ça s’est déjà vu que des gens déplacent des choses pendant la nuit pour vous faire croire que vous perdez la boule. Dans les vieux films policiers, par exemple, il n’est pas rare qu’une blonde pulpeuse avec des yeux en velours bleu, des dents de vampire, des lèvres de sangsue, une taille de guêpe et deux obus à la place des seins, engage, pour une raison X ou Y (le plus souvent parce que la pauvre chérie est victime d’un odieux chantage qui risque de faire capoter son mariage avec son mari richissime et vieillissant, ce qui aurait pour conséquence désastreuse de la priver de toute ressource et l’obliger à retourner racoler le micheton dans les bars des hôtels de luxe), un détective privé alcoolique au charme ravageur qui ne se départit jamais de son humour caustique et sa décontraction à toute épreuve. Un beau soir, la créature arrive en courant dans le bureau de ce dernier, toujours impeccablement coiffée en dépit d’une course effrénée sous une pluie battante, pour lui annoncer qu’elle vient d’être témoin d’un meurtre horrible et qu’elle préfère venir le trouver lui plutôt que la police car elle ne tient aucunement à être mêlée officiellement à cette sombre histoire. Après quelques hésitations, durant lesquelles le privé pose avec insistance son regard acéré sur les courbes vertigineuses de la fille, il consent à enfiler son imper (faute de mieux), allumer une clope, poser son chapeau sur sa tête, et la suivre sur les lieux du drame afin de vérifier l’authenticité de ses dires. Tous deux prennent place dans sa grosse voiture cabossée, traversent la ville en sens inverse, toujours sous la pluie et sans desserrer les dents (la tension est palpable, l’atmosphère en prise directe sur le triphasé, on sent bien qu’il ne faudrait pas grand-chose pour qu’ils se jettent l’un sur l’autre et que le détective lui roule une de ces putains de pelles d’anthologie dont il a le secret sans même enlever sa clope de sa bouche et tout en gardant un œil avisé sur la route), et quand ils arrivent sur place, le cadavre a disparu sans laisser la moindre trace. Le sang a été nettoyé et les meubles et objets renversés remis en place comme si de rien n’était. Du coup la fille passe pour une affabulatrice bonne pour l’hôpital psychiatrique, se lance dans des explications à n’en plus finir, toutes aussi dépourvues de sens les unes que les autres, et finalement, histoire de ne pas s’être déplacé pour rien, le privé lui roule une pelle d’anthologie avec la clope au bec et le chapeau sur la tête, avant, grand seigneur, de la ramener discrètement chez son mari cocu (parce qu’elle le trompe, bien entendu, et plutôt deux fois qu’une, la salope, avec tout ce qui lui tombe sous la chatte, du jardinier à tablette de chocolat au chauffeur ancien boxeur en passant par le puceau de service et le comptable érotomane).

Mais ce qui arrive assez fréquemment avec des cadavres ou des objets de valeur dans des appartements haussmanniens du huitième arrondissement de Paris, des villas en bord de mer, des cabanes au fond des bois ou même de simples chambres de bonne situées au dernier étage d’immeubles insalubres, se produit nettement plus rarement avec des hôtels entiers, particuliers ou non, car même si vous disposez de moyens exceptionnels pour déplacer un tel établissement dans son entièreté, il vous faudra impérativement opérer à une heure très avancée de la nuit. Et même de nuit, si l’hôtel, comme c’est généralement le cas des hôtels, particuliers ou non, est situé dans ce qu’on appelle le centre-ville ou sa périphérie immédiate, je doute fort que vous arriviez à le déplacer par la voie des airs sans attirer l’attention de personne. Ne serait-ce que celle des forces de l’ordre, par exemple, qui effectuent des rondes régulières dans les centres-villes, ou encore celle des fêtards attardés qui, même déchirés au point d’éprouver les plus vives difficultés à mettre un pied devant l’autre, seraient les premiers surpris de voir un tel édifice passer au-dessus de leurs têtes échevelées.

J’ai fait le tour du pâté de maisons un certain nombre de fois, que je n’ai pas jugé nécessaire de compter, avant que quelqu’un se décide enfin à monter dans sa putain de bagnole, glisser la clef de contact dans la fente prévue à cet effet, démarrer le moteur et se déplacer d’un lieu à un autre pour des raisons qui lui appartenaient et ne m’intéressaient en aucune façon.

Se déplacer a toujours fait partie des préoccupations majeures de l’espèce humaine, et elle le fait maintenant de plus en plus vite pour occuper le maximum d’espace en un minimum de temps, rêvant d’ubiquité, téléportation et autres colonies interstellaires pour assouvir ses coupables penchants. La question qui se pose est la suivante : que faire quand on est une espèce éminemment invasive qui ne dispose d’aucun autre prédateur qu’elle-même ? L’autorégulation est-elle une option viable ? Peut-on légitimement compter sur la tempérance d’une espèce dont la voracité légendaire ne laisse aucune place à l’expectative ou l’introspection, sinon pour une poignée de marginaux au crâne rasé qui vivent reclus dans des ruines vieilles de plusieurs siècles (pour vivre heureux vivons caché, c’est bien cul nu) ? Arrêtons de manger de la viande, les animaux sont nos amis pour la vie et en plus c’est mauvais pour la santé, s’époumone la jeunesse 2.0 qui se teint les cheveux en rose fluo, flotte dans des fringues XXL et revendique le droit de pouvoir jouer au foot quand on est une fille et à la poupée quand on est un garçon, ainsi que la liberté de choisir tel ou tel sexe si l’on estime, notamment à l’adolescence quand on est en plein quête de soi et recherche de la vérité existentielle, que le sien n’est pas conforme. On se demande bien au nom de quoi une fille ne pourrait pas venir au monde avec des attributs masculins, et inversement un garçon avec ce qui fait habituellement les charmes de la féminité. La nature fait des erreurs, c’est à nous qu’il revient de les corriger. Elle en fait même un peu trop, raison pour laquelle il serait peut-être temps, alors que la cinquième génération de téléphone sans fil vient de voir le jour et que nous n’avons aucunement l’intention de nous arrêter en si bon chemin, de songer à la sortir du jeu une bonne fois pour toutes. Après des centaines de millions d’années de règne sans partage, l’heure de la retraite a sonné. Je bande donc je suis, peut-être, mais pas forcément un garçon. Et oui, c’est vrai, j’ai une jolie petite paire de loches bien rebondies qui prend le frais dans mon Lise Charmel à balconnet, mais ce n’est pas pour autant que je suis une fille. Merde, il est temps de mettre un terme à cette vision étriquée du genre humain ! La nature est vieille, dépassée, ses injonctions n’ont plus aucun sens dans le contexte actuel, elle a perdu le contact avec les nouvelles générations, l’élève a dépassé le maître, Dieu est mort, vive Dieu ! La tyrannie des sexes, qui veut qu’on soit maçon ou garagiste quand on est un garçon et danseuse nue ou secrétaire de direction quand on est une fille, a fait son temps. Quant à cette pratique d’un autre âge que j’évoquais précédemment, qui consiste à élever des animaux dans des conditions déplorables et les assassiner pour se repaître de leur chair en toute impunité, elle encourage la bestialité qui est en nous et nous rabaisse au rang des plus vils représentants d’un passé à jamais révolu. Devenons de paisibles herbivores, abandonnons tout esprit de compétitivité, toute trace d’ego malsain, d’individualisme forcené, éclairons-nous à la bougie et arrêtons de boire du sang comme des vampires poussiéreux. Nous sommes des milliards sur terre, tous animés des meilleures intentions, et je ne doute pas qu’avec un peu de bonne volonté il soit possible de faire en sorte que tout se passe pour le mieux dans le meilleur des mondes. Nous venons en paix, amis Terriens, et apportons dans nos valises les nouvelles technologies de la félicité éternelle. Il y a trop, bien trop longtemps que vous vous entretuez bêtement (pour des raisons d’une telle frivolité que nous peinons encore à comprendre la nature réelle de vos motivations), il vous faut maintenant apprendre à vivre en bonne intelligence. Enculez-vous les uns les autres, a dit en substance notre Sauveur bien-aimé avant de finir cloué sur une croix entre deux malfrats et de repartir au Ciel la queue entre les jambes, et ne faites pas aux truies ce que vous n’aimeriez point qu’on vous fasse, bande de putois malodorants ! Vous n’aimeriez pas qu’on vous mange, n’est-ce pas, même si certains d’entre vous l’ont fait pendant un certain temps avant de prendre conscience que cette pratique n’était pas sans doute la mieux adaptée à l’établissement d’une paix durable entre les peuples. Alors laissez ces pauvres bêtes tranquilles, laissez-les se gaver de glands, gambader joyeusement dans la luzerne, et chassez de votre tête cette obsession de vouloir à tout prix les transformer en jambon, saucisson, pâté de tête et andouillette.

Mais je m’égare, une fois de plus (et je ne vous cache pas que parler de jambon, saucisson, pâté de tête et andouillette, m’a donné une solide envie de me restaurer sans plus attendre).

Pour en revenir à ce que je disais précédemment, avant de m’embarquer dans cette diatribe futuriste qui, je l’imagine, risque de faire grincer quelques dents aussi bien dans les clapets réactionnaires de l’extrême-droite décomplexée que la bouche en cœur des bobos de la Rive Droite, un de nos plus gros problèmes sur terre est qu’il y a tellement de voitures partout qu’il n’y a plus moyen de se garer nulle part. S’il y a encore moyen de trouver une place à la cambrousse, entre deux troupeaux de vaches garés en double file, la chose est devenue quasiment impossible dans les centres-villes saturés du 21e siècle.

C’est ainsi que nous étions des dizaines, que dis-je des dizaines, des centaines à tourner autour du Caribbean Hôtel, dont certains qui tournaient jour et nuit depuis des semaines et avaient fini par développer un tel degré d’exaspération qu’il aurait suffi de la plus minuscule étincelle pour mettre le feu aux poudres et déclencher une guerre civile comme on n’en avait pas connu depuis la Fronde et le régime de Vichy. Aussi opportuniste que la hyène qui profite d’un instant d’inattention du guépard pour lui subtiliser la gazelle qu’il vient de passer des heures à traquer dans la brousse, j’ai profité de ce qu’une personne handicapée (il s’agissait en fait d’un individu parvenu à un tel degré d’obésité que ses jambes disparaissaient presque entièrement sous une cascade de plis graisseux) remonte péniblement dans son SUV pour me substituer à elle en toute illégalité. Particulièrement bien équipé, je disposais en effet, en plus d’une affichette sur laquelle était inscrit en lettres majuscules «INTERVENTION POLICE» que je plaçais bien en évidence sur le tableau de bord, d’un macaron HANDICAPÉ qui me permettait d’outrepasser régulièrement mes droits, y compris, je l’avoue humblement, quand je n’étais pas spécialement en intervention. Après tout, le maintien de l’ordre est une chose essentielle si on veut espérer que la société ait une chance de survivre aux dissensions internes et autres dérèglements intestinaux responsables des flatulences qu’elle émet en permanence. J’ajoute que les handicapés, physiques et mentaux, du reste, avec tout le respect que j’ai pour eux (j’ai moi-même d’innombrables amis handicapés avec lesquels je passe d’excellents moments, à tel point que c’est tout juste si je vois la différence avec mes amis valides, pour la plupart à peine plus intéressants, même si c’est tout de même plus facile de s’adonner aux joies de l’escalade ou faire un footing en forêt de Rambouillet avec une personne valide qu’un cul-de-jatte), doivent prendre conscience que leurs problèmes personnels, aussi cruels et injustes soient-ils, je suis le premier à le reconnaître, ne doivent pas pour autant entraver l’action du bras séculier de la justice. Je suis tout à fait pour qu’on leur réserve des places ici et là, mette tout en œuvre pour leur faciliter la tâche au maximum, ne les priver d’aucunes des réjouissances auxquelles ont droit les gens normaux, mais je ne tiens pas à les avoir dans les pattes à tout bout de champ. Imaginez, par exemple, que vous êtes en train de courser un  pickpocket ou un vendeur à la sauvette dans les rues de la cité, chose qui nous arrive malheureusement plus souvent qu’à notre tour à nous autres gens d’armes et de police, et qu’un handicapé vous barre la route en se traînant comme une limace en plein milieu du trottoir. Vous faites quoi ? Vous l’évitez soigneusement, au risque de laisser filer le contrevenant, ou le traitez sans discrimination, comme n’importe quel citoyen, autrement dit lui foncez dessus et le percutez violemment sans vous soucier un seul instant des conséquences ? Il y a des moments, dans l’existence, où il faut savoir faire des choix qui ne sont pas toujours agréables et faciles à assumer, et si un jour on doit me couper les deux bras ou les deux jambes (ou les deux, enfin les quatre, plus la bite et tout ce qui dépasse), eh bien j’essaierai de me faire aussi discret que possible pour ne pas emmerder le monde, tel Raymond Burr dans L’Homme de fer (pour le cas où je devrais continuer à enquêter cloué dans un fauteuil roulant, au risque que tous les criminels se foutent de ma gueule et rêvent de me pousser dans les escaliers).

Bon, blague à part (parce que je blaguais, bien sûr, vous n’avez tout de même pas cru un seul instant que j’étais à ce point dépourvu de sens civique que je n’hésitais pas à me garer sur les places réservées aux handicapés, des gens qui n’ont pas eu de chance dans la vie et méritent bien un minimum de compassion de la part de celles et ceux qui jouissent de la totalité de leurs facultés), je n’ai pas eu à user ou abuser de mes prérogatives pour réussir à me garer. Car en effet, non loin de l’entrée principale du Caribbean, un espace entre deux citadines d’entrée de gamme attendait qu’on vienne l’occuper, chose que je me suis empressé de faire séance tenante, exécutant avec maestria un créneau à montrer en boucle dans toutes les écoles de conduite. Ce que je veux dire par là, c’est que s’il existait un Nobel du créneau, récompense que malheureusement aucune autorité compétente n’a encore songé sérieusement à attribuer, les gens étant tous des abrutis qui ne voient pas plus loin que le bout de leur nez, j’aurais pu empocher les onze millions de couronnes suédoises sans la moindre difficulté, soit environ un million d’euros qui m’auraient permis de mettre un peu de beurre dans les épinards desséchés de mon ordinaire. Mais s’il arrivait, par extraordinaire, que le comité Nobel norvégien soit assez con pour accorder à Donald Trump le prix tant convoité (mais c’est avec un immense soulagement que j’apprends à l’instant même, preuve qu’il y a encore un vague semblant de justice en ce bas monde, que le prix Nobel de la Paix vient d’être décerné à Maria Corina Machado, femme politique qui tente courageusement, au péril de sa liberté et sans doute sa vie, de faire obstacle aux sordides manigances de l’ignoble Nicolas Maduro, l’actuel président du Venezuela, prêt à toutes les exactions pour se maintenir au pouvoir et continuer à s’enrichir honteusement sur le dos de ses concitoyens), s’il arrivait, disais-je, qu’une telle aberration (à peu près aussi absurde, si vous voulez mon avis, que l’idée de voir des requins tomber du ciel ou des parasites extraterrestres mal intentionnés atterrir dans une forêt de Caroline du Sud) se produise, je foncerais aussitôt à Oslo (en voiture bien sûr, au volant de ma fidèle et puissante Kangoo, et il va de soi que j’effectuerais un créneau parfait devant le numéro 51 de la rue Henrik Ibsen) afin d’exiger que le Nobel du Créneau soit créé immédiatement pour m’être remis dans la foulée.

C’est avec un ensemble parfait, au millimètre près, comme si on avait répété la scène pendant des mois sous la direction d’un des plus grands chorégraphes de tous les temps (même si je ne suis pas certain que Balanchine, Cunningham, Béjart ou Pina Bausch auraient accepté de faire la choré de deux types insignifiants et mal réveillés en train de sortir d’une Kangoo à peu près aussi glamour qu’un crapaud barbotant dans une flaque d’eau croupie), que Greg et votre serviteur, tels deux Titans nés des amours coupables d’Ouranos et Gaïa (je rappelle quand même que Gaïa est plus ou moins la mère d’Ouranos, ce qui fait de cette union l’inceste fondateur de la mythologie grecque, inceste d’où sont issus, outre les Titans, une série de monstres comme les Cyclopes et les Hécatonchires, et que c’est encore du ventre de Rhéa, fécondée par son propre frère Cronos, que sortiront les dieux de l’Olympe, somme toute une belle bande de dégénérés qui ne font pas franchement honneur à la profession), sommes sortis du véhicule sus-mentionné. Au cinéma, la scène aurait été tournée au ralenti, en slow motion, comme disent nos amis américains qui ont toujours une longueur d’avance sur tout (ou de retard, suivant l’endroit où on se trouve), et servie accompagnée d’un morceau de choix comme L’entrée des dieux au Walhalla, scène finale de L’Or du Rhin de Richard Wagner, lui-même connu pour être une assez belle ordure prête à tout pour assouvir ses plus bas instincts, à commencer coucher avec la fille de Franz Liszt, Cosima, âgée de vingt-quatre ans de moins que lui, qui est aussi, accessoirement, la femme de son meilleur ami, le pianiste, compositeur (assez peu doué il est vrai) et chef d’orchestre Hans von Bülow. Comme quoi les plus sombres histoires d’inceste et de trahison n’empêche pas le génie de s’exprimer, pour le meilleur, le pire, le pire du meilleur et le meilleur du pire.

Greg (qui s’était mis au tango depuis quelques semaines, histoire de rencontrer des femmes superbes, bien sûr, au regard de braise, à la taille de guêpe et la croupe incendiaire, mais aussi de faire un peu d’exercice pendant ses rares heures de loisir, ce qui ne serait pas du luxe car il avait pris pas mal de bide ces derniers temps), s’était offert, en plus d’une paire de chaussures de danse en cuir souple avec talons de 22 mm à absorption de chocs et semelles antidérapantes, un Bersa Thunder 380 CC, spécialement conçu pour assurer une protection discrète en toute circonstance sans pour autant renoncer à une redoutable puissance de feu. Connu aussi aussi sous le nom de 9 mm court, avec une douille de 17,3 mm (l’une des nombreuses munitions créées dans les années 1910 par le regretté John Moses Browning, un petit gars de l’Utah, membre de l’Eglise de Jésus-Christ des saints des derniers jours, qui avait un foutu sens des affaires et n’ignorait pas que l’homme est un loup pour l’homme, et que malgré tout l’amour que notre Seigneur exige qu’on lui porte il vaut quand même mieux éviter de tourner le dos à son prochain), le calibre 380 ACP vous permettra d’exploser en toute fraternité le crâne d’un individu qui aurait la mauvaise idée de s’en prendre à votre intégrité physique ou au contenu de votre portefeuille, que je vous souhaite aussi dodu et rembourré que les fesses de Kim Kardashian, Nicki Minaj, Jennifer Lopez et Kylie Jenner réunies. J’ajoute que sa poignée ergonomique en polymère texturé viendra se réfugier dans le creux de votre main tel un chaton craintif et ronronnant. Quand on sait à quel point les narcotrafiquants sont des gens qui aiment consommer local, et ont une telle foi dans la qualité de leurs produits qu’ils n’hésitent pas à les exporter aux quatre coins du monde, on comprend mieux que les membres des cartels sud-américains soient de fervents adeptes de la marque Bersa, originaire de Buenos Aires, de même que nos amis allemands ne jurent que par le Walther PPK, et italiens par le 80X Cheetah de Beretta. Dieu que les gens peuvent être chauvins !

Et chauvin, je l’étais sans doute aussi (même si peut-être pas autant que nos amis allemands et italiens, lesquels n’ont d’ailleurs pas toujours été nos amis, il faut bien le dire, voir NBPPBP, Note en Bas de Page Pas en Bas de Page), puisque c’était entre les mains expertes d’une manufacture d’armes et cycles française, sise en la bonne ville de Saint-Etienne et jadis réputée pour la robustesse et la fiabilité de ses fusils de chasse, que j’avais choisi de remettre ma vie.

\textsc{NBPPBP} : Je pense surtout à nos amis allemands qui sont allés jusqu’à s’installer chez nous sans nous demander notre avis, dans le Nord d’abord, puis l’ensemble du pays, hormis la Corse et les départements du Sud-Est réservés à Mussolini. Fils de militant socialiste révolutionnaire, puis ancien instituteur devenu dictateur sous le nom de Guide Suprême de la République Sociale Italienne (Duce en italien), ce dernier sera désavoué par le Grand Conseil (avec la bénédiction du roi qui s’irrite de son omniprésence), arrêté et emprisonné dans les Abruzzes, au Campo Imperatore. Par chance, Hitler a vent de ses emmerdements, et comme il n’est pas du genre à laisser tomber ses amis dans la débine, il appelle son vieux pote Otto Skorzeny, SS-Hauptsturmführer farouchement anticommuniste de son état, et envoie un commando des forces spéciales du tristement célèbre Sonder Lehrgang Oranienburg pour libérer Musso (ou Mumu, comme l’appellent ses rares amis ploutocrates, tous sexuellement déviants, Mumu lui-même ne cachant pas ou peu une certaine appétence pour les très jeunes filles).

De retour aux affaires, Mumu s’autoproclame Président Directeur Général en Chef de la République de Salò (ou des Salauds, suivant l’orthographe retenue), régime vaguement cryptocommuniste sous tutelle nazie qui est loin de faire l’unanimité dans le pays.

En avril 45, sous la pression des Alliés qui entendent bien récupérer les vins de Rinaldi, Conterno, Mascarello et Giacosa, la mozarella di bufala, le moliterno truffé, le jambon de Parme, la mortadelle et le guanciale, sans parler des trésors artistiques proprement dit présents dans tous les coins et recoins de cette région bénie des dieux, il décide de fuir en emportant deux mallettes dont le mystérieux contenu reste aujourd’hui encore marqué d’un point d’interrogation (peut-être de la truffe blanche d’Alba ou du Storico Ribelle de 10 ans d’âge). Arrêté par la Résistance italienne, condamné à mort par le Comité de libération nationale, il est exécuté, en même temps que sa maîtresse et âme damnée Clara Petacci, par des partisans dans une ferme des environs de Dongo, non loin du lac de Côme, endroit idyllique s’il en est mais funeste pour les amants maudits du fascisme, dont les corps sans vie finiront tristement pendus par les pieds sur la piazza Loreto de Milan (avec ceux de Nicola Bombacci, Alessandro Pavolini et Achile Starace).

La mallette n’a jamais été retrouvée, mais on se demande bien ce que le Duce trimballait avec tant d’acharnement dans son ultime cavale. La célébrité de celui ou celle qui retrouvera ce trésor, sans doute escamoté par les partisans (à moins que Musso, comprenant que tout était fini, ne l’ait enfoui quelque part avant de sombrer dans les abîmes de l’Histoire), est assurée.

Pendant des années, la Manufacture d’Armes de Saint-Etienne a été une référence dans le monde de la Mort, tant sur le plan humain qu’animal. Ses fusils de chasse, par exemple (Falcor, Robust, Simplex et Idéal, des noms qui font rêver et traduisent assez bien une certaine idée de la destruction), jouissaient d’une excellente réputation de fiabilité et solidité, et ses armes de guerre faisaient le bonheur des courageux jeunes gens qui avaient mis leur vie au service de la Nation. En échange de leurs bons et loyaux services, la Patrie reconnaissante mettait à leur disposition du matériel de qualité pour exterminer son prochain dans les meilleures conditions, avec le maximum de confort et d’efficacité. En plus de ses modèles propres, issus du génie créatif de ses ingénieurs (des artistes de la mort, virtuoses de l’homicide, disciples de Zénon d’Élée, Anaxagore et Mélissos, pour qui le tir à balle réelle était un art que l’on se devait de porter à son plus haut degré de finitude dénazifiée, l’expression la plus ontologiquement pure de la vérité au sens présocratique et phénoménologique du terme), Manufrance (nom commercial de la Manufacture d’armes et cycles de Saint-Etienne, pionnière de la vente par correspondance) restera à jamais dans les mémoires pour avoir assemblé des armes aussi légendaires que le Beretta M12, le G3 de Heckler \& Koch et le lance-roquettes antichar LRAC F1 de la  Luchaire Défense SA, société anonyme au capital de 4 millions de francs. Mais pour moi, outre le Chassepot de 1866 et le canon de 75 de 1897, véritables fleurons d’une approche plus moderne, progressiste, pour ne pas dire humaniste de la guerre, le chef-d’œuvre absolu de la Manufacture d’Armes de Saint-Etienne reste incontestablement celui que j’appelle affectueusement Manu, le petit Manu, autrement dit le pistolet automatique Le Français dans sa version de poche, calibre 6.35, jouet que le commissaire Ottavioli gardera précieusement sur lui jusqu’à la fin des années 70 (avant de passer à des modèles plus consistants pour s’adapter à la puissance de feu croissante des nouvelles générations de malfrats). Je sais que les jeunes d’aujourd’hui pensent que toutes les choses qui se sont passées avant le jour de leur naissance sont les reliques nauséabondes d’une époque révolue, mais je tenais à leur faire savoir, quitte à passer pour un vieux con dépravé, un vieux débris obsolète, qui était vraiment Pierre Ottavioli. Figure du 36, c’était d’abord un homme d’honneur et un de ces flics à l’ancienne comme on n’en fait plus, à l’image d’un Charles Pellegrini, un Roger Marion, un Robert Broussard ou encore un Marcel Guillaume (le modèle du Maigret de Simenon). En ce temps là, somme toute pas si lointain, flics et voyous se donnaient la réplique dans une ambiance, sinon d’admiration réciproque et d’estime à proprement parler, au moins de respect mutuel, ce qui poussait chacun à se surpasser au profit d’une cause commune qui dépassait largement le cadre exigu de la simple individualité : celle du grand banditisme, grandeur qui ne définissait pas la violence extrême et aveugle qui s’y exerçait, mais la qualité supérieure (et parfois volontiers franchouillarde, un tantinet nationaliste, je vous le concède, au sens pur porc du terme, les expressions «à l’ancienne» et «qualité supérieure» désignant aussi bien des produits du terroir tels que le saucisson sec ou la blanquette de veau) de ses intervenants.

Voilà comment Greg Lussier, Bersa Thunder, le petit Manu et moi-même nous sommes retrouvés devant la porte du Caribbean Hôtel, une entrée majestueuse trônant au centre d’une de ces majestueuses façades à colonnades qu’un Louis Le Vau, un Robert de Cotte, un Charles Le Brun, un François Mansart, un Claude Perrault ou encore un Jacques Gondouin de Folleville, pourquoi pas (le «bon Gondouin», comme l’appelait Louis XV, d’abord jardinier du château de Choisy avant de se lancer dans l’architecture sous l’égide du Roi qui semblait le tenir en haute estime), auraient été fiers d’inscrire au catalogue de leurs réalisations les plus significatives, même s’ils n’avaient encore qu’une très vague idée de ce que serait un jour le style colonial dans toute sa rigidité phallique et son paternalisme débonnaire. Force, hélas, et croyez bien que je suis le premier à le déplorer, était de constater que l’édifice avait perdu une bonne partie de sa superbe. La façade, sans un ravalement d’urgence, risquait de s’effondrer à tout moment, ensevelissant au passage d’innocentes victimes dont le seul tort aurait été de se trouver là au mauvais moment. S’ensuivrait alors, avec une implacable dynamique, le cours normal des choses, en partant du postulat que la catastrophe se produirait de nuit plutôt que de jour, l’obscurité étant un excellent adjuvant de l’angoisse, la terreur et la dramaturgie : cacophonie des sirènes hurlant à tout va, féérie lumineuse des gyrophares multicolores, présence massive des forces de l’ordre pour sécuriser le périmètre et permettre aux personnels de santé de s’acquitter au mieux de leur mission,  arrivée tonitruante des vautours surexcités de l’info en continu, forêt de micros tendus aux rares témoins oculaires et survivants de ce qui pourrait bien rester dans les anus comme une des pires tragédies de l’histoire de l’humanité (après le tremblement de terre de Shaanxi en 1556, l’explosion de La Valette en 1634, le naufrage du Scipion dans la baie de Samana en 1782, la catastrophe du Victoria Hall en 1883, le vol 123 de la Japan Airlines en 1985, l’effondrement du toit de la patinoire de Bad Reichenhall en 2006 et la terrible bousculade d’Antananarivo avant un concert de Paul Bert Rahasimanana - alias Rossy - en 2019), corps sans vie évacués sur des civières, membres épars récupérés ici et là et aussitôt congelés dans le vain espoir d’être un  jour restitués à leurs propriétaires, badauds en pantoufles et robes de chambre prêts à vendre père et mère pour apercevoir ne serait-ce qu’une goutte de sang ou un morceau de cervelle sur la chaussée, téléphones portables en surchauffe et vidéos de l’événement faisant le tour de la planète en une fraction de seconde. Tristesse envahissante du monde, nullité cosmique et décomplexée de l’espèce humaine en voie de décomposition, sidération intersidérale, folle envie d’appeler Dieu en PCV (laid moins de vain temps ne pleuvent pas qu’au naître, homme aime titre qu’un nombre inca le cul glabre de choses dont ils mourirons singe ah mais avoir an tendu pas relais, l’orthographe, par exemple, sachant que nous aussi casserons nos pipes sans jamais avoir entendu parler d’un nombre d’autant plus incalculable de choses qu’elles ne cesseront de s’accumuler pendant les siècles et les siècles qui suivront notre mort, siècles dont j’ai malheureusement toutes les raisons de penser qu’ils ne seront peut-être pas aussi nombreux que prévu à suivre le cortège du temps) pour le questionner une nouvelle fois sur la nature exacte de ses motivations, tenter une dernière fois de comprendre par quelle aberration il s’est mis en tête de créer, à son image paraît-il (Genèse 1:26-28 LSG), ce qui n’est soit dit en passant pas très flatteur pour lui, une espèce aussi débile et dérisoire que la nôtre. Si le but était de nous faire passer par toutes les étapes de la médiocrité pour arriver enfin à quelque chose de présentable, alors on peut dire que nous n’en sommes encore qu’au tout début de notre évolution. Par contre, si le but était d’expier à travers nous quelque faute originelle qu’il aurait lui-même commise, alors on peut dire que l’objectif est entièrement atteint et qu’il serait peut-être temps de songer à mettre un terme à nos souffrances, chose que nous sommes par ailleurs tout à fait en mesure de réaliser par nos propres moyens. J’ai toujours dit, et je le maintiens avec la plus extrême vigueur, que tout ce que nous faisons et accomplissons avec tant de fierté n’est finalement qu’une maigre resucée à visage humain des œuvres de la nature, dont nous ne faisons que reproduire les faits et gestes en les adaptant à nos besoins, lesquels sont d’autant plus importants que s’impose chaque jour davantage l’évidence de notre inaptitude à vivre en harmonie avec le monde. Chacune de nos actions est soumise à l’utilisation d’une prothèse correspondante, une voiture pour rouler, un bateau pour naviguer, une fusée pour aller dans l’espace, des couverts pour manger, des verres pour boire, des armes pour tuer, des écrans pour voir ce qui se passe autour de nous, des chambres pour dormir, des salles de bain pour se laver, des sex-toys pour forniquer (même si c’est haram de s’en servir et si le guide suprême iranien Ali Khamenei s’est fendu d’une fatwa à leur encontre), des tables pour poser des trucs dessus, des chaises pour s’assoir, des chaussures pour marcher, des claviers pour écrire, des mots pour le dire, etc, etc, etc. Nous sommes tous des infirmes de naissance qu’on équipe de prothèses sans cesse plus sophistiquées pour les transformer en sportifs de haut niveau. Les performances sont remarquables, si on veut, mais le prix à payer bien trop élevé pour la majeure partie d’entre nous. En nous dotant des moyens intellectuels nécessaires pour surpasser notre condition, la nature s’est tiré une balle dans le pied. Sa légitimité a pris du plomb dans l’aile, elle a vu ses prérogatives contestées et son champ d’action se transformer lentement en peau de chagrin. Elle a pris conscience que si l’Homme en avait un jour les moyens, il n’hésiterait pas à la détruire, comme il n’hésite pas à détruire tout ce qui fait obstacle à ses ambitions démesurées. Lui, dans le même temps, s’est rendu compte que la nature ne lui était plus d’aucune utilité. Non seulement elle n’avait cessé de lui mettre des bâtons dans les roues, l’obligeant à surmonter des épreuves qui menaçaient jusqu’à la survie de son espèce, mais les rares satisfactions qu’elle lui procurait en échange ne faisaient que renforcer sa servitude et exacerber sa frustration. À défaut de vivre caché, sur une île déserte ou reclus entre les hauts murs de quelque monastère situé au sommet d’une montagne inaccessible, il lui fallait, pour vivre heureux et donner la pleine mesure de ses capacités, construire un monde à son image, entièrement artificiel, dont il pourrait contrôler le fonctionnement jusque dans ses moindres rouages. Quant à cette belle intelligence, cette conscience exceptionnelle qui lui servait soi-disant à accomplir des miracles, elle-même devait renoncer au naturel pour se lancer à corps perdu dans les délices de l’artifice. La machine, à terme (même si elle le fait déjà dans de nombreuses situations), est vouée à remplacer l’être humain, bien trop fragile et approximatif dans tous les secteurs d’activité. Car enfin, si le rêve de l’Homme a toujours été de vaincre la mort, il est évident qu’il lui faut d’abord vaincre la vie pour y parvenir. Ce n’est que lorsqu’il aura percé les mystères de l’obsolescence programmée qu’il pourra enfin s’affranchir des limites du temps. Il pourra alors vivre le cœur léger, même si ce cœur n’est qu’une machine, et envisager l’avenir sans cette épée de Damoclès de la Mort suspendue en permanence au-dessus de la tête. Comment, je vous le demande, se consacrer sereinement à une tâche si vous savez que tout peut s’arrêter d’un instant à l’autre ? La nature, en nous condamnant à vivre dans cette incertitude, dans l’urgence d’une fin aussi certaine qu’imprévisible, a fait acte de cruauté absolue. Car même si elle constitue parfois une source de motivation, cette urgence est d’abord et avant tout un instrument de torture diabolique. C’est ainsi que nous mettons au monde des enfants, que nous sommes censés aimer plus que tout au monde, en sachant pertinemment qu’il nous faudra les abandonner à leur triste sort. Il ne faut pas s’étonner que le principe même de la reproduction, véritable rouleau-compresseur de l’existence, machine à broyer du vivant qui nourrit les enfants du sang de leurs parents, soit aujourd’hui dans le collimateur des nouvelles générations. Là encore, cette fatalité qui nous condamne à engendrer notre propre succession, si elle peut sembler flatteuse à première vue, revient en réalité à signer notre arrêt de mort et entériner le fait que nous allons passer le restant de nos jours à assister au spectacle pitoyable de notre déchéance. Et personne, croyez-le bien, ne se privera de vous faire sentir à quel point vous ne servez plus à rien, si tant est que vous ayez jamais servi à quelque chose, sinon vous plier en quatre sans jamais vous plaindre pour que votre progéniture ne manque de rien. Et si un jour votre enfant vient vous trouver et vous dit qu’il n’a pas demandé à naître, répondez-lui que vous non plus n’avez rien demandé et vous seriez volontiers passé de sa présence si vous aviez su à quoi il allait ressembler. Il se peut alors, pour se venger bêtement d’une adversité qu’il ne soupçonnait pas, que, dans un geste exagérément théâtral, le rejeton en question décide de mettre fin à ses jours. Dites-vous bien, dans ce cas-là, que vous n’êtes pas davantage responsable de sa mort que vous ne l’étiez de son existence, et que toutes ces considérations oiseuses ne seront bientôt plus de mise dans le nouveau monde merveilleusement virtuel et orgasmiquement artificiel, véritable feu d’artifice de joie de vivre cybernétique sur fond de neurosciences sexy dopées à la myéline homéostatique, que nous nous proposons de créer. Plus personne, alors que tout le monde savait pertinemment que vous n’étiez pas en mesure de les assumer, ne viendra vous reprocher d’avoir failli à vos devoirs parentaux au profit de vos ambitions personnelles. Les psychologues autoproclamés de la connerie institutionnelle vous le répètent assez comme ça, se gargarisant à l’envi d’éléments de langage auxquels eux-mêmes ne comprennent pas un traître mot : vos enfants ne sont pas les vôtres, ne vous appartiennent pas, alors personne ne viendra vous demander de les fabriquer vous-même (souvent au prix d’atroces souffrances, la nature, que son principe d’économie, sinon de radinerie, pousse à entasser le maximum d’accessoires dans un minimum d’espace, n’ayant pas jugé nécessaire de doter la femme d’un orifice digne de ce nom pour mettre son enfant au monde, ce qui signifie que donner la vie a longtemps été synonyme de perdre la sienne) et vous en sentir responsable jusqu’à la fin de vos jours. La nature, qui n’a eu de cesse de nous harceler depuis des millénaires, sera domestiquée jusqu’au moindre brin d’herbe, la moindre touffe de poil, réduite au silence le plus abyssal, et il nous sera alors possible de promener virtuellement des microbes en laisse sur les trottoirs de l’infinitésimal. Et quand on en aura marre d’être éternel, que le temps aura totalement disparu et que Dieu lui-même sera venu publiquement reconnaître qu’il n’existe pas et n’a jamais existé ailleurs que notre imagination dévoyée, il nous suffira d’exercer une légère pression sur notre nombril pour mettre un terme à notre inexistence.

Le hall du Caribbean Hôtel était tellement vaste qu’un McDonnell-Douglas C-17 Globemaster III de l’US Air Force aurait pu s’y poser sans problème si la porte d’entrée avait été ne serait-ce qu’un poil plus large. De la même façon, avec une aisance comparable, un troupeau de buffles d’Afrique au grand complet aurait pu y séjourner en toute quiétude si le carrelage et la moquette avaient été remplacés par de l’herbe, n’ayant aucunement à redouter les balles des riches chasseurs occidentaux qui rêvent d’accrocher des têtes coupées de Big Five sur les murs de leurs résidences hors de prix. Et ceci pour une raison très simple : le Caribbean Hôtel était, sinon interdit aux Blancs, au moins réservé aux gens de couleur (au sens large du terme, c’est à dire que les Na’vi, Yondu, le Dr Manhattan - à ne bien évidemment pas confondre avec le Mr Manatane de Benoît Poelvoorde - et même l’ignoble Yellow Bastard de Sin City pouvaient y être admis en montrant patte blanche, au même titre qu’un Hellboy ou encore un Géant Vert, que son épiderme verdâtre et l’abominable odeur de maïs en boîte qui se dégage de sa personne ont définitivement mis au ban de la société), ce qui revenait sensiblement au même. Ce statut particulier, nettement discriminatoire, n’était finalement pas pire que d’interdire aux pauvres l’accès à nombre de manifestations culturelles ou écoles privées sous le prétexte qu’ils n’ont pas les moyens de se les offrir (ce qui entretient incidemment l’idée que ce sont tous des crétins incultes et sans avenir, proposition inacceptable même si non totalement dépourvue de fondement). On fait semblant de s’en émouvoir, d’en dénoncer l’injustice, mais la vérité c’est que la discrimination par l’argent ne choque plus personne depuis belle lurette, à commencer par les pauvres eux-mêmes qui trouvent tout naturel de claquer des fortunes pour aller applaudir des gens qui gagnent en une fraction de seconde ce qu’il leur faut des mois de labeur intensif, ingrat et notoirement sous-payé pour acquérir.

À cette heure matinale, l’endroit était désert.

Quelques vagues grooms en costume folklorique s’agitaient ici et là, sans doute pour faire croire qu’il se passait quelque chose alors qu’il ne se passait strictement rien, je peux en témoigner sur la vie de feu ma grand-mère maternelle adorée Alexandrine Chéron, née Lemaître, décédée en juin 2022 alors qu’elle survolait la cordillière des Andes à bord du Cessna Skylane 182 jaune canari qu’elle s’était offert pour son quatre-vingt-septième anniversaire. Cette femme, une femme de tête qui avait toujours placé l’indépendance au-dessus de tout et ne s’en était jamais laissé compter par tous les beaux-parleurs qui avaient croisé sa route (exception faite de mon grand-père Philibert, dont le charme ravageur, le sens aigu de la probité et les moyens financiers assez conséquents avaient eu raison de sa résistance), cette femme, disais-je, restait pour moi l’archétype absolu de l’aventurière intrépide au physique de reine de beauté. Les photos d’elle que j’avais vu quand elle n’était encore qu’une adolescente frondeuse ou une splendide jeune femme dont le regard bleu d’acier ne laissait planer aucun doute sur le caractère farouche et la soif de liberté, m’avaient fait forte impression. Aujourd’hui encore, parvenu à un certain degré de maturité dans l’existence, il m’est difficile de regarder ces photos, de qualité très médiocre pour la plupart, sans faire aussitôt l’objet de turbulences intérieures d’une violence inexplicable. Certes, c’était ma grand-mère et je l’adorais, mais de là à me mettre à chialer comme un gosse dès que je tombe sur une photo d’elle en maillot de bain, je pense qu’il y a tout de même un pas qui ne devrait pas être franchi avec une telle allégresse. Je me souviens, quand j’étais petit, qu’elle était encore un très belle femme pour son âge. Je dirais même anormalement, comme si un philtre de jeunesse éternelle la protégeait des ravages du temps. Bon, je reconnais que celui-ci avait fini par la rattraper, car à bientôt cent ans elle ressemblait quand même davantage à une vieille momie édentée qu’à une biche au teint frais comme la rosée du matin. Cela dit, au milieu des ruines subsistaient encore quelques reliques des splendeurs du passé, aussi scintillantes que des pépites dans le lit boueux d’une rivière.

Greg a dit, la voix traversée par un vieux frisson de peur ancestrale telle que l’homme n’en avait plus connu depuis que le dernier spécimen d’ours de Deninger s’est éteint dans une grotte de la sierra d’Atapuerca, près de Burgos : Je déteste cet endroit.

Ce à quoi j’ai répondu immédiatement, sans lui laisser le temps de s’enfoncer davantage dans les profondeurs sombres et humides de l’effroi : Pas moi.

C’était vrai, du reste, je le trouvais plutôt sympathique, cet endroit. Sauf peut-être la décoration, qui laissait quelque peu à désirer. En effet, le propriétaire des lieux, sans doute frappé de démence, n’avait rien trouvé de mieux à faire que de transformer l’endroit en une espèce de vieux musée de province rempli d’objets poussiéreux tout droit sortis d’un film d’horreur des années 50. Les animaux empaillés, par exemple, faisaient un peu désordre dans un établissement de cette catégorie, revendiquant un niveau de standing à priori incompatible avec la présence d’un groupe de hyènes au milieu du salon. L’hippopotame non plus n’avait rien à faire là, pas plus que les antilopes, le léopard en train de déchiqueter un phacochère (les entrailles étaient particulièrement bien imitées), ou encore le vautour qui déployait ses ailes au-dessus de la Réception, prêt à se jeter sur le client venu réclamer sa clé.

\textsc{Greg} : On se croirait dans Psychose.

\textsc{Moi} : En plus exotique.

Je sais qu’il existe, de nos jours enténébrés, des jeunes gens qui n’ont jamais vu un film en noir et blanc, ni entendu parler d’Alfred Hitchcock et encore moins de Robert Bloch.

Grand admirateur de Lovecraft avec lequel il entretient une longue relation épistolaire, Bloch est pourtant une des figures majeures de la littérature fantastique et horrifique américaine. Passionné par les histoires de monstres en tout genre, il s’intéresse de près à une certaine catégorie de prédateurs sexuels qui n’hésitent pas à tuer pour assouvir leurs fantasmes déviants. Un certain Edward Theodore Gein, par exemple, vient de défrayer la chronique. Il semblerait que le décès de sa mère, une fanatique protestante qui détestait les hommes, ait eu un effet désastreux sur sa personnalité. Après sa mort, le fiston, alors âgé de trente-neuf ans, commence à péter très sérieusement les plombs. Les gens normaux, au moins ceux qui croient en Dieu et une vie après la mort, se rendent au cimetière pour fleurir les tombes et prier pour le salut des âmes de leurs défunts. Pas lui. Quand il a constaté, malgré ses suppliques répétées et ses incantations au clair de lune, que sa mère ne semblait pas décidée à refaire surface, aux grands maux les grands remèdes, il est revenu avec sa plus belle pelle pour la sortir de terre. Il a ramené son butin à la ferme et s’est livré sur lui a des pratiques que la morale réprouve. Et le jour où il s’est lassé de son jouet, mu par des pulsions dans le détail desquelles je préfère ne pas entrer (je m’en voudrais qu’un enfant innocent, tombé par hasard sur cet ouvrage, se retrouve traumatisé par sa lecture), il est retourné au cimetière pour s’approvisionner. Que des cadavres de femmes, bien sûr, qu’il rapportait jalousement chez lui pour se fabriquer des trophées tous plus macabres les uns que les autres. Si vous aviez des envies bizarres, comme équiper votre salon avec un canapé en cuir de femme ou vous balader dans les rues de la ville avec une veste du même matériau sur le dos, c’est Eddie qu’il fallait aller voir. Nul doute, avec tous les cinglés en liberté, qu’il aurait pu se faire pas mal de fric en vendant ses créations au lieu de les garder pour lui. Mais s’il était doué pour la couture (peut-être pas autant que Paul Poiret ou Jean Patou, mais il avait son petit savoir-faire), il n’avait aucun sens du commerce et ne tenait aucunement à ce que ses activités s’ébruitent. Certains, du fait de la nature quelque peu discutable de leurs activités, sont condamnés à la clandestinité, ce qui est un moindre mal comparé aux risques qu’ils encourent. Toujours est-il que ce qui n’était au début qu’un passe-temps bien innocent, censé lui changer les idées et l’aider à supporter les affres de la solitude, est rapidement devenu une quête obsessionnelle. Il lui en fallait toujours plus, et les ressources que la nature met généreusement à notre disposition se révèlent parfois largement insuffisantes. En clair, les gens ne mouraient pas assez vite pour suivre le rythme effréné de sa créativité. Eddie, qui n’avait jusqu’ici connu que les plaisirs solitaires en essayant d’échapper au regard accusateur des crucifix disséminés un peu partout dans la baraque, venait de découvrir avec émerveillement les joies de l’amour physique avec une vraie femme parfaitement consentante et entièrement soumise à ses désirs, qu’il pouvait profaner par tous les orifices sans que sa mère, désormais transformée en abat-jour, descente de lit et autre rideau de douche, vienne le menacer des foudres de l’enfer. Il pouvait désormais faire tout ce qu’il voulait, même si les chairs faisandées qu’il malaxait avaient parfois tendance à se déliter sous ses doigts, tout comme il n’était pas toujours très agréable de fourrer sa langue dans des cavités buccales débordant d’asticots. Pour toutes ces raisons (pénurie de matière première et besoin de chaleur humaine), Eddie s’est mis à rêver de faire l’amour à des femmes encore tièdes qu’il aurait lui même choisies, au lieu de se contenter d’articles de récupération ayant depuis longtemps dépassé leur date de péremption. Il se fait la main sur Mary Hogan, de Pine Grove, dont il garde la tête pour se rappeler des bons moments passés en sa compagnie, avant de jeter son dévolu sur Bernice Worden, une sexagénaire qu’il croise régulièrement dans les rues de Plainfield et à laquelle il n’ose déclarer sa flamme, ne disposant pas du bagage technique nécessaire pour s’exprimer clairement et avoir la moindre chance de retenir l’attention d’une femme aussi belle et raffinée. En désespoir de cause, il met un terme brutal à son existence, la ramène dans sa tanière et peut enfin, à l’abri des regards et sans craindre le ridicule, lui témoigner toute l’étendue de la passion qui le consume jour et nuit. Mais quand ils s’étonnent de ne plus la voir et apprennent l’étrange disparition de Bernice, des voisins signalent à la police avoir vu à plusieurs reprises un type bizarre rôder autour de chez elle. Ce type bizarre, c’est Ed Gein, un gars du coin qui vit seul dans une ferme pourrie des environs de la ville. C’est le fils de George et Augusta Gein, le petit dernier d’une fratrie de deux garçons. Ils sont tous morts sauf lui. Il n’est plus tout à fait le même depuis le décès de sa mère, à laquelle il vouait une adoration sans bornes. C’était une maîtresse femme qui menait son monde à la baguette, une protestante rigoriste qui ne tolérait pas les écarts de conduite. Eddie avait la réputation d’un gars à la limite du handicap mental, mais toujours bien poli et prêt à rendre service. Quand les flics ont débarqué chez lui, ils ont été saisis par une pestilence telle que le contenu de leur estomac leur est aussitôt remonté dans le fond de la gorge. Ensuite, ils ont eu droit à une visite guidée du petit musée des horreurs qu’Eddie s’était aménagé à domicile. Il ne leur a pas fallu longtemps pour comprendre qu’ils venaient de tirer le gros lot.

C’est ainsi que Robert Bloch, quand il a eu vent de l’affaire, s’est attelé à la rédaction que ce qui allait devenir son roman le plus célèbre : PSYCHOSE. Alfred Hitchcock, le petit gros qui adorait foutre la trouille aux gens et fantasmer sur les créatures de rêve, blondes pour la plupart, qu’il engageait pour tourner dans ses films, lit le livre, et, en bon maniaque sexuel féru de psychanalyse qu’il est, décide aussitôt de l’adapter à l’écran. Gein n’est plus le péquenaud attardé du Wisconsin qui a fait frissonner le pays tout entier, mais le jeune et timide Norman Bates, célibataire endurci qui tient un motel pas très réjouissant en bordure de nationale. Ténébreux à souhait, posant sur les êtres et les choses un regard d’une étrange fixité, Norman prend soin de sa vieille mère malade et empaille des oiseaux pendant son temps libre. On aurait tendance à lui donner le bon dieu sans confession, mais il a aussi des petites manies qui pourraient facilement le faire passer pour un vilain garçon. Par exemple, et cela n’a rien à voir avec une quelconque passion contrariée pour le bricolage, il adore faire des trous dans les murs pour mater les clientes en petite tenue. Il faut dire que sa libido et ses conditions de vie difficiles lui occasionnent de sérieux troubles du comportement. Sévère et très à cheval sur les principes (à défaut d’autre chose), sa mère se transforme instantanément en redoutable bras armé de la justice divine pour châtier les Messaline qui viennent agiter leurs appas sous les yeux exorbités de son petit chéri d’amour. Lui-même, conscient des agissements de la vieille femme qui n’a hélas plus toute sa tête, se doit de tout mettre en œuvre pour la protéger. Il n’est donc pas rare que des visiteurs un peu trop curieux disparaissent sans laisser de trace.

Je citerai aussi, dans un style nettement plus gore et pétaradant, le cultissime Massacre à la tronçonneuse de Tobe Hooper : une famille de cannibales vit dans la nostalgie du passé, le bon vieux temps où l’abattoir du coin faisait vivre tout le comté. Ils ont un sens de la décoration assez particulier, en rapport avec leur ancienne profession de boucher. Le papy, qui doit avoir dans les cent cinquante ans et survit au grenier dans des conditions d’hygiène déplorables, n’avait pas son pareil pour assommer un bœuf à coups de marteau. Il passe le plus clair de son temps à somnoler dans son fauteuil, mais la vue d’un bon plat de tripes ou une belle andouillette suffit à le ramener à la vie et lui redonner la pêche de ses vingt ans. Cinq potes en vacances, deux filles et deux garçons plus un troisième qui n’est autre que le frère handicapé de l’une d’entre elles, débarquent dans les environs de Round Rock, au Texas. Sur la route, ils font la connaissance de divers membres de la famille, dont le père à la station-service, un type bizarre avec une tête de rat, et un des fils débiles qu’ils font l’erreur de prendre en stop. Ce dernier, qui n’a manifestement pas toute sa tête, s’intéresse de très près à Franklin, le frère handicapé de Sally Hardesty, la copine de Jerry. Quand le débile, après avoir tenu des propos inquiétants, sort un couteau pour se tailler un steak dans le cul de Franklin, les cinq gens lui demandent de prendre congé. En guise de réponse, il s’entaille le creux de la main en rigolant et menace de tuer tout le monde. Ils finissent par réussir tant bien que mal à le faire sortir, mais le débile, qui se balade avec des animaux morts dans son sac, les maudit ouvertement et imprime l’empreinte sanglante de sa main en guise de signe cabalistique sur la carrosserie du van. Pendant ce temps, les autorités alertent l’opinion sur le fait que les profanations de sépultures se multiplient dans le secteur, accompagnées d’actes blasphématoires particulièrement scandaleux et répugnants. En effet, après avoir été extraits de leurs tombes, ce qui représente déjà une grave atteinte à leur intégrité, les cadavres sont ensuite installés comme des épouvantails dans le cimetière. Seuls des gens souffrant de graves troubles mentaux, sans doute membres d’une secte s’adonnant à des orgies sexuelles et des sacrifices de nourrissons, sont capables de telles horreurs. Les cinq jeunes gens, partis à la recherche de la maison où Sally et Franklin ont passé une partie de leur enfance, finissent par mettre la main dessus. Grande est leur déception quand ils constatent qu’il n’en reste que des ruines. Mais quand on est jeune, beau, qu’on a toute la vie devant soi, qu’on n’a pas un gramme de graisse sur le corps, qu’on a des tablettes de chocolat quand on est un garçon et un petit cul rond comme un ballon moulé dans un micro-short en jean quand on est une fille, il faut un peu plus que ce genre de contretemps pour entamer votre bonne humeur. C’est ainsi que Kirk et Pam, l’autre sympathique petit couple de notre fine équipe de randonneurs estudiantins, décident d’aller faire trempette dans le doux ruisseau clapotant aperçu en cours de route. Ce faisant, ils aperçoivent, au détour d’un sentier tortueux, une propriété isolée dont le moins qu’on puisse dire est qu’elle n’inspire pas franchement confiance. Ils l’ignorent encore, et nous aussi même si on commence à entrevoir la sinistre vérité à travers le voile trompeur de la joie de vivre et l’insouciance, mais cette bicoque délabrée n’a rien de la petite maison dans la prairie. On a tous en tête cet endroit idyllique où des gens charmants vous accueillent à bras ouverts comme si vous faisiez partie de la famille depuis toujours. Vous tombez en panne en rase campagne, frappez à la porte, une mère de famille sexuellement attirante (en anglais Mother I’d Like to Fuck, NDLR) vous ouvre la porte, les cheveux en bataille et le décolleté largement ouvert sur des perspectives vertigineuses dignes de la Vallée des plaisirs, et vous offre aussitôt une énorme part de tarte à la citrouille que vous n’avez pas intérêt à refuser si vous ne voulez pas qu’elle vous arrache les yeux avec ses ongles de trente centimètres de long taillés en pointe. Vous vous dites «Pourquoi moi, pourquoi tant de bonheur, est-ce que la chance serait enfin en train de tourner en ma faveur ?», et elle vous annonce que le type encadré dans le photo, celui-là même que vous regardez avec insistance, les yeux remplis de crainte parce qu’il a l’air d’une brute épaisse et que vous ne pouvez vous empêcher de penser qu’il ne sera peut-être pas enchanté de vous trouver en compagnie de sa femme en rentrant d’une dure journée de labeur, elle vous annonce qu’il est décédé l’année dernière, pulvérisé par un poids lourd sur l’interstate 35, une des routes les plus meurtrières de tout le Texas. Vous aimeriez sauter de joie jusqu’au plafond, sombrer sans retenue dans la plus infâme concupiscence, mais, Dieu merci, la bienséance et la volonté de ne pas passer pour un sale con et un gros porc lubrique vous empêchent de décoller. Quoi qu’il en soit, il semblerait en effet que la chance soit en train de tourner en votre faveur. Et attendez, ce n’est pas tout. Vous lui posez prudemment la question de savoir s’il elle lui a trouvé un remplaçant, et elle vous répond, tout en posant sur vous un regard tropical qui vous transforme le bas-ventre en fourmilière, que non, elle n’a pas la tête à ça, que c’est encore trop tôt pour songer à la bagatelle, alors que des vieux bouts de vêtements et des lambeaux de chair pourrie s’accrochent encore aux os de ce pauvre Edward qui s’agite dans le fond de sa tombe. Bien sûr, vous ne comprenez que trop bien, il faut laisser du temps au temps, répondez-vous en baissant pudiquement les yeux sur les profondeurs de son décolleté et vous fustigeant intérieurement de l’absence totale de moralité qui vous caractérise, ajoutant d’une voix de fausset qu’il commence à se faire tard et que vous n’allez pas tarder à prendre congé, même si vous n’avez nulle part où aller et courez le risque d’atterrir dans un motel pourri tenu un jeune homme féru de taxidermie. Que nenni, s’insurge-t-elle en projetant sur vous une onde de choc parfumée qui vous transporte au septième ciel, ce n’est pas parce que ce pauvre Edward est mort dans des circonstances tragiques qu’elle doit pour autant renoncer aux valeurs de charité chrétienne qui ont toujours été les siennes. Vous trépignez de joie intérieurement en entendant ces paroles de réconfort, aussi douces qu’un tapis de mousse sous vos pieds nus. Un bonheur n’arrive jamais seul, écrivait Marie de Rabutin-Chantal, marquise de Sévigné, dans une lettre à sa fille datée du 20 novembre 1676, et vous n’allez pas tarder à en faire l’expérience. En effet, la porte s’ouvre et une jeune fille, plus belle que le jour et la nuit réunis, fait sont entrée dans la pièce. Et cette charmante jeune fille, sommairement emballée dans une robe à fleurs qui ne cache pas grand-chose de son anatomie dévastatrice, c’est tout simplement Candy, la fille de Janet, la jeune femme qui a eu la gentillesse de vous offrir l’hospitalité en plus d’une généreuse part de tarte à la citrouille (vous détestez ça mais vous forcez quand même à l’avaler jusqu’à la dernière miette, conscient que cette mise en bouche pourrait bien constituer les prémisses d’un festin autrement réjouissant).

Voilà comment les choses pourraient se passer si nous vivions dans un monde meilleur, et surtout si quelqu’un d’autre que Tobe Hooper avait écrit et réalisé le film. Dans le cas présent, vous le savez parce que vous avez déjà vu le film ou n’êtes tout simplement pas tombé de la dernière averse, la résidence en question, à peu près aussi accueillante qu’une décharge à ciel ouvert infestée de rats et de cafards gros comme des rats, n’est autre que celle des cannibales de service, la famille Sawyer. Kirk va être le premier à y passer, suivi de près par Pam qui finit dans le congélo après avoir été empalée sur un croc de boucher. Je vous passe les détails, mais je vous promets que la scène ne manque pas de piquant. La curiosité est un vilain défaut, et si vous en doutez encore, faites confiance à Bubba Sawyer pour vous en faire la démonstration aussi sonore que tranchante. Géant consanguin doté d’un cerveau de batracien, Bubba sait qu’il

est moche comme un pou trisomique et ne supporte pas son reflet dans un miroir. Il porte un masque en peau humaine, arraché au crâne de l’une de ses nombreuses victimes. Bûcheron dans l’âme, il découpe les voyageurs de passage à la tronçonneuse avant de les refiler à son frère Drayton, le cuisinier de la famille, qui se charge de les transformer en fromage de tête (bon, ça, le fromage de tête, avec des pickles, des cornichons et un bon verre de vin blanc sec) et chili con carne (un des meilleurs de la région, si l’on en croit les voyageurs qui ont eu le privilège de le déguster, et surtout de survivre à la dégustation). Après Kirk et Pam, c’est au tour de Jerry, le petit copain de Sally parti à leur recherche, de se faire défoncer la gueule à coups de marteau par Bubba Sawyer, plus connu dans le milieu SM sous le nom de Leatherface. Sally, restée avec son frère handicapé pour veiller sur lui, n’a aucune nouvelle de ses amis et voit la nuit tomber avec une anxiété grandissante. Je suppose que Hooper, quand il a écrit le scénario, s’est dit que ce serait sympa de voir un type en fauteuil roulant se faire découper à la tronçonneuse par un géant demeuré avec un masque de cuir sanguinolent sur le visage. S’en prendre avec une telle violence à une créature sans défense ne pourrait que faire date dans l’histoire du septième art et du film d’horreur en particulier, genre mineur que ses plus ardents défenseurs considèrent comme le nec plus ultra de la création artistique. Il a également compris que le film serait encore plus crédible si l’on insinuait perfidement qu’il n’était en aucun cas une fable horrifique issue du cerveau malade d’un réalisateur indépendant plus ou moins détraqué, mais la relation fidèle et sans complaisance d’événements s’étant réellement déroulés quelque part au fin fond du Texas. Oui, braves gens, tandis que vous ronflez paisiblement dans vos pavillons de banlieue, bien à l’abri derrière vos portes blindées et vos système d’alarme dernier cri, le 357 Magnum sous l’oreiller, il existe encore des endroits reculés où la sauvagerie la plus extrême s’exerce en toute liberté. Pendant que vos sirotez vos cocktails au bord de vos piscines à quatre-vingt mille dollars, insensibles à la misère du monde, des jeunes gens dans la fleur de l’âge se font découper à la tronçonneuse par des mongoliens de deux mètres de haut au visage masqué. Pensez-y, quand vous vous gaverez de pop corn devant la télé, et n’oubliez pas de vous renseigner à deux fois avant de partir à l’aventure. Vous vous croyez plus malin que les autres, mais pourriez vous aussi vous exposer à de très graves dangers. Et vous, parents, tremblez quand vos enfants décident de partir en vacances à l’autre bout du monde, certains que la fraîcheur de la jeunesse leur ouvrira toutes les portes et les préservera du mauvais sort. Elle pourrait tout aussi bien leur ouvrir grand les portes de l’enfer.

Dans la même veine, je me ferai également une joie de mentionner le très glauque et méconnu Maniac de William Lustig : Franck Zito ne fait pas partie de ces gens qui ont eu la chance de vivre dans un foyer harmonieux, avec des parents aimants qui ne passent pas leur temps à picoler et s’engueuler, et élèvent leurs enfants dans le respect des saines valeurs du travail et l’amour du prochain. Non. Quand le père de Franckie n’était pas en train de s’arsouiller au bistrot, il rentrait à la maison pour tabasser sa femme et ses gosses. Un jour pas fait comme un autre, il s’est barré et on ne l’a jamais revu. Quant à la mère de Franckie, elle ne valait guère mieux que son père. Elle-même largement alcoolique et dépendante de nombreuses substances prohibées, elle se vendait au plus offrant pour arrondir ses fins de mois. Contrairement à d’autres prostituées dans son genre, elle bossait à domicile et le petit Franckie assistait fréquemment aux ébats de sa mère et ses amants de passage, une «clientèle» dont je vous laisse imaginer l’élégance naturelle et la distinction.  En dépit de son comportement inadapté et totalement dépourvu de chaleur humaine, Franckie avait élevé sa mère au rang d’idole absolue et rêvait du jour où les gros porcs qui lui passaient sur le corps trouveraient leur juste châtiment. Ce n’était pas elle la coupable, mais ces ordures qui profitaient de sa faiblesse pour assouvir leurs coupables penchants. Et lui aussi était coupable de ne pas pouvoir la protéger. Sa mère reçoit des hommes chez elle, mais tous n’ont pas les mêmes intentions. Certains, par exemple, s’intéressent davantage à Franckie qu’à sa mère. Ils vont même jusqu’à faire des offres très alléchantes pour s’octroyer le droit de faire de lui ce que bon leur semble. Sa mère, toujours à court de fric pour payer ses doses, explique à Franckie qu’il va devoir faire à ces messieurs certaines des choses qu’elle-même fait avec eux, notamment avec sa bouche. Franckie hésite, mais sa mère insiste en disant qu’ils risquent de lui faire du mal s’il ne le fait pas. Il ne voudrait pas que les vilains messieurs fassent du mal à sa maman, n’est-ce pas ? S’il est bien sage, maman lui fera un gros câlin quand ils seront partis. Des années plus tard, longtemps après la mort de sa mère, Franck a développé de sérieux troubles de la personnalité. En clair, il est complètement barré, incapable de se comporter normalement en société. Comme Hugh Hefer, il vit entouré de mannequins. Sauf qu’il ne vit pas à Beverly Hills, dans un manoir de deux mille mètres-carrés avec (entre autres) salle de sport, piscine, tennis, zoo privé et parc paysager avec grotte comme dans les contes de fées, et que ses amis ne s’appellent pas John Lennon, Elvis Presley, Alec Baldwin, Sylvester Stallone, Pamela Anderson, Cameron Diaz, Leonardo DiCaprio ou Kim Kardashian. Oh non. Franckie n’a aucun ami, sa vie est loin d’être un conte de fées, et ses mannequins à lui n’ont rien à voir avec les créatures plantureuses au sourire éclatant qui s’agglutinent comme des mouches autour du patron de Playboy. Ses mannequins à lui ne marchent pas, ne parlent pas, ne respirent pas. Ils ne sont pas morts, non, c’est juste qu’ils n’ont jamais été en vie, même s’ils ont toutes les apparences de la réalité. Ce sont juste des objets, des poupées à taille humaine qu’on habille pour exposer dans les boutiques de fringues. Plastiquement parfaites, certes, mais surtout parfaitement en plastique. C’est précisément cette inertie, alliée à cette troublante ressemblance, qui peut donner des idées à certains individus sexuellement perturbés. Et même à d’autres, qui n’ont à priori rien de commun avec les pervers en question, mais sont juste fatigués d’avoir à composer avec les exigences d’un ou une partenaire en chair et en os (vous le voyez, j’essaie d’être aussi inclusif que possible, conscient que les fanatiques de la non-binarité intersexuelle et transidentitaire seraient trop contents de planter ma tête au bout d’une pique au moindre faux pas). Pour beaucoup d’hommes, confrontés à la difficulté croissante de vivre en couple (depuis, pour faire court, que la femme n’est plus cette petite chose fragile et soumise qu’on pouvait engrosser à loisir et asservir en toute tranquillité), la femme parfaite serait une poupée sexuelle ou un androïde hyperréaliste offrant tous les avantages d’un être humain sans en avoir les inconvénients. Notons, au passage, qu’une telle évolution des mœurs éviterait bon nombre de féminicides perpétrés par des conjoints alcooliques et violents. Car si un être vivant ne vous appartient pas, il n’en va pas de même pour un humanoïde en silicone. Même les prédateurs sexuels les plus abjects, comme l’ont prouvé les récentes affaires de ventes de poupées pédopornographiques sur Shein, AliExpress et Amazon, pourraient y trouver leur compte. Non seulement de nombreux enfants ne seraient plus traumatisés par leurs agissements, et des familles brisées dans la foulée, mais de nombreuses vies seraient épargnées, sachant que les pires d’entre eux n’hésitent pas à tuer pour s’assurer du silence de leurs victimes. Sur le plan de la déontologie, je reconnais que la pilule est assez difficile à avaler. C’est déjà assez moche de savoir qu’on vit entouré de porcs qui ne songent qu’à abuser de nos enfants (certains vous diront que c’est encore plus moche de ne pas savoir qui ils sont, et ils n’auront sans doute pas complètement tort), on ne va pas en plus leur fournir de quoi assouvir leurs fantasmes en toute légalité. Cela reviendrait, en quelque sorte, à officialiser leur existence et accepter de vivre avec eux en bonne intelligence. C’est d’autant plus difficile à imaginer qu’on sait pertinemment qu’il y aura toujours des brebis encore plus galeuses que les autres pour s’affranchir des règles en vigueur, jetant le discrédit sur toute la communauté de gentils pédophiles respectueux qui se contentent de tripoter des poupées en silicone.

Lustig, après de rapides débuts dans le porno, s’est tourné vers le cinéma de genre, le psycho-killer movie en l’occurrence. Un beau jour de l’année 1979, il est allé trouver son pote Joe Spinell, abonné aux seconds rôles de flics véreux et truands sans envergure, pour lui proposer de se glisser dans la défroque peu avenante de Franck Zito, propriétaire d’une boutique de mannequins et tueur de femmes psychotique obsédé par le souvenir de sa mère prostituée. Spinell, qui n’avait sans doute rien de mieux à foutre, a accepté, et s’est impliqué dans le film au point de participer à l’écriture du scénario. Vivre avec des femmes en plastique, c’est bien, mais ça ne suffit pas à faire un film d’horreur digne de ce nom. Pas mal de trucs ont déjà été faits, il va falloir trouver quelque chose d’un peu original si on veut avoir une chance de sortir du lot et décrocher un Oscar. À l’époque, la plupart des tueurs écumaient les magasins de bricolage pour s’équiper. Agrafeuse, burin, clé à molette, disqueuse, faucille, fourche, hache, machette, marteau, masse, meuleuse, pelle, perceuse, pied à coulisse, pioche, pistolet à clous, poinçon, tournevis, tronçonneuse, scie sauteuse, spatule, tout y passait dans la joie et la bonne humeur. Si on avait pu tuer quelqu’un à coups de tire-bouchon, ouvre-boîte ou lime à ongles, je vous assure qu’on ne se serait pas gêné pour le faire. J’estime toutefois que le couteau, qui peut sembler désuet, sinon vulgaire à première vue, reste une valeur sûre, à condition bien évidemment qu’il soit de taille conséquente et manipulé par quelqu’un qui connaît son affaire. Et c’est exactement ce que se sont dit nos deux compères : Zito, qui n’a aucun goût particulier pour le bricolage, va se servir d’un bon vieux couteau de chasse des familles. Et à quoi peut bien servir un bon vieux couteau de chasse des familles, hein, je vous le demande ? Eh bien à dépecer des animaux, par exemple, ou tailler des morceaux de bois pour en faire des armes redoutables, comme Stallone dans Rambo. Mais Zito crèche à New York, pas dans la jungle, et ce n’est pas dans les rues de la Grosse Pomme qu’on risque de croiser une biche ou tapir. Il fallait donc trouver quelque chose de plus en rapport avec le profil de sociopathe schizophrène de l’intéressé.

Spinell, avec sa gueule de type qu’on n’avait pas envie de croiser de nuit au détour d’une ruelle sombre et humide, au pavé glissant et aux trottoirs envahis de sacs poubelles éventrés, était un grand fan de western. Comme vous le savez, les westerns sont ces films qui évoquent les différents aspects de la conquête de l’Ouest. Les Rosbifs ont débuté avec treize colonies sur la côte Est, tandis que les Français étaient plutôt bien installés au centre du pays. Ils s’étaient même fait quelques potes indiens pas trop regardants avec lesquels ils commerçaient pour tenter s’intégrer harmonieusement dans le paysage, même s’il était clair que ça risquait de péter un jour ou l’autre. Plus au Sud, on trouvait les Espagnols qui continuaient à creuser des trous un peu partout pour trouver de l’or. Il y avait aussi quelques Hollandais qui gravitaient dans le secteur, arrivé dans le sillage de Peter Stuyvesant, mais ils ne faudrait pas longtemps pour le rejeter à la mer en cas de conflit. On trouvait un peu de tout, en fait, mais pas en quantité suffisante pour représenter une réelle menace. Tous ces gens bricolaient dans l’espoir de tirer un jour leur épingle du jeu. Un jour, les treize colonies en ont eu marre des exigences de la Couronne et décidé de s’affranchir une bonne fois pour toutes de ses directives. Ils avaient pris tous les risques et entendaient bien créer un nouvel empire dont ils seraient les seuls maîtres après Dieu. La Mère Patrie resterait à tout jamais dans le fond de leur cœur, bien entendu, tatouée à l’encre rouge de la conquête normande et la révolte des barons, mais ils avaient aujourd’hui d’autres chattes à fouetter, et non des moindres. Dans la foulée, ils ont viré les Français et le reste, laissant les Espagnols se dépatouiller avec l’Amérique du Sud. Je vous la fais courte, historiquement approximative, mais c’est grosso modo comme ça que les choses se sont passées. Comme ils n’occupaient qu’une toute petite partie du très vaste territoire sur lequel ils venaient de poser le pied, ils ont repris leur bâton de pèlerin et décidé d’aller voir ce qui se passait un peu plus loin. Ils n’ont pas tardé à se rendre compte que, en dehors des colons blancs et esclaves noirs qu’ils avaient amenés dans leurs valises, il existait une race d’autochtones dont il n’allait peut-être pas être aussi facile de se débarrasser. Même s’il s’agissait de sauvages peinturlurés tels qu’on en avait maintes fois croisés dans les expéditions passées, ils risquaient de sérieusement entraver la marche triomphale de la civilisation occidentale vers l’avenir radieux du libéralisme économique. Dès qu’un brave type de colon sans histoire s’installait quelque part, tranquille, avec sa petite famille et ses quelques têtes de bétail qui broutaient paisiblement dans le champ voisin, ces abrutis lui tombaient dessus en poussant des hurlements stridents et saccageaient tout sur leur passage, allant jusqu’à violer les femmes et enlever les enfants pour les élever à la sauce indienne. Déjà qu’il fallait se coltiner les hors-la-loi qui foutaient la merde dans les saloons, se battaient en duel à tous les coins de rues et détroussaient les voyageurs (il leur arrivait aussi de violer les femmes), ça allait devenir difficile de faire son beurre s’il fallait avoir en permanence cette bande d’emplumés sur le dos. La tâche était d’autant plus ardue qu’ils n’étaient pas constitués en une seule armée, qu’on aurait pu vaincre en une seule fois, mais une multitude de tribus éparpillées aux quatre coins du pays. L’avantage, c’était que les tribus en question n’étaient pas toujours en très bons termes, ce qui permettait, avant bien sûr de se retourner contre eux, de pactiser avec les uns pour affaiblir les autres. Et comme c’était des sauvages, c’est à dire des gens qui vivaient sous des huttes en peau de lapin, se trimballaient les couilles, lâchaient des caisses dans la plus totale insouciance, fumaient des pipes de trois mètres de long et ne disposaient que d’un armement rudimentaire pour faire valoir leurs droits à la terre de leurs ancêtres, il ne devrait pas être trop difficile de leur faire fermer leur grande gueule une bonne fois pour toutes. Car dis-toi bien ceci, ô vil suppôt de l’impérialisme américain dont l’existence même représente un grave danger pour la survie de l’espèce à laquelle j’ai la malchance d’appartenir : il existait, dans les antiques contrées de la lointaine Amérique, des kyrielles de tribus qui se partageaient les vastes plaines de l’Ouest, du Texas à l’Etat de Washington en passant par la Lorraine, l’Arizona, l’Idaho, le Colorado, le Montana, l’Iowa, le Nouveau-Mexique, le Nevada, l’Oregon, l’Utah, l’Arkansas, le Kansas tout court, le Wyoming et l’entrée de service. En plus ou moins bonne intelligence, il est vrai, car il leur arrivait fréquemment de se taper sur la gueule comme des gamins de cinq ans qui se disputent un jouet. Ces grands enfants avaient de longues conversations avec la lune, les pierres et les ruisseaux, dansaient avec les loups, et respectaient chaque chose, même la plus insignifiante, comme un membre à part entière de leur famille. Chaque année, ils attendaient le retour des bisons pour remplir les frigos, faire le plein de peau et viande séchée, et les bisons étaient tellement nombreux qu’il y en avait largement assez pour tout le monde. Leur vie s’écoulait ainsi, au gré du vent et des saisons. Ils n’en demandaient pas davantage et pensaient que les choses dureraient ainsi jusqu’à la fin des temps. Quand il a vu débarquer l’homme blanc, avec ses chemises carreaux, ses jeans et ses santiags, l’Indien a tout de suite compris que les jours heureux étaient terminés. Il faut dire que l’homme blanc s’est vite montré envahissant, totalement irrespectueux des usages, coutumes et règles en vigueur, arrogant, violent, dominateur, certain de la supériorité de son espèce, se comportant comme le roi du monde en pays conquis. Il a bien essayé de résister, mais son matos de fortune ne faisait pas le poids face à la puissance de feu de l’envahisseur. Le temps de décocher une flèche et il s’était déjà pris trois balles dans le buffet. La lutte était inégale et s’est achevée comme elle devait s’achever : par la victoire écrasante du pot de fer contre le pot de terre, sa prise de contrôle totale du territoire et sa relégation aux archives de la vieille nation indienne, réduite à survivre sur les lopins de terre indignes généreusement alloués par le gouvernement fédéral des Etats Unis d’Amérique. Le Visage Pâle est arrivé («et je vis un cheval fauve, et celui qui était monté dessus avait nom la Mort, et l’Enfer suivait après lui», Apocalypse 6:8) et il a dit : vous êtes ici chez moi, faites vos valises et foutez le camp. Fini les chants et les danses au clair de lune au son du tambour, les discussions interminables au coin du feu et la chasse au bison dans les vertes prairies du Wyoming et de l’Oklahoma. On est venu ici pour s’établir, entre gens civilisés, on n’a pas besoin de sauvages pour tenir la chandelle. On va faire des trous partout pour trouver de l’or et du pétrole. Ouais, avec des grosses machines qui fument et font du bruit. Grâce à ça on va devenir très riche et on va transformer vos terrains vagues en cités radieuses et tentaculaires. On va aussi construire des lignes de chemin de fer pour transporter nos marchandises d’un bout à l’autre du pays, des grosses bagnoles pour mettre plein d’essence dedans et une route pour aller de Chicago à Santa Monica. On va vous la mettre bien profond et vous ne pourrez rien faire pour nous en empêcher. Car la terre n’appartient à personne, même si les membres d’une même famille y sont enterrés depuis des générations, et comme elle n’appartient à personne, elle appartient à tout le monde, et donc à moi en particulier. Tu ne croyais tout de même pas que l’état de grâce allait durer éternellement, que t’allais continuer à te la couler douce sous ton tipi, caracoler dans les Rocheuses et fumer le calumet de la paix pendant que nous, fleurons de l’espèce humaine et garants de la civilisation, on allait rester sans rien dire coincé sur un rocher de l’autre côté de l’Atlantique ? Non, mon gars, on n’a pas fait tout ce chemin pour enfiler des perles et chanter des cantiques. Donc, je te le dis tout net, ce qui va se passer maintenant, c’est qu’on va te parquer dans des endroits pourris où même les rats et les cafards ne voudraient loger pour rien au monde. Mais comme on n’est pas des monstres, on va te filer du tord-boyau à volonté pour que tu puisses passer tes journées à picoler pour oublier que tu t’es fait entuber dans les grandes largeurs. Des questions ? Quoi ? Si nous aussi on picole ? Bien sûr, qu’on picole, mais nous on a de bonnes raisons pour le faire. Autre chose ? Non ? Dans ce cas ramassez vos cliques, vos claques, et foutez-moi le camp d’ici ! L’aigle s’est envolé, le coyote s’est tu, le chien de prairie a regagné sa tanière, le loup ne chante plus sous la lune (non, il n’a plus envie depuis que les Indiens sont partis), mais soyez au moins certains d’une chose, farouches guerriers des sauvages plaines de l’Ouest, c’est que vos noms resteront à jamais gravés dans nos mémoires, la mienne et celle de tous ceux qui croient encore en la justice et la liberté (rires) : Cheyennes, Sioux, Pawnees, Kiowas, Crows, Shoshones, Arapahos, Comanches, Cherokees, Apaches, Têtes Plates, Nez Percés, Pieds Noirs, Culs d’Oursins et Sacs à Puces. Vos âmes pures chevauchent désormais dans les grandes plaines de l’au-delà aux côtés de Wakan Tanka, Pah, Shakouroun, du corbeau, du vieil homme et du grand lièvre. Amen, tchuss, Allahu akbar et gode save the Gouine !

Spinell (je rappelle à toutes fins utiles qu’on est en train de parler de Maniac, le film de Lustig) n’avait pas fait les années d’études nécessaires à l’obtention d’un diplôme d’historien en bonnet difforme (ce qui n’est pas mon cas non plus, je vous rassure tout de suite, pas plus que celui de Lustig ou Caroline Munro, l’actrice principale du film, et je ne parle même pas de Gail Lawrence, Kelly Piper et Sharon Mitchell), mais il avait vu suffisamment de westerns pour savoir à quoi s’en tenir au sujet des Amérindiens. Et il adorait ça, les Amérindiens. Lui-même, d’ailleurs, avec sa sale gueule de psychopathe au look ringard et aux yeux exorbités, aurait très bien pu tourner dans un western horrifique du genre The Wind, Dead in Tombstone ou The Burrowers, ou encore, dans un style plus classique, l’intemporel Jesse James contre Frankenstein. Que celles et ceux qui ne l’ont pas vu au moins une bonne douzaine de fois soient immédiatement traduits en justice et bannis à tout jamais des salles de cinéma de France et de Navarre ! En l’an de grâce 1966, Jesse James contre Frankenstein et Billy the Kid contre Dracula sortent coup sur coup. L’émotion est grande (nulle), la foule (ne) se presse (pas), la critique se déchaîne (l’ignore totalement). On doit cette prouesse technique à William Beaudine, génie méconnu dont personne n’a jamais entendu parler et tout le monde se contrefout éperdument. Baudine vient du cinéma muet, où il avait l’habitude d’enchaîner les films par treize à la douzaine. Je considère pour ma part que la réputation de films comme La Parade du rire, Le Retour de Philo Vance et l’inénarrable Gorille de Brooklyn mériterait d’être revue à la hausse, n’en déplaise aux amateurs des frères Dardenne, Xavier Dolan et Bruno Dumont. On me murmure dans l’oreillette qu’il avait prévu d’enchaîner avec Doc Holliday contre la Momie et Calamity Jane contre Jack l’Eventreur, films qu’il n’a malheureusement pas eu le temps de tourner pour des raisons de santé. Il va de soi que je me ferais un plaisir de le faire moi-même si j’avais un peu de temps et d’argent devant moi. J’appellerais l’agent de Scarlett Johansson et je lui dirais : j’ai un rôle pour Scarlett. Lui : Ah bon ? Quelle heureuse nouvelle ! Moi : Oui, le vais tourner Calamity Jane contre Jack l’Eventreur avec mon téléphone portable et j’ai pensé à elle pour le rôle principal. Je vais aussi écrire la musique du film et jouer le rôle de Jack l’Eventreur. Lui : Pas con. Moi : Et tu sais quoi ? Lui : Non. Moi : J’ai pensé à Scarlett pour le rôle de Calamity Jane. Lui : Excellente idée, je vais voir si elle est disponible. Je te rappelle dans une heure. Moi : Okay, pas de problème. Il faut savoir que lui et moi on est comme cul et chemise, même si c’est généralement plutôt moi qui fait le cul et lui la chemise. Lui, une heure plus tard comme convenu : Désolé, vieux, mais ça ne va pas être possible. Moi : Non ? Lui : Si. Elle doit tourner dans un remake de L’Exorciste signé Mike Flanagan. Moi : Elle doit jouer quoi ? Lui : On ne sait encore pas trop. Peut-être Regan MacNeil, devenue une séduisante jeune femme, qui doit à nouveau faire face à des démons bien décidés à lui faire dire des horreurs et tourner la tête à 360 degrés. Ou alors Linda, la fille que le père Karras et Sharon Spencer, la secrétaire de Chris MacNeil, ont eu en secret, et qui se transforme en succube sous l’influence de Lamashtu, une divinité summérienne qui n’est autre que la propre femme de Pazuzu. Moi, déçu : Bon, tant pis. Je vais voir si Leven Rambin est disponible. Lui : La fille du Dr Sloan dans Grey’s Anatomy ? Moi : Oui, je l’ai trouvée très bonne dans Mank de David Fincher. Lui : Elle a joué dedans ? Moi : Oui, on l’aperçoit dans une scène ou deux. Elle crève l’écran, je trouve. Lui : Mmmmouais, je suis pas sûr. Prends plutôt Amanda Seyfried, à ce moment-là. Ou Sydney Sweeney (NDLR : recordwoman du cri le plus long et l’infanticide le plus spectaculaire dans Immaculate de Michael Mohan). Moi : OK, merci, c’est pas grave. De toute façon, je ne pense pas que Scarlett aurait accepté de jouer gratos dans mon film. Lui : Je ne pense pas non plus. Mais on ne sait jamais, Scarlett est une jeune femme tout à fait imprévisible.

Bon sang, je n’en reviens pas qu’on ait pu passer en une fraction de seconde des Sioux à Sydney Sweeney, laquelle, soit dit en passant, pourrait se révéler tout à fait attractive dans un western gore où elle porterait des jeans American Eagle et jouerait la fille d’un pionnier enlevée par des Indiens cannibales. D’ailleurs, je suis à peu près certain que si Franck Zito avait croisé Sydney Sweeney dans les rues de New York, il aurait aussitôt fait une fixette sur elle et rêvé jour et nuit de lui offrir une petite coupe. Et quand je parle de petite coupe, il ne s’agit bien évidemment pas d’une petite coupe de Champagne, millésimé ou non, mais d’une coupe de cheveux. D’après Hérodote, les Scythes, farouches guerriers venus des steppes pontiques d’Anatolie, avaient pour habitude de boire le sang de leurs victimes. Grand bien leur fasse, il ne faisait pas toujours très chaud dans le secteur, les combats étaient souvent longs et éprouvants, on peut comprendre qu’ils aient eu besoin de se détendre un peu après l’effort. C’est le genre de situation où on est bien content de boire un petit verre de vin chaud, par exemple. Comme ils n’en avaient pas sous la main, une bonne rasade de sang frais faisait l’affaire. Ils auraient pu s’en tenir là, mais non. Après avoir étanché leur soif, ils tranchaient la tête de leur ennemi pour la ramener au chef qui les félicitait en leur donnant de grandes tapes dans le dos. Les rires et les chants fusaient autour du feu qui crépitait gaiment sous la voûte étoilée, dans une saine ambiance de camaraderie où tous les excès étaient autorisés. Ensuite, les femmes et les enfants étaient violés par les participants qui, il faut bien le dire, puaient méchamment de la gueule à force de boire du sang et s’empiffrer de viande crue. En guise de souvenir, il était d’usage de conserver le cuir chevelu de la victime. Une fois débarrassée de ses impuretés, la chose était utilisée en tant que serviette pour la table, ce qui ne manquait quand même pas d’une certaine classe. Et quand on était un valeureux guerrier et qu’on en avait accumulé plein au cours de ses nombreuses campagnes, il était de bon ton de les coudre ensemble pour s’en fabriquer des vêtements. Un petit gilet bien seyant, par exemple, ou une paire de moufles. Ed Gein n’aurait pas fait mieux, sauf que lui préférait déterrer des cadavres plutôt que risquer de prendre un mauvais coup sur un champ de bataille. Zito, pour sa part, se contentait plus modestement de scalper des filles pour habiller le crâne de ses mannequins, leur donner un peu de cette humanité qui leur faisait cruellement défaut.

Et enfin, last but not least comme disent nos amis anglo-saxons, incontestables spécialistes du genre, l’inoubliable Silence des agneaux de Jonathan Demme (d’après le bouquin de Thomas Harris, journaliste et écrivain de seconde zone qui a eu la riche idée de transformer un psychiatre réputé, brillant, cultivé, mélomane et gastronome, en sociopathe manipulateur, sadique et cannibale), dans lequel un certain Jame Gumb, alias Buffalo Bill, jeune homme perturbé par une identité sexuelle mal définie, lui aussi adepte des travaux de couture, enlève des femmes pour leur voler leur peau. Il utilise notamment, pour les faire monter dans son van, le coup dit «du bras dans le plâtre», un stratagème emprunté à Ted Bundy. Ce dernier, élevé par les parents de sa mère qui lui ont longtemps fait croire qu’ils étaient se parents et qu’elle était sa sœur, n’a jamais réussi à s’intégrer parfaitement au monde des gens normaux. D’autant moins, si l’on en croit certaines des rumeurs sournoises qui couraient dans le voisinage, que son grand-père était son père, autrement dit qu’il était le fruit pourri d’une relation incestueuse entre sa mère et son grand-père. Le coup dit «du bras dans le plâtre» est librement inspiré des pratiques de Ted. Ce dernier, en effet, n’avait pas de van, mais une Coccinelle jaune. Il n’avait pas non plus le bras dans le plâtre, technique un peu complexe à mettre en œuvre, mais en écharpe. Ce qui est certain, en tout cas, c’est qu’il se pointait sur les campus et se servait de cette prétendue infirmité pour s’attirer les bonnes grâces de ses victimes. Il leur demandait de l’aider à charger des objets lourds et encombrants dans le coffre de sa voiture, et dès qu’elles étaient affairées à se rendre utiles il en profitait pour les agresser lâchement par derrière. Teddy était un petit malin qui avait un gros problème avec les femmes. Il avait aussi toutes sortes de vilains penchants qui le tenaient éloigné d’une vie épanouie en société. C’est dommage, car en dépit d’un monosourcil pas très gracieux dont il n’a semble-t-il jamais jugé utile de se départir, Bundy appartenait plutôt à la catégorie des beaux gosses charismatiques qui n’ont qu’à claquer des doigts pour que les plus belles femmes leur tombent dans les bras, la tête en arrière, les yeux révulsés, le souffle court et la bouche entrouverte. Rien à voir avec la plupart de nos tueurs en série hexagonaux, qui sont vieux, moches, puent du bec et n’ont finalement pas d’autre choix que s’en prendre à des enfants en bas âge ou des handicapés. Un Emile Louis, par exemple, aurait eu beau se pointer sur des campus avec le bras en écharpe dans une Coccinelle jaune, je doute fort qu’une étudiante, même grosse et moche, ait consenti à lever le petit doigt pour l’aider à charger des objets lourds et encombrants dans le coffre de sa voiture. Son physique ingrat et son QI de mouche à merde l’ont contraint à s’attaquer à des proies sans défense, de pauvres créatures qu’il reluquait dans le rétro de son bus avant de les agresser sexuellement et les laisser pour mortes dans des endroits déserts. De la même façon, un Michel Fourniret, malin comme un renard mais vieux et surtout terriblement moche, aussi appétissant qu’un rat d’égout tombé d’une benne à ordures, devait recourir aux services de sa compagne demeurée pour embarquer des gamines dans sa camionnette et les conduire à leur mort. Un peu à la façon d’un Albert Fish, alias le Vampire de Brooklyn, l’Ogre de Wysteria, the Grey Man ou encore le Moon Maniac, cet espèce de taré made in USA, oiseau de malheur à tête de charognard qui aimait torturer les enfants et s’enfoncer des aiguilles dans le cul. Entendez par là que les Etats Unis ont eux aussi eu leur lot de types moches et déjantés dont la seule apparence aurait suffi à vous faire sauter à pieds joints dans le premier vaisseau en partance pour l’espace. On ne peut pas éternellement se flageller, même s’il est clair que les tueurs yankees ont globalement plus de classe que les nôtres. Un Ed Kemper, for exemple, deux mètres pour cent cinquante kilos de barbaque et un QI de surdoué, donne forcément à réfléchir, surtout quand on sait qu’il s’est payé le luxe, après avoir assassiné ses grands-parents, massacré un certain nombre d’étudiantes et planté des fléchettes dans la tête coupée de sa mère, de se rendre gentiment à la police. On se dit qu’il ne faut finalement pas grand-chose pour que l’esprit humain se mette à dysfonctionner dans les grandes largeurs. Parce que si tel n’est pas le cas, si dysfonctionnement il n’y a pas, alors on est en droit de se poser de sérieuses questions sur l’avenir de notre espèce.

\textsc{Greg} : On se croirait dans Psychose.

\textsc{Moi} : En plus exotique.

Greg, étreignant nerveusement la crosse du Bersa Thunder 380 CC dissimulé sous sa veste : J’ai peur.

\textsc{Moi} : Ne t’en fais pas, je m’occupe de tout.

\textsc{Lui} : C’est bien ce qui me fait peur.

Comme à peu près tout ce qui se trouvait dans l’hôtel, à part les animaux empaillés, Greg et moi, le type de la Réception appartenait à la catégorie de ce qu’on appelle pudiquement les «personnes de couleur». Greg, qui était en train de glisser lentement dans l’alcoolisme dit «mondain», n’était plus très loin d’appartenir à la catégorie très convoitée des personnes de couleur rouge, comme les Amérindiens, bien sûr (encore que cela ne concerne originellement qu’une très petite partie d’entre eux qui avaient pour habitude de s’enduire la peau d’ocre rouge), mais aussi la grande communauté des victimes de coups de soleil (qui ne constituent pas une ethnie à proprement parler), et surtout la seule et unique personne de Donald Trump (le fameux «rouge Trump» qu’on retrouve non seulement sur le visage de l’intéressé, mais aussi les «power ties», casquettes et autres produits dérivés en vente libre sur le Trump Store).

Sur le revers de sa veste, le Réceptionniste portait un badge sur lequel était inscrit «Dumo». J’en ai déduit qu’il s’agissait de son prénom, lequel, à une lettre près (un B en l’occurrence), aurait été celui du sympathique éléphanteau des studios Disney, animal avec lequel, mis à part un certain embonpoint, il ne partageait pas de ressemblance directe. En effet, il avait des oreilles minuscules, à peine plus développées que des branchies, et un nez court et épaté, comme si on avait tapé dessus pendant des heures et des heures avec un rouleau à pâtisserie, un maillet ou tout autre objet contondant, en bois de préférence, autrement dit un organe aux antipodes de cette chose longue et majestueuse qu’on appelle une trompe.

Personne n’a jamais vu un œuf avec des cheveux (sauf en Chine, bien sûr, terre de tous les mystères y compris les plus absurdes et incongrus, je pense notamment à cette manie qu’ils ont de bouffer des œufs bouillis des heures durant dans de la pisse de collégien, pratique tout de même un peu étrange dont on peine à entrevoir clairement la finalité, sauf en Chine, disais-je, où un villageois des environs de Quanzhou, dans le sud-est du pays, a découvert un œuf couvert de poil dans le ventre d’une truie, curieux objet dont la valeur marchande, s’il s’agit bien de la concrétion attendue, pourrait avoisiner le million de dollars), mais on s’attend à en trouver, ne serait-ce quelques uns, sur le crâne d’un être humain. Même celui d’un Chinois chauve, un bonze, par exemple, ou un vieil herboriste du Sichuan, sur lequel on finit toujours par en dénicher un qui traîne au détour d’un pli ou une oreille. Et pourtant, aussi incroyable que cela puisse paraître, je vous fiche mon billet de longues heures de recherche, à la loupe ou au microscope, n’auraient pas révélé la présence du moindre élément de ce type sur le crâne de Dumo, entièrement revêtu d’une substance laquée plus proche de la boule de billard que du cuir chevelu.

Par ailleurs, je ne sais pas si vous avez déjà essayé de pianoter sur le clavier d’un ordinateur avec des boudins créoles à la place des doigts (ou des saucisses de Toulouse), mais même si vous ne l’avez jamais fait, je suis certain que vous n’aurez aucun mal à vous représenter la difficulté de la tâche. Et compatir du même coup, en bon chrétien, bouddhiste ou musulman que vous êtes (je milite en effet, à défaut de l’abandon pur et simple de toute espèce de religion, hormis peut-être le vaudou, la santeria et accessoirement le Temple Satanique de Greaves \& Jarry à Salem, pour une approche plus œcuménique, mystique, néo-romantique et libérale de la Foi, enfin délivrée de ses chaînes et autres clous christiques plantés dans les chairs tuméfiées de l’Espérance et la Rédemption), au calvaire de Dumo, qui se trouvait exactement dans la très pénible situation que je viens d’essayer, au travers de métaphores charcutières à mon sens assez pertinentes, de décrire aussi fidèlement que possible. Car en effet, le pauvre vieux s’échinait à pianoter sur un clavier dont les touches étaient dix fois trop petites pour ses extrémités digitales. Forcément, la virtuosité s’en trouvait grandement altérée, sa bonne humeur également, et l’ampleur de la tâche mobilisait l’entièreté de ses facultés intellectuelles. Son état de concentration, extrême, n’ouvrait aucune brèche sur le monde extérieur. On aurait pu forcer l’entrée de l’hôtel au bulldozer sans qu’il s’en aperçoive, et un troupeau de buffles lancés à pleine vitesse aurait pu faire de même sans obtenir davantage de résultat. Des gouttes de sueur perlaient sur son front, avant de descendre le long de ses joues, tels des petits animaux rampants laissant une trace humide dans leur sillage, et s’écraser lourdement en contrebas, sur le clavier de l’ordinateur, ce qui obligeait Dumo à l’essuyer sans arrêt avec des mouchoirs en papier, les mêmes (enfin d’autres, sortis des mêmes paquets) dont il se servait pour s’éponger le visage aussi souvent que possible avant de les jeter sans précaution dans la corbeille sise à proximité (il ne prenait pas le temps de viser, autant dire que les trois quarts atterrissaient à côté). De la même façon que certains se rongent les ongles, manipulent un objet quelconque, remuent frénétiquement telle ou telle partie de leur corps (le plus souvent la jambe, comme s’ils étaient pressés de s’enfuir), ou se triturent fébrilement une mèche de cheveux (toujours la même, comme si elle était rattachée à certaines terminaisons nerveuses déterminantes pour le succès de l’opération), lui se caressait machinalement le haut du crâne avec le plat de la main. Ce faisant, il se retrouvait avec une main pleine de sueur qu’il devait à son tour essuyer (avec un de ces mouchoirs en papier dont il faisait un usage compulsif, comme je l’ai indiqué, mais aussi très souvent avec certaines parties parmi les plus accessibles de ses propres vêtements, avec les conséquences désastreuses que l’on imagine en termes d’apparence et de propreté), avant de la tremper à nouveau dans la sueur de son crâne, l’essuyer à nouveau, et ainsi de suite, devenant ainsi la proie du jeu de dupe dont il était le principal artisan.

N’entrevoyant aucune issue favorable à cette effroyable tragédie, je me suis vu, l’espace d’un court instant, dégainer Manu et tirer une balle dans la tête de ce pauvre Dumo.

Et me suis aussitôt ravisé, bien entendu, conscient que le remède aurait quand même été quelque peu disproportionné. J’ajoute que des témoins se trouvaient sur les lieux, assez peu nombreux, certes, mais non moins vigilants, qui se seraient fait une joie de témoigner en ma défaveur. Quand on est, comme moi, quelqu’un dont l’essentiel de l’activité consiste à envoyer ses concitoyens croupir derrière les barreaux d’une cellule, il ne fait pas bon se retrouver dans la même situation. Tous n’attendent qu’une seule chose, en dehors de la libération conditionnelle ou la réduction de peine : vous voir débarquer pour vous mettre en pièces. Le monde carcéral est une poubelle dans laquelle on entasse les rebuts de la société, ses déjections, les corps étrangers qu’on extirpe de son épiderme. On aimerait tirer la chasse une bonne fois pour toutes, mais l’éthique commande de les maintenir en vie, leur offrir le gîte et le couvert avant de les relâcher dans la nature avec l’espoir que ce séjour à l’ombre leur aura rafraîchi les idées. Mais si vous placez des voleurs, des violeurs et des assassins dans le même périmètre, qu’est-ce qui va se passer à votre avis ? Eh bien soit ils s’entretuent, ce que tout le monde souhaite intérieurement sans oser publiquement l’avouer, soit ils se serrent les coudes entre confrères et ressortent de là gonflés à bloc comme jamais, bien décidés à unir leurs forces pour prendre leur revanche sur la société, lui faire payer au centuple le préjudice subi. En enfer, les enfants de chœur virent leur cuti. Rien de tel, pour former des criminels endurcis, que de leur tanner le cuir en prison. Le raisin de la haine fermente pour engendrer le nectar du mal. L’ennui, c’est qu’on ne sait pas quoi en faire, et que les entretenir ad vitam æternam finit par coûter cher à la société. Imaginez un instant que tout délinquant, aussi jeune et minime soit-il, soit éliminé physiquement au moindre faux pas, réduit en cendres et rayé des cadres de la civilisation, en partant du principe que les chances qu’il s’amende sont nulles, et que même si par miracle il y parvenait, le jeu n’en vaut de toute façon pas la chandelle. Autant former tout de suite des gens instruits et compétents plutôt que se casser le cul (et la tirelire) à essayer de rattraper de justesse des repris de justice. Pourquoi, en gros, ne pas travailler à l’américaine ? On sait que la police ne peut pas être partout, en permanence à tous les coins de rues, l’agent du maintien de l’ordre n’a pas quatre bras et encore moins le don d’ubiquité, alors pourquoi ne pas permettre aux honnêtes gens de faire eux-mêmes le sale boulot. Vous travaillez d’arrache-pied pour gagner honnêtement votre vie, vous revenez tranquillement du cinéma ou du restaurant après une dure journée de labeur, comme les parents de Bruce Wayne, et vous vous faites sauvagement agresser par un junkie nauséabond qui en veut à votre portefeuille et aux bijoux de votre femme. Bienvenue à Gotham City. Vous attendez quoi, qu’il vous assassine froidement dans une ruelle sombre et humide ? Non, bien sûr : vous sortez votre flingue (vous ne sortez jamais sans lui, il est votre ami pour la vie)et faites sauter le caisson du rat d’égout sans autre forme de procès (je sais bien qu’il faut que les juges et les avocats gagnent leur croûte, mais ce serait bien qu’ils ne le fassent pas au détriment de notre santé). Et hop, vous faites d’une paire deux couilles : non seulement vous sauvez votre portefeuille et accessoirement votre femme et (surtout) ses bijoux (qui vous ont coûté un bras, parce que vous, contrairement à certains qui ne s’embarrassent pas de principes, vous payez vos dettes rubis sur l’ongle), mais en plus vous participez gracieusement au ramassage des ordures ménagères, lesquelles, on le sait, sont de plus en plus envahissantes dans nos cités laissées à l’abandon. D’accord, vous êtes une grosse merde et votre conscience vous gratte un peu le cul pendant un jour ou deux, mais au moins vous assumez vos responsabilités civiques et méritez pleinement votre qualificatif de citoyen modèle (et pouvez toujours aller vous confesser au curé de la paroisse qui se fera un plaisir de vous donner l’absolution). Fini de faire le tri, on ratisse large et on fait le ménage à la louche, tout doit disparaître. Même chose pour le petit voyou qui vole une barre de céréales chez l’épicier du coin : négatif, votre Horreur, je plaide coupable, mon client est une ordure et on ne va pas s’amuser à lui taper gentiment sur les doigts alors qu’on sait pertinemment qu’il va recommencer à la première occasion. On le tue tout de suite, on gagne du temps, de l’argent, et tout le monde est content (sauf lui et sa famille, bien sûr, mais ça on s’est fout, c’est de leur faute, ils n’avaient qu’à mieux l’éduquer, et d’ailleurs on va éradiquer le nid dans la foulée). Je ne sais pas, moi. J’essaie juste de trouver des solutions pour éviter que nos femmes et nos enfants se fassent tripoter dans les transports en commun, nos petits vieux dépouiller par des individus peu scrupuleux (alors que les croquemorts s’en chargent très bien en toute légalité). Oui, d’accord, pour le petit voyou qui a volé une barre de céréales chez l’épicier du coin, on n’est peut-être pas obligé de le tuer tout de suite. On peut commencer par lui couper la main qui a commis le larcin, histoire de lui faire comprendre que ce n’est pas parce qu’on n’a pas d’argent pour s’acheter à manger qu’il faut se croire autorisé à dérober le bien d’autrui. Dans ce cas-là, on ne mange pas, voilà tout, ou alors, si on a vraiment très faim, on se trouve un travail honnête pour subvenir à ses besoins. Ce n’est pas le travail qui manque, vous savez. Bon, il est certain que compte tenu de son niveau d’études, le petit voyou en question ne va pas pouvoir prétendre à un salaire mirobolant. Mais il gagnera tout de même suffisamment pour s’acheter honnêtement une jolie petite pomme, et il la mangera avec d’autant plus de satisfaction qu’elle aura de la valeur pour lui. Si elle représente la moitié de son salaire, par exemple, vous pouvez être certain qu’il la mangera avec énormément de satisfaction, en savourant chaque bouchée jusqu’à l’extase. Alors que le riche, lui, le pauvre, ne prend plus aucun plaisir à manger une pomme, ou alors seulement si elle recouverte d’une feuille d’or et si on a pris soin de remplacer les pépins par des diamants. Comme le pauvre, le riche ne prend plaisir à manger une chose que si elle représente une partie non négligeable de son salaire, ce qui signifie qu’il n’est pas évident de se nourrir quand on a la malchance de gagner cent ou cent cinquante mille euros par mois. Même la truffe blanche ou le caviar à cinq mille balles le kilo, il faut déjà en avaler pas mal avant de commencer à ressentir un semblant de bien-être, une vague sensation de plaisir. D’autre part, le riche ne prend plaisir à être riche que s’il en permanence le référent de la misère sous les yeux, de même que le pauvre ne prend plaisir à l’être que s’il a en permanence le référent de l’abondance et la richesse sous les yeux. Et c’est exactement ce qu’on s’efforce de lui fournir, avec la presse people, les jeux débiles où des animateurs blindés de fric se foutent ouvertement de sa gueule, les centres commerciaux rutilants, les offres soi-disant spéciales et les supermarchés qui créent l’illusion de l’abondance en débordant de produits identiques dont seule la présentation diffère. Il se repaît des faits et gestes des riches et des puissants, se saigne aux quatre veines pour assister à des matchs de foot où les joueurs gagnent trois ou quatre mille fois le SMIC, et dans le même temps se plaint que la vie est trop chère et fonce sur les flics qui tentent de le contrôler alors qu’il roule bourré à cent à l’heure dans les rues de la ville. Il estime sans doute, en compensation de la vie de merde qu’il a courageusement accepté d’assumer, avoir droit à une certaine tolérance de la part de Nation reconnaissante. Le problème, c’est que la Nation ne lui reconnaît que le droit de se taire, estimant pour sa part que lui accorder le droit de vivre est déjà un privilège inestimable (mais un mal nécessaire à la survie du régime). Il se comporte d’autant plus mal qu’il n’a pas grand-chose à perdre, à part sa vie, à laquelle il ne tient pas plus que ça, ou sa liberté, qui ne lui sert pas à grand-chose s’il ne peut pas rouler bourré, insulter les gens et dégrader le bien d’autrui. On pourrait croire, quand il s’émerveille devant une voiture de luxe qu’il ne pourra jamais s’offrir, une star qu’il ne pourra jamais approcher, une actrice somptueuse qu’il ne pourra jamais soumettre à ses exigences sexuelles (et heureusement pour elle, la pauvre), que c’est de la convoitise. Mais non, ça le rend vraiment heureux et fier d’être pauvre, d’appartenir à la noble corporation des traîne-savates alcooliques et édentés (les fameux «sans-dents» du président Hollande, politicien médiocre mais humoriste réputé), et il ne changerait de catégorie sociale pour rien au monde. Si on lui donnait dix millions de dollars en petites coupures usagées dans un sac poubelle, il ne les prendrait pas. Enfin si, il les prendrait, mais il ne saurait pas quoi en faire, le pauvre. Autant donner de la confiture à un cochon. Il continuerait à boire de la piquette, mais il en boirait dix fois plus compte tenu de son budget illimité, et son espérance de vie, déjà aussi mince qu’une feuille de papier à cigarette (cigarettes qu’il fume par paquets entiers, de même qu’il s’adonne sans retenue à tous les vices à sa disposition), s’en trouverait réduite d’autant. Il essayerait bien de le placer à droite à gauche, sur les conseils avisés de quelque escroc en col blanc, de s’acheter une grosse voiture, une grosse bagnole, une grosse maison avec une grosse femme et des gros enfants, mais il se retrouverait vite entouré d’une nuée de parasites, contraint de se séparer de tous ses «amis» qui ne seraient en réalité qu’une bande de profiteurs de la pire espèce, totalement dénués de scrupules, et ne pourrait espérer aucun salut de la part des riches, lesquels continueraient éternellement à le considérer comme un corps étranger et opposer une fin de non-recevoir à ses tentatives de rapprochement. Sa vie, qui était déjà un pur calvaire, tournerait au cauchemar, car il se verrait contraint, après que sa femme ait demandé le divorce et tenté de le soulager de la moitié de sa fortune, de terminer ses jours dans la solitude la plus effroyable, passer ses journées à picoler des grands crus au bord de sa piscine et rouler sans but sur les hauteurs de Cannes au volant de sa Maserati flambant neuve. Quelle horreur ! La raison pour laquelle les riches, les vrais, les de pères en fils depuis des générations, n’ont que des amis riches, qui ne sont pas des amis, du reste, car il est bien entendu que le riche n’a pas d’ami, mais seulement des amis riches, autrement dit des amis qui n’en sont pas, au mieux des compagnons d’infortune et de perdition, hermétiques à la compassion mais rompus à toutes les subtilités du secret bancaire et la fiscalité paradisiaque, c’est précisément que les riches, qui le sont déjà par définition, et ce depuis longtemps, le plus longtemps possible de façon à ne laisser planer aucun doute sur le sujet, n’ont à priori aucune raison d’abuser d’eux. Le riche et le pauvre doivent être maintenus à bonne distance l’un de l’autre, ni trop proche ni trop éloignée, de sorte qu’ils puissent continuer à s’observer dans l’irrespect mutuel et l’aversion réciproque. C’est sur cet équilibre instable que repose l’avenir de la civilisation occidentale, dont on peut craindre, effectivement, qu’elle ne s’écroule à tout moment. Et j’ai envie de dire à mes amis développés, tant sur le plan de la morphologie, la technologie et la culture, très supérieurs à la moyenne dans leur approche de l’existence, mélomanes d’exception et lecteurs assidus des plus grands philosophes, qui, soudain pris de court par la déferlante qui menace de les engloutir, cherchent désespérément mon regard dans les ténèbres pour se raccrocher à quelque chose de profondément humain, sensible et empathique : oui, mes frères, c’est mal barré, mais tout n’est pas perdu. Grâce à l’oncle Sam, l’Armée rouge et le Soviet suprême, et même sans l’appui du gnome coréen à tête de poupon maléfique qui nous déteste cordialement, j’affirme sans une once de tremblement dans la voix que nous disposons de suffisamment d’ogives nucléaires pour maintenir à distance les hordes de pauvres et autres déshérités qui se pressent à nos portes. Disparaissez Marcheurs blancs, émissaires de la Mort, et laissez-nous jouir en toute liberté de nos Livrets A, LEP et PEL. Foutez-nous la paix et laissez-nous toucher en paix nos allocations familiales, revenus de solidarité active et autres primes de Noël. Non, vous ne toucherez pas à nos comptes en Suisse, pas plus qu’à nos épouses et encore moins nos filles, et n’espérez pas tremper un jour vos fesses ramollies dans nos piscines, vos mouillettes impies dans de jaune de nos œufs, ni vos lèvres perfides dans le champagne de nos coupes.

J’ai chargé Greg de faire le guet.

Comment j’ai fait ça ?

Rien de plus simple : je l’ai pris solennellement à part, dans un renfoncement de la cage d’escalier digne d’un palais des mille et une nuits, et, après lui avoir fait prêter allégeance à la cause et jurer fidélité à ma personne, je lui ai confié la lourde responsabilité de surveiller nos arrières, autrement dit me signaler immédiatement tout fait ou geste qui lui semblerait un tant soit peu suspect.

Quel grand moment d’émotion ! Je le vois encore fondre en larmes, tel un gros bébé au visage anguleux et la ceinture abdominale légèrement relâchée, l’exercice n’étant pas son fort, et s’écrouler comme une merde à mes pieds, bouleversé par l’immense honneur que je lui accordais. Je ne voudrais pas entrer dans les détails, mais il avait toujours souffert d’un manque de reconnaissance total de la part de son père, qui n’avait cessé de le dénigrer et lui faire sentir que, quoi qu’il fasse, il ne parviendrait jamais à se hisser à la hauteur de ses exigences. C’est dur pour fils de se faire traiter comme une sous-merde par son propre géniteur, celui auquel on pense devoir la vie (alors qu’il n’y est pour rien, en fait), c’est pas terrible pour la confiance, et il faut souvent des années de psychanalyse pour s’en remettre, et encore jamais complètement. Ce type était clairement une ordure qui n’aurait jamais dû avoir d’enfant. Seulement voilà, la nature se fiche que les ordures, assassins, nazis, trafiquants de drogue, politiciens corrompus et autres, se reproduisent au même titre que les honnêtes gens. Vous pouvez être le type le plus dégénéré qui soit, vous finirez toujours par trouver une fille aussi déglinguée que vous qui sera ravie d’offrir son ventre à vos ébats. La nature ne fait pas de différence entre ses enfants, tous ont les mêmes chances de se reproduire, et elle fait en sorte que chacun trouve chaussure à son pied, même s’il pue des pieds et s’il s’agit d’une vieille godasse trouée sortie d’une poubelle. Elle a donné à tous le pouvoir de niquer, le mode d’emploi et les outils pour le faire. Et elle a fait en sorte, histoire d’être bien sûre de ne pas rater son coup, que tous ne pensent qu’à ça vingt-quatre heures sur vingt-quatre, sans distinction de revenus, quotient intellectuel, familial ou autre. Et sans se balader avec la trique au vent vingt-quatre heures sur vingt-quatre, ce qui pourrait se révéler rapidement pénible et handicapant. D’où le coup de l’engin à géométrie variable, facile à ranger discrètement dans un fond de culotte pour ne pas donner l’alerte en permanence, créer un état de panique générale impossible à endiguer. La femme, par contre, avec ses seins non rétractables, dispose d’un handicap certain en la matière. Facile à repérer, il lui est d’autant plus difficile d’échapper à la concupiscence de ses concitoyens. On peut dire que la nature lui a joué un tour pendable. C’est la fable du pot de miel et du saumon. Imaginez qu’on vous dise : Tu vois cette forêt ? Elle est infestée d’ours bruns, des gros qui n’ont qu’une seule idée en tête : bouffer. Et maintenant, tu vois cette table ? Dessus, il y a un pot de miel et un saumon. Tu choisis l’un ou l’autre, ce que tu préfères, tu le poses en équilibre sur ta tête, et tu vas te balader au milieu des ours. Il est clair que vous avez toutes les chances de vous faire arracher la tête à tous les coins d’arbres. Depuis, la vie de la femme est un véritable cauchemar, un parcours du combattant qui force l’admiration. À moins de s’emmailloter sous trente ou quarante couches de fringues, de dissimuler son visage et l’abondante chevelure soyeuse et parfumée dont elle est naturellement pourvue, elle est condamnée à vivre dans un monde où ses chances de survie, sexuellement parlant, flirtent avec le zéro absolu. Les Arabes, qui l’obligent à se couvrir de la tête aux pieds avant de mettre le nez dehors, ne laissant la place qu’à une ouverture grillagée ou une fente minuscule pour lui permettre de respirer et voir sans être vue, sont en fait d’ardents protecteurs de la femme contre leur propre tyrannie. Non seulement ils reconnaissent implicitement qu’ils sont tous des obsédés sexuels incapables de contrôler leurs pulsions, ce qui est déjà une belle preuve de courage et d’abnégation, de recul sur soi (même s’il ne s’accompagne pas nécessairement d’un sens de l’humour à toute épreuve), mais ils lui indiquent (de façon assez autoritaire, il est vrai) le moyen de se prémunir contre leurs ardeurs, sachant qu’eux-mêmes, à moins de se crever les yeux, sont incapables d’avoir une approche saine et dépassionnée de la plastique féminine. Un banc de morues ne peut pas évoluer sans protection au milieu des récifs. Les requins sont partout, affamés (ou pas, du reste, car le requin a toujours faim, qu’il ait déjà mangé ou non), et la moindre écaille qui scintille à l’horizon les plonge aussitôt dans un état de frénésie proche de la démence. Même s’il n’y a pas énormément de requins dans le golfe Persique, et si la morue n’est pas le plat favori des Iraniens, nos amis Arabes, nobles descendants du royaume de Saba et des Omeyyades, ont compris la nécessité de protéger la femme de la violence de leur désir. Pas question pour eux d’envoyer des filles en hot-pants de cheerleaders se trémousser au milieu d’une caravane de bédouins qui viennent de traverser le désert à dos de chameau, des gens dont les burnes, sous l’effet de la chaleur et des chocs répétés, ont atteint un tel degré de dilatation qu’elles sont susceptibles d’exploser à tout moment, comme les graines du cornichon d’âne ou de l’herbe à Robert qu’on effleure par inadvertance. Ce n’est que dans la plus stricte intimité, à l’abri des regards concupiscents de la meute en chaleur, que la femme peut se dévoiler. L’efficacité est indéniable, mais on peut critiquer la méthode, qui est aux antipodes de notre façon de procéder. En effet, nous avons décidé de traiter le mal par le mal. Puisque nous sommes tous des gros porcs lubriques incapables de contrôler nos pulsions, autant y aller à fond et affronter nos démons à bras-le-corps. Le danger, en dissimulant la femme, c’est que cette occultation ne fasse qu’exacerber les passions et entraîner l’imaginaire vers des horizons aussi dangereux qu’insoupçonnés. Si vous sentez qu’on vous cache quelque chose, ou cherche à vous le cacher, votre envie de savoir est d’autant plus vive, intense, et peut rapidement virer à l’obsession. Vous imaginez des choses qui ne sont pas conformes à la réalité et courez droit à la déception, la frustration. Sauf, bien sûr, si vous vous gardez de toute extrapolation intempestive et conservez intact votre sens du merveilleux. Mais cela n’est pas donné à tout le monde. Pragmatiques, nous avons décidé qu’il valait mieux jouer la carte de la transparence : aux premiers rayons de soleil (oui, il ne fait pas toujours une chaleur à crever dans nos contrées) sur fond de ciel bleu dégagé et d’oiseaux qui chantent dans les arbres à la végétation naissante, la femme, à l’instar de l’homme, pourra elle aussi se trimballer à moitié nue dans les rues de la cité, confiante dans le fait que nous saurons garder nos distances et éviter regards et réflexions salaces concernant tout ou partie de son anatomie. Ne sommes-nous pas des gens civilisés, qui n’avons nul besoin de recourir à de grossiers stratagèmes pour conserver élégance et dignité ? Après tout, les Africains arrivent très bien, dans leurs contrées lointaines, à vivre en permanence entourés de femmes largement dévêtues sans développer aucun ressentiment particulier. Tout se passe très bien pour eux, ils s’accouplent quand bon leur semble, de la façon qui leur plaît, et personne ne trouve rien à redire à la situation. C’est bien la preuve, si on n’est pas complètement con, qu’on n’est pas obligé de couvrir sa femme de la tête aux pieds pour éviter les problèmes. De toute façon, quoi qu’on fasse et qui qu’on fesse, il y aura toujours des emmerdeurs pour s’affranchir des règles et traverser en dehors des clous.

J’ai relevé Greg, l’ai serré fort dans mes bras, tel un père qui verrait son enfant pour la dernière fois, ou une maman ours qui serrerait dans ses bras son bébé ours avant de le laisser partir seul dans les profondeurs de la forêt infestée de méchants chasseurs alcooliques et réactionnaires, puis me suis dirigé d’un pas lourd mais décidé vers la Réception.

Nous étions, je le rappelle, en territoire ennemi. Le coup pouvait venir de n’importe où, n’importe quand, et se présenter sous n’importe quelle forme, y compris la plus innocente, comme ces enfants qui surgissent dans le djebel et s’avancent vers vous avec une bombe à la ceinture, le sourire aux lèvres et les bras chargés de dattes. Je te raconte pas le méchoui, sidi. Alors oui, je sais ce que vous allez me dire : pas la peine faire d’en faire des caisses, de sombrer dans le délire paranoïaque des références au djihad et à l’apocalypse selon Saint-Maclou ! On se calme et on boit frais à Saint-Tropez, comme disait le regretté Max Pécas, petit producteur de navets connu pour la qualité de ses bulbes (La Baie du désir, Je suis une nymphomane, Marche pas sur mes lacets, Mieux vaut être riche et bien portant que fauché et mal foutu, etc). Vous êtes juste deux touristes mal réveillés dans le hall d’un hôtel de luxe, en plein jour, armés jusqu’aux dents qui plus est, on ne voit donc pas très bien ce qui pourrait vous arriver, même s’il est clair que vous n’êtes pas à proprement parler les bienvenus dans le contexte, compte tenu de la connerie qui vous anime et la prétention abyssale qui vous caractérise, sans parler de la couleur de peau blafarde et grassouillette d’asticot avec laquelle vous avez eu le toupet de venir au monde. Certes, ce n’est pas de gaité de cœur qu’on vous voit fouler le plancher ancestral de cet édifice prestigieux, mais ce n’est pas comme si vous étiez deux chérubins prépubères égarés dans un congrès de pédophiles, ou, pire encore, deux Juifs handicapés, homosexuels et communistes ayant atterri, suite à une cascade d’événements tous plus rocambolesques les uns que les autres, en plein milieu d’une réunion de nostalgiques du Troisième Reich.

C’est donc plein d’espoir et la poitrine gonflée d’orgueil que Greg, rasséréné par les propos lénifiants (j’aimerais, au risque d’altérer la fluidité de la narration, avoir une pensée émue pour le jeune adulte dont le niveau de vocabulaire - et de culture générale - n’excède pas celui d’un enfant de cinq ans d’il y a cinquante ans et qui, s’accrochant à chaque phrase tel un naufragé à une planche de bois vermoulu, essaie malgré tout courageusement de lire ce livre, au demeurant moins hermétique que Finnegans Wake, Le Roi pâle ou La Maison des feuilles, et lui dire que non, tout n’est pas perdu, à condition bien sûr qu’il cesse immédiatement d’ingurgiter des torrents de soupe indigeste avec des gros morceaux de caca qui pue à l’intérieur sur X, TikTok, WhatsApp, Instagram et les autres, sans quoi il parviendra à un niveau - level, je traduis en anglais par charité chrétienne, pour l’aider à ne pas décrocher totalement - de décérébration si stratosphérique que même ses propres enfants, qu’il aura bien évidemment achetés en solde sur Internet et renvoyés plusieurs fois pour vice de forme, au risque de se retrouver avec des articles ne correspondant plus du tout à ses attentes, inexistantes de toute façon, ne le reconnaîtront plus) que je venais (je vous avais prévenu qu’on allait perdre en lisibilité) de lui déverser à flux tendu dans le creux de l’oreille, s’est élancé, tel un jeune faon qui s’éveille dans la forêt enchantée de ses ancêtres, bercé de mille senteurs et sonorités à la fois étranges et familières qui l’émerveillent autant qu’elles l’interloquent, sur le sentier lumineux (rien à voir avec le parti du camarade Gonzalo, je vous rassure tout de suite) de l’épanouissement personnel, respirant à pleins poumons l’air nouveau de la confiance retrouvée.

Après quoi je me suis présenté à l’accueil, arborant le sourire satisfait du voyageur qui vient d’effectuer une confortable traversée de l’Atlantique en jet privé, entouré d’un essaim d’hôtesses vibrionnantes laissant flotter dans leur sillage un subtil bouquet d’exhalaisons paradisiaques, et j’ai attendu que Dumo, toujours très affairé à son travail, daigne s’intéresser à moi.

Ce qu’il n’a pas fait, manquant à tous ses devoirs avec une impudence désarmante de naturel.

Mais aussi terriblement irritante.

Du coup, ma pression artérielle est montée en flèche.

Et il m’arrive, quand ma pression artérielle monte en flèche, de faire des choses que je ne fais pas en temps normal, ou nettement

moins, comme par exemple tatouer mes initiales au fer rouge sur la poitrine des gens, leur arracher les yeux et les remplacer par des balles de ping pong, leur couper le nez, la langue et les oreilles avec des ciseaux rouillés, ou encore leur ouvrir la boîte crânienne à l’emporte-pièce pour leur siroter la cervelle à la paille.

Voilà comment je me suis retrouvé dans un état d’exaspération que je ne me souvenais plus avoir atteint depuis le jour où Zarina, après avoir englouti un nombre indéterminé de Girofliers du Clair de Lune (cocktail dévastateur à base d’amaretto, grappa, sirop de framboise et clou de girofle), m’avait sauvagement agressé sous le prétexte fallacieux que j’aurais, je dis bien «aurais», car je ne me souviens absolument pas de l’avoir fait, approché sa sœur d’un peu trop près. Accusation profondément injuste, perfide et mensongère, dont je peine aujourd’hui encore, longtemps après les faits, à me relever (mais ça va, hein, j’ai suivi une thérapie assez musclée à base d’électrochocs et produits stupéfiants qui me permet enfin d’entrevoir le bout du tunnel, même si je passe encore par des phases de mélancolie semblables à d’épaisses nappes de brume qui colle à la peau et s’insinue au plus profond de votre être). Car tenez-vous bien, à en croire les dires de l’écervelée, nos lèvres (les miennes et celles de Tosca, sa sœur jumelle, en tout point semblables aux siennes, hormis peut-être une imperceptible inflexion à la commissure, dessinant, en de très rares occasions, une microscopique et fugace fossette en sortie de joue) se seraient pratiquement abouchées au cours d’un échange passionné concernant les relations pour le moins sulfureuses du roi Zog d’Albanie avec une certaine Tatiana Visirova, subtil mélange d’épices chinoises, roumaines et russo-polonaises savamment assemblées en une seule et même petite personne passablement dévergondée, assez peu douée pour les études, mais suffisamment à l’aise pour se produire dans le plus simple appareil sur la scène des Folies-Bergère et déclencher des tonnerres d’applaudissements à chacune de ses apparitions.

J’ai pris sur moi pour ne pas sortir Manu et exploser le crâne bosselé du crétin des Alpes qui me faisait face.

Vers la fin du XVIIIe siècle, à l’aube des sports d’hiver et la révolution technologique, les premiers touristes parviennent au sommet des monts alpins et découvrent avec stupeur qu’ils ne sont pas seuls. D’étranges créatures, simiesques, contrefaites, et affligées pour la plupart de goitres spectaculaires, vivent dans ces contrées reculées. Il devient rapidement évident que ces créatures, en dépit de leurs malformations et leur intelligence limitée, ne représentent aucun danger particulier. De retour en ville, ils écument les salons pour faire part de leur découverte, ne lésinant pas sur les détails les plus atroces, n’hésitant pas à forcer le trait pour les besoins de la narration.  Le bourgeois frissonne, les jeunes filles se pâment et tombent inanimées dans les bras des beaux parleurs, leurs lèvres entrouvertes exhalant le souffle rauque des désirs inassouvis. En secret, elles rêvent d’étreintes avec ces monstres innommables qu’on imagine membrés comme des taureaux, portant jusqu’aux genoux de lourdes paires de burnes couvertes d’une épaisse fourrure. Qui sont ces humanoïdes ? Sont-ils les descendants des Nains d’Erebor, dont ils ont la taille réduite, les traits grossiers, la silhouette massive et la force de cheval ? Sont-ils consanguins, cannibales, sodomites ? Quelles sont leurs idoles ? S’adonnent-ils à des rites païens placés sous le signe de la nécrophilie et la zoophilie ? On envisage un temps de les exterminer, ou les réduire à l’esclavage, les faire travailler à des tâches ingrates comme des bêtes de somme, puis on choisit de les parquer dans des sanatoriums pour les étudier à loisir et tenter de percer le mystère de leur existence et leur laideur extrême. L’armée envisage un temps de transformer les mâles en super-combattants pour les envoyer sur le front en première ligne en cas de conflit. Leur aspect menaçant devrait suffire à pousser l’ennemi à rebrousser chemin sans demander son reste. Quelques années plus tard, la science livre son verdict : non, il ne s’agit pas d’une espèce d’hommes semi-préhistoriques ayant miraculeusement échappé à la civilisation, et encore moins de créatures fantastiques sorties du ventre de villageoises engrossées par des trolls, des elfes ou des lutins, mais de pauvres hères souffrant de la thyroïde en raison du manque d’iode. Peu festif mais vrai. Grâce aux progrès de la médecine, le crétin des Alpes a aujourd’hui disparu, même s’il y a toujours autant de crétins dans les Alpes que partout ailleurs.

Je me suis raclé le fond de la gorge, et Dumo, le crétin des Alpes au crâne bosselé trempé de sueur, a enfin daigné jeter un œil sur mon humble personne.

Il a dit : Bonjour monsieur.

J’ai répondu : Bonjour Dumo.

Il a ajouté : Que puis-je faire pour votre service ?

J’ai immédiatement reconnu le son de sa voix : c’était celle que j’avais entendu cette nuit dans l’interphone, quand notre joyeuse petite bande était venue sonner à la porte du Caribbean.

L’accueil n’avait pas été des plus chaleureux, et ne l’était pas davantage maintenant.

J’ai demandé : Vous me reconnaissez ?

\textsc{Lui} : Pas le moins du monde, monsieur. Vous désirez une chambre ?

\textsc{Moi} : Oui.

\textsc{Lui} : Désolé, monsieur, mais l’hôtel est complet.

\textsc{Moi} : Ça m’étonnerait.

\textsc{Lui} : Vraiment, monsieur ?

\textsc{Moi} : C’est à vous que j’ai eu affaire cette nuit, n’est-ce pas ?

\textsc{Lui} : Cette nuit, monsieur ? Ça m’étonnerait, je n’étais pas de service.

\textsc{Moi} : Vous l’étiez.

\textsc{Lui} : Vous croyez ?

\textsc{Moi} : J’en suis certain. Tout comme je suis certain que vous vous rappelez très bien de moi.

\textsc{Lui} : Si vous le dites.

\textsc{Moi} : Donc, vous savez qui je suis.

\textsc{Lui} : Pas le moins du monde, monsieur. Mais la première chose qu’on vous apprend, à l’école hôtelière, c’est de ne jamais contredire le client, quelles que soient les inepties qu’il profère.

\textsc{Moi} : Dans ce cas, on doit vous apprendre aussi qu’il ne faut jamais contredire un policier dans l’exercice de ses fonctions.

\textsc{Lui} : Monsieur est policier ?

\textsc{Moi} : Commissaire Beauvais, de la police judiciaire.

\textsc{Lui} : Policier ou pas, je vous assure que l’hôtel est complet.

\textsc{Moi} : Je ne pense pas, non.

\textsc{Lui} : C’est votre droit le plus strict.

\textsc{Moi} : Et je sais très bien que c’est à vous que j’ai parlé cette nuit. Vous m’avez parlé de Gustav Nachtigal, Bismarck, Göring, et expliqué que cet hôtel est officiellement rattaché à l’ambassade de Namibie et interdit aux Blancs.

\textsc{Lui} : Exact. Et il me semble aussi vous avoir demandé de revenir avec un mandat en bonne et due forme. Vous l’avez ?

\textsc{Moi} : Non.

\textsc{Lui} : Dans ce cas, je ne suis pas certain de pouvoir faire grand-chose pour vous. D’autant que je suis très occupé, comme vous pouvez le constater.

\textsc{Moi} : Vous m’avez également affirmé être un métis du Cap, fils d’une mère française et d’un héros de la bataille de Salt River. Au cours de cette bataille, restée légendaire dans la longue et douloureuse histoire de l’Afrique du Sud, les indigènes ont vaincu les Portugais et forgé leur réputation de courage et de férocité.

\textsc{Lui} : Je n’ai rien d’autre à rajouter.

\textsc{Moi} : Je ne suis pas venu en ennemi. Il se trouve que je suis à la recherche d’un collègue et ami d’enfance, et que j’ai de bonnes raisons de penser qu’il est venu ici cette nuit.

Lui, passant la main sur son crâne trempé de sueur avant de l’essuyer sur sa bedaine : Je vous l’ai dit, cet hôtel est interdit aux Blancs.

\textsc{Moi} : Mon ami est Noir, descendant en droite ligne des King’s American Dragoons du colonel Benjamin Thompson, comte de Rumford, membre de la Royal Society et auteur d’un remarquable Essai sur les cheminées, avec des propositions pour les améliorer, les empêcher efficacement de fumer, économiser le combustible, et rendre les habitations plus confortables et salubres.

\textsc{Lui} : Intéressant.

\textsc{Moi} : C’est comme je vous le dis.

\textsc{Lui} : Dans ce cas, il est peut-être venu ici. Je peux vérifier dans le registre, si vous me dites son nom.

\textsc{Moi} : Titus Beaugendre, comme un beau gendre.

J’ai sorti mon téléphone et affiché une photo de Titus, prise dans son jardin alors qu’on était en train de faire griller des steaks par une belle soirée d’été.

\textsc{Moi} : Tenez, c’est lui.

L’autre, jetant un coup d’œil à la photo : Ça ne me dit rien. Attendez, je consulte le registre.

Dans la foulée, j’ai affiché la photo de la Gardienne de la Nuit, prise à son insu peu de temps avant qu’elle disparaisse avec Titus dans les profondeurs de la nuit : J’ai tout lieu de penser qu’il se trouvait en compagnie de cette personne.

Lui, levant le nez de son registre : Ça ne me dit rien non plus. Vous connaissez son nom ?

\textsc{Moi} : Elle a dit s’appeler Atiena, et exercer la profession de gardienne de la nuit.

\textsc{Lui} : Gardienne de la nuit, dites-vous ?

\textsc{Moi} : Oui.

\textsc{Lui} : Drôle de métier.

Lui toujours, m’arrachant presque le téléphone des mains : Permettez !

\textsc{Moi} : Je vous en prie.

Lui, fixant la photo avec attention, le visage de plus en plus déformé par la perplexité : On dirait…

\textsc{Moi} : On dirait ?

\textsc{Lui} : On dirait Repentance.

\textsc{Moi} : Repentance ?

\textsc{Lui} : Repentance Whittingham. Drôle de nom, n’est-ce pas ?

\textsc{Moi} : Pour le moins. D’origine anglaise, je suppose ?

\textsc{Lui} : Aucune idée. Tout ce que je sais, c’est qu’elle travaille ici comme femme de chambre. Une très belle femme, du reste.

\textsc{Moi} : Vous en êtes certain ?

\textsc{Moi} : Que c’est une très belle femme ? Oui, pour autant que je puisse en juger. Après, c’est une question de goût, bien sûr.

\textsc{Moi} : Non. Je veux dire : vous êtes certain qu’elle travaille comme femme de chambre dans cet établissement ?

Lui, me restituant le téléphone et retournant à son registre : Oui, si c’est bien Repentance. Mais j’avoue que la ressemblance est assez troublante. Par contre, je ne vois aucune trace d’un quelconque Titus Beaugendre.

\textsc{Moi} : Il a pu donner un faux nom.

Lui, refermant le registre et plongeant avec vigueur ses yeux légèrement globuleux dans les miens, comme s’il cherchait à m’hypnotiser : C’est possible. Mais à moins qu’il soit venu déguisé, il me semble que je reconnaîtrais son visage.

Moi, incapable de détacher mon regard du sien : Vous m’avez dit que vous n’étiez pas de service cette nuit.

J’ai senti ma gorge se nouer, comme si des mains invisibles étaient en train de m’étrangler.

\textsc{Lui} : J’ai dit ça ?

Moi, avec la désagréable impression de me tenir en équilibre sur des jambes en caoutchouc : Vous l’avez dit.

\textsc{Lui} : Cette nuit, vous dites ?

\textsc{Moi} : Excusez-moi, mais je ne me sens pas très bien.

\textsc{Lui} : Vous dites que je n’étais pas de service cette nuit ?

Moi, transpirant à grosses gouttes : Il fait une chaleur à crever, ici, vous ne trouvez pas. Vous n’auriez pas un mouchoir à me prêter, par hasard ?

Il pouvait difficilement refuser, vu qu’il en avait des tonnes.

\textsc{Lui} : Mais si, bien sûr.

\textsc{Moi} : Merci, c’est très aimable à vous.

\textsc{Lui} : Mais je vous en prie, c’est tout naturel.

J’ai détaché mon regard du sien, me suis épongé le front, et j’ai commencé à me sentir un peu mieux.

\textsc{Lui} : Ça va ?

\textsc{Moi} : Mieux, merci.

\textsc{Lui} : Cette nuit, disiez-vous ?

\textsc{Moi} : Quoi, cette nuit ?

\textsc{Lui} : Vous disiez que je n’étais pas de service cette nuit ?

J’ai senti qu’il tentait à nouveau d’exercer ses talents de prêtre vaudou à mon encontre, mais j’ai réussi à parer le coup.

\textsc{Moi} : Non, c’est vous qui l’avez dit.

\textsc{Lui} : J’ai dit ça ?

\textsc{Moi} : Oui, vous l’avez dit.

Une lueur de désagrément s’est baladée sur son visage, tel un mille-pattes qui court le long des murs à la tombée du jour.

\textsc{Lui} : Dans ce cas, je me suis trompé. Ce n’est pas cette nuit que je n’étais pas de service, mais celle d’avant. Nous avons des horaires très irréguliers, il n’est pas toujours facile de s’y retrouver.

Je commençais à reprendre du poil de la bête : Je vois. Et cette Repentance Machin-Chose, elle est de service aujourd’hui ?

Lui, affichant un sourire aussi avenant qu’un banc de piranhas dans les eaux troubles de l’Amazone : Whittingham, monsieur. Une très jolie femme, avec des yeux d’une couleur assez indéfinissable.

\textsc{Moi} : Un vert très pâle, si je ne m’abuse.

\textsc{Lui} : Je ne saurais dire exactement. L’hôtel compte près d’une centaine d’employés, vous comprendrez que je ne peux pas me souvenir de la couleur des yeux de chacun d’eux.

Moi, de nouveau en pleine possession de mes moyens : Je comprends admirablement. Mais vous n’avez pas répondu à ma question.

Lui, exhibant une rangée de crocs qui n’auraient pas dépareillé dans le bec d’un requin-bouledogue : Et pourquoi le ferais-je ? Vous avez un mandat, une commission rogatoire ? je suis accusé de quelque chose ?

\textsc{Moi} : Rien de tout ça, rassurez-vous. Il s’agit juste d’une petite discussion entre gens de bonne compagnie, tout ce qu’il y a de plus informelle.

\textsc{Lui} : C’était quoi, déjà, votre question ?

\textsc{Moi} : Est-ce que cette Repentance Whit-je-sais-plus-quoi est de service aujourd’hui ?

\textsc{Lui} : Whittingham, monsieur. «La colonie du peuple blanc», en anglais.

\textsc{Moi} : Vous avez fait des études de langues ?

\textsc{Lui} : Disons que j’en parle un certain nombre.

\textsc{Moi} : Et donc ?

\textsc{Lui} : Donc rien. La compagnie cherchait un réceptionniste polyglotte, elle a fait appel à mes services.

\textsc{Moi} : Je parlais de Repentance Whittingham.

\textsc{Lui} : Très belle femme.

\textsc{Moi} : Oui. Mais elle est de service, aujourd’hui ?

\textsc{Lui} : Je ne sais pas.

\textsc{Moi} : Vous avez sans doute l’emploi du temps des employés.

\textsc{Lui} : Oui, je l’ai, en effet. Mais je ne suis pas certain d’avoir envie de le consulter. D’ailleurs, je ne suis pas là pour ça. Je suis là pour accueillir les gens et faire en sorte qu’ils passent un agréable séjour dans notre établissement.

\textsc{Moi} : Vous m’obligeriez.

\textsc{Lui} : Vous vouliez savoir si votre ami était ici, il ne l’est pas, je crois que nous avons fait le tour de la question. Si vous voulez réserver une chambre, pour vous ou quelqu’un d’autre, il y a un petit hôtel très sympathique à deux rues d’ici. Je peux vous donner son nom et son adresse, si vous le désirez.

\textsc{Moi} : Pourquoi j’ai l’impression que vous vous foutez de ma gueule ?

Lui, tout sourire : Je ne me permettrais pas, monsieur le commissaire.

Une des choses qui m’auraient fait le plus plaisir, à ce moment-là, aurait été d’être pilote de chasse, que des touristes français (venus massacrer des lions, des éléphants et des rhinos) soient sauvagement assassinés et décapités par des bushmen dans le désert du Kalahari, et que la guerre soit déclarée à la Namibie par un président de la République aux abois, en totale perdition, disposant d’une cote de popularité frisant le ridicule, suspecté de corruption passive, fraude fiscale, trafic de drogue, proxénétisme aggravé, prise illégale d’intérêts, détournement de fonds publics et association de malfaiteurs, prêt à tout, y compris rayer un pays de la carte, pour redorer son blason aux frais de la princesse. J’aurais quitté les lieux sans dire un mot, rejoint ma base, serais monté dans mon Rafale et revenu larguer tout mon stock de bombes et de missiles sur le Caribbean Hôtel, ne laissant à sa place qu’un cratère fumant jonché de gravats et de cadavres entremêlés.

Greg, que je n’avais pas entendu arriver, s’est pointé derrière moi et m’a donné quelques coups de coude dans le dos pour me signaler sa présence et son intention de s’entretenir avec moi.

\textsc{Moi} : Quoi, qu’est-ce qui se passe ?

Il a alors attiré mon attention, de façon non verbale, en le pointant du doigt, sur un fait qui venait de se produire et pouvait se révéler d’une importance capitale pour la suite de notre enquête : une porte d’ascenseur venait de s’ouvrir, non loin d’une girafe en train de brouter les feuilles en plastique d’un arbre factice, et de cette ouverture venait de surgir ce qui ressemblait comme deux gouttes d’eau à une femme de chambre poussant un chariot adapté à l’exercice de ses fonctions.

Croiser une femme de chambre est chose assez courante dans un hôtel, je vous le concède, mais ce qui la rendait particulièrement intéressante dans le cas présent, c’était que la femme de chambre en question ressemblait elle-même comme deux gouttes d’eau à Repentance Whittingham, alias Atiena, la Gardienne de mes deux.

Repentance semblait parfaitement détendue, comme une fille à qui tout sourit dans la vie et qui n’a à priori aucune raison de s’inquiéter pour son avenir.

Je dis bien «à priori», car toute personne qui a quelque chose à se reprocher et croise mon chemin a du souci à se faire pour son avenir. D’ailleurs, toute personne qui croise mon chemin, même si elle n’a rien à se reprocher, a du souci à se faire pour son avenir. Je peux très bien, sur un coup de tête, parce que je me suis levé du pied gauche ou de la couille droite, parce qu’il tombe sur le monde un fin crachin qui m’indispose, parce que ma femme me trompe avec mon chien, parce que j’ai écumé sans succès toute ma garde-robe à la recherche d’une paire de chaussettes qui ne soit pas trouée au niveau du gros orteil, pour tout un tas de raisons passant par tous les stades de l’insignifiance, décider de lui loger une balle de crâne, avant d’exécuter tous les témoins de la scène (ce qui a chaque fois constitue une nouvelle scène qui peut avoir de nouveaux témoins, et ainsi de suite, de sorte qu’il faut parfois éliminer tout un quartier, voire une ville entière et une bonne partie de la campagne environnante, pour se débarrasser enfin du dernier témoin, ou, si je suis dans un bon jour, ce qui n’arrive pratiquement jamais, lui laisser la vie après lui avoir coupé la langue et crevé les yeux et les tympans pour m’assurer de son innocuité, ce qui représente un surcroît de travail non négligeable pour finalement pas grand-chose, sachant que la plupart des gens préfèrent largement être morts plutôt que vivre dans les conditions épouvantables que je viens de décrire, sans voir ni entendre quoi que ce soit, ni être en capacité de prononcer le moindre mot, même s’il n’y a pas grand-chose à dire quand on en est réduit à un tel degré d’infirmité), regagner tranquillement mon domicile avec mes fringues couvertes de sang, me faire couler un bain, mettre un CD de Funki Porcini (The Mulberry Files, par exemple, ou At The Edge Of The World), me glisser dans l’eau tel un reptile sournois, une larve machiavélique, un cafard dans le plus simple appareil, et me laisser glisser sans résistance dans la tiédeur parfumée de l’oubli en sirotant un grand verre de cognac.

\textsc{Greg} : Tu vois ce que je vois ?

\textsc{Moi} : Oui, c’est Repentance Whittingham.

\textsc{Lui} : Qui ça ?

\textsc{Moi} : Atiena, la Gardienne de l’ennui. Son vrai nom, c’est Repentance Whittingham.

\textsc{Lui} : Comment tu sais ça ?

\textsc{Moi} : C’est Dumo qui me l’a dit.

\textsc{Lui} : Qui ça ?

\textsc{Moi} : Dumo, le réceptionniste.

\textsc{Lui} : On fait quoi ?

\textsc{Moi} : On va juste lui poser quelques questions, gentiment, poliment, comme des gens bien éduqués, et on verra bien ce qu’il en ressort.

\textsc{Lui} : Et s’il n’en ressort rien ?

\textsc{Moi} : On surveille l’entrée, on attend qu’elle sorte, on l’embarque et on l’emmène au labo pour la cuisiner à l’abri des regards indiscrets.

Greg, outré : Mais c’est parfaitement illégal.

\textsc{Moi} : En effet.

\textsc{Lui} : C’est contraire aux droits de l’homme, la femme, et tout ce qui s’ensuit. C’est très grave.

\textsc{Moi} : Si on s’en tient au code pénal, il s’agit effectivement d’un enlèvement passible d’une lourde peine de prison, surtout s’il s’accompagne d’actes de torture et de barbarie.

\textsc{Greg} : Désolé, mais je ne mange pas de ce pain-là. Ma réputation et mon honneur sont en jeu, sans parler de ma licence professionnelle.

\textsc{Moi} : Ne t’en fais pas, j’assume l’entière responsabilité de la mission. En cas de pépin, il te suffira de nier avoir eu connaissance de mes agissements, j’abonderai dans ton sens et tu t’en sortiras sans une égratignure. Mais trêve de bavardage. Cet endroit est une véritable jungle, nous ne sommes pas les bienvenus, et le personnel fera tout ce qui est en son pouvoir pour nous mettre des bâtons dans les roues, voire nous éliminer et faire disparaître nos corps dans les profondeurs de l’hôtel. Il faut agir vite, pendant que la cible est encore en vue, sinon on va perdre sa trace au détour d’un couloir. Elle a l’avantage du terrain, qu’elle arpente depuis de longues années et dont elle connaît chaque recoin par cœur.

\textsc{Greg} : On n’en sait rien.

\textsc{Moi} : Quoi ?

\textsc{Lui} : Si elle travaille ici depuis de longues années.

\textsc{Moi} : Non, mais on peut le supposer. On doit le faire, même, car il ne faut jamais sous-estimer l’adversaire. Allons-y !

Au moment où je prononçais ces mots, le regard de Repentance Whittingham a croisé le mien.

Il ne lui a pas fallu plus d’une demi-seconde pour me reconnaître et comprendre que ma présence en ces lieux n’était pas forcément la meilleure nouvelle de l’année.

Elle a lâché son chariot et s’est précipitée vers l’entrée avec une telle célérité (je ne saurais la quantifier au juste, mais je pense qu’on ne devait pas être loin de celle de la sonde Parker Solar Probe soumise à l’influence gravitationnelle de Vénus) que Greg et moi en sommes restés comme deux ronds de flan, bras ballants, yeux écarquillés et bouche grande ouverte, l’air si profondément idiot qu’on n’a pas été étonnés de voir tout le personnel présent éclater de rire et se foutre ouvertement de notre gueule.

Le temps de reprendre nos esprits, ravaler notre rancœur et prendre sur nous pour ne pas dégommer tout le monde à coups de flingues, et on s’est lancés à la poursuite de la fugitive.

Repentance Whittingham s’est engouffrée dans une Cooper S garée un peu plus loin, le genre de petite bombe qui permet de se faire la malle en vitesse quand on n’a pas la conscience tranquille, horriblement envie de pisser, ou simplement de se faire plaisir sur le bitume quand on est amateur de sensations fortes. Vous en connaissez beaucoup, vous, des femmes de chambre qui roulent en Cooper S ? Moi pas. On est déjà sur de la femme de chambre de qualité supérieure, haut de gamme, pur porc, élevée au grain sous la mer (d’où ces petits arômes iodés fleurant bon la moule à marée basse qui font la joie des gastronomes en culotte courte, en slip, ou même à poil pour les plus intrépides), ou alors une femme de chambre qui, en plus de passer une partie de son temps à faire des lits et récurer des chiottes, se livre à d’autres activités pour arrondir ses fins de mois (ce qui n’est pas une fin en soi, ni en soie, les humoristes apprécieront). À moins, bien sûr, qu’il ne s’agisse d’une femme qui a ou vient d’hériter d’une somme d’argent assez conséquente, ou de gagner au loto, et continue néanmoins à exercer sa profession parce qu’il s’agit pour elle non pas de quelque chose d’essentiellement alimentaire, avant tout destiné à la nourrir elle et le gamin trisomique qu’elle élève seule depuis le décès par arme à feu de son conjoint alcoolique et violent, mais d’une authentique passion qu’elle n’entend pas abandonner, même pour tout l’or du monde.

Repentance Whittingham a fait rugir le moteur de son petit bolide, pas très content d’être réveillé en sursaut (le genre de moteur qui n’est pas du matin, préfère largement s’exprimer à la tombée de la nuit sur les grands boulevards ou les petites routes de campagne), puis s’est enfuie en laissant dans son sillage des traces de gomme brûlée additionnées d’un épais nuage de gaz d’échappement, riches, comme chacun le sait, en monoxyde de carbone, dioxyde d’azote et autres hydrocarbures aromatiques polycycliques très mauvais pour la santé. Mais la petite peste se foutait royalement que les gens s’intoxiquent en respirant ses déjections. Tout ce qui l’intéressait, c’était de rouler comme une dingue dans les rues de la ville, sans se soucier un instant des dommages collatéraux, vies brisées et familles en pleurs que sa conduite irresponsable pouvait occasionner.

Si vous voulez mon avis, Repentance Whittingham n’était pas encore prête à se repentir, implorer la clémence des instances supérieures de l’univers, demander pardon à qui que ce soit, ni même éprouver le moindre soupçon d’un commencement de début de vague regret pour toutes les mauvaises actions dont elle s’était rendue coupable au cours de son existence entièrement dévolue à la domination d’autrui et sa soumission sans réserve à ses exigences les plus dégradantes, injustes, obscènes et farfelues.

De mon côté, après m’être quelque peu pris les pieds dans le tapis et avoir bien failli m’empaler sur la corne du buffle qui trônait à l’entrée, je déployais à présent l’ensemble de mes facultés motrices et capacités énergétiques pour rejoindre mon véhicule au plus vite.

J’étais plus ou moins dans les temps, jusqu’au moment où Greg a mis le pied sur une crotte de chien et ripé en beauté avant de s’étaler de tout son long dans le caniveau. Cette chute s’est accompagnée d’une bordée d’injures et de grossièretés que je n’aurai pas la faiblesse de reproduire en ces lignes, mais sachez qu’elles s’adressaient non seulement à la gent canine, ce ramassis de bâtards attardés tout juste bons à se renifler le cul et gueuler sans raison, mais surtout leurs soi-disant propriétaires, maîtres ou appelez ça comme vous voudrez, sombres crétins et abrutis de première classe qui considèrent l’espace public comme des chiottes à clébard. D’après Greg, on devait non seulement leur infliger une forte amende en cas de non-ramassage de crotte, mais aussi les condamner à ingérer l’objet du délit en mastiquant longuement chaque bouchée.

N’ayant rien d’autre sous la main pour nettoyer la semelle de sa chaussure, Greg, toujours en jurant comme un charretier, s’est mis à la racler furieusement contre le bord du trottoir, maudissant la négligence des usagers et l’incapacité des élus locaux à assurer la propreté de leur territoire. Dans ces conditions, l’interdiction massive et définitive de tout canidé dans l’espace public lui apparaissait comme la seule mesure raisonnable à prendre, le chien devant être réservé à un usage purement domestique, comme les fonctions de gardiennage ou d’ami de substitution pour les gens qui vivent dans la solitude la plus extrême, ou encore la pratique de la chasse en milieu forestier, endroit où ces sales bêtes peuvent pisser et chier à tout bout de champ sans que cela prête à conséquence, au même titre que les renards dont ils sont les proches cousins. Le chat, par contre, qui jamais ne s’abaisserait à chier devant tout le monde et encore moins laisser sa crotte à l’abandon, avait toute sa place dans la cité, d’autant que sa présence dissuadait les rats et autres rongeurs malintentionnés de s’installer ouvertement dans nos murs.

Cette opération de nettoyage nous a pris quelques précieuses secondes supplémentaires, de sorte que Repentance Whittingham, alias la Gardienne de la Nuit et accessoirement la femme de chambre la plus rapide du monde, avait déjà quelques longueurs d’avance sur nous quand nous nous sommes enfin lancés à sa poursuite.

Mais, grâce aux deux cent cinquante bourrins survitaminés attelés à ma charrette, il ne m’a pas fallu longtemps pour la rattraper. Le seul détail véritablement incommodant dans cette affaire, c’était que l’habitacle de la Kangoo, en dépit des efforts désespérés de Greg pour s’en défaire, empestait la merde de chien.

Me voyant débouler dans son rétro, elle a accéléré à son tour, prenant tous les risques pour me distancer, et je dois bien admettre qu’elle était loin d’être maladroite avec un volant entre les mains. Si un jour on la virait de son emploi de femme de chambre, elle pourrait toujours se recycler dans la course automobile. Cela dit, on n’était pas sur la Whaanga Coast ou au Panzerplatte. En ville, les limitations de vitesse sont réduites au plus bas et les panneaux de signalisation nombreux pour les faire respecter. Cela dit, faire partie de la noble famille des Représentants de l’Ordre présente tout de même quelques avantages, comme par exemple celui de disposer d’un gyrophare et une sirène pour s’affranchir des règles de circulation en vigueur. L’usager doit être averti quand des cinglés roulent à tombeau ouvert dans les rues de la cité. Il doit savoir que les forces de l’ordre (au même titre que celles du feu et de la santé), toujours sur la brèche pour garantir sa sécurité au péril de la leur, sont prioritaires quand des actions de ce type sont engagées sur la chaussée. Même si mon véhicule de service, qui se trouvait aussi être mon véhicule personnel (un ludospace très agréable en conduite normale et pratique pour transporter des objets lourds et encombrants), n’était pas une réplique exacte de la Pursuit Special de Mad Max, elle n’en constituait pas moins une arme de destruction massive une fois lancée à pleine vitesse dans les rues de la ville. J’ai donc mis en branle le système d’avertissement sonore et lumineux dont j’étais le dépositaire.

Au volant de sa Cooper S, totalement insensible aux appels à la raison que je multipliais à son égard, Repentance Whittingham filait comme une flèche et je lui suçais la roue (à défaut d’autre chose) au plus près, ne laissant entre nos deux véhicules que l’épaisseur d’une feuille de papier à cigarette. Dans les films d’action, les gens se collent au cul, roulent portière contre portière, et n’hésitent pas à froisser de la tôle pour arriver à leurs fins. Ils s’en foutent, c’est la production qui paye. Dans le cas présent, je le répète, il s’agissait d’un véhicule que j’avais payé de ma poche, rubis sur l’ongle, le prix d’une bouchée de très bon pain, je vous le concède, puisque je l’avais acheté d’occasion, mais dans lequel j’avais par la suite investi de confortables sommes d’argent pour le transformer en authentique machine de guerre habilement dissimulée sous les dehors inoffensifs d’un utilitaire sans autre prétention que celle de l’être, utile. On sait par expérience que sapiens sapiens (environ deux millions de glandes sudoripares, vitesse de pointe aux alentours de 45 km/h, inventeur de la bombe atomique, du petit salé aux lentilles et du saut à l’élastique) est très souvent au moins aussi attaché à sa voiture qu’à sa femme et ses enfants. Il la dorlote, la bichonne, la caresse amoureusement, se mire avec délectation dans sa carrosserie rutilante, s’enfonce jusqu’aux ouïes dans ses sièges en cuir pleine fleur pour savourer le chant rauque et mélodieux de son six cylindres sublimé par une ligne d’échappement optimisée, lui récure les jantes à la brosse à dents et polit les chromes à la peau de chamois. Il ne fait aucun doute, si la chose était matériellement possible, qu’il n’hésiterait pas à s’accoupler avec elle dans les positions les plus torrides du kamasutra automobile. On imagine alors facilement la suite, aussi terrifiante que fascinante : pour une raison quelconque, échappant totalement à l’entendement des apprentis sorciers du transhumanisme, la bagnole tombe enceinte. Après quelques semaines d’une grossesse tumultueuse, elle donne naissance à une sorte de monstre de Frankenstein supersonique, à mi-chemin entre l’homme et la Formule 1. À l’instar du singe et du nain (je ne fais bien entendu aucune analogie qualitative entre l’un et l’autre), la chose est dotée de membres inférieurs nettement plus courts que la moyenne, avec des cuisses très puissantes. Capable de se mouvoir efficacement aussi bien à la verticale qu’à l’horizontale, c’est toutefois dans cette dernière configuration qu’elle produit les accélérations les plus fulgurantes et pulvérise haut la main les records de vitesse des animaux les plus rapides de la planète. Et je parle ici uniquement des mammifères terrestres, comme le guépard, le lièvre et l’antilope, et de la poiscaille, comme l’espadon, le marlin et le thon banane. J’exclus volontairement les volatiles, comme le faucon, l’aigle et le martinet, et plus encore les insectes, minuscules créatures sex pedibus aux performances exceptionnelles, sachant que la vorace cicindèle, par exemple, qui ne dépasse pas les deux centimètres de long, flirte allègrement avec les sept cents kilomètres/heure, à tel point que ses propres yeux n’arrivent plus à suivre et qu’elle peine à distinguer ses proies dans le feu de l’action. Mais s’il fallait à tout prix trouver un alter ego vaguement humain à cette abomination mutante, ultime rejeton des passions déviantes de la sexualité automobile considérée comme un des beaux-arts, c’est clairement du côté des speedsters Jay Garrick, Wally West, Barry Allen, Pietro Maximoff (Vif-Argent), Danica Williams et Hunter Zolomon (alias Zoom ou Reverse-Flash) qu’il conviendrait de se tourner.

Bref, on s’en fout.

Ce qui est certain, c’est que personne, à commencer par moi, n’aurait pu se douter de ce qui allait se passer dans un très proche avenir.

Traquée, comme on le répète toutes les trente secondes dans ces pathétiques émissions de télé pour handicapés mentaux et baltringues en surcharge pondérale spécialisées dans les faits divers sanglants, cold cases et autres affaires judiciaires retentissantes dont tout le monde se contrefout éperdument, Repentance Whittingham a pris tous les risques pour tenter de nous semer. Elle aurait pu, pour rester dans la ligne cinématographique adoptée précédemment, être interprétée par la très contagieuse Antonia Thomas, père anglais, mère jamaïcaine, mélange aux yeux verts hautement détonant, susceptible de déclencher des tempêtes de niveau 5 dans les parties basses de la sphère anatomique : vents violents, slips arrachés et projetés à des lieues à la ronde, couilles tuméfiées, décharges à répétition, danger de priapisme et crise cardiaque. Pour celles et ceux qui ne seraient pas au courant, je me dois de préciser qu’Antonia, depuis quelques années maintenant, exerce la noble profession d’actrice au pays de Très Sa Gracieuse Majesté la Reine de Mes Deux. Hélas, son talent n’est pas reconnu à sa juste valeur par ces cons de rosbifs qui n’y entendent rien à l’art, la bouffe et la beauté féminine, rien à rien en général, sans quoi ils ne passeraient pas leur temps à se prosterner comme des carpettes décérébrées devant le Royal Vampire qui leur suce la moelle pour maintenir ses privilèges et son train de vie pharaonique. La vérité, c’est que ces insulaires, comme tous les gens de leur sorte, sont des asociaux de première classe, arrimés telles des patelles paranoïdes à leur foutu rocher. L’ineptie atavique et l’aveuglement quasi systémique de ses compatriotes, alcooliques pour la plupart il faut bien le dire, ont conduit à cette absurdité stratosphérique qui ne cesse de susciter mon exaspération (et celle, je veux le croire, de toutes les âmes sensibles qui ont encore le sens du beau et vouent, à travers ses productions les plus inimitables et abouties, un culte irréductible à la nature : si on veut la voir (Antonia) en activité et se prendre à rêver de serrer son petit corps frémissant entre ses bras pantelants, lui bouffer le nez à pleine bouche et éventuellement se livrer sur elle à des activités que la morale réprouve (je m’en excuse d’avance auprès d’elle et tous les membres de sa famille, mais je ne fais que traduire le sentiment général des heureux élus qui ont eu la chance de croiser sa route, hommes, femmes et animaux confondus, y compris insectes et batraciens, heureuses la mouche qui pète et la grenouille qui coasse sur son épaule dénudée), on doit malheureusement se satisfaire de somnoler avec une demi-molle devant des teen dramas et autres séries télé aussi diversement mémorables que Misfits et Lovesick, sans oublier Good Doctor aux côtés de Freddie Highmore, alors frais émoulu de la série Bates Motel. À noter que ce même Freddie Highmore, à l’instar d’une Olivia Cooke (délicieuse) ou un Nestor Carbonell (ténébreux à souhait avec ses sourcils épais et son regard de braise), qui lui donnent la réplique dans ladite série (Bates Motel, très réussie au demeurant), peine lui aussi à s’extirper des griffes du petit écran. Seule Vera Farmiga (qui joue le rôle de Norma Louise Bates, la mère de Freddie Highmore dans Bates Motel), née de parents ukrainiens, ancienne de la St. John the Baptist Ukrainian Catholic School de Syracuse, dans l’État de New York, et actrice pour laquelle j’éprouve une appétence aussi bizarrement obsessionnelle que dépourvue de tout caractère sexuel (même si je la trouve excessivement désirable, paradoxe que je peine à expliquer, je l’avoue, et ne tiens même pas spécialement à le faire, tant je préfère que le mystère reste entier, mais que j’aurais néanmoins, s’il fallait absolument tenter de fournir un élément de réponse, tendance à mettre sur le compte de l’aura de maternité quasi virginale qui émane de son adorable personne), s’en sort avec les honneurs. Croyez-moi ou non, mais c’est un bien triste monde que celui dans lequel la plupart d’entre nous survivent avec l’énergie du désespoir, et heureusement qu’il existe des gens comme Antonia Thomas et surtout Vera Farmiga pour badigeonner d’un peu de baume nos cœurs meurtris par les assauts répétés de l’existence.

Donc, comme je vous le disais, l’horreur est montée d’un cran.

Poussée dans ses derniers retranchements, telle une mygale acculée dans le fond de sa tanière, Repentance Whittingham s’est décidée à jouer le tout pour le tout. Refus de priorité, dépassement par la droite, emprunt de voie non autorisée, non respect des règles les plus élémentaires de bonne conduite et du bien circuler ensemble, elle a en quelques minutes pulvérisé tous les records détenus par les pires chauffards de la planète, prouesse d’autant plus remarquable qu’elle n’était à priori pas sous l’emprise de l’alcool ou d’une quelconque drogue, de synthèse ou autre. Franchement, si je n’avais pas eu entre les mains le volant sport de ma Kangoo Interceptor Turbo +, capable de franchir le mur du son par simple pression du gros orteil sur la pédale d’accélérateur, je crois que j’aurais été incapable de suivre le mouvement et obligé de renoncer, la mort dans l’âme et la larme à l’œil, à mettre un terme à la folle cavale de la Gardienne de la Nuit. Mais c’est mal me connaître que de croire que j’allais renoncer aussi facilement. J’avais en effet, outre une formation de cavalier de la Garde républicaine et d’enquêteur subaquatique (niveau bac ou RNCP 4), choses qui, quand on aime l’eau et les chevaux, peuvent toujours servir en cas de crise, suivi les cours de pilotage VRI du commandant Valentin Deschanel, spécialisé dans les go fast et les interventions en zone urbaine à forte densité, star incontesté de la discipline, hélas tragiquement décédé quelques mois auparavant dans un accident de la circulation, et ce alors qu’il rentrait tranquillement chez après avoir fait ses courses à la supérette du coin. Il faut dire qu’il était en T-MAX 530 (quarante cinq chevaux à six mille cinq cent tours/minute, le scooter préféré des dealers) et que le choc avait été d’une violence inouïe (et non pas inuite comme le prétendent certains imbéciles patentés dont on se demande encore comment ils ont réussi à ne pas faire virer de l’école et passer leur bac avec succès, le concept, je le rappelle, n’ayant aucun rapport particulier avec le Groenland et sa population autochtone, pas plus qu’avec la baleine, l’ours polaire et le caribou, même si l’ours polaire, plus encore que le caribou ou le pacifique cétacé, est capable de faire preuve de violence quand il se sent menacé, à noter la rime plutôt riche avec cétacé), j’en veux pour preuve ses mandarines, yaourts aux fruits (il adorait les yaourts aux fruits, personne n’est parfait) et blancs de poireau éparpillés dans tout le périmètre. Une vraie boucherie, et pour la perte sinon d’un ami à proprement parler, même si nous entretenions toujours d’excellentes relations, au moins d’un mentor qui m’avait enseigné les joies de la glisse et du talon-pointe.

Tandis que nous nous trouvions sur une artère passablement fréquentée, à slalomer dangereusement entre les usagers qui nous maudissaient au passage, Repentance Whittingham a tenté une manœuvre particulièrement osée, sinon suicidaire, qui consiste à doubler à l’aveugle dans un virage en priant le ciel que personne n’ait la mauvaise idée de se pointer en face. Ce cas de figure, systématiquement représenté dans les courses-poursuites au cinéma, donne toujours lieu, même si on sait que ça va passer de justesse, à des moments de franche rigolade, surtout quand le pauvre type qui arrivait paisiblement en sens inverse finit les quatre roues en l’air dans le décor, les cheveux en bataille et les fringues en lambeaux, et contemple, dépité, sa belle voiture toute neuve bonne pour la casse. C’est le genre de sitation cruelle et parfaitement injuste qui, reconnaissons-le à notre corps défendant, trouve bien souvent un écho favorable dans le public, preuve que l’être humain est encore loin d’en avoir fini avec ses vieux démons.

Sauf que dans la vraie vie, sans caméra ni perchman, les choses ne se passent pas toujours comme on voudrait. Le scénario n’est pas écrit, on improvise en permanence, et il n’est pas possible de retourner quarante fois la scène pour obtenir satisfaction. La première est la bonne, il faut être au top tout de suite et ne surtout pas compter sur le montage ou les effets spéciaux pour arrondir les angles.

Et ce qui devait arriver arriva : au moment où Repentance effectuait sur les chapeaux de roues son dépassement non autorisé pour cause d’absence totale de visibilité dans un virage particulièrement dangereux, un Sprinter (utilitaire léger de chez Mercedes, ndlr) arrivait en sens inverse. À son volant, se trouvait un triste sire qui lui-même roulait à tombeau ouvert parce qu’il était en retard à son boulot, peintre en bâtiment en l’occurrence. Non content de baigner dans son jus comme une grosse merde fraîchement pondue, il écoutait du Spear of Longinus à fond les ballons, fumait clope sur clope et braillait comme un veau dans l’habitacle enfumé de sa poubelle.

On ne va pas se mentir, se voiler la fesse, tenter de minimiser pieusement les faits : le choc a été d’une violence impitoyable, provoquant un de ces vacarmes épouvantables qui évoquent irrésistiblement un tremblement de terre, une explosion due au gaz ou une attaque de missiles russes, et la projection dans l’atmosphère d’une pluie de débris automobiles comparables aux scories d’une éruption volcanique de grande ampleur.

Disons-le tout net, la Mini ne faisait pas le poids face au Sprinter. Elle a été littéralement pulvérisée sous l’impact. Tout est allé tellement vite que le Sprinter (dont le conducteur, qui avait pris une cuite retentissante la veille, n’était sans doute pas au mieux de sa forme) n’a même pas esquissé la moindre tentative de dégagement. De mon côté, tandis que Greg tétanisé par la peur n’avait même plus la force de hurler, se contentant d’ouvrir un bec de cent pieds de long d’où aucun son ne sortait, j’ai sauté à pieds joints sur le frein et réussi de justesse, grâce à mon sens aigu du pilotage et ma parfaite connaissance de l’arme de destruction massive qui me tenait lieu de véhicule, à éviter le massacre. Après avoir embouti la Mini de plein fouet, le conducteur a cédé à la panique, et le Sprinter a continué sa route en zigzaguant dangereusement avant d’aller s’écraser à grand bruit contre un platane.

Je suis sorti de la voiture (le premier, Greg avait besoin d’un peu de temps pour se remettre de ses émotions) et me suis précipité au chevet de Repentance Whittingham. Enchevêtrée dans un amas de tôle indescriptible, j’ai constaté qu’elle donnait encore quelques vagues signes de vie. Naturellement, j’ai aussitôt appelé les secours. Même si j’avais quelques raisons de lui en vouloir, notamment celle de m’avoir fait risquer ma vie dans cette course-poursuite endiablée, l’idée de voir une aussi belle chose disparaître à tout jamais de la surface de la Terre me semblait intellectuellement irrecevable. Et oui, pas la peine de hurler, je sais très bien que les femmes, et les êtres vivants en général, ne sont pas des choses, même si on pourrait très bien partager le monde entre choses vivantes ou non, le fait d’être en vie se signalant essentiellement par sa nature dégénérative et éphémère. La vie, en sursis permanent dans le couloir de la mort, est une situation des plus inconfortables. L’être humain, conscient de cette précarité, a tôt fait de sombrer dans le doute et la paranoïa. Il tente, par tous les moyens, de prolonger son existence. Mais pourquoi s’acharner à vivre dans un monde aussi hostile, avec une sentence de mort épinglée au milieu du front ? Le danger vient de partout, y compris de l’intérieur, et peut surgir à tout moment, y compris celui où on s’y attend le moins. Malheur à l’inconscient, galvanisé par la jeunesse ou les stupéfiants, et souvent les deux, qui tente de tromper la mort, car elle finira un jour ou l’autre par l’emporter. La chose vivante, condamnée à disparaître, l’est aussi à se reproduire pour se survivre à elle-même. Sans cette fonction essentielle, toute vie est impossible. Elle peut aussi se dupliquer par ses propres moyens, sans l’aide d’un tiers, mais elle ne peut échapper au processus. La chose, par contre, la vraie, parfaite en soi, se suffit à elle-même et n’a nul besoin de ce stratagème pour perdurer. Son existence, sinon définitive, est au moins certaine et indéniable. Elle n’a nul besoin de régénérescence, renaissance, reset ou mise à jour, nul besoin d’évoluer, de s’adapter à son environnement. Elle est intemporelle, exempte de tout questionnement, toute remise en question. C’est ici, sans doute, que s’opère le distinguo entre la chose naturelle et l’objet fabriqué, artificiel. L’objet, en effet, est soumis à un objectif qui le dépérennise, le voue, au même titre que l’être vivant, à l’obsolescence et la disparition, l’obligeant à se transformer sans cesse pour subsister. C’est ainsi que l’objet, la machine, par exemple, transcende sa nature inanimée pour se rapprocher artificiellement du principe vital. À travers l’homme, l’objet se déplace, pense, vit et meurt. Il est, en quelque sorte, l’ultime avatar de son désir d’affranchissement des lois de l’existence. Depuis des millénaires qu’il se torture inutilement les méninges, l’homme n’aspire qu’à une seule chose : devenir une machine, un automate capable, sans le moindre effort, la moindre prise de tête, crise de conscience ou autre, le moindre doute sur la nature de ses actes, d’accomplir des miracles et de battre tous les records, y compris de longévité, de réduire définitivement au silence et l’impuissance sa vieille ennemie la Mort. L’acte sexuel, par exemple, se doit d’être entièrement dévolu au plaisir, et non plus à cette fonction dégradante et avilissante qu’est la reproduction. Ainsi, ce piège odieux tendu par la Nature pour nous contraindre à signer notre arrêt de mort, entériner l’acte de décès de nos années de jeunesse et d’insouciance, se transforme en gag de l’arroseur arrosé. Débarrassé de ces oripeaux d’un autre âge, le plaisir sexuel peut enfin s’exercer sans limite ni contrainte, faire feu de tout bois. Jouissons sans entrave, morale ou autre, et laissons les basses-œuvres de la chose reproductive à celles et ceux qui n’ont d’autre moyen de subsistance que de fabriquer des enfants à la chaîne. Voilà un monde plus juste, où chacun se spécialise dans son domaine de compétence et vit pleinement sa vie sans remords ni regret. Dans ce monde plus juste, les nantis, dont la recherche du plaisir est le principal, sinon seul et unique domaine de compétence, viennent au secours de ceux qui ont voué leur existence à la douleur. Voir les autres comme des objets, des instruments qu’on peut manipuler à loisir pour satisfaire ses exigences, se satisfaire, est en droite ligne du chemin d’excellence que nous nous sommes tracé. Notre objectif, je le rappelle, est de devenir des machines de guerre, utraperformantes, conçues pour dominer et conquérir le monde. Dans un premier temps, bien sûr, avant de s’attaquer au reste de l’univers, et j’en profite au passage pour rendre grâce à Elon Musk (visionnaire sud-africain de race blanche dont le caractère ultra-reproducteur, à priori paradoxal, s’explique uniquement par son appétence pour les femmes jeunes et jolies bien entendu, mais surtout son narcissisme exacerbé et la volonté de se dupliquer à l’infini) d’avoir eu la présence d’esprit de se ménager (à lui et quelques fidèles) des bases arrières dans l’espace afin de ne pas être pris au dépourvu le moment venu, sachant que les gens, hélas bien trop rares, qui voient un peu plus loin que le bout de leur nez, commencent à se trouver singulièrement à l’étroit sur ce lopin de terre étriqué qu’est la planète bleue. Toutes les créatures faibles et geignardes qui auront l’impudence de se mettre en travers de notre marche triomphale seront impitoyablement exterminées.

Et voilà pourquoi cette chère Repentance Whittingham, qui, j’ose le dire, était l’irréfutable incarnation de l’éternel féminin en son sens le plus mythologique (autant dire carrément mytho, archétypal, archi-typique, désespérément romantique, goethien, gothique, et, en un mot comme en cent, au moins en ce qui me concerne car tous les égouts sont dans la nature, assez éloigné de la blonde platine à la Ginger Rogers, d’autant que j’ai une sainte horreur de la comédie musicale et que Fred Astaire m’a toujours fait penser à un sosie virevoltant de Stan Laurel) du terme, n’avait à mon sens nul besoin de se livrer à cette malheureuse tentative de dépassement, au sens propre comme au figuré, laquelle ne pouvait, en définitive, que la confronter brutalement à sa propre finitude.

Quand il a été constaté qu’elle était encore en vie, même si celle-ci ne tenait plus qu’à un fil, Greg et moi, tandis que les badauds (à commencer par le conducteur de la vénérable Ford Fiesta que la Mini avait tenté de doubler) commençaient à affluer de toute part, nous sommes dirigés vers la carcasse du Sprinter.

Le baiser passionné qu’il avait échangé avec le tronc du platane lui avait explosé le moteur et fait voler le parebrise en éclats.

À l’intérieur, se trouvait une vieille connaissance que Greg, pour avoir suffisamment enquêté sur l’affaire Tiago Alvarez (Sally Robinson le harcelait encore quasi quotidiennement pour le pousser à franchir le Rubicon en éliminant l’assassin présumé de son bien-aimé) a parfaitement identifiée dès le premier coup d’œil. Aussi improbable que cela puisse paraître, il s’agissait ni plus ni moins que de l’ignoble néonazi psychopathe, pervers et pyromane Noé Desmarais, fondateur, avec les sieurs Aymeric Jégou et Milo Monteil, de la sympathique petite association de malfaiteurs connue sous le nom de Disciples de la Colère. Personnellement, je préférais la dénomination de Disciples de la Connerie, qu’ils avaient poussé à un rare degré de perfection.

Petite piqûre de rappel pour la route : un beau jour, un certain Léopold Chiasson de Bellisle, comte de son état, s’éprend de son garde-chasse, un jeune dieu répondant au nom de Robert Pleimelding. Dès qu’il voit son petit cul apparaître au coin d’un bois (quand il est à quatre pattes en train de ramasser des châtaignes, par exemple, ou de renifler une truffe avec son flair de lévrier), le comte est tellement excité qu’il serait prêt à enfiler cul sec le premier sanglier qui lui passe à portée d’entrejambe. Car le problème, voyez-vous, c’est que Robert est marié, et que rien ne prouve qu’il ait envie de se faire ramoner la tuyauterie par son employeur (même si à l’époque, je dis ça je dis rien, les gens n’étaient peut-être pas aussi tatillons sur les conditions de travail et les droits du citoyen, autrement dit rechignaient moins à payer de leur personne pour se donner les moyens de réussir dans la vie). Fou de désir, vous l’aurez compris, Bellisle lui sort le grand jeu : havane, cognac centenaire, grosse bûche dans la cheminée du salon, préludes de Chopin avec légers craquements d’époque par un Arthur Rubinstein au sommet de son art, peignoir en soie savamment entrouvert pour laisser deviner une anatomie aussi avantageuse que remarquablement bien conservée, interminable discussion sur la question de savoir lequel du 12 ou du 20 est le meilleur calibre pour la chasse à la bécasse, sachant que le plomb de 8 bourre grasse reste sans doute le meilleur compromis en toute circonstance, etc, etc. Petit hommage en passant à la vieille sorcière de la Madrague récemment disparue, ex-héroïne de La Vérité (vaudeville judiciaire et musical tragique signé Henri-Georges Clouzot), même si, qu’on le veuille ou non, ses amitiés politiques douteuses avec la fille cadette du caliborgnon de la Trinité-sur-Mer risque de ternir durablement l’image de passionaria de la cause animale qu’elle entendait laisser. Bellisle, disposant de moyens quasi illimités pour parvenir à ses fins, ne tarde pas, en dépit de la résistance héroïque qu’elle lui oppose, à réduire sa proie à sa merci. La romance peut commencer, et se prolonger ainsi jusqu’à la mort du comte, qui laisse alors assez d’argent à Robert et sa famille pour vivre confortablement en chantant les louanges et bénissant chaque jour le nom de leur généreux donateur. Dans la foulée, l’aristocrate lui octroie également les quelques cinquante hectares de forêt (et pas de la friche dégueulasse, hein, du terrain vague hérissé d’arbres faméliques tout juste bon à stocker des déchets nucléaires, non, de la bonne vieille forêt bien épaisse regorgeant du plus fin gibier et des meilleurs champignons comestibles poussant comme une bénédiction divine  au pied des plus beaux fûts), de forêt, disais-je, que Robert, même s’il n’a plus le pas aussi souple qu’auparavant, continue à arpenter inlassablement pendant ses vieux jours, la pipe au bec, comme il l’a toujours fait et rêve de le faire encore longtemps après sa mort dans les forêts enchantées du paradis. C’est là, au cours d’une de ses balades avec sa fidèle Greta (un drahthaar, croisement de griffon kortals et de braque allemand à poil court, excellent chien d’arrêt qui, le cas échéant, n’hésitera pas un instant à se jeter à l’eau pour repêcher une gélinotte criblée de plombs), son juxtaposé Chapuis Progress Grand Luxe calibre 12/70 (crosse anglaise en noyer premier choix, plaque de couche en bois de rose et sujets animaliers gravés en taille douce sur les contre-platines et le dessous de bascule, une arme de collection qu’il n’aurait jamais eu les moyens de s’offrir sans les largesses de Chiasson) et sa gibecière à rabat en cuir pleine fleur Lazzaro Bernardini (encore du cousu main qui cadre assez peu avec le statut de garde-chasse à la petite semaine de l’intéressé), qu’il tombe sur une vision d’horreur qui restera à tout jamais gravée dans sa mémoire : un corps (humain, le corps), qu’une quelconque entité malfaisante a manifestement tenté de détruire intégralement par le feu.

S’il était envisageable que le sujet, pour des raisons diverses (démence, fanatisme religieux ou mystique particulière, satanisme, pratique du vaudou, sur fond d’addiction à l’alcool ou toute autre drogue dure sous quelque forme que ce soit), ait tenté de mettre fin à ses jours de cette façon que je qualifierai de pour le moins médiévale, il l’était nettement moins qu’il se soit fait désintégrer par des extraterrestres en train de reconnaître les lieux dans la perspective d’une prochaine invasion. Tout bien considéré, le plus crédible restait que le pauvre avait été la victime d’actes de torture et de barbarie de la part d’un ou plusieurs individus qu’il restait à identifier et punir à la hauteur de leurs agissements.

Dépêché sur les lieux en compagnie de Zaahid Shirani, légiste d’origine hindoue avec lequel j’avais, dans un premier temps, noué des relations purement amicales, avant que celles-ci ne se transforment en relations quasi familiales (le ténébreux personnage avait fort habilement mis le grappin sur Tosca, la sœur jumelle de ma dulcinée), j’avais d’abord pensé à un règlement de comptes entre truands, la technique dite «du barbecue» étant souvent utilisée par ce genre de clientèle pour tenter de faire entrave à l’action de la justice (infraction, je le rappelle en passant au cas où certains d’entre vous auraient dans l’idée de céder à la tentation, passible au bas mot de trois ans d’emprisonnement et de 75 000 euros d’amende).

C’était compter sans la perspicacité de Shirani, limier diabolique capable de débusquer sans effort la plus microscopique aiguille dans la plus monumentale botte de foin. L’animal avait réussi, Dieu sait comment, à dresser le profil génétique de la victime et le comparer à celui de Tiago Alvarez, un ambulancier gay qui s’était récemment évaporé dans la nature, disparition jugée plus qu’inquiétante sur laquelle, par le plus grand des hasards, enquêtait Grégoire Lussier, un autre mien ami, pour le compte d’une (un) certaine Sally Robinson, sosie de Danny DeVito en jupon. Grâce à Cerqueira, gorille portugais au cerveau de moule qui tentait tant bien que mal de dissimuler son homosexualité à son entourage (je l’avais surpris en flagrant délit de racolage sur la voie publique, et menacé de le foutre en taule et ébruiter son petit secret s’il ne se montrait pas coopératif), à commencer par les membres de sa fine équipe de sympathisants d’extrême-droite, le lien entre Alvarez et Desmarais avait pu être établi. D’après le grand singe en question, c’était Noé Desmarais, suprémaciste blanc, homophobe et pyromane, qui avait carbonisé Alvarez au lance-flammes, en joyeuse compagnie de deux autres ordures de son espèce, Aymeric Jégou et Milo Monteil.

J’ai brandi ma plaque et hurlé à la cantonade : POLICE !!! CIRCULEZ, Y A RIEN À VOIR !!!

On le sait, les gens adorent les trucs scabreux. Ils vous diront que non, mais le fait est que s’il se passe quelque chose d’horrible quelque part, ils ont toutes les peines du monde à ne se précipiter pour jeter un œil. Ils sont comme ces charognards qui reniflent l’odeur de la viande froide et rappliquent ventre à terre pour prendre part au festin. Avant ils se contentaient de regarder, maintenant ils filment, ce qui leur permet d’une part de revoir la scène encore et encore, d’autre part d’en faire profiter celles et ceux qui n’auraient pas eu la chance d’y assister. Avant, ils n’avaient que leur parole à opposer aux sceptiques et aux jaloux, maintenant ils ont les images pour preuve de leur bonne foi. Ces images, d’ailleurs, peuvent leur assurer une petite rente s’ils sont les seuls à les posséder. Les chaînes d’info en continu se feront une joie de les récupérer pour les diffuser en boucle auprès du grand public. Naturellement, plus c’est ignoble et horrifique, et plus le rapport est important. Tout est fait, en ce bas monde, pour flatter les plus bas instincts de la communauté. C’est ainsi que les réseaux sociaux, ramassis de crétins décérébrés et de sociopathes à la petite semaine, peuvent se développer à la vitesse d’une portée de cafards dans un placard rempli de victuailles. Les vautours parlent aux vautours, partagent l’info en temps réel, vingt-quatre heures sur vingt-quatre. Les hyènes ricanent, les requins de la finance entrent dans la danse et se remplissent la panse. Honni soit qui mal y pense, et longue vie en passant à la reine Victoria, au prince de Galles, à Ed Wood (le comte d’Halifax, pas l’auteur de La Fiancée du monstre et Necromania, qualifié par les frères Medved de «plus mauvais réalisateur de tous les temps», ce qui est plutôt un compliment venant de réacs dans leur genre), au gentilhomme huissier de (pas à) la verge noire, au duc de Kent et à Naruhito, empereur du Japon, à sa gracieuse épouse Masako Owada, et bien sûr à leur fille Aiko, princesse de Toshi et Grand-cordon de l’ordre de la Couronne précieuse, au même titre que la reine d’Espagne et la princesse Basma de Jordanie, également (pour celles et ceux que ça intéressent, même si je me doute bien qu’ils ne sont légion) Grand-Croix de l’ordre royal de l’Étoile polaire de Suède. Les gens que je viens de citer, citoyens haut-placés de l’univers, quasi divinités vénérées par des peuples tout entiers (lesquels, il faut bien le reconnaître, ne barbotent pas toujours dans l’opulence et ont par conséquent d’autant plus de mérite à admirer des gens qui se goinfrent sur leur dos), bénéficient d’une protection toute particulière pour garantir leur intimité. Mais vous, qui n’avez pas de particule et encore moins de sang royal qui circule dans vos veines, sachez que quel que soit l’endroit où vous vous trouviez (j’allais dire cachiez, sans doute ce que vous auriez de mieux à faire), il y aura toujours quelqu’un pour vous filmer à votre insu. Non pas que votre vie présente un quelconque intérêt, mais le voyeurisme est aujourd’hui parvenu à un tel degré d’omniprésence qu’il est devenu impossible de s’y soustraire. Fini le bon temps où on pouvait agresser une petite vieille en toute sécurité, tabasser un étranger ou harceler sexuellement sa secrétaire sans que la moitié de la planète soit au courant dans les secondes qui suivent. Aujourd’hui, outre les caméras de surveillance qui fleurissent à tous les coins de rues, il faut compter sur les particulier qui vivent en permanence l’œil rivé à l’écran de leur smartphone. L’être humain a toujours été sujet au voyeurisme, c’est vrai, mais ce qui était jadis honteux peut aujourd’hui s’afficher au grand jour en toute légalité. De la même façon, l’exhibitionnisme n’est plus réservé à une certaine catégorie de personnel, comme les politiciens, les artistes ou les gens qui se baladent à poil sous des imperméables même quand il ne pleut pas, de préférence à la sortie des écoles. Non, vous pouvez maintenant être une parfaite nullité totalement dépourvue de charisme et réussir à capter l’attention de millions de followers encore plus nuls que vous. On dit follower parce que suiveur ou suiveuse c’est pas terrible, sinon franchement péjoratif, au même titre que disciple, qui fait un peu gourou d’une secte de demeurés, ou encore abonné, qui fait référence à la téléphonie des années 70 et cadre assez mal avec l’idée de modernité véhiculée par les nouvelles technologies.

Avant, si vous étiez un gros pervers de voyeur, vous aviez vu Psychose trente ou quarante fois, et, sans aller jusqu’à empailler des oiseaux, appliquiez la méthode dite «Norman Bates», c’est-à-dire que aviez acheté une perceuse et fait des trous un peu partout dans votre baraque pour vous rincer l’œil. Mais comme vous n’aviez pas de motel perdu au fin fond de la cambrousse, et qu’en plus vous aviez hérité d’une tête qui n’inspirait pas vraiment confiance, le plus dur était de trouver des femmes qui, pour une raison ou pour autre, acceptent de franchir le seuil de votre porte. Par chance, vous aviez un peu de famille, des sœurs, des tantes, des nièces, et même un fils qui s’était dégoté une femme somptueuse qui lui avait donné de nombreux et beaux enfants, dont trois filles pas piquées des hannetons, et qui, malgré le fait que vous sembliez bizarre çà tout le monde (le petit dernier, auquel vous fichiez une trouille bleue, ne se résignait à vous embrasser que contraint et forcé par ses parents, et encore n’aurait-il pas fait pire grimace si on l’avait forcé à rouler une galoche à une vieille méduse échouée sur la plage des Sablettes à La Seyne-sur-Mer), venait vous rendre visite de temps à autre. Vous pouviez alors, l’œil rivé à l’un ou l’autre des nombreux trous qui garnissaient votre intérieur, celui de la salle de bain ou des toilettes notamment, satisfaire pleinement vos répugnantes ambitions. Certes, tout cela demandait une préparation minutieuse et un art consommé de la dissimulation, car si jamais quelqu’un avait découvert votre petit manège, vous pouviez définitivement dire adieu au peu de vie sociale qu’il vous restait, et accessoirement vous préparer à aller finir vos jours en prison, endroit où on pratique encore la séparation des sexes et où on ne dispose de toute façon d’aucun outil pour faire des trous dans les murs. Aujourd’hui, tout se passe comme si un petit malin, un visionnaire et bienfaiteur de l’humanité, conscient de la détresse qui accablait ses semblables et bien déterminé à leur venir en aide, avait trouvé le moyen de faire des trous partout dans le monde pour que tout le monde puisse se rincer l’œil sans limite d’âge ni de distance. Des tas d’exhibitionnistes frustrés, contraint de vivre dans la honte et le déni, ont enfin pu afficher leur différence au grand jour. Tu veux voir mon zizi (pensée émue pour Francky Vincent, inoubliable auteur de La Braguette d’or, Alice ça glisse et La Chatte de la voisine), ma petite chérie ?  Non ? Pas de problème, je te le montre quand même. Oui, je sais, tu n’as que dix ans, mais il n’est jamais trop tôt pour s’instruire. Je vais te montrer à quoi ça sert, ce qui se passe quand on le secoue, et tu me remercieras plus tard. Voilà quand même une approche autrement constructive et conviviale du bien vivre ensemble (endroit coloré et agréable, salles bien conçues, belle initiative de la municipalité), qui devrait, sauf imprévu, largement contribuer à apaiser les tensions qui divisent les peuples et dressent inutilement les gens les uns contre les autres. Grâce à Internet, chacun peut enfin vivre sa passion sans se soucier du qu’en-dira-t-on. Ces échanges entre gens de bonne compagnie, hors de tout jugement, sont le lubrifiant qui permet à la machine sociale de tourner sans accroc. Désormais, si un accident se produit quelque part et que ne pouvez y assister, d’abord parce que personne n’a jugé utile de vous prévenir, ce qui est déjà assez grave, et ensuite parce qu’on ne peut pas être partout en même temps, soyez rassuré : même s’il se produit en haute mer, au sommet de l’Himalaya ou aux abords de la planète Mars, les rescapés auront pris soin de filmer la scène sous toutes les coutures. Vous assisterez, comme si vous y étiez, à la lente agonie de celles et ceux qui n’auront pas eu la chance de mourir sur le coup (que vous aurez vu mourir aussi, ne vous inquiétez pas). S’il s’agit d’un naufrage, vous verrez les moins chanceux se faire dévorer par les requins, et le reporter improvisé, animé par une foi intense et la volonté farouche de servir l’info à tout prix, préférera mille fois se faire dévorer lui-même plutôt que de renoncer à sa mission. Et s’il a encore la force, dans un ultime sursaut d’orgueil, de filmer sa fuite effrénée dans les eaux glacées de l’océan, vous verrez, en un sublime ralenti magnifié par les filtres idoines, les terribles prédateurs se rapprocher inexorablement de leur proie et la mettre en pièces jusqu’à ce que soit tranchée, en un ultime gros plan spectaculaire, la main qui tenait courageusement l’appareil hors des flots écumeux et rougis par le sang. Si on est au sommet de l’Himalaya, ce qui peut arriver aussi, quelque part entre le Népal et le Tibet, et que les survivants, pris dans une crevasse et gravement blessés dans leur chute, n’ont pas d’autre choix que de s’entredévorer pour survivre, il y en aura toujours pour filmer pendant que les autres passent à table. Je pense aussi à la Grande Guerre (celle de 14-18 pour les ignares), époque où la photographie n’en était encore qu’à ses premiers balbutiements en noir et blanc. On dispose bien de quelques images rudimentaires, d’une netteté approximative, mais quelle perte pour l’humanité de ne pas avoir pu voir et savoir ce qui se passait réellement dans les tranchées. Les derniers poilus sont morts, et la plupart yoyotaient sérieusement dans les dernières années de leur existence, chose bien sûr tout-à-fait normale dont il ne saurait ne leur être fait grief, surtout quand on a vécu les horreurs de la guerre. Difficile néanmoins, dans ces conditions, de se fier à leur témoignage, même si on ne rendra jamais assez hommage au courage et la ténacité dont ils ont fait preuve pour empêcher le pays de tomber aux mains de ceux qui n’étaient pas encore des enfoirés de nazis mais n’allaient pas tarder à le devenir. Si Manfred von Richthofen, par exemple, alias le Baron Rouge, avait pu filmer tout ce qui s’est passé à bord de son Fokker, je pense que ces images feraient aujourd’hui encore la une du box-office. Pour ce qui est de la planète Mars, avec les transmissions satellites actuelles, on pourrait vivre le massacre en direct sans aucune difficulté, une bière à la main, confortablement installé dans le canapé de son salon, entouré de poupées sexuelles hyperréalistes en provenance de Chine ou du Japon. Une certaine idée du bonheur, un peu déviante, peut-être, mais tellement contemporaine. Il faut vivre avec son temps, nom de Dieu, et le temps est aux pédophiles qui montrent leur bite sur les réseaux sociaux, aux profanateurs de sépultures qui vont pisser sur la tombe de Badinter, aux dirigeants irrédentistes et mégalos qui font main basse sur les richesses du monde en agitant l’épouvantail d’une troisième guerre mondiale, aux tueries de masse dans les écoles et les universités, aux ballots de coke qui s’échouent sur les plages de la Côte d’Opale, aux riches de plus en plus riches, aux pauvres de plus en plus pauvres, aux cons de plus en plus cons, aux morts de plus en plus morts, aux gamines qu’on shoote de force pour les prostituer dans des chambres d’hôtel sordides, aux réfugiés qui se noient dans l’océan (ils n’ont même pas la chance de tomber sur un ballot de coke qui leur permettrait de s’offrir des vêtements secs et des vrais faux papiers), aux gens de plus en plus gros à force de bouffer de la merde (qui est en vente libre, contrairement au crack, aux sels de bain, au fentanyl et à la kétamine, très nocifs je le rappelle, alors que l’ayahusaca et les champignons magiques, produits naturels s’il en est, sont toujours là pour vous procurer voyage et dépaysement à petit prix), aux gens qui s’exhibent à moitié à poil dans des escape games débiles et autres programmes affligeants (liste non-exhaustive pour prendre pleinement la mesure de l’ampleur des dégâts : Bâtard Academy, Qui veut épouser mon fils ? Personne merci, L’Amour est dans ton cul, Norbert crétin d’office, Mariés au premier coup de bite, La Villa des culs brisés, Danse avec les porcs, The Voice : La Poisse, L’île de la perdition, Le Meilleur bon à rien, Crétin Express, Astikoh-Lanta, Top Larbin, Cauchemar dans les latrines avec le trois-quarts centre étoilé de la cuisine française, etc, etc, etc) suivis par des millions de téléspectateurs qui sont à priori des gens normaux comme vous et moi (j’espère bien que non, sinon tout est foutu), j’en passe et des meilleures, des vertes, des pires et des pas mûres.

Donc, comme je le disais avant d’expliquer les raisons de cet emportement, j’ai brandi ma plaque et hurlé à la cantonade : POLICE !!! CIRCULEZ, Y A RIEN À VOIR !!!

Et j’ai ajouté, ce qui n’était pas absolument nécessaire techniquement mais très satisfaisant au niveau du bien-être et l’épanouissement personnel : FOUTEZ LE CAMP, BANDE DE CHAROGNARDS !!!!!!

La plupart de ces enfoirés ont foutu le camp, conformément à mes instructions on ne peut plus claires, mais certains, plus tenaces et affamés que d’autres, ont continué à rôder dans les parages, se planquant derrière les arbres et les tas d’ordures pour s’adonner à leur vice. Je précise que, suite à une grève prolongée des agents de propreté urbaine, plus communément appelés éboueurs, les poubelles vomissaient leurs entrailles et l’espace public s’était transformé en décharge à ciel ouvert, avec diverses conséquences, dont au moins deux passablement préoccupantes sur le plan sanitaire et social : d’une part attirer toute une faune de bestioles peu regardantes sur l’état de fraîcheur de leur alimentation, d’autre part empuantir l’atmosphère de façon significative. Même les oiseaux, qui d’ordinaire se plaisaient à faire des vocalises dans la ramure environnante, avaient déserté la place.

Les pompiers, les flics, les secours sont arrivés, toutes sirènes hurlantes.

Desmarais, assis au volant de sa bagnole comme si de rien n’était, à ceci près qu’il était tout de même légèrement disloqué de toute part, ne présentait aucune blessure apparente. L’idée qu’il ait pu survivre à l’accident était pour moi d’une extrême contrariété. J’étais, vous le savez, bien décidé à le buter. J’avais beau tourner et retourner la question dans tous les sens, je ne voyais aucune circonstance atténuante à accorder à cette ordure. Même s’il avait eu une enfance horriblement malheureuse, si son père l’avait élevé dans le culte du Troisième Reich et obligé à apprendre Mein Kampf par cœur dès son plus jeune âge, il devait disparaître sans laisser de traces. Et comme il y avait de fortes chances que sa femme partage ses opinions, qui n’étaient d’ailleurs pas des opinions mais seulement une expression idéologisée de la haine, la bêtise et la vulgarité, il aurait été prudent de l’exterminer dans la foulée. Même chose pour ses gosses, auxquels qu’il avait vraisemblablement inculqué, comme son père l’avait fait avec lui, les valeurs de l’idéologie nazie, valeurs qu’eux-mêmes se feraient un devoir de transmettre à leur progéniture. Mais peut-être les avait-il découvertes lui-même, comme un grand, après une quelconque tragédie qui l’avait laissé profondément meurtri et gorgé d’amertume dans un monde cruel où il n’avait plus sa place. Imaginez par exemple que sa mère, après avoir pris conscience (un peu tard, malheureusement) que son mari n’était qu’une brute, un alcoolique et un crétin antisémite, raciste et xénophobe, ait cédé aux tentations de l’adultère avec un employé du gaz venu relever les compteurs (et les jupes par la même occasion). C’est déjà très embêtant, car chacun sait que la brute raciste et xénophobe est très à cheval sur les valeurs morales et la fidélité conjugale (surtout en ce qui concerne sa femme), mais si l’employé du gaz en question cumule avec une rare insolence toutes les tares les plus rédhibitoires aux yeux de ladite brute, on coure droit à la catastrophe. Même s’il avait été le parfait aryen dans toute sa splendeur, la chose aurait été difficile à avaler. Mais imaginez, ne serait-ce qu’un instant, qu’il soit juif, arabe, communiste, franc-maçon, bisexuel, et, pour couronner le tout, affligé de quelque handicap mineur ou discrète malformation congénitale (un sexe anormalement développé, par exemple, et capable de soutenir une érection des heures durant sans montrer le moindre signe de faiblesse). Le mari trompé, quand il découvre le pot aux roses (en l’occurrence des roses fanées dont les tiges putréfiées baignent dans l’eau croupie), entre dans une fureur noire. Après être allé se recueillir une dernière fois sur la tombe de Rudolf Hess à Wunsiedel, en Bavière, il endosse son plus bel uniforme de la SS, vérifie que le chargeur de son semi-automatique Luger P08 est plein à craquer, puis se dirige d’un pas ferme vers la chambre d’hôtel où il sait que les amants ont l’habitude de se retrouver pour donner libre cours à leur frénésie sexuelle. Là, il les trouve au lit, en train de forniquer comme des bêtes, et la vision de cette espèce d’animal velu en train de prendre sa femme en levrette le pousse à commettre l’irréparable. Il l’abat d’une balle en pleine tête, chose qui, et c’est pour dire à quel point cet employé du gaz est anormalement constitué, n’entrave en rien sa vigueur sexuelle. Même mort, la tête à moitié arrachée, il continue à limer furieusement la malheureuse couverte de sang et de cervelle qui hurle tant et plus. Désireux de mettre un terme à ses souffrances, notre homme l’abat à son tour d’une balle dans la tête. Mais le monstre, loin de s’arrêter à ce genre de contretemps, continue à besogner le cadavre comme si de rien n’était. Le malheureux, qui n’est alors pas loin de perdre la raison, vide son chargeur sur le zombie qui est toujours en train de s’acharner sur la dépouille de sa volage moitié. Voyant que tout cela est sans effet, et comprenant que le monstre, qui le dévisage en se léchant de façon obscène les babines ensanglantées, et au sujet duquel j’ai indiqué précédemment qu’il était à voile et à vapeur, ne va pas tarder à s’en prendre sexuellement à lui, il recharge son arme et se fait sauter la cervelle. Et, détail sordide que je ne pas encore eu le courage de vous révéler, Desmarais père ne s’est pas rendu seul à cette expédition punitive qui vient de tourner au cauchemar. Non, il a emmené avec lui son fils Noé, treize ans moment des faits, afin qu’il soit témoin de l’indignité de sa mère. Le petit Noé, donc, a assisté à tout. Butalement plongé jusqu’aux ouïes dans la triste réalité d’une existence qu’il n’avait pas choisie, il a vu sa mère adorée mourir des mains de son père, puis son père se donner la mort pour échapper aux assauts de l’amant de sa femme transformé en zombie. Lui-même, Noé, n’a dû son salut qu’à une fuite effrénée dans les couloirs de l’hôtel, avant que le zombie ne soit finalement carbonisé au lance-flammes par le GIGN. On peut comprendre, et je suis tout disposé à le faire, que de tels événements aient influencé durablement le cours de son existence et précipité sa chute. On peut comprendre que sa raison en ait été ébranlée, et qu’il ait développé un certain penchant pour les arts du feu et la pyrotechnie. Néanmoins, quels que soient le trauma originel, l’éducation pourrie et l’état de dégradation mentale auquel il était parvenu, on ne pouvait continuer à le laisser sévir impunément. Face à un détraqué de ce calibre, on ne pouvait en aucun cas se permettre de miser sur la clémence ou les bons sentiments. On aurait pu le mettre en prison, bien sûr, et il aurait passé le restant de ses jours à ruminer sa vengeance. Mais à quoi bon, puisqu’il était irrécupérable, et que même s’il l’avait été, ça aurait de toute façon été un boulet accroché au pied de la société. Même s’il avait cessé de foutre le feu aux gens, il aurait continué à répandre ses idées malsaines et participer à la déliquescence ambiante. On aurait aussi pu le foutre en HP et le bourrer de médocs jusqu’à sa mort. Dans un cas comme dans l’autre, à part faire bosser les toubibs et les gardiens, le retour sur investissement aurait été nul. En humaniste convaincu, je suis et ai toujours été un farouche adversaire de la peine de mort, en ce sens que la mort n’est pas une peine mais une nécessité, au mieux un incident de parcours. C’est sur la mort des uns, le terreau de leur cadavre, que fleurit la vie des autres. Et je suis d’avis que la société ne doit en aucun cas se salir les mains en effectuant les sales besognes nécessaires à sa survie. C’est à certains citoyens, plus dévoués que d’autres à la cause commune, de s’y coller, et la justice n’a rien à voir là-dedans. On peut m’objecter que ce n’est pas à moi de décider qui doit ou non cesser de respirer. Ce serait vrai si je décidais de quoi que ce soit, mais ce n’est pas le cas. Je ne décide de rien du tout, pas plus que le marteau ne décide de taper sur la tête du clou. Je fais ce que j’ai à faire en toute décontraction, en toute humilité. J’essaie de le faire bien, en bon artisan respectueux de la tradition et de ses outils, à l’image de mon père qui, après l’armée, avait œuvré un temps comme tueur à gage pour un caïd de la drogue, avant de partir planter ses choux dans des contrées plus vertes. Même s’il était au service de l’une d’entre elles, et non des moindres, mon père n’exécutait que des ordures, raison pour laquelle j’ai toujours considéré qu’il exerçait une profession de salubrité publique. D’autant qu’il travaillait proprement, sans excès de zèle, ce qui explique sans doute que personne ne soit jamais revenu de l’au-delà pour se plaindre de ses agissements. Comme lui, à ceci près que je fais dans le bénévolat et n’attends aucune espèce de reconnaissance de la part de qui que ce soit, je supprime certaines personnes peu fréquentables parce qu’il faut bien que quelqu’un le fasse, et que cette mission m’a été dévolue par des forces supérieures sur lesquelles je n’ai aucune autorité. Et non, je ne suis pas ce genre de cinglé qui entend des voix lui murmurer dans le creux de l’oreille qu’il doit prendre les armes et endosser sa plus belle, blanche et étincelante armure pour aller défendre la veuve et l’orphelin. Personne ne me parle, ne me dit quoi faire. J’agis en toute simplicité, aussi naturellement que l’agneau tète sa mère avant que le loup ne la croque, en toute simplicité lui aussi.

Cela dit, pour en revenir à la situation présente, je n’avais rien contre le fait que le destin fasse le boulot à ma place. Après tout, je n’étais pas responsable de l’existence de cette tête de nœud, et n’avais donc aucune raison particulière de m’infliger la corvée de le faire disparaître.

Comme beaucoup d’urgentistes souffrant de handicap visuel, le docteur Sébastien Charrier portait des lunettes Lazarus Pierson en titane avec verres antibuée, antireflet, antitout, charnières flexibles et plaquettes nasales antidérapantes, spécialement conçues pour procurer le maximum de confort aux membres du corps médical. L’urgentiste de terrain, confronté à des conditions de travail souvent difficiles, se doit d’être parfaitement équipé pour donner le meilleur de lui-même. Charrier était le genre d’homme qui s’entretient physiquement et intellectuellement pour être toujours au top des ses capacités. Il était marié et père de trois enfants. Sa femme le trompait avec son gynécologue, le docteur Rémi Durand, qui était aussi le meilleur ami de son mari, mais il s’agit là d’un cas de figure tellement banal que de nos jours plus personne n’y prête attention. D’autant qu’on ne pouvait pas dire à proprement parler qu’elle le trompait avec le docteur Rémi Durand, puisque le docteur Charrier était parfaitement au courant de ses agissements. Non seulement il était au courant, mais lui-même couchait avec la femme du docteur Durand, son meilleur ami, lequel meilleur ami était bien entendu lui aussi parfaitement au courant des agissements de son épouse. Charrier était couvert de poil, comme un singe, et sa libido était semblable à une bête sauvage qu’il avait toutes les peines du monde à tenir en laisse. Déjà, quand il était petit, ses parents faisaient venir à la maison des amis avec lesquels ils organisaient des orgies sexuelles où tout le monde était convié, y compris les enfants. Il ne s’agissait pas d’inceste à proprement parler, mais le fait est que sa mère et la plupart de ses frères et ses sœurs l’avaient sucé à de nombreuses reprises. Même son grand-père, qui ne crachait pas sur les petits garçons à l’occasion, lui avait plus d’une fois sucé la bite, tout comme lui-même, dans un souci d’équité et de retour d’ascenseur, avait plus d’une fois sucé la bite et avalé la semence grand-paternelle. C’est dans ce contexte qu’il avait décidé, comme son père, lui-même chirurgien de renom, et son grand-père avant lui, généraliste unanimement apprécié pour ne pas dire vénéré, de consacrer sa vie à sauver celle des autres. Mais quand on passe le plus clair de son temps à côtoyer la mort, il faut bien se détendre un peu en rentrant chez soi. Tout le monde l’avait bien compris, raison pour laquelle des partouzes étaient régulièrement organisées chez les Charrier ou les Durand, tous deux possesseurs de vastes maisons de maîtres dans la campagne environnante. Les Charrier, par exemple, étaient les heureux propriétaires d’un château du XVIIIe (quinze pièces, parc paysager de deux hectares, piscine et dépendances), dans le Loiret. Naturellement, quand on le voyait débarquer comme ça sur les lieux d’une catastrophe, on ne pouvait pas se douter que le docteur Sébastien Charrier était obsédé par les jeunes et jolies infirmières qui gravitaient autour de lui. Tellement qu’il lui arrivait de leur proposer, pour arrondir leurs fins de mois (tout le monde sait que le métier d’infirmière, qui exige non seulement des compétences médicales, mais aussi beaucoup de patience et d’empathie face à des gens en situation de stress et de faiblesse extrême, est très insuffisamment rémunéré et ne suscite plus guère de vocations), de venir passer le week-end avec lui et ses amis à la campagne. Tout ce qu’elles avaient à faire était de venir dans leur tenue de travail, autrement dit en petite tenue sous leur blouse (en coton de préférence, dessous sexy appréciés), et de s’occuper gentiment des pensionnaires. Il leur était également demandé de ne faire aucune différence entre homme et femme, de quelque âge, condition sociale (elles n’avaient pas de souci à se faire, il n’y aurait que des gens bien élevés et très à l’aise financièrement) ou origine ethnique que ce soit, les séminaires en question étant placés sous le signe de la confraternité intergénérationnelle, culturelle et sexuelle entre les peuples. Toute pratique, tant qu’elle ne portait pas (gravement, on ne pouvait jamais totalement exclure tel ou tel suçon ou légère trace de morsure, lesquels, de toute façon, feraient l’objet d’un dédommagement adapté au préjudice) atteinte à l’intégrité physique de la personne, devait être acceptée et satisfaite dans la joie et la bonne humeur la plus exubérante. Si elles respectaient à la lettre ces quelques directives, leur train de vie pourrait connaître une embellie spectaculaire. À elles le coupé sport, les robes de princesses, les bijoux chatoyants, les restaurants étoilés et les voyages en première classe aux Seychelles et à Bora Bora. La plus stricte discrétion était bien entendu de mise, sachant que la plupart des gens sont bien trop étroitement cadenassés dans un carcan de valeurs morales d’un autre âge pour admettre que certains aient impunément accès à des plaisirs auxquels ils n’auront jamais droit.

Jusqu’au jour où l’une des infirmières en question, une certaine Alena Benesch (blonde, le teint pâle, avec des grands yeux noisette et un visage d’ange tombé du ciel), est allée porter plainte au commissariat le plus proche.

Et devinez qui se trouvait dans ce commissariat le plus proche ?

Votre serviteur.

Et c’est à lui qu’on a refilé la patate chaude.

Vêtu de mon plus chouette costume, mon plus charmant sourire aux lèvres, accompagné d’une poignée de flics en uniforme pour renforcer l’aspect solennel de la démarche, je suis allé sonner à la porte de Charrier, dans les beaux quartiers, là où toutes les baraques, ceintes de hauts murs hérissés de tessons de bouteilles ou de clôtures électriques, sont équipées d’alarme dernier cri qui retentissent au moindre frémissement, déclenchant aussitôt le bouclage de la zone et le parachutage sur site des meilleurs éléments du RAID et du GIGN. Des maîtres-chiens lourdement armés patrouillent dans les rues, et des snipers surveillent H24 les alentours du haut de miradors placés à tous les coins de rues.

Il convient également, si l’on ne souhaite pas finir ses jours en fauteuil roulant, de se méfier des pièges à loup dans les parcs et mines antipersonnel dans les allées de jardin.

Eh bien croyez-le ou non, j’ai été reçu comme un cheveu sur la soupe.

C’est tout juste si on ne pas claqué la porte au nez.

En fait, ça n’aurait pas été pire si j’avais été un huissier de justice ou un témoin de Jéhovah .

Je parle d’un témoin de Jéhovah mâle, bien sûr, avec les oreilles décollées et un physique d’expert-compatible, ou une bonne femme de cinquante piges mal fagotée avec des dents jaunes et un fort strabisme divergent.

Parce que je vous fiche mon billet que si j’avais été un témoin de Jéhovah femelle avec la plastique de Scarlett Johansson, Jenna Ortega, Rachel McAdams ou Gal Gadot, le tout emballé dans une blouse d’infirmière ultra sexy avec les boutons prêts à exploser sous la pression des formes généreuses qu’elle s’efforçait désespérément de contenir, on m’aurait déroulé le tapis rouge, offert des fleurs, un verre de Champomy, et aussitôt proposé un emploi à plein temps comme garde d’enfants à domicile, gouvernante, femme de ménage, jardinière de légumes ou n’importe quoi d’autre pour me garder à portée de braguette et tenter de m’entraîner corps (surtout) et âme dans la spirale du vice.

Poussé dans ses retranchements, Charrier a admis qu’il lui arrivait de recevoir des filles dans son château du XVIIIe, mais que celles-ci étaient majeures, traitées avec les mêmes égards que les autres invités, sans que ne soit jamais exigée aucune contrepartie de leur part.

À l’époque j’étais comme toi, lecteur, jeune et fougueux inspecteur, épris de justice et de liberté, révolté à l’idée que les puissants abusent de leurs prérogatives en toute impunité. Je rêvais, à cheval sur mon blanc destrier, engoncé dans ma plus resplendissante armure et coiffé de mon plus bel heaume, de voler au secours de la veuve et l’orphelin en proie aux persécutions de nobles dévoyés, d’usuriers pervers et de crapules sans foi ni loi. D’étranges rumeurs circulaient au sujet du château de la Frétoise, près de Montargis, entre Préfontaines et Corquilleroy.

Quelques années auparavant, une fille avait disparu sans laisser de traces dans le secteur. Son vélo, ainsi qu’une de ses chaussures et une pince à cheveux, avaient été retrouvés sur la route de Nargis, au lieu-dit du Bois au Notaire. La pauvre enfant venait tout juste de fêter ses dix-sept ans. Les recherches n’ont rien donné, mais les allées et venues nocturnes du côté de la Frétoise avaient éveillé les soupçons de plusieurs personnes du voisinage, même si l’endroit est particulièrement isolé. Le propriétaire des lieux, un certain Sébastien Charrier, avait été entendu. Il avait déclaré n’être au courant de rien, mais s’était montré des plus désagréables, traitant les enquêteurs avec dédain, n’hésitant pas à leur faire sentir qu’ils n’étaient que des moins-que-rien indignes de sa compagnie, des pauvres types auxquels il ne se résignait à adresser la parole que contraint et forcé. Le même sort m’a été réservé lorsque je l’ai entendu à mon tour, suite au dépôt de plainte de l’infirmière qui affirmait avoir été violentée durant son séjour au château. Un soir, par exemple, après un repas bien arrosé, on l’avait trainée de force dans le salon et obligée à se dévêtir entièrement devant les invités, tous complètement bourrés et incapables de la moindre retenue. Elle se souvient que certains, à commencer par le docteur Charrier lui-même, avaient le sexe à l’air et se masturbaient outrageusement devant elle. C’est ce même docteur Charrier qui lui avait ensuite copieusement aspergé les parties génitales avec de la crème chantilly, avant de faire venir Hermann, son dogue allemand, pour qu’il nettoie la zone à grands coups de langue sous les éclats de rire et les encouragements obscènes de l’assistance. Ames sensibles s’abstenir. Je vous laisse néanmoins imaginer, si toutefois vous en avez le courage, la terreur de cette jeune femme, livrée à la frénésie d’un monstre de près de quatre-vingt-dix kilos qui aurait très bien pu ne pas se satisfaire de cet amuse-gueule et décidé de passer sans plus tarder au plat de résistance. Quelle horreur ! Si vous ajoutez à cela le caractère extrêmement humiliant et dégradant de la scène, vous transpirez à grosses gouttes et prenez aussitôt la mesure du traumatisme subi.

Naturellement, cette enflure de Charrier a nié les faits avec la dernière énergie, affirmant que la fille mentait pour se faire du fric sur son dos. Selon lui, ce tissu de conneries grosses comme un troupeau d’éléphants risquait de jeter un voile crasseux sur la blancheur immaculée de sa réputation. En conséquence, il a menacé de porter plainte en retour pour diffamation. Son avocat était le genre de Pavarotti du barreau capable de faire passer un curé pédophile pour un honnête serviteur de Dieu. Il a également claironné qu’il allait s’empresser de solliciter ses relations dans les plus hautes sphères de l’État, afin que celles et ceux qui avaient prêté une oreille complaisante à ces élucubrations soient sanctionnés à la hauteur de leurs agissements. Peu de temps après, j’ai été convoqué chez le dirlo qui m’a gentiment expliqué que Charrier était un type comme ça, un héros des temps moderne qui ne comptait pas ses heures pour sauver la vie des gens. Charrier n’était pas le Gilles de Rais de Montargis, le Barbe Bleue de la Frétoise, l’Ogre de Préfontaines qui nourrit ses molosses avec de la chair humaine, et il fallait séance tenante arrêter de l’emmerder avec cette histoire de dogue allemand bouffeur de chatte à la chantilly. Tout cela ne tiendrait pas une seconde devant les tribunaux. D’autant que l’infirmière en question, même si elle en avait toutes les apparences, était loin d’être une princesse de conte de fée, aussi pure et innocente que la rosée du matin ou la fleur des champs fraîchement éclose. Ses états de service faisaient mention d’un certain nombre de délits qui cadraient assez mal avec le numéro de novice de couvent des Ursulines qu’elle avait tenté de nous faire avaler : excès de vitesse, conduite en état d’ivresse et usage de stupéfiants.

Résultat des courses : Alena Benesch a retiré sa plainte et Charrier s’en est sorti sans une égratignure, lavé de tout soupçon, blanc comme la neige qui recouvrait jadis les vastes plaines de notre enfance.

Naturellement, Benesch a été gentiment remerciée par la Direction et priée d’aller exercer ses talents de garde-malade dans un autre établissement, si possible dans une autre galaxie, à des années-lumière de la planète Terre. Charrier, grand seigneur, lui a versé un petit pécule pour l’aider à tenir le coup en attendant de trouver un nouveau job. Pas rancunier pour un sou, il lui a glissé dans le creux de l’oreille qu’il y aurait toujours un bol de soupe pour elle à la Frétoise. Et pas seulement de la soupe : il avait fait le plein de chantilly, et Hermann, au bord de la dépression depuis son départ précipité, la réclamait avec insistance.

Tu te demandes peut-être, ami lecteur dont je connais la sagacité et la soif de transparence, comment je suis au courant de tous ces détails concernant la vie privée de Charrier, notamment son enfance dévoyée au sein d’une famille sexuellement dysfonctionnelle ?

La réponse est simple : quand j’enquête, j’enquête, ce qui signifie que je passe au crible tous les éléments ayant trait à l’enquête en question, y compris les plus insignifiants. J’effectue des recoupements, fouille dans les tiroirs, les archives, les boîtes à chaussures, les armoires à linge et les cartons à chapeau, épluche les livrets de comptes, les carnets de liaison, les journaux intimes, visite les jardins secrets, déterre les cadavres et plonge tête baissée dans les profondeurs existentielles des protagonistes de l’affaire. Ce travail de fourmi, long et ingrat, me permet de débusquer sans coup férir l’ivraie qui se dissimule au sein du bon grain. Ainsi, Charrier avait des frères et sœurs avec lesquels il n’était peut-être pas forcément dans les meilleurs termes du monde, au point qu’il ne parlait quasiment plus à certains d’entre eux. C’est vers eux qu’il fallait se tourner pour obtenir de précieuses informations sur un frère qui ne leur inspirait plus que mépris et répulsion. Il devenait alors possible de reconstituer pas à pas le parcours d’un Charrier, comprendre comment avait pu s’échafauder dans son cerveau malade cette passion dévorante pour les dogues allemands, les infirmières et la chantilly.

Voilà comment, ami lecteur, j’ai pu apprendre toutes ces choses passionnantes sur la vie du docteur Sébastien Charrier, et te donner ainsi l’impression que j’étais dans le secret des dieux, alors que tout cela n’était finalement rien d’autre que le fruit d’un travail lent et minutieux. Cela dit, oui, je peux aussi être un tout petit peu dans le secret des dieux, car, sauf le respect que je te dois, je ne suis pas non plus obligé de tout te dire. C’est encore moi le seul maître à bord de ce rafiot littéraire qui vogue sans relâche sur les flots écumeux de la syntaxe, la rhétorique, la synecdoque, l’oxymore, la litote, la métaphore, l’antonomase, l’allégorie, le truisme et le paradoxe. Donc oui, je ne suis pas obligé de tout de dire, et peux choisir à mon gré le moment de te dévoiler ou pas tel ou tel aspect de l’histoire qui nous occupe.

Charrier m’a immédiatement reconnu.

Il a sorti de sa poche une seringue remplie d’un liquide verdâtre qu’il a essayé de me planter dans le bras.

Je ne sais toujours pas quel était ce liquide verdâtre de merde, de la pisse de rat, du jus de cadavre ou autre chose, mais quelque chose me dit que je serais mort dans d’atroces souffrances s’il avait réussi à me l’injecter.

J’ai esquivé le coup, une courte lutte s’est engagée, à l’issue de laquelle il s’est retrouvé au sol.

J’ai tenté de le ramener à de meilleurs sentiments avec un bon coup de latte dans les parties, mais il a réussi à m’attraper le pied, le tordre et me faire perdre l’équilibre.

Je me suis retrouvé au sol à mon tour, la cheville endolorie, pendant que Charrier se relevait d’un bond avec l’élégance d’un sportif de haut niveau.

Greg s’est jeté sur lui pour l’étrangler, mais Charrier l’a vu arriver et reçu avec une série de coups qui l’ont laissé sans voix, notamment le crochet au foie et l’uppercut à la mâchoire, tous deux assénés avec une précision d’autant plus redoutable que l’anatomie n’avait aucun secret pour lui.

Greg, faisant sienne la devise de l’inspecteur Harry Callahan dans Magnum Force, «l’homme sage est celui qui connaît ses limites», n’a pas jugé utile de se révéler après avoir mordu la poussière.

Dans une autre vie, Charrier avait été champion de France universitaire de boxe anglaise. Il avait perdu en vitesse, son jeu de jambes n’était plus ce qu’il était et ses réflexes s’étaient quelque peu émoussés, mais il lui restait encore largement de quoi faire illusion sur un ring.

Pas  besoin d’être diplômé de l’ESSEC et encore moins d’être le principal actionnaire de la Royal Caribbean Cruises Ltd. pour comprendre que la situation était en train de se barrer méchamment en couille.

Il m’a fallu développer des trésors de résistance à la douleur et de volonté farouche de survivre dans ce monde cruel qui est le nôtre pour réussir enfin à retrouver cette position qui, au même titre que le cheval, la femme et le dromadaire, compte au rang des plus belles conquêtes de l’homme, je veux bien sûr parler de la bipédie. Grâce à elle, nous avons tout le loisir de conserver la pleine et entière jouissance de nos membres supérieurs, nos mains en particulier, ce qui nous a permis d’accomplir des miracles qui seraient à tout jamais restés hors de portée si nous avions été contraints de nous déplacer à quatre pattes.

C’est alors que j’ai vu Charrier se diriger à grands pas vers moi, les poings serrés et les traits horriblement déformés par la haine, l’envie de me détruire entièrement, me mouliner, me torréfier, me hacher menu, me réduire en cendre, m’éradiquer définitivement de la surface de la Terre (510 millions de kilomètres carrés tout de même, dont 70\% de flotte, rivières, lacs et profondeurs océaniques peuplés de créatures aussi étranges que primitives).

La situation était d’autant plus préoccupante qu’il avait un scalpel à la main, instrument dont les qualités de tranchant ne sont plus à démontrer.

Greg au sol, et apparemment bien décidé à y rester, je ne pouvais plus compter que sur moi.

Et Manu, bien sûr, fidèle serviteur de la Loi qui ne m’avait jamais trahi, ne s’était jamais enrayé, n’avait jamais connu la moindre avarie en plus de vingt ans de bons et loyaux services, vingt longues années d’épreuves traversées côte à côte, la main (ou la crosse, si vous préférez) dans la main, le doigt sur la détente.

Il y a des moments, dans l’existence, où l’heure n’est plus aux conciliabules,  atermoiements et autres vaines tergiversations.

Quand l’ennemi fond sur vous, l’écume aux lèvres, et que vous êtes clairement en infériorité numérique, le mieux est encore de faire feu sans se poser de question.

C’est ce que j’ai fait, à deux reprises.

Le premier projectile a raté sa cible, à savoir la tête de Charrier, mais le second a fait mouche.

Son crâne s’est ouvert comme un œuf à la coque, et sa cervelle a été projetée dans les airs, tel un drôle d’objet volant mal identifié.

Elle a effectué quelques tours sur elle-même, avant d’atterrir sur l’épaule d’un bonze qui circulait en trottinette électrique sur le trottoir d’en face. Soyons clair : j’ai eu beau poser la question à tous les témoins qui avaient assisté à la scène, et dieu sait qu’il y en avait un paquet, aucun n’a été en mesure de m’expliquer ce que ce foutu bonze faisait là, en toute illégalité qui plus est, ce qui n’est à priori pas dans le style des bonzes, toujours respectueux des lois, adeptes de la discrétion et désireux de se fondre dans la foule (même si l’espèce de soutane orange dans laquelle ils se trimballent n’est sans doute pas le meilleur moyen d’y parvenir), sachant qu’on n’avait pas vu de bonze dans le secteur depuis au moins trois ou quatre siècles, en admettant qu’on en ait jamais vu un, raison pour laquelle personne ne s’attendait à en voir un, et encore moins au guidon d’une putain de trottinette électrique, le bonze n’étant généralement pas pressé et préférant faire usage de ses pieds, à l’ancienne, pour se transporter d’un point à un autre.

Le bonze a tourné la tête, vu la cervelle sur son épaule, poussé un cri (oui, les bonzes aussi poussent des cris, peut-être pas autant que les gens normaux, les anachorètes ou les membres des autres congrégations, mais ils en poussent aussi), tenté de s’en débarrasser, perdu le contrôle de sa trottinette, réussi de justesse à éviter un bac à fleurs, avant d’aller s’écraser sur une borne anti-stationnement, effectuer un vol plané d’anthologie et se retrouver les quatre fers en l’air au beau milieu de la chaussée. Tandis qu’il tentait de se relever, un bus est arrivé à pleine vitesse et lui a roulé sur la tête, laquelle a explosé comme une vieille citrouille pourrie en répendant son contenu sur le bitume.

Coupez !

En fait non, les choses ne sont exactement passées de cette manière.

Vous le savez comme moi, la réalité est une tambouille désespérément fade. Si on veut lui donner un semblant de saveur, il ne faut pas lésiner sur les épices. C’est d’ailleurs ce que la plupart des gens, dès qu’ils ont un moment de libre, s’emploient à faire.

Le mensonge, par exemple, ou le fait de travestir plus ou moins subtilement la vérité, sont des pratiques courantes en la matière.

Quand Roger Borniche (ex-comique troupier reconverti en flic à la Sûreté puis écrivain à succès) raconte comment il a serré Émile Buisson, alias l’ignoble Fatalitas, ennemi public numéro 1, dans la petite auberge de Normandie où ce dernier était tranquillement en train de déjeuner (pâté de campagne, avec salade et cornichons, tripes au calva, camembert et tarte aux fraises, il s’agit d’un scoop mondial puisque le contenu de ce déjeuner n’avait encore jamais été dévoilé, on se demande d’ailleurs bien pourquoi quand on sait à quel point les gens sont friands de ce genre de détails), il enjolive copieusement la scène, allant jusqu’à faire croire que c’est sa propre femme, Martine, qui a passé les menottes à Buisson (alors, je le rappelle, que deux autres flics, dont c’est le métier de menotter les gens, étaient présents dans la salle). Sacré Roger ! Non, en réalité, Martine n’était là que pour endormir la méfiance de Buisson, individu extrêmement dangereux en permanence sur le qui-vive, et en aucun cas risquer de prendre un mauvais coup en procédant elle-même à son arrestation.

Moi-même, en l’occurrence, qui n’ai rien à envier à Roger Borniche, à ceci près (manquerait plus que ça !) que je n’ai jamais été comique troupier, chansonnier ou agent de sécurité dans un grand magasin (il n’y a pas de sot métier, je vous l’accorde, mais ça montre bien à quel point l’approche du maintien de l’ordre était différente en ce temps-là, même si aujourd’hui encore il n’est pas rare que d’anciens acteurs de seconde zone accèdent aux plus hautes fonctions de l’État), ne rechigne pas à mettre un peu de piment dans le ragoût fadasse de l’existence, quitte à rétablir, une fois la supercherie découverte, l’exacte vérité des faits.

Dans le cas présent, je suppose que la présence d’un bonze en trottinette a dû sembler bizarre aux plus méfiants d’entre vous, d’autant qu’il n’y a aucun monastère dans les environs, et que même s’il y en avait un, il n’est pas du tout certain que les moines auraient l’autorisation d’utiliser un tel moyen de locomotion.

Je vous rassure tout de suite : il n’y en avait pas.

Pas à ma connaissance, en tout cas.

Par contre, le docteur Sébastien Charrier, lui, était bien là.

Je lui ai dit : Docteur Charrier ! Si je m’attendais à vous trouver ici !

\textsc{Charrier} : On se connaît ?

\textsc{Moi} : Vous ne vous rappelez pas ?

\textsc{Charrier} : Me rappeler de quoi ? On s’est déjà vu quelque part ?

\textsc{Moi} : L’affaire Alena Benesch, ça vous dit quelque chose ?

\textsc{Charrier} : Attendez voir… Mais oui, bien sûr, cette petite pute qui avait essayé de me faire chanter en prétendant que je lui avais fait bouffer la chatte par Hermann, mon dogue allemand !

\textsc{Moi} : Oui. C’est moi qui étais chargé de l’enquête. Il va bien, au fait ?

\textsc{Lui} : Qui ? Hermann ?

\textsc{Moi} : Oui.

\textsc{Lui} : La pauvre bête est morte de chagrin il y a quelques années de cela. Je crois qu’elle s’était beaucoup attachée à cette petite. Le coup de foudre existe aussi chez les animaux, vous savez.

\textsc{Moi} : J’ignorais.

\textsc{Lui} : Oui, je sais, on pense toujours que ce ne sont que des brutes épaisses incapables de sentiment. Eh bien il n’en est rien, ils sont beaucoup plus sensibles qu’on ne le pense.

\textsc{Moi} : Je suis vraiment désolé.

\textsc{Lui} : C’est gentil à vous. Je l’ai enterré au fond du jardin et vais quotidiennement me recueillir sur sa tombe.

\textsc{Moi} : C’est l’avantage d’avoir une grande propriété.

\textsc{Lui} : Vous connaissez la Frétoise ?

\textsc{Moi} : J’y suis allé une fois ou deux. Très bel endroit, à la fois authentique et élégant.

Lui, manifestement ému : Oui, un joyau historique rénové avec passion au cœur d’un environnement préservé. Nous sommes actuellement en train d’aménager le colombier pour y faire une chambre d’ami. C’est une bonne idée, vous ne trouvez pas ?

\textsc{Moi} : Excellente

\textsc{Lui} : Il faudra venir dîner un de ces soirs. Vous êtes marié ?

\textsc{Moi} : Non, pas encore.

\textsc{Lui} : Une fiancée, alors. Ravissante, je suppose. Il faudra penser à nous l’amener.

\textsc{Moi} : Je n’y manquerai pas.

\textsc{Lui} : Quoiqu’il en soit, pour en revenir à cette petite garce d’Alena Benesch, je continue de penser qu’elle n’a eu que ce qu’elle méritait. Il m’est arrivé de penser que j’avais peut-être été un peu trop dur avec elle. Je ne suis pas un mauvais homme, vous savez, et toujours de mon mieux pour réparer mes torts. Si torts il y a, bien entendu. La vérité, c’est que je ne me sens coupable de rien en ce qui la concerne. Toujours est-il que l’autre jour, croyez-le ou non, elle est venue sonner à ma porte en disant qu’elle était dans une misère noire et avait besoin d’un petit coup de main.

\textsc{Moi} : Et qu’est-ce que vous avez fait ?

\textsc{Lui} : Ça reste entre nous, mais ma chère épouse, qui est une vraie salope soit dit en passant, adore me regarder baiser avec une autre femme. On a tous nos petites manies, n’est-ce pas. Alena Benesch a pris un peu de poids, c’est vrai, mais elle est encore tout à fait comestible. Je lui ai proposé de la sodomiser sous les yeux de ma femme, si elle n’avait rien contre le fait de se faire défoncer le cul par un ancien interne des hôpitaux de Paris. En échange, je pourrais essayer de faire jouer mes relations pour qu’elle retrouve un poste dans une clinique privée. J’ai un ami qui adore les femmes plutôt bien en chair, un violeur notoire qui a pour habitude d’abuser de ses patientes quand elles sont dans les vapes. Je ne sais que ça ne se fait pas, mais c’est un ami et je ne me vois pas le balancer aux flics. D’autant que la plupart d’entre elles ne se souviennent de rien à leur réveil. Je vous choque ?

\textsc{Moi} : Un peu, oui. Et celles qui se souviennent, je peux savoir ce que vous en faites ?

\textsc{Lui} : On les envoie chez le psy.

\textsc{Moi} : Un ami à vous, je suppose ?

\textsc{Lui} : Evidemment. Comme elles n’ont que de très vagues souvenirs sur lesquels elles sont incapables de mettre un nom ou un visage, le psy leur explique qu’elles sont en pleine bouffée délirante, sans doute liée à l’un ou l’autre de ces traumas d’enfance mal gérés, ou pas gérés du tout parce que totalement passés sous les radars, qui refont surface après des années d’enfouissement, comme des saletés de zombies qui refoulent du bec et tentent de vous bouffer tout cru. Mais dieu merci, on n’est plus au Moyen Âge. De nos jours, on peut être cinglé sans se retrouver en train de griller sur un bûcher. Les chercheurs bossent comme des dingues, pour des salaires de misère, et nous disposons de molécules de plus en plus sophistiquées pour remettre un peu d’ordre dans les cerveaux détraqués.

\textsc{Moi} : N’empêche qu’il abuse d’elles.

\textsc{Lui} : Oui, on peut dire ça. Mais à ce moment-là, on peut aussi dire qu’il les viole avec ses doigts en procédant aux examens d’usage.

\textsc{Moi} : Oui, enfin, ce n’est pas tout à fait la même chose. Là, les viole carrément avec sa bite.

\textsc{Lui} : Je sais, c’est moche. Très moche, même, mais il prétend avoir un meilleur diagnostic avec sa verge qu’avec les autres outils dont il dispose, trop grossiers à son goût. Je sais que c’est faux, qu’il ment, qu’il se ment à lui-même, mais il n’est pire sourd que celui qui ne veut rien entendre. J’ai beau essayer de lui ouvrir les yeux, il s’obstine dans le déni. Je lui dis : Roman (il s’appelle Roman), mon ami, je t’en supplie, va voir un psy. Un jour ou l’autre, une patiente va se réveiller pendant que tu es en train de l’ausculter avec ta bite, en tout bien tout honneur, et il va en résulter un de ces putains de scandales qui éclaboussent la profession toute entière. Pense à tes collègues, tous ces gens qui se battent becs et ongles pour que les gens se bourrent de médocs jusqu’à cent ans. Il me répond : oui, je ferais bien quelques séances d’hypnose, mais j’ai peur de me faire violer pendant mon sommeil. Je ne fais aucune confiance à tous ces enfoirés de psys ! Vous le voyez, on n’en sort pas. À propos, vous êtes toujours dans la police ?

\textsc{Moi} : Oui, plus ou moins.

\textsc{Lui} : Dans ce cas, je compte sur votre discrétion. Vous savez ce que c’est : les gens sont méchants, ils n’aiment pas les riches. Dès que vous gagnez un peu plus de pognon qu’eux, ils font tout ce qui est en leur pouvoir pour vous mettre des bâtons dans les roues. Heureusement qu’ils n’en ont aucun, sinon celui de descendre dans la rue pour agiter des banderoles et se gargariser de slogans anticapitalistes, sans quoi ils seraient pires que tous ces dictateurs qui dirigent le monde d’une main de fer. Je peux savoir ce qui s’est passé, ici ?

\textsc{Moi} : Accident de la circulation.

\textsc{Lui} : Belle boucherie !

\textsc{Moi} : Oui. Vous pensez qu’elle s’en sortira ?

\textsc{Lui} : Contusions multiples, hémorragie interne au niveau de la cavité abdominale, possible trauma crânien, j’en passe et des meilleurs. Elle est dans la coma. Nul ne peut dire quand elle en sortira, si elle en sort, et encore moins dans quel état. Vous lui vouliez quoi, à cette petite ?

\textsc{Moi} : L’interroger. J’ai de bonne raison de penser qu’elle est impliquée dans la disparition d’un collègue de travail, qui se trouve aussi être un de mes plus proches amis.

\textsc{Lui} : J’ai moi-même perdu un excellent ami.

\textsc{Moi} : Vraiment ?

\textsc{Lui} : Oui, très récemment. Il s’appelait Rémi Durand et était gynécologue.

\textsc{Moi} : Ce nom me dit quelque chose.

\textsc{Lui} : Bien évidemment, que ça vous dit quelque chose. Si vous avez enquêté sur la Frétoise, vous n’ignorez pas que Rémi y passait quasiment tous ses week-ends, plus une bonne partie des vacances scolaires et ses congés de maternité.

\textsc{Moi} : Je vois. Qu’est-ce qui s’est passé, au juste ?

\textsc{Lui} : On l’a retrouvé pendu dans son garage.

\textsc{Moi} : Pendu ???!!!!!!!!!!

\textsc{Lui} : Dans son garage, oui.

\textsc{Moi} : D’après mon expérience personnelle, qui est tout de même loin d’être négligeable, il est assez rare que les gens se pendent dans leur garage, ou alors seulement s’ils sont victimes de harcèlement sur les réseaux sociaux ou viennent d’apprendre fortuitement qu’ils sont atteints d’une maladie grave qui ne leur laisse que quelques heures à vivre. Ils ont généralement bien trop de respect pour leur voiture pour lui imposer une telle humiliation.

\textsc{Lui} : N’est-ce pas. Au lieu de ça, Rémi était en pleine forme et venait tout juste de s’offrir une Porsche Cayman GT4 RS dont il était extrêmement fier.

\textsc{Moi} : C’est troublant, en effet.

\textsc{Lui} : Sincèrement, mon ami, vous pensez vraiment qu’un type qui a une Porsche Cayman GT4 RS dans son garage a la moindre envie de mettre fin à ses jours ?

\textsc{Moi} : Quelle couleur, la Cayman GT4 RS ?

\textsc{Lui} : Jaune, avec pack Clubsport, réservoir de 90 litres et trousse de premiers secours en alcantara !

\textsc{Moi} : Une merveille.

\textsc{Lui} : Absolue ! Vous êtes comme moi, n’est-ce pas ?

\textsc{Moi} : Comment ça ?

\textsc{Lui} : Vous ne croyez pas un instant à la thèse du suicide.

\textsc{Moi} : Le suicide d’un gynéco bien dans sa peau qui vient de s’offrir une Porsche Cayman GT4 RS jaune avec pack Clubsport, réservoir de 90 litres et trousse de premiers secours ?

\textsc{Lui} : Oui, et toutes les options disponibles, l’intérieur full cuir bien évidemment, mais aussi les rétros extérieurs à capteurs de pluie, la reconnaissance des panneaux de signalisation, les tapis de sol en peau de fesse, les jantes en or massif, le système audio Bose et les ceintures de sécurité couleur chair !

\textsc{Moi} : Terrifiant ! J’ai beau tourner et retourner le problème dans tous les sens, je ne vois pas comment un type qui a tout ça dans son garage pourrait avoir la moindre envie de se suicider.

\textsc{Lui} : Et imaginez que ce même type soit marié à une jeune femme somptueuse dont il pourrait être le père ! Vous croyez vraiment qu’un tel homme aurait envie de mettre fin à ses jours ?

\textsc{Moi} : En aucun cas. Vous avez contacté le procureur ?

\textsc{Lui} : Bien évidemment, vous me prenez pour qui ! J’ai exigé qu’une expertise médico-légale soit diligentée dans les plus brefs délais, et j’ai demandé à y assister personnellement.

Moi, sortant un Hemingway Short Story de ma poche : Vous avez raison, on n’est jamais mieux servi que par soi-même. Vous fumez ?

\textsc{Lui} : Vous avez enquêté sur moi, non ?

\textsc{Moi} : Un peu, oui. Enquête de routine, vous savez ce que c’est.

\textsc{Lui} : Dans ce cas, vous devez savoir que je ne crache pas sur un petit cigare de temps à autre.

\textsc{Moi} : Et c’est tout à votre honneur. Tu en veux un aussi, Greg ?

\textsc{Greg} : Je ne voudrais surtout pas déranger.

\textsc{Moi} : Mais pas du tout, voyons, qu’est-ce que tu vas imaginer. C’est juste que comme je sais que tu ne fumes pas, ou quasiment pas, je me suis dit qu’il n’était peut-être pas complètement indispensable de te proposer un cigare.

\textsc{Greg} : Je ne fume pas en temps normal, c’est tout à fait vrai. Mais aujourd’hui, les circonstances sont suffisamment exceptionnelles pour que je fasse une exception à la règle.

\textsc{Moi} : C’est la raison pour laquelle, en dépit des éléments dont je viens de te faire part, que je me suis permis de te demander si tu voulais toi aussi un cigare.

\textsc{Greg} : Dans ce cas, je te répondrai ceci : oui, Djef, même s’il est vrai que je ne fume pas ou quasiment pas, c’est avec le plus grand plaisir que je vais accepter le cigare que tu m’offres si gentiment.

\textsc{Moi} : Bien. On en était où, nous ?

\textsc{Charrier} : Vous veniez de me proposer un cigare.

\textsc{Moi} : Que vous aviez accepté, c’est bien ça ?

\textsc{Lui} : C’est bien ça.

\textsc{Moi} : Et avant ?

\textsc{Lui} : Avant, on était en train de parler de la mort suspecte de mon ami Rémi Durand, gynécologue ayant pignon sur rue qui n’avait aucune raison de mettre fin à ses jours, et ce d’autant moins qu’il était marié à très jolie fille beaucoup plus jeune que lui et venait tout juste de s’offrir une Porsche Cayman GT4 RS jaune avec pack Clubsport, réservoir de 90 litres et trousse de premiers secours en alcantara, le nec plus ultra en matière de chic automobile.

\textsc{Moi} : Vous pensez qu’on l’a tué ?

\textsc{Lui} : Je ne vois pas d’autre explication.

\textsc{Moi} : Il s’agit peut-être d’un accident.

\textsc{Lui} : Vous plaisantez ?

\textsc{Moi} : On ne sait jamais. Imaginez un type qui décide d’aller dans son garage pour bricoler un truc au plafond, accrocher une corde, par exemple. Il monte sur un tabouret, et, pour une raison ou pour une autre, perd l’équilibre et se retrouve avec la corde enroulée autour du cou. Il se débat, tente désespérément de se raccrocher au tabouret. Mais il donne un coup de pied dans le tabouret, le tabouret tombe, et notre homme se retrouve pris au piège.

\textsc{Lui} : Ridicule !

\textsc{Moi} : Ou alors, il a peut-être eu, pendant un bref instant, l’idée de mettre fin à ses jours. On croit connaître ses amis, et on découvre parfois qu’ils nous ont caché des choses pendant des années. François Vérove, alias «le Grêlé», a vécu pendant trente-cinq sans attirer les soupçons. Ancien gendarme, bon père de famille, qu’est-ce que vous croyez qu’il s’est passé quand ses proches ont appris qu’il s’agissait en fait d’un tueur en série pédophile de la pire espèce, responsable d’au moins une bonne demi-douzaine de meurtres, et sans doute beaucoup plus, la liste exacte de ses victimes n’ayant jamais pu être établie avec certitude ?

\textsc{Lui} : Il sont tombés des nues, je suppose.

\textsc{Moi} : Peut-être que votre ami Rémi vous cachait des choses, lui aussi. Vous seriez surpris d’apprendre le nombre de gens qui ont une double vie.

\textsc{Lui} : Vous insinuez que Rémi était un tueur en série pédophile ?

\textsc{Moi} : Pas le moins du monde. Enfin, on ne sait jamais. Vous m’avez dit qu’il avait une femme beaucoup plus jeune que lui. Peut-être qu’elle le trompait et qu’il ne l’a pas supporté.

\textsc{Lui} : Elle ne l’a jamais trompé. Elle couchait avec d’autres hommes, c’est vrai, mais Rémi était parfaitement au courant et couchait lui aussi avec d’autres femmes, en toute transparence.

\textsc{Moi} : La vôtre, par exemple.

\textsc{Lui} : Par exemple. Mais j’ai souvent couché avec la sienne. J’ai toujours considéré comme une chose parfaitement normale que mes amis couchent avec ma femme, et m’autorisent à faire de même avec la leur. Je ne sais pas si vous êtes marié, lieutenant…

\textsc{Moi} : Commandant.

\textsc{Lui} : … commandant, mais si vous l’êtes vous devez savoir à quel point il est rébarbatif de coucher toujours avec la même femme, quel soit l’amour qu’on lui porte. L’amour et le sexe sont deux choses totalement différentes, qui ne devraient rien avoir à faire ensemble. Le désir est une chose, l’amour en est une autre, et la confusion qui règne entre les deux est la source de nombreux problèmes. Vous pouvez vivre avec quelqu’un toute votre vie, et continuer à l’aimer, mais certainement pas à le désirer comme au premier jour. Si certains y arrivent, tant mieux pour eux, mais moi ce n’est pas mon cas. Je ne vois pas au nom de quoi je devrais m’interdire d’être attiré par d’autres femmes et de coucher avec elles si le désir est réciproque. Quel spectacle pathétique de voir tous ces vieux types, mariés depuis des siècles à une femme qui ne ressemble physiquement plus à rien, baver comme des malades sur le cul des gamines qui passent à leur portée. Mais vous savez ce qui me fait le plus marrer ?

\textsc{Moi} : Non, dites-moi.

\textsc{Lui} : C’est de voir les jeunes mariés errer dans les rayons des supermarchés.

\textsc{Moi} : Mais encore ?

\textsc{Lui} : Madame marche en tête, et monsieur suit, l’air grognon et la mine déconfite, poussant un caddie rempli jusqu’aux ouïes de couches-culottes, lait en poudre et toute une ribambelle de produits ultra transformés bourrés d’huile de palme hydrogénée, amidon modifié, agents de texture, isolats de protéines et autres perturbateurs endocriniens diversement cancérigènes. Il a pourtant toutes les raisons d’être heureux, le jeune père de famille : il vient d’acheter un pavillon avec un petit lopin de terre pour faire pousser deux patates et trois petits pois, de changer de bagnole et d’accéder aux joies de la paternité. Seulement voilà, madame vient de prendre vingt kilos en neuf mois et il sait pertinemment qu’elle ne réussira jamais à s’en débarrasser. D’autant qu’elle ne fera pas le moindre effort pour ça, pour la bonne et simple raison qu’elle veut un autre enfant et part du principe que c’est pas la peine de suer sang et eau pour perdre vingt kilos si c’est pour en reprendre trente dans la foulée. Et elle se dit aussi que si son mari l’aime, il aura toujours autant envie d’elle même si elle ressemble à un éléphant de mer. Du coup, elle va garder ses vingt kilos et en reprendre une bonne vingtaine de plus pendant sa prochaine grossesse, perdant définitivement toute chance de retrouver un jour sa taille d’antan. Résultat des courses : ils vont commencer à s’engueuler, divorcer dans un an, deux ou trois si tout va bien, et madame va aller s’inscrire à la salle de sport du coin dans l’espoir de trouver un nouveau pigeon pour lui témoigner un peu d’affection. Vous ne trouvez pas que ça fait froid dans le dos ?

\textsc{Moi} : Vu comme ça, ce n’est effectivement pas très engageant.

\textsc{Lui} : Franchement répugnant, vous voulez dire !

\textsc{Moi} : J’aimerais qu’on en revienne à Rémi. Vous n’avez rien remarqué de bizarre pendant les jours ou les semaines qui ont précédé son décès ?

\textsc{Lui} : Non, rien du tout. Je vous le répète, Rémi allait parfaitement bien et n’avait aucune raison de mettre fin à ses jours. Il est évident que quelqu’un l’a tué en essayant de faire passer le crime pour un accident, et je ne doute pas que l’analyse médico-légale le confirmera.

\textsc{Moi} : Vous lui connaissez des ennemis ?

\textsc{Lui} : Les riches ont des tas d’ennemis.

\textsc{Moi} : Vous, peut-être ?

\textsc{Lui} : Moi ? Vous êtes fou !

\textsc{Moi} : Vous m’avez dit qu’il couchait avec votre femme. On a vu des gens en tuer d’autres pour moins que ça.

\textsc{Lui} : Je vous ai dit aussi que je m’en foutais, et que je couchais aussi avec la sienne. Il nous arrivait aussi de coucher tous ensemble, si vous voulez tout savoir.

\textsc{Moi} : Drôles de pratiques.

\textsc{Lui} : Je ne vous demande pas de vous joindre à nous.

\textsc{Moi} : Je peux vous poser une question ?

\textsc{Lui} : Si vous y tenez. Au fait, je ne sais pas si je vous l’ai dit, mais ce cigare est excellent. Il vient de Cuba, je suppose.

\textsc{Moi} : Non, de République dominicaine. Vous avez entendu parler d’Arturo Fuente ?

\textsc{Lui} : Non.

\textsc{Moi} : C’est un de ces émigrés espagnols qui ont fui Cuba pendant la guerre hispano-américaine. Il a atterri à Tampa, en Floride, et s’est lancé dans la fabrication de cigares avec des feuilles en provenance de Cuba. Une production d’abord confidentielle, limitée à quelques milliers de cigares par an roulés dans le salon et la cuisine par les membres de la famille. Ensuite, quand ça commençait à plutôt bien marcher, il y a eu le Che, Castro et la révolution cubaine, avec pour conséquence la rupture des relations diplomatiques entre Cuba et les USA. Du coup, la manne cubaine s’est tarie. Fuente a donc commencé à se fournir au Mexique et à Porto Rico, avec un succès mitigé, avant de changer son fusil d’épaule et aller s’installer au Nicaragua, nouvel Eldorado du cigare et dictature bananière sous contrôle américain. Mais à la fin des années 70, la révolution sandiniste a éclaté, les Somoza ont été foutus à la porte, et la fabrique Fuente, emblème d’une époque révolue, a été entièrement détruite par les flammes. Nouvel exil à Santiago, en République dominicaine, avec pour tout bagage un solide savoir-faire et une volonté farouche de tout casser. Aujourd’hui, associée à la famille Newman, Fuente produit des dizaines de millions de cigares par an, dont quelques uns parmi les plus réputés et onéreux de la planète. Celui que vous êtes en train de fumer, par exemple, le Short Story de la série Hemingway, est un vibrant hommage à l’écrivain qui a passé une bonne partie de sa vie à Cuba et appréciait tout particulièrement les Cohiba, une des plus prestigieuses marques de cigares.

\textsc{Lui} : Les meilleurs, à ce qu’il paraît.

\textsc{Moi} : Les plus chers, en tout cas. Oui, c’est ce que disent les snobs qui fument pour se donner un genre et n’y connaissent rien. C’était sans doute vrai avant l’embargo, et jusqu’à la fin des années 70 ou 80, mais depuis le cigare a fait son chemin un peu partout dans le monde et l’hégémonie cubaine n’est plus d’actualité, notamment en ce qui concerne le rapport qualité-prix. En cause le développement des marchés internationaux comme l’Inde, la Chine et le Moyen-Orient. Le pays n’arrive plus à suivre, avec pour conséquences une tendance à la surproduction, une baisse notoire de la qualité de fabrication et une hausse constante des prix. Les cigares autrefois abordables ont pulvérisé toutes les limites de la décence tarifaire. Le SIGLO VI de Cohiba, par exemple, grand classique s’il en est, se négocie aujourd’hui aux alentours de cent-dix euros pièce, et certaines séries spéciales montent à trois ou quatre cent. D’autre part, même s’il représente toujours dans l’imaginaire collectif la référence absolue en matière de cigare, force est de constater que le havane peine à se renouveler, innover tant sur la plan de la forme que du fond, tandis que les autres rivalisent de créativité pour exciter les papilles du consommateur.

\textsc{Lui} : Si vous le dites.

\textsc{Moi} : Je l’affirme haut et fort et ne cesserai de le clamer jusqu’à mon dernier souffle, n’en déplaise aux crétins prétentieux qui ne jurent que par le havane !

\textsc{Lui} : Grand bien vous fasse.

\textsc{Moi} : Maintenant, si vous le voulez bien, j’aimerais vous poser une petite question. Rien de personnel, rassurez-vous.

\textsc{Lui} : De quoi s’agit-il ?

\textsc{Moi} : De l’individu assis au volant de cette voiture.

Je parlais de Noé Desmarais, qui n’avait pas bougé un cil depuis le début de la conversation.

\textsc{Lui} : Vous voulez savoir s’il est mort, c’est ça ?

\textsc{Moi} : J’aimerais bien, oui.

\textsc{Lui} : C’est un ami à vous ?

\textsc{Moi} : Pas exactement, mais c’est une histoire un peu longue à raconter. Tout ce que je peux vous dire, c’est qu’il arrivait en face quand la Mini a essayé de doubler la Fiesta. Et ça a fait un grand BOUM !

Lui, agitant le truc qui pendait à son cou : Vous savez ce que c’est ?

\textsc{Moi} : Oui, un stéthoscope.

\textsc{Lui} : Mais pas n’importe lequel. C’est un Redmann, la Rolls du stéthoscope. Avec ça, vous pouvez entendre respirer un moucheron et battre le cœur d’un ver de terre, qui en possède cinq soit dit en passant.

\textsc{Moi} : Et alors ?

\textsc{Lui} : Alors cet homme est mort, il n’y a aucun doute là-dessus.

\textsc{Moi} : Vous en êtes sûr ?

\textsc{Lui} : Sûr et certain. Vous n’oseriez tout de même pas mettre en doute mes compétences ?

\textsc{Moi} : Loin de moi cette idée absurde.

\textsc{Lui} : Dans ce cas, vous ne m’en voudrez pas de prendre congé. J’ai encore des tas de vie à sauver qui m’attendent.

FIN

Provisoire, bien entendu.

Il est toujours extrêmement douloureux de mettre un point final à un récit qui a occupé de longs mois de votre existence, pompé une bonne partie de votre énergie et mobilisé toutes vos facultés créatrices. C’est comme dire au revoir à un vieil ami, lui serrer une dernière fois la main sans savoir si on le reverra un jour. On la garde longtemps dans le creux de la sienne, comme un petit animal blessé, on se refuse obstinément à la lâcher. Et puis on rentre chez soi, triste, et on avale une belle assiette de rognons de veau à la crème pour se donner une contenance, tenter d’oublier que toutes les choses ont une fin, les meilleures comme les pires, ce qui est une bonne chose pour les pires mais moins bonne pour les bonnes.

Sous la pression de mes fans, qui commencent à trouver le temps long (vous m’avez manqué, vous aussi), et surtout de mon éditeur qui a grand besoin de renflouer les caisses de sa modeste entreprise (les temps sont durs pour tout le monde, et ce serait pour lui un crève-cœur de devoir vendre son Ferretti Custom Line 97 ou hypothéquer sa villa de Saint-Raphaël pour sauver les meubles), je me vois dans l’obligation de remettre à plus tard un certain nombre des affaires en cours.

Je pense notamment à Jaya, la fille adorée de mon ami Zaahid Shirani, tombée entre les griffes d’un certain Simon Keskula, guide spirituel et maître incontesté d’un secte post-apocalyptique connue sous le nom d’Alliance de la Révélation. J’ai promis à Zaahid de tout mettre en œuvre pour que Jaya rentre au bercail et que le monstre qui la tenait sous sa dépendance soit définitivement mis hors d’état de nuire. Et comme je suis un homme de parole, je vais faire ce que j’ai dit. Et surtout, je ne manquerai pas de vous narrer par le détail comment, par quel stratagème machiavélique, ruse subtile et technique d’infiltration digne des meilleurs services de renseignement, et accessoirement usage immodéré de la force, sinon la violence la plus aveugle et éthiquement condamnable, je serai parvenu à mes fins.

En attendant, je sais qu’une double question vous ronge la cervelle aussi sûrement qu’un rat affamé s’attaque à un morceau de gruyère ou un vieux quignon de pain : Repentance Whittingham, alias la Gardienne de la Nuit ou la femme de ménage la plus rapide du monde, est-elle sortie du coma, et quid de Titus Beaugendre, porté disparu après une rencontre tant fortuite que suspecte avec la demoiselle en question ?

Eh bien… on n’en sait trop rien, à vrai dire, mais je ne manquerai pas de vous le faire savoir si j’apprends quelque chose à ce sujet. Tout ce que je sais pour l’instant, et c’est assez maigre je vous le concède, c’est qu’elle a disparu du jour au lendemain de sa chambre d’hôpital. Et comme je doute fort qu’elle ait été capable de le faire par ses propres moyens, l’action d’un tiers n’est pas à exclure.

Pour ce qui est de Titus, je vous propose un petit flashback juste avant le mot FIN, au moment où le docteur Charrier nous a annoncé que lui et son Redmann, la Rolls du stéthoscope, étaient formels sur le fait que Noé Desmarais ne ferait plus jamais joujou avec les allumettes, ni d’ailleurs avec quoi que ce soit d’autre, l’envie de faire joujou avec quelque objet ou organe que ce soit lui étant définitement passée. Desmarais refroidi, la joyeuse petite bande de néonazis des Disciples de la Colère était en partie démantelée. Ne restait plus qu’à exterminer les sieurs Monteil et Jégou, individus peu recommandables auxquels j’entendais bien réserver un traitement à la hauteur de leurs exploits. Pour ce faire, je m’étais dit que ça pourrait être sympa de transformer ma salle de bain en chambre à gaz. On enlevait Monteil et Jégou, on les obligeait à revêtir un pyjama rayé, on les affamait pendant quelques semaines, puis, quand ils commençaient à sentir si mauvais que même les mouches à merde s’enfuyaient à tire-d’aile à leur approche, on leur offrait une petite douche gratuite au Zyklon B, le célèbre insecticide à base de cyanure de la Deutsche Gesellschaft fur Schadlingsbekampfung (il faut reconnaître que les Allemands ont un certain talent pour créer des mots de quinze kilomètres de long totalement imprononçables pour toute personne non germanique, à tel point que je me demande si ce n’est pas une des raisons principales de leur manque de popularité et relatif isolement sur la scène internationale, outre le fait qu’ils construisent des voitures rapides très appréciées des trafiquants de drogue, boivent beaucoup de bière et sont nuls en cuisine). Après quoi on en faisait des steaks hachés, merguez, saucisses et andouillettes, et on organisait une grande fiesta dans le quartier avec barbecue à gogo jusqu’à épuisement des stocks. On joignait l’utile à l’agréable, et je doute fort que la police s’amuserait à aller fourrer son nez dans les cuvettes de chiottes du voisinage. Je ne connais pas les effets du piment et des épices sur les composés organiques, mais je suppose qu’il est tout à fait possible de rechercher des traces d’ADN dans une merguez ou une chipolata. Si on le faisait plus souvent, quelque chose me dit qu’on pourrait avoir des surprises de taille.

\textsc{Greg} : Je vais appeler Sally Robinson. Elle sera sûrement ravie d’apprendre que Desmarais n’est plus de ce monde.

\textsc{Moi} : Et moi je vais appeler Bérénice pour lui dire qu’on est toujours sans nouvelles de Titus.

\textsc{Greg} : C’est quand même bizarre, cette histoire.

\textsc{Moi} : On nage en pleine absurdité.

\textsc{Greg} : Je me pose des questions.

\textsc{Moi} : À propos de quoi ?

\textsc{Lui} : Titus. Il aime sa femme, ses gosses, je ne comprends pas pourquoi il a suivi cette fille sans discuter.

\textsc{Moi} : C’est toujours pareil : on pense connaître les gens, et on découvre des zones d’ombre qu’on ne soupçonnait pas.

\textsc{Greg} : Tu penses qu’il est encore en vie ?

\textsc{Moi} : Je n’en ai aucune idée. Peut-être qu’il est enchaîné quelque part, en train de se faire violer et torturer par une bande de suprémacistes blancs homosexuels.

\textsc{Greg} : Ou alors il a décidé que sa vie ne correspondait plus à ses rêves et il a foutu le camp sans laisser d’adresse.

\textsc{Moi} : Il m’a souvent parlé de son envie de retourner en Sierra Leone pour renouer avec ses origines et retrouver la trace de ses ancêtres. Je crois savoir que l’idée ne plaisait pas plus que ça à Bérénice.

\textsc{Greg} : Il t’avait semblé bizarre, ces derniers temps ?

\textsc{Moi} : Pas plus que d’habitude. Il a toujours été un peu mystique. Il y a quelque temps, il s’est découvert une passion pour Edward Tylor, un anthropologue british du XIXème qui s’intéressait à l’animisme, au totémisme et à la cosmogonie. Titus potassait aussi des trucs sur la culture yoruba et les babalaos.

\textsc{Greg} : Les quoi ?

\textsc{Moi} : Les babalaos. Ce sont des espèces de sorciers qui pratiquent la divination avec des noix de palmier.

Pendant ce temps, les pompiers avaient découpé la Mini pour extraire Repentance Whittingham de son cercueil de tôle.

La femme de ménage la plus rapide du monde avait ensuite été transportée toutes sirènes hurlantes vers l’hôpital le plus proche.

\textsc{Greg} : Et Bérénice ?

\textsc{Moi} : Quoi, Bérénice ?

\textsc{Greg} : Tu lui as parlé, récemment ?

\textsc{Moi} : Bien sûr, que je lui ai parlé récemment ! Elle m’appelle toutes les dix secondes pour savoir ce qui est arrivé à son cher et tendre. D’ailleurs c’est bizarre, ça fait un moment qu’elle ne m’a pas rappelé.

\textsc{Greg} : Elle n’a rien remarqué de particulier, elle non plus ?

\textsc{Moi} : Elle, non. Mais moi, oui.

\textsc{Greg} : Comment ça ?

\textsc{Moi} : J’ai remarqué quelque chose de particulier. Pas à propos de Titus, mais de Bérénice, justement.

\textsc{Greg} : Ah bon ?

\textsc{Moi} : Oui. Depuis quelque temps, Bérénice a un comportement inhabituel, que je ne lui connaissais pas auparavant.

\textsc{Greg} : Tu peux être plus précis.

\textsc{Moi} : C’est assez indéfinissable, en fait. C’est quelque chose dans son attitude générale, sa façon de s’habiller, de parler, et même de se déplacer.

\textsc{Lui} : Elle ne se déplace plus comme avant ?

\textsc{Moi} : Elle se déplace bizarrement, avec une démarche plus féline que d’habitude.

\textsc{Greg} : Plus féline ?

\textsc{Moi} : Oui, plus féline. Je te l’ai dit, c’est assez indéfinissable. Cela dit, en la voyant, je me suis dit que quelque avait changé en elle, comme si elle n’était plus tout à fait la même.

\textsc{Greg} : Des soucis personnels, peut-être.

\textsc{Moi} : Même son odeur n’est plus la même.

\textsc{Greg} : Elle a peut-être tout simplement changé de parfum.

\textsc{Moi} : Ne sois pas idiot. Si tu veux vraiment que je te dise ce que je pense…

\textsc{Greg} : Oui, j’aimerais bien.

\textsc{Moi} : … je vais te le dire, mais arrête de me couper sans arrêt ! Si tu veux vraiment que je te dise ce que je pense, je ne serais pas étonné qu’elle voie quelqu’un d’autre.

\textsc{Greg} : Non ???!!!

\textsc{Moi} : Si.

\textsc{Greg} : Mais alors, ça change tout !

\textsc{Moi} : Comment ça ?

\textsc{Greg} : Si elle voit quelqu’un d’autre, il se peut que Titus soit au courant, ce qui expliquerait son comportement bizarre de ces derniers temps.

\textsc{Moi} : Titus n’était pas bizarre ces derniers temps. Il n’est devenu bizarre qu’hier soir, quand cette fille a débarqué de nulle part. Dès qu’elle est entrée dans son champ de vision, c’était comme si son esprit avait quitté son corps pour aller gambader tel un guépard dans les vastes plaines du Kalahari, royaume de ceux qui suivent l’éclair et ramassent par terre. Tu te rappelles quand elle lui a dit «toi venir avec moi dans chambre», l’étrange lueur qu’il y avait dans ses yeux quand elle prononcé ces mots ?

\textsc{Greg} : Les yeux de qui ?

\textsc{Moi} : Ceux de Titus, bien sûr ! Tu le fais exprès, ou quoi ?

\textsc{Lui} : Excuse-moi, mais j’ai un peu de mal à te suivre dans tes divagations. Et non, je n’ai pas vraiment fait attention à la lueur qu’il y avait dans les yeux de Titus quand elle a prononcé ces mots. Je devais être occupé ailleurs. Cela dit, si Bérénice a effectivement un amant comme tu le prétends, ça pourrait expliquer bien des choses.

\textsc{Moi} : Je ne prétends rien du tout. Je dis juste qu’il n’est pas totalement impossible, compte tenu des éléments dont je dispose, que Bérénice ait un amant.

\textsc{Greg} : Quels éléments ?

\textsc{Moi} : Des éléments…

\textsc{Greg} : Tu veux dire des éléments vraiment… vraiment…

\textsc{Moi} : Compromettants, oui. Mais ce serait peut-être à toi de m’en dire un peu plus sur le sujet…

Greg, feignant la surprise : À moi ?

\textsc{Moi} : Oui, à toi. Je crois savoir que Bérénice est une excellente danseuse de tango…

\textsc{Greg} : C’est possible, en effet.

\textsc{Moi} : Ne fais pas comme si tu ne le savais pas.

\textsc{Greg} : J’en ai vaguement entendu parler.

\textsc{Moi} : Un peu plus que vaguement, je pense. Je crois savoir aussi que tu t’es toi-même découvert une passion pour Carlos Gardel, Rodolfo Biagi, Francisco Canaro, El Pibe et Anibal «El Gordo» Troilo, pour n’en citer que quelques uns.

Greg, de plus en plus mal à l’aise : One day i was at home, et non pas reincarnated as the 7th Prince, en train de bouffer des raviolis en boîte, quand j’ai entendu une chanson de La Tana, Susana Rinaldi de son vrai nom, à la radio. Stupeur et tremblements, un déclic s’est produit dans ma tête. J’ai aussitôt balancé mes raviolis à la poubelle et décidé de changer de vie. Ma passion pour le tango était née.

\textsc{Moi} : Et tu t’es acheté une paire de Danilo Pinto à mille cinq cent balles !

\textsc{Greg} : Quand on aime on ne compte pas.

\textsc{Moi} : Des Danilo Pinto dont tu ne te sépares quasiment plus. La preuve, tu les as aux pieds ce matin, alors qu’on n’était pas franchement partis pour danser le tango.

\textsc{Greg} : Elles sont très confortables.

\textsc{Moi} : Je n’en doute pas. Mais ça n’explique pas pourquoi tu t’es inscrit dans la même école de danse que Bérénice…

\textsc{Greg} : Je me trompe ou tu es en train d’insinuer des choses pas très catholiques à mon sujet ?

\textsc{Moi} : Je n’insinue pas, j’affirme. J’affirme, mon cher Grégoire, que Bérénice et toi dans comme des Argentins jusque tard dans la nuit au Palazzo Cristal Club de la rue Féret. Des Argentins désargentés, peut-être, mais des Argentins tout de même.

\textsc{Greg} : C’est faux !

Moi, d’un calme olympien : Non, c’est authentiquement vrai.

\textsc{Lui} : Comment tu peux savoir ça ?

\textsc{Moi} : Je le sais, c’est tout.

\textsc{Lui} : C’est Bérénice qui te l’a dit ?

\textsc{Moi} : Ah ah ! Donc tu admets que Bérénice et toi vous trémoussez comme des dingues au Palazzo ?

\textsc{Lui} : On danse, c’est tout.

\textsc{Moi} : Oui, avec la femme de Titus. Titus à qui tu t’es bien gardé de dire que tu dansais le tango avec sa femme. Reconnais que tes méthodes sont particulièrement sournoises.

\textsc{Greg} : Je suppose qu’il est au courant.

\textsc{Moi} : Non, il ne l’est pas. Enfin, pas officiellement. Et ce n’est pas tout.

Greg a ouvert la bouche. J’ai senti qu’il voulait dire quelque chose, plaider sa cause, mais il s’est aussitôt ravisé, comprenant qu’il avait le droit de garder le silence et que tout ce qu’il dirait pourrait être retenu contre lui.

J’ai donc repris le fil de mon réquisitoire : On vous a vu sortir main dans la main et vous roulez des pelles sur le trottoir !

Greg a bruyamment ravalé la salive qui affluait dans le fond de sa gorge : Pardon ?

L’accusation était ignoble, je l’admets, mais hélas non dépourvue de fondement.

\textsc{Moi} : Pelles, galoches, palots, patins, appelle comme tu veux, toujours est-il qu’on vous vu vous boulotter les amygdales !

Lui, feignant assez maladroitement la stupéfaction (même un acteur de sitcom ou téléfilm français aurait largement fait mieux) : Hein ? Quoi ? Comment ? On ? Qui ça, on ?

Comme un bon journaliste ne cite jamais ses sources, un bon flic ne révèle jamais le nom de ses tontons.

Mais à vous, à toi, lecteur (je pense qu’on en est arrivé à un stade de notre relation où on peut raisonnablement envisager de se tutoyer), je peux bien le dire.

Je te connais suffisamment pour savoir, si je ne le fais pas, que tu vas psychoter, te creuser les méninges et douter de tout, y compris de Dieu, l’abbé Pierre, Donald Trump et Vladimir Poutine, t’interroger sur le sens de la vie, les dangers de la 5G, la réalité augmentée, du deep learning et de l’IA générative, jusqu’à en perdre le boire et le manger, le sommeil aussi, passer tes nuits les yeux grand ouverts à fixer le plafond, le cœur battant, à suer sang et eau dans le fond du lit conjugal de moins en moins conjugué, insensible aux appels désespérés de ta femme pour que tu t’intéresses enfin un peu à elle, ses désirs, ses rêves, son corps à l’abandon. Je sais que tu vas errer sans but dans les rues de la ville, telle une bête malade, chercher refuge dans des lieux de perdition, boire plus que de raison, et que tes amis, lassés de tes absences et ton humeur chagrine, vont te quitter un par un. Puis ce sera au tour de ta moitié, devenue au fil du temps ton tiers, ton quart, ton huitième, une fraction de plus en plus insignifiante de toi-même, de faire ses valises. Ayant épuisé les trésors de patience dont elle disposait, elle retournera chez sa mère, laquelle en sera enchantée pour au moins trois raisons : 1. elle vit seule avec sa chatte et se fait chier comme un rat mort ; 2. elle a toujours été extrêmement possessive et aurait voulu garder sa fille pour elle toute seule ; 3. elle t’a toujours considéré comme une erreur de la nature et un bon à rien. Mais ta femme ne partira pas seule : elle emportera avec elle la seule chose qui ne retenait encore accroché du bout des doigts à l’existence, je veux bien sûr parler de tes trois enfants. Quand tu rentrais à la maison, après ce qui aurait dû être une dure journée de labeur, ils te donnaient du «monsieur», non parce que vous êtes de la haute et avez conservé les usages de la vieille noblesse française (vous êtes des roturiers de la pire espèce, sans une once de sang royal dans les veines), mais parce qu’ils ne te reconnaissaient tout simplement plus. Et quand je dis «aurait dû être» une dure journée de labeur, c’est parce que tu as perdu ton boulot, bien sûr, et passes tes journées assis sur un banc à ruminer tes vilaines et vaines pensées, et envisager le moyen le plus expéditif de mettre un terme à tes souffrances. Tel Jean-Claude Romand, tu iras toquer à la porte des moines de la congrégation de Solesmes, à Notre-Dame de Fontgombault, mais ceux-ci, persuadés d’avoir affaire à un suppositoire de Satan, refuseront de t’accueillir. Par charité chrétienne, sur l’insistance du père-abbé qui est un homme profondément bon et ouvert d’esprit, cette bande de clowns tonsurés te proposera toutefois une place de jardinier au monastère Saint-Joseph de Séguéya, en Guinée-Conakri, poste que tu auras la fermeté de refuser d’une part parce que tu n’as pas la main verte, d’autre part parce que tu as toujours eu une peur bleue des Noirs, lesquels restent pour toi des créatures étranges aux yeux globuleux et à la bouche remplie de dents. C’est ainsi, seul sur le canapé du salon, la libido en berne devant les chaînes d’info en continu qui achèveront de te liquéfier le cerveau, un vieil exemplaire du Désespéré de Léon Bloy à portée de main, entouré de bouteilles vides, de toiles d’araignées et de déchets alimentaires en voie de putréfaction, tu dépériras lentement dans l’indifférence générale, après que ton père t’ait déshérité et ta mère se soit désagrégée de chagrin comme une vieille chaussette pourrie. Jusqu’au jour où tes voisins (avec lesquels tu n’as jamais entretenu la moindre relation), incommodés par l’odeur, appelleront la police qui se fera une joie de venir défoncer ta porte et découvrir avec horreur ton cadavre grouillant d’asticots.

Aussi, parce que je te suis reconnaissant des efforts que tu as consentis pour arriver jusqu’ici, je vais te le dire.

En fait, aussi étrange et stupéfiant que cela puisse paraître, je tenais cette information capitale de la bouche même de Sam Girard, cuisinier hors pair et ancien des forces spéciales de l’armée de terre, qui se trouvait résider non loin du Palazzo Cristal Club. C’est là, depuis le bar où il avait coutume de se désaltérer et refaire le monde avec d’autres habitués, assez rugueux pour la plupart, qu’il les avait distinctement aperçus en train de se livrer à des activités qui avait plus à voir avec les bordels de Buenos Aires qu’avec le Rio de la Plata et ses danses locales. Prudent, et habitué au secret comme la plupart des militaires, il ne s’en était ouvert qu’à moi. Pourquoi moi ? Eh bien mais tout simplement parce que je suis une personne de confiance, comme vous n’aurez sans doute pas manqué de le remarquer. Si vous êtes dans la détresse, affective ou autre, je saurai trouver les mots justes pour remettre sur les rails la locomotive cabossée de votre existence. Je dis ça, mais je n’ignore pas que c’est dans la nature des gens de raconter leur vie au premier venu. Ils ne vous voient pas comme un être humain, mais comme une benne à ordures dans laquelle déverser leurs immondices. Ce sont les mêmes qui abandonnent leurs papiers gras sur la chaussée et leurs mégots de clopes dans la forêt, au risque de foutre le feu. Peu importe l’environnement, l’important est de se débarrasser à bon compte de ses déchets. Et si vous avez la faiblesse de prêter ne serait-ce qu’une vague oreille aux divagations de l’un d’entre eux, vous pouvez être certain qu’ils ne seront pas long à faire la queue devant chez vous pour faire leurs besoins sur votre paillasson. Pour ce qui est des soi-disant secrets qu’ils sont censés garder, dites-vous bien que c’est un fardeau dont ils se débarrasseront à la première occasion. Ils vous refilent la patate chaude et reprennent leur petite vie tranquille comme si de rien n’était, la conscience aussi légère et vierge qu’au jour de leurs premiers vagissements. Libre à vous de la refiler à quelqu’un d’autre, qui à son tour la refilera à quelqu’un d’autre, et ainsi de suite jusqu’à ce que tout le monde soit au courant. Au courant de quoi, c’est une autre question. Car il va de soi, au cours du périple mouvementé qui a été le sien, que le récit n’a plus grand-chose à voir avec ce qu’il était à l’origine.

Dans le cas présent, la patate chaude en question était une première main, passée directement du producteur au consommateur, de la terre à l’assiette. Rien à voir avec ces cochonneries bourrées d’additifs tous plus toxiques les uns que les autres. Sam l’avait récoltée à la source, et me l’avait livrée telle qu’elle, sans la nettoyer, lui faire subir le moindre traitement qui aurait pu altérer sa texture et sa saveur. Elle avait de la mâche, du goût, et le mieux, pour les préserver, était de la cuisiner le moins possible. Certains disent qu’on peut sublimer le produit, le rendre encore meilleur qu’il n’est en réalité, mais si une chose est parfaite en soi, ce que les mêmes ne cessent de répéter avec les yeux révulsés et des trémolos dans la voix, alors je ne vois pas l’intérêt d’en rajouter, sinon celui de s’en attribuer indûment les mérites grâce à un tour de passe-passe discutable. Par exemple, le type qui essaie de vous vendre un jambon prétendument incomparable, va d’abord vous beurrer la tartine d’une épaisse couche de superlatifs concernant son produit, sans la moindre pudeur. Il pourrait se contenter de vous en couper une tranche et attendre patiemment votre réaction. Mais non, il prépare le terrain, vous assure que vous allez participez à une expérience unique dont vous sortirez à jamais transformé, vivre un moment de grâce absolue, toucher du doigt les arcanes de la mystique plotinienne et approcher au plus près des plus grands mystères qui agitent l’humanité depuis la nuit des temps. Vous êtes l’élu, celui qui a été choisi pour assister à la Révélation. À tel point, même si vous le pensez profondément, que vous hésiterez à lui dire que son soi-disant merveilleux jambon de pays n’est rien d’autre qu’une merde infâme indigne du plus vil clébard. Pire encore, vous serez dans l’obligation d’en faire l’acquisition alors que vous n’en avez pas la moindre envie. Et comme vous aurez été bien reçu (c’est tout juste si on ne sera pas allé vous chercher au milieu de la rue pour vous traîner dans la boutique), qu’on vous aura déroulé le tapis rouge, traité comme une personnalité de premier plan, vous culpabiliserez d’autant plus de ne pas souscrire à l’offre proposée. Celle-ci, du reste, sera largement prohibitive, pour ne pas dire totalement déconnectée de la réalité, mais c’est un abus dont vous ne pourrez hélas faire mention sans passer aussitôt pour un effroyable radin, un pique-assiette qui profite de la générosité de ses hôtes pour s’empiffrer à bon compte. Ainsi, vous aurez bel et bien assisté à la Révélation de votre propre connerie, la facilité avec laquelle il est possible de vous la fourrer bien profond. Voilà pourquoi, lorsque vous êtes en affaires, et c’est un petit conseil que vous donne comme ça en passant, partez toujours du principe que votre partenaire est une ordure qui ne pense qu’à vous arnaquer, fera tout pour endormir votre vigilance et se barrer avec la caisse à la première occasion. Ne faites confiance à personne, à commencer vous-même, car vous êtes et serez toujours votre pire ennemi, d’autant plus dangereux qu’il est enclin à tout vous pardonner, y compris vos plus monumentales erreurs.

Donc, pour en revenir à notre affaire, je ne pouvais pas répondre à la question de Greg (Hein ? Quoi ? Comment ? On ? Qui ça, on ?) autre chose que : Désolé, mais je ne peux pas te le dire.

\textsc{Greg} : Je ne sais pas qui t’a raconté ça, mais je peux t’assurer qu’il n’y a rien entre Bérénice et moi. Crois-le ou non, mais c’est tout à fait par hasard qu’on s’est retrouvé dans ce club de danse.

\textsc{Moi} : Dans ce cas, je trouve étrange que tu ne m’en aies jamais rien dit.

\textsc{Lui} : L’occasion ne s’est pas encore présentée, voilà tout. C’est tout récent, je te le rappelle.

\textsc{Moi} : Récent ou pas, je pense que si tu avais rencontré Bérénice dans un club de danse, tu n’aurais pas manqué de m’en faire part. À moins, bien sûr, que tu aies une bonne raison de ne pas le faire.

Greg, aussi convaincant qu’un politicien qui affirme n’avoir jamais eu connaissance des agissements de Jeffrey Epstein alors que son nom revient plus de cinq mille fois dans les dossiers afférents, et n’être allé à Little Saint James qu’une ou deux fois, et encore uniquement pour jouer au tennis, prendre des bains de mer et pêcher le marlin : Je suis vraiment très triste.

\textsc{Moi} : Ah bon ?

\textsc{Lui} : Ben oui, que tu puisses penser une chose pareille de moi.

\textsc{Moi} : Pourquoi tu ne dis pas tout simplement la vérité ?

\textsc{Lui} : Quelle vérité ?

\textsc{Moi} : Que tu te tapes la femme de Titus. En plus de Lou et peut-être une demi-douzaine d’autres, va savoir. À partir de là, on peut tout imaginer.

\textsc{Lui} : Comme quoi, par exemple ?

\textsc{Moi} : Eh bien, par exemple, que Bérénice et toi avez imaginé un plan machiavélique pour vous débarrasser du mari gênant. Vous louez les services de cette prétendue femme de ménage et Gardienne de mes couilles, qui est en réalité une redoutable tueuse à gage, et vous faites croire à tout le monde qu’elle et Titus sont partis couler des jours heureux quelque part sous les Tropiques.

Greg, avec pour la première depuis longtemps une vague touche de sincérité dans le regard : C’est du grand n’importe quoi !

Bon, je reconnais qu’il m’arrive parfois d’avoir des idées farfelues, sinon franchement débiles, et celle-ci figurait en bonne place dans le haut du panier. Si je n’avais pas cette profonde connaissance de moi-même, fruit d’une longue et attentive fréquentation de mon humble personne, j’en viendrais presque à douter de l’état de ma santé mentale. Mais s’il faut douter à tout prix, j’aime autant douter de celle des autres. Greg est un ami et le restera, même s’il couchait avec sa propre mère, ce qui ne risque pas d’arriver puisqu’elle est morte, paix à son âme. Mais peut-être l’ont-ils fait, se sont-ils vautrés tels des hippopotames en rut dans la boue de l’inceste, des sangsues boursouflées de désir dans la vase abjecte de la consanguinité. Si tel est le cas, je n’en ai jamais rien su et préfère n’en jamais rien savoir. Cela dit, Greg m’avait souvent donné cette bizarre impression de traverser l’existence sur la pointe des pieds, en catimini. On ne le sentait pas réellement investi dans les affaires du monde. Je n’irais pas jusqu’à dire qu’il se foutait royalement de tout, ce serait sans doute exagéré, mais il y avait tout de même chez lui une forme de dilettantisme qui flirtait parfois assez dangereusement avec l’absentéisme pur et simple. Je pense que ça ne lui aurait fait ni chaud ni froid si on lui avait annoncé qu’il allait mourir demain. Ses traits ne seraient pas décomposés, ni sa gorge nouée. Contrairement à ces condamnés à mort qui s’accrochent désespérément à la vie et pleurent toutes les larmes de leur corps au moment fatidique, abandonnant au pied du gibet les derniers oripeaux de leur dignité, c’est le cœur léger qu’il aurait offert son cou à la corde du bourreau. Un jour, on lui avait fait cadeau de la vie. C’était, comme à nous tous, son premier cadeau d’anniversaire. Sauf que pour lui, le cadeau en question ressemblait davantage au jackpot d’une loterie infernale dont il aurait été non pas le grand gagnant, l’heureux élu, mais le grand perdant, la principale victime. Quand certains passent le restant de leurs jours à les remercier de leur avoir donné la vie, permis de gambader sur terre tels des faons émerveillés, d’autres vouent leurs géniteurs aux gémonies et n’aspirent qu’à leur faire payer le prix de cette trahison. Quand on aime vraiment quelqu’un, on ne lui donne jamais le jour. Certes, on n’aura jamais le plaisir de faire sa connaissance, contempler sur ses traits la marque de sa glorieuse lignée, mais au moins on n’aura pas son sang sur les mains. Tel un Marc-Antoine Jacquinot, avec lequel il partageait de nombreux traits de caractère, Greg n’avait jamais jugé utile de faire valoir ses droits à la paternité. Tous deux étaient, quoi qu’on en pense, en parfaite adéquation avec cette tendance de la jeunesse actuelle qui consiste à ne plus faire d’enfant, au point que d’aucuns commencent à s’en inquiéter sérieusement. C’est chronophage, ça coûte bonbon et ça ne rapporte le plus souvent que des emmerdements. D’autre part, avec les familles sans cesse recomposées, faites de bric et de broc, la notion de descendance perd de sa pertinence. D’autant qu’il ne sera bientôt plus nécessaire de payer de sa personne, la reproduction pouvant être assurée in vitro à partir d’une sélection aléatoire d’échantillons prélevés sur l’ensemble de la population, avec tous les avantages d’un brassage ethnique intempestif. Et sans doute même qu’un jour, ovule et spermatozoïde pourront être fabriqués de toutes pièces en laboratoire, ce qui permettra non seulement de s’affranchir du concept de race, problématique s’il en est, mais aussi d’exercer un contrôle total sur la démographie, tant sur la plan de la quantité que la qualité. Il est loin le temps où on accouchait dans la douleur, et où la question se posait souvent de savoir si on devait sauver la vie de la mère ou celle de l’enfant. Ce choix, il faut bien le reconnaître, était rarement favorable à la parturiente, surtout si l’enfant à naître était un mâle (pour un bien, en quelque sorte). Enfin, le monde fait aujourd’hui figure d’un tel bourbier que beaucoup rechignent à y plonger d’innocentes créatures qui n’ont rien demandé à personne. À quoi bon les envoyer au casse-pipe, les condamner à livrer une guerre perdue d’avance. Et surtout, à quoi bon rajouter du malheur au malheur quand il y a déjà de part le monde des tas d’orphelins qui crèvent la gueule ouverte dans le caniveau, sont livrés pieds et poings liés à une horde de prédateurs qui abusent d’eux en toute impunité. Bien sûr, la plupart d’entre eux sont des boules de haine et de ressentiment difficiles à gérer, mais l’heure est venue de se détacher de ces vieux principes éculés de filiation et de maternité. La reproduction doit être assurée de façon mécanique et désintéressée. Après tout, faire des enfants n’est pas un investissement à long terme, une assurance-vie pour de futurs vieux qui n’ont pas envie de crever seuls comme des chiens dans un pavillon de banlieue décrépi. C’est une nécessité biologique, si on veut que l’espèce perdure, qui doit être appréhendée de façon scientifique et non plus sentimentale et affective, avec la niaiserie résignée de ces jeunes parents qui endurent H24 les réflexions idiotes de leur entourage radotant. Notre existence n’est que le fruit du hasard, et nos enfants ne nous appartiennent pas davantage que nous ne nous appartenons à nous-mêmes. Ton corps t’appartient… Mon cul, oui ! C’est pas comme si tu étais allé l’acheter au supermarché du coin, et tu auras beau l’entretenir aussi bien sinon mieux que ta baraque ou ta bagnole, ça ne changera rien à l’affaire. Si tu peux toujours revendre ou refiler ta baraque et ta bagnole à tes enfants, personne ne voudra de ton corps. Ta carcasse sera dévolue aux vautours et tu te feras ratisser jusqu’à l’os, à l’exception bien sûr de tes dents en or, prothèses de hanche et autres babioles imputrescibles qui seront récupérées par la communauté. Car n’oublie jamais ceci, ma frère, mon sœur ou qui que tu sois : ton corps est la propriété de Dame Nature qui te l’a généreusement (plus ou moins, certains auraient des raisons de se plaindre) prêté pour te servir d’enveloppe physique pendant ton séjour sur terre. Il faut le restituer le jour de ta mort, sensiblement dans l’état où tu l’as trouvé le jour de ta naissance, c’est à dire peu fonctionnel et totalement dépendant des autres pour assurer sa survie. Sauf que cette fois, les autres n’ont pas la moindre envie de te donner la becquée, te porter sur leurs épaules et te torcher le cul. Ils n’ont plus envie de te courir après quand tu t’enfuis à quatre pattes à travers la maison en ricanant comme un dingue et lâchant des caisses à tout bout de champ. Ils sont fatigués de faire tes courses, hurler parce que ton sonotone déconne, te ramasser parce que tu n’arrêtes pas de tomber (ça te rapproche de la tombe), aller te récupérer au commissariat parce qu’on t’a encore chopé en train de te balader à poil dans la rue, taguer des obscénités sur les murs et pisser sur les devantures de magasins. Ils n’ont qu’une hâte, c’est que tu casses ta pipe et arrêtes de faire chier le monde avec tes jérémiades. Même tes gosses, que tu appelles madame ou monsieur parce que tu ne les reconnais plus, commencent à trouver le temps long. Eux et leurs propres enfants auraient mieux à faire que de passer leurs dimanches à l’EHPAD, mais comme ils savent que tu es capable de les déshériter sur un simple coup de tête, ils continuent à venir pour ne pas risquer de passer à côté du pactole. Mais toi tu t’en fous, parce que tu as cinq ans dans ta tête, et que tu sais que ton grand âge te permet de faire à peu près tout et n’importe quoi sans que personne n’ose lever le petit doigt. Alors tu vas voir le caïd du coin, met de la schnouff dans ta chicorée et t’achètes un flingue pour braquer une banque. Quand les flics arrivent, tu défourailles à tout-va, mais personne ne riposte parce que ça ferait désordre de buter un vieux qui n’a plus toute sa tête. On attend que le chargeur soit vide, puis on vient te chercher pour t’escorter gentiment jusqu’à l’asile le plus proche, asile dont tu ne tarderas d’ailleurs pas à t’évader pour mettre à nouveau la ville à feu et à sang, passer en justice pour dégradation de bien public, et faire marrer tout le tribunal en te foutant de la gueule des juges et désavouant publiquement ton avocat pour cause d’incompétence chronique et inculture caractérisée.

Bref, les jeux sont faits à notre insu, et nous ne sommes que des billes qui virevoltent sur la table de la roulette en espérant tomber sur le bon numéro. Beaucoup, aujourd’hui, semblent assez peu motivés pour participer à ce qui leur apparaît de plus en plus comme un jeu de dupe, une vaste supercherie. La reproduction ne fait plus recette, la contraception bat son plein. Chez les femmes principalement, qui sont en première ligne. Si elles éprouvaient jadis une certaine fierté à tripler de volume en neuf mois et se traîner comme des vaches à lait en priant le ciel que tout se passe bien au moment de l’accouchement, elles ne semblent plus aujourd’hui très réceptives à la démarche. Physiquement parlant, elles préfèrent le style Coca-Cola au style Orangina, la taille de guêpe à l’embonpoint gravidique. Dans le monde moderne, biberon, couche, siège-bébé, poussette et dépression ne font plus rêver personne, les gens ayant autre chose à foutre que passer leur temps à pouponner. La grossesse, vécue comme un parcours du combattant digne des Forces Spéciales dans l’enfer djihadiste du sanctuaire de Tofagala, est une chose beaucoup trop sérieuse pour être confiée à des gens dont ce n’est pas le métier. Seules des femmes surentraînées, au mental d’acier, peuvent se permettre d’affronter cette épreuve sans s’exposer aux ravages du stress post-traumatique. Donner la vie, au même titre que la prendre, exige des compétences qui ne sont pas à la portée du premier venu. Face aux difficultés croissantes de l’existence, la progéniture elle-même semble accessoire, un luxe dispensable réservé à une élite qui a les moyens de ne pas s’impliquer dans la vie familiale et surtout faire élever ses enfants par d’autres, des professionnels spécialement formés pour gérer les parcours scolaires, les conflits de cour de récréation, les chagrins d’amour et les crises d’adolescence.

Voilà pourquoi, pour en revenir à Greg, même s’il brûlait d’une quelconque flamme pour Bérénice, ce qui était fort probable à moins que Sam n’ait été abusé par ses sens (le fait est qu’il avait tendance à picoler, en plus de certaines mauvaises habitudes toxicologiques contractées au cours de ses années de baroud au sein de la Légion), j’avais tout lieu de penser que ce modeste incendie ne survivrait pas à la première averse.

Un flic en uniforme s’est pointé pour me faire part d’une nouvelle de la toute première importance, susceptible d’éclairer l’affaire en cours d’un jour radicalement nouveau. Il avait mis la main, en jetant négligemment un œil (réflexe professionnel) dans le coffre de la Mini, sur une importante quantité de ce qu’il est convenu d’appeler un produit stupéfiant. En l’occurrence, il s’agissait de toute évidence de cet ester méthylique de benzoylecgonine plus connu sous le nom de chlorhydrate de cocaïne, ou cocaïne tout court, alcaloïde tropanique très recherché pour ses propriétés psychoactives. À vue de groin, il y en avait une bonne dizaine de kilos, soit une valeur marchande avoisinant les sept cent mille euros. Il va sans dire que la présence de ce chargement pour le moins compromettant expliquait à lui seul la conduite (dangereuse) de la très belle et sulfureuse Repentance Whittingham, la femme de ménage la plus rapide du monde qui ne se contentait apparemment pas de faire succinctement le ménage dans un hôtel de luxe pour personnes de couleur, hôtel de luxe qui d’ailleurs, au vu des récents événements, n’était peut-être pas seulement, pour reprendre mot pour mot l’excellente définition du Larousse, le respectable «établissement commercial mettant à la disposition d’une clientèle itinérante des chambres meublées pour un prix journalier» (et excessivement élevé, ajouterai-je) qu’il prétendait.

Voilà, c’est pour l’instant.

Tout ce que je puis ajouter, à l’heure où j’écris ces lignes, c’est qu’on est toujours sans nouvelles de Titus Beaugendre.

Est-il besoin de préciser que je m’associe pleinement à sa femme et ses enfants, dont je partage l’affliction, pour dénoncer l’indolence, sinon l’inaction des pouvoirs publics, et exiger que toute la lumière soit faite au plus vite sur cette affaire.

Naturellement, vous serez les premiers informés dès que j’en saurai un peu plus à ce sujet.

En attendant, je vous souhaite bon vent (dans tous les sens du terme, y compris bien sûr l’expulsion plus ou moins sonore et trébuchante de ces gaz intestinaux qu’il serait dangereux de conserver indéfiniment à l’intérieur de soi).

\textsc{PS} : Si je n’aime pas les fins, toujours décevantes et redondantes, inutiles, je n’aime pas non plus les titres, ennuyeux, racoleurs et commerciaux, profondément réducteurs et indigents.

Si vous lisez MADAME BOVARY sur une couverture, par exemple, vous viennent aussitôt en tête des adjectifs peu flatteurs comme bovin ou bavarois, lesquels, à moins d’être féru de produits laitiers ou de culture germanique, ne donnent guère envie d’ouvrir le livre.

De la même façon, si vous lisez DRACULA, vous viennent aussitôt en tête des images à caractère sexuel pour le moins déplacées, même si, je vous l’accorde, les canines du vampire qui pénètrent dans la chair tendre d’une gorge féminine ne sont pas totalement exemptes de sensualité.

J’avais, dans un premier temps, pensé appeler ce livre L’ÉCUME DES ABAT-JOUR, en hommage à Boris Vian, auteur que je connais de nom et de réputation, mais dont je confesse, à ma grande honte, n’avoir jamais lu le moindre livre. Je ne pouvais donc pas, ne serait-ce que par éthique littéraire, retenir ce titre, assez rigolo par ailleurs.

J’ai ensuite pensé l’appeler LES GLAPISSEMENTS DE L’ENNUI, en référence aux CROASSEMENTS DE LA NUIT (titre français assez disgracieux de STILL LIFE WITH CROWS) de Douglas Preston \& Lincoln Child, deux auteurs de romans policiers teintés de fantastique que j’apprécie particulièrement, même si leur prose à quatre mains est loin d’être un modèle d’intelligence et de créativité, tant sur la forme que le fond (il est bon, parfois, de s’autoriser une certaine dose de médiocrité). J’ai rapidement laissé tomber l’idée, divertissante, certes, mais sans grand intérêt.

J’ai alors pensé, de façon plus impersonnelle, l’intituler LIVRE UN ou MON PREMIER ROMAN, mais la connotation enfantine m’est apparue par trop manifeste.

Si j’ai finalement, en désespoir de cause, résolu de tisser la métaphore alimentaire, c’est parce que manger (et boire) reste une de mes activités favorites. C’est aussi, accessoirement, la garantie de ne plus avoir à se creuser le chou pour trouver un titre, la gastronomie internationale ne manquant pas de recettes aussi populaires que savoureuses. J’aurais aussi bien pu appeler ce bouquin CASHER BLUES, en référence subtile et odorante au CASHEL BLUE, ce fromage irlandais à pâte persillée, ou JAMBALAYA, best-seller de la cuisine cajun, TOUOP-TOUOP KELONG, recette camerounaise à base de banane plantain et poisson fumé, ou encore MOROS Y CRISTIANOS, du nom de ce délicieux plat cubain à base de riz blanc et haricots noirs, sans que personne n’y trouve rien à redire, ni trahir aucunement la nature conviviale et épicée du présent ouvrage.


