%
% lettre-de-presentation-denoël.tex
% Compilation : latexmk -pdf lettre-de-presentation-denoël.tex
%
\documentclass[12pt,a4paper]{article}

\usepackage[utf8]{inputenc}
\usepackage[T1]{fontenc}
\usepackage{ebgaramond}
\usepackage[french]{babel}
\usepackage{microtype}
\usepackage{graphicx}
\usepackage{booktabs}
\usepackage{array}
\usepackage[left=30mm,right=25mm,top=28mm,bottom=25mm]{geometry}

\setlength{\parindent}{0pt}
\setlength{\parskip}{7pt}

\begin{document}
\pagestyle{empty}

\bigskip

Madame, Monsieur,

Je viens vous proposer le manuscrit du premier roman de mon ami cher Éric Mugnier,
\textit{Tête de veau ravigote}, dont voici le pitch :

\medskip
\begin{quote}
Quand le père Vidal disparaît dans des circonstances troubles\\
et que des colis macabres commencent à circuler dans la ville,\\
le commandant Beauvais se lance dans une enquête labyrinthique\\
qui le mènera des caves de l'Église à l'ambassade de Namibie.

\medskip
Entouré de Titus Beaugendre, son fidèle équipier,\\
et d'une galerie de personnages hauts en couleur,\\
Beauvais arpente un monde contemporain\\
où l'absurde le dispute à l'horreur.
\end{quote}
\medskip

Écrit dans le style d'un polar, ce n'en est pas un : où sont les énigmes à résoudre,
les fausses pistes, les coups de théâtre ? Plutôt qu'un polar, un ovni. Une logorrhée
vengeresse et jubilatoire, écrite en \textit{free writing} / \textit{spontaneous prose},
où l'humour potache s'achoppe à des scènes \textit{gore} à la limite du soutenable,
entrelacée d'amples digressions érudites : sur les chats, les dinosaures, l'Église,
les pavillons de banlieue, l'unité 731, Pannonica de Koenigswarter, la 'Ndrangheta.

L'\textbf{Horreur} (prononcé à la Marlon Brando dans Apocalypse Now), voilà le thème du livre.

Toute forme de morale est étrangère aux protagonistes — au héros-narrateur en premier.
Les digressions, d'un pessimisme schopenhauérien, révèlent une humanité sans Dieu,
désespérée, tragique. Le ton de celles-ci n'est jamais didactique, jamais emphatique,
toujours factuel, genre \og Alain Decaux raconte \fg{}. Narrer L'Horreur sur ce ton
d'entomologiste provoque l'apparition spontanée et involontaire de la sensation d'ironie
dans l'esprit du lecteur, un mécanisme inné de défense de son intégrité mentale. Bien
plus efficace qu'un bouquin qui dirait benoîtement : \og Regardez ça, c'est mal !
Ouh comme c'est vilain ! \fg{}. Un bruit de fond, un acouphène, un filtre optique
polarisé, un voile de gaze sur la peau. La tentaculaire progression de la Camorra
dans le Naples de l'\textit{Amica Geniale} d'Elena Ferrante, jamais explicite, trop prégnante.

\textit{Tête de veau ravigote} me semble trouver naturellement sa place chez Denoël,
maison qui a su, depuis Céline, accueillir les textes qui bousculent — ceux qui mêlent
la noirceur absolue à une ambition littéraire sans compromis. La collection qui a publié
Palahniuk en France n'est sans doute pas étrangère à cet esprit. En outre, il n'échappera
pas à l'éditeur qu'un texte de cette nature recèle un potentiel de controverse, un côté
\og comment peut-on écrire de pareilles choses ?! \fg{} qui, de Huysmans à Houellebecq,
a toujours su nourrir le débat public et, accessoirement, faire vendre.

Je suis la tête de gondole d'un groupe d'amis de Éric Mugnier, convaincus du potentiel
de ses écrits. Nous sommes convenus de l'aider en toute amitié et absolu désintéressement
dans les démarches fastidieuses inhérentes à la recherche d'un éditeur. Éditeur qui saura,
au travers des défauts du texte, déceler ce potentiel, partager notre conviction et
entreprendre avec Éric Mugnier le travail éditorial qui lui fera trouver son public.

\medskip
Respectueusement,\\[4pt]
Christophe Thiebaud

\bigskip

% ─── Biographie ────────────────────────────────────────────────────────────────
\begin{tabular}{@{}p{3cm}p{11cm}@{}}
\toprule
\textsc{Biographie} & \\
\midrule
& Éric Mugnier a 65 ans ; \\
& habite la préfecture de l'un des départements les plus boisés de France, la Haute-Marne ; \\
& a été adopté tout petit via l'Assistance Publique par une mère seule de la moyenne bourgeoisie catholique locale ; \\
& n'a jamais eu d'activité professionnelle \og classique \fg{} : emploi, métier, profession, poste, travail, fonction, rien de tout cela ; \\
& a pu consacrer sa vie à la création, dans de nombreux arts : peinture, musique, chansons, et ici, écriture ; \\
& son auteur favori est Joris-Karl Huysmans. \\
\bottomrule
\end{tabular}

\bigskip

% ─── Coordonnées ───────────────────────────────────────────────────────────────
\begin{tabular}{@{}p{3cm}p{11cm}@{}}
\toprule
\textsc{Coordonnées} & \\
\midrule
& Éric Mugnier \\
& 33-11 rue Bouchardon \\
& 52000 Chaumont \\
& ericmugnier730@gmail.com \\
& 06 82 03 69 48 \\
\bottomrule
\end{tabular}

\bigskip\bigskip

% ─── Photos ────────────────────────────────────────────────────────────────────
\begin{minipage}[t]{0.45\linewidth}
  \centering
  \includegraphics[width=\linewidth]{../images/Mugnier0.jpg}\\[4pt]
  \small 1984
\end{minipage}%
\hfill
\begin{minipage}[t]{0.45\linewidth}
  \centering
  \includegraphics[width=\linewidth]{../images/Mugnier4.jpg}\\[4pt]
  \small 2020
\end{minipage}

\end{document}
