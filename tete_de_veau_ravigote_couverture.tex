\documentclass[tikz, border=0pt]{standalone}

% ─── PACKAGES ─────────────────────────────────────────────────────────────────
\usepackage[utf8]{inputenc}
\usepackage[T1]{fontenc}
\usepackage{ebgaramond}
\usepackage[french]{babel}
\usepackage{xcolor}
\usepackage{tikz}
\usetikzlibrary{calc}
\usepackage{microtype}
\usepackage{ragged2e}
\usepackage{graphicx}

% ─── COULEURS ─────────────────────────────────────────────────────────────────
\definecolor{gallimardred}{RGB}{185,10,10}
\definecolor{creme}{RGB}{255,252,240}
\definecolor{dosgris}{RGB}{245,242,232}

% ─── PAS DE NUMÉRO DE PAGE (inutile avec standalone) ─────────────────────────


% ─── RÉSUMÉ & BIO (modifier ici) ──────────────────────────────────────────────
\newcommand{\resumequatriemecouv}{%
  Quand le père Vidal disparaît dans des circonstances troubles et que des
  colis macabres commencent à circuler dans la ville, le commandant Beauvais
  se lance dans une enquête labyrinthique. Mais le polar n'est ici qu'un fil
  rouge dérisoire : ce qui compte, ce sont les digressions – sur les chats,
  les dinosaures, l'Église, les pavillons de banlieue, la décadence de
  l'humanité.

  \medskip
  Roman-fleuve picaresque dans la lignée de Cervantes et Rabelais,
  \textit{Tête de veau ravigote} mêle la rage sadienne au pessimisme
  schopenhauerien. Éric Mugnier y déploie une prose incandescente, truculente
  et désespérée, qui rappelle autant \textit{Là-bas} de Huysmans que les
  pamphlets du XVIIIe~siècle. Inachevable, et délibérément inachevé, ce
  premier roman ignore la résolution et préfère la dérive : une logorrhée
  vengeresse et jubilatoire.%
}

\newcommand{\bioauteur}{%
  Éric Mugnier est né en 1960. Il vit à Chaumont-en-Bassigny. Peintre,
  compositeur et parolier, il se consacre depuis quarante ans à la
  création. \textit{Tête de veau ravigote} est son premier roman.%
}

% ─── DOCUMENT ─────────────────────────────────────────────────────────────────
\begin{document}

% ─── CANVAS TIKZ (unité = 1mm, origine bas-gauche) ───────────────────────────
% Zones :
%   Fond perdu gauche   :  0 –   3 mm
%   4e de couverture    :  3 – 143 mm  (140 mm)
%   Dos                 : 143 – 161 mm ( 18 mm)
%   1ère de couverture  : 161 – 301 mm (140 mm)
%   Fond perdu droit    : 301 – 304 mm
%   Fond perdu bas/haut :  0 –   3 mm / 208 – 211 mm

\begin{tikzpicture}[x=1mm, y=1mm, font=\normalfont]

  % ── FOND GLOBAL ────────────────────────────────────────────────────────────
  \fill[creme] (0,0) rectangle (304,211);

  % ── DOS (fond légèrement plus sombre) ──────────────────────────────────────
  \fill[dosgris] (143,0) rectangle (161,211);
  % Filets de séparation dos
  \draw[black!30, line width=0.2pt] (143,0) -- (143,211);
  \draw[black!30, line width=0.2pt] (161,0) -- (161,211);

  % ─────────────────────────────────────────────────────────────────────────
  % ══ 1ÈRE DE COUVERTURE (161 – 301 mm) ═══════════════════════════════════
  % ─────────────────────────────────────────────────────────────────────────

  % Double rectangle rouge (style NRF/Gallimard)
  \draw[gallimardred, line width=0.5pt]
    (166,  9) rectangle (298, 202);
  \draw[gallimardred, line width=0.3pt]
    (167, 10) rectangle (297, 201);

  % ── Auteur ─────────────────────────────────────────────────────────────
  \node[anchor=north] at (231, 191)
    {\fontsize{10.5}{12}\selectfont\textsc{éric mugnier}};

  % Filet fin rouge sous l'auteur
  \draw[gallimardred, line width=0.4pt] (176,182) -- (286,182);

  % ── Titre (petit caps, grand, centré) ──────────────────────────────────
  \node[anchor=north, text width=115mm, align=center] at (231, 177)
    {\fontsize{16}{20}\selectfont\textsc{tête de veau ravigote}};

  % Filet fin rouge sous le titre
  \draw[gallimardred, line width=0.4pt] (176,138) -- (286,138);

  % ── Mention "roman" ────────────────────────────────────────────────────
  \node[anchor=north] at (231, 134)
    {\fontsize{10}{12}\selectfont\textit{roman}};

  % ── Ornement typographique centré ──────────────────────────────────────
  \draw[gallimardred!60, line width=0.3pt]
    (228, 75) -- (234, 75);
  \draw[gallimardred!60, line width=0.3pt]
    (231, 72) -- (231, 78);
  \fill[gallimardred!60] (231,75) circle (0.8mm);

  % ── Logo CT (1ère, centré dans l'espace libre) ────────────────────────────
  \node[anchor=center] at (231, 104)
    {\includegraphics[width=14mm]{ct_watermark.png}};


  % ─────────────────────────────────────────────────────────────────────────
  % ══ DOS (143 – 161 mm) ═══════════════════════════════════════════════════
  % ─────────────────────────────────────────────────────────────────────────

  % Auteur en haut (horizontal, police réduite)
  \node[anchor=north, rotate=0] at (152, 199)
    {\fontsize{6.5}{8}\selectfont\textsc{é. mugnier}};

  % Titre vertical (texte de bas en haut)
  \node[rotate=90, anchor=center] at (152, 112)
    {\fontsize{8}{10}\selectfont\textsc{tête de veau ravigote}};

  % ── Logo CT (dos, bas, discret) ───────────────────────────────────────────
  \node[anchor=south] at (152, 6)
    {\includegraphics[width=6mm]{ct_watermark.png}};


  % ─────────────────────────────────────────────────────────────────────────
  % ══ 4ÈME DE COUVERTURE (3 – 143 mm) ═════════════════════════════════════
  % ─────────────────────────────────────────────────────────────────────────

  % ── Résumé
  \node[
    anchor=north west,
    text width=113mm,
    align=justify,
    inner sep=0pt
  ] at (15, 191)
  {%
    \fontsize{9}{13.5}\selectfont
    \resumequatriemecouv
  };

  % Filet séparateur au-dessus de la bio
  \draw[black!40, line width=0.3pt] (15, 41) -- (131, 41);

  % ── Bio auteur ─────────────────────────────────────────────────────────
  \node[
    anchor=north west,
    text width=113mm,
    align=justify,
    inner sep=0pt
  ] at (15, 38)
  {%
    \fontsize{8}{11}\selectfont
    \bioauteur
  };

\end{tikzpicture}

\end{document}
