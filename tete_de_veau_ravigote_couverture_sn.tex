\documentclass[tikz, border=0pt]{standalone}

% ─── PACKAGES ─────────────────────────────────────────────────────────────────
\usepackage[utf8]{inputenc}
\usepackage[T1]{fontenc}
\usepackage{ebgaramond}
\usepackage[french]{babel}
\usepackage{xcolor}
\usepackage{tikz}
\usetikzlibrary{calc}
\usepackage{microtype}
\usepackage{graphicx}

% ─── COULEURS SÉRIE NOIRE ─────────────────────────────────────────────────────
\definecolor{snjaune}{RGB}{255,210,0}     % jaune Série Noire
\definecolor{snnoir}{RGB}{10,10,10}       % noir profond
\definecolor{snblanc}{RGB}{240,240,240}   % blanc cassé texte

% ─── DOCUMENT ─────────────────────────────────────────────────────────────────
\begin{document}

% ─── CANVAS TIKZ (unité = 1mm, origine bas-gauche) ───────────────────────────
% Zones :
%   Fond perdu gauche   :  0 –   3 mm
%   4e de couverture    :  3 – 143 mm  (140 mm)
%   Dos                 : 143 – 161 mm ( 18 mm)
%   1ère de couverture  : 161 – 301 mm (140 mm)
%   Fond perdu droit    : 301 – 304 mm
%   Fond perdu bas/haut :  0 –   3 mm / 208 – 211 mm

\begin{tikzpicture}[x=1mm, y=1mm, font=\normalfont]

  % ── FOND GLOBAL NOIR ───────────────────────────────────────────────────────
  \fill[snnoir] (0,0) rectangle (304,211);

  % ─────────────────────────────────────────────────────────────────────────
  % ══ 1ÈRE DE COUVERTURE (161 – 301 mm) ═══════════════════════════════════
  % ─────────────────────────────────────────────────────────────────────────

  % ── Filet jaune de séparation ──────────────────────────────────────────
  \draw[snjaune, line width=0.8pt] (166, 200) -- (297, 200);

  % ── Auteur ─────────────────────────────────────────────────────────────
  \node[anchor=north, text=snblanc] at (231, 197)
    {\fontsize{11}{13}\selectfont\sffamily ÉRIC MUGNIER};

  % ── Titre (jaune, très grand, impactant) ───────────────────────────────
  \node[
    anchor=north,
    text width=125mm,
    align=center,
    text=snjaune
  ] at (231, 163)
    {\fontsize{22}{24}\selectfont\sffamily\bfseries
     TÊTE DE VEAU\\[1mm]RAVIGOTE};

  % ── Filet jaune sous le titre ──────────────────────────────────────────
  \draw[snjaune, line width=0.8pt] (166, 121) -- (297, 121);

  % ── Mention roman ──────────────────────────────────────────────────────
  \node[anchor=north, text=snblanc] at (231, 118)
    {\fontsize{9}{11}\selectfont\sffamily\itshape roman};

  % ── Logo CT (entre "roman" et l'ornement) ────────────────────────────────
  \node[anchor=center] at (231, 88)
    {\includegraphics[width=14mm]{ct_watermark.png}};

  % ── Ornement central : trois tirets horizontaux jaunes ─────────────────
  \foreach \y in {70, 65, 60} {
    \draw[snjaune!70, line width=0.6pt] (215, \y) -- (247, \y);
  }


  % ─────────────────────────────────────────────────────────────────────────
  % ══ DOS (143 – 161 mm) ═══════════════════════════════════════════════════
  % ─────────────────────────────────────────────────────────────────────────

  % Bande jaune pleine sur le dos
  \fill[snjaune] (143,0) rectangle (161,211);

  % Auteur en haut du dos (texte noir sur jaune)
  \node[anchor=north, text=snnoir] at (152, 199)
    {\fontsize{6}{7.5}\selectfont\sffamily\bfseries MUGNIER};

  % Titre vertical (noir sur jaune)
  \node[rotate=90, anchor=center, text=snnoir] at (152, 106)
    {\fontsize{8.5}{10}\selectfont\sffamily\bfseries TÊTE DE VEAU RAVIGOTE};

  % ── Logo CT (dos, bas, discret) ───────────────────────────────────────────
  \node[anchor=south] at (152, 8)
    {\includegraphics[width=6mm]{ct_watermark.png}};


  % ─────────────────────────────────────────────────────────────────────────
  % ══ 4ÈME DE COUVERTURE (3 – 143 mm) ═════════════════════════════════════
  % ─────────────────────────────────────────────────────────────────────────

  % ── Résumé
  \node[
    anchor=north west,
    text width=113mm,
    align=justify,
    text=snblanc,
    inner sep=0pt
  ] at (15, 188)
  {%
    \fontsize{9}{13.5}\selectfont
    Quand le père Vidal disparaît dans des circonstances troubles et que des
    colis macabres commencent à circuler dans la ville, le commandant Beauvais
    se lance dans une enquête labyrinthique. Mais le polar n'est ici qu'un fil
    rouge dérisoire : ce qui compte, ce sont les digressions – sur les chats,
    les dinosaures, l'Église, les pavillons de banlieue, la décadence de
    l'humanité.

    \medskip
    Roman-fleuve picaresque dans la lignée de Cervantes et Rabelais,
    \textit{Tête de veau ravigote} mêle la rage sadienne au pessimisme
    schopenhauerien. Éric Mugnier y déploie une prose incandescente, truculente
    et désespérée, qui rappelle autant \textit{Là-bas} de Huysmans que les
    pamphlets du XVIIIe~siècle. Inachevable, et délibérément inachevé, ce
    premier roman ignore la résolution et préfère la dérive : une logorrhée
    vengeresse et jubilatoire.%
  };

  % Filet séparateur jaune
  \draw[snjaune!50, line width=0.4pt] (15, 41) -- (131, 41);

  % ── Bio
  \node[
    anchor=north west,
    text width=113mm,
    align=justify,
    text=snblanc!70,
    inner sep=0pt
  ] at (15, 38)
  {%
    \fontsize{8}{11}\selectfont
    Éric Mugnier est né en 1960. Il vit à Chaumont-en-Bassigny. Peintre,
    compositeur et parolier, il se consacre depuis quarante ans à la
    création. \textit{Tête de veau ravigote} est son premier roman.%
  };

\end{tikzpicture}

\end{document}
