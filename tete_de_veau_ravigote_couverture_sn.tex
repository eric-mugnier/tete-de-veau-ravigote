\documentclass[tikz, border=0pt]{standalone}

% ─── PACKAGES ─────────────────────────────────────────────────────────────────
\usepackage[utf8]{inputenc}
\usepackage[T1]{fontenc}
\usepackage{ebgaramond}
\usepackage[french]{babel}
\usepackage{xcolor}
\usepackage{tikz}
\usetikzlibrary{calc}
\usepackage{microtype}

% ─── COULEURS SÉRIE NOIRE ─────────────────────────────────────────────────────
\definecolor{snjaune}{RGB}{255,210,0}     % jaune Série Noire
\definecolor{snnoir}{RGB}{10,10,10}       % noir profond
\definecolor{snblanc}{RGB}{240,240,240}   % blanc cassé texte

% ─── DOCUMENT ─────────────────────────────────────────────────────────────────
\begin{document}

% ─── CANVAS TIKZ (unité = 1mm, origine bas-gauche) ───────────────────────────
% Zones :
%   Fond perdu gauche   :  0 –   3 mm
%   4e de couverture    :  3 – 143 mm  (140 mm)
%   Dos                 : 143 – 161 mm ( 18 mm)
%   1ère de couverture  : 161 – 301 mm (140 mm)
%   Fond perdu droit    : 301 – 304 mm
%   Fond perdu bas/haut :  0 –   3 mm / 208 – 211 mm

\begin{tikzpicture}[x=1mm, y=1mm, font=\normalfont]

  % ── FOND GLOBAL NOIR ───────────────────────────────────────────────────────
  \fill[snnoir] (0,0) rectangle (304,211);

  % ─────────────────────────────────────────────────────────────────────────
  % ══ 1ÈRE DE COUVERTURE (161 – 301 mm) ═══════════════════════════════════
  % ─────────────────────────────────────────────────────────────────────────

  % ── Filet jaune de séparation ──────────────────────────────────────────
  \draw[snjaune, line width=0.8pt] (166, 200) -- (297, 200);

  % ── Auteur ─────────────────────────────────────────────────────────────
  \node[anchor=north, text=snblanc] at (231, 197)
    {\fontsize{11}{13}\selectfont\sffamily ÉRIC MUGNIER};

  % ── Titre (jaune, très grand, impactant) ───────────────────────────────
  \node[
    anchor=north,
    text width=125mm,
    align=center,
    text=snjaune
  ] at (231, 163)
    {\fontsize{22}{24}\selectfont\sffamily\bfseries
     TÊTE DE VEAU\\[1mm]RAVIGOTE};

  % ── Filet jaune sous le titre ──────────────────────────────────────────
  \draw[snjaune, line width=0.8pt] (166, 121) -- (297, 121);

  % ── Mention roman ──────────────────────────────────────────────────────
  \node[anchor=north, text=snblanc] at (231, 118)
    {\fontsize{9}{11}\selectfont\sffamily\itshape roman};

  % ── Ornement central : trois tirets horizontaux jaunes ─────────────────
  \foreach \y in {70, 65, 60} {
    \draw[snjaune!70, line width=0.6pt] (215, \y) -- (247, \y);
  }


  % ─────────────────────────────────────────────────────────────────────────
  % ══ DOS (143 – 161 mm) ═══════════════════════════════════════════════════
  % ─────────────────────────────────────────────────────────────────────────

  % Bande jaune pleine sur le dos
  \fill[snjaune] (143,0) rectangle (161,211);

  % Auteur en haut du dos (texte noir sur jaune)
  \node[anchor=north, text=snnoir] at (152, 199)
    {\fontsize{6}{7.5}\selectfont\sffamily\bfseries MUGNIER};

  % Titre vertical (noir sur jaune)
  \node[rotate=90, anchor=center, text=snnoir] at (152, 106)
    {\fontsize{8.5}{10}\selectfont\sffamily\bfseries TÊTE DE VEAU RAVIGOTE};


  % ─────────────────────────────────────────────────────────────────────────
  % ══ 4ÈME DE COUVERTURE (3 – 143 mm) ═════════════════════════════════════
  % ─────────────────────────────────────────────────────────────────────────

  % ── Résumé
  \node[
    anchor=north west,
    text width=113mm,
    align=justify,
    text=snblanc,
    inner sep=0pt
  ] at (15, 188)
  {%
    \fontsize{9}{13.5}\selectfont
    Quand le père Vidal disparaît dans des circonstances troubles et que le
    Brain Catcher sème la terreur, le commandant Beauvais se lance dans une
    enquête labyrinthique qui le mènera des caves de l'Église aux secrets les
    plus enfouis de la République. Entouré de Titus Beaugendre, son fidèle
    équipier, et d'une galerie de personnages hauts en couleur, Beauvais arpente
    une France contemporaine où l'absurde dispute à l'horreur.

    \medskip
    Roman-fleuve d'une ambition rare, \textit{Tête de veau ravigote} mêle polar
    halluciné, satire sociale et méditation sur la violence. Éric Mugnier y
    déploie une prose incandescente, truculente et érudite, qui rappelle les
    grandes heures du roman noir français.%
  };

  % Filet séparateur jaune
  \draw[snjaune!50, line width=0.4pt] (15, 41) -- (131, 41);

  % ── Bio
  \node[
    anchor=north west,
    text width=113mm,
    align=justify,
    text=snblanc!70,
    inner sep=0pt
  ] at (15, 38)
  {%
    \fontsize{8}{11}\selectfont
    Éric Mugnier vit en France.
    \textit{Tête de veau ravigote} est son premier roman.%
  };

\end{tikzpicture}

\end{document}
