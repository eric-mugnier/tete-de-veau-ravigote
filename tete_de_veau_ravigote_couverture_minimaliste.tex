\documentclass[tikz, border=0pt]{standalone}

% ─── PACKAGES ─────────────────────────────────────────────────────────────────
\usepackage[utf8]{inputenc}
\usepackage[T1]{fontenc}
\usepackage{ebgaramond}
\usepackage[french]{babel}
\usepackage{xcolor}
\usepackage{tikz}
\usetikzlibrary{calc}
\usepackage{microtype}
\usepackage{graphicx}

% ─── COULEUR ──────────────────────────────────────────────────────────────────
\definecolor{encre}{RGB}{18,18,18}
\definecolor{gristexte}{RGB}{90,90,90}
\definecolor{bleutnuit}{RGB}{28,40,64}

% ─── DOCUMENT ─────────────────────────────────────────────────────────────────
\begin{document}

% ─── CANVAS TIKZ (unité = 1mm, origine bas-gauche) ───────────────────────────
%
% Gabarit TheBookEdition — format A5 (148×210 mm) — 465 pages
%
%   Fond perdu          : 5 mm de chaque côté
%   Marge de sécurité   : 5 mm en retrait des traits de coupe (texte/image)
%
%   Zones (x) :
%     Fond perdu gauche   :   0 –   5 mm
%     4e de couverture    :   5 – 153 mm  (148 mm)
%     Dos                 : 153 – 182 mm  ( 29 mm)
%     1ère de couverture  : 182 – 330 mm  (148 mm)
%     Fond perdu droit    : 330 – 335 mm
%
%   Zones (y) :
%     Fond perdu bas      :   0 –   5 mm
%     Contenu             :   5 – 215 mm  (210 mm)
%     Fond perdu haut     : 215 – 220 mm
%
%   Centres :
%     4e de couverture    :  79 mm
%     Dos                 : 167,5 mm
%     1ère de couverture  : 256 mm

\begin{tikzpicture}[x=1mm, y=1mm, font=\normalfont]

  % ── FOND BLANC ABSOLU ───────────────────────────────────────────────────────
  \fill[white] (0,0) rectangle (335,220);

  % ═══════════════════════════════════════════════════════════════════════════
  % ══ 1ÈRE DE COUVERTURE (182 – 330 mm) — centre : 256 mm ═════════════════
  % ═══════════════════════════════════════════════════════════════════════════

  % ── Auteur : petites caps, très espacées, centré entre haut et filet ────────
  \node[anchor=center, text=bleutnuit] at (256, 185)
    {\fontsize{14}{17}\selectfont\textls[250]{\textsc{éric mugnier}}};

  % ── Filet unique : l'unique élément graphique ─────────────────────────────
  \draw[bleutnuit, line width=0.25pt] (193, 160) -- (319, 160);

  % ── Titre : garamond regular, grand, aéré, une seule ligne ────────────────
  \node[
    anchor=north,
    text width=120mm,
    align=center,
    text=bleutnuit
  ] at (256, 156)
  {\fontsize{26}{32}\selectfont
    Tête de veau ravigote};

  % ── Mention "roman" ────────────────────────────────────────────────────────
  \node[anchor=north, text=bleutnuit] at (256, 123)
    {\fontsize{11}{13}\selectfont\itshape roman};

  % ── Logo CT (1ère, bas droite, discret) ──────────────────────────────────
  \node[anchor=south, opacity=0.4] at (256, 10)
    {\includegraphics[width=6mm]{ct_watermark.png}};

  % ═══════════════════════════════════════════════════════════════════════════
  % ══ DOS (153 – 182 mm) — centre : 167,5 mm ══════════════════════════════
  % ═══════════════════════════════════════════════════════════════════════════

  % Filets de séparation, très effacés
  \draw[encre!12, line width=0.2pt] (153,0) -- (153,220);
  \draw[encre!12, line width=0.2pt] (182,0) -- (182,220);

  % Auteur horizontal en haut du dos
  \node[anchor=north, text=encre] at (167.5, 207)
    {\fontsize{6}{7.5}\selectfont\textls[120]{\textsc{éric mugnier}}};

  % Titre vertical (de bas en haut)
  \node[rotate=90, anchor=center, text=encre] at (167.5, 110)
    {\fontsize{8.5}{10}\selectfont\textls[80]{\textsc{tête de veau ravigote}}};

  % ── Logo CT (dos, bas, discret) ───────────────────────────────────────────
  \node[anchor=south, opacity=0.4] at (167.5, 10)
    {\includegraphics[width=6mm]{ct_watermark.png}};

  % ═══════════════════════════════════════════════════════════════════════════
  % ══ 4ÈME DE COUVERTURE (5 – 153 mm) — centre : 79 mm ═══════════════════
  % ═══════════════════════════════════════════════════════════════════════════

  % ── Résumé ─────────────────────────────────────────────────────────────────
  \node[
    anchor=north west,
    text width=118mm,
    align=justify,
    inner sep=0pt,
    text=encre
  ] at (17, 200)
  {%
    \fontsize{9}{16}\selectfont
    Quand le père Vidal disparaît dans des circonstances troubles et que des
    colis macabres commencent à circuler dans la ville, le commandant Beauvais
    se lance dans une enquête labyrinthique. Mais le polar n'est ici qu'un fil
    rouge dérisoire : ce qui compte, ce sont les digressions – sur les chats,
    les dinosaures, l'Église, les pavillons de banlieue, la décadence de
    l'humanité.\\[16pt]
    Roman-fleuve picaresque dans la lignée de Cervantes et Rabelais,
    \textit{Tête de veau ravigote} mêle la rage sadienne au pessimisme
    schopenhauerien. Éric Mugnier y déploie une prose incandescente, truculente
    et désespérée, qui rappelle autant \textit{Là-bas} de Huysmans que les
    pamphlets du XVIIIe~siècle. Inachevable, et délibérément inachevé, ce
    premier roman ignore la résolution et préfère la dérive : une logorrhée
    vengeresse et jubilatoire.%
  };

  % Filet séparateur fin
  \draw[encre!30, line width=0.25pt] (17, 36) -- (140, 36);

  % ── Bio ─────────────────────────────────────────────────────────────────────
  \node[
    anchor=north west,
    text width=118mm,
    align=justify,
    inner sep=0pt,
    text=gristexte
  ] at (17, 33)
  {%
    \fontsize{8}{11.5}\selectfont
    Éric Mugnier est né en 1960. Il vit à Chaumont-en-Bassigny. Peintre,
    compositeur et parolier, il se consacre depuis quarante ans à la
    création. \textit{Tête de veau ravigote} est son premier roman.%
  };

\end{tikzpicture}

\end{document}
