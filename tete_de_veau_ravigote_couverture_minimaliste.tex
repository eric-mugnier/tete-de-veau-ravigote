\documentclass[tikz, border=0pt]{standalone}

% ─── PACKAGES ─────────────────────────────────────────────────────────────────
\usepackage[utf8]{inputenc}
\usepackage[T1]{fontenc}
\usepackage{ebgaramond}
\usepackage[french]{babel}
\usepackage{xcolor}
\usepackage{tikz}
\usetikzlibrary{calc}
\usepackage{microtype}

% ─── COULEUR ──────────────────────────────────────────────────────────────────
\definecolor{encre}{RGB}{18,18,18}
\definecolor{gristexte}{RGB}{90,90,90}

% ─── DOCUMENT ─────────────────────────────────────────────────────────────────
\begin{document}

% ─── CANVAS TIKZ (unité = 1mm, origine bas-gauche) ───────────────────────────
%
% Gabarit TheBookEdition — format A5 (148×210 mm) — 465 pages
%
%   Fond perdu          : 5 mm de chaque côté
%   Marge de sécurité   : 5 mm en retrait des traits de coupe (texte/image)
%
%   Zones (x) :
%     Fond perdu gauche   :   0 –   5 mm
%     4e de couverture    :   5 – 153 mm  (148 mm)
%     Dos                 : 153 – 182 mm  ( 29 mm)
%     1ère de couverture  : 182 – 330 mm  (148 mm)
%     Fond perdu droit    : 330 – 335 mm
%
%   Zones (y) :
%     Fond perdu bas      :   0 –   5 mm
%     Contenu             :   5 – 215 mm  (210 mm)
%     Fond perdu haut     : 215 – 220 mm
%
%   Centres :
%     4e de couverture    :  79 mm
%     Dos                 : 167,5 mm
%     1ère de couverture  : 256 mm

\begin{tikzpicture}[x=1mm, y=1mm, font=\normalfont]

  % ── FOND BLANC ABSOLU ───────────────────────────────────────────────────────
  \fill[white] (0,0) rectangle (335,220);

  % ═══════════════════════════════════════════════════════════════════════════
  % ══ 1ÈRE DE COUVERTURE (182 – 330 mm) — centre : 256 mm ═════════════════
  % ═══════════════════════════════════════════════════════════════════════════

  % ── Auteur : petites caps, très espacées, discret ─────────────────────────
  \node[anchor=north, text=encre] at (256, 205)
    {\fontsize{12}{14}\selectfont\textls[250]{\textsc{éric mugnier}}};

  % ── Filet unique : l'unique élément graphique ─────────────────────────────
  \draw[encre, line width=0.25pt] (193, 160) -- (319, 160);

  % ── Titre : garamond regular, grand, aéré, une seule ligne ────────────────
  \node[
    anchor=north,
    text width=120mm,
    align=center,
    text=encre
  ] at (256, 156)
  {\fontsize{24}{30}\selectfont
    Tête de veau ravigote};

  % ── Mention "roman" ────────────────────────────────────────────────────────
  \node[anchor=north, text=gristexte] at (256, 123)
    {\fontsize{11}{13}\selectfont\itshape roman};

  % ═══════════════════════════════════════════════════════════════════════════
  % ══ DOS (153 – 182 mm) — centre : 167,5 mm ══════════════════════════════
  % ═══════════════════════════════════════════════════════════════════════════

  % Filets de séparation, très effacés
  \draw[encre!12, line width=0.2pt] (153,0) -- (153,220);
  \draw[encre!12, line width=0.2pt] (182,0) -- (182,220);

  % Auteur horizontal en haut du dos
  \node[anchor=north, text=encre] at (167.5, 207)
    {\fontsize{6}{7.5}\selectfont\textls[120]{\textsc{éric mugnier}}};

  % Titre vertical (de bas en haut)
  \node[rotate=90, anchor=center, text=encre] at (167.5, 110)
    {\fontsize{7.5}{9}\selectfont\textls[80]{\textsc{tête de veau ravigote}}};

  % ═══════════════════════════════════════════════════════════════════════════
  % ══ 4ÈME DE COUVERTURE (5 – 153 mm) — centre : 79 mm ═══════════════════
  % ═══════════════════════════════════════════════════════════════════════════

  % ── Résumé ─────────────────────────────────────────────────────────────────
  \node[
    anchor=north west,
    text width=118mm,
    align=justify,
    inner sep=0pt,
    text=encre
  ] at (17, 175)
  {%
    \fontsize{9}{14}\selectfont
    Quand le père Vidal disparaît dans des circonstances troubles et que le
    Brain Catcher sème la terreur, le commandant Beauvais se lance dans une
    enquête labyrinthique qui le mènera des caves de l'Église aux secrets les
    plus enfouis de la République. Entouré de Titus Beaugendre, son fidèle
    équipier, et d'une galerie de personnages hauts en couleur, Beauvais arpente
    une France contemporaine où l'absurde dispute à l'horreur.

    \medskip
    Roman-fleuve d'une ambition rare, \textit{Tête de veau ravigote} mêle polar
    halluciné, satire sociale et méditation sur la violence. Éric Mugnier y
    déploie une prose incandescente, truculente et érudite, qui rappelle les
    grandes heures du roman noir français.%
  };

  % Filet séparateur fin
  \draw[encre!30, line width=0.25pt] (17, 40) -- (140, 40);

  % ── Bio ─────────────────────────────────────────────────────────────────────
  \node[
    anchor=north west,
    text width=118mm,
    align=justify,
    inner sep=0pt,
    text=gristexte
  ] at (17, 37)
  {%
    \fontsize{8}{11.5}\selectfont
    Éric Mugnier vit en France.
    \textit{Tête de veau ravigote} est son premier roman.%
  };

\end{tikzpicture}

\end{document}
