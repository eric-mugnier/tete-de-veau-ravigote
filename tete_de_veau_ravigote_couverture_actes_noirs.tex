\documentclass[tikz, border=0pt]{standalone}

% ─── PACKAGES ─────────────────────────────────────────────────────────────────
\usepackage[utf8]{inputenc}
\usepackage[T1]{fontenc}
\usepackage{ebgaramond}
\usepackage[french]{babel}
\usepackage{xcolor}
\usepackage{tikz}
\usetikzlibrary{calc}
\usepackage{graphicx}
\usepackage{microtype}

% ─── COULEURS ─────────────────────────────────────────────────────────────────
\definecolor{actesred}{RGB}{212,32,32}
\definecolor{greyovl}{RGB}{218,213,203}
\definecolor{greybio}{RGB}{180,180,180}

% ─── DOCUMENT ─────────────────────────────────────────────────────────────────
\begin{document}

% ─── CANVAS TIKZ (unité = 1mm, origine bas-gauche) ───────────────────────────
% Zones :
%   Fond perdu gauche   :  0 –   3 mm
%   4e de couverture    :  3 – 143 mm  (140 mm)
%   Dos                 : 143 – 161 mm ( 18 mm)
%   1ère de couverture  : 161 – 301 mm (140 mm)
%   Fond perdu droit    : 301 – 304 mm
%   Fond perdu bas/haut :  0 –   3 mm / 208 – 211 mm

\begin{tikzpicture}[x=1mm, y=1mm, font=\normalfont]

  % ── FOND GLOBAL NOIR ───────────────────────────────────────────────────────
  \fill[black] (0,0) rectangle (304,211);

  % ─────────────────────────────────────────────────────────────────────────
  % ══ 1ÈRE DE COUVERTURE (161 – 301 mm) ═══════════════════════════════════
  % ─────────────────────────────────────────────────────────────────────────

  % ── Cadre rouge (4 mm de trait) ─────────────────────────────────────────
  \draw[actesred, line width=4mm] (163, 2) rectangle (299, 209);

  % ── Auteur (rouge, gras sans-serif, centré) ─────────────────────────────
  \node[anchor=north, text=actesred] at (231, 200)
    {\sffamily\bfseries\fontsize{7.5}{9}\selectfont ÉRIC MUGNIER};

  % ── Titre (blanc, Garamond, grand, centré) ──────────────────────────────
  \node[
    anchor=north,
    text width=118mm,
    align=center,
    text=white
  ] at (231, 188)
  {\fontsize{22}{26}\selectfont
    TÊTE DE VEAU\\[4pt]RAVIGOTE};

  % ── Médaillon ovale avec photo ──────────────────────────────────────────
  \begin{scope}
    \clip (231, 76) ellipse (31.5mm and 40mm);
    % Fond crème derrière la photo (visible si la photo ne couvre pas tout)
    \fill[greyovl] (199.5, 36) rectangle (262.5, 116);
    % Photo
    \node[anchor=center, inner sep=0] at (231, 76)
      {\includegraphics[width=63mm, height=80mm]
        {images/tranche_tete_de_veau.jpg}};
  \end{scope}

  % ─────────────────────────────────────────────────────────────────────────
  % ══ DOS (143 – 161 mm) ═══════════════════════════════════════════════════
  % ─────────────────────────────────────────────────────────────────────────

  % Filets rouges sur les bords du dos
  \draw[actesred, line width=4mm] (143, 0) -- (143, 211);
  \draw[actesred, line width=4mm] (161, 0) -- (161, 211);

  % Titre + auteur vertical centré
  \node[rotate=90, anchor=center, text=white] at (152, 106)
    {\sffamily\fontsize{4.5}{6}\selectfont
      ÉRIC MUGNIER \quad•\quad TÊTE DE VEAU RAVIGOTE};

  % ─────────────────────────────────────────────────────────────────────────
  % ══ 4ÈME DE COUVERTURE (3 – 143 mm) ═════════════════════════════════════
  % ─────────────────────────────────────────────────────────────────────────

  % ── Cadre rouge ─────────────────────────────────────────────────────────
  \draw[actesred, line width=4mm] (5, 2) rectangle (141, 209);

  % ── Résumé (centré)
  \node[
    anchor=north,
    text width=112mm,
    align=center,
    text=white,
    inner sep=0pt
  ] at (73, 200)
  {%
    \sffamily\fontsize{8.5}{13}\selectfont
    Quand le père Vidal disparaît dans des circonstances troubles et que des
    colis macabres commencent à circuler dans la ville, le commandant Beauvais
    se lance dans une enquête labyrinthique. Mais le polar n'est ici qu'un fil
    rouge dérisoire : ce qui compte, ce sont les digressions – sur les chats,
    les dinosaures, l'Église, les pavillons de banlieue, la décadence de
    l'humanité.

    \medskip
    Roman-fleuve picaresque dans la lignée de Cervantes et Rabelais,
    \textit{Tête de veau ravigote} mêle la rage sadienne au pessimisme
    schopenhauerien. Éric Mugnier y déploie une prose incandescente, truculente
    et désespérée, qui rappelle autant \textit{Là-bas} de Huysmans que les
    pamphlets du XVIIIe~siècle. Inachevable, et délibérément inachevé, ce
    premier roman ignore la résolution et préfère la dérive : une logorrhée
    vengeresse et jubilatoire.%
  };

  % Filet séparateur
  \draw[white!25!black, line width=0.2pt] (14, 41) -- (132, 41);

  % ── Bio (centré, gris clair)
  \node[
    anchor=north,
    text width=112mm,
    align=center,
    text=greybio,
    inner sep=0pt
  ] at (73, 38)
  {%
    \sffamily\fontsize{7.5}{11}\selectfont
    Éric Mugnier est né en 1960. Il vit à Chaumont-en-Bassigny. Peintre,
    compositeur et parolier, il se consacre depuis quarante ans à la
    création. \textit{Tête de veau ravigote} est son premier roman.%
  };

\end{tikzpicture}

\end{document}
