%
% tete_de_veau_ravigote_sommaire.tex
% Sommaire tenant sur une page — lit directement le .toc du document principal.
% Compilation : latexmk -pdf tete_de_veau_ravigote_sommaire.tex
%
\documentclass[9pt,a5paper]{article}

\usepackage[utf8]{inputenc}
\usepackage[T1]{fontenc}
\usepackage{ebgaramond}
\usepackage[french]{babel}
\usepackage{microtype}
\usepackage{tabularx}
\usepackage[left=14mm,right=11mm,top=10mm,bottom=8mm,noheadfoot]{geometry}

\renewcommand{\baselinestretch}{1.0}
\setlength{\parindent}{0pt}
\setlength{\parskip}{0pt}

% ─── Macros du .toc ────────────────────────────────────────────────────────────

% Définitions de référence pour l'identification par \ifx dans \tocPeek.
\newcommand{\tocchapnum}[1]{#1}
\newcommand{\tocchapsep}{}
\def\chapternumberline#1{}   % entrées auto (Acte N) : no-op

% Hook babel
\def\babel@toc#1#2{}

% ─── Rendu en table 3 colonnes ────────────────────────────────────────────────
%
% Architecture sans groupe autour de & ni \\ (évite "misplaced alignment tab").
%
% Sentinelles internes (tokens rares, absents du .toc)
\def\TOCNIL{}    % délimite la capture du 1er token dans \tocPeek
\def\TOCSTOP{}   % délimite la description dans \tocChapRow / \tocSectRow

% \contentsline{type}{contenu}{page}{}
\renewcommand{\contentsline}[4]{%
  \tocProcess{#2}{#3}%
}

% Passe le contenu "nu" (avant \TOCNIL) pour l'isolation du 1er token,
% et la page en braces {#2} pour que \tocPeek capture le numéro entier.
\newcommand{\tocProcess}[2]{%
  \tocPeek#1\TOCNIL{#2}%
}

% Drapeau : premier chapitre → pas de strut extra avant lui.
\newif\iffirstchap \firstchaptrue

% Inspecte le 1er token du contenu (#1) :
%   #2 = reste du contenu jusqu'à \TOCNIL
%   #3 = numéro de page (en braces, donc multi-chiffres capturé entièrement)
\long\def\tocPeek#1#2\TOCNIL#3{%
  \ifx#1\chapternumberline
    % entrée auto "Acte N" → ignorer
  \else\ifx#1\tocchapnum
    \tocChapRow#2\TOCSTOP{#3}%
  \else
    % #1=\tocchapsep ; #2 = " description"
    \tocSectRow#2\TOCSTOP{#3}%
  \fi\fi
}

% Ligne de chapitre :
%   appel : \tocChapRow{romain}description\TOCSTOP{page}
%   #1 = romain, #2 = description (jusqu'à \TOCSTOP), #3 = page
%   & et \\ au niveau externe du tableau (pas de groupe autour).
%   Un strut invisible \rule{0pt}{13pt} crée de l'espace visuel au-dessus
%   des lignes de chapitre (sauf la première).
\long\def\tocChapRow#1#2\TOCSTOP#3{%
  \iffirstchap\global\firstchapfalse\else\rule{0pt}{13pt}\fi
  #1 & #2 & #3 \\[2pt]%
}

% Ligne de section :
%   appel : \tocSectRow " description"\TOCSTOP{page}
%   #1 = " description" (espace initial absorbé par \ignorespaces), #2 = page
\long\def\tocSectRow#1\TOCSTOP#2{%
  & \ignorespaces#1 & #2 \\[1pt]%
}

% ─── Document ──────────────────────────────────────────────────────────────────
\begin{document}
\pagestyle{empty}

\vspace*{3mm}
{\centering\normalfont\scshape\large sommaire\par}
\vspace{5mm}

{\small\scshape
% \setcounter{tocdepth}{...} parsème le .toc entre les entrées et démarre
% des lignes parasites dans le tableau ; on le neutralise ici localement.
\def\setcounter#1#2{}%
\setlength{\tabcolsep}{4pt}%
\begin{tabularx}{\linewidth}{@{}lXr@{}}
\documentclass[10pt,a5paper,twoside,openright]{memoir}

% ─── PACKAGES ─────────────────────────────────────────────────────────────────
\usepackage[utf8]{inputenc}
\usepackage[T1]{fontenc}
\usepackage{ebgaramond}
\usepackage[french]{babel}

% ─── PAGE GEOMETRY (140×205 mm – Gallimard Blanche) ──────────────────────────
\setstocksize{210mm}{148mm}
\settrimmedsize{210mm}{148mm}{*}
\settrims{0mm}{0mm}
\setlrmarginsandblock{22mm}{17mm}{*} % inner/outer
\setulmarginsandblock{22mm}{25mm}{*} % top/bottom
\setheadfoot{10mm}{12mm}
\setheaderspaces{*}{6mm}{*}
\checkandfixthelayout

% ─── TYPOGRAPHY ───────────────────────────────────────────────────────────────
\renewcommand{\baselinestretch}{1.1} % Interligne
\setlength{\parindent}{4mm}          % Alinéa
\setlength{\parskip}{0pt}            % Pas d'espace entre paragraphes

% Pas d'alinéa après titres de chapitre
\setlength{\afterchapskip}{2\onelineskip}

% ─── CHAPTER STYLE ────────────────────────────────────────────────────────────
\makeatletter
\makechapterstyle{gallimard}{%
  \renewcommand{\chapnamefont}{\normalfont\scshape\centering}
  \renewcommand{\chapnumfont}{\normalfont\scshape\centering}
  \renewcommand{\chaptitlefont}{\normalfont\scshape\centering}
  \renewcommand{\printchaptername}{}
  \renewcommand{\chapternamenum}{}
  \renewcommand{\printchapternum}{%
    \chapnumfont\thechapter}          % plus sobre : juste le numéro
  \renewcommand{\afterchapternum}{\par\vskip 2\onelineskip}
  \renewcommand{\printchaptertitle}[1]{}
}
\makeatother
\chapterstyle{gallimard}

% ─── RUNNING HEADERS : folio seul, sans auteur ni titre ───────────────────────
\makepagestyle{gallimard}
\makeevenhead{gallimard}{}{}{}        % en-tête paire : vide
\makeoddhead{gallimard}{}{}{}         % en-tête impaire : vide
\makeevenfoot{gallimard}{}{\thepage}{}
\makeoddfoot{gallimard}{}{\thepage}{}
\makepsmarks{gallimard}{%
  \nouppercaseheads
  \createmark{chapter}{both}{nonumber}{}{}
}
\pagestyle{gallimard}

% Pages de début de chapitre : sans en-tête ni folio
\aliaspagestyle{chapter}{empty}

% Pages blanches intercalées par \cleardoublepage : toujours vides
\makeatletter
\renewcommand{\cleardoublepage}{%
  \clearpage
  \if@twoside
    \ifodd\c@page\else
      \hbox{}\thispagestyle{empty}\newpage
    \fi
  \fi
}
\makeatother

% ─── DOCUMENT ─────────────────────────────────────────────────────────────────
\begin{document}
\pagestyle{empty}

% ═══ PAGE DE TITRE (p. 1, impaire) ═══════════════════════════════════════════
\begin{center}
\vspace*{6\onelineskip}
{\scshape\large éric mugnier}

\vspace{8\onelineskip}
{\scshape\huge tête de veau ravigote}

\vspace{12\onelineskip}
{\itshape roman}
\end{center}
% ═══ PAGE DE COPYRIGHT (verso du titre, p. 2) ════════════════════════════════
\newpage
\thispagestyle{empty}
\vspace*{\fill}
{\footnotesize\noindent
\textcopyright{} Éric Mugnier, 2026\\
Tous droits réservés pour tous pays\\
Reproduction interdite\\[1.5\onelineskip]
{\scriptsize\itshape Édition préparée par Christophe Thiebaud}
}
\cleardoublepage                        % p. 5 (impaire) → dédicace

% ═══ DÉDICACE (impaire) ═══════════════════════════════════════════════════════
\vspace*{10\onelineskip}
\begin{flushright}
\itshape\small
A ma grand-mère maternelle Alexandrine Chéron, née Lemaître, \\
décédée en 2022 alors qu'elle survolait la cordillère des Andes \\
à bord du Cessna Skylane 182 jaune canari qu'elle s'était, \\
contre l'avis général et tout particulièrement celui des compagnies \\
d'assurances (qui, bien qu'elle soit en pleine forme, la jugeait \\
tout de même peut-être un peu âgée pour piloter), offert en grande \\
pompe pour son quatre-vingt-septième anniversaire. Le crash aérien, \\
d'une extrême violence, n'a pas permis de retrouver l'intégralité \\
des morceaux de la chère vieille dame, en partie calcinée et dévorée \\
par les condors qui, on ne le sait pas assez, adorent la viande bien \\
cuite. Sa mort laisse un grand vide dans mon cœur.
\end{flushright}
\cleardoublepage                        % p. 6 blanche, → p. 7 (impaire) chapitre 1

% ═══ MAIN TEXT ════════════════════════════════════════════════════════════════
\pagestyle{gallimard}

\chapter{Acte 1}

\noindent Une décharge d’un milliard de volts a déchiré le ciel comme un vieux chiffon bouffé aux mites, aussitôt suivie d’une déflagration à vous faire se dresser les cheveux sur la tête et s’éjecter les dents des gencives.

Quelques heures plus tôt, drainés par des bourrasques dignes de l’apocalypse, les nuages avaient rappliqué en masse et plongé le secteur dans les ténèbres.

Gulav (Gulav Behram, dit «le Kurde») était une des pires pourritures qui aient jamais vécu à la surface de cette terre de merde, ce trou à rat qui avait vu ces cons de Romains clouer Jésus sur une croix et ces fumiers de Nazis exterminer les Juifs par treize à la douzaine. Depuis, les réjouissances n’avaient cessé de s’enchaîner à un rythme frénétique. L’horreur montait en puissance, les limites de l’abjection semblaient pouvoir être repoussées indéfiniment.

Personne ne savait d’où il sortait et tout le monde s’en foutait, à commencer par lui. Quelqu’un qui n’a aucun avenir n’a pas à se soucier de son passé. Il vivait de rapine, vol, viol, escroquerie, extorsion et autres saloperies du même genre.

En un mot comme en cent, Gulav était né pour faire le mal et n’avait cessé de s’employer à mériter pleinement ses galons de pourriture hors norme. Il donnait envie de dégueuler à tout le monde, mais les gens se retenaient pour ne pas finir avec un couteau dans le dos. En bon clébard amateur de levrette et de chiennes en chaleur, Gulav était un vicieux qui faisait ses coups par derrière. Un pro, quoi.

Or donc, ce jour-là, vla ti pas qui faisait un temps à ne pas mettre un chien dehors, précisément, mais cette sous-merde de Gulav, cet étron magistral chié dans la douleur par l’anus même de Satan qui s’était planqué tel un ver solitaire infernal dans les intestins de sa mère, était pire que le pire charognard vomi par les entrailles de la putréfaction. Il était, à lui seul, la preuve irréfutable de la non-existence de Dieu. Désolé, je m’en excuse aussi platement qu’une limande auprès des grenouilles de bénitier, cathos de droite antisémites et autres enculeurs de louveteaux, mais si Dieu avait créé une bouse comme Gulav, alors il ne méritait pas qu’on lui adresse la moindre prière, sinon celle de se faire tout petit dans son coin et fermer sa gueule à tout jamais.

Gulav s’est dit que c’était le moment idéal pour se faire une petite bicoque. Il la surveillait depuis un bout de temps et savait que les proprios s’étaient fait la malle. En vacances, certainement. Déjà que c’était des retraités qui n’en foutaient pas la rame, ils trouvaient encore le moyen de se payer du bon temps au frais de la princesse. Pendant que les jeunes se cassaient le cul pour trois francs six sous, ces fossiles vivants s’en foutaient plein la gueule au club Med ou au camping de Palavas-les-Flots. Gulav leur préparait une petite surprise pour leur retour. Non seulement il allait tout saccager, mais il ne se gênerait pas pour marquer son territoire en se soulageant sur le tapis du salon. Arsène Lupin laissait sa carte de visite, lui, c'était de la merde.

Comme à son habitude, Gulav s’est pointé dans une voiture volée, une caisse tellement pourrie que son propriétaire ne se donnerait même pas la peine de signaler sa disparition. De toute façon, personne ne se donnerait la peine de la rechercher.

Il a profité d’une brève accalmie pour se ruer hors de la poubelle, le pied-de-biche à la main. Gulav était d’une laideur telle que même les cafards dégueulaient en croisant son chemin. Les seules filles auxquelles il pouvait prétendre, sans avoir recours au viol sur fond de tortures à faire passer le marquis de Sade pour un bienfaiteur de l’humanité, étaient les paumées défoncées jusqu’à l’os qui n’auraient pas hésité à s’accoupler avec un troupeau de sangliers en échange d’une dose de came. Les gens n’imaginent pas, avachis dans le canapé en simili de leur médiocrité quotidienne, à quel point il se passe des trucs moches à deux pas de chez eux. Et s’ils parviennent malgré tout à se l’imaginer, ils s’en foutent. Au contraire, le fait de savoir que des gens crèvent la gueule ouverte dans le caniveau participe à leur bonheur. Ils ne vous le diront jamais, bien sûr, et finiront même par se convaincre que c’est faux, mais le fait est que le malheur des autres contribue très largement à leur félicité.

C’était un de ces petits pavillons de merde comme on en trouve un peu partout dans les endroits destinés à accueillir les gens qui ne souhaitent pas s’entasser dans des immeubles de cinquante étages à l’isolation douteuse. Ils préfèrent pourrir dans un cercueil individuel, en tout point semblable à celui du voisin (hormis quelques petites touches personnelles dont on se passerait volontiers), mais soigneusement délimité par des barrières physiques plus ou moins hermétiques que nul ne doit s’aviser de franchir sans laissez-passer sous peine de prendre une décharge de calibre 12 à travers la gueule. Ils veulent un petit lopin de terre avec des nains de jardin, une fontaine en plastoc, une cabane branlante pour ranger leurs outils et une pelouse tondue au ras des pâquerettes. Ils veulent la complète panoplie du parfait connard qui a tout bien réussi dans sa vie, y compris sa mort. Ses frais d’obsèques lui ont coûté tellement de pognon qu’il est pressé de crever pour en profiter. Je serais d’avis, pour économiser un  peu d’espace, d’enterrer tous ces crétins insignifiants dans leurs jardins, sous les chrysanthèmes et bégonias qu’ils arrosent à l’eau de source à longueur de journée. Les locataires s’y succéderaient et chacune de ces cryptes des temps modernes aurait droit à son petit cimetière personnel.

La plupart de ces boîtes à chaussures sont aujourd’hui occupées par des vieux (ou futurs vieux) qui votent (très) à droite et ne savent pas se servir d’un téléphone portable. Ils détestent les étrangers et les gens (surtout les étrangers) qui se garent devant chez eux. Ils ont des gosses, parce qu’avoir des gosses faisait partie de leur plan de carrière, leur vie rêvée de crétins obsolètes voués à la décrépitude. D’autant qu’avoir des gosses ne demande pas de qualités particulières : il suffit d’avoir un équipement en bon état de marche et de faire les poubelles pour trouver une conjointe. Dès qu’ils ont fini par dénicher un emploi à peu près stable, ils se sont fait construire (certains l’ont construit de leurs propres mains, on n’est jamais mieux servi que par soi-même) un de ces foutus pavillons pour former à leur image une portée de rejetons édentés qui porteront à leur tour le flambeau de la connerie vers des horizons insoupçonnés. Je devrais plutôt dire leur nid, qu’il aurait sans doute fallu dératiser avant qu’il soit trop tard. Leurs gosses ont quitté le navire dès qu’ils ont été en âge de le faire. En fait de navire, il s’agissait plutôt d’une coquille de noix sur un océan de merde balayé par les bourrasques incessantes d’un destin aussi haineux qu’une meute de chiens enragés. Aujourd’hui, ces mêmes gosses, qui ont aussi des gosses et un foyer, attendent qu’ils crèvent pour solder la bicoque et se payer du bon temps avec le fric. Ils sont la copie conforme de leurs parents, le portable en plus (auquel ils s’accrochent comme des désespérés pour tenter de donner un sens à leur vie, se sentir autre chose qu’un pion en perdition sur un échiquier trop grand pour lui), et se ruinent pour élever des gosses qui leur fausseront compagnie à la première occasion. Ils ont, accroché au mur de leur salon, le calendrier idéal d’une vie réussie, et entendent bien cocher toutes les cases avant que le croque-mort referme sur leur tronche déconfite le couvercle de leur boîte en sapin.

On pourrait dire que Gulav n’avait jamais eu de chance. On pourrait dire aussi qu’il était tellement con que tout ce qu’il entreprenait ne pouvait faire autrement que tourner rapidement au vinaigre.

Ce soir-là, je n’étais pas en service. Je dis ça parce que je suis flic, et en service la plupart du temps comme la plupart des flics. Flic est un boulot de merde, mais comme tous les boulots sont des boulots de merde, je me suis dit que quitte à faire un boulot de merde autant se balader librement avec un flingue à la ceinture. Et puis, à part soldat, c’est le dernier métier où vous avez le droit de tuer des gens en toute impunité. Sauf que si vous êtes soldat, vous êtes obligé de vivre H24 avec d’autres crétins au crâne rasé dans une caserne à la con. C’est quand même plus cool de vivre dans le vrai monde avec ses propres fringues sur le dos, sans avoir à marcher au pas, ramper dans la boue et se faire postillonner à la gueule à longueur de journée par une brute épaisse à haleine de chiotte. Et puis à l’armée, pour tuer des gens, il faut aller à la guerre, et là, vous avez de fortes chances d’en prendre une avant d’avoir eu le temps de vous faire plaisir. Trop risqué. Bien sûr que les malfrats aussi sont dangereux, mais dans la police on peut se permettre de les flinguer sans raison et bricoler une petite mise en scène pour enfumer une hiérarchie plus ou moins complaisante.

Ce vieux Gulav avait un casier long comme le bras, mais, allez savoir pourquoi, il finissait toujours par se retrouver dehors. Oui, je sais, c’est le mal du temps de foutre les prisonniers dehors. Plus assez de place dans les prisons. Un jour, il faudra demander à l’habitant d’en prendre un ou deux chez lui pour participer à l’effort collectif. À charge pour lui de les attacher dans le cave ou le grenier. Pour ce qui est de Gulav, il était tellement chiant que même les taulards n’en voulaient plus. Quand il se retrouvait en cabane, il ne fallait pas plus de deux ou trois jours pour que les détenus se mutinent et fassent une grève de la faim pour exiger son expulsion. Donc on le foutait dehors en le priant de se tenir à carreau. Naturellement il n’en faisait rien, parce qu’il était bien trop con pour comprendre un traître mot de ce qu’on lui racontait. On en avait tellement marre de voir sa gueule qu’on finissait par ne même plus se donner le mal de lui courir après.

Je m’étais souvent dit qu’un jour ou l’autre, si je ne voulais pas devenir fou à force de tourner en boucle avec cet abruti dans le paysage, je devrais me résoudre à mettre un terme définitif à ses pitoyables activités.

Il semblait que ce jour soit enfin arrivé.

Par le plus grand des hasards, puisque je rentrais paisiblement chez moi après une petite balade en ville.

Cette bouse sur pattes avait mal choisi son jour pour faire des siennes. Quand je dis «choisi» c’est une façon de parler, parce qu’en réalité il ne se passait pas une journée sans qu’il trouve le moyen de faire chier le monde d’une façon ou d’une autre.

Donc, quand j’ai vu l’autre crétin des Alpes garer sa poubelle sur le trottoir d’en face, un sourire est apparu sur mon visage, découvrant une rangée de quenottes à faire pâlir d’envie le grand méchant loup en personne (et je ne parle même pas de l’Hydre de Lerne, Alien, Smaug et Godzilla, vulgaires animaux de compagnie tout juste bons à bouffer des croquettes, miauler au coin du feu et se tortiller d’aise quand on leur gratte le bide).

Je me suis garé un peu plus loin et l’ai regardé se faufiler dans les ténèbres liquides tel le reptile abject qu’il n’avait cessé d’être depuis le jour maudit de sa naissance, neuf mois environ après que son abruti alcoolique de géniteur avait cru malin de fourrer son zguègue dans le cloaque maternel, sorte de jungle marécageuse infestée de bestioles qui auraient fait s’enfuir à toutes jambes le plus endurci des aventuriers. Même un Livingstone (je parle de David, pas de Jonathan le goéland qui n’aurait pour rien au monde accepté de mettre une plume dans ce bourbier) aurait longuement hésité avant de tenter l’expédition. Il n’était pas rare, au détour d’un buisson malodorant grouillant d’araignées venimeuses et autres grenouilles tueuses, d’y croiser des créatures mutantes à mi-chemin entre le tapir et l’anaconda, l’alligator et le singe hurleur, la panthère noire et la sangsue de cinq mètres de long. Ce terrain vague, cette zone de non-droit hantée par les serviteurs du Mal aurait dû être éradiquée de la carte depuis bien longtemps. D’ailleurs, quand je dis «cru malin» (cf. ci-dessus), je devrais plutôt dire «tragique concours de circonstances qui l’avait, au détour d’une beuverie dans l’un ou l’autre de ces bars louches où il avait ses habitudes, conduit à commettre un geste irréparable dont l’Humanité n’allait pas tarder à payer le prix fort». Si la Nature avait ne serait-ce qu’un semblant d’intelligence, elle ferait en sorte que de telles abominations n’aient aucune chance de se produire. Au lieu de ça,  cette conne pousse ses ressortissants à forniquer à tout-va sans se soucier des conséquences.

Grand seigneur, je lui ai laissé un peu de temps pour se mettre à l’aise, prendre ses marques. Après quoi, j’ai vérifié que «Manu», mon 6.35 Manufrance (arme de collection, chargeur sept coups qu’il convient de vider d’une traite sur l’agresseur pour s’assurer d’une réelle efficacité létale, pas très moderne, je vous l’accorde, voire vieillot, mais j’aime flinguer français) était chargé à bloc et suis, avec la souplesse d’un félin d’une grâce infinie (un peu empâté, le félin, diront les mauvaises langues), sorti de ma caisse pour lui emboîter le pas.

Je vous rappelle que des rideaux de flotte dégringolaient du ciel illuminé par un feu d’artifice d’éclairs tonitruants, situation assez désagréable sur le plan épidermique, mais globalement favorable pour progresser de façon discrète dans un environnement hostile. Si quand même, vous ne m’enlèverez pas de l’idée que ce genre de quartier discret aux trottoirs impeccables, peuplé de patriotes armés jusqu’aux dents, reste un environnement assez hostile. Souvent âgés et insomniaques, maniaques, paranoïaques, durs de la feuille et à moitié aveugles, pétris de ressentiment à l’égard d’une existence qui n’a cessé de leur pourrir consciencieusement la vie, les patriotes en question ont tendance à tirer dans le tas au moindre pet de travers.

Comme indiqué précédemment, Gulav était une saleté de clébard sournois qui faisait ses coups par derrière. Au lieu de passer par devant, il avait préféré faire le tour de la baraque et s’attaquer à la porte de derrière, laquelle donnait sur une vague terrasse et un jardinet ceint d’une haie de thuyas du plus bel effet. Et à en juger par l’état de celle-ci (la porte de derrière, pas la haie de thuyas taillée au cordeau), il n’avait pas à proprement parler choisi la méthode douce. Au lieu de travailler la serrure en souplesse, comme tout monte-en-l’air qui se respecte, il l’avait démontée au pied-de-biche en arrachant la moitié de la porte. Ce type était la honte de la profession, et l’intersyndicale des cambrioleurs m’avait adressé plusieurs suppliques pour faire cesser le massacre.

Trempé comme une soupe, le 6.35 à la main, je me suis faufilé à l’intérieur. J’avais tout du héros des temps modernes, l’espion sur l’échiquier international du Mal, le cavalier noir, le fou prend la reine (par derrière), la tour dans le cul du roi, le reptile subtil qui se glisse en silence dans les interstices du vice pour que surgisse in extremis la justice.

L’endroit, même pour un héros des temps modernes lancé à pleine vitesse dans les corridors de la vengeance, était d’un classicisme déprimant qui ne donnait aucune envie de s’attarder. Se sauver en courant, plutôt. Aucune personnalité, que des objets bas de gamme accumulés au cours d’une existence totalement dépourvue d’intérêt, aussi morne et plate qu’un paysage de Wallonie. Il fallait vraiment que Gulav ait du temps à perdre pour s’introduire là-dedans, ou alors il disposait d’informations laissant entendre qu’un magot était caché quelque part. Mais c’était aussi improbable que de voir une soucoupe volante atterrir dans le jardin et une fournée de Télétubbies en descendre pour ramasser des choux avant de rentrer chez eux. Gulav était du genre menu fretin : un petit collier par-ci, une bague par-là, une petite culotte fraîchement portée, quelques pièces de monnaie suffisaient à son bonheur. Son pied, il le prenait en violant l’intimité des gens ou les gens eux-mêmes à l’occasion, les dames d’un certain âge de préférence, bien mûres, à la limite du blet, qu’il prenait plaisir à martyriser avant de prendre la fuite en saccageant tout sur son passage, non sans leur avoir auparavant tatoué un G entre les omoplates avec la pointe de son couteau à cran d’arrêt. Un G comme Gulav (ou encore Gibbon, Gencive, Gaz puant, Génocide, Glauque, Gonorrhée, Gras du bide, Gymnosperme, etc), sa carte de visite sur peau humaine.

Après une rapide visite du propriétaire, j’ai trouvé mon Gulav à quatre pattes dans le salon en train de braquer le faisceau de sa torche sous une commode Louis XV. Du faux, bien sûr, du Louis le Bien-Aimé made in Taïwan in 1995 par des enfants en bas âge élevés à l’alcool de riz et à coups de trique dans le bas du dos. Je ne sais pas ce que cet abruti cherchait et m’en foutais royalement. Seule la vue réjouissante de son postérieur gentiment offert à la semelle de ma chaussure offrait pour moi un quelconque intérêt.

Il s’est pris son coup de pompe pleine bourre et est allé gentiment s’écraser contre la commode Louis XV dont il inspectait les dessous avec tant d’attention. Il a laissé tomber la torche au passage. J’ai profité de ce qu’il était en train de rassembler ses esprits pour la ramasser et la lui braquer en pleine face.

Il s’est mis à cligner des yeux, l’air bovin, la gueule ouverte, le nez en sang, et une série de grognements dont j’ai été incapable de décrypter la signification éventuelle est sortie de son gosier.

\textsc{Votre serviteur} (pour info, je m’appelle Djeferson Beauvais, avec un D et un seul F, je vous le dis maintenant, ce sera fait) : Ça va, Gulav ?

Allez savoir pourquoi, mon père était un fan de Thomas Jefferson, riche propriétaire terrien et planteur de Charlottesville, dans le comté d’Albemarle en Virginie, mathématicien, naturaliste (il fréquentait le baron von Humboldt, membre de l’Académie des sciences et président de la Société géographique de Paris), horticulteur, inventeur (d’une machine à faire les nouilles et du cylindre de chiffrement polyalphabétique qui porte son nom), lockiste, rousseauiste et polyglotte, troisième président of the United States of America après George Washington et le peu connu John Adams, pote de Condorcet et d’Alembert (du temps où il était ambassadeur à Paris), grand amateur de bonne chère, notamment les macaronis, les gaufres, le muscat de Frontignan et le vin en général, spécialités qu’il se fait un devoir de ramener dans ses valises pour les faire découvrir à ses concitoyens enthousiastes. Esclavagiste (on lui attribue un cheptel de près de six cents esclaves), certes, comme tout bon Américain du Sud qui se respecte, et pas que les Américains du Sud, du reste, même s’ils étaient et sont restés parmi les plus réfractaires à l’émancipation des Noirs, l’exploitation des races dites «inférieures» étant alors une pratique assez répandue dans le monde, mais progressiste et nullement insensible aux charmes de la communauté noire, des femmes (jeunes) notamment, et tout particulièrement ceux d’une certaine Sally Hemings, laquelle, si on en croit la rumeur et surtout les analyses ADN réalisées à la fin des années 90 sur certains de ses descendants, lui aurait donné au moins un fils, élevé dans le plus grand secret dans sa propriété de Monticello. Mon paternel (paix à son âme corrompue et maintes fois souillée par le péché) étant manifestement infoutu d’orthographier correctement le nom de ce glorieux personnage, ou désireux de le franciser ou pire encore de faire preuve d’originalité, chose qui ne lui ressemblait pas, j’avais hérité de ce prénom ridicule.

\textsc{Lui} : Qu’est-ce que vous faites là, commandant ?

\textsc{Moi} : Rien, je passais dans le coin, j’ai vu de la lumière, je me suis dit : tiens, je parie que ce vieux Gulav est encore en train de faire des siennes. J’entre, je jette un œil et là bingo qui je vois à quatre pattes dans le salon en train de fureter sous la commode ? mon vieux copain Gulav, plus laid et répugnant que jamais ! Tu cherches quelque chose de particulier ?

Mon Gulav, chafouin : j'ai perdu une boucle d’oreille.

\textsc{Moi} : Il serait temps que t’arrêtes de tapiner, mon lapin, c’est plus de ton âge. En plus, ta perruque est de travers, on dirait une vieille folle.

\textsc{Lui} : Vous allez me laisser partir ?

\textsc{Moi}, lui agitant le 6.35 sous le nez : Et comment ! Je vais même t’aider à le faire. Tu sais qui c’est, ça ?

Il a hoché la tête en faisant la moue (un truc horrible qui lui donnait l’air d’un vieux mérou constipé) pour signifier que non seulement il n’en savait rien, mais qu’il s’en fichait comme de sa dernière dent creuse.

\textsc{Moi} : C’est Manu, mon fidèle équipier. Et Manu, il ne peut vraiment pas te saquer.

\textsc{Lui} : On peut trouver un arrangement, chef. Je vous file la moitié du butin et vous me laissez filer.

\textsc{Moi} : C’est tentant, je l’avoue. Sauf que Manu n’est pas du tout de cet avis.

C’est là qu’il a essayé de se jeter sur moi.

C’était bien essayé mais Manu a été plus rapide. Perso, j’aurais pu me contenter de lui sonner la cloche à coups de crosse, mais Manu était du genre hypersensible, surtout au niveau de la détente. Le coup est parti très vite (900 km/h environ) et la balle est entrée direct dans le cœur du sujet, celui de Gulav en l’occurrence. Comme j’ai déjà eu le plaisir de vous l’expliquer, le 6.35 est un calibre sympathique mais peu efficace en termes de rentabilité mortifère, raison pour laquelle une seconde balle s’est aussitôt échappée du canon en direction du crâne de l’intéressé, suivie d’une troisième qui allée s’échouer dans une zone assez indistincte mais néanmoins sensible de sa masse corporelle.

Il s’est affalé et mis à couiner comme un porc à l’agonie. Vraiment aucune dignité. Je ne suis pas spécialiste de la question, mais à vue de nez il lui restait quelques minutes à se tordre douleur avant de crever. J’aurais pu tirer un fauteuil, allumer un cigare (j’avais justement un Don Carlos de Fuente qui trainait dans le fond de ma poche) et le regarder tranquillement se vider de son sang, mais il commençait à se faire tard et j’avais hâte de retrouver la quiétude de mon foyer, mes vieux livres poussiéreux et la tiédeur de mes draps.

J’ai donc, dans un accès d’humanité assez inhabituel, vidé le restant du chargeur dans sa carcasse, mettant du même coup un arrêt définitif à la carrière de Gulav Behram, dit «le Kurde». Plus jamais les vieilles dames en chemise de nuit ne verraient débarquer sa sale gueule dans l’embrasure de leurs portes, plus jamais il ne prendrait congé de ses copains de beuverie en leur plantant un couteau dans le dos, plus jamais il ne fendrait des crânes à coups de hache ni ne volerait des bagnoles pour défoncer les devantures de magasin. Une nouvelle ère commençait.

Cela dit, je suis un perfectionniste. Certains salopent tout et repartent sans se soucier des conséquences, moi j’ai été élevé dans le respect des autres et moi-même à travers eux. Je n’allais pas laisser traîner cette grosse merde de Gulav au milieu du salon. Je l’ai attrapé par les pieds, tirer jusqu’à l’entrée, devant laquelle il avait eu la riche idée de garer sa poubelle, et installé au volant comme si de rien n’était. Avant de partir, j’ai refait le plein et lui ai relogé une demi-douzaine de bastos dans le buffet, histoire d’être bien certain qu’il n’allait pas de réveiller et se faire la malle.

Celui ou celle qui allait tomber sur cette grosse andouille truffée de plombs le lendemain en promenant son chien allait faire une drôle de tête.


\noindent Ça a officiellement commencé le jour où Rose Barbet, née Lortie, 75 ans, a reçu un putain de colis. Un colis assez lourd et volumineux. Quelques instants après la livraison, alors qu’elle avait encore le colis dans les mains, Thierry, son mari, est rentré avec une baguette sous le bras. Chaque jour, plus ou moins à la même heure, il rentrait avec une baguette sous le bras, la baguette du pain quotidien que sa femme et lui partageait religieusement depuis bientôt quarante ans (même si c’est lui qui en mangeait les trois quarts). Un record de longévité qui aurait pu leur valoir une citation à l’Ordre national du Mérite conjugal, ou, pour Thierry, à comparaître devant la justice pour entreprise de destruction à long terme d’une personne psychologiquement vulnérable, Rose ayant le profil type de l’épouse soumise et endurante, corvéable à merci, prête à tous les sacrifices pour assurer la survie de son ménage.

Donc, comme je le disais, Thierry est rentré avec sa baguette sous le bras, et, après avoir jeté un coup d’œil déjà légèrement embué par l’alcool (il avait profité de sa sortie pour s’enfiler quelques verres avec ses potes, rituel auquel il ne dérogeait que dans les cas de force majeure, et potes qu’il retrouvait en fin d’après-midi pour s’enfiler à nouveau quelques verres, nettement plus nombreux que ceux qu’il avait ingurgités dans la matinée, sachant qu’il avait entretemps avalé une bonne demi-bouteille de vin pendant le repas de midi, bouteille qu’il finirait le soir avant de s’endormir devant la télé) sur sa femme et surtout le colis, il a posé la question suivante : Qu’est-ce que c’est que ça ?

Rose a répondu : Je sais pas.

\textsc{Lui} : T’as commandé quelque chose ?

C’était assez peu probable dans la mesure où Rose n’avait jamais rien commandé de sa vie, et qu’on la voyait assez mal, à 75 piges, se mettre à commander des trucs sur Amazon (dont elle n’avait d’ailleurs jamais entendu parler, pas plus qu’elle ne savait se servir d’un ordinateur ou un smartphone). Non seulement elle n’avait jamais rien commandé, mais personne ne lui avait jamais rien envoyé de plus conséquent qu’une lettre ou une carte postale, lesquelles lettres provenaient la plupart du temps d’organisme de recouvrement de créances diverses, et rarissimes cartes postales des quelques résidus familiaux dont elle pouvait encore se prévaloir ici et là.

Elle a dit : Non.

Il a posé sa baguette sur la table de la cuisine et dit : Donne-moi ça.

Elle lui a donné le paquet, il a constaté que ses nom et adresse figuraient bien sur l’étiquette, puis, après avoir examiné l’objet sous toutes ses coutures, il s’est dit qu’il y avait peu de chances qu’il s’agisse d’une bombe (qui se donnerait la peine de faire sauter sa femme et un crétin comme lui ?) et s’est résolu à l’ouvrir. À noter que le paquet ne faisait aucune mention d’un quelconque expéditeur.

Je vous fais grâce des détails et autres péripéties concernant l’ouverture de ce colis, sachez seulement que l’expéditeur en question n’avait pas lésiné sur les moyens pour en garantir l’intégrité.

Quelques minutes plus tard, le temps que les époux Barbet retrouvent leurs esprits (surtout Rose, qui avait fané d’un coup en découvrant le pot aux roses — c’est d’un goût douteux, je vous l’accorde, mais il faut bien décompresser un peu quand les événements se précipitent), les forces de l’ordre, votre serviteur en tête, débarquaient dans la place toutes sirènes hurlantes. Le grand jeu à l’américaine, avec crissements de pneus et portières qui claquent. Il y avait aussi, histoire de gonfler un peu le niveau sonore, les services d’urgence pour ranimer Rose qui gisait sans connaissance dans le canapé à fleurs du salon.

Thierry et le colis étaient là aussi, bien sûr, plus quelques voisins attirés par le vacarme qu’il a fallu faire dégager à coups de pompe dans le cul.

Thierry, pour se remettre de ses émotions, était en train de siffler une flasque de cognac bon marché.

Il faut dire que le contenu du colis n’était pas des plus ordinaires.

Pour faire court, il s’agissait de ce qu’on appelle communément une tête, humaine en l’occurrence.

Naturellement, lorsqu’on trouve une tête (humaine surtout, mais on se poserait aussi des questions s’il s’agissait d’une tête de chien ou de mouton) dans un colis, la première chose à faire est de savoir à qui elle appartient, ou plutôt appartenait, car son propriétaire n’est plus en état de la réclamer.

Il faut aussi se demander comment et pourquoi cette tête a atterri dans ce colis, et accessoirement, quand ce colis a été réceptionné par quelqu’un, qui le lui a adressé et pourquoi à lui en particulier.

Telles sont les interrogations auxquelles l’enquêteur se voit rapidement confronté.

Pour mener à bien sa mission, il dispose d’un certain nombre de ressources scientifiques, parmi lesquelles le précieux concours d’un médecin légiste, étrange personnage qui, à défaut de soigner les gens, tente de comprendre pourquoi ils sont morts.

Voilà comment la mystérieuse tête s’est retrouvée sous le regard expert du docteur Zaahid Shirani, personnage complexe aux yeux de braise pour qui la mort n’avait aucun secret. Ses grands-parents, des intellectuels hindous, avaient fui le Bengale après la partition de 47, pour se réfugier à Londres d’abord, après un parcours riche en rebondissements, puis en France.

Si l’on s’en tient aux connaissances actuelles de la communauté scientifique en matière de vie et de mort, un être humain quel qu’il soit n’a que très peu de chances de survivre à l’ablation de sa tête. Mais il peut aussi être mort bien avant qu’on la lui coupe, chose que Shirani devrait s’employer à déterminer.

La réponse n’a pas tardé à tomber : oui, il était mort bien avant qu’on la lui coupe.

Et comment était-il mort ?

Il serait, en l’absence de corps, difficile de répondre à cette question.

Par contre, la tête présentait quelques particularités qui pouvaient se révéler intéressantes dans le cadre d’une enquête policière.

J’ai ouvert la porte du Frigo et une forte odeur de shit m’a aussitôt sauté au visage.

Le Doc se trouvait un peu plus loin, pétard au bec, en train de faire joujou avec la tête.

J’ai dit : Ça va, Doc ?

Il a tourné vers moi son beau visage buriné par les embruns du Pacifique Sud (cet enculé passait toutes ses vacances à Bora-Bora, à faire du bateau et bouffer de la langouste) et répondu : On fait aller, inspecteur.

Il a sorti le pétard de son bec et l’a tendu dans ma direction : Vous en voulez ?

\textsc{Moi} : Jamais ! Ça me file des migraines et des boutons sur la gueule.

\textsc{Lui} : Vous avez tort, c’est du tout premier choix. Je le fais venir tout spécialement des fertiles vallées du Cachemire. Une goutte de Feni, peut-être ?

À chacune de mes visites, il me proposait un verre de Feni, un alcool de cajou très populaire dans son pays de sauvages, et ma réponse était toujours la même : Non merci.

J’avais commis l’erreur, par charité chrétienne, de tremper une fois mes lèvres dans cette abomination liquide, et je vous prie de croire que l’expérience m’avait servi de leçon.

Je lui ai demandé s’il avait du nouveau concernant la tête que je lui avais confiée.

Il a répondu, avec la moue du type qui n’a pas pour habitude de se casser le cul pour des prunes et pour qui avoir du nouveau est d’une telle banalité qu’il ne convient même plus d’en faire état : Approchez, je vais vous montrer.

Il m’a alors expliqué que la boîte crânienne en question (celle d’un homme qu’une quarantaine d’années pourvu d’une dentition exécrable) avait été méticuleusement décalottée, un peu à la façon d’un œuf à la coque, et entièrement vidée de son contenu. Pourquoi faire me direz-vous ? Eh bien tout simplement pour remplacer ledit contenu par un autre, à savoir des croquettes pour chien à l’agneau et au riz dont la marque restait à déterminer. Après quoi la partie supérieure du crâne avait été soigneusement remise en place et hermétiquement soudée, de sorte qu’il fallait vraiment se pencher de près sur la question pour déceler quoi que ce soit.

J’ai sorti un Hemingway Short Story que je me suis aussitôt inséré entre les dents, avant de l’allumer à la flamme de mon briquet (c’est un perfecto très facile à allumer), et j’ai demandé au Doc : Une idée, Doc ?

\textsc{Le Doc} : Je suis légiste, pas psychiatre.

\textsc{Moi} : Vous pensez que seul un dingue peut avoir fait une chose pareille ?

\textsc{Lui} : Non, pas forcément. C’est peut-être juste un type qui aime les chiens et a le sens de l’humour. Est-ce qu’il faut être dingue pour tuer quelqu’un ? Bien sûr que non. Beaucoup de gens ont envie de le faire et certains ne parviennent pas toujours à se retenir. Est-ce qu’il faut être dingue pour farcir le crâne de quelqu’un avec des croquettes pour chien à l’agneau et au riz ? J’ai envie de dire non, pas davantage.

Votre humble serviteur, nettement plus détendu après avoir tiré quelques bouffées de cigare : Soit. Mais vous ne pensez pas qu’il faut être dingue pour envoyer une tête au premier venu ? Les Barbet sont des gens sans histoire, totalement insignifiants, qui ont été manifestement choisis au hasard.

\textsc{Le Doc} : C’est assez drôle, non ?

\textsc{Moi} : Vous trouvez ?

\textsc{Lui} : Tant qu’à faire de s’amuser à farcir des têtes avec des croquettes pour chien, autant les envoyer au premier venu. C’est une question de logique. Une logique un peu particulière, certes, mais une logique quand même. Dites donc, mon vieux, je pense que je vais aller faire un tour rue de la Cloche, demain soir. Vous m’accompagnez ?

\textsc{Moi} : Une autre fois, Doc. Je n’ai vraiment pas la tête à ça.

Pour info, parce que je suppose que vous brûlez d’en savoir plus, je peux vous confier qu’il se trouve rue de la Cloche un endroit connu des initiés sous le nom de Narcisse Rose. Cet endroit, admirable s’il en est, n’est ni plus ni moins que ce qu’il faut bien appeler un lupanar. Mais attention, pas n’importe lequel. C’est un endroit à l’ancienne, très classe, dans le plus pur respect des traditions de la Troisième République (temps béni des maisons closes, avant que Marthe Richard\nf{Marthe Richard, née Marthe Betenfeld (1889--1982), aventurière, espionne et femme politique française. Ancienne prostituée et espionne pendant la Première Guerre mondiale, conseillère municipale de Paris en 1945, elle déposa l’amendement qui entraîna la fermeture des maisons closes en France par la loi du 13 avril 1946. Sa biographie recèle de nombreuses affabulations. \source{fr.wikipedia.org/wiki/Marthe\_Richard}}, péripatéticienne repentie et mythomane notoire, plus pathétique qu’aristotélicienne, ne réclame officiellement leur fermeture en 45 et ne l’obtienne un an plus tard), d’un goût très sûr, un peu orientalisant, fréquenté uniquement, tant au niveau de la clientèle que du personnel, par des gens de toute première qualité, en gros de la pute fermière élevée au grain et du gentleman premier choix.

Je ne suis pas très porté sur la chose, mais il m’arrive, dans mes moments de déprime les plus intolérables, quand j’ai lu tous les livres et fumé tous les cigares, d’aller trouver refuge dans les bras d’une de ces filles qui font profession du bien-être de leurs concitoyens. Mais bon, ça va, je ne vais pas non plus entrer dans les détails de ma vie privée, ni m’excuser parce qu’il m’arrive parfois de me sentir seul au point de rechercher la compagnie d’une personne physiquement attractive et affectivement neutre. Un peu de fraîcheur, même abondamment souillée par ces pratiques impies que la nature inflige à ses subordonnés, ne fait pas de mal de temps à autre.

Zaahid Shirani avait parlé de logique, et la logique voulait que la plaisanterie ne s’arrête pas en si bon chemin.

C’est maintenant que je vais être obligé de vous parler, même si je n’en ai pas la moindre envie, du dénommé Dylan Passereau, chauffeur routier de son état, grossier personnage au physique assez peu avantageux et aux capacités intellectuelles proches de l’inexistence, ce qui n’est bien évidemment, je m’empresse de le préciser, pas le cas de tous les chauffeurs routiers, même si on ne peut décemment nier qu’un certain nombre d’entre eux s’inscrivent pleinement dans cette définition. Car c’est bien lui, Dylan Passereau, homme de route, qui avait, pas plus tard que le lendemain, hérité d’un second colis, renfermant cette fois les organes génitaux d’une personne de sexe masculin conservés dans un bocal de liquide incolore qui s’est avéré plus tard n’être ni plus ni moins que de l’alcool ménager.

Il avait aussitôt appelé les forces de l’ordre, et comme il était le second, dans un même périmètre (il résidait à moins de cinq cents mètres de chez les Barbet), à recevoir un colis de ce genre, on pouvait facilement imaginer que le service trois pièces dans le formol et la tête farcie de croquettes pour chien appartenaient au même individu.

Les analyses ont confirmé que c’était le cas.

Les flics, à commencer par moi, ont enquêté, et, à force de recoupements, sont arrivés à cerner deux ou trois suspects aussitôt placés sous surveillance. D’autre part, on avait appris que le père Clément Vidal, 45 ans, curé de l’église Notre Dame du Perpétuel Secours, avait disparu sans laisser de trace quelque temps avant que la vague de colis commence à déferler sur la ville.

Les analyses ont également confirmé que la tête et les organes génitaux, plus quelques menus articles réceptionnés ici et là par des habitants ulcérés de la ville, appartenaient bel et bien au père Vidal. Au rayon des dommages collatéraux, il est à noter que le pied droit de l’intéressé avait causé la mort de Valérie Renou, retraitée de la fonction publique dont le cœur d’artichaut n’avait pas supporté la vision d’horreur qui lui était imposée.

Solange Jacquard était ce qu’on appelle une personne très pieuse. Piteuse, aussi, car, à bientôt quatre-vingt-dix ans, elle était dans un état de conservation pour le moins discutable.

Solange allait à la messe tous les dimanches, bien sûr, et hurlait les cantiques avec une voix de fausset qui faisait se lézarder les murs de la petite église où elle avait ses habitudes (et les tympans de ses voisins, lesquels s’étaient progressivement éloignés, de sorte qu’elle se retrouvait maintenant seule dans le périmètre déclaré zone sinistrée par le diocèse). Elle avait toujours été très laide, ce qui n’avait pas facilité son rapprochement avec les humains, mais n’était pas rancunière pour un sou, ce qui lui avait permis d’entretenir d’excellentes relations avec Dieu, lequel lui devait bien ça compte tenu de l’existence misérable qu’il l’avait contrainte à mener depuis le jour de sa naissance.

Tous les troisièmes jeudis du mois, la vieille taupe allait se confesser. Allez savoir pourquoi, elle était quasi quotidiennement la victime éplorée de vilaines pensées à caractère sexuel qui pesaient lourdement sur sa conscience passablement délabrée. Oui, ça peut sembler bizarre à quatre-vingt-dix ans, mais il n’était pas rare que son regard vitreux s’attarde exagérément sur la poitrine débordante ou la croupe avenante de telle ou telle frétillante jeune fille. Attirée par les personnes de son sexe depuis sa plus tendre enfance, cette conne n’avait jamais trouvé le moyen de peloter une belle paire de miches ou bouffer une chatte bien juteuse à pleines dents, d’où une certaine aigreur qui suintait par tous les pores de sa peau de vieille fille desséchée. Il lui arrivait aussi, pendant ses crises de délirium nocturnes, telle une vierge en proie aux assauts du démon, de se fourrer des cierges et des courgettes dans le fion.

Heureusement pour elle, tous les troisièmes jeudis du mois, le père Beaubois l’attendait à dix-sept heures tapantes pour écouter religieusement ses inepties et redonner à son âme putréfiée un semblant de fraîcheur divine.

Ce jeudi-là, donc, comme à son habitude, Solange Jacquard, plus laide et aigrie que jamais, la chatte épilée de frais car elle ne supportait pas d’avoir un rat d’égout dans la culotte (et surtout elle aimait sentir le contact du tissu sur ses chairs), s’est pointée à son rendez-vous, le souffle court et le pied creux. Elle avait hâte de lui annoncer que de nouveaux locataires, les Brochard (elle était allée renifler le nom sur la boîte aux lettres), venaient d’emménager en face de chez elle.

Laurent et Frédérique Brochard, la cinquantaine, avaient deux enfants, un garçon et une fille. Le garçon devait avoir dans les treize-quatorze ans, la fille pas loin de dix-sept ou dix-huit. C’était cette dernière, d’une beauté assez dommageable collatéralement, qui avait retenu l’attention de Solange, au point qu’elle qui ne dormait déjà pas beaucoup passait maintenant ses nuits à s’astiquer le bouton. Les vilaines pensées voletaient autour d’elle comme un essaim de mouches à merde, se posant avec insistance sur les points les plus sensibles de sa personne, ne lui laissant aucun répit. À ce stade de décomposition morale, seul le père Beaubois pouvait encore son âme des flammes de l’enfer.

Solange connaissait le chemin. Après quelques rapides signes de croix en direction de l’autel et des autorités religieuses statufiées ici et là, elle est allée droit sur le confessionnal où l’attendait le père Beaubois.

Elle s’est agenouillée à l’endroit réservé aux pénitents, a débité les formules d’usage, et attendu que le père Beaubois l’autorise à vider son sac. Mais le père Beaubois, dont elle devinait le visage à travers la grille, est resté de marbre. Peut-être s’était-il endormi à force d’attendre. Elle a répété les formules d’une voix plus forte, mais le père Beaubois n’a pas esquissé l’ombre d’un mouvement ni daigné ouvrir la bouche pour l’autoriser à poursuivre. Peut-être qu’il ne dormait pas mais qu’il était mort. Après tout, on peut faire une crise cardiaque n’importe où. Peut-être qu’il avait reçu le diable en personne en confession et que son cœur avait lâché, comme celui du père Merrin dans L’Exorciste, quand cette saleté de Pazuzu l’insulte copieusement et traite sa mère de pute.

Après avoir beuglé en vain pour attirer son attention, elle s’est résolue à s’extraire de son logement pour aller s’assurer que le père Beaubois était encore de ce monde.

D’une main tremblante, elle a ouvert la porte centrale et aussitôt poussé un long hurlement qui s’est répercuté sans fin sous les arches de l’église, heureusement déserte à cette heure-ci.

Le père Beaubois était bien là, assis à sa place, mais on lui avait ouvert le ventre, et tout ce qui se trouvait à l’intérieur se trouvait maintenant dans un seau en plastique positionné entre ses jambes. Solange Jacquard, qui n’avait pas gerbé depuis ses dix-sept ans, année de sa première et dernière cuite, a aussitôt recraché les trois parts de tarte aux pommes qu’elle avait ingurgitées avant de venir, lesquelles parts de tarte à moitié digérées ont atterri dans le seau du père Beaubois, avec les restes de blanquette de son repas de midi.

Je suis arrivé une demi-heure plus tard, en compagnie de Titus Beaugendre, mon plus fidèle lieutenant, et d’une escouade de la police scientifique. Il s’est avéré qu’en plus de ses pieds et mains, le père Beaubois avait été délesté de ses organes génitaux (lesquels ne lui servaient en principe pas à grand-chose, mais on sait aujourd’hui que le vœu de chasteté est loin d’être unanimement respecté par les membres du clergé, et ce quelle que soit leur taille). Ensuite, son crâne d’œuf avait été ouvert avec la même précision chirurgicale que celui du père Vidal, également vidé de sa matière cérébrale, puis farci avec des croquettes pour chien dont il y avait tout lieu de penser qu’il s’agissait de croquettes à l’agneau et au riz de la marque Waterflox, uniquement disponible en boutique de luxe pour animaux (le genre de boutique où on trouve aussi des casquettes en velours avec des trous pour les oreilles et des petits blousons en cuir pour chien). Enfin, la calotte prélevée sur le crâne lors de son ouverture, d’un diamètre sensiblement équivalent à celle qui recouvrait jadis la tonsure des clercs (sans doute pour qu’ils n’attrapent pas un rhume de cerveau en déambulant dans les travées glaciales des édifices religieux), avait été soigneusement remise à sa place et la cicatrice maquillée avec une telle habileté qu’elle était presque impossible à distinguer.

La vieille bique tenait tellement à rester auprès de son cher curé qu’on a dû l’expulser à coups de pompe dans le cul. Déjà qu’elle avait foutu de la tarte plein la blanquette, on n’allait pas la laisser continuer à saboter la scène de crime. Après quoi, alors que j’avais instamment donné l’ordre de ne laisser entrer personne, pas même le Pape, une espèce de prélat, genre nonce, évêque, cardinal ou Dieu sait quoi, s’est pointé, avec sa calotte mauve sur le crâne, ses narines pleines de poils, ses oreilles décollées et ses yeux exorbités derrière d’épaisses lunettes à monture d’écaille, et a commencé à gesticuler et couiner comme un beau diable en disant qu’il tenait à être informé de tout et tenu au courant heure par heure du déroulement de l’enquête. Naturellement, il n’était pas question que la presse entende parler de cette regrettable affaire, sans doute l’œuvre d’un fou échappé de l’hôpital psychiatrique le plus proche. On l’a foutu dehors aussi, même s’il a été un peu difficile à manœuvrer que la vieille gouine, jurant qu’il se plaindrait de nos méthodes en haut lieu. Titus, homme de conviction mais anticlérical primaire capable de réactions d’une rare violence en présence d’une soutane, tenait absolument à lui défoncer la gueule, de sorte qu’il m’a fallu user de toute ma force de persuasion pour le ramener à de meilleurs sentiments. Une fois le calme revenu, on a pu se remettre à bosser dans de bonnes conditions, même si un public nombreux, alerté par le va-et-vient incessant des sirènes et gyrophares, commençait à se masser de façon inquiétante aux portes de l’édifice.

Certaines personnes portent bien leur nom, d’autres pas.

Dylan Passereau, par exemple, portait assez mal le sien.

Vous vous rappelez de lui ? C’est le routier qui avait reçu la bite et les couilles du père Vidal dans un bocal.

Alors que le passereau est généralement un petit animal frais et léger qui virevolte de branche en branche en sifflant des airs entraînants, Dylan était quant à lui une sorte de monstre préhistorique haut de deux bons mètres et taillé à la serpe dans un fût de séquoia, pourvu de surcroît d’une voix caverneuse qui renvoyait aux premiers âges de l’humanité. Plus proche du gorille des montagnes que de l’oiseau-mouche, il émanait de sa personne une forte odeur de tripes à la mode de Caen. Ses épaules étaient si larges qu’il explosait systématiquement toutes les vestes qu’il tentait d’enfiler. Allez savoir pourquoi, sans doute parce qu’il me faisait l’effet d’un pithécanthrope capable des abominations les plus régressives, je n’avais jamais réellement cru à son histoire de bocal de couilles. Il m’était instantanément venu à l’esprit qu’il avait très bien pu se les envoyer à lui-même, histoire de narguer la police en se faisant passer pour une victime du Brain Catcher, surnom que j’avais trouvé pour notre tueur de curés en raison de sa manie de prélever la cervelle de ses victimes. Sans me vanter, je passais pour avoir le nez assez creux lorsqu’il s’agissait de démasquer les imposteurs, et Passereau me semblait avoir le profil idéal de l’enculé de service.

Naturellement, je ne m’étais ouvert de mes doutes à personne, tenant à m’occuper moi-même de Passereau au cas où j’arriverais à établir sa culpabilité, ce dont je ne doutais guère. Même Titus, qui me filait régulièrement des coups de main pour expurger la planète de ses abcès les plus purulents, n’était pas au courant. Avec lui, Maël Robineau, Samuel Girard et Grégoire Lussier, j’avais fondé une petite entreprise de nettoyage spécialisée dans le tri sélectif des ordures. Les heureux élus faisaient l’objet d’un traitement de faveur dont les modalités variaient en fonction de leurs méfaits, mais qui se terminait invariablement par leur mise en pièces et la dispersion de celles-ci dans les égouts de l’histoire de l’humanité. Nous formions une petite équipe très efficace, mais chacun de nous se réservait le droit de chasser en solitaire. C’est très agréable de planifier, monter minutieusement une opération et la voir se dérouler sans accroc, mais il arrive parfois qu’un lièvre surgisse à l’improviste, vous passe entre les jambes, et il serait dommage de le laisser filer. Le côté monstrueux de Passereau, inhabituel et digne des films d’horreur les plus réfrigérants, stimulait mon instinct de limier, et j’avais décidé qu’il serait pour moi et moi seul.

J’étais certain, à 99,99 pour 100, qu’il était le Brain Catcher.

Par chance il travaillait à l’international, avec comme destinations principales l’Allemagne, la Belgique et le Luxembourg, ce qui fait que lorsqu’il partait on pouvait être trois ou quatre jours, voire davantage, sans voir sa sale gueule de troll des cavernes dans les parages. Figurez-vous, preuve que la nature fait vraiment n’importe quoi, qu’il avait quand même réussi à trouver une femme et lui faire un enfant, une fille en l’occurrence. Naturellement, la femme en question avait rapidement pris conscience de l’erreur monumentale qu’elle avait faite en s’accouplant avec ce dégénéré, et s’était empressée de prendre la fuite avec sa fille sous le bras. C’était une chose dont je ne pouvais que me féliciter, d’abord pour elles parce qu’elles avaient sans doute échappé au pire, ensuite pour moi car le fait qu’il vive seul facilitait grandement mon travail.

J’ai profité de son absence pour aller faire un tour chez lui, de nuit bien sûr, parce que la nuit tous les chats sont gris et que je suis moi-même un gros matou affichant un certain penchant pour la bouteille. La preuve, c’est que juste avant cette visite je m’étais envoyé les trois quarts d’une bouteille de Gevrey-Chambertin signé Clovis Cuvillier \& Fils à Morey, lequel divin nectar m’avait placé dans l’exact état d’insouciance recherché pour mener à bien cette mission, à risque modéré il faut bien le dire. Inutile de préciser que bien au chaud dans le fond de ma poche, mon fidèle Manu était prêt à pointer le bout de son canon et cracher le plomb à la moindre alerte.

J’ai utilisé une clé passe-partout (achetée pas cher sur Internet, j’en avais plusieurs qui me permettaient de venir à bout de la plupart des serrures, hormis peut-être quelques rares modèles de fabrication tchécoslovaque datant du rideau de fer et de la cession de la Ruthénie subcarpatique à l’URSS en 45) pour m’introduire sans dommage dans le nid d’amour de mon routier préféré. Le Brain Catcher, si c’était bien lui, ne devait pas savoir que quelqu’un avait violé son intimité pendant son absence. Il devait continuer à mener sa petite vie bien tranquille sans se douter de rien, jusqu’au jour où je lui tomberais dessus sans crier gare et lui ferais regretter le jour où son père et sa mère s’étaient rencontrés avant de forniquer pour donner naissance à une des pires abominations jamais conçues par des voies naturelles. L’idée peut sans doute choquer quelques intellectuels de gauche, universitaires affiliés à la LCR de Gérard Filoche et autres membres du Nouveau Parti Anticapitaliste farouchement attachés au respect des droits de l’homme et autres valeurs aussi obsolètes que la liberté, l’égalité et la fraternité, toutes choses avec lesquelles la populace se récure copieusement le fondement, mais il semble évident que certaines personnes devraient être interdites de reproduction (et dans le cas où ils seraient arrêtés après avoir mis bas dans quelque endroit tenu secret, éliminés sans autre forme de procès en même temps que leur immonde progéniture).

La bicoque, située dans un des quartiers les plus pourris de la ville (il suffisait de prononcer son nom pour que les gens se mettent à claquer des dents et transpirer à grosses gouttes), dégageait la même odeur de tripes à la mode de Caen que son propriétaire. Notre homme n’avait pas de chien, ou plutôt n’en avait plus, car même les animaux ne supportaient pas sa présence et se carapataient à la première occasion. Peu de temps après le départ de sa femme et sa fille, Pancho, un Ratier de Majorque dans la force de l’âge, d’humeur habituellement enjouée, avait disparu à son tour. Outre des croquettes Waterflox, croquettes de luxe pour chien gâté, je cherchais une planche à découper spéciale être humain avec les outils appropriés, des congélateurs pleins à craquer, plus des murs entiers d’étagères surchargées de conserves comme on en trouve chez les péquenauds qui passent leur temps à cueillir des haricots, des patates et des citrouilles, zigouiller des lapins, des poules et des cochons, quand ils ne sont pas dans la forêt en train de défourailler sur tout ce qui bouge au calibre 12.

Je n’ai pas trouvé de croquettes Waterflox, ni de congélos pleins à craquer (juste un réfrigérateur rempli de canettes de bière bas de gamme et de bouffe avariée), jusqu’au moment où je me suis retrouvé face à une minuscule porte planquée derrière une tenture bouffée aux mites à l’effigie de Ganesh. Figurez-vous que le dieu de la sagesse, seigneur des catégories, principe du nombre et protecteur des moissons, était en train de danser ou sautiller sur une espèce de pouf en agitant sa trompe et ses deux paires de bras potelés autour de son corps grassouillet, ses principaux attributs dans les mains. Au moins, en Inde, les dieux n’ont pas peur du ridicule, ce qui vaut d’ailleurs pour les Indiens en général, qui adorent les vaches maigres, les bus bondés avec des gens sur le toit, les couleurs criardes, les comédies musicales ineptes et se baigner tout habillés dans les eaux croupies du Gange.

Cette porte, qui menaçait de tomber en poussière si on s’avisait de la brusquer, n’était autre que celle de la cave, endroit qui dégageait une forte odeur de moisi, fosse septique et rat crevé, cocktail qui ne donnait pas franchement envie de s’aventurer plus loin. Surtout que pour descendre, il fallait emprunter un escalier en bois dont chaque marche semblait sur le point de s’effondrer quand on posait le pied dessus. Pour couronner le tout, un interrupteur, assez vieux pour figurer en tête de gondole dans le musée international des interrupteurs, était relié à une ampoule crasseuse pendant nue au bout d’un fil décharné, laquelle émettait juste assez de lumière pour ne pas se marcher dessus dans la pénombre.

J’ai dû mettre en marche la torche de mon téléphone pour me frayer un chemin dans ce trou à rats. Après avoir fureté dans tous les coins et recoins, chose qui ne m’a permis de dénicher que quelques insectes rampants et autres araignées manifestement peu satisfaits de se voir dérangés dans leurs activités sournoises, je me suis vu dans l’obligation de conclure à l’improductivité de ma démarche. Il n’y avait rien dans cette cave susceptible d’incriminer mon principal suspect, ce qui le faisait passer de son statut initial à individu lambda, certes particulièrement moche, vulgaire et inquiétant, mais apparemment indemne des faits ignobles que je brûlais d’envie de coller sur son dos d’une largeur impressionnante. Rien, dans ce que j’avais vu ici, ne pouvait laisser penser que Dylan Passereau était le Brain Catcher, même s’il avait le profil idéal pour ce genre de cas de figure.

Je suis rentré chez moi la queue entre les jambes, au volant de ma Renault Kangoo de dix ans d’âge, ce qui je vous le concède n’est pas spécialement une bagnole de super flic, avec la ferme intention de pousser plus avant mes investigations sur les étranges cas des pères Vidal et Beaubois. Ce n’était pas tous les jours qu’on retrouvait des hommes d’Église en pièces détachées dans des bocaux (qu’on aurait pu étiqueter, par exemple, «~Les Conserves du Vatican~», et distribuer gratuitement aux fidèles lors des grandes occasions).

J’ai retrouvé ma bouteille de Gevrey aux trois quarts vide qui m’attendait bien sagement sur la table de la cuisine, et en passant près d’elle je l’ai clairement entendue me chuchoter dans le creux de l’oreille, d’une voix déchirante qui n’aurait pu laisser personne indifférent, moi encore moins compte tenu de la passion que je nourrissais pour les vins de Bourgogne : finis-moi avant d’aller te coucher. Cette proposition tombait d’autant mieux que j’avais bien besoin d’un petit remontant. J’ai donc sifflé le verre de vin en question et, histoire de digérer un peu mieux la déconvenue que je venais d’encaisser, me suis servi un petit Cognac dans la foulée, que j’ai siroté religieusement, le cul profondément enfoncé dans mon fauteuil club préféré, un Short Story au bec, en écoutant la Huitième de Chostakovitch\nf{Dmitri Chostakovitch (1906--1975), compositeur soviétique né à Saint-Pétersbourg. Sa Huitième Symphonie en ut mineur op. 65 (1943), créée en pleine Seconde Guerre mondiale, est une œuvre sombre et d’une tension extrême, souvent interprétée comme une méditation sur l’horreur du conflit. Elle fut un temps bannie par le régime stalinien. \source{fr.wikipedia.org/wiki/Dmitri\_Chostakovitch}}, œuvre majeure du maître de Saint-Pétersbourg dont la tension dramatique me paraissait idéalement adaptée aux turbulences qui m’agitaient intérieurement.

La question qui revenait sans cesse à mon esprit, ainsi qu’une hideuse ritournelle aux intonations grinçantes, n’était autre que celle-ci : pourquoi s’en prendre à des prêtres ?

Je doutais fort, même si je dispose d’un sens de l’humour à toute épreuve, qu’il s’agisse d’un exercice de style ou une simple déclaration d’anticléricalisme primaire. On sait aujourd’hui que les prêtres n’ont pas toujours le comportement irréprochable qu’on serait en droit d’attendre de leur sacerdoce. Nombre d’entre eux semblent avoir le plus grand mal à résister aux attraits de la chair. Non seulement beaucoup n’y résistent pas, mais il semblerait que certains s’y adonnent avec une coupable frénésie, fort peu en accord avec la rigueur morale qu’on attend d’eux. J’en veux pour preuve les récentes autant que stupéfiantes révélations concernant l’abbé Pierre\nf{Henri Grouès, dit l’abbé Pierre (1912--2007), prêtre catholique français, fondateur du mouvement Emmaüs (1949) et auteur de l’appel de l’hiver 1954 en faveur des sans-abri. À partir de 2024, de nombreuses victimes ont témoigné de violences sexuelles qu’il leur aurait infligées, des faits couverts pendant des décennies par son entourage et par l’Église. \source{fr.wikipedia.org/wiki/Abb\%C3\%A9\_Pierre}}, héros de la cause chrétienne que l’on croyait pourtant au dessus de tout soupçon. Même si la vie m’a appris qu’on ne peut raisonnablement se fier à personne, j’en suis resté comme deux ronds de flan. Derrière son sourire bonasse entouré de poils drus, Castor méditatif (son totem chez les scouts, assez bien trouvé quand on sait à quel point le castor travaille avec sa queue) dissimulait les noires pensées d’un des pires prédateurs sexuels de la chrétienté. Sous sa soutane frétillait un goupillon assez peu orthodoxe, une anguille sournoise qui n’attendait qu’une occasion de se glisser entre les cuisses des filles. À ses nombreuses décorations, Légion d’honneur, Croix de guerre et autres, je serais d’avis d’ajouter celle de grand officier de l’Ordre national des Violeurs patentés et autres Abuseurs sexuels compulsifs. Cela dit, je ne vais pas m’acharner sur lui ni sur sa hiérarchie qui, désireuse de se focaliser uniquement sur la bonté de son âme, a couvert ses agissements pendant des décennies. Finalement, et c’est tout à son honneur, sa lutte acharnée contre la misère incluait aussi la misère sexuelle, celle des prêtres en particulier. De même que pour nombre de nos personnalités les plus en vue, qu’il s’agisse de nos plus grands acteurs, hommes de science ou dirigeants, aujourd’hui poursuivies par des scandales à répétition dont on peine à connaître l’ampleur et la nature exactes, ne faut-il pas tolérer quelques petits égards de conduite eu égard aux services inestimables rendus à la Nation ? Le célibat des prêtres pose un problème de fond, nous le savons tous, car on ne voit pas très bien pourquoi il faudrait s’abstenir de se tripoter la nouille et tirer sa crampe sous prétexte de servir Dieu. D’un autre côté, il semble assez difficile de ne pas voir une certaine équivoque dans le fait de prêcher la retenue des sens d’une part, sinon l’abstinence pure et simple, et d’autre part abuser de ses prérogatives pour satisfaire outrageusement ses plus bas instincts. Il m’est souvent arrivé, en croisant quelque moine courbé sous le poids de sa charge méditative et soumis plus que tout autre au vœu de chasteté, de l’imaginer en train de s’adonner sans retenue aux plaisirs solitaires dans le secret de sa cellule, quitte à se flageller ensuite jusqu’au sang pour expier sa faute. Il est bien cruel, quand on sait à quel point l’être humain ne vit que pour la recherche effrénée de sa satisfaction immédiate et personnelle, d’exiger de l’homme d’Église qu’il renonce à toute forme de jouissance, sinon celle de l’extase mystique dont la nature pourrait d’ailleurs elle-même prêter à confusion.

À la lumière de ces réflexions, il devenait facile d’imaginer les raisons qui pouvaient pousser quelqu’un à s’en prendre à des prêtres. J’admets que le fait de les couper en morceaux et leur farcir la tête de croquettes pour chien laissait planer de sérieux doutes sur la santé mentale de l’auteur des faits. Pour des raisons évidentes, notamment le fait que les femmes ne fassent plus la cuisine, aient perdu toute expérience dans le domaine de la découpe des viandes et le farcissage des légumes (ne parlons pas des boîtes crâniennes), il y avait fort peu de chances que ces crimes atroces soient l’œuvre de l’une d’entre elles, même la plus repoussante des camionneuses, la plus épouvantable des gouines, la plus monstrueuse des brouteuses de minous. Je penchais davantage en faveur de la vengeance d’un individu de sexe mâle, homosexuel pourquoi pas, en tout cas un homme dont lui-même ou un être cher, un fils par exemple (qui aurait mis fin à ses jours ou sombré dans la démence, laissant son géniteur dans le désespoir le plus extrême), aurait été victime des agissements inqualifiables et répétés d’un homme d’Église apparemment au-dessus de tout soupçon.

Pour conforter cette hypothèse, il importait donc de se livrer sans plus attendre à une expertise approfondie de la vie privée des ecclésiastiques concernés.

C’est ce que j’ai commencé à faire dès le lendemain, en sollicitant une entrevue avec la plus haute autorité du diocèse, en la personne éminemment respectable et en tout point admirable de monseigneur Mathéo Riqueti, cardinal de son état.

Mathéo Riqueti était un homme de belle allure, au regard aussi pénétrant que la bite de l’abbé Pierre dans le cul d’une novice terrassée par son magnétisme et sa ferveur religieuse, habitué à ce que les gens se prosternent à ses pieds et boivent ses paroles avec avidité. La première chose que j’ai pensé, en le voyant, c’est que si ça se trouve il passait le plus clair de son temps à enculer des enfants de chœur dans les coins sombres des églises (qui n’en manquent pas, il faut bien le dire). Okay, je retire ce que j’ai dit, honte à moi, je cours de ce pas me suspendre à des crochets de boucher et me rouler dans les braises pour expier mes vilaines pensées. Satan, sors de ce corps !

Il m’a reçu dans un petit salon sympathique, avec fauteuils aussi absorbants que du PQ triple épaisseur, tentures de velours rouge et meubles en bois précieux servant de supports à des bibelots raffinés.

Je lui ai dit que j’enquêtais sur les meurtres des pères Vidal et Beaubois, que j’avais des raisons de penser que ce n’était pas par hasard qu’on les avait pris pour cible, et que j’aurais aimé en savoir un peu plus sur leur compte. Leur vie privée, notamment.

Ses épaules se sont quelque peu ratatinées et sa tête est rentrée dans son cou. Un frisson d’angoisse imperceptible (pour tout œil aussi diaboliquement avisé que le mien) a parcouru son visage poupin surmonté d’une élégante collerette de cheveux grisonnants.

\textsc{Moi} : Waterflox, ça vous parle ?

\textsc{Lui} : Non. C’est quoi ?

\textsc{Moi} : Une marque de croquettes pour chien.

\textsc{Lui} : Quel rapport avec les pères Vidal et Beaubois ?

\textsc{Moi}, sortant un Short Story de la poche de ma veste : Ça vous dérange si je fume ?

\textsc{Lui} : Oui, je suis vraiment désolé. Donc ?

\textsc{Moi}, reniflant le cigare avec envie avant de le remballer : Donc quoi ?

\textsc{Lui} : Quel rapport avec les pères Vidal et Beaubois ?

\textsc{Moi} : On leur a farci le crâne avec des croquettes pour chien de la marque Waterflox. Des croquettes de luxe à l’agneau et au bœuf. Vous n’êtes pas censé le savoir, on n’en a pas parlé dans les journaux.

\textsc{Lui} : Dans ce cas, je ne vois pas pourquoi vous me posez la question.

\textsc{Moi} : Déformation professionnelle, histoire de vérifier que l’info n’a pas fuité.

\textsc{Lui} : L’œuvre d’un dément, à n’en pas douter.

\textsc{Moi} : C’est aussi mon avis.

\textsc{Lui} : Ou du diable en personne !

\textsc{Moi} : Si c’est le cas, on va avoir du mal à l’attraper.

\textsc{Lui} : Quoi qu’il en soit, je ne vois pas très bien ce que je peux faire pour vous, à part vous proposer un petit verre de vino santo. Celui-ci vient du domaine Badia à Coltibuono\nf{Badia a Coltibuono (abbaye de la «~bonne récolte~»), domaine viticole situé en Toscane dans le Chianti Classico, au nord de Gaiole in Chianti. Fondé au XI\textsuperscript{e} siècle par les moines bénédictins de l’ordre de Vallombrosa, il produit des vins rouges et un \textit{vin santo} réputé.\source{fr.wikipedia.org/wiki/Badia\_a\_Coltibuono}}, fondé il y a plus de mille ans par les moines bénédictins de Vallombrosa\nf{L’ordre de Vallombrosa, branche réformatrice des Bénédictins, fut fondé vers 1038 par saint Jean Gualbert près de Florence. L’abbaye-mère de Vallombrosa, dans les Apennins toscans, est encore habitée par des moines. L’ordre, dit «~silvestrin~» ou «~vallombrosain~», insiste sur la pénitence et la prière contemplative. \source{fr.wikipedia.org/wiki/Ordre\_de\_Vallombreuse}}, au nord de Gaiole. Il mûrit six ans dans des petits fûts de châtaignier avant d’être commercialisé.

\textsc{Moi} : Je ne dis pas non.

Il a hurlé : Dardariel, s’il vous plaît !

Sorti de nulle part, un loufiat en soutane s’est approché en rampant plus bas que terre : Monseigneur ?

\textsc{Monseigneur} : Apportez-nous deux verres de vino santo, s’il vous plaît.

\textsc{Le loufiat} : Tout de suite, Monseigneur.

Quelques instants plus tard, j’avais en main un petit verre finement ciselé rempli du précieux nectar aux reflets cuivrés, apporté par un Dardariel qui s’était évaporé avant que j’aie eu le temps de cligner des yeux.

\textsc{Moi}, reniflant le breuvage avec délectation : À votre avis, qu’est-ce qui peut pousser un détraqué à s’attaquer à des hommes d’Église ?

\textsc{Lui} : On a coutume de dire que les voies du Seigneur sont impénétrables. J’ajouterai que celles du démon le sont tout autant.

\textsc{Moi} : Vraiment ?

\textsc{Lui} : Mais oui.

\textsc{Moi} : Écoutez, cher monsieur, je ne vais pas y aller par quatre chemins. Je sais que l’Église, comme l’Armée, a le goût du secret. On lave son linge sale en famille, avec ses propres règles, on se croit au-dessus des lois. Vous n’avez pas le fond de culotte très propre, loin s’en faut, mais vous faites tout pour empêcher les mauvaises odeurs d’arriver jusqu’aux narines du profane. Je ne suis pas spécialement porté sur la religion, vous l’aurez compris, mais deux curés viennent de se faire étriper avec une violence rare. On les a découpés en morceaux et on leur a farci le crâne avec des croquettes pour chien Waterflox. Vous n’avez pas de chien, Monseigneur ?

\textsc{Lui}, sirotant tranquillement son vino santo : Les chiens demandent beaucoup d’attention, inspecteur, et mes obligations envers Dieu et les Hommes ne me laissent aucun répit. C’est tout juste si j’ai le temps de faire une petite partie de golf de temps à autre.

\textsc{Moi} : Alors écoutez bien ceci : les croquettes pour chien Waterflox bénéficient de recettes élaborées par les meilleurs cuisiniers du monde. Il s’agit ici d’agneau à la sauce hoisin accompagné de riz délicatement parfumé au jasmin, de nature à flatter les truffes les plus exigeantes, une recette réservée aux chiens de race tout spécialement concoctée par Anada Sintawichai, un des meilleurs chefs de Bangkok. Le docteur Zaahid Shirani, légiste à la PJ qui a examiné les corps avec la plus grande attention et que vous connaissez peut-être, étant donné que lui-même joue au golf avec assiduité, est formel : ces putains de croquettes pour chien sont un produit de luxe réservé aux bourses les plus rebondies, et le travail pratiqué au niveau des boîtes crâniennes des victimes relève de la plus haute technicité.

\textsc{Lui} : Donc, si je comprends bien, le tueur a des notions de médecine et les moyens de ses ambitions. Pourquoi pas un chirurgien ?

\textsc{Moi} : Oui, comme pour Jack l’Éventreur\nf{Jack l’Éventreur (\textit{Jack the Ripper}), tueur en série non identifié qui assassina au moins cinq prostituées dans le quartier de Whitechapel, à Londres, entre août et novembre 1888. Son identité n’a jamais été établie avec certitude. L’hypothèse d’un meurtrier aux connaissances chirurgicales fut avancée dès l’époque.\source{fr.wikipedia.org/wiki/Jack\_l\%27\%C3\%89ventreur}}. J’y ai pensé, figurez-vous. Mais on n’est pas à Whitechapel et ce n’est pas pour ça que je suis venu vous voir. Ce que je veux savoir, c’est si les pères Vidal et Beaubois ont fait l’objet de signalements concernant des faits à caractère sexuel.

En entendant ces mots, le visage de monseigneur Mathéo Riqueti, jusqu’ici d’une parfaite impassibilité, est soudain retombé tel un vieux soufflé dans le fond de sa gamelle.

\textsc{Lui} : Vous voulez parler de…

\textsc{Moi} : D’attouchements sur mineurs, par exemple. Je pense plus particulièrement à des sodomies de jeunes garçons ou des fillettes contraintes de pratiquer des fellations pendant les cours de catéchisme.

\textsc{Lui} : Vous êtes ignoble !

\textsc{Moi} : Non, mais je fais un métier difficile dans un monde qui l’est.

\textsc{Lui} : Je dispose effectivement d’un certain nombre d’informations concernant les pères Vidal et Beaubois. Je suis prêt à vous en faire part, à titre personnel, pour l’avancement de votre enquête, mais je vous demanderai de garder ça pour vous. Vous savez que notre traverse depuis quelque temps une période difficile, accumulant scandale sur scandale, chose qui pourrait, à terme, conduire à la complète désertion de nos locaux. Si Dieu est parfait, irréprochable, rempli d’amour et de bonté, il arrive que ses émissaires le soient un peu moins. Ce ne sont que des hommes, après tout, avec leurs qualités et leurs défauts, des défauts contre lesquels ils luttent avec le concours de la foi et toute la force de leurs convictions. Mais il arrive parfois que leur volonté faiblisse face aux appels répétés de la chair, que la tentation soit trop forte. Le père Vidal est~-- était~-- un homme remarquable, dévoué corps et âme au salut de ses ouailles, mais je me suis laissé dire qu’il lui arrivait parfois de céder à la tentation. Plus d’une fois, les bénévoles chargés de l’entretien de son église l’ont surpris en train de se livrer à des actes que la morale réprouve. Les rares plaintes déposées ont été classées sans suite, rayées des fichiers sur l’insistance des plus hautes autorités religieuses. Moi-même, eu égard à ses états de service, suis parfois intervenu en sa faveur.

\textsc{Moi} : Et le père Beaubois ?

\textsc{Lui} : Une ordure de la pire espèce, mais avec des relations très haut placées qui lui ont toujours permis de s’en sortir sans une égratignure.

\textsc{Moi} : Il aimait les petits culs ?

\textsc{Lui}, embarrassé : La formule n’est pas très élégante, c’est le moins qu’on puisse dire.

\textsc{Moi} : Parce que vous trouvez élégant de tringler des gamins ?

\textsc{Lui} : Non, pas davantage. D’après ce que je sais, le père Beaubois n’avait pas de préférence marquée en la matière. C’était un opportuniste qui faisait feu de tout bois, sans distinction d’aucune sorte du moment que la victime n’avait pas encore atteint sa majorité.

\textsc{Moi} : Je vois. Alors que le père Vidal…

\textsc{Lui} : À ma connaissance, le père Vidal ne s’intéressait qu’aux petits garçons.

\textsc{Moi} : De toute façon, il ne fait pour moi guère de doute que l’auteur des crimes est un individu de sexe mâle. Il a dû avoir affaire aux pères Vidal et Beaubois et décidé de passer à l’acte après avoir longuement ruminé sa vengeance. Vous savez, quand on passe sa jeunesse à se faire enculer, on en garde souvent quelques mauvais souvenirs. Je pense qu’il est grand temps que l’Église commence à payer un peu le laxisme dont elle fait preuve depuis des siècles, je dirais même la lâcheté impardonnable qu’elle a tendance à afficher en toute circonstance. Votre vin est excellent mais je mentirais en disant que m’inspirez la moindre trace de sympathie, Monseigneur. Vous êtes comme tous ces curetons qui cachent sous leur soutane des montagnes de secrets inavouables. Votre voix est douce comme le miel mais votre cœur plus dur et sec que le granit. Allez vous faire foutre !

\textsc{Lui} : Dardariel !

Je ne vous l’ai peut-être pas dit, n’en n’ayant point vu l’intérêt sur le coup, mais Dardariel avait une des pires sales gueules qu’il m’ait été donné de voir au cours d’une existence qui ne m’avait pourtant pas ménagé sur le sujet. Non seulement il était le majordome de Riqueti, son homme à tout faire y compris sans doute se faire enculer le soir au fond des bois en chantant des cantiques, mais il remplissait aussi avec une certaine efficacité les fonctions de chauffeur et garde du corps.

\textsc{Dardariel} : Monseigneur ?

\textsc{Monseigneur} le roi des enculés : Cet odieux personnage vient de m’insulter et me manquer de respect comme rarement quelqu’un s’est permis de le faire. Frappez-le, s’il vous plaît.

Dardariel portait une tenue à mi-chemin entre la robe de bure à capuche et le costume de ninja. Je pense qu’il avait dû faire du catch ou un truc dans le genre avant d’entrer dans les ordres. Ou alors Monseigneur l’avait rencontré dans le Marais et décidé de le prendre (par derrière de préférence) à son service après une rapide formation aux textes sacrés et autres rituels religieux.

\textsc{Dardariel} : Avec joie, Monseigneur.

Je ne sais pas pourquoi, je ne lui avais rien fait, mais Dardariel n’avait pas l’air de me porter dans son cœur. Sans doute que ma gueule de parfait enfoiré de flic sournois ne lui revenait pas.

Dardariel s’est approché et planté devant moi : Levez-vous, s’il vous plaît.

\textsc{Moi} : Non, je suis très bien assis.

\textsc{Lui} : Comme vous voudrez.

Il m’a attrapé par le colbac et projeté à travers la pièce.

C’était la première fois qu’un membre du clergé me faisait subir un tel affront. J’ai senti la haine monter en moi tel un tsunami dans un de ces films-catastrophes de merde que l’industrie du cinéma américain tourne à la chaîne à grand renfort d’acteurs à chier, d’effets 3D et de connerie artificielle du style Chatte-J’ai-pété.

J’ai dit, en ramassant mes morceaux et essayant de les remettre dans le bon ordre : vous savez ce qu’il en coûte de s’en prendre à un fonctionnaire de police dans l’exercice de ses fonctions ?

\textsc{Lui} : Vous auriez dû y penser avant de manquer de respect à Monseigneur.

\textsc{Moi} : Penser à quoi ?

Ses yeux ont fait trois tours dans leurs orbites, et j’ai compris que son cerveau avait un peu de mal à suivre le rythme. Il n’avait certainement pas été recruté pour ses capacités intellectuelles.

Par contre, niveau qualité des articulations, j’ai pu goûter à la dureté de ses poings quand il m’a envoyé une droite en pleine mâchoire. Pas très appuyée, la droite, mais tout de même assez pour me clouer le bec pendant quelques dixièmes de seconde (ce qui n’est pas chose facile, vous l’aurez sans doute remarqué).

Quand je l’ai vu rappliquer dans ma direction, alors que j’en étais encore à me remettre les dents en place dans leurs alvéoles, je me suis dit qu’il était temps de passer aux choses sérieuses. Rapide comme la foudre, ma main a filé dans ma poche revolver et sorti Manu qui bouillait d’impatience de se joindre à la fête. Il avait des fourmillements dans la détente.

Manu ratait rarement ses entrées, et je crois pouvoir dire que celle-ci a été appréciée à sa juste valeur. Dardariel d’abord, qui fondait sur moi avec la rapidité du faucon, a stoppé net sa progression. Il faut dire que Manu le regardait droit dans les burnes, de quoi faire réfléchir même le plus abstinent des moines. Monseigneur ensuite, qui s’est ratatiné dans le fond de son fauteuil comme s’il avait vu le diable en personne.

\textsc{Moi} : Franchement, messieurs, vous commencez à me taper sur le système, au propre comme au figuré. Je vais essayer de prendre sur moi et faire comme s’il ne s’était rien passé. Dardariel, mon petit Dardariel, viens donc voir par ici.

Dardariel s’est approché, à reculons, ses petits poings tellement serrés qu’on pouvait entendre les jointures craquer à des lieues à la ronde. Il avait beau être con comme un balai, il s’était rendu compte que mon invitation n’avait rien de cordial.

Dès qu’il a été à portée de main, je lui ai balancé un bon coup de crosse en pleine gueule, histoire de lui faire comprendre, ainsi qu’à son connard d’employeur, que les forces de l’ordre gardaient le contrôle de la situation quoi qu’il advienne, et que ce n’était certainement pas une bande de curetons mal ensoutanés qui allaient foutre le bordel dans ma circonscription.

Je pense que c’est un signe : quand les hommes d’Église commencent à se conduire comme la pire des racailles de banlieue, alors il est grand temps de changer d’air. Mais pas question pour autant d’aller me faire chier sur Mars avec Elon Musk, Trump, Poutine, Bolsonaro, Meloni et toutes les faces de pet dictatoriales et nazifiantes de la planète. Il va falloir trouver une alternative. Laquelle, je ne sais pas encore, mais j’y travaille.

\textsc{Moi} : Écoutez, Monseigneur (mot prononcé avec tout le mépris dont j’étais capable), il va me falloir le maximum de détail concernant les plaintes déposées à l’encontre des pères Vidal et Beaubois. Qui, quand et pourquoi. Si vous me mangez gentiment dans la main, sans faire de vaguelettes, j’essayerai de passer l’éponge sur vos agissements et le fait que vous êtes une grosse pourriture qui pue et me retourne le cœur à chaque que je pose les yeux dessus. Dans le cas contraire, je vous promets une campagne de pub dont vous aurez du mal à vous remettre.

Apparemment convaincu par la solidité de mes arguments, Riqueti avait fini par lâcher des infos qui allaient m’être d’une grande utilité pour la suite de mon enquête.

Sauf que, comme tous les enfoirés de son espèce, drapés ou non des oripeaux de la religion, non seulement il n’avait aucune confiance en son prochain, ce que je pouvais difficilement lui reprocher, mais il avait la rancune tenace.

Ce même jour, en fin d’après-midi, on avait fait une descente dans la cité des Alisiers pour remuer un peu la merde. On s’était fait repérer dès notre arrivée, méchamment caillasser, et on avait préféré battre en retraite plutôt que de risquer le mauvais coup. En désespoir de cause, on s’était rabattu sur un pédophile dont le dossier traînait au-dessus de la pile depuis quelques semaines. On avait gentiment sonné à sa porte, comme des gens civilisés, et ce gros tas de merde était venu ouvrir en slip et débardeur crasseux, des morceaux de chips plein la barbe. Il était en train de s’empiffrer devant la télé. On lui a dit qu’on savait des choses sur lui, il a dit qu’il ne voyait pas de quoi on voulait parler, qu’il avait effectivement eu quelques petits problèmes par le passé, mais que ça lui avait servi de leçon et qu’il avait définitivement mis un terme à sa carrière d’exhibitionniste à la sortie des écoles. L’ennui, c’était qu’on avait des informations qui semblaient prouver le contraire. Son ordi était sous surveillance depuis un bout de temps, et ses connexions à répétition sur des sites pédocriminels les plus dégueulasses du web ne plaidaient pas en sa faveur. Titus Beaugendre, mon homme de confiance, compagnon d’infortune, toujours partant pour la grande aventure, carrossé comme un ours et monté comme un taureau (on aurait dit qu’il pissait avec un tuyau d’arrosage), s’était fait tripoter dans les vestiaires de son club de foot quand il avait entre huit et dix ans. Il avait arrêté le foot mais l’expérience lui avait laissé un très mauvais souvenir, de sorte que la vision d’un pédophile, même de très loin, de dos et dans le brouillard, lui causait des migraines insupportables. Quand l’autre a commencé à faire le malin, il n’a pu s’empêcher de lui coller quelques crochets dans le gras du bide pour lui faire fermer sa gueule. Le représentant de l’ordre passe la majeure partie de son temps sous tension, il faut bien qu’il décompresse quand l’occasion se présente. On a saisi le matos et embarqué le pédophile, histoire de le cuisiner un peu à tête reposée.

Le soir même, après une journée de boulot bien remplie comme vous avez pu le constater, je rentrais tranquillement chez moi, avec la satisfaction du devoir accompli, quand je me suis rendu compte que quelque chose clochait dans le périmètre. Dans le rétro de ma Kangoo, plus exactement. Sauf à être devenu complètement parano, ce qui n’était pas impossible vu les cadences infernales auxquelles nous étions soumis, une voiture me suivait. Une Alfa Romeo Giulietta noire aux vitres teintées, le genre de caisse pilotée neuf fois sur dix par une tête à claques qui se prend pour Fangio et finit encastré dans le premier platane qui passe à sa portée. Nous, les vrais flics de terrain, disposons d’un sixième sens qui nous permet de savoir instantanément si quelqu’un nous suit ou pas. Cela dit, conscience professionnelle oblige, j’ai fait trois fois le tour du quartier pour m’en assurer. Il me suivait, ou alors quelqu’un le suivait lui aussi et il habitait le même quartier que moi. Mais la plupart des flics, vous le savez comme moi, ont toutes les peines du monde à croire aux coïncidences. Je ne faisais pas exception à la règle.

Je me suis garé près de chez moi, suis descendu tranquillement de ma voiture, comme si de rien n’était, et me suis dirigé vers l’entrée. Comme je m’y attendais, l’Alfa m’a dépassé au ralenti pour aller se garer à quelques mètres de là.

Je résidais au sixième étage d’un immeuble qui, s’il ne datait pas exactement d’hier, n’en présentait pas moins toutes les garanties nécessaires en termes de confort et d’habitabilité. Il y avait l’eau courante, le gaz, l’électricité et du double vitrage aux fenêtres. Il y avait même un ascenseur dans lequel on pouvait loger deux ou trois personnes (pas trop grosses) sans risquer la chute libre. Comme il tombait régulièrement en rade, produisant des bruits atroces qui donnaient tout sauf envie de monter dedans, je pouvais garder la forme en effectuant des allers et retours dans les escaliers. Ce n’était pas le cas de ma voisine d’en face, une vieille bique sèche comme un coup de trique avec des yeux de fouine, une langue de pute et des touffes de cheveux irrégulièrement plantées sur le crâne, qui se retrouvait en grande difficulté à chaque fois que ce foutu ascenseur déclarait forfait. Ses enfants venaient la voir une fois tous les trente-six du mois et ses petits enfants ne rappliquaient que pour lui taxer du fric (qu’elle ne manquait pas de leur donner, sachant pertinemment qu’ils ne viendraient plus sans ça). Une fois, je l’avais croisée à mi-chemin dans l’escalier, à l’agonie avec sa baguette sous le bras et ses poireaux qui dépassaient du cabas. Au lieu de l’achever, ce qui aurait été un acte de charité chrétienne de haute valeur morale, j’avais commis l’erreur de lui proposer mon aide. Depuis, elle surveillait mes allées et venues et me mettait à contribution dès qu’elle arrivait à me mettre le grappin dessus. Quand je n’étais pas de corvée de courses, ou autre, elle m’attrapait par la manche et me tirait chez elle pour avaler un verre de porto. Le porto, c'est sucré, pas trop alcoolisé, les vieux adorent ça. Elle avait aussi trois chats à peu près aussi vieux qu’elle qui chiaient et pissaient partout. L’un d’eux, un rouquin vaguement tigré avec des oreilles bouffées aux mites, s’échappait dès qu’il pouvait pour venir chier sur le paillasson devant ma porte. D’ailleurs, en règle générale, il s’échappait dès qu’il pouvait pour faire régner la terreur dans le quartier. Cette sale bête s’appelait Korax. Un nom de dragon, de créature maléfique, je me demandais bien où la vieille était allée chercher ça. Je m’en plaignais régulièrement à l’intéressée qui m’offrait un verre de porto pour se faire pardonner, une piquette proche du vinaigre qui me filait des aigreurs d’estomac pour le restant de la journée. Je pense qu’on aurait pu décimer une ville entière avec son porto. Peut-être qu’elle le fabriquait elle-même avec du sang de cochon et de l’acide sulfurique, mais il était impossible d’imaginer qu’elle telle horreur soit en vente libre dans le pays des droits de l’homme, l’art de vivre et la gastronomie. Je commençais à me demander si la vieille, sous ses allures de grand-mère inoffensive, n’était pas en fait une redoutable sorcière ayant survécu aux bûchers de l’Inquisition. Korax me détestait. Il y avait des tas d’endroits où il aurait pu chier et pisser sans faire chier le monde, mais il tenait absolument à le faire devant ma porte. Je voyais qu’il me détestait quand il me regardait fixement avec ses grands yeux verts, avec un air de défi, en remuant nerveusement la queue et poussant des grognements rauques comme s’il allait se jeter sur moi. Il avait au moins trois siècles et je me disais qu’il allait bien finir par crever. Mais non, son aversion pour moi le maintenait solidement accroché à la vie, lui donnait la force de continuer à se battre pour me pourrir la mienne. Quels que soient ses sentiments à mon égard, je dois admettre que Korax n’était pas un chat comme les autres. Par exemple, croyez-le ou non, il avait appris à prendre l’ascenseur. Il savait sauter sur le bouton pour l’appeler, et une fois à l’intérieur, sauter à nouveau sur le bouton pour le faire aller et venir. Il lui arrivait aussi, de plus en plus souvent (il commençait à se faire vieux), de se faufiler entre les jambes des gens pour profiter du voyage. Quand je dis les jambes des gens, je pense aux miennes en particulier, entre lesquelles il adorait se faufiler, pas tellement parce que mes jambes sont des jambes entre lesquelles il est plus agréable de se faufiler que d’autres, mais parce qu’il savait que la manœuvre m’exaspérait et que sa présence m’horripilait. À tel point que des fois, quand je le voyais roder dans les parages, prêt à se jeter dans mes pattes pour profiter de l’ascenseur en faisant semblant de ronronner, je préférais me farcir les six étages à pinces. Mais au lieu de lâcher l’affaire, il se les farcissait lui aussi, en se frottant dans mes jambes pour essayer de me faire tomber. Contrairement au chien, animal grégaire au tempérament servile, le chat n’aime pas les gens, qu’il tient pour responsables d’avoir fait de lui une carpette à sa mémère, tout juste bon à faire ses pattes sur un coussin en ronronnant bêtement. Dans animal domestique, il y a animal et domestique. Le chien est plus domestique qu’animal, tendance larbin dévoué corps et âme à son maître, alors que le chat est un animal qui non seulement déteste les gens, mais se déteste lui-même pour s’être laissé avilir par une espèce qu’il estime, sans doute à juste titre, très inférieure à la sienne. Jamais il ne nous pardonnera ce que nous avons fait de lui, et jamais il ne se pardonnera de nous avoir laissé faire. Reste que dans son esprit, c’est lui a domestiqué l’homme et pas le contraire. À ce titre, il a tous les droits et droit à toutes les déférences. Caractériel et sournois de nature, veule et manipulateur, autoritaire, il faut être d’une naïveté sans bornes pour lui accorder la moindre confiance. C’est d’ailleurs, n’en doutons pas, la principale raison de la fascination qu’il exerce sur les gens, plus ou moins conscients d’avoir entre les mains une grenade dégoupillée qui peut leur péter à la gueule à tout instant. Quand un chat vous regarde de travers en plissant les yeux, cela signifie grosso modo : quel dommage que je ne fasse pas trente ou quarante kilos de plus, car j’aurais le plus grand plaisir à t’arracher la tête et te pisser dans le cou.

Depuis le dernier passage du réparateur, trois semaines plus tôt, l’ascenseur semblait reparti pour une nouvelle vie. Il marchait du feu de Dieu, gravissant les étages à la vitesse d’une fusée et les dévalant à celle d’un skieur de haute montagne, de sorte que c’était tout juste si on avait le temps de monter à bord avant d’être arrivé. Des rumeurs circulaient comme quoi des gens auraient été coupés en deux en essayant de monter à bord. Apparemment, la vioque et son maudit chat avaient échappé au sinistre. Certes, l’engin produisait toujours des bruits suspects et autres turbulences dignes d’un avion de chasse en plein passage du mur du son, mais c’était sa façon d’exprimer mécaniquement sa joie d’avoir retrouvé la pleine et entière jouissance de ses facultés motrices.

Je suis monté dedans (ou plutôt glissé furtivement à l’intérieur, en essayant de ne pas me faire couper en deux) et j’ai appuyé sur le 6. Mais une fois arrivé en haut, au lieu de rentrer chez moi et m’envoyer un ou deux verres de blanc (une bouteille de Chablis Grenouilles m’attendait au frais, vous pensez si j’étais un tantinet pressé de lui rendre hommage, d’autant que la journée avait été particulièrement douce pour la saison et que la soif n’avait cessé de me tirailler) comme j’aurais aimé pouvoir le faire, je me suis planté devant l’ascenseur et j’ai attendu patiemment qu’il redescende, après avoir sorti Manu de sa cachette et vérifié qu’il était chargé à bloc, prêt à cracher son venin à la plus légère sollicitation de mon index droit (lequel, bénéficiant d’une autonomie relative, était sujet à de fréquentes crispations réflexes en période de stress). On n’arrête pas un éléphant avec du 6.35, mais, à bout portant, ça reste une arme tout à fait dissuasive pour le commun des mortels. Comme de juste, Korax, la bête immonde, féline incarnation des forces démoniaques sur terre, digne héritier du chat égyptien au regard fourbe et à la griffe acérée, s’est pointé dans mon dos. Avant même que j’aie pris conscience de sa présence, il était déjà en train de se frotter copieusement dans mes jambes (je n’ai rien dit pour ne pas faire d’esclandre, sachant qu’il était capable de piquer une crise si je tentais de faire obstacle à sa volonté), laissant des touffes de poil roux sur mes bas de pantalon d’une fraîcheur jusqu’ici impeccable. Il pensait peut-être que j’étais sur le point de monter dans l’ascenseur. Plus probablement, il avait reniflé que la situation était tendue et jugé par conséquent le moment idéalement choisi pour venir fourrer son museau dans mes affaires. Je n’ai pas vu d’où il sortait exactement, mais j’étais prêt à parier qu’il venait d’aller lâcher une belle grosse prune bien juteuse sur mon paillasson, petit cadeau de bienvenue dans lequel j’avais mis le pied si souvent que j’avais épuisé mes ressources en matière de calcul mental. Le pire c’était le matin, quand je sortais de chez moi la tronche encore ensuquée de sommeil et les paupières mi-closes. J’avais beau connaître la chanson, il m’arrivait encore, à une fraction de seconde près, de tomber dans le piège.

Je n’ai pas eu à attendre bien longtemps.

Quelqu’un a appelé l’ascenseur, et je ne doutais pas que ce quelqu’un se trouvait au rez-de-chaussée et qu’il s’agissait de l’homme à la Giulietta. En voyant l’ascenseur prendre le large, Korax m’a jeté un regard mauvais et s’est éloigné en trottinant en direction des escaliers, avec cette façon particulière qu’ont les chats de trottiner en levant la queue comme si ça les faisait marrer de vous montrer leur trou du cul. Ce trou du cul fonctionne comme un troisième œil qui vous fixe avec sa pupille marronnasse en ayant l’air de dire : je t’emmerde, connard. Mais comme je vous l’ai dit, Korax savait se servir de l’ascenseur. Il n’avait donc pas besoin de moi pour le prendre, mais adorait m’imposer sa présence de félin sournois et arrogant. Il adorait me faire chier, par exemple gratter ses puces dans ma direction dans l’intention manifeste qu’elles atterrissent sur moi et me vident de mon sang. Mais un jour ou l’autre, à force de me pousser à bout, de jouer avec mes nerfs, il aurait affaire à Manu, son œil de cyclope, son petit corps en acier trempé, froid et glacé comme la mort (glacé aurait sans doute suffi, mais deux précautions valent mieux qu’une), et surtout sa détente si hypersensible qu’il suffisait qu’un moucheron passe à proximité pour la déclencher (d’ailleurs moi-même, quand je le trimballais dans ma poche, j’évitais soigneusement tout mouvement brusque pour éviter de prendre une balle perdue).

Comme prévu, l’ascenseur est monté jusqu’au sixième.

Ça aurait aussi bien pu être la mère Ouvrard, Maria de son prénom, ma voisine d’en face et diabolique maîtresse de cet enfoiré de Korax, âme damnée des forces du Mal, la Voisin du sixième, connue pour empoisonner ses victimes avec de fortes doses de porto frelaté.

Tout comme ça aurait pu être Marc-Antoine Jacquinot, le prof de philo qui résidait à l’autre bout du couloir, un type d’une politesse exemplaire, irréprochable, mais dont le comportement laissait clairement entendre qu’il ne tenait aucunement à faire plus ample connaissance avec les autres locataires. C’était quelqu’un, voyez-vous, qui passait le plus gros de son temps à l’intérieur de lui-même, et n’en sortait que pour planer à mille lieues au-dessus des vicissitudes de l’existence, solitaire et majestueux dans le noir ciel de la connerie humaine, si loin qu’il avait fini par perdre tout contact avec la triste réalité de la vie bassement matérielle et intellectuellement indigente de ses contemporains. C’était le genre de type qui n’avait pas la télé et vivait dans le noir en permanence, rideaux tirés, s’éclairant à la bougie et se nourrissant de fruits et gâteaux secs, ainsi que de plats préparés qu’il réchauffait au micro-ondes, seule et unique concession de sa part au monde moderne, et consommait directement dans leur emballage, l’idée même de salir une assiette et se voir contraint de la laver lui apparaissant comme une aberration sans nom. Il n’avait bien entendu ni femme ni enfant, et encore moins d’animal domestique, chien, chat ou canari, tortue ou poisson rouge, lièvre ou musaraigne. La mère Ouvrard, toujours à la recherche de nouvelles victimes, avait maintes fois essayé de lui faire monter ses courses dans l’escalier, quand l’ascenseur faisait des siennes, s’affichant à mi-chemin du trajet, recroquevillée sur les marches telle une vieille chenille en voie de décomposition, une crotte de chien abandonnée dans le caniveau, physiquement au bout du rouleau, à l’article de la mort. Mais Jacquinot, qui n’était pas tombé de la dernière averse, avait toujours très habilement refusé, d’une voix grave et tranquille suintant la gentillesse, l’amour de l’autre, se confondant en excuses bouleversantes d’humilité, mettant en avant une faiblesse extrême liée à toutes sortes de problèmes de santé, de tares chroniques qui laissaient la médecine perplexe, à commencer par une série de hernies discales inexplicables qui le faisaient cruellement souffrir depuis son plus jeune âge. Folle de rage, dont elle ne laissait rien paraître, la vieille sorcière rêvait de lui administrer quelques doses de porto qui mettraient un terme définitif à ses souffrances. Même Korax, qui n’était pourtant pas du genre timide, hésitait à s’approcher de lui. Il faut dire que les rares fois où il s’y était risqué, il s’était pris des coups de la lourde sacoche en cuir qui accompagnait Jacquinot dans la majeure partie de ses déplacements.

Mais je savais, animé par ce sixième sens qui est l’apanage des plus fins limiers, qu’il ne s’agissait pas de lui ni de la mère Ouvrard.

Aussi, quand la porte de l’ascenseur s’est ouverte, n’ai-je pas été autrement surpris de me retrouver nez-à-nez avec ce cher Dardariel, exécuteur des basses œuvre de monseigneur Mathéo Riqueti. Figurez-vous que ce crétin, qui s’attendait à tout sauf à se retrouver aussi vite en face de moi, était en train de visser tranquillement un réducteur de son au bout d’un Glock 17 à canon fileté. Autrement dit, la visite qu’il s’apprêtait à me rendre n’avait rien de courtoise.

Quand il a vu Manu, auquel il avait déjà présenté et qui lui avait laissé un douloureux souvenir au coin de la mâchoire, et qu’il a entendu ma voix chaude et onctueuse lui intimer l’ordre de mettre les mains aussi haut que possible au-dessus de sa tête de nœud et sortir de l’ascenseur en marchant sur des œufs s’il ne voulait pas repeindre l’habitacle avec sa cervelle de moineau, il a réalisé un peu tard que ses compétences en matière de filature laissaient fortement à désirer.

Dès qu’il a été dehors, j’ai récupéré le Glock et lui ai attaché les mains dans le dos avec la paire de menottes que je trimballais sur moi en permanence.

Je lui ai demandé : T’as un permis de port d’arme, mon petit Dardariel ?

Il a répondu : Oui, je fais du tir au stand de la police.

Ce à quoi j’ai rétorqué : Et un permis de connerie, t’en as un aussi ?

Il m’a regardé en serrant les dents, et son visage était aussi chargé de haine que le ciel de Pompéi était chargé de cendres et autres vapeurs toxiques après l’éruption du Vésuve en 79 après J.-C.

J’ai remballé Manu, non sans lui avoir auparavant tendrement caressé la crosse en signe d’indéfectible affection, puis j’ai retourné Dardariel et lui ai collé le canon de son Glock dans le creux des reins : On va aller faire un tour. Si tu fais le moindre geste suspect, respire trop fort ou lâche un pet de travers, je te bute sans hésiter.

On allait monter dans l’ascenseur quand Korax, émissaire velu des forces occultes qui gouvernent le monde, s’est ramené, la queue en l’air et le trou de balle bien en évidence.

Il m’a toisé, et pour la première fois j’ai cru déceler une lueur de respect, sinon de sympathie, dans le fond de son regard.

Une fois en bas, j’ai dit à Dardariel : Passe devant, je te suis. Et rappelle-toi ce que je t’ai dit : au moindre pet de travers, je te bute.

Korax est resté dans l’ascenseur, puis il a sauté sur le 6 pour remonter gratter à la porte de la mère Ouvrard, laquelle devait être en train de somnoler comme une grosse merde en robe de chambre et pantoufles devant Qui veut gagner des morpions ou Les fions de l’amour.

Le cerveau de Dardariel tournait à plein régime pour imaginer une issue favorable à la situation. Après tout il n’avait rien fait de mal, à part se retrouver dans l’ascenseur d’un immeuble qui se trouvait, par le plus grand des hasards, être celui dans lequel résidait Djeferson Beauvais, inspecteur spécialisé dans la traque des criminels les plus endurcis, tueurs en série, violeurs d’enfants et autres saloperies du même genre. Certes il était, au moment des faits, porteur d’un Glock 17 et se trouvait précisément en train de l’équiper d’un silencieux quand Beauvais lui était tombé dessus, mais il était détenteur d’un permis de port d’arme, et, à sa connaissance, aucune loi n’interdisait spécifiquement de visser un silencieux à un pistolet dans un ascenseur, même s’il reconnaissait volontiers l’existence d’endroits sans doute mieux adaptés à ce type d’activité. Il était, d’autre part, titulaire d’une licence séminariale (il était l’auteur d’une thèse sur les mouvements pentecôtistes américains et les dérives sectaires du Renouveau charismatique), inspiré par et d’un diplôme de prêtre garantissant la parfaite moralité de sa personne. Ajoutez à cela qu’il ne travaillait rien moins que pour le compte de monseigneur Mathéo Riqueti, évêque du Sanctuaire de Ddarr (lieu saint de l’Église catholique et romaine) et ami personnel du \foreignlanguage{italian}{cardinale di tutti li cardinali} Prospero Cangelosi, chef de chœur au Vatican et bras droit de Sa Sainteté le Pape, et vous conviendrez qu’il est difficile, sinon inconvenant, de suspecter un tel homme de quelque mauvaise intention que ce soit, funeste à plus forte raison. Il exigerait la relaxe immédiate et les excuses publiques du sieur Djeferson Beauvais, voire sa radiation définitive des services d’élite de la police nationale. Sa rétrogradation au niveau de simple agent de la circulation, condamné à faire le pied-de-grue à la sortie des écoles, ne serait pas pour lui déplaire.

Je lui ai demandé : T’es garé où ?

Il a fait, avec un geste du menton en direction de l’endroit : Là-bas, un peu plus loin.

\textsc{Moi} : Okay, on y va.

\textsc{Lui} : Vous allez faire quoi ?

\textsc{Moi} : On va faire un tour en bagnole.

Quand la Giulietta a été en vue, je lui ai demandé, en appuyant ma demande d’une légère pression sur le Glock : Où sont les clés ?

\textsc{Lui} : Dans la poche de ma veste.

\textsc{Moi} : Laquelle, je te prie ?

\textsc{Lui} : La droite.

J’ai pris les clés et ouvert le coffre de la Giulietta.

\textsc{Lui} : Vous faites quoi, là ?

\textsc{Moi} : Tu vois bien, j’ouvre le coffre.

\textsc{Lui} : Pourquoi faire ?

\textsc{Moi} : Je t’ai dit qu’on allait faire un tour en bagnole. Allez, monte là-dedans.

\textsc{Lui} : On va où ?

\textsc{Moi} : Faire un tour en bagnole. J’ai toujours rêvé de conduire une Alfa. Je t’en prie, installe-toi.

\textsc{Lui}, de plus en plus inquiet : Pourquoi il faut que je monte dans le coffre ?

\textsc{Moi} : Parce que je te le demande. Je n’ai aucune confiance en toi, alors je préfère te savoir dans le coffre pendant que je conduis.

\textsc{Lui} : Je peux savoir où on va ?

\textsc{Moi} : Monte, tu verras bien.

Il a fini par monter, non sans rechigner dans les grandes largeurs, et dès qu’il a été à l’intérieur, après avoir pris soin de vérifier que personne n’était en train de reluquer alentour, je me suis permis, à sa plus grande surprise (tout est allé très vite et il n’a pas eu le temps de formuler la moindre objection), de lui loger trois balles dans le buffet qui ont définitivement mis un terme à ses activités de mercenaire de Dieu.

L’avantage d’habiter un quartier tranquille, c’est qu’on peut flinguer les gens en toute sérénité, sans se soucier de témoin éventuel qu’il faudrait flinguer à son tour et ainsi de suite jusqu’à ce qu’il n’y ait plus personne sur terre. C’est sans doute ce qui arrivera un jour, mais je ne me sentais pas vraiment d’attaque pour éliminer un par un huit milliards et quelque de témoins gênants. Le jour venu, ce n’est pas un par un que les gens crèveront mais par treize à la douzaine, décimés par des armes chimiques, bactériologiques ou nucléaires, voire naturelles si la planète finit par se fâcher pour de bon comme il semble qu’elle soit en train de le faire.

Je me suis glissé au volant de la Giulietta et j’ai appelé Titus pour lui annoncer que je rappliquais dare-dare avec un truc dans le coffre dont je comptais bien me débarrasser au plus vite.

Le truc en question s’appelait Dardariel, et il s’agissait d’une sorte de moine de combat qui bossait pour Mathéo Riqueti. Cet enfoiré s’était pointé à mon domicile dans l’intention manifeste de me buter. Seulement je l’avais repéré en moins de deux et lui avait tendu un petit guet-apens dans lequel il s’était précipité comme un débutant. J’avais aussi pas mal avancé sur le Brain Catcher et les croquettes Waterflox. Passereau, même s’il avait le profil de l’emploi et avait un temps possédé un Ratier de Majorque, était selon moi définitivement hors de cause, même si j’étais certain qu’on aurait pu trouver des trucs pas très reluisants sur son compte en fouillant dans les recoins et autres interstices nauséabonds de sa pitoyable existence. De toute façon, ne serait-ce que financièrement parlant, les croquettes Waterflox, n’étaient pas à la portée de ses bourses.

Par contre, d’après cette ordure de Riqueti, lequel s’était pas mal fait tirer l’oreille avant de cracher le morceau, les pères Vidal et Beaubois faisaient plus que tripoter des enfants de chœur et enfiler louveteaux et jeannettes pendant les camps de vacances. Car comme le disait fort justement le regretté Lord Robert Stephenson Smyth Baden-Powell of Gilwell\nf{Robert Baden-Powell (1857--1941), général britannique, héros du siège de Mafeking lors de la guerre des Boers (1899--1902). Fondateur du mouvement scout mondial (\textit{Scouting for Boys}, 1908) et des Louveteaux (1916). Grand Commandeur de l’ordre royal de Victoria, il mourut au Kenya en 1941. \source{fr.wikipedia.org/wiki/Robert\_Baden-Powell}}, héros de la guerre des Boers et Grand Commandeur de l’ordre royal de Victoria : «~Les louveteaux ont une recette simple pour se procurer du bonheur : ils s’arrangent pour en donner aux autres~». De quoi entrevoir des perspectives quasiment illimitées et rayonnantes de lumière céleste pour tout prédateur sexuel digne de ce nom. Aux dernières nouvelles, Vidal et Beaubois faisaient partie d’une organisation pédocriminelle incluant non seulement des ecclésiastiques plus ou moins gradés dans la hiérarchie religieuse, mais aussi des individus venant de toutes disciplines et horizons dont les identités, si elles venaient à être connues du grand public, pourraient créer un séisme sans précédent sur le territoire national et même au-delà. J’ignorais ce que Riqueti savait au juste et jusqu’à quel point il était impliqué, mais sa tentative pour me faire taire prouvait que l’affaire était d’une gravité hors norme.

Titus Beaugendre avait deux caractéristiques dont je ne vous ai pas encore parlé, non que je veuille faire des secrets mais tout simplement parce que l’occasion ne s’était pas encore présentée : un, c’était un homme de couleur, noire en l’occurrence, et deux, c’était un homme marié. Le fait qu’il soit de couleur n’a pas de réelle importance pour la suite de l’intrigue, inexistante par ailleurs (je n’aime pas tout ce qui est intrigue, mystère et cachotterie, coup de théâtre et rebondissement de mes deux), mais je tenais à le signaler pour que les choses soient claires. Si je ne l’avais pas écrit noir sur blanc, il y a de grandes chances que votre cerveau de lecteur de race blanche l’ait imaginé comme un mâle caucasien issu des chasseurs-cueilleurs du Paléolithique supérieur ou des bergers Yamnaya des steppes pontiques de basse Volga, et cela, je ne puis me résoudre à l’accepter. D’autant qu’une telle méprise, sans être nécessairement catastrophique, pourrait néanmoins conduire à des contresens regrettables.

D’origine africaine, donc (Sierra Leone en l’occurrence, descendant d’un loyaliste noir des King’s American Dragoons débarqué à Freetown à la fin du XVIIIe siècle), et marié, Titus vivait avec femme (Bérénice) et enfants (deux, un garçon et une fille, Paul et Virginie) dans un pavillon de banlieue (hérité de ses parents, je vous épargne leurs noms et le reste, ça n’a aucun intérêt pour la suite de l’enquête, enquête dont on se contrefout accessoirement, mais bon, c’est un autre sujet) assez moche mais doté d’un petit lopin de terre pour faire pousser des patates et des laitues, ce que notre homme s’employait à faire dès qu’il avait un moment de libre. Sauf que des moments de libres, il en avait rarement, entre son boulot de flic et ses activités parallèles de justicier masqué, de sorte que c’était plus généralement Bérénice et les enfants (Paul surtout, car Virginie, férue de littérature, préférait se plonger dans la lecture assidue des œuvres complètes de Rousseau, Montesquieu, Voltaire, Sainte-Beuve, et bien sûr Bernardin de Saint-Pierre\nf{Jacques-Henri Bernardin de Saint-Pierre (1737--1814), écrivain et botaniste français, auteur de \textit{Paul et Virginie} (1788), roman idyllique situé à l’île Maurice, et d’\textit{Études de la nature} (1784). Élève du père La Caille à Caen, il devint l’ami et le disciple de Jean-Jacques Rousseau. \source{fr.wikipedia.org/wiki/Bernardin\_de\_Saint-Pierre}}, ancien élève de l’abbé La Caille\nf{Nicolas-Louis de La Caille (1713--1762), astronome français, prêtre catholique, professeur au Collège Mazarin à Paris. Il dirigea l’expédition au cap de Bonne-Espérance (1750--1754) qui lui permit de cataloguer dix mille étoiles australes et de mesurer un arc de méridien dans l’hémisphère sud. \source{fr.wikipedia.org/wiki/Nicolas-Louis\_de\_La\_Caille}} au collège des Jésuites de Caen) qui s’occupaient des patates et des laitues, ainsi d’ailleurs que d’à peu près tout dans la maison.

Il était précisément en train de fumer une clope dans le jardin quand il a reçu mon appel concernant Giulietta et son encombrant chargement. Dès qu’elle me voyait rappliquer, Bérénice savait que j’allais lui emprunter son mari et qu’elle ne le reverrait plus que tard dans la nuit, quand il rentrerait (ivre ou non) sans faire de bruit (ou en essayant d’en faire le moins possible) avant de la rejoindre subrepticement sous la couette. Elle voyait en moi un trublion de la vie domestique, un empêcheur de tourner en rond, un casse-couille de première. Pour elle, son mari était un homme droit et honnête que ses mauvaises fréquentations amenaient parfois à déraper, lesquelles mauvaises fréquentations se résumaient essentiellement à un seul et unique nom : le mien. Malgré tout, même si elle rêvait régulièrement de voir mon corps pendu au bout d’une corde ou réduit en charpie dans un amas de tôle, elle ne pouvait s’empêcher s’éprouver pour moi une certaine tendresse mêlée de fascination morbide, du même genre que celle que Titus nourrissait à mon égard. Il était, je dois le dire, le seul ami d’enfance qui me soit resté fidèle. Très tôt, alors que nous n’étions encore que des adolescents en proie aux affres de la puberté (et que j’étais personnellement couvert d’acné qui a laissé des traces sur ma peau), j’ai décelé chez lui un sens aigu de la justice et de réelles aptitudes à la violence. À de nombreuses reprises, il m’a sauvé la mise alors que j’étais en passe de me faire massacrer par un adversaire d’une bande rivale. Je le suis un peu plus maintenant, mais à l’époque je n’étais pas très costaud. Hargneux, oui, mais pas très costaud. Et aussi assez habile intellectuellement, ce qui me permettait de remplir les fonctions de chef d’équipe et cerveau des opérations.

Naturellement, il fallait attendre que la nuit soit bien noire pour aller déposer le colis dans un endroit tranquille.

D’irrésistibles effluves de coulis (colis, coulis, certains mots tissent entre eux d’étranges résonances) de tomates fraîches, huile d’olive et viande de bœuf rôtie parvenaient à mes naseaux frémissants dans l’air du soir. La fenêtre de la cuisine était grande ouverte, et c’était le chemin qu’ils empruntaient pour arriver jusqu’à moi. Pas besoin d’avoir un nez crochu, fumer la pipe et s’appeler Sherlock Holmes pour comprendre que la maîtresse des lieux était en train de concocter sa légendaire sauce bolognaise, une sauce dans laquelle elle mettait, outre ses talents culinaires, tout son amour et sa sensualité. Je suis aussitôt allé lui présenter mes salutations respectueuses, laissant discrètement pendre ma langue pour signifier que le pauvre célibataire que j’étais n’avait rien à manger dans son frigo, ce qui n’était pas totalement faux.

\textsc{Titus}, qui connaissait le langage des langues pendantes, a proposé : Tu mangeras bien avec nous.

\textsc{Moi}, pauvre célibataire aussi sournois qu’un serpent glissant sans bruit telle une babouche furtive dans le sable du désert chauffé à blanc : Je ne voudrais pas déranger.

\textsc{Bérénice} : Tu sais bien que tu ne nous déranges jamais, Djef.

Elle mentait presque aussi bien que moi.

J’ai ajouté : Je crois que si je devais n’emporter qu’une seule chose sur une île déserte pour y passer le restant de mes jours, ce serait les spaghetti bolognaise de Bérénice.

J’exagérais sans doute un peu, mais le fait est qu’entre ça et les œuvres complètes du grand poète élisabéthain Edmund Spenser\nf{Edmund Spenser (vers 1552--1599), poète anglais de la période élisabéthaine, auteur de \textit{La Reine des fées} (\textit{The Faerie Queene}, 1590--1596), vaste épopée allégorique en douze livres projetés dédiée à la reine Élisabeth I\textsuperscript{re}. Il est considéré comme le «~prince des poètes~» de la Renaissance anglaise. \source{en.wikipedia.org/wiki/Edmund\_Spenser}}, pour lequel j’ai pourtant une admiration sans bornes (je pense notamment à son chef-d’œuvre La Reine des fées), je n’aurais pas hésité pas un seul instant.

Titus, qui n’attendait que ça, s’est précipité à la cave (on aurait dit le piaf dans Bip Bip et Vil Coyote) pour en revenir quelques fractions de seconde plus tard en compagnie de deux fiasques de Chianti Rodolfo Dragone, un breuvage pour lequel il affichait une appétence proche de l’addiction. Moi un peu moins, étant donné son acidité décapante, mais je comptais sur la cuisine de Bérénice pour atténuer le feu qui ne manquerait pas de me déchirer les entrailles après quelques verres de cette piquette machiavélique. Quant à Titus, sa corpulence de gorille des montagne s’accompagnait d’une résistance hors du commun à l’alcool. Il aurait pu avaler les deux fiasques cul sec sans ressentir la moindre trace d’ébriété, alors que des gens normaux comme vous et moi n’auraient pas survécu à l’ingestion de la première. Après être passés par toutes les couleurs de l’arc-en-ciel, nous nous serions effondrés comme des masses pour ne plus jamais refaire surface. Eh oui, la vie n’est pas juste, car si nous ressentons tous les effets délétères de la soif ou la désespérance, nous ne sommes pas tous égaux devant l’alcool.

Vers minuit, soit deux fiasques de Rodolfo Dragone plus tard (finalement pas si mauvais ; les premiers verres sont un peu raides, je vous le concède, mais avec un peu de persévérance on finit par lui découvrir des mérites insoupçonnés), Paul et Virginie (respectivement 14 et 16 ans pour celles et ceux que ça intéresse) avaient regagné leurs chambres depuis longtemps, la compagnie des adultes n’étant pour eux qu’une péripétie assez désagréable à laquelle il convenait de se soustraire au plus vite. Bérénice, sans doute soucieuse d’en apprendre un peu plus sur nos projets et s’assurer qu’on n’allait pas faire de connerie majeure dès qu’elle aurait le dos tourné, nous avait tenu la jambe un peu plus longtemps avant de tirer sa révérence, vaincue par nos blagues graveleuses de mâles alpha et nos itérations inflexibles concernant le secret professionnel et le fait que moins on en sait mieux on se porte.

Oui, nous avions des choses à faire, effectivement, mais le seul fait d’aborder le sujet, même de façon évasive comme c’était le cas, constituait déjà une grave entorse au devoir de réserve auquel nous étions soumis.

Non, l’opération ne présentait pas de danger particulier, et encore moins pour des professionnels de notre trempe.

Pourquoi la mener au milieu de la nuit ?

Eh bien, même si je n’étais pas censé répondre à cette question, le fait est que certaines choses nécessitent un peu plus de discrétion que d’autres, d’où l’obligation de travailler de nuit plutôt que s’exposer en plein jour. On pouvait, à la rigueur, classer ces activités nocturnes dans la catégorie des heures supplémentaires, évidemment non rémunérées pour cause de restrictions budgétaires drastiques. Suite à un certain nombre de malversations, hausse des matières premières, déboires commerciaux, fuites des capitaux, cupidité des actionnaires et autre train de vie somptuaire au plus haut niveau de l’État, les finances publiques étaient au bord du gouffre et le Peuple tout entier (surtout les pauvres, bien sûr, qui sont les plus nombreux et habitués à se faire mettre bien profond avec le sourire comme s’il s’agissait d’une marque d’affection de la part de leurs exploitants) allait une fois de plus devoir mettre la main à la poche pour leur sauver la mise. Question de solidarité avec ses dirigeants, lesquels, dès que la navire flamboyant de la Nation voguerait à nouveaux sur les flots argentés d’un avenir radieux, avaient fait la promesse de lui renvoyer grassement l’ascenseur, l’art de faire des promesses et ne jamais les tenir étant la première chose qu’on apprend dans les grandes écoles d’administration.

Et alors ?

Alors rien. Simplement, certaines affaires se traitent dans la pénombre plutôt que la lumière aveuglante des projecteurs. Il se trouvait que Titus et moi-même, sans être des agents secrets à part entière, devions parfois remplir des missions officiellement inexistantes en dehors de nos heures de travail. Rien, pas même le Chianti ou les somptueux spaghetti bolognaise dont elle venait de nous régaler, ne serait de nature à nous délier la langue. Même sous la torture, par exemple si on nous avait agrafé les paupières et obligés à regarder en boucle Super Mario Bros.\nf{\textit{Super Mario Bros.} (1993), film de science-fiction réalisé par Rocky Morton et Annabel Jankel, adaptation du jeu vidéo de Nintendo. Avec Bob Hoskins et Dennis Hopper, il fut un échec commercial et critique retentissant, régulièrement cité parmi les pires adaptations de jeux vidéo. \source{fr.wikipedia.org/wiki/Super\_Mario\_Bros.\_(film)}} de Rocky Morton et Annabel Jankel, jamais, je dis bien jamais, même si après vingt-quatre heures non-stop de Super Mario Bros. il avait fallu s’enquiller tous les pires films d’exploitation du cinéma pré-code hollywoodien, comme par exemple l’histoire tragique de cette secrétaire devenue pute dans Safe in Hell de William Wellman, jamais nous n’aurions livré la moindre information concernant une affaire en cours.

Et non, Bérénice, je suis désolé mais le monde n’est pas qu’un parc d’attraction peuplé de mascottes inoffensives et surdimensionnées qui font le bonheur des petits et des grands, le tout sur fond de musique de fête foraine affligeante et odeurs poisseuses de barbe à papa. Derrière les masques souriants se dissimulent des grimaces de super-vilains qui n’attendent qu’une occasion de faire régner la terreur sur terre. Et il se trouve, par le plus grand des hasards, que Titus et moi sommes précisément chargés, sinon de mettre un terme définitif à leurs agissements, au moins de modérer leurs ardeurs.

Et moi, quand je modère, je modère sans modération.

Vaincue par nos arguments fallacieux, ou plutôt profondément exaspérée par le tissu de conneries que nous n’avions cessé de lui tricoter depuis des heures, Bérénice avait fini par aller se coucher, laissant le champ libre à nos élucubrations judiciaires.

Une heure plus tard environ, la seconde fiasque de Chianti arrivant à son terme, j’ai sorti un Hiram \& Solomon Fellow Craft Gran Toro de la poche de ma veste, le partenaire idéal pour se remettre les idées en place en fin de soirée. Il n’était bien entendu pas question de fumer à l’intérieur, la fumée de cigare (et l’odeur de tabac froid qu’elle laisse derrière elle) étant considérée par la majeure partie des gens (hommes et femmes confondus, même si ces dernières, habituées à ne respirer que des odeurs flatteuses de cosmétiques, sont nettement plus intransigeantes sur le sujet) comme une des pires agressions olfactives qui soient. Comme je ne tiens pas à m’énerver inutilement, je ne ferai pas de commentaire sur la question. Titus, qui était lui-même tenu de griller ses clopes à l’extérieur pour ne pas pourrir l’atmosphère en plus de nuire à la santé de sa femme et ses enfants, a proposé qu’on aille se jeter quelques verres de Monte Negro (un grogue cap-verdien qu’on venait de lui faire découvrir et sur lequel il ne tarissait pas d’éloges) dans le jardin, histoire de fignoler les derniers préparatifs de notre expédition.

Le plan était simple et juteux : j’allais reprendre le volant de l’Alfa, avec son précieux chargement de merde ecclésiastique dans le coffre, et me rendre dans un no man’s land tellement pourri que même les rats et les cafards l’avaient déserté depuis longtemps. Je connaissais un terrain vague, parsemé de carcasses de voitures désossées, situé aux confins d’une zone industrielle abandonnée, livrée à la poussière, aux ronces, chiendent et détritus en tout genre, qui ferait parfaitement l’affaire. De son côté, avec son propre véhicule (un break 307 SW), Titus me collerait au train jusqu’à destination, ce qui lui permettrait non seulement de me donner un coup de main pour le gros œuvre, mais aussi de me ramener à bon port une fois la sale besogne accomplie. Par «~sale besogne~» j’entendais : mise à feu de Giulietta en vue de la destruction totale dudit véhicule et son contenu par les flammes. Ainsi, sa carcasse calcinée jusqu’à l’os viendrait s’ajouter à celles déjà présentes sur les lieux.

Le voyage s’est déroulé sans encombre, et une fois sur place j’ai ouvert le coffre pour vérifier que tout se passait pour le mieux dans le plus infect des mondes. D’autant que Titus, histoire de lui rendre un dernier hommage avant son départ pour l’au-delà, tenait à voir le macchabée.

Imaginez ma surprise en découvrant que le macchabée en question ne l’était pas tout à fait, en dépit des trois balles que je lui avais logées dans le buffet. Ratatiné dans le coffre, il nous dévisageait avec des yeux injectés de sang en tentant de nous agripper avec ses doigts crochus qui s’agitaient vainement dans le vide. D’autre part, des gargouillis sortaient de sa bouche en même temps que des bulles de bave ensanglantée. Même pour quelqu’un comme moi, habitué à voir des trucs horribles, c’était assez désagréable à regarder.

Titus a dit : Je croyais que tu l’avais buté.

Ce à quoi j’ai répondu, de bonne foi : Moi aussi.

\textsc{Lui} : Qu’est-ce qu’on fait, maintenant ?

\textsc{Moi} : On referme le coffre et on le fait cramer vivant.

En entendant ces mots, les gargouillis ont redoublé d’intensité dans le gosier du moine et ses yeux se sont mis à danser la salsa dans le fond de leurs orbites. Manifestement, brûler vif dans le coffre d’une voiture ne correspondait pas du tout à l’idée qu’il se faisait de la mort.

Titus semblait sensiblement du même avis.

Il a dit : C’est barbare, non ?

J’ai répondu : Oui, mais je te rappelle que les curés ne se sont pas gênés pour cramer des tas de pauvres filles sous prétexte que c'étaient des sorcières. Il serait peut-être temps de leur rendre un peu hommage, non, tu ne crois pas ?

\textsc{Lui} : Aux sorcières ?

\textsc{Moi} : Oui, aux sorcières et toutes les autres victimes de l’Inquisition.

\textsc{Lui} : Quand même, ça se fait pas de brûler un moine.

\textsc{Moi} : D’ailleurs, à propos de sorcières, tu ne trouves pas un peu bizarre qu’il soit encore en vie avec trois balles dans le corps ?

\textsc{Lui} : Maintenant que tu le dis.

\textsc{Moi} : La dernière fois que j’ai ouvert le coffre, il était bel et bien mort, je peux te l’assurer. Et maintenant, il ne l’est plus.

Titus s’est signé à plusieurs reprises avant de s’enfuir à toutes jambes en agitant les bras et poussant des cris dans une langue que je ne connaissais pas.

J’ai hurlé : Titus, reviens ici tout de suite !!!

Il s’est ramené, penaud, et a dit : Tu crois que c’est un mort-vivant ?

Voilà ce qui arrive quand on s’enquille non-stop les onze saisons de The Walking Dead sans boire autre chose que de la bière ni bouffer autre chose que des poignées de chips. J’entends d’ici les mauvaises langues insinuer que je présente Titus, qui est une personne de couleur (noire et l’occurrence, le jaune étant pris par les Asiatiques, le rouge par les Indiens, le vert par les Martiens, le bleu par les Schtroumpfs et les Na’vis, et le blanc par les Aryens et autres Indo-européens suprémacistes occidentaux), comme un sauvage craintif et superstitieux, un colosse aux pieds d’argile bête comme ses pieds, un singe à peine descendu de son cocotier, persuadé qu’un type peut revenir à la vie après avoir reçu trois balles dans le buffet et séjourné pendant des heures dans le coffre d’une Alfa Romeo (et ce n’est pas une question de marque, le résultat aurait été le même dans le coffre d’une Fiat ou une BM). C’est d’autant plus ridicule que non seulement Titus est un ami d’enfance, preuve que tout petit déjà je ne m’arrêtais pas à des considérations de ce genre, et ensuite parce que j’ai toujours été un fervent défenseur des droits de l’homme et du citoyen, y compris la femme, et, dans une moindre mesure, l’animal quel qu’il soit, à part peut-être quelques espèces profondément nuisibles dont il ne viendrait à l’idée de personne de sauver l’existence.

Voilà pourquoi, après avoir posé une main virile et paternelle sur l’épaule d’un Titus claquant des dents et flageolant des genoux à la seule vue du moine qui agitait ses longs doigts crochus en roulant des yeux dans le coffre, je lui ai solennellement déclaré, d’une voix calme et posée, aussi noire et profonde qu’une pinte de Guinness dans un pub de Kilmore Quay (ah le cri des fous de Bassan le soir au coin du feu) : Titus, mon ami, comme les vampires et les loups-garous, les morts-vivants n’existent pas. Par contre, le corps humain est capable de bien des choses et on est encore loin d’être au courant de tout. Je pensais avoir tué cet abruti, mais il n’était pas tout à fait mort et il s’est réveillé pendant le trajet. Voilà tout.

\textsc{Lui} : Tu crois ?

\textsc{Moi} : J’en suis sûr.

\textsc{Titus} : N’empêche qu’on ne peut tout de même pas le brûler vif !

\textsc{Moi} : Non, bien sûr, tu as raison. Il faut l’achever avant de procéder à l’incinération.

Je lui ai tendu le Glock : Tu veux t’en charger ?

\textsc{Lui} : Non, merci. Il faut leur tirer une balle dans la tête, sinon on n’est pas sûr qu’ils sont morts.

\textsc{Moi} : Qui ça, ils ?

\textsc{Lui} : Les morts-vivants.

\textsc{Moi} : Ça n’est pas un mort-vivant, Titus. C’est juste un vivant qui n’est pas encore tout à fait mort.

Pendant qu’on discutait de savoir s’il était mort ou vivant, ou les deux à la fois, le dénommé Dardariel se vidait de son sang dans le coffre de l’Alfa, ce qui ne prêtait pas à conséquence puisque personne n’aurait jamais à le nettoyer. Il était de mon devoir d’être humain et responsable de mettre un terme aux souffrances dudit Dardariel. J’ai approché le Glock de sa tête, positionné le canon contre sa tempe, puis, sans me soucier de ses yeux hagards qui tentaient désespérément de me faire comprendre que ce que je m’apprêtais à faire était un acte ignoble totalement indigne de l’espèce dont j’étais en principe un représentant légal, après quelques secondes qui n’étaient pas d’hésitation mais simplement le fait de quelqu’un qui est ennemi de la précipitation et tient à ajuster son coup au millimètre près, j’ai appuyé sur la détente.

Tandis que sa cervelle giclait joyeusement aux alentours, les yeux de Dardariel se sont révulsés une dernière fois dans leurs orbites, comme des rats pris au piège, puis ses paupières se sont closes définitivement sur un monde qui tournerait largement aussi bien, sinon mieux, sans lui. Cela dit, compte tenu de sa conduite parmi nous, il avait du souci à se faire concernant l’accueil que les autorités célestes allaient lui réserver (Saint Pierre allait être furax et l’envoyer moisir au Purgatoire pendant les cent cinquante prochaines générations). Moi aussi j’en avais, du souci à me faire, sauf que je savais pertinemment que non seulement il n’y avait pas de vie après la mort, mais qu’il n’y en avait pas non plus avant, ou alors si peu que ça ne valait même pas la peine d’en parler.

On avait, en passant, rempli quelques bidons d’essence et fait le plein de l’Alfa à une station-service. Un réservoir plein à craquer nous assurait une crémation réussie, dans les règles de l’art. Les bidons en question se trouvaient dans le coffre de la 307. On a attendu quelques instants, histoire d’être sûrs que Dardariel n’allait pas se réveiller encore une fois, sortir du coffre en titubant, avec ses yeux injectés de sang, son crâne explosé et sa cervelle qui lui sortait par les oreilles et les trous de nez, et se jeter sur nous pour nous dévorer les entrailles et nous transformer en zombies. J’ai profité de l’occasion pour rallumer le Hiram \& Solomon qui croupissait entre mes dents depuis un certain temps. Titus, en homme prévoyant qui avait le sens des voyages et de la distraction (il n’avait pas son pareil pour organiser les pique-niques, les excursions en montagne et les anniversaires pour les gosses avec plein de bonbons et de ballons multicolores, autant de choses dont la seule évocation suffisait à me donner des sueurs froides), avait pensé à amener le fond de la bouteille de Monte Negro, de sorte qu’on pouvait se rincer le gosier à la fraîche tout en gardant un œil sur le macchabée. Pour un peu, ça aurait été un moment de poésie à en chialer des larmes de sang.

Je sentais que l’émotion n’était pas loin de nous submerger tel un raz de marée venu des Antipodes, aussi me suis-je dit à l’intérieur de moi-même qu’il était peut-être temps de songer à allumer le barbecue.

Ne pouvant garder une idée aussi brillante pour moi, je m’en suis ouvert à Titus, qui n’a rien trouvé de mieux à faire que pondre la réplique suivante : je me sentirais plus tranquille si tu lui tirais encore une ou deux balles dans la tête.

\textsc{Moi} : Si ça peut te faire plaisir.

Le mort était déjà mort et re-mort, mais peu importe, je lui ai logé quelques dizaines de grammes de plomb supplémentaires dans la cervelle. Il n’a pas bronché d’un millimètre, à part que ce qui avait été autrefois sa tête ressemblait maintenant à un steak haché baignant dans une sauce pas très ragoûtante.

Je me suis demandé si les comportements irrationnels dont il nous arrivait de faire preuve n’expliquaient pas en grande partie le parcours chaotique de notre espèce sur terre, lequel parcours, à force de cahoter, risquait de finir par se casser la gueule pour de bon. C’était le genre de fulgurance molle du genou qui me traversait parfois l’esprit, au plus fort de la nuit, surtout quand je m’apprêtais à faire cramer quelqu’un dans le coffre d’une bagnole. Il y a soixante-six millions d’années (à quelques mois près), un caillou (un gros, très gros caillou, d’une dizaine de kilomètres de diamètre) en provenance de Jupiter s’est écrasé sur le Yucatan, creusant un trou de cent kilomètres de diamètre par trente de profondeur, de quoi envoyer assez de cochonneries dans l’atmosphère pour soumettre la planète à une série d’évènements climatiques dévastateurs tels que : effet de gril dans un premier temps, avec flambée des températures et retombées de particules incandescentes qui crament tout sur leur passage, puis formation d’un écran de poussières toxiques en haute atmosphère qui plonge la planète dans le froid et les ténèbres pendant des décennies, le tout suivi d’une remontée brutale des températures due à la présence massive de gaz à effet de serre.

En gros, tout ce qui vit à la surface passe à la trappe, à commencer par nos bons gros vieux dinosaures qui s’ébrouaient tranquillement à la surface du globe. Voilà comment, alors que ça n’a aucun rapport direct avec l’histoire qui nous occupe, il m’arrive de songer à ce qui serait advenu si l’astéroïde avait raté sa cible et l’extinction K-T n’avait jamais eu lieu. Peut-être que les dinosaures seraient toujours parmi nous et qu’on pourrait aujourd’hui se balader avec un Tricératops ou un T-Rex miniature en laisse en guise d’animal de compagnie. On se demande aussi pourquoi Asteriornis maastrichtensis, sorte de poulet préhistorique qui hantait les basses-cours du Crétacé, ou encore Vegavis iaai, sympathique canard dont les restes vieux de soixante-dix millions d’années ont été retrouvés au nord de la terre de Graham, dans l’Antarctique, ont survécu alors qu’ils n’avaient à priori aucun endroit où se cacher, contrairement à celles et ceux qui pouvaient se réfugier dans les profondeurs de la terre ou la mer. Ce séisme a-t-il eu une influence directe sur l’Évolution au point que sans lui, par exemple, l’espèce humaine n’aurait jamais vu le jour ? Aux dernières nouvelles, après Morotopithecus bishopi en Ouganda et Nyanzapithecus alesi au Kenya, Archicebus achilles, un singe chinois de cinquante-cinq millions d’années à peine plus gros qu’une souris, est ce qui se fait de plus vieux en matière de primate. Peut-être que toute cette joyeuse bande de primates n’aurait jamais vu le jour sans l’astéroïde de Chicxulub.

Ce qui pose la question suivante : l’Évolution a-t-elle un plan de carrière, prévoyant à court ou moyen terme l’émergence de telle ou telle espèce ? Est-ce que, quoi qu’il arrive, nous devons nécessairement passer par ici pour en arriver là ? Fascinant, non ? Oui, mais aussi complètement chiant, raison pour laquelle je vous propose de retourner sans plus tarder à nos moutons, lesquels, je le rappelle en passant, descendent tous du Mouflon oriental, lui-même descendant du Mouflon chinois Ovis shantungensis, ce qui ne nous rajeunit pas.

J’ai dit, avec une légère pointe d’agacement dans la voix : C’est bon, on peut y aller, maintenant ?

\textsc{Titus} : Oui, c’est bon.

On a récupéré les bidons d’essence et arrosé la caisse de fond en comble, des jantes au toit ouvrant en passant par la boîte à gants et la banquette arrière, avec une attention toute particulière pour Dardariel qui, en plus de son sang et sa cervelle, baignait littéralement dans le carburant. Il restait quelques allumettes extra longues dans la boîte qui me servait à allumer mes cigares. J’en ai craqué quelques-unes que j’ai dispatchées un peu partout dans le véhicule, lequel s’est aussitôt mis à flamber dans une ambiance pestilentielle de fin du monde.

Il ne nous restait plus qu’à prendre congé avant que le cocktail Molotov nous explose à la gueule.

Adieu Giulietta, belle italienne aux formes arrondies et courbes affriolantes, telle une Gina Lollobrigida\nf{Gina Lollobrigida (1927--2023), actrice et photographe italienne, icône du cinéma des années 1950--1960. \textit{La Proie des vautours} (\textit{Buona Sera, Mrs. Campbell}, 1968, réal. Melvin Frank) est l'une de ses dernières grandes comédies, où elle joue une Italienne qui a prétendu à trois Américains que chacun est le père de sa fille. \source{fr.wikipedia.org/wiki/Gina\_Lollobrigida}} dans La Proie des vautours ou une Sophia Loren\nf{Sophia Loren (née en 1934), actrice italienne, première comédienne non anglophone à remporter l'Oscar de la meilleure actrice (\textit{La Ciociara}, 1961). \textit{La Comtesse de Hong-Kong} (\textit{A Countess from Hong Kong}, 1967) fut le dernier film de Charlie Chaplin comme réalisateur, dans lequel elle donna la réplique à Marlon Brando. \source{fr.wikipedia.org/wiki/Sophia\_Loren}} dans La Comtesse de Hong-Kong (cette dernière référence permet de boucler en douceur la boucle avec Ovis shantungensis, le mouflon chinois dont je parlais précédemment, même s’il n’est bien évidemment pas question de comparer Sophia Loren à une quelconque sorte de ruminant).



\noindent Quelques jours plus tard, les journaux ont parlé d’un corps retrouvé carbonisé dans le coffre d’une voiture, au cœur d’une vaste friche industrielle abandonnée bien connue pour servir de lieu de rendez-vous à toute une faune de pervers qui venaient se tripoter la nouille entre les tas d’immondices.

L’incendie avait été d’une telle violence que la voiture et le cadavre avaient été nettoyés jusqu’à l’os.

L’inspecteur Djeferson Beauvais, chargé de l’enquête, n’avait pas caché sa terreur face à une scène de crime qu’il estimait digne des pires scènes de guerre et attentats perpétrés contre des populations innocentes par des fanatiques religieux de la pire espèce. Le mot «~pire~» (qui rime avec empire et vampire, ça n’a pas de lien direct mais quand même ça fait réfléchir) était celui qui revenait le plus fréquemment dans sa bouche, et il confessait avoir acquis, au fil du temps et des abominations sans cesse plus atroces auxquelles il était confronté, la certitude que quelque chose ne tournait pas rond chez l’être humain. D’après lui, ça posait un problème de fond sur lequel il faudrait bien que l’humanité décide de se pencher un jour ou l’autre. Et le plus tôt serait le mieux, car au train où allaient les choses, il ne nous resterait bientôt plus que nos yeux pour pleurer.

Avis largement partagé par le docteur Zaahid Shirani, légiste vedette de la PJ dont le regard pénétrant et les interventions régulières sur les plateaux de télé réjouissaient les amateurs d’énigmes policières et crimes sanglants, qui ne cachait pas la difficulté de faire correctement son travail dans des conditions aussi extrêmes.

Djeferson Beauvais, interrogé par la presse sur fond de carcasses de voitures, sol jonché de détritus et arbres faméliques tendant leurs membres décharnés vers un ciel éternellement gris, s’est déclaré fermement décidé à tout mettre en œuvre pour découvrir le fin mot de l’affaire, ajoutant avoir toute confiance en Zaahid Shirani pour mettre la main sur les indices les plus infimes pouvant conduire à l’arrestation du monstre responsable de cette tragédie.

Vous l’aurez compris, outre le fait que j’étais très embêté d’avoir à me rechercher moi-même pour me traduire devant la justice, j’étais aussi très remonté contre Riqueti, que je tenais pour personnellement responsable de la majeure partie de mes emmerdements. S’il n’avait pas envoyé ce tueur pour me dessouder, je n’aurais pas eu besoin de le faire cramer dans le coffre de sa voiture. À celles et ceux qui objecteraient que je n’étais peut-être pas obligé de le faire cramer dans le coffre de sa putain de voiture, je répondrai que l’objection ne tient pas face à la réalité du fait que tout cela n’aurait jamais eu lieu si cet imbécile n’était pas venu chez moi dans l’intention de me faire passer de vie à trépas. La chronologie fait loi, et je ne vois absolument pas en quoi le fait d’avoir fait cramer cet individu dans le coffre de sa voiture infirmerait la validité de mon raisonnement.

Je lui ai donc filé le train, à cet enfoiré, et quand j’ai vu qu’il se dirigeait vers sa luxueuse maison de campagne située dans un environnement protégé et ceinte de murs infranchissables, avec pour seule compagnie son nouveau chauffeur au volant de sa BMW Série 7 High Security (blindage renforcé, vitres multicouches résistant aux tirs de gros calibre, intérieur luxe tout confort, moteur V8 capable d’accélérations décoiffantes, vision infrarouge, technologie de pointe et communications sécurisées), je me suis dit que c’était le moment rêvé d’aller lui poser quelques petites questions sur ses activités récentes, notamment le fait qu’il n’hésite pas à payer des gens pour en tuer d’autres, chose ô combien décevante de la part d’un homme censé représenter Dieu sur terre. On ne jouait pas dans la même catégorie, c’est vrai, mais ce n’était pas un cureton de luxe qui allait me chier dans les bottes.

Non, ce qui m’inquiétait le plus, outre les murs infranchissables de la propriété qu’il allait bien falloir que je me démerde pour franchir d’une manière ou d’une autre, c’était le remplaçant de Dardariel, un type aussi large que haut, avec une tronche de primate mal dégrossi, des oreille en chou-fleur et des battes de baseball à la place des bras. Avec son costard-cravate et son chapeau de cowboy sur la tête, sans doute la seule fantaisie que lui autorisait son employeur, il ressemblait à un flic ricain genre Texas ranger qui tire sur tout ce qui bouge au moindre battement de cil. Et vu sa largeur d’épaules, il était certainement aussi capable de couper des bûches en deux avec le tranchant de la main et assommer un bœuf à coups de bite. Cette fois, j’avais intérêt à la jouer fine si je ne voulais pas me retrouver entre quatre planches.

La veille, et croyez bien que je n’en suis pas fier, j’étais au Narcisse Rose avec Zaahid Shirani et on avait passé la soirée à boire des Girofliers du Clair de Lune (un cocktail à base de rhum, champagne et clou de girofle, spécialement conçu pour faire sortir les morts de leurs tombes et les asticots de leurs claquos) jusque tard dans la nuit. On avait aussi pas mal parlé de la Compagnie des Indes orientales\nf{La Compagnie des Indes orientales fut fondée en plusieurs versions concurrentes~: britannique (\textit{East India Company}, 1600), néerlandaise (\textit{VOC}, 1602) et française (1664, sous Colbert). Instruments du capitalisme colonial, ces sociétés par actions détenaient le monopole du commerce avec l'Asie. La version anglaise gouverna l'Inde de facto jusqu'en 1858. \source{fr.wikipedia.org/wiki/Compagnie\_des\_Indes\_orientales}}, sujet qui semblait le passionner, et surtout du Bangladesh, pays d’où étaient originaires ses grands-parents vénérés, ainsi que du périple invraisemblable qu’ils avaient accompli pour arriver jusqu’à nous. Pour couronner le tout, on avait dû subir les assauts à répétition d’une paire de jumelles florentines, Tosca et Zarina Brizzi, lesquelles, mues par le vice propre aux filles de leur espèce, tenaient absolument à ce qu’on finisse la soirée à quatre dans leur suite du Jade Mountain Hôtel de l’avenue des Rossignols (très bel endroit au demeurant, dans lequel je n’avais encore jamais eu le privilège de mettre un orteil), offre que j’avais poliment déclinée pour des raisons éthiques d’une part, ayant été élevé dans le respect des traditions et valeurs morales en vigueur dans ce pays, pratiques d’autre part, étant promis à une grosse journée le lendemain. En soi, ma seule présence dans ce lieu de perdition était déjà la preuve d’un grave manquement de ma part aux exigences de ma profession, et je me verrais contraint, sitôt de retour dans la pénombre glacée de ma cellule, de me flageller jusqu’au sang pour obtenir le pardon de Dieu le Père, Christ ressuscité et tous les saints apôtres de la Parole divine. Je dois néanmoins reconnaître, à mon corps défendant, que Zarina exerçait sur moi une attraction qui n’avait rien à voir avec les montagnes russes ou le train fantôme. Non, on était plus proche des forces cosmiques qui régissent l’univers. Cela dit, quand on est pourvu d’une volonté d’acier, comme c’est mon cas, on arrive fort bien à se sortir de ce genre de traquenard sexuel dans lequel la nature essaye sans cesse de nous entraîner, la garce.

Shirani, très en verve, m’avait parlé du corps calciné dans le coffre de l’Alfa, réduit peu ou prou à l’état de cendre, et je lui avais répondu avec aplomb qu’il s’agissait sans doute d’un banal règlement de compte entre dealers, comme cela était malheureusement fréquent dans le quartier. Quant à la Giulietta, elle appartenait à un certain Kévin Charrier, domicilié au 18 rue des Nénuphars, que je m’étais fait un plaisir de soumettre à un interrogatoire musclé. En toute méchanceté gratuite, il va sans dire, car il était évident que le particulier, vendeur chez Alfa, n’avait strictement rien à voir dans l’affaire. Au contraire, il s’était fait piquer sa voiture le jour-même et, en bon citoyen responsable et respectueux des lois, était allé aussitôt s’en plaindre au commissariat le plus proche. Là, le planton de service, un petit gros au crâne dégarni et la moustache alerte, lui avait déclaré entre deux bâillements que tout serait diaboliquement mis en œuvre pour la retrouver dans les meilleurs délais. Il s’agissait d’une affaire d’ampleur nationale, d’une priorité absolue qui allait mobiliser les plus hautes instances et les troupes d’élite de la Nation.

Et puis, chemin faisant, on en était venus à parler de choses très intimes concernant la vie du légiste, des choses que je savais déjà plus ou moins, rumeur oblige, mais qu’il ne m’avait encore jamais fait l’honneur de me confirmer de vive voix, et je dirais même très vive voix, assez pour couvrir le niveau sonore de la musique d’ambiance. D’excellente qualité, du reste, la musique en question, dans le plus pur style Five Spot Café, le fameux club de jazz situé dans le Lower East Side de Manhattan\nf{Le \textit{Five Spot Café}, ouvert en 1956 au 5 Cooper Square dans le Lower East Side de Manhattan, fut la scène centrale du jazz moderne et du free jazz entre 1956 et 1967. Thelonious Monk y tint une résidence historique en 1957~; Ornette Coleman y créa le scandale de son quartette en 1959. \source{fr.wikipedia.org/wiki/Five\_Spot\_Caf\%C3\%A9}}. À l’époque, je parle de la fin des années 50, le quartier était une zone sinistrée dans laquelle les honnêtes gens auraient préféré se faire amputer des deux jambes plutôt que d’y mettre un pied. La plupart des résidents étaient des Noirs, et une bonne partie de ces Noirs étaient des musiciens de jazz. Pas le genre pépère à la Glenn Miller, Benny Goodman ou Duke Ellington, ce truc de riches pour s’encanailler dans des bars à putes. Non, un genre plus méchant, sauvage, imprévisible, laissant la part belle à l’imagination débridée de types qui en avaient gros sur la patate, en perdition pour la plupart. Pour eux, l’instrument était une arme qui leur permettait de pleurer en silence ou hurler à la mort, suivant l’état d’ébriété dans lequel ils se trouvaient. Leurs noms ? Vous les connaissez aussi bien que moi, sinon il est grand temps de vous mettre à écouter autre chose que du Sofiane Pamart et du Ibrahim Maalouf : Charlie Parker, John Coltrane, Shadow Wilson, Cecil Taylor, Johnny Griffin, Roy Haynes, Ahmed Abdul-Malik, Don Cherry, Ornette Coleman, Miles Davis, pour ne citer que les plus célèbres, et bien sûr le grand Thelonious Monk\nf{Thelonious Sphere Monk (1917--1982), pianiste et compositeur de jazz américain originaire de Rocky Mount (Caroline du Nord). Figure de proue du bebop, son style harmonique déconcertant et ses silences calculés en firent une des personnalités les plus singulières du jazz. Il vécut les six dernières années de sa vie chez Pannonica de Koenigswarter dans le New Jersey, presque muet. \source{fr.wikipedia.org/wiki/Thelonious\_Monk}} qui reste une énigme pour nombre d’amateurs. C’est dans ce foutoir savamment désorganisé, expressionniste, abstrait, dans lequel trainaient également toute une faune d’artistes dégénérés dans le genre de ceux qu'Hitler rêvait d’exterminer en même temps que les Juifs, les communistes, les francs-maçons et les homosexuels, qu’est arrivée «~la Baronne~», Pannonica de Koenigswarter\nf{Kathleen Annie Pannonica de Koenigswarter, née Rothschild (1913--1988), surnommée «~la Baronne~». Issue de la branche britannique des Rothschild, elle devint la principale mécène du jazz à New York, hébergeant et soutenant Charlie Parker (mort dans sa chambre en 1955) et Thelonious Monk (qui vécut chez elle ses six dernières années). \source{fr.wikipedia.org/wiki/Pannonica\_de\_Koenigswarter}}, Nica pour les intimes, riche héritière au caractère bien trempé, folle de jazz, militante des droits de l’homme, grande amie des minorités oppressées et des Noirs en particulier. À titre d’exemple, c’est dans sa suite de l’hôtel Stanhope\nf{Charlie Parker (1920--1955), saxophoniste alto américain surnommé «~Bird~», figure fondatrice du bebop. Il mourut le 12 mars 1955 à 34 ans dans la chambre new-yorkaise de Pannonica, à l’hôtel Stanhope. Le médecin légiste, stupéfait par l’état de son corps ravagé, lui attribua entre 50 et 60 ans. \source{fr.wikipedia.org/wiki/Charlie\_Parker}}, en face de Central Park, que Charlie Parker, usé par la drogue, rendra son dernier souffle à trente-quatre ans (il en paraissait le double). Née Rothschild, Nica avait à dix ans quand son père vénéré, banquier et naturaliste fou, s’est tranché la gorge dans un moment de déprime. Charles\nf{Charles Rothschild (1877--1923), banquier londonien de la maison Rothschild et entomologiste passionné. Il décrivit plusieurs centaines d’espèces de puces, dont \textit{Xenopsylla cheopis}, principal vecteur de la peste bubonique. Atteint de troubles mentaux sévères (possiblement une encéphalite léthargique), il se donna la mort en 1923 dans une maison de repos. \source{fr.wikipedia.org/wiki/Charles\_Rothschild}}, comme son frère Lionel, vouait une passion immodérée à l’entomologie. Quand il n’était pas en train de glander dans son bureau de Thames Street, dans la City, il battait la campagne un filet à la main avant de rentrer chez lui au pas de gymnastique pour épingler ses trouvailles dans des boites en carton. Il avait aussi une passion dévorante pour les puces, ces saletés de suceuses de sang capable de sauter plus de trois cents fois leur taille, soit l’équivalent pour un être humain d’un bond au-dessus de la Vostok Tower\nf{La Vostok Tower (374 m), achevée en 2016 dans le quartier d’affaires de Moskva-City (Complexe de la Fédération) à Moscou, est le deuxième plus haut gratte-ciel de Russie. Son nom évoque le programme spatial soviétique Vostok. \source{fr.wikipedia.org/wiki/Vostok\_Tower}} du Complexe de la Fédération, à Moskva-City. Sur les quelques vingt mille espèces aujourd’hui répertoriées, nombre qui file le vertige et montre à quel point les parasites les plus assoiffés de sang disposent d’une formidable capacité d’adaptation, et je ne dis pas ça pour nous, Charles Rothschild en a décrit quelques centaines, ce qui n’est déjà pas si mal pour l’époque même si ça ne suffit pas à faire de lui un naturaliste de premier plan. D’autant qu’il était blindé de thune et disposait de tout le loisir nécessaire pour jouer les Jean-Henri Fabre\nf{Jean-Henri Fabre (1823--1915), entomologiste et naturaliste français, auteur des dix volumes des \textit{Souvenirs entomologiques} (1879--1907). Victor Hugo le surnomma «~l'Homère des insectes~». Son nom est associé à plus de 200 espèces et genres d'insectes. \source{fr.wikipedia.org/wiki/Jean-Henri\_Fabre}}. Mais bon, ce n’est pas si courant pour un banquier de se passionner pour les insectes, même si l’esprit affûté ne manquera pas de voir un lien direct entre les puces qui sucent le sang des gens et les banquiers qui leur sucent leur pognon. Pannonica s’appelait en réalité Kathleen Annie Pannonica de Rothschild, Pannonica étant le seul resté à la postérité, et elle avait épousé le baron Jules de Koenigswarter, ancien de Janson-de-Sailly et diplômé de l’École des Mines, héros de la Campagne de Tunisie et du débarquement à Cavalaire-sur-Mer sous les ordres du général de division Jean Touzet du Vigier\nf{Jean Touzet du Vigier (1896--1969), général de corps d'armée français. Il commanda la 1re Division blindée lors de la Campagne de Tunisie (1943) puis le débarquement de Provence (août 1944) à Cavalaire-sur-Mer à la tête du 1er Corps de la 1re Armée française. \source{fr.wikipedia.org/wiki/Jean\_Touzet\_du\_Vigier}}. Il a une bonne tête mais ce n’est pas forcément le mec le plus marrant qui soit. En même temps, comme il n’est jamais là, sa présence ne constitue pas un obstacle majeur à leur union. Pannonica, prénom étrange s’il en est, vient de Pannonie, ancienne province de l’Empire romain correspondant plus ou moins à l’actuelle Slovénie, région où Charles s’est offert le luxe de mettre la main sur une nouvelle espèce de papillon de nuit. Le 1er juin 1954, Thelonious Monk, 37 ans, originaire de Rocky Mount en Caroline du Nord (état qui, je le rappelle, appliquait entre 1927 et 74 le programme de stérilisation forcée contre les Noirs et les Indiens, entreprise génocidaire validée par le juge à la Cour suprême des États-Unis Oliver Wendell Holmes\nf{Oliver Wendell Holmes Jr. (1841--1935), juge associé à la Cour suprême des États-Unis (1902--1932). Dans l'arrêt \textit{Buck v.\ Bell} (274 US 200, 1927), il rédigea la décision validant la stérilisation forcée, concluant~: «~Three generations of imbeciles are enough.~» Cette décision ne fut jamais officiellement annulée. \source{fr.wikipedia.org/wiki/Oliver\_Wendell\_Holmes\_Jr.}}), est à Paris pour participer au troisième Salon du Jazz, salle Pleyel\nf{La salle Pleyel, inaugurée en 1927 au 252 rue du Faubourg-Saint-Honoré à Paris (8e), est l’une des grandes salles de concert françaises. Le \textit{Salon du jazz} du 1er juin 1954 y réunit pour la première fois de nombreux jazzmen américains devant un public parisien, dont Thelonious Monk, Sidney Bechet et Gerry Mulligan. \source{fr.wikipedia.org/wiki/Salle\_Pleyel}}, en même temps que Sidney Bechet, le Gerry Mulligan Quartet, Jonah Jones, Claude Luter et Martial Solal. Il se fait copieusement huer par la foule en délire, qui le traite de singe débile et maniaque, avant de se faire éreinter par la critique qui émet elle aussi l’idée qu’il serait plus à sa place à sauter de branche en branche dans la jungle qu’à jouer du piano dans une salle de concert. Dans son malheur, il croise celle qui deviendra sa plus fidèle groupie, indéfectible et riche amie, j’ai bien sûr nommé Pannonica de Koenigswarter, qui décide aussitôt de le prendre sous son aile et faire en sorte qu’il ne manque plus jamais de rien. Il faut savoir que Monk souffrait des mêmes désordres mentaux que Charles, le père de Nica, et que leur comportement était similaire à bien des égards, ce qui explique sans doute pourquoi elle a pris soin de lui pendant près de trente ans, dans l’espoir de ne pas le voir se trancher la gorge à son tour. C’est ainsi qu’il a passé les six dernières années de sa vie chez elle, dans un état de délabrement physique et moral croissant, sans toucher à un piano et se pissant dessus toutes les trente secondes pour cause de dérèglement aigu de la prostate, et s’est éteint pour ainsi dire dans ses bras de sa bienfaitrice, dans le New Jersey (comme pas mal d’autres musiciens noirs, qui adoraient mourir dans le New Jersey, sans doute un des endroits les plus agréables pour mourir aux USA, et plus spécialement dans le comté de Bergen, à Englewood, où dès 1906 l’écrivain et militant socialiste Upton Sinclair\nf{Upton Sinclair (1878--1968), romancier américain, auteur de \textit{La Jungle} (1906), réquisitoire contre les abattoirs de Chicago qui provoqua une réforme sanitaire fédérale. Socialiste militant, il fonda en octobre 1906 à Englewood (New Jersey) l’\textit{Helicon Home Colony}, communauté utopique détruite par un incendie d’origine suspecte en mars 1907. \source{fr.wikipedia.org/wiki/Upton\_Sinclair}} fonde sa Helicon Home Colony, sorte de communauté pré-hippie ravagée par un incendie d’origine suspecte peu de temps après son installation).

Donc, comme je vous le disais, alors même que les sœurs Brizzi nous reluquaient avec insistance en se passant ostensiblement la langue sur les lèvres (et j’étais terrorisé tant je sentais que celle de gauche, dont je ne savais pas encore qu’elle s’appelait Zarina, mourait d’envie de me dévorer tout cru, et encore plus tant je devais bien admettre que cette perspective cannibalistique exerçait sur moi un véritable attrait), Zaahid Shirani m’a confié avec un soulagement manifeste (merci le Giroflier du Clair de Lune, breuvage magique qui n’a pas son pareil pour délier les langues les plus ratatinées dans le fond de leur gosier) qu’il avait une fille, Jaya (enfin, je savais déjà qu’il avait une fille et qu’elle s’appelait Jaya, mais il s’était toujours montré extrêmement évasif à son sujet et j’avais toujours eu la délicatesse de ne pas insister), qui lui causait bien du souci depuis des années, plus précisément depuis que sa mère Faustina, d’origine galicienne (elle avait vu le jour à Saint-Jacques-de-Compostelle), avait avalé assez d’alcool et de cochoncetés pharmaceutiques pour décimer un troupeau d’éléphants dans la force de l’âge.

Jaya avait douze ans au moment des faits, et, sans doute parce que le destin, d’une cruauté sans limites, prend plaisir à s’acharner sur des proies faciles, c’était elle qui avait découvert le corps en rentrant du collège. Elle avait passé une bonne demi-heure à essayer de ranimer sa mère, avant le retour de son père qui n’avait eu d’autre choix que de constater le décès. Suite à cette tragédie, le comportement de Jaya, brillante et pleine de vie jusqu’alors, avait commencé à changer, et ses résultats scolaires à piquer du nez. Il aurait pu s’agir d’un épisode transitoire, somme toute bien compréhensible dans le contexte, mais la situation, loin de s’améliorer, n’avait fait qu’empirer, les tentatives pour enrayer la chute n’ayant donné strictement aucun résultat.

À dix-sept ans, elle avait fait une tentative de suicide peu convaincante pour alerter sur la détresse psychologique qui ravageait son existence, telle une meute de rats rongeant peu à peu sa santé mentale. Zaahid avait pris la mesure de la chose et s’était efforcé d’agir en conséquence, ne faisant qu’aggraver la situation au fur et à mesure de ses initiatives, toutes plus malheureuses les unes que les autres.

L’année suivante, Jaya était en totale perdition, rupture scolaire et avec la société en général, et le conflit avec son père avait atteint sinon un point de non-retour, au moins de difficulté majeure à espérer s’en sortir un jour. Autant dire qu’ils ne se parlaient quasiment plus, les rares tentatives dans ce sens aboutissant inévitablement à un échange de hurlements laissant craindre le pire. Zaahid n’était pas d’une fidélité à toute épreuve, c’est vrai, il aimait le sexe, à deux ou à plusieurs, surtout avec des professionnelles, et il lui arrivait parfois de faire preuve d’un caractère difficile, sans parler d’accès de violence qu’il peinait à contrôler en dépit de sa consommation élevée de cannabis, mais de là à lui reprocher la mort de sa femme, il y avait un pas que Jaya ne s’était pas gênée pour franchir allègrement.

Aujourd’hui âgée de vingt-deux ans, méconnaissable, Jaya multipliait les addictions et n’hésitait pas à se prostituer pour satisfaire ses besoins. Elle avait défoncé toutes les barrières, arraché toutes les digues, pulvérisé tous les cadres. C’est dans ces conditions abominables qu’elle avait croisé la route d’un certain Simon Keskula, fondateur de l’Alliance de la Révélation, secte post-apocalyptique établie au fin fond des Alpes de Haute-Provence, au pied de la Montagne de Lure. Keskula, depuis longtemps dans le collimateur de la MIVILUDES\nf{Mission interministérielle de vigilance et de lutte contre les dérives sectaires, créée par décret en mai 2002 auprès du Premier ministre. Elle observe et analyse les mouvements à caractère sectaire et conseille les pouvoirs publics et les particuliers. \source{fr.wikipedia.org/wiki/Miviludes}}, n’en était pas à son coup d’essai, puisqu’on lui devait déjà la création du Monster Gang (une organisation genre X-Men censée regrouper des gens aux capacités hors du commun) et de l’Ordre de la Lune Noire (structure opaque en relation avec des groupuscules d’extrême droite), tous deux de sinistre mémoire et frappés d’interdiction par les autorités compétentes. Lesquelles, jusqu’à présent, se cassaient les dents sur l’Alliance de la Révélation, qui semblait bénéficier de la protection de certaines personnalités influentes.

Mais ceci est une autre histoire, aussi tragique que passionnante, sur laquelle je ne manquerai pas de revenir ultérieurement.

La BM a emprunté une voie privée qui conduisait à l’entrée de la propriété.

J’ai continué mon chemin comme si de rien n’était, avant de faire demi-tour et revenir me positionner dans les parages, hors de vue de Riqueti et sa bande de malandrins qui auraient pu être tentés d’observer les alentours à la jumelle.

L’entrée, en plus de ses grilles de quatre mètres de haut, étant équipée de caméras de surveillance interdisant formellement de se livrer à toute forme d’excentricité inconsidérée, j’ai jugé plus intelligent de faire le tour et trouver une brèche pour m’introduire discrètement dans le périmètre. Il fallait, pour atteindre le mur d’enceinte, s’aventurer dans une jungle d’orties et arbustes épineux qui n’avait pas grand-chose à envier, question impénétrabilité, à la forêt de Bwindi, en Ouganda\nf{La forêt impénétrable de Bwindi, en Ouganda, est inscrite au patrimoine mondial de l'Unesco depuis 1994. Elle abrite environ la moitié de la population mondiale de gorilles des montagnes (\textit{Gorilla beringei beringei}), soit environ 400 individus sur les 700 recensés à l'état sauvage. \source{fr.wikipedia.org/wiki/For\%C3\%AAt\_imp\%C3\%A9n\%C3\%A9trable\_de\_Bwindi}}, où vit la plus grande population de gorilles des montagnes encore en activité, espèce à laquelle appartenait vraisemblablement le nouveau chauffeur (lui-même potentiellement en voie de disparition) de Riqueti. Ce dernier (Riqueti, pas son chauffeur que je n’avais pas encore le plaisir de connaître, et je vous avouerai que je n’étais spécialement pressé de combler ce manque) avait clairement évoqué l’existence d’une organisation pédocriminelle de grande ampleur, connue sur OnionLand sous le nom de League of the Unknown Ribbon (la LUR, ce qui ne veut pas dire grand-chose car on ne voit pas très bien à quoi le ruban inconnu en question fait allusion, à moins que ce ne soit celui entourant le paquet-cadeau d’enfants innocents livrés à la barbarie de leurs tortionnaires) à laquelle les pères Vidal et Beaubois auraient appartenu. J’avais aussitôt mis mon pote Maël Robineau, expert en cybercriminalité, pirate informatique à ses heures et justicier à la petite semaine, sur le coup. Il avait épluché l’Oignon (le dark web, NDLR), notamment la nuit (lorsque sa femme et ses enfants dormaient à l’étage), dans le sous-sol de sa maison transformé en base secrète et centre de recherches ultrasensibles équipé de tout le matériel de pointe en matière de sécurité informatique et sécurité tout court (légèrement paranoïaque sur les bords, il avait fait installer une porte blindée avec serrure biométrique à empreinte digitale, du genre de celles qu’on trouve à Fort Knox ou à la banque de France), mais ses investigations n’avaient rien produit de réellement concluant. Je ne doutais pas de la réalité de cette organisation, sur laquelle j’aurais peut-être l’occasion de me pencher un jour ou l’autre, mais j’étais persuadé que Riqueti me dissimulait des choses de toute première importance. Faisait-il lui-même partie de cette organisation ? J’avais tendance à penser que non, sans quoi il n’aurait pas pris le risque de m’en parler, ou alors juste pour le plaisir de me coller un contrat sur la tête, ce qui n’était pas impossible de la part d’un esprit tordu comme le sien.

Quoi qu’il en soit, j’étais bien décidé à lui tirer les vers du nez une bonne fois pour toutes, voire carrément lui couper le nez en question au ras des joues avec un couteau rouillé et le lui enfoncer bien profond dans la rondelle, ne serait-ce que pour l’amener à réfléchir sur ce qu’il avait fait de sa vie et s’il avait des raisons d’en être fier.

Après une bonne demi-heure de recherches, à me démener dans la végétation en regrettant à chaque instant de n’avoir pas pensé à emporter une machette, j’ai fini par dénicher une vieille porte en bois qui n’avait pas dû être utilisée depuis l’attribution du prix Nobel de philosophie à Jean Baptise Pourri pour ses travaux sur la structure discontinue de la connerie. Elle était tellement vermoulue qu’elle ne tenait plus que par l’opération du Saint-Esprit, lequel (car immenses sont ses pouvoirs) se manifestait en l’occurrence sous la forme d’un exosquelette de lierre dont la chose (porte) était entièrement recouverte. J’ai entrepris d’arracher courageusement de grosses touffes du lierre en question, ce qui m’a pris un certain temps et permis de m’esquinter gentiment les mains, puis j’ai appuyé de tout mon poids sur la porte qui s’est affaissée en faisant un bruit bizarre évoquant furieusement celui que produit un éléphant de mer en glissant sur une plage de Californie.

À ce stade, je me suis demandé pourquoi je m’amusais à passer par derrière alors qu’il m’aurait suffi de passer par la porte principale en présentant ma carte de flic. Je me suis demandé : mais quelle est donc cette force obscure qui te pousse à prendre des chemins de traverse, emprunter des sentiers sinueux et semés d’embûches alors qu’il te suffirait de passer par l’entrée principale, celle que tous les gens normaux empruntent le plus naturellement du monde ?

N’ayant pas trouvé de réponse satisfaisante à cette question, sinon que je voulais bénéficier d’un effet de surprise incompatible avec une pratique standard, j’ai continué ma route.

J’étais à présent dans un parc, du genre de ceux que les gens riches aiment bien avoir chez eux pour frimer devant les invités et faire la bamboche au clair de lune, et j’ai profité de l’occasion pour m’assoir sur un banc qui tendait ses petites planches moussues à mon fessier douloureux (oui, je ne vous l’ai pas dit et aurais préféré ne pas avoir le faire pour éviter de passer pour un manche, mais j’avais glissé sur une pierre, étais parti en vol plané et m’étais rétamé lourdement sur le cul). Ledit banc étant invisible de la propriété, j’avais tout le loisir de m’allumer un petit cigare et savourer quelques bouffées avant de passer à la phase deux de mon opération commando. J’avais aussi tout le loisir d’appeler Maël et Titus pour leur demander de me rejoindre au plus vite, car j’avais maintenant la très nette et désagréable impression que mon entreprise, plus que téméraire, flirtait avec le suicidaire. Autrement dit j’étais venu, avec pour seul et unique partenaire Manu, un 6.35 à sept coups auquel j’attachais une grande valeur sentimentale mais dont l’efficacité létale était pour le moins discutable, me jeter dans la gueule du loup. Manu, un Le Français dans sa version de base «~modèle de poche~», avait appartenu à mon grand-père Philibert, communiste et résistant de la première heure, qui s’en était servi à plusieurs reprises pour faire sauter le caisson à des traîtres ou des nazis (eh oui, il n’avait pas toujours été le vieillard paisible et amateur de pêche à la mouche qu’il était en fin de carrière). Je le voyais souvent chez lui étant enfant, dans la vitrine où il reposait, et je passais des heures à le contempler. Un jour, mon grand-père m’a raconté l’histoire de cet objet, permis de le toucher, le prendre en main, expliqué son fonctionnement, et finalement autorisé à faire mes premiers cartons dans la cour. Je devais avoir une douzaine d’années, et n’étais alors qu’un bambin innocent aux pupilles dilatées par la curiosité. Plus tard, peu de temps avant sa mort, il me l’a officiellement légué, évènement qui figure encore aujourd’hui parmi les plus marquants de mon existence. Naturellement, pour les missions à haut risque nécessitant une puissance de feu supérieure, je disposais de mon arme de service, bien sûr, laquelle était malheureusement restée à la maison, au même titre que mon Desert Eagle 44 magnum dont je commençais à regretter amèrement la présence.

Le banc était confortable, aussi doux à mon coccyx qu’un tapis de mousse fraîche et parfumée dans laquelle s’ébrouent de minuscules insectes aux couleurs vives. C’est donc en toute décontraction que j’ai allumé un Gurkha Ghost Shadow, cigare dominicain emballé dans une feuille d’Arapiraca brésilien sombre et huileuse, et pris le temps de me détendre un peu avant d’appeler Maël. La nuit commençait à tomber, lentement mais sûrement, sur mes frêles épaules de justicier masqué de l’ombre qui erre le balai à la main, tel un super-technicien de surface, dans les ruelles sombres et humides de la cité tentaculaire pour faire régner un semblant de propreté dans un monde voué à la crasse et la putréfaction. Supertechno à Goddam City, jeu de mots foireux pour un monde foireux. Même si cette fois il ne s’agissait pas d’une ruelle sombre et humide, infestée de rats et cafards comme des poissons dans l’eau parmi ordures et immondices, mais d’un repaire perdu au fond des bois dans lequel les Forces du Mal semblaient bien avoir établi leur QG.

J’ai dit : Maël ?

\textsc{Maël} : Oui.

\textsc{Moi} : Salut, c’est Djef.

\textsc{Lui} : Salut, Djef.

\textsc{Moi} : Je suis chez Riqueti.

\textsc{Lui} : C’est cool.

\textsc{Moi} : Pas trop, non. L’endroit est une vraie forteresse. J’ai dû passer par derrière pour m’introduire.

\textsc{Lui} : Qu’est-ce que je peux faire pour toi ?

\textsc{Moi} : T’es dispo, là ?

\textsc{Lui} : Négatif. Ma fille a quarante de fièvre et ma femme n’est pas là. Pourquoi ?

\textsc{Moi} : Eh ben, à dire la vérité, je me sens un peu seul, ici. Je me suis dit que tu pourrais venir me tenir compagnie.

\textsc{Lui} : Désolé, vieux, mais ça ne va pas être possible. Barre-toi si tu sens que c’est trop dangereux, on y retournera en force une autre fois.

\textsc{Moi} : Laisse tomber, je vais me démerder.

\textsc{Lui} : T’as essayé Greg ?

Pour info, Grégoire (Lussier) faisait aussi partie de notre joyeuse bande de justiciers d’opérette. Diplômé de l’Essec, analyste M\&A chez Reckless \& Knot (banque surnommée «~La Machine à Laver~» parce que les affaires qu’elle traitait n’entraient pas toujours dans le cadre de la plus stricte légalité), il avait, le jour où cette dernière s’était retrouvée impliquée jusqu’aux ouïes dans ce qu’il est convenu d’appeler un scandale financier de grande ampleur, pris conscience qu’il faisait un boulot de merde dans un monde de merde et décidé de tout plaquer pour voler au secours de la veuve et l’orphelin, surtout quand la veuve était jeune et jolie, riche de préférence, et l’orphelin héritier d’une immense fortune comme Bruce Wayne, Largo Winch ou Steve Jobs. Profondément humaniste, il avait soudain compris qu’il ne pouvait pas continuer à se goinfrer éhontément pendant que d’autres crevaient la dalle et se faisaient piétiner par des gens comme lui, des requins de la finance qui nageaient en cercle autour des navires en perdition, attendant qu’ils sombrent pour dévorer leurs occupants. Il avait donc, après avoir été licencié avec de copieuses indemnités, ouvert un cabinet de détectives privés, lequel lui servait également de couverture pour des opérations plus radicales. Privé est un métier à risque, on ne le répétera jamais assez. La preuve en est qu’il se trouvait présentement dans une chambre d’hôpital, après avoir été blessé par balle dans l’exercice de ses fonctions.

\textsc{Moi} : Il est à l’hosto.

\textsc{Maël} : Ah bon ? Comment se fait-il que je ne sois au courant de rien ?

\textsc{Moi} : Je viens seulement de l’apprendre, figure-toi. Apparemment c’est arrivé il y a deux jours, pendant qu’il enquêtait sur une affaire d’adultère à priori tout ce qu’il y a de banal. Sauf que l’amant n’est pas du genre commode et lui a tiré dessus alors qu’il allait procéder au constat.

\textsc{Lui} : C’est grave ?

\textsc{Moi} : Il a été touché au bras, rien de grave à priori. Greg savait que le type avait un casier long comme le bras, qu’il trafiquait avec des gens peu recommandables et avait l’habitude de se trimballer avec un calibre sur lui. Il s’était équipé en conséquence et a finalement réussi à flinguer ce connard après une course poursuite dans les escaliers.

\textsc{Lui} : Il est mort ?

\textsc{Moi} : Non, pas tout à fait, il s’en tire avec un genou en vrac et la rate explosée. Les flics attendent qu’il se rétablisse pour lui coller une inculpation de tentative de meurtre sur le dos.

\textsc{Lui} : Ben merde, alors !

\textsc{Moi} : Comme tu dis. Le client, un bijoutier plein aux as de la rue Edgar Morel, est ravi, et Greg va empocher une rallonge substantielle en remerciement de ses bons et loyaux services. Il paraît même que le gars lui aurait demandé de buter aussi sa femme dans la foulée, en faisant passer ça pour un accident. Greg lui a répondu sèchement que le métier de tueur à gage exigeait des compétentes qu’il n’avait pas, et qu’il préférait faire comme s’il n’avait rien entendu.

\textsc{Maël} : Le fait est qu’il n’a jamais tué personne pour de l’argent.

\textsc{Moi} : C’est exact.

\textsc{Lui} : Et Sam ?

\textsc{Moi} : Il est en vacances à Zanzibar, tous frais payés dans une villa 5 étoiles avec vue sur mer, piscine, plage privée, lit king size, clim, minibar, coffre-fort et jardin luxuriant. J’ai cru comprendre qu’il a rendu quelques petits services à un homme d’affaires tanzanien qui a su se montrer particulièrement reconnaissant, preuve que tes frères africains ne sont pas tous des sauvages qui vénèrent des idoles païennes et coupe les couilles des albinos pour s’en faire des grigris.

(NB : je reviendrai plus tard~-- ou pas, suivant mon humeur et les besoins de l’action~-- sur le cas du capitaine Samuel Girard, appartenant lui aussi à notre joyeuse petite bande de justiciers d’opérette, pour la plupart dotés d’une moralité douteuse, certes, mais terriblement humains et sympathiques, même si parfois amenés à commettre des actes plus ou moins répréhensibles aux yeux d’une justice hélas trop souvent aveugle, molle du genou, ou tout simplement dépassée par les événements, tant les affaires à traiter sont nombreuses et les réductions de personnel incompatibles avec le serein exercice d’une profession toujours plus corsetée et exigeante tant sur le plan ethnique qu’éthique et tactique.)

\textsc{Lui} : Il y en a qui ont de la chance.

\textsc{Moi} : Tu l’as dit, bouffi. C’est pas des honnêtes travailleurs comme toi et moi qui auront un jour les moyens d’aller se dorer la pilule sous les Tropiques ! Et pourtant c’est pas faute de se décarcasser, tu seras d’accord avec moi.

\textsc{Maël} : Ouais. Reste plus que Titus, dans ce cas.

\textsc{Moi} : Ouais, je vais l’appeler, mais il va encore se faire sonner les cloches par sa femme.

\textsc{Maël} : Je te souhaite bonne chance. Cela dit, tu devrais pouvoir t’en sortir sans trop de difficulté avec le logiciel que je t’ai installé.

\textsc{Moi} : J’y compte bien.

J’ai raccroché et tiré quelques bouffées de cigare en regardant la nuit dégouliner sur le paysage comme du chocolat fondu sur une glace à la vanille (la comparaison est osée et passablement vide de sens, mais bon, je me suis dit que ça mettrait un peu de poésie pâtissière dans un monde aussi sec qu’une vieille tranche de jambon racornie sur le bord d’une assiette ébréchée).

Puis j’ai appelé Titus, sachant pertinemment que j’allais encore déchaîner la foudre dans le ciel sans nuage (ou presque) de sa vie de famille exemplaire.

J’ai dit : Titus ?

\textsc{Titus} : Oui.

\textsc{Moi} : Salut, c’est Djef.

\textsc{Lui} : Salut, Djef.

\textsc{Moi} : Je suis chez Riqueti.

\textsc{Lui} : C’est cool.

\textsc{Moi} : Pas trop, non. L’endroit est une vraie forteresse. J’ai dû passer par derrière pour m’introduire.

\textsc{Lui} : Qu’est-ce que je peux faire pour toi ?

\textsc{Moi} : T’es dispo, là ?

\textsc{Lui} : J’allais allumer le barbecue.

\textsc{Moi} : Laisse tomber le barbecue, vieux, je suis dans la merde.

\textsc{Lui} : T’as pensé à ma femme ?

\textsc{Moi} : J’y pense jour et nuit.

\textsc{Lui} : Et mes gosses, tu y as pensé à mes gosses ?

\textsc{Moi} : Et comment, que j’y ai pensé ! Je n’y penserais pas davantage s’il s’agissait de mes propres enfants. Cela dit, pour une fois, ça ne leur fera pas de mal de bouffer autre chose que des saucisses et des merguez.

\textsc{Lui} : J’ai prévu des steaks hachés et du boudin créole.

\textsc{Moi} : Tu sais quoi ?

\textsc{Lui} : Non. Quoi ?

\textsc{Moi} : Je me suis pointé ici ne pensant que ça allait être une partie de plaisir, mais je me rends compte que ça risque d’être un peu plus compliqué que prévu. J’ai vraiment besoin de toi.

\textsc{Lui} : Bérénice va être furax.

\textsc{Moi} : Elle comprendra si tu lui dis que c’est une question de vie ou de mort.

\textsc{Lui} : Mes gosses vont être furax.

\textsc{Moi} : Tu sais quoi ?

\textsc{Lui} : Non. Quoi ?

\textsc{Moi} : Tu allumes le barbecue et tu viens me rejoindre après. Bérénice n’aura plus qu’à faire cuire la barbaque, c’est quand même pas sorcier de poser trois saucisses sur une grille.

\textsc{Lui} : Je te trouve bien arrogant.

\textsc{Moi}, estimant que le terme «~arrogant~» n’était peut-être pas le plus approprié mais jugeant préférable de n’en pas faire état : Tu viens ou tu viens pas ?

\textsc{Lui}, après un temps d’hésitation, sans doute dû au fait qu’il avait d’abord songé à m’insulter copieusement avant de se raviser : Comme si j’avais le choix.

\textsc{Moi} : Non, en effet.

\textsc{Lui} : Quoi, non ?

\textsc{Moi} : Non, en effet, tu n’as pas le choix.

\textsc{Lui} : Bien sûr que non, que je ne l’ai pas.

\textsc{Moi} : Ben non.

\textsc{Lui}, bougon : Quand même, tu pourrais prévenir avant.

\textsc{Moi} : Je ne savais pas que j’allais venir.

\textsc{Lui} : Ben voyons.

\textsc{Moi} : Si, je t’assure. J’effectuais une petite filature de routine, pépère, juste comme ça au cas où, et je m’apprêtais à remballer quand j’ai vu Riqueti monter dans la bagnole. Je pensais même passer boire l’apéro chez toi, figure-toi.

\textsc{Lui} : Ah bon ?

\textsc{Moi} : Oui, j’avoue que l’idée m’a traversé l’esprit.

\textsc{Lui} : Ben t’aurais pu me prévenir.

\textsc{Moi} : J’allais le faire, figure-toi. Sauf que quand j’ai vu Riqueti monter dans la bagnole, mon sang n’a fait qu’un tour et j’ai décidé de le suivre.

\textsc{Lui} : T’as des idées bizarres, parfois.

\textsc{Moi}, tirant sur mon cigare les fesses au frais dans la mousse : Ouais, je sais.

\textsc{Lui}, retrouvant un semblant de calme après la bourrasque d’énervement qui venait de le traverser (Titus avait bien des défauts, comme celui de s’emporter facilement, mais pas celui d’être rancunier, ce qui signifie qu’il pouvait vous hacher menu sous le coup de l’émotion, puis le regretter aussitôt et passer le restant de son existence à recoller les morceaux à la pince à épiler) : Et t’es où, au juste ?

\textsc{Moi} : Au château de Marmont. Tu vois où c’est ?

\textsc{Lui} : Non.

\textsc{Moi} : Tu vois Draguillon ?

\textsc{Lui} : Vaguement, oui.

\textsc{Moi} : T’as un GPS, non ?

\textsc{Lui} : Ouais, bien sûr.

\textsc{Moi} : Eh ben t’as qu’à le mettre.

\textsc{Lui} : Et je tape quoi ?

\textsc{Moi} : Je viens de te le dire : château de Marmont. C’est pas vraiment un château, du reste. Plutôt une grosse baraque bourgeoise en forme de château.

\textsc{Lui} : Ouais. Un château, quoi.

\textsc{Moi} : Si on veut.

\textsc{Lui} : C’est loin ?

\textsc{Moi} : Non, tout près de Draguillon.

\textsc{Lui} : Et je te retrouve où ?

\textsc{Moi} : Appelle-moi quand t’arrives, je te donnerai la marche à suivre.

\textsc{Lui} : Okay, je me mets en route. Essaie de te tenir à carreau en attendant.

\textsc{Moi} : Je vais faire le maximum.

Il est allé voir sa femme qui s’affairait en cuisine à préparer une salade composée à base de riz, tomate, oignon, œuf dur et poivron, s’est approché d’elle aussi doucement que sa stature imposante le lui permettait, puis lui a glissé dans le creux de l’oreille, la voix chargée d’appréhension : Chérie ?

\textsc{Bérénice}, sans lever la tête de son saladier : Oui ?

\textsc{Titus} : Je suis désolé, mais…

\textsc{Elle}, tournant vers lui un visage déjà empourpré par la colère : Quoi encore ?

\textsc{Lui} : Djef vient de m’appeler.

\textsc{Elle} : Et alors ?

\textsc{Lui} : Il est au château de Marmont.

\textsc{Elle} : Il a gagné au loto ?

\textsc{Lui} : Non. C’est juste que…

\textsc{Elle} : Que quoi, mon chéri ?

\textsc{Lui} : Ben… il a des ennuis, à ce que j’ai cru comprendre.

\textsc{Elle}, narquoise : Non ? C’est vrai ?

\textsc{Lui} : Il semblerait, oui.

\textsc{Elle}, d’une voix d’une braise qui couve sous la cendre, prête à s’enflammer au moindre courant d’air : Rappelle-moi, tu es marié avec qui ? Avec Djef ou avec moi ?

\textsc{Lui}, muscles tendus, prêt à prendre la fuite si une fourchette ou un rouleau à pâtisserie se dirigeait vers lui à une vitesse anormalement élevée : Avec toi, chérie, tu le sais bien.

\textsc{Elle} : Et tu préfères passer la soirée qui ? Avec Djef ou tes enfants et moi ?

\textsc{Lui} : Les enfants et toi, chérie, tu le sais bien. C’est juste que Djef se trouve dans une situation assez délicate, et…

\textsc{Elle}, tailladant une pauvre tomate innocente à grands coups de couteau de boucher : Djef est toujours dans une situation délicate, vingt-quatre heures sur vingt-quatre, c’est sa spécialité de se fourrer dans des situations délicates ! Et qui est-ce qu’il appelle ensuite ? Je te le donne en mille ? Toi, toujours toi, et toi tu rappliques à chaque fois qu’il te sonne ! Ne me dis pas qu’il n’a pas d’autres amis que toi, des amis qui ne seraient pas obligés de planter femme et enfants à chaque fois que Djef appelle parce qu’il est dans une situation délicate ! Y en a marre, de Djef ! Il ferait mieux de se trouver une femme et faire des gosses, au lieu de se fourrer dans des situations délicates et faire chier le monde à tout bout de champ pour qu’on l’aide à s’en sortir ! À croire qu’il le fait exprès !

\textsc{Titus}, écaillant un œuf avec ses gros doigts pour bien montrer que lui aussi était capable de s’occuper de tâches ménagères, ces petites choses de la vie quotidienne en apparence insignifiantes, humbles, sinon franchement casse-couille, mais qui constituent en réalité le fertile substrat d’une vie de couple réussie : Oui, c’est vrai, tu as raison. (Règle numéro un : ne jamais contrarier une femme qui tient un couteau de boucher.) Je crois savoir qu’il est inscrit sur des sites de rencontre pour célibataire de moins de cinquante ans.

\textsc{Bérénice} : Tu parles !

\textsc{Titus} : Si, je t’assure. C’est un garçon très sensible, tu le sais aussi bien que moi. L’ennui, c’est qu’il fait peur aux filles avec ses yeux de fou et les trucs bizarres qu’il raconte en permanence.

\textsc{Elle} : En effet, il n’est pas très net.

\textsc{Lui} : Complètement cinglé, tu veux dire ! (Règle numéro deux : faire semblant d’abonder dans son sens.)

\textsc{Elle}, un poil plus détendue : N’empêche qu’il suffit qu’il te siffle pour que tu rappliques ventre à terre !

\textsc{Titus} : Je te trouve un peu sévère, pour le coup. Je te rappelle qu’il m’a déjà sauvé plusieurs fois la mise, et qu’il n’hésiterait pas à le refaire si l’occasion se présentait.

\textsc{Elle} : Justement ! Ce type a un don pour attirer les embrouilles, et je n’ai pas envie que ça se termine mal un jour l’autre.

\textsc{Titus} : Oui, mon amour.

\textsc{Elle} : Te fiche pas de moi !

\textsc{Lui} : Je me fiche pas de toi, je sais très bien que c’est pas facile de vivre avec un flic.

\textsc{Elle} : Surtout un flic qui fait des heures supplémentaires. Non rémunérées, bien entendu.

\textsc{Lui} : Je sais. Mais dans ce métier, on ne va pas loin si on commence à compter ses heures. On n’est pas au théâtre, le crime ne fait jamais relâche.

Bon, je n’étais pas là pour entendre ça, mais je trouve que c’est plutôt pas mal dit pour un type qui n’est pas (objectivement, sans jugement de valeur) ce qui se fait de mieux en matière d’explication de texte et littérature comparée.

C’est à ce moment-là que Virginie, seize ans, qui était dans sa chambre en train de lire le Traité sur la Musique de saint Augustin (elle jouait du piano depuis l’âge de dix ans), a fait son apparition dans la cuisine. Elle avait entendu du bruit et tenait à s’enquérir de la situation.

\textsc{Virginie} : Tout va bien ?

\textsc{Bérénice} : Très bien. C’est juste que ton père va encore nous planter là pour aller faire les quatre cents coups avec son copain Djef.

\textsc{Titus} : Je suis vraiment désolé, mais il s’agit d’un cas de force majeure.

\textsc{Bérénice} : Question de vie ou de mort.

\textsc{Titus} : C’est ça. Je ne devrais sans doute pas vous le dire, mais il se trouve en ce moment-même chez un individu dont on ne connaît pas encore exactement le niveau de dangerosité. Il pensait s’en sortir tout seul, raison pour laquelle il n’avait pas jugé bon de me prévenir, mais il craint des complications et pense que nous ne serons pas trop de deux pour y faire face. C’est mon devoir de lui prêter main forte.

\textsc{Bérénice}, narquoise : Oui, bien sûr.

\textsc{Virginie} : Il est gentil, Djef.

\textsc{Bérénice} : Oui, très. Mais il est aussi très chiant.

\textsc{Titus}, consultant sa montre : Je suis d’accord avec vous sur toute la ligne, les filles : il est très chiant et très gentil, et vous êtes mes deux princesses adorées sans qui la vie n’aurait aucun intérêt pour moi.

\textsc{Bérénice} : Mais tu dois partir tout de suite pour voler au secours de ton ami Djef, c’est ça ?

\textsc{Titus} : C’est ça.

Son téléphone a sonné.

Et devinez qui c’était ?

C’était moi, bien sûr, l’empêcheur de tourner en rond, le briseur de paix dans les ménages : Titus ?

\textsc{Lui} : Quoi encore ?

\textsc{Moi} : T’es où ?

\textsc{Lui} : Chez moi.

\textsc{Moi} : T’es pas encore parti ?

\textsc{Lui} : Si, presque.

\textsc{Moi} : Qu’est-ce que tu fous, bordel ?

\textsc{Lui} : J’ai une femme et des enfants, au cas où tu ne l’aurais pas remarqué.

\textsc{Moi} : Contrairement à moi qui suis libre comme l’air, je sais, merci. Désolé de te déranger en plein conseil de famille, mais il se passe des choses, ici.

\textsc{Lui} : Ah bon ? Rien de grave, j’espère.

\textsc{Moi} : Hélas non. Mauvaise nouvelle pour vous : je suis encore en vie. Par contre, Riqueti et son chauffeur viennent de s’absenter. Je me suis dit qu’on pourrait en profiter pour fouiller la baraque.

\textsc{Lui} : C’est une bonne idée.

\textsc{Moi} : Merci. Fais vite, je t’attends.

\textsc{Lui} : Je suis là dans une demi-heure.

\textsc{Moi} : Okay. Je vais explorer un peu les alentours en attendant.

J’ai raccroché et commencé à fureter dans le secteur.

Le parc était cossu, les arbres centenaires, la piscine olympique, les dépendances nombreuses. Il y avait même une petite chapelle dans laquelle Monseigneur devait donner des messes privées pour les notables du coin. Des messes noires, peut-être, durant lesquelles, coiffé de sa plus belle mitre, il sacrifiait des nourrissons, chiait sur des rondelles de pain azyme, pissait dans des calices, buvait du sang humain et s’accouplait avec des animaux, le tout pour s’attirer les bonnes grâces de Lucifer et accéder à l’immortalité. Peut-être même que c’était à cet endroit précis que se tenaient les sessions extraordinaires de la League of the Unknown Ribbon, sinistre association de malfaiteurs dont les ramifications souterraines s’étendaient bien au-delà de l’Hexagone.

Ladite chapelle était fermée à clé.

Qu’à cela ne tienne, je disposais de tout le matériel nécessaire pour venir à bout de la serrure.

Quelques instants plus tard, donc, j’étais à l’intérieur de la chapelle. J’en ai fait rapidement le tour sans rien déceler de suspect. Je m’attendais au moins à trouver, je ne sais pas, moi, des crucifix à l’envers ou une statue de Lucifer avec des pieds de bouc, des ailes de chauve-souris et un braquemart comac poli par des années de dévotion tant orale que manuelle. Mais rien de tout ça, pas même un petit bocal avec un fœtus à l’intérieur ou une bible satanique écrite avec du sang de lépreux sur de la peau humaine rongée par la petite vérole. Pas la moindre bouteille de vin de messe remplie de pisse, effigie de la Vierge avec des nichons énormes, des lunettes de soleil et une carotte dans le cul. Je dois dire que j’attendais un peu mieux de Riqueti et sa bande de dégénérés de la LUR. Même si, pour être tout à fait exact, je ne disposais pour l’instant d’aucune preuve tangible de son appartenance à une telle organisation, raison pour laquelle (découvrir de telles preuves) je me trouvais ici.

Trois quarts d’heure environ après notre dernier échange téléphonique, j’ai reçu un appel de Titus me signalant qu’il venait d’arriver. Il avait repéré ma bagnole et s’était garé à côté.

Je lui ai dit que j’étais passé par derrière pour rentrer, mais que le chemin était un peu compliqué, pour ne pas dire digne d’un parcours du combattant dans une jungle infestée de serpents et truffée de mines antipersonnel, et que par conséquent, vu que les locataires étaient partis, j’allais le faire passer par la porte principale, ce qui nous permettrait de gagner du temps et lui éviterait de revenir chez lui en lambeaux et de se faire tabasser par sa chère et tendre. Grâce à mon passe de compète, l’opération ne prendrait que quelques secondes.

Ou minutes, peut-être, parce que la serrure s’est révélée un peu plus coriace que prévu.

J’étais en train de m’échiner dessus quand mon téléphone a sonné. J’ai laissé sonner.

J’ai fini par triompher de la serrure et laisser entrer un Titus qui commençait manifestement à trouver le temps long, d’autant qu’il avait promis à sa femme de faire aussi vite que possible. À ce rythme-là, il n’était pas près de retrouver la chaleur du foyer que je l’avais forcé à abandonner.

J’ai jeté un coup d’œil à mon téléphone : un O6 inconnu avait laissé une trace sur ma messagerie. La tentation de l’écouter était trop forte. C’était Zarina Brizzi, la fille que j’avais rencontré la veille au Narcisse Rose, et à qui j’avais, dans un moment d’égarement, commis l’erreur de refiler mon numéro de téléphone. Elle n’avait pas perdu de temps. Elle me disait qu’elle et sa sœur avaient passé une excellente soirée en notre compagnie (j’étais là-bas avec Zaahid Shirani), qu’elle-même, à titre personnel, avait particulièrement apprécié ma conversation, et qu’elle espérait qu’on pourrait se revoir dans les plus brefs délais. C’était d’autant plus étrange que je n’avais pas dû prononcer plus de dix ou quinze mots dans la soirée, et n’avais rien fait, à mon sens tout cas, pour susciter un tel engouement. Il faut croire que mon charme dévastateur s’exerçait à mon corps défendant. Cela dit, j’aurais mauvaise grâce de prétendre que Zarina, sa beauté plastique indéniable, son esprit d’une vivacité peu commune et son accent italien aussi doux et enveloppant qu’un manteau en plumes de marabout Yves Saint Laurent\nf{Yves Saint Laurent (1936--2008), couturier français né à Oran. Directeur artistique de Dior à 21 ans, il fonda sa propre maison en 1961. Il est notamment crédité d'avoir introduit le smoking féminin (1966) et popularisé le prêt-à-porter de luxe. \source{fr.wikipedia.org/wiki/Yves\_Saint\_Laurent\_(couturier)}} (élevé dans le respect du bien-être animal et des cycles biologiques, je parle du marabout), me laissaient de marbre.

\textsc{Titus} : Qui c’est ?

\textsc{Moi} : Une fille que j’ai rencontrée hier soir au Narcisse Rose.

\textsc{Lui} : Ah ah, les affaires reprennent, on dirait.

\textsc{Moi} : Ne dis pas de conneries !

\textsc{Lui} : Elle s’appelle comment, si ce n’est indiscret ?

\textsc{Moi} : Zarina.

\textsc{Lui} : C’est joli, Zarina.

\textsc{Moi} : Oui, bon, on s’en fout. Amène-toi, on a déjà perdu assez de temps.

\textsc{Lui} : Quand est-ce que tu me la présentes ?

\textsc{Moi} : T’as fini, oui !

\textsc{Lui} : Excuse-moi d’être content pour toi. En plus, je pense qu’une femme ne serait pas trop pour remettre un peu d’ordre dans ton existence.

\textsc{Moi} : On a échangé trois mots autour d’un verre, c’est peut-être un peu tôt pour publier les bans.

\textsc{Lui} : N’empêche que t’as l’air drôlement mordu.

\textsc{Moi} : Je ne vois pas ce qui te fait dire ça.

\textsc{Lui} : Je ne sais pas, la façon que t’as d’en parler avec des étoiles plein les yeux.

\textsc{Moi} : Tu te fous de moi ?

\textsc{Lui} : Non, c’est vrai, t’as des étoiles plein les yeux.

\textsc{Moi}, le fusillant du regard : Laisse tomber, tu veux.

\textsc{Lui}, pénible : Tu vas la revoir bientôt ?

\textsc{Moi} : Écoute, vieux, je t’aime bien, t’es comme un frère pour moi, mais si tu continues comme ça, c’est toi qui vas en voir, des étoiles !

\textsc{Lui} : Okay, ça va, pas la peine de s’énerver. On fait comment, alors ?

\textsc{Moi} : C’est pas compliqué : tu montes la garde pendant que je visite la maison.

\textsc{Lui} : Et s’ils reviennent ?

\textsc{Moi} : Tu me fais signe et on se barre en toute discrétion.

\textsc{Lui} : En toute discrétion ?

\textsc{Moi} : Oui, en toute discrétion. T’as pas oublié ton flingue, j’espère ?

\textsc{Lui}, sortant son Glock 17 et me l’agitant sous le nez : Tu me prends pour qui, pépère ? Bien sûr que non, que je l’ai pas oublié !

\textsc{Moi} : Parfait.

\textsc{Lui} : Sauf que je ne pense pas en avoir besoin si on se barre en toute discrétion, comme tu dis.

\textsc{Moi} : On ne sait jamais.

J’ai fait le tour de la bicoque, Titus sur mes talons, et rapidement trouvé ce que je cherchais : le moyen de m’introduire en toute illégalité dans un domicile qui n’était pas le mien. Ce moyen était une porte, et on y accédait en descendant quelques marches d’escalier, tellement étroites et abruptes qu’on était, à moins de chausser du 32, ce qui correspond peu ou prou à la pointure d’un gamin de six ans (pour info, je chaussais du 44 et Titus un bon 50), obligé de se déplacer en crabe, au risque de se niquer la cheville et se foutre la gueule par terre à tout moment. En d’autres termes, ladite porte était clairement celle de la cave. Celle-ci avait quand même la particularité d’être blindée, preuve que la cave en question devait recéler des trésors liquides avec lesquels le propriétaire des lieux n’avait aucune envie des mécréants se rincent la glotte. Monseigneur était fin gourmet, aimait arroser ses plats des meilleurs crus. Pas de problème, j’étais moi-même féru de gastronomie et de vins fins et veloutés. On dit souvent que c’est autour d’une bonne table que les grandes causes se négocient. Et j’ai envie de m’écrier : mais oui, bien sûr, c’est tellement vrai ! Si Riqueti, au lieu de m’envoyer ses tueurs, m’avait fait les honneurs de sa cave, je vous fiche mon billet que les hommes de goût que nous sommes auraient su se montrer suffisamment adultes pour balayer leurs dissensions d’un revers de la patte. Alors oui, c’est vrai, je n’étais pas venu pour ça (comme souvent, d’ailleurs, je ne savais pas très bien pourquoi j’étais venu, ce que je faisais là, il s’agissait tout au plus d’une vague idée, le pressentiment qu’une tragédie d’ampleur hellénistique était en train de se nouer dans l’ombre propice de quelque profond caveau), mais je vous prie de croire que si d’aventure tel ou tel précieux flacon s’avisait de me faire de l’œil, je n’allais certainement pas passer à côté en faisant comme si je n’avais rien vu. Merde, ce n’est pas parce que certains prennent les cochons de lait pour des sangliers albinos (croyez-le ou non, mais le cas s’est vu en forêt de Beaumont-le-Roger, dans l’Eure, bien triste histoire sur laquelle je n’aurais malheureusement pas le plaisir de m’appesantir davantage, étant donné que le présent ouvrage ne traite ni d’élevage ni de chasse) qu’il faut se laisser marcher sur les sabots ! En un mot comme en cent, quatre en l’occurrence : l’occasion fait le lardon.

J’ai sorti mon matos de cambrioleur, avec Titus dans mon dos qui commençait à manifester certains signes de nervosité (en dépit de sa stature de grand singe, et vous me connaissez assez pour savoir que je ne dis pas ça parce qu’il s’agit d’une personne de couleur, c’était quelqu’un d’assez nerveux, qui prenait sur lui en permanence pour ne pas tout exploser sur son passage, comme cette porte, par exemple, qu’il se serait fait une joie de défoncer à coups de poings si je n’avais pas été là pour le ramener à la raison), et aussitôt attaqué à la serrure avec tout le doigté dont j’étais capable, c’est-à-dire énormément.

Elle était coriace, mais moi aussi.

Après dix bonnes minutes de tripatouillage intensif, elle a rendu les armes.

Après la porte, l’escalier, toujours aussi étroit et d’autant plus casse-gueule que s’y ajoutait une bonne dose d’humidité supplémentaire, s’enfonçait dans les profondeurs de la terre. Par chance, un interrupteur situé en haut à droite permettait de faire toute la lumière sur la situation. En tant que chef des opérations, c’était à moi que revenait le privilège de passer le premier.

J’ai dit à Titus, qui maugréait derrière moi : Fais gaffe, ça glisse.

\textsc{Lui} : Merci, j’avais pas remarqué.

\textsc{Moi} : Je n’aime pas trop te sentir dans mon dos.

\textsc{Lui} : Je protège tes arrières.

\textsc{Moi} : Mouais. Et si tu glisses, je dégringole avec toi.

\textsc{Lui} : La confiance règne.

Quand vous avez un mec qui chausse du 50 et qui est obligé de se plier en quatre pour descendre un escalier aussi étroit qu’un trou de souris, et que ce mec se trouve dans votre dos suffisamment près pour que vous puissiez sentir son souffle vous rafraîchir la nuque, j’estime qu’on est en droit de se faire un brin de muguet (alternative florale à «~un peu de souci~»).

Environ trois mètres plus bas, comme il ne fallait pas être un génie pour s’en douter, se trouvait une cave qu’on aurait pu aisément transformer en boîte de nuit ou court de tennis. L’espace était occupé par des tonneaux vides, certains étant retournés pour faire office de tables et permettre ainsi la dégustation in situ des crus les plus prometteurs, et les murs couverts de casiers de bouteilles dont la totalité (je parle des bouteilles) devait bien avoisiner, à vue de nez, les deux ou trois mille, soit de quoi plonger en état de sidération tout amateur un peu sérieux.

J’ai dit à Titus, la voix brisée par l’émotion : Je crois que je vais en avoir pour un moment.

\textsc{Lui} : Comment ça ?

\textsc{Moi} : Tu ne crois quand même pas que je vais repartir sans jeter un œil là-dessus.

\textsc{Lui} : Jeter un œil ?

\textsc{Moi} : Oui, et prélever quelques échantillons de valeur que nous aurons le plaisir de partager ou revendre sur Internet pour arrondir nos fins de mois difficiles.

\textsc{Lui} : T’as vraiment aucune moralité !

\textsc{Moi} : Non, mais j’ai soif. Soif de vivre et déguster les meilleurs vins, que ce ne soit pas toujours les mêmes qui en profitent.

\textsc{Lui} : Bérénice adore le Bordeaux.

\textsc{Moi} : Je crois qu’on devrait pouvoir trouver ce qu’il faut. Tu sais quoi ?

\textsc{Lui} : Non ?

\textsc{Moi} : Changement de plan.

\textsc{Lui} : Pourquoi, il y en avait un ?

\textsc{Moi} : Plus ou moins. Tu sais ce que je pense ?

\textsc{Lui} : Non, et j’aimerais autant ne pas le savoir.

\textsc{Moi} : Je pense qu’on se complique bien trop l’existence.

\textsc{Lui} : Ah bon.

Je lui ai tendu les clés de ma caisse : Oui. Tu vas remonter, aller chercher ma caisse et venir te garer carrément ici, le cul à la porte de la cave.

\textsc{Lui} : Je te rappelle qu’on n’est pas chez nous.

\textsc{Moi} : On s’en fout. Il faut savoir se faire plaisir, de temps à autre.

\textsc{Lui} : Et si les autres rappliquent ?

\textsc{Moi}, avec des intonations dans la voix que je ne me connaissais pas, rendu fou par la vue des bouteilles qui scintillaient devant mes yeux exorbités comme autant d’étoiles qu’il suffisait de tendre le bras pour atteindre : On va faire aussi vite que possible.

\textsc{Lui} : D’accord, mais s’ils reviennent et qu’ils nous trouvent en train de piquer leurs bouteilles ?

\textsc{Moi} : On avisera. Pour info, j’ai appris récemment que Riqueti est originaire de Spezzano, le siège du clan Terracciano.

\textsc{Lui} : Connais pas.

\textsc{Moi} : Faudrait voir à te tenir un peu courant, ma vieille. Le clan Terracciano, c’est la mafia calabraise. T’as entendu parler de la mafia calabraise, quand même ?

\textsc{Lui} : Oui, bien sûr.

\textsc{Moi} : Alors tu sais que c’est pas des tendres. Si j’en crois mes informateurs, Riqueti bosse pour eux et se sert de son influence pour renforcer celle de l’organisation.

\textsc{Lui} : Tu crois ?

\textsc{Moi} : J’en suis sûr. Si je te dis Piero Bottaro ?

\textsc{Lui} : Je te réponds inconnu au bataillon.

\textsc{Moi} : Et Santo Termine ?

\textsc{Lui} : Qui ça ?

\textsc{Moi} : Santo Termine. Lui et Bottaro sont des pontes de la ’Ndrangheta\nf{La ’Ndrangheta, organisation criminelle originaire de Calabre (Italie du Sud), est considérée par Europol et l’Office des Nations unies contre la drogue et le crime comme la mafia la plus puissante et la plus internationale du monde. Son chiffre d’affaires annuel est estimé à plus de 50 milliards d’euros, essentiellement issu du trafic de cocaïne. \source{fr.wikipedia.org/wiki/’Ndrangheta}}, elle-même en cheville avec la mafia albanaise. Termine est spécialisé dans le trafic d’œuvres d’art, les paris clandestins et l’immobilier, Bottaro dans la fausse monnaie, les comptes offshore et la fraude aux subventions européennes. J’ai mis Maël sur le coup et j’ai la preuve, photos à l’appui, que Riqueti, Termine et Bottaro se connaissent depuis des lustres. Donc Riqueti est une ordure et on ne va pas se gêner pour lui piquer son pinard !

\textsc{Titus} : Et s’il rapplique avec son garde du corps ?

J’ai peut-être pas l’air, comme ça, mais je suis le genre de type très malin qui a toujours qui a toujours une longueur d’avance sur les évènements, un joker dans le fond de sa poche. Je ne dirais pas que je suis capable de lire l’avenir dans le fond d’un verre de Chambertin, mais pas loin.

\textsc{Moi}, sortant deux chiffons noirs des poches de mon pantalon : S’ils rappliquent, on enfile ça et on leur tombe sur le dos par derrière. On les maîtrise, on les attache dans un coin, et on finit tranquillement ce qu’on a à faire.

\textsc{Lui} : C’est quoi ?

\textsc{Moi}, triomphant d’ingéniosité : Des cagoules, mon vieux ! Avec ça sur la tronche, ils ne sauront jamais qui a fait le coup.

\textsc{Lui} : On aurait peut-être pu les mettre tout de suite.

\textsc{Moi} : Pourquoi faire ?

\textsc{Lui} : Il y a peut-être des caméras de surveillance qui nous filment depuis le début.

\textsc{Moi} : Et tu crois sans doute que je n’y ai pas pensé.

\textsc{Lui} : Je sais pas. Tu y as pensé ?

\textsc{Moi} : Je croyais que tu me connaissais un peu mieux, depuis tout ce temps. Bien sûr, que j’y ai pensé !

J’ai sorti mon téléphone et lui ai mis sous le nez : Tu sais ce que c’est, ça ?

Lui, levant les yeux au ciel : Oui, un téléphone.

\textsc{Moi} : Exact.

J’ai tripoté le téléphone pendant quelques secondes et lui ai à nouveau collé sous le nez : Et ça, tu sais ce que c’est ?

\textsc{Lui}, jetant un œil blasé sur l’objet : Euh…. non, c’est quoi ?

\textsc{Moi} : Le progrès, mon vieux. Avec ce programme dernier cri, je peux scanner les systèmes de surveillance et les désactiver. C’est précisément ce que j’ai fait pendant que tu discutais avec ta femme et prenais tout ton temps pour te ramener ici.

\textsc{Lui}, interloqué : Merde, alors !

\textsc{Moi}, essayant de ne rien laisser paraître de l’autosatisfaction qui m’agitait intérieurement : Ouais. Des caméras, il y en a effectivement un peu partout ici. Mais grâce à ce petit bijou de technologie, j’ai pu localiser le logiciel de surveillance et le pirater avec une facilité déconcertante dès que Riqueti s’est absenté. Même chose pour les alarmes censées prévenir les flics en cas d’intrusion. Autrement dit, on peut se balader dans le périmètre en toute tranquillité. Je ne te cache pas que c’est Maël qui m’a filé le tuyau, installé le programme et indiqué la marche à suivre.

\textsc{Lui} : Impressionnant !

\textsc{Moi} : Ouais, c’est assez fascinant. Plus on essaye de se protéger et plus on s’expose, en quelque sorte. Allez, va chercher la caisse et viens te garer aussi près que possible pour qu’on n’ait pas trop à se fatiguer en déménageant le butin.

Une heure plus tard, le coffre de la Kangoo était plein à craquer de tout ce que j’avais pu trouver de plus rare et prestigieux dans la cave de Riqueti (ne vous inquiétez pas, j’aurai l’occasion d’y revenir).

Je ne vous l’ai peut-être pas dit, mais ma caisse, même si elle n’avait pas grand-chose à voir avec l’Aston DB5 de ce crétin de James Bond ou le Tumbler de cet autre abruti notoire de Bruce Wayne, était quand même équipée de gadgets très utiles pour sauver ses fesses en cas de coup dur. Par exemple, un jeu de plaques minéralogiques rotatives me permettait de dissimuler ma véritable identité en un battement de cils. Ce tour de force s’effectuait grâce à un interrupteur habilement dissimulé parmi les commandes du tableau de bord, et on ne pouvait espérer détecter la supercherie qu’en se livrant à un examen approfondi des plaques en question. C’était Nathan Lussier (un des nombreux frères de Greg), garagiste de son état, féru de nouvelles technologies et de nazi porn (Train spécial pour Hitler, Ilsa la louve des SS, Camp N°7, Le Lac des morts-vivants, autant de chefs-d’œuvre absolus réservés à un public averti), en plus du bel canto (fan de Donizetti, il écoutait Lucia di Lammermoor en boucle en jouant de la pompe à vide et de la clé de 12) et de la chasse à l’arc (c’était son côté comte Zaroff, sauf que lui utilisait un arc à poulies Martin Cougar Vintage MT-6 en fibre de carbone, le même que Stallone dans John Rambo, quatrième volet de la saga avec un John Rambo qui commence sérieusement à piquer du nez dans sa soupe), qui avait procédé à l’installation avec tout le génie dont il était capable. L’optimisation (la plupart du temps, reconnaissons-le, parfaitement illégale) de véhicules ordinaires était sa grande passion. Par exemple, le moteur à essence 1,6 16v de ma fidèle Kangoo développait à l’origine une petite centaine de poneys. C’était bien suffisant pour doubler un poids lourd et effectuer ses livraisons en temps et en heure, mais totalement ridicule pour espérer se livrer à des courses-poursuites délirantes dans des conditions extrêmes telles que centres-villes aux terrasses bondées et trottoirs encombrés de gens en fauteuil roulant et de mères de familles avec des landaus, autoroutes à contresens ou routes de montagne sinueuses bordées de falaises d’un côté et de précipices de l’autre. Eh bien, croyez-le ou non, mais après être passé entre les mains magiques de Nathan, le bloc pouvait maintenant se prévaloir d’un troupeau de cent cinquante mustangs gonflés à bloc qui piaffaient d’impatience d’exploser les chronos. Grâce à Nathan, l’utilitaire léger était devenu une authentique bombe de l’asphalte. Il avait bien sûr fallu procéder aux aménagements nécessaires en termes de fiabilité et de sécurité (je ne plaisante pas avec ça), mais j’avais insisté pour que rien ne se fasse au détriment de la discrétion exemplaire qui avait toujours été ma règle de conduite : pas d’échappement tonitruant, de roues de camion, aileron de requin et autres prises d’air intempestives. Pour tout le monde, j’étais et devais rester le parfait brave type qui roulait sans accroc, respectait scrupuleusement les limitations de vitesse et les panneaux de signalisation, n’oubliait jamais d’attacher sa ceinture, mettre son clignotant et vérifier la pression de ses pneus avant de partir en vacances (chose qu’il ne faisait bien évidemment jamais vu que son compte en banque était dans le rouge dès le 13 du mois).

Histoire de ne pas prendre le risque de se retrouver coincés comme des rats si l’évêque du Sanctuaire de Ddarr et son gorille des montagnes revenaient à l’improviste, j’ai suggéré à Titus d’aller remettre la Kangoo à sa place, pas trop près pour ne pas attirer l’attention des fouineurs éventuels, mais pas trop loin non plus pour qu’on ne soit pas obligés de se taper des kilomètres au pas de charge pour s’exfiltrer de la zone. Mes affinités avec le sport se limitaient à quelques rares disciplines, et la course à pied n’en faisait pas partie.

Telle était la première partie de sa mission, qu’il avait acceptée d’assez mauvaise grâce, estimant sans doute qu’elle n’était pas à la hauteur de ses capacités. D’autant que je lui avais, avec une insistance de nature à taper sur les nerfs du plus flegmatique des interlocuteurs, intimé l’ordre de rouler sur des œufs pour ne pas risquer d’esquinter mon précieux chargement.

La seconde (partie de sa mission), tout aussi périlleuse, était la suivante : rester en haut, trouver un emplacement stratégique, surveiller activement les alentours et me prévenir aussitôt en cas de danger. J’avais une totale confiance en Titus, mais vous savez ce que c’est, avec le stress, on se met facilement à radoter et ergoter sur des sujets mineurs. Et du stress, en dépit de l’excitation qui m’habitait en pensant à la cargaison de ma Kangoo et autres objets de valeur que je ne manquerais pas de dénicher en fouillant la maison (et de voler aussi, oui, je savais que c’était mal, au moins sur le plan de la morale chrétienne, mais les arcanes de la notion de propriété échappaient encore en grande partie à mon entendement, même si je parvenais quand même, sans trop de difficulté, à me mettre dans la peau de celui qui se fait dévaliser et en éprouve un certain ressentiment), j’en étais quand même porteur d’une quantité respectable.

Je m’apprêtais à gagner le rez-de-chaussée quand quelque chose a attiré mon attention. C’était à peine audible, mais, en tendant l’oreille, on pouvait distinguer des bruits bizarres s’apparentant à des gémissements ou une respiration étouffée, un peu comme si quelqu’un était en train de scier du bois dans une pièce adjacente avec une scie en mousse, ou encore de s’échiner à monter une mayonnaise avec une brosse à dents. Je me suis dirigé vers l’endroit d’où semblaient provenir les bruits en question, à savoir un des murs les plus glaireux et répugnants qu’il m’ait été donné de voir (et je vous prie de croire que j’en ai vu, des murs glaireux et répugnants, au cours de ma longue carrière de fonctionnaire véreux), contre lequel, au prix d’un effort titanesque, j’ai néanmoins réussi à coller une oreille. J’ai longtemps hésité entre la droite et la gauche, pour finalement choisir la droite, laquelle m’est apparue sur le moment, pour des raisons que je ne saurais décrire avec précision, comme la plus performante et la moins fragile des deux. Il faut savoir que mes oreilles et moi étions nés le même jour, à la même heure et au même endroit, de sorte que nous formions une équipe très soudée, comme les cinq doigts de la main qui eux-mêmes se connaissent depuis toujours et entretiennent une relation très étroite, quasi fusionnelle, leur permettant d’accomplir des prouesses techniques qui laissent sans voix la plupart des habitants de cette planète, et ne manqueraient certainement pas de clouer le bec à la plupart des touristes extraterrestres de passage sur terre.

Et la réponse est : oui, il y avait quelqu’un, à n’en pas douter.

En y regardant de plus près (faisant fi de l’odeur de marée basse qui s’en dégageait, mêlant poissons morts et vieilles touffes de varech pourri), je me suis rendu compte que ce mur n’en était pas vraiment un, mais plutôt une cloison enduite d’un revêtement lui donnant toutes les apparences du mur qu’il n’était pas.

En m’éclairant avec mon téléphone, j’ai fini par découvrir une protubérance suspecte sur laquelle j’ai décidé d’appuyer, certain qu’il s’agissait d’un mécanisme d’ouverture habilement dissimulé dans le décor.

Bingo ! La cloison a coulissé sur quelques dizaines de centimètres, pas de quoi laisser passer un troupeau de gnous lancés à pleine vitesse avec une meute de lions à ses trousses (plus quelques guépards et une escadrille de vautours en soutien aérien), mais assez pour qu’une personne affichant un tour de taille raisonnable (comme moi, par exemple) puisse se glisser à l’intérieur sans avoir à rentrer son ventre.

Non loin de là, j’ai avisé un interrupteur qui ne demandait qu’à être activé, ce que je me suis empressé de faire, n’éprouvant aucune jouissance particulière à rester dans le noir plus longtemps que nécessaire.

Et la lumière fut.

Et là, franchement, que s’écrier d’autre que (je m’en excuse d’avance auprès de toutes les âmes sensibles et éprises de justice divine que ces paroles pourraient choquer) : NOM DE DIEU DE BORDEL DE MERDE DE FILS DE PUTE D’ENCULÉ DE SA RACE !!!

Dans le fond de la pièce, il y avait ce qui ressemblait comme deux gouttes de sueur à une chaise électrique, dans une présentation à peine plus évoluée que celle qui avait servi à faire griller le jeune George Junius Stinney Jr.\nf{George Junius Stinney Jr. (1929--1944), 14 ans, fut le plus jeune condamné à mort exécuté aux États-Unis au \textsc{xx}e siècle. Son procès, expédié en une journée sans défense effective devant un jury exclusivement blanc, fut jugé inique dès l'époque. En 2014, un tribunal de Caroline du Sud annula sa condamnation et le déclara officiellement innocent. \source{fr.wikipedia.org/wiki/George\_Stinney}} (14 ans) le 16 juin 44 au pénitencier de Columbia. Soupçonné d’avoir sauvagement assassiné deux gamines parties cueillir des maypops dans la campagne ensoleillée du comté de Clarendon, en Caroline du Sud, George Stinney avait commis trois erreurs fatales : 1 : les deux gamines en question (Betty June Binnicker et Mary Emma Thames, respectivement âgées de 11 et 8 ans) étaient blanches ; 2 : il habitait près de chez elles et leur parlait de temps en temps ; et 3, la pire de toutes : il avait eu la mauvaise idée de naître noir, ce qui signifie qu’il a été «~interrogé~» par des flics blancs sans l’assistance d’un avocat, et qu’il s’est ensuite retrouvé devant une salle d’audience exclusivement composée de suprémacistes blancs surexcités à l’idée de bouffer du nègre. Les gens de couleur, privés de leurs droits civiques, étaient invités à rester tranquillement chez eux s’ils ne tenaient pas à se faire lyncher. De toute façon, ce n’était que partie remise, car ils finiraient tôt ou tard par croiser le Klan qui se ferait un plaisir de foutre le feu à leurs baraques et les pendre haut et court.

Saucissonné sur cette chaise, entièrement nu, un individu respirait bruyamment, difficilement, dans un état proche de l’inconscience, le corps couvert d’ecchymoses. Son état général, plus qu’alarmant, laissait présager le pire dans un très proche avenir. Cela dit, en faisant vite, on devait encore pouvoir le sauver.

Je me suis approché, dans l’idée de lui soutirer quelques informations sur son identité et les raisons de sa présence ici, et lui ai donné quelques petites tapes sur l’épaule pour tenter de le faire revenir à lui.

Il a poussé un grognement, assez flippant je dois dire, puis, au prix d’un effort manifestement surhumain, comme si elle avait été lestée de plomb, il a soulevé une paupière et posé sur moi un regard vitreux dans lequel j’ai vu ou cru voir passer, aussi furtive qu’un Lockheed Martin F-22 Raptor dans le ciel de Téhéran, une vague lueur d’espoir. Quand je dis «~soulevé une paupière~», j’entends soulevé de quelques centièmes de millimètres, ce qui ne lui permettait pas de se faire une idée très précise de la situation. Assez, cependant, pour comprendre que le visage sympathique et bienveillant qui se tenait en face de lui n’était pas celui de l’un ou l’autre de ses habituels tortionnaires.

Il a commencé à s’agiter un peu, un filet de bave au coin des lèvres, et sa mâchoire a produit une série de craquements suspects, attestant qu’il s’efforçait de former des mots pour exprimer son point de vue.

J’ai dit : Du calme, mon vieux, tout va bien.

C’est le genre de répartie qui ne veut pas dire grand-chose, sinon rien, mais qu’on se sent néanmoins obligé de prononcer quand on se retrouve seul dans une cave devant un type à poil attaché à une chaise électrique.

Un gargouillis est sorti de sa bouche, en même temps qu’un filet de bave ensanglantée.

J’en ai remis une couche : Ne vous inquiétez pas, je vais vous sortir de là.

Et c’est là que mon téléphone a sonné, que j’ai décroché, ne serait-ce que parce que ce sont des choses qui se font quand son téléphone sonne, et entendu la voix de Titus qui chuchotait, manifestement en proie à quelque chose qui sans être tout à fait de la panique n’en était pas très éloigné : Djef ?

\textsc{Moi} : Oui ?

\textsc{Lui} : Tu m’entends ?

\textsc{Moi} : Oui, je t’entends. Tu devineras jamais ce que j’ai trouvé ici.

\textsc{Lui} : Remonte en vitesse, les autres sont rentrés !

\textsc{Moi} : T’es sûr ?

\textsc{Lui} : Oui, je suis sûr ! La BM est là !

\textsc{Moi} : Et c’est maintenant que tu me préviens !

\textsc{Lui} : J’étais en train de couler un bronze dans un bosquet. Je les ai pas entendus rentrer.

\textsc{Moi} : Ils t’ont vu ?

\textsc{Lui} : Je crois pas, non.

\textsc{Moi} : OK, j’arrive !

\textsc{Lui} : Magne-toi !

J’ai dit au type sur la chaise : Je suis obligé de partir. Mais je vais bientôt revenir, ne vous en faites pas.

J’étais sur le point de faire demi-tour quand j’ai entendu une voix dans mon dos.

Elle s’exprimait avec un léger accent italien, et j’avais déjà eu l’occasion de l’entendre : Je peux savoir ce que vous faites chez moi, inspecteur ?

Je me suis retourné d’un coup, et retrouvé en face de Riqueti et son gorille à chapeau de cowboy et oreilles en chou-fleur. Je n’ai rien contre les cowboys, si ce n’est que la plupart sont quand même des grosses brutes sans cervelle qui ont piqué leur terre aux Indiens (mais vous allez me dire que c’est de bonne guerre, qu’après tout les Indiens, s’ils avaient été moins nuls et arriérés, attachés à des croyances stupides, ne se seraient peut-être pas fait mettre la misère et retrouvés entassés comme des clodos dans des réserves insalubres), mais je n’ai jamais pu encaisser le chou-fleur. Déjà, tout petit, mon crétin de père devait me tabasser jusqu’au sang pour m’en faire avaler une bouchée. En plus de ses oreilles indigestes, le gorille avait un nez tellement écrasé qu’il lui recouvrait presque la totalité du visage. C’était le genre de type qui foutait la trouille même de dos, et quand vous étiez derrière lui, vous imploriez le ciel pour qu’il ne se retourne pas.

Riqueti a dit : Doucement, inspecteur, pas de geste brusque. Fouillez-le, Niccolo.

Donc, le singe s’appelait Niccolo et répondait à son prénom, preuve que, contre toute apparence, il n’était pas totalement dépourvu d’intelligence.

Il s’est approché, avec ses mains énormes qui ressemblaient à des gants de boxe sans gants de boxe (et dans une de ces mains il y avait ce qui ressemblait furieusement à un pistolet semi-automatique Smith \& Wesson Bodyguard 380, un jouet qu’il valait mieux éviter de laisser traîner dans une chambre d’enfant), et n’a pas mis longtemps à dénicher Manu qui tentait de se planquer dans le fond de ma poche tel un petit animal apeuré.

Après quoi, il m’a décoché un petit sourire avec des dents impeccables qui n’étaient manifestement pas d’origine. Je préférais ça plutôt qu’il me décoche une droite en pleine poire.

\textsc{Riqueti} : Donnez-le-moi, je vous prie.

Niccolo lui a remis le flingue.

\textsc{Riqueti}, examinant Manu sous toutes les coutures : Quelle drôle de petite chose.

\textsc{Moi} : C’est Manu.

\textsc{Lui} : Manu ?

\textsc{Moi} : Oui, une arme de collection made in France.

\textsc{Lui} : Vous croyez qu’on peut tuer quelqu’un, avec ça ?

\textsc{Moi} : Avec un peu de bonne volonté.

Il a braqué le flingue dans ma direction et j’ai cru ma dernière ou avant-dernière heure arrivée. J’avais beau tenter de me persuader que Manu, reconnaissant son seigneur et maître (cet homme remarquable qui l’avait sauvé des oubliettes de la grande histoire des armes à feu et avec lequel il avait partagé tant de bons moments à la chasse aux nuisibles), refuserait de faire feu, j’avais un peu de mal à y croire.

Au dernier moment, juste avant d’appuyer sur la détente, Riqueti a détourné le canon.

\textsc{Lui}, apparemment enchanté de la bonne blague qu’il venait de me faire : Mais dites-moi, c’est qu’il marche du feu de Dieu, votre petit engin !

J’ai préféré, pour ne pas risquer de me laisser aller à un mouvement d’humeur qui n’aurait fait qu’envenimer la situation, garder le silence.

\textsc{Riqueti} : Bien, trêve de plaisanterie. Je peux savoir à qui vous parliez ?

\textsc{Moi} : Qui ça ? Moi ?

\textsc{Lui} : Oui, vous, au téléphone.

\textsc{Moi} : Un ami.

\textsc{Lui} : Un ami ? Donnez-moi ce téléphone, s’il vous plaît.

\textsc{Moi} : Non, c’est privé.

\textsc{Lui} : Ici aussi c’est une propriété privée. Ce n’est manifestement pas ça qui vous arrête. Le téléphone, Niccolo.

Niccolo m’a arraché le téléphone de la main, intervention qui m’a fait à peu près le même effet que si une murène avait jailli de son trou pour m’arracher un bras, et l’a tendu à Riqueti qui a jeté un coup d’œil à la liste d’appels et rappelé le dernier numéro.

Titus a décroché : Allo ?

Silence.

\textsc{Titus} : Djef ?

Silence.

\textsc{Titus} : Djef, tu m’entends ?

\textsc{Riqueti} : Oui.

\textsc{Titus} : Qu’est-ce que tu fous, bordel ? Pourquoi tu me rappelles ?

Silence.

\textsc{Titus}, de plus en plus interloqué : Djef ?

\textsc{Riqueti} : Ne vous inquiétez pas, monsieur…. monsieur comment, déjà ?

\textsc{Titus} : Vous n’êtes pas Djef !

\textsc{Riqueti} : Non, je ne suis pas Djef. Mais ne vous en faites pas, il est entre de bonnes mains. Les miennes, en l’occurrence, et celles de Niccolo.

Comprenant enfin qu’il était en train de se faire mener en bateau, et prenant du même coup conscience de la gravité de la situation, Titus a aussitôt raccroché.

Je n’ai pas eu le temps de vous dire, tant j’étais pris dans le feu de l’action, que Riqueti tenait un chien en laisse.

Ce chien, à poil ras, de taille moyenne, affublé de longues oreilles pointues et d’une queue de rat, ne semblait pas le moins du monde intéressé par ma présence. J’en connais qu’autres qui auraient grogné et montré les dents, mais lui était au-dessus de ça. Il me considérait comme une quantité négligeable et tenait à me le faire sentir, exprimer par son détachement hautain que je n’avais pour lui pas plus d’importance qu’une vieille feuille de laitue ou une pomme blette.

\textsc{Riqueti}, voyant que je reluquais son clebs : Je vous présente Terzo, inspecteur. C’est un Cirneco de l’Etna\nf{Le Cirneco dell’Etna est une race de lévrier sicilienne parmi les plus anciennes d’Europe, attestée dès 2 500 av.\ J.-C. par des pièces de monnaie et des mosaïques. La villa romaine du Casale (Piazza Armerina, Sicile, \textsc{iv}e s.), inscrite au patrimoine mondial de l’Unesco, en présente de remarquables représentations de chasse. Aristote le mentionne dans son \textit{Histoire des animaux} (livre IX). \source{fr.wikipedia.org/wiki/Cirneco\_dell\%27Etna}}, un lévrier de très haute lignée dont on retrouve la trace dès l’Antiquité. En Sicile, par exemple, sur les murs de la villa du Casale à Chiazza, on peut voir des scènes de chasse parfaitement conservées sur lesquelles il figure. Vous noterez la noblesse de ses traits. Aristote lui-même en a fait une description assez convaincante dans son Histoire des animaux. Mais je suppose que tout cela vous indiffère.

\textsc{Moi} : Totalement, en effet.

\textsc{Lui} : Vous avez tort, Terzo est un remarquable limier. Je suis sûr qu’il pourrait vous en remontrer à bien des égards.

\textsc{Moi} : Pour chasser le lapin, peut-être. Personnellement, je m’attaque à des proies nettement plus grosses.

\textsc{Lui} : Bien trop pour vous, je le crains. Vous avez un mandat de perquisition ?

\textsc{Moi} : Un quoi ?

\textsc{Lui} : Ne faites pas l’imbécile. Vous n’avez pas de mandat, vous n’êtes pas en service, pour moi vous n’êtes qu’un citoyen lambda pris en flagrant délit de violation de domicile. La légitime défense m’autorise à faire justice moi-même.

\textsc{Moi} : En vertu de quoi ? De l'Édit de Nantes ?

\textsc{Lui} : Non, de l’article 122-6 du Code Pénal qui m’autorise à repousser, de nuit, toute entrée par effraction, violence ou ruse, dans un lieu habité, le mien ne l’occurrence. Le cas de figure me semble avéré.

\textsc{Moi} : Mon cul, oui ! Vous savez ce qu’il en coûte de s’en prendre à un représentant de l’ordre ?

\textsc{Lui}, une grimace censée ressembler à un sourire aux lèvres : De l’ordre, c’est vous qui le dites. Moi, je dirais plutôt un représentant du chaos, un agent du désordre.

Puis, s’adressant à Niccolo et son Bodyguard 380 : Niccolo, mon ami, allez donc faire une tournée d’inspection pour voir si tout se passe bien. J’ai dans l’idée que notre ami n’est pas venu seul.

Niccolo a hoché la tête et s’est éloigné en trottinant comme un bouledogue en rut. Tout ce que j’avais entendu de lui, jusqu’à présent, se résumait à quelques vagues grognements laissant à penser qu’il maîtrisait difficilement l’usage de la parole.

\textsc{Moi} : Vous êtes sur une pente savonneuse, monseigneur.

\textsc{Lui} : Vraiment ?

\textsc{Moi} : Oui, mais il est encore temps de tout arranger. Vous me rendez mon arme, vous me laissez partir et je ferme les yeux sur vos activités illicites. On ne vous l’a peut-être pas dit, mais la peine de mort est interdite en France depuis le 9 octobre 81. Autrement dit, même pour jouer, il est interdit de faire griller des gens dans une cave.

\textsc{Lui} : Je suppose, par contre, qu’il est autorisé de les faire griller dans le coffre d’une voiture.

\textsc{Moi} : On ne va pas se chamailler pour si peu. Vous me laissez partir, et tout le monde continue à faire griller des gens où bon lui semble. Je suppose que vous avez d’excellentes raisons de vous en prendre à cet individu.

\textsc{Lui} : Excellentes, en effet.

\textsc{Moi}, désignant le type sur la chaise, lequel sortait progressivement de sa torpeur et tentait, toujours sans réel succès, d’articuler quelque chose qui ressemble vaguement à un langage connu : Et on peut savoir lesquelles ?

Un certain nombre de petites choses commençaient à tournoyer dans ma cervelle comme des mouches autour d’une charogne. Ce qui m’avait mis la puce à l’oreille, si j’ose dire, c’était le chien. Jusqu’à présent, je n’avais rien trouvé permettant de relier Riqueti au Brain Catcher, mais maintenant j’avais Terzo, chien de race de son état, dont j’imaginais assez bien qu’il n’était pas du genre à bouffer de la pâtée premier prix. Par contre, les croquettes Waterflox à l’agneau et au riz d’Anada Sintawichai, à peu près aussi chères au kilo que le caviar de Beluga albinos ou la Tuber magnatum piémontaise, pouvaient fort bien constituer son ordinaire.

\textsc{Lui} : Disons qu’il a commis des erreurs regrettables, et que le moment est venu pour lui de payer.

\textsc{Moi} : Quel genre d’erreurs ?

\textsc{Lui} : Du genre sexuel, si vous voyez ce que je veux dire.

\textsc{Moi}, après quelques instants d’un silence si épais qu’un pet de moucheron aurait fait l’effet d’une déflagration thermonucléaire : C’est vous, n’est-ce pas ?

Un sourire sorti des tréfonds de son âme dévoyée a transformé son visage jusqu’ici relativement anodin en authentique preuve de l’existence du diable : Moi quoi ?

\textsc{Moi} : C’est vous, le Brain Catcher !

\textsc{Lui} : Je ne vois pas de quoi vous voulez parler. Je suis monseigneur Mathéo Riqueti, évêque du Sanctuaire de Ddarr et ami personnel du cardinal Prospero Cangelosi, le chef de chœur au Vatican. Je ne vois rien de répréhensible là-dedans.

\textsc{Moi}, à propos du type sur la chaise : Et lui, qui est-ce ?

\textsc{Riqueti}, le visage toujours affreusement déformé par cette forme de démence rare qui atteint parfois les ecclésiastiques en fin de carrière : Un homme d’Église, lui aussi. Il s’agit du père Marian Granet, si vous tenez vraiment à tout savoir.

\textsc{Moi}, d’une voix claire et nette : Vous êtes le Brain Catcher !

\textsc{Lui} : Mmmoui, j’en ai vaguement entendu parler, en effet. Sachez que je n’ai rien à voir avec ce triste personnage.

\textsc{Moi} : N’empêche que c’est bien vous qui tuez des prêtres et leur farcissez le crâne avec des croquettes Waterflox !

\textsc{Lui} : Waterflox, dites-vous ? C’est bizarre, c’est justement la marque préférée de Terzo. Vous pensez que mon chien a quelque chose à voir dans cette terrible histoire ?

\textsc{Moi} : Je pense surtout que vous êtes complètement cinglé !

\textsc{Lui} : Vous n’êtes pas sans savoir que la définition de la maladie mentale reste extrêmement sujette à caution. Mais dites-moi, inspecteur, avez-vous déjà assisté à une exécution ?

\textsc{Moi} : Oui, bien sûr.

\textsc{Lui} : Vraiment ?

\textsc{Moi} : Oui, même que la plupart du temps c’est moi qui fais office de bourreau.

\textsc{Lui} : Je vous parle d’une exécution sur la chaise électrique.

\textsc{Moi} : Merci, j’avais compris. Non, évidemment, je n’ai jamais assisté à une exécution de ce genre.

\textsc{Lui} : Et que diriez-vous d’y assister ?

\textsc{Moi} : Vous n’allez pas me dire que cette relique est en état de marche ?

\textsc{Lui} : Cette relique, comme vous dites, est une exacte réplique, à un léger détail près, de la chaise sur laquelle s’est assis le tueur en série sado-maso, pédophile et cannibale Albert Fish\nf{Albert Fish (1870--1936), surnommé «~le Vampire de Brooklyn~» ou «~l'Homme gris~», tueur en série américain reconnu coupable d'au moins trois meurtres d'enfants. Masochiste autoproclamé, il s'était planté des aiguilles dans le périnée~; elles provoquèrent des courts-circuits lors de son exécution à Sing Sing le 16 janvier 1936. \source{fr.wikipedia.org/wiki/Albert\_Fish}}, alias le Vampire de Brooklyn, le 16 janvier 1936. On lui reproche notamment d’avoir étranglé la petite Grace Budd, 10 ans, avant de la découper en morceaux et la manger entièrement. Outre le fait qu’il entendait des voix lui ordonnant de violer, castrer et tuer des petits garçons, il adorait s’enfoncer des aiguilles dans le rectum et se fouetter jusqu’au sang avec une planche à clous de sa confection. Les aiguilles en question ont provoqué des courts-circuits pendant l’exécution, obligeant, pour le plus grand bonheur de Fish, le bourreau à s’y reprendre à plusieurs fois pour arriver à ses fins. L’homme que vous voyez assis sur cette chaise, le père Granet, prétend lui aussi que c’est la voix de Dieu qui lui ordonne de s’en prendre à des petits garçons. Il prétendait tellement que j’ai dû lui en couper une bonne partie pour le faire taire. Vous voyez ce bouton rouge, là-bas ?

\textsc{Moi} : Non.

\textsc{Lui} : Mais si, vous le voyez.

\textsc{Moi} : Je ne tiens pas du tout à le voir.

\textsc{Lui} : Il suffit d’appuyer dessus pour faire rôtir le père Granet, et c’est vous-même qui allez vous en charger.

Le père Granet, qui n’avait plus de langue mais avait encore ses oreilles, s’agitait de plus en plus, parfaitement conscient du caractère critique de sa situation. Il aurait aimé se plaindre, exprimer des objections, faire valoir son point de vue, mais le moignon de langue qui se tortillait dans le fond de sa gorge tel un rat pris au piège ne lui permettait pas de faire entendre sa voix de façon satisfaisante. Il était, je l’ai dit, entièrement nu, et la chaise avait été conçue de telle sorte qu’il lui était possible de se soulager sans avoir à se déplacer. En clair, il était assis sur une chaise percée avec un seau positionné en dessous pour recueillir ses déjections, à la façon d’une cuvette de chiotte.

Alors je ne sais pas ce qui s’est passé, si ses muscles se sont relâchés ou quoi, toujours est-il qu’il s’est mis à uriner en même temps qu’il continuait à tenter désespérément de s’exprimer, les deux formant un mélange que je n’hésiterai pas à qualifier d’assez indigeste. En matière de spectacle avilissant, de façon de rabaisser un homme plus bas que terre, le forcer à renoncer à toute espèce de dignité, Riqueti avait atteint un degré de raffinement qui témoignait d’un réel sens de l’esthétique.

Cela dit, je n’étais pas prêt pour autant à lui servir d’homme de main, d’où ma réponse ferme et définitive : C’est hors de question !

\textsc{Lui} : Vous êtes en mon pouvoir, maintenant, et n’avez pas d’autre choix que d’accéder à mes désirs. Sauf si vous préférez vous assoir sur cette chaise après lui, et mariner dans vos excréments pendant un temps indéterminé avant que quelqu’un se décide à mettre un terme à vos souffrances. Si vous ne le faites pas pour lui, faites-le pour vous. Vous n’avez pas hésité à carboniser Dardariel dans le coffre de sa voiture, je ne vois pas ce qui vous empêche d’en faire autant avec lui. Si ça peut vous décider, sachez que le père Granet n’est pas seulement un pédophile de la pire espèce, mais aussi un assassin qui n’hésite pas à sacrifier des nourrissons pour s’attirer les bonnes grâces de Lucifer. Il perpétue l’exercice occulte des messes noires, à l’image d’un Eustache Blanchet\nf{Eustache Blanchet (?--v. 1440), prêtre poitevin, aumônier de Gilles de Rais. Accusé d'avoir participé à des invocations démoniaques, il fut condamné lors du procès de 1440 mais gracié après avoir témoigné contre Gilles de Rais. \source{fr.wikipedia.org/wiki/Gilles\_de\_Rais}}, un Joseph-Antoine Boullan\nf{Joseph-Antoine Boullan (1824--1893), abbé français défroqué, fondateur de la Société réparatrice. Accusé de satanisme et de pratiques occultes, il servit de modèle au chanoine Docre dans \textit{Là-bas} (1891) de Huysmans. \source{fr.wikipedia.org/wiki/Joseph-Antoine\_Boullan}}, un Étienne Guibourg\nf{Étienne Guibourg (v. 1610--1686), abbé français impliqué dans l'affaire des Poisons (1679--1682). Accusé d'avoir célébré des messes noires pour la marquise de Montespan, maîtresse de Louis XIV, il mourut emprisonné sans jugement à la forteresse de Besançon. \source{fr.wikipedia.org/wiki/\%C3\%89tienne\_Guibourg}}, ou plus récemment un Matthias Schuster, faux pasteur mais véritable escroc qui non seulement couchait avec sa sœur et les enfants qu’il avait eus avec elle, mais s’adonnait à des rituels sataniques dans les ruines de l’abbaye de Grérac, en Corrèze, dont il avait fait l’acquisition pour une bouchée de pain en même temps qu’un hameau abandonné situé en plein cœur de la forêt de Montaulogne, soi-disant pour les remettre en état. Je suppose que vous en avez entendu parler.

\textsc{Moi} : Ça me dit vaguement quelque chose, en effet.

Pendant ce temps, Niccolo avait repéré Titus qui se dissimulait entre deux haies de thuyas, le Glock en main.

Il s’est approché en silence, par derrière, et lui a collé le canon de son flingue entre les omoplates, tout en disant ceci : Laisse tomber ton flingue, doucement, tout doucement, et mets les mains en l’air.

Titus a laissé tomber le Glock, mais au lieu de lever sagement les mains, il s’est brusquement retourné et a saisi le bras qui tenait le Bodyguard.

Une lutte s’est engagée, au cours de laquelle Niccolo a accidentellement appuyé sur la détente, provoquant une détonation qui a attiré l’attention de Riqueti, lequel, je vous le rappelle, me tenait en joue avec Manu.

\textsc{Riqueti}, en entendant le coup de feu, a eu le mauvais réflexe de tourner la tête.

Telle la mangouste qui se jette sur le serpent, j’ai profité de l’occasion pour lui balancer un coup de latte dans les parties. Les saintes couilles de notre ami l’évêque ont fait le yoyo dans son slip. Riqueti est tombé à genoux, la gueule ouverte, les yeux exorbités, avant de se rouler par terre en poussant des beuglements de veau à l’abattoir. Au passage, il a laissé tomber Manu et la laisse du clebs, lequel, jusqu’ici d’une placidité à toute épreuve, je dirais même d’une indifférence rare, s’est instantanément transformé en bête fauve tout droit sortie d’une époque où bestialité et sauvagerie exacerbées régnaient en maîtresses absolues sur la terre de nos ancêtres.

Pourvu d’une détente assez impressionnante, le canidé est passé à trois mètres au-dessus de ma tête au moment même où je me penchais pour ramasser Manu. Il a fini sa course à l’autre bout de la pièce, et le temps qu’il réussisse à freiner en catastrophe, faire demi-tour et repartir à la charge, j’avais eu largement le temps de récupérer mon vieux compagnon de route.

Je lui ai serré chaleureusement la crosse, sur laquelle étaient gravées les initiales de mon grand-père maternel, PC pour Philibert Chéron (et Parti Communiste, auquel il appartenait effectivement, de toutes les forces de sa foi vibrante en l’avenir de l’Homme et la perspective d’un monde meilleur, sans misère ni injustice), le père de ma génitrice adorée Valentine Chéron (VC, oui), résistant de la première heure et tueur de nazis digne (même s’il n’était pas juif, à ma connaissance en tout cas) d’un Aldo l’Apache ou un Donny Donowitz, et lui ai appuyé une première fois sur la détente qu’il avait souple et onctueuse.

Terzo ne se trouvait plus qu’à quelques mètres de moi quand la première balle, tirée au jugé je dois bien le dire, l’a atteint à la cuisse. Il a poussé un jappement de douleur tout à fait déchirant mais continué à courir comme si de rien n’était, comme si sa fureur le rendait insensible à la douleur.

Pourquoi ne pas le dire, au risque de passer pour un crétin sentimental d’une naïveté crasse à la limite de la niaiserie la plus inacceptable, mais il se trouve que j’ai une certaine tendresse pour les animaux. Au même titre que nous, ils font partie de la grande famille du vivant et tâchent de tirer au mieux, avec les moyens rudimentaires dont ils disposent, leur épingle du jeu sadique auquel ils sont contraints de participer. Cela dit, même si je savais fort bien que ce pauvre Terzo agissait davantage comme une machine à tuer décérébrée qu’un authentique criminel dont la seule et unique joie consiste à infliger de la souffrance à ses semblables, je n’allais pas me laisser égorger comme un mouton pendant la fête de l’Aïd. L’instinct de survie n’est-il pas la force principale qui régit notre existence ? Il faut, en permanence, se protéger d’une multitude d’agressions de toutes natures, dont on ne sait jamais trop quand ni comment elles vont nous tomber sur le dos, ce qui tend à générer du stress et de l’angoisse pouvant aller jusqu’à la psychose et la paranoïa (laquelle va nous pousser à détruire ce qui nous entoure à titre préventif, nous chuchoter dans le creux de l’oreille «~tue untel ou untel si tu veux sauver ta peau~»). Les mécanismes de défense sont au cœur de notre activité, et l’être humain, conscient de cette triste réalité, a mis au point un certain nombre de dispositifs très utiles en cas de danger. Et parmi ceux-ci, même si on ne pouvait plus parler à son égard de fleuron de la technologie moderne, le revolver Le Français de Manufrance (ici en modèle de poche 6.35, l’équivalent national du Walther thuringeois) restait un exemple tout à fait convaincant de l’ingéniosité déployée par l’être humain pour se protéger de l’adversité.

Les deuxième et troisième coups sont partis quasiment coup sur coup, c’est le cas de le dire, et je dois admettre en toute modestie que tous deux ont atteint leur cible, je veux bien sûr parler de ce pauvre Terzo qui ne faisait finalement rien d’autre qu’effectuer son boulot de chien, en toute innocence.

Hélas, tout a une fin, et jamais plus personne ne l’entendrait aboyer, chose qu’il ne faisait de toute façon que très rarement, compte tenu de son pédigrée et son éducation hors pair. C’est avec beaucoup de dignité, après un ultime soubresaut, qu’il a rendu son âme à Anubis, dieu des chiens, auquel il ressemblait d’ailleurs comme deux gouttes d’eau (au même titre que deux espèces très proches, le chien de garenne des Canaries d’une part, redoutable chasseur de lapin s’il en est, et surtout le lévrier de Malte qui serait un descendant direct de Tesem, le chien de Khéops, deuxième pharaon de la IVe dynastie dont la pyramide se dresse à Gizeh, non loin de celles de Khéphren, son fils, et Mykérinos, le fils de son fils).

Riqueti, qui commençait tout juste à se remettre de ses émotions, a poussé un long cri d’effroi en voyant son animal adoré étendu dans la poussière. Quelqu’un de plus charitable que moi lui aurait sans doute laissé le temps de pleurer à loisir son compagnon disparu, mais j’étais d’assez mauvaise humeur et peu enclin à faire preuve de compassion. En guise de condoléances, je lui ai braqué mon flingue sous le nez, et ordonné de s’étaler face contre terre, les bras en croix.

Pendant ce temps, Titus était aux prises avec Niccolo.

À première vue, la bataille semblait inégale. D’un côté Titus, force de la nature dépassant le mètre quatre-vingt-dix et capable de décorner un buffle à mains nues, et de l’autre Niccolo, son chapeau de cowboy ridicule, ses oreilles en chou-fleur et son nez épaté, d’une carrure impressionnante, certes, et doté de paluches surdimensionnées, mais dont la taille s’apparentait à celle de ce qu’on aurait pu appeler un «~nain de grande taille~», c'est-à-dire grand pour un nain mais totalement ridicule pour un être humain normalement constitué. Attention, je ne prétends aucunement que les nains sont des êtres humains anormalement constitués. Non, je dis juste que certaines particularités physiques permettent de les distinguer à coup sûr du commun des mortels. Et qui sait, peut-être un jour régneront-ils en maîtres sur la Terre, ce qui ne serait d’ailleurs pas une mauvaise idée car on pourrait réduire considérablement la hauteur sous plafond des appartements et loger ainsi beaucoup plus de monde. Mais bon, on n’en est pas encore là (cela dit, petite prévision gratuite en passant, je ne serais pas étonné que l’être humain rétrécisse dans les siècles à venir, jusqu’à retrouver une taille proche de celle qu’il avait à la préhistoire).

Donc, si on s’en tenait aux dimensions des belligérants, la bataille semblait inégale. Sauf que Niccolo était un boxeur hors pair, aussi difficile à toucher qu’une anguille. Titus, après un corps-à-corps viril pendant lequel il avait pu juger de la force hors du commun de son adversaire, était néanmoins parvenu à lui faire lâcher son engin (je parle bien sûr de son Smith \& Wesson Bodyguard 380, 380 pour 380 ACP, soit le 9 mm court conçu dès 1903 par le très inventif John Moses Browning\nf{John Moses Browning (1855--1926), armurier américain, l'un des inventeurs les plus prolifiques de l'histoire des armes à feu. Outre la cartouche .380 ACP (1903), il conçut le Colt M1911, le Browning Hi-Power, la mitrailleuse M2 «~Ma Deuce~» et le Winchester Model 1894. \source{fr.wikipedia.org/wiki/John\_Moses\_Browning}} pour équiper le Colt Pocket Hammer). Un Titus, je le rappelle, qui avait lui-même laissé tomber son Glock quelques instants auparavant.

Niccolo a entamé les débats par un direct au corps qui s’est échoué sur les abdos de Titus, naturellement bien doté de ce côté-là. Si c’était moi qui avais pris ce coup, je crois que je serais encore allongé par terre à regarder danser les étoiles dans le ciel. Il était évident, à voir la façon dont il se déplaçait, que Niccolo avait de nombreuses années de boxe derrière lui. Titus s’est dit qu’un crochet au foie suivi d’une droite à la mâchoire serait une combinaison intéressante à essayer. En face de lui, Niccolo dansait avec la légèreté d’une libellule, distribuant au passage des coups dont la plupart faisaient mouche. Le crochet au foie de Titus n’est pas arrivé à destination, et la mâchoire tant convoitée n’était plus là quand son poing s’est présenté. Pendant ce temps, les jabs continuaient à pleuvoir et atteindre leur cible avec une précision diabolique. Titus, même s’il était loin d’avoir inventé l’eau tiède, était quand même pourvu de facultés intellectuelles lui permettant de se faire une idée raisonnable de la situation. Et le moins qu’on puisse dire, c’est que la sienne était tout sauf enviable. Il ressemblait à un de ces sacs de sable sur lesquels les boxeurs frappent à tour de bras pour s’entraîner. À aucun moment il n’avait le temps de s’organiser tant Niccolo faisait preuve d’une rapidité déconcertante. Il pensait éviter un coup, et venait au contraire s’empaler sur un autre qu’il n’avait pas vu venir. Sa résistance était mise à rude épreuve et la défaite semblait inévitable.

C’est alors qu’un coup de théâtre, de ceux qui ont fait la renommée des tragédies antiques, s’est produit.

Alors que Niccolo continuait à tabasser gentiment sa proie, s’amuser avec elle, la réduire méthodiquement à l’impuissance, il a mis le pied dans un trou. Sa cheville a émis un craquement de mauvais augure, et l’expression de joie sadique qui tapissait son visage s’est aussitôt changée en voile noir et grimaçant. D’un coup, qui ne devait rien à Titus, le Rudolf Noureev\nf{Rudolf Noureev (1938--1993), danseur et chorégraphe soviétique, considéré comme le plus grand danseur de ballet du \textsc{xx}e siècle. Il fit défection à l’Ouest le 17 juin 1961 à l’aéroport du Bourget (Paris) lors d’une tournée du Ballet Kirov, puis dirigea le Ballet de l’Opéra de Paris (1983--1989). \source{fr.wikipedia.org/wiki/Rudolf\_Noure\%C3\%AFev}} des rings s’est transformé en pantin désarticulé incapable de mettre un pied devant l’autre, boitillant de façon ridicule en essayant de se protéger de son mieux.

Vous pensez bien que Titus, en bon opportuniste qu’il était, a profité de l’occasion pour lui balancer tout ce qu’il avait dans la tronche, avec une violence décuplée par l’humiliation qu’il venait de subir. Une avalanche de coups s’est abattue sur Niccolo, lequel n’a pas tardé à mettre un genou, puis les deux à terre.

Titus, après lui avoir, en guise de cerise sur le gâteau, décoché un dernier coup de tatane dans les ratiches, a récupéré son Glock et le Bodyguard. Il était maintenant seul maître à bord, et l’autre le regardait avec des yeux remplis d’un mélange scintillant de haine et de désespoir.

Titus lui a collé le canon du Bodyguard contre la tempe, et il a dit : Bouge pas.

\textsc{Niccolo} : Tu fais le malin, maintenant, mais tu m’aurais jamais eu si je m’étais pas pris les pieds dans le tapis.

\textsc{Titus} : C’est vrai que t’es coriace.

\textsc{Niccolo} : Laisse-moi filer et on en reste là. Tu peux garder mon flingue en souvenir, si tu veux.

\textsc{Titus}, au moment même où son téléphone se remettait à sonner : Y a intérêt, que je vais le garder !

Il a décroché, tout en surveillant l’autre du coin de l’œil, et entendu ma voix chaude et enveloppante comme un manteau de fourrure en léopard des neiges qui lui murmurait dans le creux de l’esgourde : Titus ?

\textsc{Lui} : Ouais.

\textsc{Moi} : Alors, t’en es où ?

\textsc{Lui} : Je suis maître de la situation.

\textsc{Moi} : Il est mort ?

\textsc{Lui} : Non, mais je le tiens en joue.

\textsc{Niccolo}, entre ses dents : T’as vraiment eu de la chance, enfoiré !

\textsc{Moi} : Parfait, alors bute-le.

\textsc{Lui} : Tu crois ?

\textsc{Moi} : Oui, et après viens me rejoindre en bas. Tu ne devineras jamais ce que j’ai trouvé !

\textsc{Lui} : Vas-y, raconte.

\textsc{Moi} : Bute-le et viens me rejoindre en bas.

\textsc{Lui} : T’es sûr qu’il faut vraiment le buter ?

\textsc{Niccolo}, qui commençait à se faire du souci : Quoi ? Buter qui ?

\textsc{Titus} : Ta gueule, toi !

\textsc{Moi} : Comment ça, ta gueule ?

\textsc{Titus} : Non, pas toi, c’est l’autre qui me parle. T’es sûr qu’il faut vraiment le buter ?

\textsc{Moi} : Qu’est-ce que tu veux en faire ?

\textsc{Titus} : Je sais pas. On pourrait le laisser filer.

\textsc{Niccolo} : Oui, bonne idée. Vous en faites pas, je dirai rien à personne.

\textsc{Moi} : Pour qu’il aille nous balancer à la première occasion ? Certainement pas, non !

\textsc{Titus} : Tu crois qu’il ferait ça ?

\textsc{Niccolo}, de plus en plus inquiet pour son avenir : Quoi ? Faire quoi ?

\textsc{Titus} : Ferme-la, je t’ai dit !

\textsc{Moi} : C’est à moi, que tu parles ?

\textsc{Titus} : Non, c’est l’autre abruti qui arrête pas de me parler ! Commence vraiment à me casser les oreilles, celui-là !

\textsc{Moi} : Tu ne veux pas le buter ?

\textsc{Lui}, manifestement en proie à des questionnements existentiels qui n’étaient pas monnaie courante chez lui : J’ai pas dit ça.

\textsc{Moi} : T’es tombé amoureux de lui, ou quoi ?

\textsc{Lui} : Dis pas de conneries !

\textsc{Moi} : Non, parce que ça arrive parfois que des mecs se battent et tombent amoureux l’un de l’autre. Si c’est le cas, je vous souhaite tous les vœux de bonheur.

\textsc{Lui} : Okay, ça va.

\textsc{Moi} : Quoi ?

\textsc{Lui} : C’est bon, je vais le buter.

\textsc{Niccolo} : Ah non, ça va pas recommencer ! J’ai rien fait, moi, je suis juste un homme de main. Je vois pas pourquoi je devrais trinquer à la place des autres !

\textsc{Moi} : Je savais que je pouvais compter sur toi.

\textsc{Lui} : Tu restes au téléphone ?

\textsc{Moi} : Pourquoi faire ?

\textsc{Lui} : Je sais pas. J’aimerais juste que tu restes au téléphone pendant que je le bute.

\textsc{Moi}, compréhensif : Tu veux que je t’encourage un peu pendant que tu appuies sur la détente ?

\textsc{Lui} : Oui, je sais pas. Je le sens pas, cette fois.

\textsc{Niccolo} : Si tu le sens pas, faut pas le faire ! Il s’agit quand même de la vie d’un homme, merde !

\textsc{Moi} : Allez, vas-y, mon vieux. Prends ton temps, respire à fond, et quand tu sens que tu es prêt tu appuies sur la détente, tranquillement, sans forcer.

\textsc{Titus}, essayant de se remonter le moral : C’est vrai, ça, j’ai déjà buté plein de gens de sang-froid.

\textsc{Moi} : Bien sûr, des tas ! Et dis-toi bien que tous l’avaient amplement mérité.

\textsc{Lui} : Ouais, t’as raison.

\textsc{Moi} : Bien sûr que j’ai raison.

\textsc{Lui} : Je comprends vraiment pas ce qui se passe. En plus, ce type a vraiment une sale gueule et il vient de me dérouiller sévère. Je devrais lui en vouloir à mort.

\textsc{Moi} : Tu sais, il se passe parfois des choses curieuses, dans l’existence. On a buté des dizaines, voire des centaines de gens, et puis un jour, sans qu’on sache pourquoi, on n’arrive pas à appuyer sur la détente. Est-ce que c’est l’ampleur de la tâche qui nous submerge, l’impression de prêcher dans le désert, de se décarcasser pour des prunes ? Oui, peut-être bien, n’empêche qu’il faut se forcer à retourner au charbon encore et toujours, sans jamais céder au découragement.

\textsc{Titus} : C’est bon, je le sens bien, là !

\textsc{Moi} : Je suis de tout cœur avec toi, ma vieille !

C’est au moment précis où Titus allait enfin faire son devoir de tueur froid et sanguinaire que Niccolo, mu par cet instinct de survie qui trouve sa source au plus profond de notre génome, s’est détendu tel un fauve dans la savane, ou plutôt un phacochère surpris par un fauve dans la savane, et mis à courir de toute la force de ses jambes courtes mais puissantes dans la première direction qui s’offrait à lui.

\textsc{Moi}, commençant à trouver le temps long : Alors, ça vient ?

\textsc{Titus} : Oui, attends, je…

\textsc{Moi} : Quoi encore ?

\textsc{Titus} : Il est en train de se faire la malle !

\textsc{Moi} : Ben qu’est-ce que tu attends ? Vas-y, tire-lui dessus !

Titus a tiré à deux reprises en direction du fugitif, lequel a continué à détaler en zigzaguant comme un lapin avant de disparaître au coin d’un massif de fleurs.

\textsc{Moi} : C’est bon, tu l’as eu ?

\textsc{Titus}, pressant une troisième fois la détente : C’est qu’il est rapide, le bougre !

\textsc{Moi} : On ne peut pas en dire autant de toi.

\textsc{Lui} : Je suis vraiment désolé.

\textsc{Moi} : Tu l’as raté, si je comprends bien ?

\textsc{Lui} : Ouais, je crois bien.

\textsc{Moi} : Eh oui. Voilà ce qui arrive quand on perd son temps en tergiversations oiseuses !

\textsc{Lui} : T’inquiète, il a eu la peur de sa vie. On ne le reverra pas de sitôt. De toute façon, j’ai gardé son flingue.

\textsc{Moi} : Il en a peut-être un autre planqué quelque part.

\textsc{Titus} : Penses-tu ! Il a eu du bol de sauver ses fesses et il ne va pas venir demander son reste.

\textsc{Moi} : Mouais, reconnais que t’as quand même pas été très efficace sur ce coup-là. Bon allez, viens me rejoindre en bas, j’ai des choses intéressantes à te montrer.

Quand Titus s’est pointé, penaud, et qu’il a vu le petit spectacle que je lui réservais, à savoir un chien mort, son maître avec les couilles en vrac et un père Granet plein d’espoir ficelé sur sa chaise électrique, il n’a pu retenir un : Nom de Dieu !

\textsc{Moi} : Comme tu dis ! Est-ce que je peux te demander de tenir cette ordure en joue pendant que je détache le prisonnier ? Tu penses que c’est dans tes cordes ?

\textsc{Lui}, désolé : Oui, je sais, j’ai merdé. Mais ne t’inquiète pas, s’il a le culot de se repointer ici je lui fais sauter le caisson sans sommation.

\textsc{Moi}, commençant à détacher le père Granet qui se trouvait dans un état de délabrement physique (à commencer par son bout de langue manquant) et moral assez impressionnant : Je t’accorde qu’il s’en tire à bon compte, raison pour laquelle je doute fort qu’il remette les pieds ici.

\textsc{Titus} : C’est qui, lui ?

\textsc{Moi} : Un certain père Marian Granet, à ce que j’ai cru comprendre. C’est bien ça, Monseigneur ?

\textsc{Riqueti}, les traits déformés (déjà qu’il n’était pas trop beau au naturel) par une haine féroce : Oui, et aussi une sale ordure qui a violé et assassiné des tas d’enfants innocents !

\textsc{Moi} : Loin de moi l’idée de me comparer à un enfant innocent, mais je vous rappelle que vous avez vous aussi tenté de me faire assassiner, mon cher ami.

\textsc{Lui} : Simple malentendu.

\textsc{Moi} : Fâcheux tout de même.

\textsc{Lui} : Excusez-moi, mais je ne savais pas de quel côté vous étiez.

\textsc{Moi} : Toujours celui de la justice, soyez-en sûr. Figure-toi, mon cher Titus, que notre bon ami le Cardinal a admis à demi-mot être le Brain Catcher.

\textsc{Titus} : Non ?

\textsc{Moi} : Si.

\textsc{Riqueti}, l’air renfrogné des grands jours : Je n’ai rien admis du tout !

\textsc{Moi} : Je suis certain que si on va faire un tour à l’étage, on va trouver des croquettes Waterflox.

\textsc{Titus} : À l’agneau et au riz ?

\textsc{Moi} : Oui, riz au jasmin et agneau mariné à la sauce hoisin, conçues à l’origine par le grand chef Anada Sintawichai pour Tuani, sa chienne Thaï Ridgeback à poil bleu.

\textsc{Riqueti} : Je ne vous pardonnerai jamais d’avoir tué Terzo !

\textsc{Moi} : Oui, je dois reconnaître que ça m’embête un peu. Je ne vois pas pourquoi les animaux doivent toujours payer le prix fort pour la connerie de leurs maîtres. Bon, en même temps, s’ils n’étaient pas programmés pour voler bêtement au secours de leurs supérieurs hiérarchiques, y compris des crétins de la pire espèce qui n’ont aucune réelle estime pour eux, ce genre de chose n’arriverait pas.

Une fois le père Granet sorti de sa chaise, je me suis rendu compte qu’il tenait à peine debout et l’ai posé dans un coin pour vaquer à la suite de mes occupations. J’ai quand même suggéré à Titus de le tenir à l’œil au cas où il reviendrait soudainement à la vie et déciderait de nous jouer un tour à sa façon. Je n’avais pas vraiment de raison de douter de ce que m’avait raconté le cardinal à son sujet, mais je ne pouvais pas non plus l’exécuter sans autre forme de procès. Avant de décider de quoi que ce soit à son sujet, une enquête s’imposait.

Vaguement contrarié (il arrive qu’on se sente vaguement contrarié, sans trop savoir pourquoi, surtout que dans le cas présent les choses se déroulaient plutôt pas mal), j’ai attrapé Riqueti par le bras en disant : Monseigneur, si vous voulez bien vous donner la peine.

\textsc{Lui}, paniqué : Qu’est-ce que vous faites ?

\textsc{Moi} : Je vous en prie, installez-vous.

\textsc{Lui} : Mais vous êtes fou ! Arrêtez ça tout de suite !

\textsc{Moi}, lui enfonçant Manu dans le creux des reins et le poussant dans la chaise : J’ai dit assis !

\textsc{Lui} : Vous n’avez pas le droit de faire ça !

\textsc{Moi} : Vous vouliez bien me forcer à faire griller le père Granet.

\textsc{Lui} : C’était pour rire. De toute façon, cette machine ne fonctionne pas.

\textsc{Moi} : Dans ce cas vous n’avez rien à craindre. Attache-le, Titus.

\textsc{Lui} : Je vous préviens, j’ai le bras long. Tout ça va vous coûter très cher !

\textsc{Moi} : Il faut savoir se faire plaisir de temps en temps.

\textsc{Lui} : Je n’ai rien fait de mal. Le père Granet était consentant, demandez-lui. Il aime les sensations fortes, si vous voyez ce que je veux dire. Cette situation l’excitait au plus haut point, c’est lui qui a insisté pour que je le condamne à la chaise électrique. Il a fait de choses discutables, pas très catholiques, ou trop, et il voulait expier ses fautes dans la douleur. Je lui ai dit : mon cher, s’il n’y a que ça pour vous être agréable, j’ai exactement ce qu’il faut à la maison.

\textsc{Moi} : N’empêche que vous alliez le faire griller.

\textsc{Lui} : Non, juste le torturer un peu à l’électricité pour son plus grand plaisir.

\textsc{Moi} : Je ne sais pas si vous le savez, mais pendant une électrocution la température du corps peut monter jusqu’à 100 degrés, entraînant des effets visuellement assez insupportables. En plus de se pisser et se chier dessus, il n’est pas rare que le supplicié s’enflamme ou que ses yeux soient éjectés de leurs orbites comme des bouchons de champagne ! C’est ce qui est arrivé à Albert Clozza en 1991.

L’avantage, avec le cardinal Riqueti, c’était qu’il était chauve et qu’on pouvait par conséquent s’épargner la corvée de lui raser le crâne avant de le mettre sous tension. Restait maintenant à espérer pour lui que le petit bijou qu’il s’était fait fabriquer fonctionnait aussi bien qu’il l’avait déclaré précédemment, sachant que la chaise restait quand même un des instruments de torture les plus barbares jamais mis au point par l’être humain pour se débarrasser de son prochain. À tel point qu’aujourd’hui quasiment plus personne ne l’utilise, à part quelques états parmi les plus sympathiques, les plus accueillants et les moins racistes du territoire américain, je veux bien sûr parler de la Virginie, de la Caroline du Sud, du Tennessee et de l’Alabama, ou griller un Noir de temps à autre fait encore partie des traditions et de ces bons vieux souvenirs qu’on se plaît à évoquer le soir au coin du feu, la pipe au bec et un verre de George Dickel à la main.

Le manuel de l’utilisateur, rédigé dans une langue facile d’accès et généreusement enrichi d’illustrations de qualité, détaillait la marche à suivre pour s’assurer d’une cuisson optimale, notamment les endroits où il était préférable de placer les électrodes, les spécialistes s’étant souvent écharpés à ce sujet. Si le sommet du crâne bien dégagé arrivait toujours en tête du classement pour la première électrode, la question de savoir s’il valait mieux placer la seconde sur le haut ou le bas de la jambe, voir le pied, faisait toujours débat. Dans le cas présent, le constructeur préconisait les testicules, ce qui techniquement n’était pas une mauvaise idée mais soulevait quand même quelques légers problèmes d’éthique auxquels il semblait assez peu perméable. Cela dit, à titre expérimental, j’ai décidé d'opter pour cette solution.

\textsc{Moi} : Il y a quand même une petite question que j’aimerais vous poser, Monseigneur.

\textsc{Lui}, de mauvais poil : Fichez-moi la paix !

\textsc{Moi} : Croyez-moi, ça vous fera du bien de soulager votre conscience.

\textsc{Lui} : Elle va très bien, merci !

\textsc{Moi} : J’aimerais savoir pourquoi vous vous en prenez à des hommes d’Église, les découpez en morceaux que vous expédiez aux quatre coins de la ville, et leur farcissez le crâne avec des croquettes pour chien ?

\textsc{Lui} : Qu’est-ce qui vous fait croire que c’est moi ?

\textsc{Moi} : Qui d’autre ?

\textsc{Lui} : Je ne sais pas, moi, un fou ! Je ne suis pas fou, et vous êtes sur le point de commettre une très grave erreur judiciaire.

\textsc{Moi} : Et vous, vous étiez sur le point de faire griller le père Granet.

\textsc{Lui} : Pas du tout. Comme je vous l’ai dit, il s’agissait d’un jeu à caractère sexuel.

\textsc{Moi} : Ben voyons ! Et ce morceau de langue que vous lui avez coupé ?

\textsc{Lui} : Il arrive que les choses dérapent dans le feu l’action.

\textsc{Moi} : Ce morceau de langue, je ne serais pas étonné que quelqu’un le reçoive bientôt dans la boîte aux lettres.

\textsc{Lui} : Vous n’avez aucune preuve !

\textsc{Moi} : Mais vous, vous avez un chien. Aviez, pardon.

\textsc{Lui} : Et alors ?

\textsc{Moi} : Alors je suis prêt à parier que vous le nourrissiez avec des croquettes Waterflox. Je n’aurai aucun mal à en trouver ici.

\textsc{Lui} : Et alors, ça ne prouve rien ! Des tas de chiens mangent des croquettes Waterflox.

\textsc{Moi} : À 3000 balles le kilo ? Je crois pas, non.

\textsc{Lui} : Je parle de chiens de race qui ont la chance de vivre avec autre chose que des maîtres chômeurs et alcoolos.

\textsc{Moi} : D’autre part, je suis certain que si je fouille dans ce gros congélateur que j’entrevois là-bas dans le fond de la pièce, je vais trouver des tas de bocaux avec des vrais morceaux de curés à l’intérieur.

Riqueti s’est recroquevillé dans sa chaise en me regardant de travers. J’avais vu des photos de déments prises dans des asiles psychiatriques de sinistre renom, aujourd’hui laissés à l’abandon pour la plus grande joie des urbex en mal de sensations fortes, eh bien le cardinal ressemblait tout à fait à l’un d’entre eux. Il aurait fait merveille avec une camisole de force sur le dos.

\textsc{Moi} : Je ne comprends pas ce qui peut pousser un homme d’Église à s’en prendre à d’autres hommes d’Église.

\textsc{Lui} : Je n’ai pas toujours été un homme d’Église, jeune homme. J’ai été enfant, comme tout monde, et victime des agissements pervers d’un de ces hommes d’Église dont vous parlez. Plusieurs, même, mais un en particulier qui m’a laissé un très mauvais souvenir. Pour faire court, je l’ai menacé de tout dire à mes parents et il a tué Leo, le Cirneco de l’Etna que mon père m’avait offert pour mon cinquième anniversaire, une bête superbe qui était mon plus fidèle ami et confident, à laquelle je tenais comme à la prunelle de mes yeux. Ensuite, il a dit que si ça ne suffisait pas il s’en prendrait à Nina, ma sœur, puis à mon père et ma mère si nécessaire. J’étais terrorisé et n’ai jamais rien dit à personne. Mais plus tard, je me suis juré d’avoir ma revanche sur tous ces enfoirés, raison pour laquelle je suis entré dans les ordres avec la ferme intention de me hisser au plus haut niveau de la hiérarchie. De l’intérieur, je n’aurais aucun mal à savoir qui faisait quoi et prendre les décisions qui s’imposaient. Pour vous dire la vérité, je me fiche complètement de Dieu, Jésus et toutes ces conneries. Ma seule et unique motivation a toujours été d’étancher ma soif de vengeance.

\textsc{Moi} : Pourquoi les découper en morceaux ?

\textsc{Lui} : Il avait fait la même chose avec Leo.

\textsc{Moi} : Et leur farcir la tête avec des croquettes pour chien ?

\textsc{Lui} : Petit touche de créativité personnelle, rapport à l’expression «~mettre du plomb dans la cervelle~». Moi, c’est des croquettes pour chien, de qualité supérieure, vous l’aurez noté. On a bien le droit de s’amuser un peu, non ?

\textsc{Moi} : Bien sûr. D’ailleurs, si vous n’y voyez pas d’inconvénient, on va maintenant passer au clou du spectacle.

\textsc{Lui} : Faites ce que vous voudrez. De toute façon, Terzo est mort et je n’ai pas la force de continuer à vivre sans lui. Vous promettez de me farcir le crâne avec des croquettes pour chien une fois que je serai mort ?

\textsc{Moi} : Désolé, mais je n’ai pas que ça à faire.

\textsc{Lui} : Dommage.

\textsc{Moi} : Il ne nous reste plus qu’à vous souhaiter bon voyage.

Ainsi s’achevait, avec le cardinal Mathéo Riqueti, la sanglante épopée du Brain Catcher.

Au moment de partir, une divine surprise nous attendait, preuve que les justes sont toujours récompensés : alors que nous passions au détour d’une allée, traînant avec nous le père Granet qui suait sang et eau pour mettre un pied devant l’autre, on est tombés sur ce bon vieux Niccolo adossé à un arbre, en train de se vider tranquillement de son sang. Titus n’était pas aussi nul que je le pensais, et qu’il le croyait lui-même, car il avait, tout en suivant une conversion au téléphone (pas passionnante, au demeurant), réussi un tir létal sur une cible mouvante, ce qui n’est pas donné à tout le monde. Niccolo, dans sa fuite, avait pris une balle dans le dos, et se trouvait à présent entre la vie et la mort. Comme on n’avait pas de temps à perdre, qu’il n’avait manifestement aucune chance de s’en tirer et que même si ça avait été le cas on ne l’aurait certainement pas laissé faire, Titus s’est à nouveau servi du flingue de Niccolo pour mettre un point final au roman de gare de son existence, bien trop longue succession de pages mal écrites distillant un mortel ennui. Après quoi il a méticuleusement essuyé le Bodyguard avant de replacer l’objet entre les mains de son propriétaire. L’engin avait dû servir à buter pas mal de gens, et Titus n’avait aucun intérêt à ce que son nom soit associé à son palmarès. L’heure était venue de rendre à César ce qui lui appartenait, et César allait pouvoir aller griller en enfer avec son mentor Riqueti, lequel, lorsqu’il aurait le cul transformé en rôti de porc cramé jusqu’à l’os, aurait peut-être un petit moins envie de faire joujou avec les appareils électriques.

Quant au père Granet, à en croire ses explications rendues quelque peu vaseuses par son état physique général, il ne s’agissait aucunement de l’ignoble pédophile dont Riqueti nous avait vanté les mérites. Son seul tort, dans l’affaire, était d’avoir des penchants sexuels d’une nature un peu particulière, certes, mais qui, s’exerçant toujours entre personnes majeures et consentantes, ne sortaient pas du cadre de la légalité. Bien sûr, à titre personnel, il ne m’inspirait que la plus vive répulsion, mais je ne me sentais pas qualifié pour juger de ses agissements (lesquels, selon moi, étaient du ressort de la psychologie la plus intime), et la seule chose que je pouvais lui conseiller, à part consulter au plus vite les meilleurs spécialistes de la confusion mentale, était de se tenir aussi loin de moi qu’il lui était possible de le faire, avec tout le respect que j’avais pour lui et le calvaire qu’il venait d’endurer.

Et aussi, bien évidemment, de tenir sa langue, ou sa moitié de langue, s’il tenait à la conserver, et je ne doutais pas qu’il suivrait mes prescriptions à la lettre. Même pour un type comme lui, sensible aux délices de la douleur, qui ne se sentait jamais plus en vie que lorsqu’il était assis sur les genoux de la mort, l’expérience avait été profitable. Il allait, à présent, réfréner ses pulsions les plus morbides et s’en tenir à des pratiques moins traumatisantes, plus consensuelles. Il avait pris conscience, au cours de cette mésaventure qui avait bien failli lui être fatale, que l’existence avait un prix au-delà duquel il n’était pas disposé à s’engager.

Titus et moi, en tout point fidèles à la réputation d’excellence qui était la nôtre, allions le déposer à l’hôpital le plus proche, où il chanterait de sa plus belle voix la chanson suivante : enlevé par un sadique alors qu’il rentrait chez lui à la tombée de la nuit, il avait été soumis à de multiples sévices avant de mettre à profit un instant d’inattention de son geôlier pour réussir à s’échapper. Il ignorait totalement l’endroit où il se trouvait et avait couru des heures durant dans la forêt avant de trouver une issue. Il avait ensuite, alors qu’il errait, dans un état second, sur une route secondaire dont il serait bien incapable de préciser la localisation, été pris en stop par un individu qui l’avait conduit jusqu’ici. L’individu en question, dépourvu de tout signe particulier permettant de le distinguer du commun des mortels, était pressé et ne tenait pas s’attirer de complications inutiles, raison pour laquelle il était aussitôt reparti, fin de l’histoire. Naturellement, tout cela s’était passé comme dans un de ces cauchemars dont on se réveille en sueur, le cœur battant, les yeux écarquillés dans la pénombre, et il était bien incapable de donner des informations précises sur les conditions de sa détention. D’autant que son bourreau, chaque fois qu’il lui rendait visite, non seulement portait un masque de Grand Méchant Loup de film d’horreur qui lui couvrait intégralement le visage, mais, à l’instar d’un Batman ou un Dark Vador, s’exprimait à travers un dispositif sonore qui rendait sa voix méconnaissable.

Enfin, et j’en terminerai là, j’ai embarqué quelques flacons de Bourgogne supplémentaires avant de prendre congé, la seule idée qu’ils puissent tomber entre de mauvaises mains me rendant littéralement ivre de rage. J’estimais, entre ma première razzia et celle-ci, avoir sauvé du naufrage les plus remarquables d’entre eux, même si je m’étais vu, bien à contrecœur, dans l’obligation de faire des choix qui avaient été pour moi autant de coups de poignard dans le dos de ma passion pour le jus de raisin fermenté. Ôôôôô fermentus, fermentatis, fermentatorum, laisse à tout jamais déferler dans mon gosier avide les flots tumultueux de ta splendeur fruitée aux saveurs incomparables ! Fasse que, à jamais enchaîné au flux continu de tes liquides bienfaits, jamais mon âme ne connaisse ce dessèchement intérieur qui ronge les meilleurs d’entre nous ! Amen.



\noindent Il y a ce qu’on appelle la «~loi des séries~», comme si le destin vous imposait une thématique à laquelle il semble impossible de déroger.

Assez loin de ces considérations, tous les dimanches matin, aux premières lueurs de l’aube, Robert Pleimelding allait à la chasse. Avec sa fidèle Greta, un braque allemand à poil dur qu’il appelait «~ma fille~» ou «~fifille~», alors même qu’il avait déjà une fille, une vraie, avec laquelle il n’échangeait guère plus de trois ou quatre mots par an, et encore seulement les bonnes années.

Le rituel était immuable, et rien, pas même un cyclone déferlant sur le secteur, n’aurait pu altérer son déroulement.

Tous les dimanches matin, tandis que Claire, sa femme, ronflait encore à poings fermés dans le creux de son lit (ils faisaient chambre à part depuis de nombreuses années, leurs horaires et habitudes nocturnes ayant atteint un seuil d’incompatibilité irréductible), Robert, après avoir passé un bon quart d’heure aux toilettes et avalé un demi-litre de café noir, enfilait sa tenue de chasse et sortait sur la terrasse pour voir le soleil se lever (quand il y en avait, bien sûr, car l’astre luminescent ne parvenait pas toujours à transpercer l’épaisse couche de brouillard qui tapissait la vallée) et respirer à pleins poumons les toutes nouvelles senteurs du jour.

Accessoirement, Robert était le genre de type qui, en plus d’une réussite familiale discutable, n’avait aucun ami, pas même au bistrot du village où il se gardait bien de mettre les pieds pour éviter les ennuis. Il avait, toute sa vie durant, travaillé pour le comte Léopold Chiasson de Bellisle, propriétaire du château voisin, en tant que garde-chasse. Il avait vingt-cinq ans quand il était entré au service de Bellisle, lequel en avait une bonne vingtaine de plus. Le comte, qui vivait seul sur ses terres avec quelques domestiques, une armada de bestioles en tout genre (ça allait de la poule et la pintade au lapin en passant par les chevaux, bien sûr, comme la plupart des rupins qui se la jouent, plus un certain nombre d’espèces exotiques rapportées de ses nombreuses expéditions à l’étranger) et une meute de chiens de race, affichait un très net penchant pour les gens de son sexe. Bon, il est certain que quand on voyait la tronche de Robert quarante ans plus tard, il était difficile d’imaginer quelle petite gueule d’amour il avait été dans sa jeunesse. Non seulement il avait une bouche spécialement conçue par le designer en chef de la Création pour enfourner de la saucisse au kilomètre, mais il était équipé d’un petit cul trois étoiles dont il aurait été criminel de ne se servir que pour s’assoir ou faire ses besoins. Homme de goût, amateur d’art et de belles choses en général, Chiasson de Bellisle ne pouvait qu’être émerveillé à la vue d’un tel chef-d’œuvre. Convaincu, il avait engagé sur le champ Robert et sa jeune épouse, Robert en tant que garde-chasse et Claire, qui avait de réels talents en la matière, en tant que cuisinière.

Rapidement, les relations entre Robert et son employeur avaient pris une tournure d’un genre un peu particulier.

D’abord, Robert était anormalement bien payé pour un garde-chasse, avec un salaire correspondant davantage à celui d’un ministre que d’un type qui passe son temps à se balader en forêt avec un fusil sur l’épaule, veillant à l’équilibre de la faune et la flore, et s’assurant qu’aucun contrevenant ne passait outre les panneaux PROPRIÉTÉ PRIVÉE affichés aux quatre coins du domaine. Dans le cas contraire, il avait ordre de faire feu sans sommation et enterrer le corps sur place, à la merci des bêtes sauvages, le comte étant un fervent défenseur des méthodes à l’ancienne, le bon vieux temps où le châtelain avait toute latitude pour faire régner l’ordre sur ses terres. Autant dire que si vous vous faisiez piquer avec un panier de cèpes ou un lapin dans la musette, vous aviez intérêt à courir très vite pour passer entre les plombs. À moins, bien sûr, de présenter, comment dire, certaines caractéristiques physiques susceptibles d’amener la partie civile à envisager l’éventualité d’un règlement à l’amiable du litige. Dans ce cas, pourquoi ne pas rester à quatre pattes le nez dans la mousse, à cueillir tranquillement des champignons, pendant que s’exerce avec une saine vigueur le droit séculier de la justice ?

Même chose pour Claire, dont la rémunération ne devait pas être très éloignée de celle d’un chef trois étoiles dans un palace méditerranéen. Il faut dire que sa cuisine était assez remarquable, et qu’elle était tout à fait capable de rassasier des tablées de quinze ou vingt personnes.

Si vous ajoutiez à cela que les tourtereaux étaient logés gratos dans le pavillon de gardien, disposant, en plus d’une terrasse et d’un jardin très convenables, de trois chambres, une belle cuisine parfaitement équipée, un bureau-bibliothèque (même si Robert n’avait pas lu plus de trois ou quatre livres dans toute son existence, et encore des livres avec des images et des grosses lettres) et un grand salon cossu avec poutres apparentes, baie vitrée et cheminée assez vaste pour faire cuire un obèse du Mississippi (c’est là qu’il y en a le plus) imbibé de sirop de maïs trafiqué et huile de palme hydrogénée, il n’y avait à priori aucune raison de se plaindre, même s’il leur était demandé une totale disponibilité en retour.

Robert aimait Claire (et les roberts de Claire, si ronds, fermes et débordants d’affection qu’ils faisaient, je ne crains pas de l’affirmer au risque de passer pour un immonde pourceau, l’envie et l’admiration de tous) et Claire aimait Robert, autrement dit Claire et Robert et inversement s’aimaient d’un amour si beau et pur qu’il brillait telle une pépite dans le lit boueux de la rivière infestée de sangsues de l’existence impitoyable qui est notre lot commun. Cet édifice affectif inébranlable reposait sur des fondations solides dont le ciment principal n’était autre qu’un mot de neuf lettres commençant par un c et se terminant par un e, je veux bien sûr parler de la confiance, laquelle se rapproche davantage de la confidence que du confit d’oie (même si rien n’empêche de se confier à quelqu’un en mangeant du confit d’oie, surtout si on arrose le tout de quelques bonnes rasades de Cahors ou Madiran). À ce titre, Robert n’avait jamais caché à Claire qu’il avait, depuis sa prime enfance, des hésitations sur la nature exacte de sa sexualité. Après mûre réflexion, il en était arrivé à la conclusion qu’il aimait les femmes, très certainement, mais n’était pas pour autant insensible aux appas de son sexe. Pas assez pour se lancer à corps perdu dans une homosexualité débridée, certes, mais tout de même suffisamment pour céder de temps à autre aux sirènes de l’inversion. Il lui arrivait, à l’occasion, de fréquenter, dans une tenue qui n’était pas la sienne habituellement, certains établissements dont la nature douteuse était de notoriété publique. Certains vont jouer au casino, d’autres vont se faire défoncer le cul dans des endroits fleurant bon la sueur, le cuir et le foutre.

Un soir, sous un prétexte aussi quelconque que fallacieux, Bellisle avait invité Robert à passer le voir.

Robert avait trouvé son employeur au coin du feu, nu sous une robe de chambre élimée dans le genre de celles que portaient les aristos des temps jadis, luth en main. Quand je parle de luth, je ne parle pas de lutte gréco-romaine mais de vrai luth, cette sorte de guitare du Moyen Âge qui a plus ou moins la forme d’une larme ou une goutte d’eau, l’instrument dont un Bernard de Ventadour\nf{Bernard de Ventadour (v.~1130--v.~1200), troubadour limousin considéré comme l’un des plus grands poètes occitans du \textsc{xii}\textsuperscript{e}~siècle, célèbre pour ses chansons d’amour courtois adressées à des dames de haute noblesse. \textit{Source :} \textup{fr.wikipedia.org/wiki/Bernard\_de\_Ventadour}} se servait pour enchanter la belle Aliénor d’Aquitaine\nf{Aliénor d’Aquitaine (v.~1122--1204), duchesse d’Aquitaine et reine successivement de France (épouse de Louis~VII) puis d’Angleterre (épouse d’Henri~II Plantagenêt). Grande mécène des troubadours, elle est l’une des figures politiques les plus puissantes du Moyen Âge occidental. \textit{Source :} \textup{fr.wikipedia.org/wiki/Ali\%C3\%A9nor\_d\%27Aquitaine}} à la cour du roi de France.

Invité à s’assoir, Robert s’était aussitôt retrouvé englouti dans un fauteuil qui avait dû voir défiler des millions de culs de toutes formes et tailles à travers les âges. À peine avait-il eu le temps d’ouvrir le bec pour formuler quelques vagues remerciements qu’il s’était vu proposer un havane et sommé d’avaler un verre de cognac de la taille d’un ballon de foot. Pendant ce temps, lui-même passablement éméché, le comte, sur des accords hésitants, poussait la chansonnette d’une voix plus proche de la toile émeri que du velours. La scène frisait le ridicule, admettons-le, mais il aurait fallu être foutrement ingrat pour ne pas rendre les armes devant une telle parade nuptiale, si bien arrosée et finement orchestrée, et le fait est que si on pouvait reprocher bien des choses à Robert, comme à tout le monde du reste, l’ingratitude ne faisait certainement pas partie de ses défauts.

C’est ainsi, après avoir rapidement épuisé son répertoire, que le comte a eu beau jeu de se jeter, telle une hyène en chaleur aux babines ruisselantes de salive nauséabonde, sur la proie en partie liquéfiée qui s’étalait devant lui. Consciente que sa fin était proche, que toute résistance était inutile, celle-ci venait d’ailleurs d’écluser son troisième (on aurait dit que le comte avait des mains partout, comme Shiva, tant il était capable de continuer à jouer et chanter~-- mal, certes, mais quand même~-- tout en remplissant les verres à la vitesse de l’éclair) ballon de cognac (pour info, sachant que non seulement il ne fumait pas mais allait bientôt avoir quelque chose d’autrement consistant à se fourrer sous la dent, Robert avait décliné l’offre du havane).

La suite (ce déchaînement de frénésie sexuelle qui fait irrésistiblement penser à une meute de hyènes affamées se disputant la carcasse d’un phacochère tombé sous les crocs d’un lion) se passe de commentaires. C’est moche, la vie est moche, on n’a pas d’autre choix que faire avec (raison pour laquelle la plupart d’entre nous ont une telle soif de liberté, qu’ils n’obtiendront bien entendu jamais, même si l’argent, qui précisément achète, corrompt et dénature tout, peut leur donner l’illusion du contraire), à moins d’éviter de naître ou se faire sauter le caisson à la première occasion, mais la nature est toujours là, tapie, embusquée derrière les écrans de fumée de la raison, l’intelligence et la conscience, et quand elle réclame son dû, ses esclaves se voient dans l’obligation de satisfaire au plus vite à ses exigences. C’est d’ailleurs ce qui pousse certains à commettre des exactions que la morale réprouve, et avec elle la société dans son ensemble, qui tente désespérément de sauver les meubles.

Hélas, ou tant mieux, tout a une fin, à commencer par les meilleures choses dont la durée de vie est d’autant plus courte qu’elles se rapprochent de l’excellence.

Un jour, alors qu’il était en train de fumer un Cohiba (un Esplendido Gran Reserva, son module préféré et grand classique de la marque) en contemplant les plus belles fleurs de sa roseraie, Bellisle avait rendu son âme à Lidos, empereur des trous de balle et dieu des sodomites. La chose s’était produite en une fraction de seconde : d’abord une violente douleur dans la poitrine, aussi vive que l’éclair, suivie de quelques spasmes désordonnés, doigts crispés cherchant vainement à se raccrocher à quelque chose, et notre homme s’affaissait tel un vieux slip plein de merde dans un massif de ces fabuleuses roses noires (ça ne s’invente pas) en provenance des jardins luxuriants d’Anatolie du sud-est.

C’est le jardinier, digne héritier de la statuaire grecque mais solidement équipé sur le plan génital (contrairement aux Grecs qui ont souvent des petites bites, vous l’aurez sans doute remarqué si vous vous intéressez un tant soit peu à la question), qui avait découvert le corps. On se doute que le comte ne l’avait pas engagé pour ses seules compétences en matière de taille, tonte et bouturage. Non, le vieux salopard avait tenté à maintes reprises de le détourner du droit chemin de la reproduction, des vertes allées de l’amour courtois, allant jusqu’à lui proposer des sommes déraisonnables en échange de l’anneau de jouissance qu’il dissimulait jalousement entre ses petites fesses potelées (une version moderne du Seigneur des anneaux, si vous préférez, sous sa forme sainte trilogie de l’arrière-train : La Communauté de l’anus, Les Deux Trous et Le Retour du doigt). Mais l’animal (à poil la plupart du temps, hormis un micro-short en jean déchiré~-- tout juste si on ne lui voyait pas les boules de chaque côté~-- et une paire de bottes en caoutchouc), aussi difficile à croire que cela puisse sembler, était d’une hétérosexualité à toute épreuve, un parangon de vertu insensible à toute espèce de corruption, une sorte de réincarnation sexy du Christ. Il aurait pu le virer et le remplacer par un autre, plus coopératif, mais se plaisait énormément à la dégustation platonique de ce fruit défendu, rêvant à la jutosité de sa chair et la saveur sucrée de ses sécrétions intimes. Il existait, chez Bellisle, une force qui le poussait à toutes les perversions, et s’infliger le supplice de Tantale était sans doute pour lui le moyen d’expier une partie de ses fautes.

Prévoyant, et très épris, Bellisle, qui n’avait aucun héritier direct, avait couché (avec, mais pas que) Robert sur son testament, lui léguant une somme confortable grâce à laquelle ce dernier avait pu acquérir l’agréable demeure qu’il habitait aujourd’hui.

Et donc, comme je vous le disais, cette agréable demeure, il la quittait tous les dimanches matin pour aller à la chasse avec Greta, sa chienne, acquise elle aussi avec cet héritage ; tout comme, du reste, le juxtaposé Chapuis Progress Grand Luxe calibre 12/70 (crosse anglaise en noyer premier choix, plaque de couche en bois de rose et sujets animaliers gravés en taille douce sur les contre-platines et le dessous de bascule, grosse émotion) qu’il chérissait comme la prunelle de ses yeux et passait des heures à lustrer en écoutant des opérettes de Jacques Offenbach\nf{Jacques Offenbach (1819--1880), compositeur et violoncelliste franco-allemand, créateur de l'opéra-bouffe français. Auteur de plus de cent opérettes, dont \textit{La Belle Hélène} (1864), \textit{Orphée aux Enfers} (1858) et \textit{Les Contes d'Hoffmann} (posth., 1881). \textit{Source :} \textup{fr.wikipedia.org/wiki/Jacques\_Offenbach}}, Reynaldo Hahn\nf{Reynaldo Hahn (1874--1947), compositeur, chef d'orchestre et critique musical vénézuélo-français, ami intime de Marcel Proust. Connu pour ses mélodies et ses opérettes, notamment \textit{Ciboulette} (1923) et \textit{Mozart} (1925). \textit{Source :} \textup{fr.wikipedia.org/wiki/Reynaldo\_Hahn}} et Francis Lopez\nf{Francis Lopez (1916--1995), compositeur français spécialiste de l'opérette populaire, auteur de \textit{La Belle de Cadix} (1945) et \textit{Le Chanteur de Mexico} (1951), qui connurent un succès considérable dans l'après-guerre. \textit{Source :} \textup{fr.wikipedia.org/wiki/Francis\_Lopez}}.

Solitaire de nature, c’est seul avec Greta qu’il aimait s’adonner à sa passion. Il abandonnait volontiers le gros gibier aux viandards pour se consacrer à des cibles plus modestes, selon lui autrement intéressantes gustativement et passionnantes à traquer, comme le lièvre et la perdrix, mais surtout le plus noble, intense et raffiné d’entre tous, je veux bien sûr parler de Scolopax rusticola, migrateur à long bec et plumage mimétique plus connu sous le nom de bécasse. Naturellement, à l’occasion, il ne crachait pas sur la grive, le garenne ou le colvert. Du moment que la bête tenait dans la marmite, il n’y avait aucune raison de s’en priver. En ce qui concerne la bécasse, Greta n’avait pas son pareil pour la débusquer (au même titre que la truffe), et Claire pour la cuisiner, au four, à la ficelle, en cocotte au vin blanc ou rouge, à la truffe ou au foie gras, flambée à l’Armagnac ou avec ce très vieux Calva d’une richesse aromatique sans précédent que Robert se procurait au compte-goutte chez un producteur confidentiel (Raoul Gounelle, je ne devrais pas vous le dire mais je le fais quand même parce que vous êtes de bons clients). Quand le comte (de son vivant bien sûr, il n’allait quand même pas revenir de l’au-delà exprès pour ça) leur faisait l’honneur de venir dîner chez eux, elle ne manquait jamais de le régaler avec une bonne plâtrée de bécasse, surtout les recettes avec beaucoup de truffe et de foie gras, ou à défaut de bécasse un de ces civets de lièvre ou rôti de colvert aux châtaignes et champignons des bois (le comte en raffolait et s’en foutait plein les moustaches en poussant des grognements de satisfaction) dont elle avait le secret, après quoi leur hôte s’allumait un de ses havanes favoris (il demandait toujours la permission et personne n’osait la lui refuser) et s’endormait lourdement sur le Calva avant de rentrer à quatre pattes chez lui, Robert dans son sillage pour vérifier qu’il ne se trompait pas de direction.

Ce matin-là, donc (je ne sais pas si je vais y arriver), aux aurores, Greta (folle de joie à l’idée d’aller se dégourdir les papattes en forêt), Robert, sa gibecière à rabat en cuir pleine fleur Lazzaro Bernardini (dans les deux mille euros sur plaisirsdelachiasse.com), son fusil Chapuis de collection, un thermos de café, des vivres pour midi (dont une belle grosse gourdasse d’un Hautes-Côtes de Nuits somme toute assez gouleyant), et assez de cartouches pour soutenir un siège face au GIGN, sont tous montés dans le Land Cruiser que ce même Robert, toujours lui, s’était offert avec le pognon de Bellisle. Robert s’est installé au volant, Greta à ses côtés (elle préférait monter devant), a tourné la clé de contact, le 4x4 a hoqueté quelques instants avant de faire entendre la douce musique de son six cylindres en ligne, puis tout ce joli petit monde a pris le chemin des quelques cinquante hectares de forêt que cet enfoiré de Robert, encore et toujours lui, avait également hérité de son bienfaiteur. Une belle journée s’annonçait, pleine de soleil et de ciel bleu, et avec elle la promesse d’une partie de chasse réussie quel que soit le résultat des courses. Même s’il fallait se contenter d’un lapin ou deux, avec en prime une perdrix et quelques cèpes, la joie de se balader au grand air, en toute liberté, avec Greta gambadant à ses côtés, l’œil vif et la truffe au ras du sol, attentive au moindre bruit, au moindre frémissement, et surtout personne pour vous poser des questions idiotes et vous débiter des inepties dans les oreilles, suffirait amplement à son bonheur.

Seulement voilà, dans l’existence, tout ne se passe pas nécessairement comme prévu, raison pour laquelle il est plus prudent de ne rien prévoir du tout.

Quelques heures plus tard, il avait tiré un lièvre et ramassé quelques belles poignées de girolles qui, préparées au beurre, à l’ail et au persil, constitueraient un accompagnement idéal pour le civet. Quoi de plus merveilleux que de pouvoir se goinfrer des bienfaits que dame Nature met généreusement à la disposition de ceux qui savent où les trouver, quand d’autres, ignorant tout des mystères de la truffe et la perdrix, doivent se satisfaire de poulet de batterie et autres champignons de couche insipides.

Nous le retrouvons adossé à un chêne centenaire, en train de siroter un café en écoutant chanter les oiseaux et regardant les écureuils sauter de branche en branche, tout en caressant tendrement la tête du lièvre qu’il venait d’abattre et humant à pleins poumons une poignée des girolles qu’il venait de collecter.

Il avait, en cet instant béni des dieux, le sentiment d’appartenir à un club très fermé de privilégiés. Certes, en plus d’un Chapuis de collection et une gibecière Lazzaro Bernardini entièrement fabriquée à la main dans un cuir de très haute qualité, il ne s’agissait somme toute que de quelques hectares (une cinquantaine tout de même) de forêt giboyeuse hérités d’un généreux donateur auquel il avait lui-même fait don d’une bonne partie de sa personne et son existence. Mais tout de même, quelle joie d’être le seul à en profiter, d’en avoir la jouissance exclusive, pouvoir s’y promener en toute tranquillité de corps et d’esprit, même s’il lui arrivait parfois de croiser tel ou tel bipède humain qui s’était indûment introduit dans la place (il aurait pu l’abattre et faire disparaître le corps, comme Bellisle en son temps, mais, toujours magnanime et didactique, près du peuple auquel il avait toujours appartenu, il se contentait généralement de quelques remontrances et un simple avertissement). Certains n’auraient sans doute pas manqué de prétendre qu’une telle attitude était injuste et égoïste, un tel privilège inacceptable, surtout dans le contexte actuel où tant de gens vivaient les uns sur les autres, dans un état de promiscuité qui les poussait à la violence et l’incompréhension, ce à quoi il ne se serait pas gêné pour rétorquer que la vie était une guerre sans merci qui obligeait chacun, même le mieux intentionné d’entre nous, à défendre âprement le moindre centimètre-carré de terrain arraché à l’adversité. Ces quelques hectares de forêt étaient comme une île déserte au milieu d’un océan de béton charriant la fureur mécanique et les miasmes de la modernité. La pollution gagnait du terrain, bientôt lièvres et champignons seraient à tel point chargés d’immondices qu’ils pourriraient sur pied et deviendraient impropres à la consommation. Ensuite, quand toute forme de vie serait à l’agonie, viendraient les bulldozers pour transformer ce havre de paix en terrain vague et laisser le champ libre aux promoteurs et autres marchands de mort par correspondance. Et un jour, après avoir tout détruit, comme c’était déjà le cas dans certaines métropoles saturées de gaz toxiques et d’agressivité, aussi hostiles que les jungles d’antan, l’être humain serait obligé de faire pousser des arbres et du gazon sur le toit des immeubles, à des centaines de mètres de la terre ferme, et une fois de plus il se féliciterait ouvertement de son ingéniosité et sa capacité à survivre, tout contrôler et dominer le monde. En prévision de la disparition des espèces animales, il aura pensé, sur le modèle d’une arche de Noé de laboratoire, à conserver, soigneusement numérotées et étiquetées, les cellules souches qui lui permettront de reconstituer artificiellement un simulacre de vie, cette vie qu’il s’acharne à détruire faute de pouvoir la soumettre entièrement à sa volonté, cette garce impudente et d’un naturel désarmant qui a toujours une longueur d’avance sur lui, se permet de lui damer le pion au moment où il s’y attend le moins, à lui qui est au sommet de la chaîne alimentaire, le prédateur ultime auquel nul ne peut se soustraire, l’égal des dieux dont le souvenir s’efface peu à peu dans les méandres de la mémoire collective en déshérence. Sur ce champ de bataille et de dévastation, jonché des cadavres du vieux monde, des vestiges du passé soigneusement rassemblés dans des sanctuaires payants que les gens font la queue pour aller visiter, se dresse, de toute la hauteur de sa supériorité intellectuelle et son omnipotence technologique, revêtu de l’armure étincelante de son génie créatif, l’Homme victorieux et tout-puissant qui, à l’instar des plus hautes instances de l’univers désormais en perdition, est seul en mesure de définir les contours de l’avenir et assurer brillamment la survie d’une espèce en voie de disparition, la sienne en l’occurrence.

Tandis que, adossé à son chêne, Robert laissait son esprit vagabonder au fil de réflexions dont la hauteur de vue le laissait lui-même pantois, tant il savait son intelligence limitée d’ordinaire, Greta, qui furetait alentour à la recherche d’une piste, avait disparu au détour d’un fourré.

Sachant qu’elle ne tarderait pas à revenir, Robert a continué à siroter son café comme si de rien n’était, l’appelant de temps à autre pour s’assurer qu’elle restait bien dans le périmètre.

Et en effet, quelques instants plus tard, elle a refait surface avec un objet non identifié entre les dents, objet qu’elle s’est empressée de déposer fièrement aux pieds de son maître.

Lequel a dit, en tapotant doucement la tête de l’animal : Bonne fille !

Greta a poussé un petit jappement de satisfaction en sautillant sur place et remuant frénétiquement la queue.

Puis il a ajouté, plus pour lui-même que quelqu’un d’autre puisqu’il n’y avait personne d’autre que lui-même pour l’entendre, à part Greta qui n’avait pas les accréditations nécessaires pour répondre à ce genre de question (il arrive toutefois que certaines personnes s’adressent à leur chien comme à un être humain, avec le sentiment qu’il existe entre eux un lien si fort, intense et mystérieux que même le plus brillant poète, le plus au fait des arcanes du langage et des subtilités littéraires, serait bien incapable de traduire avec des mots) : Qu’est-ce que c’est que ça ?

Greta, si elle avait pu faire usage de la parole, n’aurait pas manqué de répondre que, pour autant qu’on puisse en juger, il s’agissait vraisemblablement d’une vieille basket pourrie calcinée jusqu’à l’os, autrement dit un objet qu’on s’attendait assez peu à découvrir en pleine forêt au milieu des ronces.

Interloqué, Robert s’est levé, a récupéré son barda, et dit à Greta en lui agitant la godasse sous le nez : T’as trouvé ça où, ma fille ?

Greta, même si elle ne parlait pas le français, savait que quand son maître se levait, récupérait son barda et lui agitait quelque chose sous le nez, cela signifiait qu’il voulait en savoir un peu plus sur l’objet en question, à commencer par l’endroit où elle l’avait trouvé.

C’est donc sans hésiter qu’elle s’est mise à trottiner devant lui pour le conduire audit endroit.

L’endroit en question était une sorte de taillis dont une bonne partie avait été réduite en cendres. Au milieu de ce tas de cendres, particulièrement nauséabond et hideux, se trouvait une horreur sans nom qui ressemblait vaguement à un corps humain, recroquevillé sur lui-même dans une expression d’indicible souffrance. Quelques touffes de cheveux et lambeaux de chair calcinée adhéraient encore au crâne noirci. Il en allait de même pour le reste du squelette, lequel présentait ici et là des morceaux de viande semblable à du charbon de bois.

Sous le choc, Robert prenait lentement conscience que sa partie de chasse venait de prendre un tour tragique. Car ici même, dans son bois à lui, sa PROPRIÉTÉ PRIVÉE dûment signalée par de nombreux panneaux d’avertissement de couleur vive, une personne avait été réduite en cendres par une ou plusieurs autres, le tout sans aucune autorisation, dans le plus total mépris de la législation en vigueur et la propriété d’autrui.

Dans le cas présent, faire le 15 était sans objet, la victime ayant peu de chances d’être ranimée.

Même chose pour le 18, les pompiers n’étant plus d’aucune utilité pour circonscrire l’incendie.

Idem pour l’alerte attentat, l’enfance en danger et le sauvetage en mer, autant de situations délicates qui ne cadraient pas avec la configuration présente.

Restait le 17, que Robert s’est empressé de composer sitôt son téléphone en main.

Shirani et moi-même, qui n’avions rien de mieux à foutre à ce moment-là (ou peut-être que si, mais la description du sinistre nous avait paru intéressante sur le coup), sommes arrivés sur le terrain quelques heures plus tard (il n’y avait pas le feu, si j’ose dire, la victime n’en étant plus à quelques minutes près), avec la PTS, bien sûr, ses gants en latex, ses masques chirurgicaux, ses pinces à épiler et ses écouvillons, plus un nombre assez considérable d’agents en uniforme censés passer les alentours au peigne fin (même si, en réalité, on est plus près du troupeau de sangliers qui ravage tout sur son passage).

À propos de Shirani, je dois vous dire que nos rapports dépassaient maintenant très largement le cadre de la simple relation professionnelle. Comme Riggs et Murtaugh dans L’Arme fatale, Mills et Somerset dans Seven, Rizzoli et Isles (parité oblige) ou même Turner et Hooch dans un style plus canin, on était, au fil du temps, devenus des vrais potes qui partageaient des vraies valeurs et se tapaient la cloche dès qu’ils en avaient l’occasion. Un certain nombre de choses nous avaient rapprochés, notamment un goût commun pour la viande froide (avec ou sans mayonnaise) et, pourquoi ne pas le dire au risque de passer pour des cons obsolètes, une appétence marquée pour les belles Italiennes carrossées comme des voitures de sport. Je sais que ce genre de comparaison mécanique ne se fait plus trop aujourd’hui, et on ne peut que se féliciter du fait que les femmes revendiquent haut et fort le droit d’être considérées comme des êtres humains à part entière, et non plus seulement des appareils ménagers ou des objets destinés à satisfaire les pulsions sexuelles de mâles sujets à la violence et la vulgarité. Bon, en même temps, je note quand même que les sexbots et autres love dolls plus ou moins perfectionnés font fureur chez nos amis Nippons, lesquels, on le sait, ont toujours été des pervers de premier plan, des prédateurs faussement timides chez qui l’obsession de la chair fraîche le dispute à un sens de l’honneur et la politesse frisant le ridicule. Cela dit, avec l’Intelligence Artificielle, il faut s’attendre à voir ce genre de compagnon de route, tant féminin que masculin, accéder à un niveau de sophistication et de réalisme de plus en plus élevé, au point de représenter une réelle menace pour l’avenir de l’humanité sur le plan relationnel et affectif. Déjà que c’était pas brillant, ça risque de devenir franchement catastrophique. Imaginez un homme (ou une femme, bien sûr, vu qu’elles sont à présent sensiblement logées à la même enseigne) qui a un four micro-ondes, un lave-vaisselle, un aspirateur robot laveur et une machine à laver : quelque chose me dit qu’il ou elle pourrait bien être tenté de s’offrir un conjoint full option, personnalisable (ou fait main sur mesure pour les plus fortunés, avec des vrais cheveux et poils humains, modèle chauffant pour donner l’illusion de la vie, hyperréaliste, tel qu’il serait pratiquement impossible de différencier le vrai du faux, hormis peut-être un usage limité de la parole et une mobilité un peu raide, comme si le sujet avait un balai dans le cul l’empêchant de se déplacer en toute fluidité), garanti deux ans constructeur, en leasing pour celles et ceux qui aiment changer régulièrement de partenaire, ou achat comptant pour celles et ceux désireux de construire une relation sur le long terme, le modèle pouvant même être équipé d’un logiciel de vieillissement pour ne pas rester éternellement jeune pendant que l’autre vieillit comme un con (contrairement à celles et ceux qui préfèrent se taper des petits jeunes plutôt que des vieux rogatons à leur image). Pour ce qui est des gosses, il est également tout à fait possible de concevoir des copies numériques, permettant de jouir de tous les avantages d’une vie de famille épanouie (rires et cris d’enfants dans une grande maison joyeuse et lumineuse pleine de charme et dotée de tout le confort moderne, grandes tablées les dimanches et jours de fête, jeux à n’en plus finir, histoires pour s’endormir, éventuellement activités pédophiles pour les proches, ces dernières étant difficilement contestables en l’absence de législation morale ou psychologique appropriée, etc) sans avoir à en supporter les inconvénients (enfants malades, chiants, ingrats et mal élevés, capables de se faire la malle à l’improviste, accumuler les conneries à la puberté, voire se retourner violemment contre leurs parents dans les cas les plus extrêmes). On peut aussi, si on tient absolument à en avoir des vrais, se procurer assez aisément des orphelins et autres enfants des rues issus des pays pauvres, et parfois même les acheter (à des tarifs d’autant plus attractifs que le vendeur est aux abois) directement à leurs parents qui en ont des tonnes et ne savent plus quoi en foutre (la contraception n’est pas le fort de ces populations autochtones), d’autant qu’ils n’ont pas les moyens de les nourrir et qu’il est assez pénible pour une mère ou un père, même alcoolique au dernier degré, de voir ses enfants crever de faim sous ses yeux. Bref, c’est un modèle de vie autrement confortable, juste et humain qui se dessine sous nos yeux ébahis dans un horizon pas si lointain, et, serais-je tenté de dire si je ne craignais pas d’être mal interprété, que s’octroyer l’assurance d’une vie sexuelle et conjugale réussie mérite bien quelques coups de canif dans la couche d’ozone et l’apparition de plus en plus fréquente de quelques séismes dévastateurs aux quatre coins de la planète, y compris dans des zones jusqu’ici relativement épargnées comme Malibu, Pacific Palisades et les hauteurs de Beverly Hills. Qu’est-ce qui est préférable ? Éviter qu’une star hollywoodienne voie sa baraque à 100 millions de dollars partir en flammes ou assurer au plus grand nombre une vie sexuelle heureuse et épanouie ?

De toute façon, les stars ont les moyens de s’offrir des assurances à la hauteur, alors que la misère sexuelle n’est couverte par rien (le pauvre bougre hagard qui se branle vingt heures par jour au péril de sa vie, devient aveugle à force de s’esquinter les yeux sur son écran d’ordinateur, accumule les tendinites et finit avec la teub en sang et un rythme cardiaque considérablement dégradé, peut toujours se gratter les couilles pour obtenir le moindre dédommagement de quelque organisme que ce soit, y compris le sien qui n’en peut plus) et conduit le plus souvent à des drames d’une violence inimaginable (qui, en plus, donnent des idées à des scénaristes peu scrupuleux, de véritables rats qui épluchent les tabloïds à la recherche des faits divers les plus sordides) et permettent à ces mêmes stars hollywoodiennes de s’illustrer dans des reconstitutions criantes de vérité, rafler des Oscars à la pelle et engranger toujours plus de pognon pour se taper des escorts à vingt mille dollars la nuit, se faire construire des villas dans des endroits paradisiaques et faire leurs courses en jet privé. Donc oui, je crois, j’affirme, même, qu’il est grand temps de se préoccuper un peu du petit peuple qui souffre en silence dans des bidonvilles voués à la mort et la destruction, tas de tôles infects prenant l’eau de toute part (heureusement qu’il ne pleut pas souvent dans ces coins-là), se nourrissant des rats, cafards et autres punaises qui pullulent entre les matelas éventrés et les amoncellements de déchets en tout genre. Qu’ils aient, au moins, le plaisir de couler des jours paisibles avec une femme aimante et des enfants obéissants, dans un confort relatif, certes, rustique pour ne pas dire préhistorique, mais dans une béatitude sexuelle et affective de tous les instants, un état d’extase conjugale permanent.

La première chose que j’ai fait, en débarquant sur les lieux du drame avec Zaahid Shirani, légiste et néanmoins ami, a été de m’adresser au propriétaire des lieux, Robert Pleimelding, lequel nous attendait à la sortie du bois. L’endroit exact du sinistre n’étant pas facile à trouver, on avait jugé préférable de se donner rendez-vous en un point facilement accessible par voie de locomotion motorisée, en l’occurrence ma somptueuse Kangoo de dix ans d’âge (j’étais, dans un élan de générosité inhabituel, et aussi parce que je n’aime pas spécialement voyager seul, passé prendre Zaahid à la morgue).

\textsc{Moi} : Vous êtes qui, au juste ?

Je le savais déjà mais j’aime bien reposer la question pour faire chier le monde. Le métier de flic n’est pas toujours rigolo, on a les distractions qu’on peut.

\textsc{Robert Pleimelding} : Robert Pleimelding. Je suis le propriétaire des lieux.

\textsc{Moi} : Vous voulez dire que ce bois vous appartient ?

Là encore j’avais parfaitement compris, mais j’appliquais la méthode Columbo (l’imper cradingue, le basset pétomane et le strabisme en moins), qui consiste à se faire passer pour une bille afin d’endormir la méfiance du suspect.

\textsc{Lui} : C’est ça.

\textsc{Moi} : Et c’est vous qui avez découvert le corps ?

\textsc{Lui}, essayant de retenir Greta qui s’était prise de sympathie pour Zaahid et tenait absolument à lui renifler les bas de pantalon (ou alors c’était la vieille odeur de cadavre rance et produits chirurgicaux qui émanait de sa personne) : Oui, c’est moi. Greta, laisse le monsieur tranquille !

\textsc{Zaahid}, qui n’était pas très à l’aise avec les animaux, en particulier les chiens : C’est quoi, comme race ?

\textsc{Robert} : Un drahthaar.

\textsc{Moi} : Un quoi ?

\textsc{Lui} : Un drahthaar, mot allemand qui signifie «~avoir des cheveux en fil de fer~». Un croisement de griffon-kortals et de braque allemand à poil court, si vous préférez.

\textsc{Moi} : Je préfère.

\textsc{Zahhid} : Un très bel animal, en tout cas.

\textsc{Moi} : Ce serait mieux si vous arriviez à le tenir.

\textsc{Robert} : C’est le cas en général. Je ne sais pas pourquoi, mais elle semble très attirée par monsieur.

\textsc{Moi} : Oui, eh bien il se trouve que «~monsieur~» n’aime pas trop les chiens, raison pour laquelle je vous serais reconnaissant de tenir cet animal à distance.

\textsc{Lui} : Au pied, Greta ! Pas bouger !

Le chien est allé s’assoir aux pieds de son maître, tout en continuant à reluquer Zaahid avec la langue pendante et des étoiles plein les yeux.

\textsc{Moi} : Et ça, c’est quoi ?

\textsc{Lui} : Un fusil de chasse.

\textsc{Moi} : Oui, je vois bien que c’est un fusil de chasse. Je vais être obligé de vous le confisquer pour expertise. Vous ne voyez pas que ce soit l’arme du crime !

\textsc{Lui} : Je suis parfaitement en règle !

\textsc{Moi} : Je n’en doute pas. N’empêche qu’il va quand falloir que j’examine votre pétoire, des fois qu’on retrouve des plombs dans le périmètre. Vous avez quoi dans cette sacoche ?

Il a ouvert la sacoche et me l’a collée sous le nez : Voyez vous-même !

\textsc{Moi} : Hun hun. C’est quoi, ça, comme oiseau ?

\textsc{Lui} : C’est pas un oiseau, c’est un lièvre.

\textsc{Moi} : Ah oui, je suis bête ! Et ça ?

\textsc{Lui} : Vous n’en avez jamais vu ?

\textsc{Moi} : Excusez-moi, mais je ne passe pas vie à glandouiller dans la forêt. On dirait des champignons.

Naturellement, je savais parfaitement qu’il s’agissait de girolles. Là encore, je travaillais mon personnage d’idiot de service pour mettre le suspect en confiance et l’inciter à se trahir. Suspect qui n’avait d’ailleurs à priori rien de suspect, si ce n’est qu’il avait soi-disant découvert le corps sur un terrain lui appartenant. Je suis d’accord que ce n’était pas comme s’il l’avait trouvé dans son salon en rentrant chez lui après une dure journée de labeur. Les traces d’effraction qu’on n’aurait pas manqué de remarquer auraient joué en sa faveur (même s’il aurait très bien pu les fabriquer lui-même), et les relevés de scène de crime auraient été autrement plus aisés qu’en pleine forêt. Ici, en dépit des interdictions placardées un peu partout, n’importe qui pouvait introduire à sa guise, en particulier la nuit, et s’adonner en toute tranquillité à des activités peu recommandables. Cela dit, quand quelqu’un appelle les flics pour leur signaler un meurtre, il y a toujours ce vieux réflexe conditionné qui les incite à placer l’intéressé en tête de liste des suspects. On ne manque pas d’exemple où l’assassin, non content d’avoir lui-même attiré l’attention de la police sur l’événement, se montre exagérément coopératif auprès des forces de l’ordre, allant parfois jusqu’à mener sa propre enquête. Cela traduit non seulement une faille psychologique importante, sinon une «~catastrophe psychique~» telle que définie par Racamier dans son article intitulé «~Entre agonie psychique, déni psychotique et perversion narcissique~» publié en 1986 dans la Revue Française de Psychanalyse, mais aussi la volonté de coller au plus près de l’enquête pour tenter de la manipuler à son profit.

\textsc{Pleimelding} : Oui, ce sont des champignons. Des girolles, pour être plus précis. Vous n’allez pas me dire que vous n’avez jamais mangé de girolles ?

\textsc{Moi} : Je ne vous dis rien du tout. C’est vous qui allez me dire des choses, à commencer par l’endroit où se trouve le corps. C’est loin ?

\textsc{Lui} : Une trentaine de minutes, en marchant d’un bon pas.

\textsc{Zaahid}, jetant un œil énamouré sur ses mocassins Guido Cattani en alligator de Louisiane (élevage, en accord avec la convention de Washington sur le commerce international des espèces menacées d’extinction) à cinq mille balles la paire : Personne n’a une paire de bottes à me prêter ?

Perso, je n’avais pas ce genre de problème avec ma vieille paire de pompes columbesques dont la semelle collée par-dessus la jambe par des asiatiques sous-payés se décollait dangereusement sur les bords. Le seul risque, c’était qu’elles lâchent en cours de route et que je me retrouve obligé de gambader en chaussette au milieu des bois, chose qui peut s’avérer romanesque à première vue, mais surtout se révéler extrêmement préjudiciable pour l’intégrité physique de ses pieds.

J’ai dit, à l’intention de Pleimelding qui semblait trouver la situation très à son goût : Laissez votre fusil à l’agent Bescond, ici présent. Thomas, s’il te plaît, tu prends le fusil de monsieur et tu le ranges en lieu sûr. Désolé, monsieur Pleimelding, mais je ne peux pas laisser un civil se balader avec calibre 12 en pleine enquête policière.

\textsc{Pleimelding} : Et si on croise un lièvre ou une perdrix, on fait quoi ?

\textsc{Moi}, lui laissant entrevoir la crosse de mon arme de service rangée dans son holster : Ne vous en faites pas, j’ai ce qu’il faut.

\textsc{Lui} : Oui, sauf que vous n’avez aucune chance de toucher quoi que ce soit avec un engin dans ce genre.

\textsc{Moi} : Eh bien ce sera pour une autre fois, voilà tout. Au cas où vous l’auriez oublié, je vous rappelle qu’on est de la police, pas un groupe de touristes allemands en villégiature sur la Côte. On n’est pas venu chasser la perdrix mais voir ce qui s’est passé dans votre bois, savoir qui a tué qui, quand et comment, et surtout mettre la main sur le responsable de ce carnage. Veuillez remettre votre arme à l’agent Bescond, s’il vous plaît.

\textsc{Pleimelding} : Soit. Mais je vous préviens tout de suite que s’il lui arrive la moindre bricole, éraflure ou quoi que ce soit, je vous traîne devant les tribunaux !

\textsc{Moi} : Pourquoi, il n’est pas assuré ?

\textsc{Lui}, manifestement outré : Bien sûr que si, qu’il l’est ! Mais au cas où vous ne le sauriez pas, l’argent ne fait pas tout, dans la vie.

\textsc{Moi} : Non, je ne savais pas.

\textsc{Lui} : Certaines choses, même si elles sont dépourvues de valeur marchande~-- ce qui n’est pas le cas de ce fusil, je vous le dis tout de suite, peuvent avoir une grande valeur sentimentale. Ça peut être n’importe quoi : une pince à cheveux ayant appartenu à votre mère, les lunettes de votre père…

\textsc{Moi} : Le stérilet de votre grand-mère ou la vieille paire de pantoufles que portait votre grand-père pour fumer sa pipe au coin feu en lâchant des caisses interminables, oui, en effet, il me semble que j’ai déjà vaguement entendu parler de ça. Sinon, sans indiscrétion, ça vaut combien, un joujou de ce genre ?

\textsc{Lui} : Dans les trente mille euros.

\textsc{Moi} : Ah oui, quand même !

Puis, à l’agent Bescond qui n’était pas spécialement connu pour être un modèle de douceur et de délicatesse, ayant toujours préféré le football américain au patinage artistique et le bûcheronnage sportif à la natation synchronisée : Thomas, tu prends le plus grand soin du fusil de monsieur, c’est bien compris ?

\textsc{Bescond} : Oui, chef.

\textsc{Moi} : Veille bien à ce qu’il ne lui arrive rien. Dans le cas contraire, c’est toi qui paieras les pots cassés. Et assure-toi qu’il n’est pas chargé, je n’ai pas envie que quelqu’un se fasse sauter le caisson avec !

\textsc{Lui}, au garde-à-vous : Chef, oui chef !

\textsc{Moi} : Mon cul, je n’entends rien ! Montrez-moi que vous en avez une paire !

\textsc{Lui}, à tue-tête : Chef, oui chef !!!

\textsc{Moi} : Je suis vache, mais je suis réglo. Aucun sectarisme racial ici. Je n’ai rien contre les négros, ritals, rouquins ou métèques ! Est-ce que c’est clair ?

\textsc{Lui} : Chef, oui chef !

\textsc{Lui} : Dites, chef…

\textsc{Moi} : Quoi encore ?

\textsc{Lui} : Vous connaissez la différence entre un rappeur et un campeur ?

\textsc{Moi} : Non, et je m’en fous !

\textsc{Lui}, hilare : C’est pourtant simple, chef : le rappeur nique ta mère et le campeur monte ta tente !

\textsc{Moi} : On se tutoie, maintenant ?

\textsc{Lui} : Chef, non chef ! C’était juste pour…

\textsc{Moi} : Mon cul ! Rompez, agent Bescond. Serez de corvée de chiottes la semaine prochaine !

\textsc{Lui} : Chef, oui chef !

Bescond, non content d’être roux, gras du bide et globalement assez désagréable à regarder (notamment en raison de ce début de calvitie et ce teint rougeaud qui le faisait ressembler étrangement à un ouakari chauve des forêts marécageuses d’Amérique du Sud, un des singes les plus laids du monde), était d’un naturel volontiers frondeur. Ses plaisanteries vaseuses et autres blagues salaces ne faisaient rire personne, ou alors seulement quelques abrutis en phase terminale de décérébration dans son genre, mais je ne pouvais m’empêcher d’éprouver certaine tendresse pour lui, ne serait-ce que par égard pour les efforts considérables qu’il fournissait pour attirer l’attention sur lui. Son répertoire de blagues était impressionnant, surtout pour quelqu’un comme moi qui n’ai jamais été foutu d’en retenir une seule. Je me demandais bien où il allait fourrer tout ça dans sa cervelle de lilliputien. D’autre part, force était de reconnaître qu’il disposait d’une culture cinématographique hors du commun, totalement inattendue de la part d’un type dont le QI ne devait pas dépasser celui du raton laveur. Le cinéma, américain notamment, n’avait aucun secret pour lui, et il connaissait par cœur des répliques entières d’un nombre considérable de films (dont certains, comme Full Metal Jacket, sur lesquels je pouvais lui donner la réplique, étant moi-même assez friand de ce genre de performance). Aussi, au lieu de lui coller des blâmes à répétition pour insulte permanente envers les grands du rire et la vis comica, je me contentais de produire une expression faciale à équidistance entre la grimace et le sourire, au mieux de le remettre gentiment à sa place en le priant d’aller voir ailleurs si j’y étais, chose qu’il s’empressait de faire en courant.

Zaahid, dont les relations avec Greta semblaient être en bonne voie puisqu’il en était maintenant à lui tapoter prudemment (gentiment à la rigueur, amoureusement serait prématuré) le haut du crâne, chose qu’elle semblait apprécier au plus haut point (dès qu’il arrêtait, elle le sollicitait en le poussant du museau pour qu’il remette ça) : Il serait peut-être temps d’aller le voir, ce corps, vous ne croyez pas ?

Pour veiller au grain et s’assurer que personne ne viendrait fourrer son groin dans nos affaires, j’avais laissé deux hommes en faction, dont l’agent Bescond, mon ouakari favori. À voir la tête de cent pieds de long qu’il affichait, ce dernier n’était manifestement pas très content d’être tenu à l’écart des festivités. Il ne s’agissait en aucun cas d’une mesure de rétorsion, mais de précaution. En effet, la dernière fois qu’il avait vu un macchabée (on avait retrouvé un clodo les tripes à l’air dans une benne à ordures, je vous prie de croire que ça ne sentait pas le muguet), il avait tourné de l’œil et mis trois semaines à s’en remettre (il avait fallu l’assistance d’une psychologue particulièrement gironde pour qu’il sorte enfin de sa torpeur). J’estime qu’un chef digne de ce nom se doit de préserver la santé mentale de ses hommes. Celle de l’agent Bescond étant déjà sérieusement entamée de naissance, je me devais de le préserver d’autant plus des chocs traumatiques qui auraient pu le faire basculer définitivement dans la démence.

Après quoi, telle une joyeuse bande de scouts en goguette, notre fine équipe s’est courageusement engagée dans la forêt.

Celle-ci était dense et touffue, à la limite de la pénétrabilité.

Au bout de cinq minutes, j’étais (et nous l’étions tous) déjà complètement paumé.

Heureusement pour nous, notre guide, ex-garde-chasse du comte de Bellisle, connaissait le coin comme sa poche. Il empruntait des chemins de traverse avec une aisance déconcertante, tournant à gauche ou à droite comme s’il avait un GPS à la place du cerveau.

Zaahid et moi marchions devant avec lui, sans oublier Greta qui elle aussi connaissait parfaitement le chemin. Les chiens ont une méthode très efficace pour s’orienter : ils pissent tous les cinq ou dix mètres. Cette méthode, soit dit en passant, est beaucoup plus efficace que celle du Petit Poucet, pourtant le plus malin des sept fils du bûcheron. En effet, je ne sais pas si vous avez déjà essayé de semer des cailloux dans la forêt, mais je peux vous dire qu’il faudrait en semer une sacrée quantité (et je parle de cailloux de grosse taille, plus proches du rocher que du grain de sable) pour espérer retrouver son chemin. C’est là, si je puis me permettre, que le conte de Perrault manque un peu de réalisme, car il aurait fallu, pour que son plan ait une chance de fonctionner, que le Petit Poucet emmène avec lui une pleine remorque de briques. D’ailleurs, là où on voit qu’il n’est pas si malin que ça, c’est quand il tente de rééditer l’expérience avec des petits morceaux de pain. Faut être con pour laisser des petits morceaux de pain dans la forêt, alors que celle-ci regorge d’oiseaux et que tout le monde sait que les oiseaux adorent les petits morceaux de pain (pas seulement les oiseaux, du reste, mais aussi les fourmis et toutes sortes de bestioles dont on ne soupçonne même pas l’existence).

Ce con de Pleimelding avait parlé d’une demi-heure de route, mais compte tenu de la populace qu’il fallait éviter d’égarer à tous les coins de rue, ou de sentier, trois bons quarts d’heure nous ont été nécessaires pour arriver à destination. Inutile de préciser que Zaahid et moi-même, peu habitués à ce genre de pérégrinations champêtres, en avions plein les pattes. La PTS et les agents en uniforme, qui avaient tout donné pour garder le rythme infernal imposé par l’ex-garde-chasse, se trouvaient dans un état similaire, en sueur et perclus de crampes.

Le spectacle n’était pas joli à regarder.

J’ai dit à Zaahid, qui reluquait la scène de crime avec un intérêt certain : Si je te dis «~Tu vois, Julien, c’est ça la chevrotine~», tu me réponds quoi ?

\textsc{Zaahid} : Que je ne m’appelle pas Julien.

\textsc{Moi} : Logique.

\textsc{Lui} : Ouais.

\textsc{Moi} : Et à part ça ?

\textsc{Lui} : À part ça quoi ?

\textsc{Moi} : Cette phrase ne te dit rien ?

\textsc{Lui} : Rien du tout.

\textsc{Pleimelding} : Je suppose que vous voulez parler de Julien Dandieu, dans Le vieux fusil.

\textsc{Moi} : Exact.

\textsc{Lui} : Le petit Julien, âgé d’une dizaine d’années, est parti chasser avec son père.

\textsc{Moi} : Il s’agit d’un flashback, bien évidemment.

\textsc{Lui} : Bien évidemment. Soudain, ils se retrouvent nez à nez avec un sanglier…

\textsc{Moi} : Un énorme sanglier.

\textsc{Lui} : Un sanglier adulte, de taille normale.

\textsc{Moi} : Non, un sanglier énorme, sans doute un de ces vieux mâles acariâtres qui chargent tout ce qui bouge.

\textsc{Lui} : Ils ne sont pas spécialement acariâtres, c’est juste qu’ils ont la vue qui baisse.

\textsc{Moi} : Peut-être, n’empêche qu’ils chargent tout ce qui bouge.

\textsc{Lui} : Oui, bon, si vous voulez. Le petit Julien et son père, donc, se retrouvent nez à nez avec un énorme sanglier…

\textsc{Moi} : Qui, comme je le disais, se met aussitôt à les charger, la bave aux lèvres. Le père prend son fusil, épaule tranquillement, tire et stoppe net le sanglier qui s’écroule le groin dans les feuilles mortes. C’est mieux quand c’est moi qui raconte, vous ne trouvez pas ?

Grimace de Pleimelding qui lui aussi aimait bien raconter les histoires et n’aimait pas que quelqu’un dise que c’était mieux quand quelqu’un d’autre, lui en l’occurrence, le faisait à sa place.

\textsc{Zaahid} : Et c’est que là que le père dit à son fils : «~Tu vois, Julien, c’est ça la chevrotine~».

\textsc{Moi} : C’est ça.

\textsc{Pleimelding}, qui ratait rarement une occasion de ramener sa science : La chevrotine, je le rappelle, est une charge de plombs de gros diamètre destinée à la chasse au gros gibier, le chevreuil en particulier. C’est dévastateur à bout portant, mais dès que la distance augmente l’efficacité diminue. À une époque, pour éviter la dispersion, on liait la charge avec du fil en laiton. Dans un cas comme dans l’autre, les animaux mouraient rarement du premier coup. Il fallait les traquer et s’y reprendre à plusieurs fois pour les exterminer. Ce n’était plus de la chasse mais de la boucherie, raison pour laquelle la chevrotine a progressivement disparu au profit de cartouches à balle unique. Longtemps interdite, elle est de nouveau autorisée dans certains départements, comme la Corse, pour essayer de contenir la prolifération des sangliers. Il y en a partout et ils n’hésitent pas à s’introduire dans les habitation pour faire les poubelles et ravager les potagers. Pour le tir à courte distance, à moins de quinze mètres, la chevrotine est tout indiquée. Il s’agit de régions où tout le monde ou presque a un fusil de chasse. Non seulement la chevrotine est bon marché, mais il n’est pas nécessaire d’être un bon tireur pour faire mouche.

\textsc{Moi} : Bon, ça y est ?

\textsc{Lui} : Excusez-moi. Je voulais juste…

\textsc{Moi} : On n’en a rien à faire, de vos histoires de chevrotine !

\textsc{Lui} : C’est vous qui en avez parlé le premier.

\textsc{Moi} : Je demandais au docteur Shirani, ici présent et manifestement atterré par la nullité de la discussion, si la phrase «~Tu vois, Julien, c’est ça la chevrotine~» lui disait quelque chose. Je me fous de la Corse et des sangliers !

\textsc{Zaahid} : Ne t’énerve pas, Djef.

\textsc{Moi} : Je ne m’énerve pas. C’est juste que cet abruti commence à me taper sur le système à ramener sa fraise à tout bout de champ !

\textsc{Zaahid} : Laisse tomber, et dis-moi plutôt ce que tu voulais me dire avec ton histoire de chevrotine.

\textsc{Pleimelding} : Il voulait parler du \textit{Vieux fusil}, un film de Robert Enrico\nf{Robert Enrico (1931--2001), réalisateur français, auteur notamment du \textit{Vieux fusil} (1975), film sur la vengeance d'un chirurgien dont la femme a été massacrée par les SS en 1944. Cinq César, dont meilleur film, meilleur acteur (Noiret) et meilleur second rôle. \textit{Source :} \textup{fr.wikipedia.org/wiki/Robert\_Enrico}} avec Romy Schneider\nf{Romy Schneider (1938--1982), actrice austro-française, icône du cinéma européen des années 1960--1970. Elle incarne Clara Dandieu dans \textit{Le Vieux fusil} (1975) et reste célèbre pour la trilogie \textit{Sissi} et sa collaboration avec Claude Sautet. \textit{Source :} \textup{fr.wikipedia.org/wiki/Romy\_Schneider}} et Philippe Noiret\nf{Philippe Noiret (1930--2006), acteur français majeur, interprète de Julien Dandieu dans \textit{Le Vieux fusil} (1975, César du meilleur acteur). Également connu pour \textit{Cinema Paradiso} (1988) et \textit{Il Postino} (1994). \textit{Source :} \textup{fr.wikipedia.org/wiki/Philippe\_Noiret}}.

\textsc{Moi} : Vous allez la fermer, oui ou merde ? Prenez un peu de distance, s’il vous plaît. Il s’agit d’une enquête policière et vous n’avez rien à faire là.

\textsc{Lui} : Je croyais que j’étais le principal suspect.

\textsc{Moi} : Pour l’instant vous n’êtes rien du tout. Continuez comme ça et je vous colle les bracelets pour entrave à la justice.

\textsc{Zaahid} : Et alors, c’est quoi cette histoire de vieux fusil ?

\textsc{Moi} : Eh bien des années plus tard, en juin 44, Clara, la femme de Julien Dandieu, alias Romy Schneider et Philippe Noiret comme a eu la délicatesse de nous le rappeler notre hôte, est assassinée par Nazis en déroute, dans le genre Oradour-sur-Glane\nf{Oradour-sur-Glane, village de Haute-Vienne où, le 10~juin 1944, des soldats de la division SS \textit{Das Reich} massacrent 643~habitants, dont des femmes et des enfants enfermés dans l'église et brûlés vifs. Le village martyr est conservé en l'état et classé monument historique. \textit{Source :} \textup{fr.wikipedia.org/wiki/Massacre\_d\%27Oradour-sur-Glane}}.

\textsc{Zaahid} : C’est moche.

\textsc{Moi} : Très moche, oui, d’autant plus que Clara…

\textsc{Pleimelding}, qui avait manifestement décidé d’en finir avec la vie : La femme de Julien Dandieu, n’est pas seulement assassinée mais littéralement massacrée par les SS, qui non contents de la violer sauvagement, finissent par la carboniser au lance-flammes.

J’ai souvent envie de tuer des gens, je l’avoue, mais lui j’avais envie de le découper très lentement avec un couteau émoussé.

Deux agents glandaient dans les parages, que j’ai apostrophés en ces termes choisis : Vous deux, là, venez voir un peu ici.

J’aimerais dire qu’ils ont rappliqué au pas de gymnastique, mais en fait non : ils ont mis trois plombes pour arriver en traînant la patte comme des grands blessés.

\textsc{Moi} : Vous voyez ce monsieur et son animal de compagnie ?

\textsc{Eux} : Oui.

\textsc{Moi} : Vous me les éjectez de la scène de crime, et au trot ! Je ne veux pas les revoir dans le périmètre jusqu’à nouvel ordre, sinon je ne réponds plus de rien !

Aussitôt dit aussitôt fait, les flics ont embarqué Greta et son maître, et j’ai enfin pu continuer à vivre ma vie sans personne pour me chier dans les bottes.

\textsc{Moi}, à Zaahid qui se trouvait présentement à quatre pattes en train de renifler le cadavre : C’est là où je voulais en venir, justement.

\textsc{Lui} : Au lance-flammes ?

\textsc{Moi} : Oui. J’ai déjà vu des gens auxquels on avait foutu le feu, mais ça ne ressemblait pas vraiment ça. T’en penses quoi ?

\textsc{Lui} : Faut voir. Ce qui est certain, c’est que les traces de crémation ne sont pas les mêmes selon la méthodologie adoptée. Si on arrose un cadavre d’essence pour y mettre le feu, il y a de fortes chances qu’on trouve des traces de crémation en dehors de la sphère anatomique stricto sensu.

\textsc{Moi} : Ce qui veut dire, en français ?

\textsc{Lui} : Qu’il y aura des projections de combustible aux alentours, alors que ce ne sera pas le cas si on utilise un lance-flammes. Les choses seront aussi très différentes selon que la personne est brûlée vive ou non.

\textsc{Moi} : C'est-à-dire ?

\textsc{Lui} : Un mort ne bouge pas, alors qu’un vivant aura tendance à se débattre, même s’il est solidement attaché. En conséquence, les traces seront différentes. Et si les poumons sont encore utilisables, ce qui ne semble pas vraiment être le cas ici, on trouvera des traces de suie à l’intérieur.

\textsc{Moi} : Et là, qu’est-ce que tu dirais, comme ça, à vue de nez ?

\textsc{Lui} : Eh bien, après un rapide examen des lieux, j’aurais tendance à dire que la personne était en vie et se tenait debout quand elle est passée à la rôtissoire.

\textsc{Moi} : Putain, comme Romy Schneider dans Le vieux fusil !

\textsc{Lui} : Si tu le dis.

\textsc{Moi} : Sauf que cette fois c’est pas du cinéma !

\textsc{Lui} : Ne t’emballe pas, c’est juste une première impression.

\textsc{Moi} : Et si c’est comme dans Le vieux fusil, peut-être qu’il faut se pencher sur la piste d’un sympathisant d’extrême-droite, un nostalgique du IIIe Reich, de la Schutzstaffel, Das Reich et la division Totenkopf, membre d’un de ces quelconques groupuscules néo-nazis qui poussent comme des champignons vénéneux sur le terreau de la connerie humaine ! Peut-être qu’il a un portrait de Hitler dans sa chambre, fréquente les sites de militaria sur le dark web et écoute des chants nazis dans son sous-sol de sa maison transformé en QG de la Kommandantur. Et dans ce cas, la victime était peut-être juive, communiste, gay, handicapée mentale, ou les quatre à la fois !

\textsc{Lui} : Oui, enfin, je trouve quand même que tu fais une fixette sur les Nazis. Ils ne sont pas les seuls à avoir utilisé le lance-flammes, arme de guerre aujourd’hui passée de mode mais très répandue à l’époque. Les Américains, par exemple, s’en sont servi en Corée et au Vietnam.

\textsc{Moi} : N’empêche que si ça s’est vraiment passé comme je le pense, on aurait tout intérêt à aller fouiner un peu du côté de l’extrême-droite. L’autre jour, j’ai vu un film de Steve Olson avec Logan Price, Finn Bowman et Savannah Ramirez. Dans un futur proche, à Brooklyn, des vampires nazis s’attaquent aux Juifs pour les vider de leur sang une bonne fois pour toutes. Sauf que les Juifs ultra-orthodoxes Borough Park ont retenu la leçon de 45 et ne sont pas du genre à se laisser génocider sans réagir. Or, aux États-Unis, terre promise de la liberté de vendre chèrement sa peau au plus offrant, il se trouve que le lance-flammes est en vente libre, comme d’ailleurs à peu près tout ce qui peut servir à se débarrasser efficacement de son prochain, à part les armes chimiques et les mines antipersonnel, en principe réservées à un usage strictement militaire. Même Elon Musk\nf{Elon Musk (né en 1971), entrepreneur sud-africano-américain, fondateur de SpaceX, Tesla et co-fondateur de PayPal. En 2018, sa société The Boring Company a commercialisé environ 20~000~unités d'un lance-flammes surnommé «~Not a Flamethrower~» à 500~dollars pièce. \textit{Source :} \textup{fr.wikipedia.org/wiki/Elon\_Musk}}, en son temps, vendait des lance-flammes pour le compte de la Boring Company, sa société de construction de tunnels. Il s’agissait plus de briquets géants que de véritables armes de guerre, mais tout de même on était assez proche de l’esprit néo-nazi high-tech dont il semble aujourd’hui se revendiquer, sachant qu’il ferait un SS tout à fait convaincant avec un uniforme sur le dos. Je me souviens qu’à l’époque, la rumeur circulait qu’il préparait une attaque massive de zombies pour booster les ventes de son gadget crématoire. Donc, les Juifs ultra-orthodoxes de Brooklyn se précipitent dans les armureries et font une razzia sur les lance-flammes, la seule arme véritablement efficace pour renvoyer les vampires en enfer. La plupart des vampires sont détruits, mais leur chef, l’ignoble Frozzan, alias le Porteur de la Mort, échappe au carnage et tombe amoureux de la belle Rachel, une jeune juive sublime aux yeux verts qui est le portrait craché de sa femme, la somptueuse Katharina, morte dans des circonstances on ne peut plus dramatiques. Nazi de la première heure et membre de la garde rapprochée du Führer, Frozzan, de son vrai nom Armin Böhmer, assiste en direct à la mort du dictateur sanguinaire, antisémite, raciste et homophobe, et sort très éprouvé de cette terrible épreuve. Il se débarrasse ensuite de son uniforme, enfile des vêtements civils empruntés à Hitler qui n’en a plus besoin et fait sensiblement la même taille que lui, et fonce retrouver sa femme dans leur appartement de la Friedrichstrasse, se frayant tant bien que mal un passage au milieu des cadavres et des gravats. Mais quand il arrive, il trouve la porte défoncée et l’appartement dévasté. Le corps de Katharina, entièrement dénudé et couvert de contusions horribles, repose au milieu du salon, sans vie. Il ne fait aucun doute qu’elle a été sauvagement violée et assassinée par les soudards de l’Armée rouge, horde barbare dont on connaît la violence extrême et l’absence totale de moralité. Fou de rage et de douleur, Böhmer renie Dieu, allant même jusqu’à uriner sur un crucifix jeté au sol dans la tourmente, et jure de revenir d’entre les morts pour se venger et poursuivre l’œuvre du Führer.

\textsc{Zaahid} : Vraiment très intéressant.

\textsc{Moi}, sachant pertinemment qu’il était en train de se foutre de ma gueule : Ouais, et encore plus quand on sait que c’est Logan Price et Savannah Ramirez qui jouent Frozzan et Rachel. Tu connais Savannah Ramirez ?

\textsc{Zaahid}, examinant la scène de crime à travers une loupe à très fort grossissement, d’une voix de fausset censée exprimer toute l’étendue de son désintérêt pour la question, en plus d’une ironie cruelle à mon encontre : Non, je n’ai pas ce plaisir.

Il y a des gens qui vont très souvent au cinéma, d’autres une fois de temps en temps, d’autres jamais. Zaahid appartenait à la troisième catégorie, laquelle regroupe des gens qui, sans être à proprement parler des hégéliens pur sucre de betterave, estiment toutefois que le cinéma, au même titre que d’autres disciplines telles que le macramé, la poterie, l’origami, Photoshop, les châteaux de sable, la cuisine de bonne femme, l’opérette, la charcuterie fine et la natation synchronisée (pour n’en citer que quelques-unes parmi les plus populaires), n’appartiennent pas à la catégorie des beaux-arts au sens noble du terme. Zaahid, par exemple, n’avait même jugé utile de se pourvoir d’un poste de télévision. Il considérait ce genre de lucarne sur le monde comme une calamité pour la paix intérieure et la connaissance profonde de soi, seul véritable chemin d’accès à la connaissance d’autrui et l’espérance d’une vie harmonieuse en société. Pour beaucoup de gens, aux ambitions plus modestes, Zaahid n’était qu’un crétin prétentieux, un de ces intellectuels autoproclamés et suffisants qui traitent les autres comme de la merde et considèrent qu’il n’y a aucune voie salutaire en dehors de celle qu’ils ont choisi de suivre.

\textsc{Moi} : Tu as tort, c’est une très belle femme.

\textsc{Lui} : Il y en a plein les rues, des très belles femmes. Et je les préfère en chair et en os.

\textsc{Moi} : Donc tu te fiches complètement de ce que je te raconte.

\textsc{Lui} : On peut dire ça.

\textsc{Moi} : Et tu ne veux surtout pas connaître la fin du Vampire de Borough Park.

\textsc{Lui}, évoluant comme un poisson dans l’eau au milieu des agents de la PTS affairés à récolter des indices : Au cas où tu ne l’aurais pas remarqué, je suis un peu occupé.

\textsc{Moi} : À ne surtout pas confondre avec Un vampire à Brooklyn de Wes Craven, avec Eddie Murphy dans le rôle d’un vampire caribéen qui débarque à Brooklyn pour assurer sa descendance. Mais je suppose que tu n’as jamais entendu parler de Wes Craven.

\textsc{Lui} : Bingo !

\textsc{Moi} : Ni de Beverly Hills Vamp, un navet qui met aux prises trois crétins patentés avec des putes vampires dans un bordel de la ville.

\textsc{Lui} : Ça me ferait mal !

\textsc{Moi} : Oui, eh bien quoi qu’il en soit, la fin du Vampire de Borough Park est assez inattendue, pour ne pas dire totalement inhabituelle. Comme je te l’ai dit, Frozzan et Rachel tombent amoureux l’un de l’autre, et il décide de faire d’elle sa compagne pour l’éternité, perspective que l’élue de son cœur accueille avec une joie sans limite mêlée quand même d’un soupçon d’appréhension. D’autant qu’il leur faut en permanence déjouer les pièges d’Obadiah McClelland, le chasseur de vampires de service, interprété par un Finn Bowman en grande forme, à mon humble avis bien meilleur que dans Les Barons de l’Aube de Trevor Wilson ou La Forêt maudite de Cameron Delgado, même si ça reste un acteur de série B.

Je me suis arrêté un instant de parler afin de mesurer l’impact de mes propos (j’estimais avoir fait un étalage assez encyclopédique de mes connaissances sur le sujet) sur l’homme qui se disait mon ami et que je considérais comme tel, mais, à en juger par l’attention qu’il me portait, cet impact n’excédait pas celui d’un moucheron percuté de plein fouet par le pare-brise d’un véhicule à l’arrêt.

Un constat d’une grande tristesse pour moi, mais il en fallait heureusement (pour vous surtout, qui rêvez de connaître la fin de l’histoire) un peu plus pour me faire taire.

C’est donc comme si de rien n’était que j’ai repris le cours de mon récit, faisant fi de l’indifférence de mon ami et néanmoins légiste fort peu cinéphile, présentement occupé à exhorter ses troupes (les agents de la PTS, ndlr) à récolter un maximum d’indices sur et autour du cadavre calciné : Bref, je te passe les détails, mais, pendant une nuit d’amour torride, Frozzan plante ses crocs acérés dans la gorge frémissante de Rachel qui exulte de bonheur. Je précise que cet enfoiré de McClelland, adepte de pratiques sexuelles déviantes, assiste à une bonne partie de la scène en regardant à travers le trou de la serrure de la chambre nuptiale, et que c’est seulement quand il a satisfait ses coupables penchants qu’il se décide à enfoncer la porte. Oui, je ne te l’ai peut-être pas dit, mais il y a une bonne dose d’érotisme dans ce film. Par exemple, quand Frozzan plante ses crocs dans la gorge de Rachel, elle est entièrement nue et il en profite pour lui malaxer copieusement la poitrine et la pénétrer sauvagement avec sa vieille queue toute rongée par les asticots qui manque de tomber en miettes.

Regard appuyé en direction de Zaahid pour mesurer l’effet produit par mes déclarations, lequel Zaahid, sentant le poids de mon regard sur sa nuque indifférente, tourne la tête et pose sur moi un œil rendu quasi vitreux par l’ennui et la condescendance. Je pensais qu’il allait au moins dire un mot, me faire part sans ambages de ses sentiments sur l’univers passionnant que je m’efforçais patiemment de lui faire entrevoir, mais sa tête est repartie dans l’autre sens et retournée à ses occupations sans prononcer la moindre syllabe, à tel point que j’ai, l’espace d’un instant, douté de l’opportunité de poursuivre mon récit.

\textsc{Moi} : Donc, comme je te le disais, McClelland enfonce la porte et se rue dans la chambre, le lance-flammes à la main, le visage atrocement déformé par la haine et la concupiscence, bientôt rejoint par un groupe de Juifs de Borough Park parmi les plus remontés, eux aussi armés de lance-flammes et bien décidés à réduire Frozzan en cendres. Mais en découvrant Rachel dans ses bras, entièrement nue, ils marquent un temps d’arrêt bien compréhensible. Du coup, Frozzan et Rachel, qui est devenue elle aussi un vampire, en profitent pour se transformer en chauves-souris et s’enfuir à tire-d’aile par la fenêtre de la chambre.

\textsc{Zaahid}, qui était d’ordinaire quelqu’un de plutôt calme et réservé : Tu commences à m’emmerder, avec tes histoires de chauves-souris !

\textsc{Moi} : Ah bon ?

\textsc{Lui} : Oui.

\textsc{Moi} : C’était juste pour détendre l’atmosphère.

\textsc{Lui} : Eh bien va la détendre ailleurs. Je comprends que tu sois stressé, ce n’est pas tous les jours qu’on assiste à un spectacle de ce genre, mais j’ai besoin de calme pour travailler.

\textsc{Moi} : Excuse-moi, je pensais bien faire.

Je mentais, bien sûr, je n’en avais rien à cirer de faire bien ou mal, et la vue du cadavre n’exerçait sur moi aucune pression particulière. C’est juste que je m’emmerdais et cherchais un moyen de faire passer le temps. Mais je voyais bien que Zaahid était concentré sur sa tâche, et je pouvais aisément comprendre que le bruit de fond incessant de mes litanies vampiriques pouvait lui taper sur le système, comme d’ailleurs celui de toute personne normalement constituée qui aurait dû les supporter sans piper mot.

Du coup, je me suis rabattu sur Pleimelding, que je savais cinéphile, et l’ai trouvé en pleine conversation avec le flic chargé de le surveiller. Quand je dis «~conversation~», il s’agissait plutôt d’un long monologue évoquant par le menu quelques-unes de ses plus mémorables parties de chasse. Le malheureux fonctionnaire, réduit à l’impuissance par le flot ininterrompu de paroles qui s’abattait sur lui, ne pouvait pas en placer une, et j’ai lu dans son regard éploré qu’il espérait que ma venue allait enfin mettre un terme à son calvaire. Au lieu de ça, il a dû rapidement se faire à l’évidence que ce qu’il venait d’endurer n’était qu’une pâle mise en bouche par rapport au plat de résistance que je m’apprêtais à lui servir.

Pleimelding, que j’ai coupé sèchement alors qu’il rabattait (il revivait la scène avec une intensité confondante) un douze cors vers son employeur, le comte Léopold Chiasson de Bellisle, propriétaire du château du même nom et de la majeure partie des terres environnantes, y compris la forêt où nous nous trouvions en ce moment même, s’est au contraire montré très intéressé par mon histoire de guerre des gangs entre vampires nazis et Juifs orthodoxes de Brooklyn. Poli, je lui ai d’abord demandé s’il s’intéressait au cinéma. Il m’a répondu que oui, preuve en était sa parfaite connaissance du Vieux fusil de Robert Enrico. Je lui ai fait remarquer qu’il s’entendrait bien avec l’agent Bescond, le rouquin gras du bide que j’avais laissé en faction à l’orée du bois et chargé de veiller à ce que son précieux fusil ne s’évapore pas dans la nature. Ce à quoi il a rétorqué que si tout le monde connaissait Jean Bouise, Romy Schneider et Philippe Noiret, tous d’excellents acteurs au demeurant, ces mêmes soi-disant cinéphiles se fichaient comme d’une guigne de savoir qui était Christian Teyras, second rôle à la notoriété confidentielle, certes, mais au talent bien réel. Car c’est lui, n’en déplaise à ces pisse-froid de la pellicule, qui joue le père de Julien Dandieu et prononce la phrase restée dans toutes les mémoires ou presque : «~Tu vois, Julien, c’est ça la chevrotine~». Je l’ai félicité, et avant qu’il ait eu le temps de me donner des détails sur la vie et l’œuvre de Christian Teyras (dont la carrière se résumait à une poignée de films tombés dans l’oubli), j’ai enchaîné avec mon histoire de guerre des gangs new-yorkais, avec d’un côté les vampires nazis de l’ancien SS Armin Böhmer, alias Frozzan, le Porteur de la Mort, et de l’autre les Juifs orthodoxes de Borough Park conduits par Zachariah Matusevitch, le chef de la communauté, assisté dans cette dure épreuve par Obadiah McClelland, spécialiste du paranormal, expert en sciences occultes et chasseur de vampires unanimement respecté.

Depuis que mon ami Grégoire Lussier, ancien analyste M\&A chez Reckless \& Knot (un scandale financier retentissant avait contraint la banque à fermer la majeure partie de ses succursales avant d’aller se refaire une santé aux Bahamas), avait ouvert son cabinet de détectives privés, les affaires marchaient plutôt bien.

Il s’était, pour faire plus sérieux (et accessoirement se sentir moins seul), adjoint les services de deux hommes de valeur, eux aussi anciens de chez Reckless \& Knot, qui partageaient la même passion que lui pour les affaires douteuses, les histoires glauques et les polars old school : Romuald Pueyrredon, dit Le Grand Acquisiteur (rapport à la fusion-acquisition, la synergie de croissance et les magouilles financières dont il maîtrisait parfaitement l’architecture sophistiquée), et Mathéo Baleya, surnommé La Balayette car il n’avait pas son pareil pour faire le ménage sur le marché, en d’autres termes se débarrasser de la concurrence.

Rien de plus normal, quand on est curieux de nature et un peu voyeur sur les bords, d’aimer fourrer son nez dans les affaires des autres, remuer la merde pour faire remonter les mauvaises odeurs, mais il arrive parfois, pour des raisons que j’ignore et sur lesquelles je continue à m’interroger avec toute la puissance cérébrale qui me caractérise, que certains, sans doute plus sanguins et moins ouverts d’esprit que d’autres, dans l’impossibilité de prendre du recul et contrôler leurs émotions, en prennent ombrage au point de se livrer à des gestes regrettables. C’était, par exemple, le cas de Yiorgos Panayiotou, qui prenait facilement ombrage de tout et n’importe quoi, y compris les choses les plus minimes, et avait tendance à accumuler des gestes dont le niveau de regrettabilité se situait toujours très au-dessus de la moyenne. En effet, non content de s’adonner au trafic de stupéfiants et autres activités peu recommandables, ce ressortissant chypriote affichait un penchant maladif pour les femmes des autres, lesquels autres n’avaient pas toujours les ressources nécessaires pour prendre la chose avec philosophie. Beau parleur, séducteur, toujours propre sur lui, bien mis de sa personne, le sourire facile, il emballait tout ce qui bouge à la vitesse d’une araignée. En quelques heures, la victime se retrouvait solidement emmaillotée dans un cocon affectif duquel il lui était impossible de s’extraire. C’est dans le cadre d’une enquête sur ce Yiorgos Panayiotou que Greg avait fait l’objet d’une blessure par balle, tirée par ce même Yiorgos, par chance aussi mauvais tireur que citoyen. La balle, en dépit des allégations de Greg qui prétendait avoir vu la Mort en face, croisé son regard de glace dissimulé sous un grand capuchon noir, n’avait fait qu’effleurer le bras, entraînant une ITT inférieure à huit jours.

Ce jour-là, à quinze heures, Greg avait rendez-vous avec une certaine Sally Robinson.

Qui était cette Sally Robinson ?

Aucune idée.

Pourquoi avait-il rendez-vous avec elle ?

Parce qu’elle l’avait appelé et qu’elle avait insisté pour prendre rendez-vous.

Pourquoi ?

Il s’agissait apparemment d’une affaire urgente de la plus extrême gravité.

Et il avait accepté ?

Oui, il avait fini par accepter de la rencontrer, même si ça ne l’arrangeait pas car il avait déjà assez de boulot comme ça, surtout que le boulot et lui ça faisait deux et qu’il s’en tenait généralement au strict nécessaire pour ne pas crever de faim.

Et qu’est-ce qu’il bouffait généralement ?

Caviar, foie gras, truffe, des choses de ce genre, copieusement arrosées des meilleurs vins et alcools.

Pas donné, donc.

Non, pas donné, raison pour laquelle il devait quand même bosser un minimum pour assurer ce train de vie de ministre.

Et pour laquelle il avait accepté de recevoir cette Sally Robinson, même s’il trouvait ce nom bizarre et n’avait à priori aucune envie de se retrouver assis face à elle dans son bureau.

En effet.

Est-ce qu’elle était à l’heure, au moins ?

Oui, avec un sens de l’exactitude quasi helvétique, et je ne vous cache pas que c’était là un excellent point pour elle, car même si le détective a l’habitude de passer des heures à poireauter, et pas toujours dans les meilleures conditions (souvent sous la pluie avec de fortes rafales de vent, pas tellement parce qu’il ne pourrait pas trouver un endroit pour attendre au sec, mais parce que la tension dramatique est d’autant plus intense que les conditions météorologiques sont désastreuses, que les éléments eux-mêmes semblent se déchaîner au rythme de l’action et traduire par leur mauvaise humeur les tourments intérieurs du héros et le destin tragique qui lui pend au nez), Greg avait horreur des gens qui ne respectent pas scrupuleusement les horaires établis. Lui-même était toujours excessivement à l’heure, à la seconde près. Il ne le faisait pas spécialement par respect pour l’autre, dont il n’avait la plupart du temps pas grand-chose à secouer, mais parce que l’espèce d’angoisse sourde et obsédante qui l’habitait en permanence lui interdisait formellement de prendre la moindre liberté avec une chose aussi fluide, insaisissable et implacable que le temps. Le temps ne se voit pas mais ses effets sont bien réels. Ils se nomment décrépitude, mort, décomposition, disparition.

Heureusement, Sally Robinson était à l’heure et Greg a pu échapper au supplice infernal des idées noires qui, dans le cas où elle serait arrivée avec ne serait-ce que deux ou trois minutes de retard, n’auraient pas manqué de tourner dans son crâne tel un essaim de mouches à merde autour d’une bouse fraîchement pondue. Naturellement, il n’aurait pas fallu non plus qu’elle arrive en avance, ce qui aurait, de fait, placé Greg dans une situation de retardataire d’une flagrante injustice. Toute sa construction mentale de la journée en cours en aurait été irrémédiablement perturbée, mise à mal, et finalement détruite, tant il est impossible, même en courant vite, de rattraper la moindre seconde de temps perdu. Tout comme il est impossible d’en gagner, du reste, d’où la précision extrême que l’on se doit d’adopter si on tient à rester dans le jeu (de dupes, bien évidemment, car on finit toujours par perdre, quels que soient les trésor d’ingéniosité que l’on développe pour s’en sortir).

Sally Robinson n’était pas à proprement parler ce qu’il est convenu d’appeler une belle femme, au grand dam de Greg qui s’attendait à voir débarquer Rhonda Fleming, Lana Turner, Gene Tierney, Lauren Bacall ou Sharon Stone.

Ne vous en déplaise, Sally Robinson n’avait rien de commun avec ces femmes fatales qui ondulent dans les romans de Chandler, Hammett, Goodis et les autres, chantent comme des sirènes et entraînent les marins d’eau douce dans les profondeurs glacées du désespoir et la décrépitude la plus totale, tant physique que morale. Si tous les gens qui croisaient le chemin de Sally Robinson avaient envie de l’éradiquer, à commencer par les femmes dont elle dégradait sérieusement l’image, c’est parce que la société se présente sous la forme d’un organisme savamment structuré et se comporte comme tel, à savoir qu’il cherche à éliminer ou expulser tout corps étranger dont il détecte la présence en son sein, expurger sans ménagement de la surface de son épiderme satiné toute excroissance disgracieuse ou furoncle gorgé de pus. Et je vous prie de croire que la présence de Sally Robinson avait été détectée, depuis longtemps, et que tous les anticorps disponibles avaient été mobilisés pour tenter de venir à bout d’un des agents pathogènes les plus redoutables de toute l’histoire de l’humanité, au moins sur le plan de l’esthétique et du bon goût, de la joie de vivre et la culture générale.

Vous vous dites : bon sang, mais où va-t-il chercher tout ça !

On a vu des gens moches, certes, mais quand même pas au point de se poser sérieusement la question de savoir si le seul fait de leur présence sur terre représente une réelle menace pour la survie de l’espèce qui nous tient le plus particulièrement à cœur : la nôtre.

Oui, eh bien si vous pensez vraiment que je pousse le bouchon un peu loin, je vous suggère de procéder à l’expérience suivante : essayez d’imaginer Danny DeVito avec des oreilles de Mickey et les nichons de Pamela Anderson.

Si vous y parvenez (ce qui en soi représente déjà un exploit non négligeable), dites-moi franchement quelle est la conclusion qui vous vient le plus spontanément à l’esprit.

Alors ? Vous êtes d’accord avec moi qu’un tel niveau de laideur est difficilement acceptable dans le monde aseptisé qui le nôtre, et pourrait, si on le laissait aller et venir en toute liberté, représenter une source de traumatisme irréversible pour les plus fragiles d’entre nous, catégorie à laquelle, heureusement pour lui, Greg n’appartenait absolument pas, même si son premier réflexe avait été de prendre ses jambes à son cou et sauter dans le premier avion en partance pour nulle part.

Il s’est dit : pas de panique, mon garçon, on est au vingt-et-unième siècle ! Ce n’est pas le siècle des lumières, plutôt celui de l’extinction des feux, mais un vrai professionnel se doit de garder la tête froide en toute circonstance.

Aujourd’hui on change de sexe comme de chemise, ou de capote, et les enfants pourront bientôt porter plainte contre leurs parents pour les avoir fait naître avec des attributs inappropriés, obtenir une rente à vie en réparation des souffrances psychologiques endurées et passer le restant de leurs jours à se dorer la pilule sur une plage de sable fin. Depuis le temps qu’on raconte que les enfants sont des petites choses fragiles qu’il faut manier avec d’infinies précaution, il fallait bien s’attendre à ce qu’ils nous reprochent jusqu’à la taille de leur sexe ou la couleur de leurs yeux. Dieu merci, la science peut réparer quelques erreurs tragiques de ce genre, mais il s’agit la plupart du temps d’un aller simple et le résultat n’est pas garanti sur facture. L’être humain, je l’ai déjà dit, livre un combat sans merci contre la nature, et ne se gêne plus pour remettre ses décisions en cause et corriger le tir en cas de besoin. Sa volonté de puissance est telle qu’il ne tolère plus que qui ou quoi que ce soit décide à sa place de ce qui est bon ou pas pour lui. La nature, qui jusqu’ici faisait autorité, est aujourd’hui contestée sans ménagement. Contrairement à Dieu, qui par essence ne fait pas d’erreur et ne peut donc faire l’objet d’aucune remontrance, la nature, qui est en quelque sorte sa version laïque, encaisse son lot de critiques et agressions caractérisées. Il faut vraiment qu’elle se fâche tout rouge pour que l’être humain commence à se poser des questions sur le bien-fondé de ses agissements, entrevoir l’idée que quoi qu’il fasse, quel que soit le sentiment de supériorité et l’autosatisfaction qui gonfle sa poitrine d’orgueil, il aura toujours une longueur de retard face aux forces telluriques qui lui ont, dans un moment d’aberration, de bug cosmique, donné le jour.

Nul doute que pour certains, Sally Robinson représentait une forme d’évolution ultime de l’être humain, de prototype d’un nouveau genre, une avancée significative sur la voie de la perfection et la guérison de tous les maux.

Pour Greg, elle (ou il, tant il était difficile de qualifier avec certitude cette créature mutante, fruit pourri tombé de l’arbre de la folie) représentait surtout une potentielle source de revenu qui, au même titre que l’investissement locatif, l’hypnothérapie et les cryptomonnaies, méritait d’être considérée avec tout le respect dû à son rang. Certes, il était assez compliqué, en voyant entrer ce modèle réduit (sa taille ne devait pas excéder le mètre cinquante) surmonté de la tête d’un homme de quarante-cinq ans, laquelle tête affichait, outre une calvitie déjà bien avancée, un double menton prononcé, des oreilles de Mickey, des sourcils épilés et une bouche tartinée de lipstick rose et satiné (intense et hydratant, parfait pour un look tendre et décontracté de tous les jours, les parties de pêche en mer, la cueillette des champignons et les soirées karaoké, alors que les rouges crémeux, plus mats et parfois collant au point de ne plus pouvoir décoller le bec de l’endroit où il se pose, sont à réserver pour les grandes occasions telles que mariage, barmitsva, première communion, enterrement de vie de garçon, proche décédé ou toute autre chose susceptible d’être enterrée, etc), il était assez compliqué, disais-je, en voyant entrer cette créature improbable et mal fagotée tout droit sortie de Queer Eye for the Straight Guy, de ne pas éclater de rire et se rouler par terre en se tordant douloureusement les côtes, surtout si on avait le malheur de poser les yeux sur les boucles d’oreille en forme de crucifix ou l’énorme paire de nichons que Sally Robinson trimballait devant elle avec une fierté manifeste. À ce stade, on frisait la provocation pure et simple.

Sally Robinson, drapée de tissu rose cochon en forme de costard trois fois trop grand et nimbée d’un parfum de tubéreuse à décimer un régiment de sapeurs-pompiers : Bonjour. Je suis Sally Robinson.

«~En v’là une nouvelle qu’elle est bonne~» s’est dit Greg en lui tendant la main : Grégoire Lussier. Entrez, je vous en prie.

La chose est entrée, et son parfum, plus efficace qu’une bonne rasade d’insecticide surpuissant, a laissé une traînée de mouches mortes dans son sillage.

Elle se déplaçait, avec une élégance toute relative, sur des chaussures à talons sensiblement de la même couleur charcutière que son costume ridicule.

\textsc{Greg}, souriant de ses plus belles dents, lui désignant un des deux fauteuils réservés à la clientèle : Asseyez-vous, je vous en prie.

\textsc{Sally}, prenant place dans le fauteuil en tortillant du cul tel un cycliste en pleine ascension du col du Galibier : Je vous remercie.

«~Mais de rien, ma bonne dame~» s’est dit Greg en allant s’assoir en face d’elle, devant le bureau sur lequel s’entassaient une foule de dossiers multicolores, vides pour la plupart, mais donnant l’impression qu’il passait ses journées et une bonne partie de ses nuits à faire don des plus belles années de sa vie à une clientèle toujours plus nombreuse et exigeante : Bien, madame Rob….

\textsc{Sally}, passant une main soigneusement manucurée sur la maigre touffe de cheveux qui garnissait encore (mais sans doute pas pour longtemps) l’arrière de son crâne : Mademoiselle.

\textsc{Greg}, exhibant à nouveau les plus beaux fleurons de sa dentition : Mademoiselle Robinson, pardon. Puis-je savoir ce qui vous amène ?

\textsc{Sally} : Eh bien voilà : j’ai un ami qui a disparu.

\textsc{Greg} : Un ami ?

\textsc{Sally} : Oui, un ami.

\textsc{Greg} : Quel genre d’ami ?

\textsc{Sally} : Un ami proche, suffisamment pour que je m’inquiète de sa disparition.

\textsc{Greg} : Je vois. Et je suppose que vous voudriez que je le retrouve ?

\textsc{Sally} : C’est ça.

\textsc{Greg} : Depuis combien de temps a-t-il disparu ?

\textsc{Sally}, des sanglots dans la voix, une voix de fausset qui sonnait comme celle d’un enfant dans le corps d’un adulte : Trois semaines jour pour jour.

\textsc{Greg} : Vous voulez boire quelque chose ?

\textsc{Sally} : Oui, je veux bien.

Greg s’est levé, dirigé vers le bar, est revenu avec une bouteille de Jack Daniel’s de deux verres : Je n’ai que ça. J’espère que ça fera l’affaire.

\textsc{Sally}, mouchant son gros nez avec des grâces de pucelle effarouchée : Ce sera parfait, merci. Excusez-moi, je suis ridicule.

«~Ça tu peux le dire, ma grosse ! Ton rimmel va couler si tu continues à chialer comme une midinette~» s’est dit Greg en remplissant les verres : Je vous prie, je vois bien que la situation n’est pas facile pour vous. Tenez, buvez ça, vous vous sentirez mieux après.

\textsc{Sally} : Merci, vous êtes vraiment très aimable.

\textsc{Greg}, intérieurement : «~Tu parles, Charles !~»

\textsc{Extérieurement} : Donc, vous dites que cette personne a disparu il y a trois semaines jour pour jour, c’est bien ça ?

\textsc{Sally}, après avoir englouti cul sec son verre de Jack Daniel’s sans sourciller : Oui. J’avais rendez-vous avec Tiago au Sugar \& Spice, mais il n’est jamais venu.

\textsc{Greg} : Tiago ?

\textsc{Sally} : Tiago Alvarez, la personne qui a disparu.

\textsc{Greg} : D’origine espagnole, je suppose.

\textsc{Sally} : Brésilienne. Vous connaissez le Sugar \& Spice ?

\textsc{Greg} : Non, ça ne me dit rien.

\textsc{Sally} : C’est un cabaret, au 127 rue Théo Cazenave.

\textsc{Greg}, sirotant son Jack Daniel’s en affichant un air profond de type chez qui la moindre syllabe prononcée déclenche un tsunami de pensées complexes qui s’entrechoquent comme des électrons dans un accélérateur de particules : Je vois.

Pour l’instant, tout ce qu’il voyait, c’était qu’une tante n’était pas venue à son rendez-vous avec une autre tante dans un repaire de tantes de la rue Théo Cazenave.

\textsc{Sally}, contemplant d’un œil morne le verre vide qu’elle tenait toujours à la main, un filet de morve au coin du nez : Cela se passait il y a trois semaines jour pour jour, et depuis je n’ai plus aucune nouvelle.

\textsc{Greg} : Je vous en sers un autre ?

\textsc{Sally} : Oui, c’est pas de refus.

À ce train-là, la bouteille ne ferait pas de vieux os.

Greg s’est exécuté, puis il est allé remettre la bouteille dans le placard pour bien signifier à mademoiselle Sally Robinson qu’il ne fallait pas confondre son officine avec un débit de boissons.

\textsc{Greg} : Il ne répond pas au téléphone ?

\textsc{Sally}, le nez dans le Jack Daniel’s : Non. J’ai laissé des tonnes de messages, en vain. Aujourd’hui son téléphone ne sonne même plus. Je vous l’ai dit : nous sommes très proches. Jamais il ne resterait trois semaines sans donner de nouvelles. J’ai la certitude qu’il lui est arrivé quelque chose, quelque chose de grave.

\textsc{Greg}, toujours compatissant : Vous ne pensez pas qu’il aurait pu… comment dire… s’évaporer dans la nature ?

\textsc{Sally} : Tiago n’avait aucune raison de s’évaporer, comme vous dites. Tout allait bien pour lui. J’aurais été le premier à le savoir s’il avait eu le moindre problème.

\textsc{Greg} : Vous êtes allé trouver la police ?

\textsc{Sally} : Non, je sais comment les flics traitent les gens dans mon genre. Je n’ai pas envie qu’ils se foutent de ma gueule, et je sais très bien qu’ils ne lèveront pas le petit doigt pour retrouver Tiago. Un pédé de plus ou de moins, c’est le cadet de leurs soucis !

Greg s’est renversé dans son fauteuil, paupières mi-closes, a hoché la tête, affiché son sourire numéro 12 (réservé aux situations délicates exigeant une grande finesse d’esprit, des nerfs d’acier et une joie de vivre à toute épreuve), puis il a plongé ses yeux (bleus, Greg était blond, grand, mince, tout le contraire de Sally) droit dans ceux de son interlocuteur, l’a fixé quelques instants sans desserrer les dents, avant de déclarer, de cette voix grave et onctueuse qui avait le pouvoir de placer celui ou celle qui l’entendait dans un état de bien-être proche de l’extase : Je comprends. Et il fait quoi, ce Tiago, dans la vie ?

\textsc{Sally} : Ambulancier.

\textsc{Greg} : Oui, ce n’est pas comme si il travaillait à la DGSE. Des problèmes de drogue, alcool ou autre ?

\textsc{Sally} : Il sniffe un peu de temps à autre, comme tout le monde. Rien de bien méchant.

\textsc{Greg} : Dois-je comprendre que vous-même…

\textsc{Sally} : Comme tout le monde.

\textsc{Greg} : Tout le monde ne sniffe pas un peu de temps à autre.

\textsc{Sally} : Vous devriez essayer. Passez donc me voir au Sugar \& Spice, j’y suis presque tous les soirs.

\textsc{Greg} (sourire numéro 3, crispé pour exprimer une gêne certaine, mais sans excès pour ne pas froisser le vis-à-vis) : J’y penserai à l’occasion. Pour en revenir à Tiago, vous ne lui connaissez pas d’ennemi, de gens qui pourraient avoir des raisons de lui en vouloir.

\textsc{Sally} : Non. C’est un garçon charmant, très ouvert…

\textsc{Greg} : Je n’en doute pas. Et vous-même ?

\textsc{Sally} : Moi aussi je suis très ouverte…

\textsc{Greg} : Non, je veux dire vous-même, vous faites quoi dans la vie ?

\textsc{Sally} : Je suis artiste de music-hall, je travaille au Sugar \& Spice.

\textsc{Greg}, après s’être raclé le fond de la gorge : Ah très bien ! Vous chantez ?

\textsc{Sally} : Oui, et je danse, aussi.

\textsc{Greg} : Bien, parfait. Au sujet de Tiago, je suppose que vous vous êtes renseignée auprès de son employeur.

\textsc{Sally} : Bien sûr, mais il n’en sait pas plus que moi. Personne ne sait rien, et tout le monde est très inquiet parce que Tiago n’est pas du genre à disparaître comme ça du jour au lendemain, sans donner de nouvelles à qui que ce soit. Il est très proche de sa mère, par exemple, et elle non plus n’a pas de nouvelles. Elle voulait appeler la police mais je l’ai persuadée de n’en rien faire. Je lui ai dit que ça risquait de coûter un peu d’argent, mais que j’allais mettre un vrai professionnel sur le coup.

\textsc{Greg} : Je crois qu’il serait quand même souhaitable de le faire.

\textsc{Sally} : Quoi ?

\textsc{Greg} : Appeler la police.

\textsc{Sally} : Vous refusez le job ?

\textsc{Greg} : Je n’ai pas dit ça. Je pense comme vous qu’ils ne feront rien, ou pas grand-chose, mais autant mettre toutes les chances de notre côté.

\textsc{Sally} : Non. Je veux, nous voulons que ce soit vous qui meniez l’enquête.

\textsc{Greg} : Je peux savoir comment vous êtes arrivée jusqu’ici ?

\textsc{Sally} : En voiture, j’essaie d’éviter les transports en commun. Pourquoi ?

\textsc{Greg} : Pour rien. Non, ce que je veux dire, c’est pourquoi moi ? Pourquoi êtes-vous venue me trouver moi et pas un autre ?

\textsc{Sally} : Vous avez été chaudement recommandé.

\textsc{Greg} : Je peux savoir par qui ?

\textsc{Sally} : Une personne pour qui vous avez travaillé et qui a été pleinement satisfaite de vos services. Je ne peux pas vous en dire plus, elle tient à garder l’anonymat.

Greg s’est creusé le chou quelques instants pour essayer de voir qui pouvait correspondre à la personne en question. En vain, d’autant que personne, jusqu’à présent, n’avait eu à se plaindre de ses services. Quand un client vient vous trouver pour enquêter sur ceci ou cela, résoudre telle ou telle affaire, on n’est pas censé enquêter sur lui, en tout cas pas officieusement, et pas davantage que ne l’exigent les données du problème, sauf bien sûr si on flaire une embrouille et a le sentiment de se faire manipuler. Il pouvait donc s’agir d’à peu près n’importe qui, en rapport direct ou indirect avec les protagonistes de l’affaire et le Sugar \& Spice.

\textsc{Greg} : C’est bon, je n’insiste pas. De toute façon, ça n’a pas grande importance.

\textsc{Sally} : Non, en effet. Alors vous acceptez ou pas ?

\textsc{Greg} : Vous connaissez mes tarifs ?

\textsc{Sally} : Oui. C’est cher mais votre réputation n’est plus à faire.

\textsc{Greg} : Nous sommes dans la moyenne de la profession. J’ajoute que nous avons aussi des forfaits très intéressants, surtout pour les enquêtes qui risquent de prendre un certain temps.

\textsc{Sally} : Nous avons constitué une petite cagnotte qui devrait nous permettre de faire face aux dépenses, même les plus imprévues. Je ne sais pas si vous êtes au courant, mais notre petite communauté LGBT, QIA+ si affinités, traverse une mauvaise passe, si j’ose m’exprimer ainsi. Durant les six derniers mois, plusieurs d’entre nous ont disparu sans laisser de trace, se sont comme qui dirait volatilisés dans la nature. Il y a peut-être dans le secteur un dingue qui s’est donné pour mission d’exterminer tous les queers, les transgenres, les non-binaires et les pédés. La police est au courant mais se fiche comme d’une guigne de ce qui peut bien nous arriver. Si un tel individu existe, il se pourrait que Tiago soit tombé entre ses mains. Alors, c’est oui ou c’est non ?

\textsc{Greg} : C’est oui.

Par le plus grand des hasards, et sans doute aussi parce qu’il avait fait venir le major Sandor Balint et son fidèle Zoltan, un malinois de cinq ans d’âge, excessivement racé et incontestablement le plus grand renifleur de sperme et sécrétions intimes en tout genre de sa génération, fleuron de la brigade cynophile, récemment élu Truffe de Platine et meilleur chien policier de tous les temps (que son maître ne manquait jamais de récompenser avec une poignée de croquettes Waterflox au sanglier, baies sauvages et légumes de saison, je suppose que le nom vous rappelle quelque chose, une affaire de sinistre mémoire qui prouve une fois encore que les tréfonds de l’âme humaine sont loin d’avoir été atteints), Zaahid avait identifié, sur un tronc d’arbre avoisinant la scène de crime, des traces de liquide séminal pour le moins suspectes. D’après lui, ces traces attestaient qu’on avait affaire à un grand malade qui s’était astiqué le poireau en regardant le corps brûler, péché mignon de certains pyromanes qui éprouvent une vive excitation sexuelle à la vue des flammes. À ce sujet, j’ai une pensée émue pour Ottis Toole\nf{Ottis Toole (1947--1996), serial killer américain originaire de Jacksonville (Floride), condamné pour plusieurs meurtres commis notamment en compagnie de Henry Lee Lucas. Il est également suspecté du meurtre d’Adam Walsh (1981), dont la disparition conduisit à la création du \textit{National Center for Missing and Exploited Children}. \textit{Source :} \textup{fr.wikipedia.org/wiki/Ottis\_Toole}}, alias le cannibale de Jacksonville, qui s’est livré à de nombreuses exactions en compagnie de son petit copain Henry Lucas\nf{Henry Lee Lucas (1936--2001), serial killer américain condamné pour onze meurtres. Il avait initialement avoué plusieurs centaines de crimes à travers les États-Unis, aveux largement rétractés par la suite et controversés. \textit{Source :} \textup{fr.wikipedia.org/wiki/Henry\_Lee\_Lucas}}, lui-même sérieusement dérangé du bocal. Il faut dire que le pauvre Ottis avait commencé à morfler dès sa plus tendre et juteuse enfance, son âge le plus fondant, étant issu d’une famille de bouseux analphabètes qui pratiquaient inceste, torture et consanguinité depuis des générations. Dès l’âge de cinq ou six ans, parfois dès le berceau, les enfants étaient quotidiennement abusés et soumis à de cruels châtiments (tant corporels que psychologiques) pour les endurcir, les préparer à la dure réalité de l’existence misérable qui les attendait, mais aussi s’assurer qu’ils seraient aux premières loges pour perpétuer la tradition, les valeurs familiales. Sa grand-mère, au lieu de faire des cookies, lui chanter de comptines pour l’endormir et lui remonter le moral quand il est triste, comme le font toutes les gentilles grands-mères, est une vieille folle qui se lave tous les trente-six du mois, picole comme un trou et pratique le satanisme à haute dose. La nuit, ils font la tournée des cimetières pour déterrer des cadavres. Quand il n’est pas sage, rechigne à démembrer un corps ou avaler une potion qu’elle a fabriquée spécialement pour lui, elle l’arrose de seaux de pisse et le barbouille d’excréments. Toutes ces activités qui, en dépit de leur caractère inhabituel, pourraient sembler ludiques à première vue, sont en réalité très traumatisantes pour un gamin de huit ans, heureusement pour lui déjà accro à l’alcool et aux drogues qui lui permettent d’envisager l’existence avec davantage, sinon de sérénité, au moins de détachement. À quatorze ans, sur les conseils avisés de sa grande sœur qui picole, se drogue et se prostitue (le tiercé gagnant pour finir en pièces détachées dans une benne à ordures), il fréquente les bars gay et loue ses services pour quelques cents. Rapidement, il se retrouve avec le fion tellement large qu’un troupeau d’éléphants aurait pu le traverser au pas de charge sans toucher les bords, et comme il a un petit pois à la place de son cerveau imbibé d’alcool et des images horrifiques accumulées pendant son enfance, il se lance dans une carrière de serial killer cannibale qui lui vaudra une certaine notoriété auprès de tous les cinglés amateurs de faits divers sordides, autant dire une bonne partie de la population mondiale. À ce sujet, je rappelle quand même que l’espèce humaine se trimballe un sérieux problème d’addiction à la violence, l’horreur et la tragédie, données stylistiques qui semblent indissociables de son activité et consubstantielles à sa nature profonde, laquelle serait donc, par voie de conséquence, fondamentalement violente, horrible et tragique, ce qui, reconnaissons-le, ne présage rien de bon pour l’avenir.

Un beau jour, parce que la vie est un merveilleux conte de fées, un monde où tout est possible, dans lequel les rêves les plus improbables se réalisent au moment où on s’y attend le moins, Ottis rencontre un beau jeune homme borgne qui répond au doux nom d’Henry Lucas. Le petit Henry a été élevé dans une cabane en rondins au milieu des bois, avec les bêtes sauvages, les tapis de mousse sous les pieds et les champignons, mais sans eau ni électricité. Viola, sa mère, alcoolique au dernier degré, l’oblige à garder les cheveux longs et aller à l’école habillé en fille, ce qui, de l’avis de tous les psychiatres consultés, n’est pas l’idéal sur le plan de la construction personnelle, la mise en place de structures psychologiques stables et socialement compatibles. Elle le tabasse quotidiennement et se prostitue sous ses yeux avec une clientèle dont je vous laisse imaginer le niveau d’éducation. En fait de poètes et hommes du monde, il s’agit surtout de brutes épaisses dont la pudeur et la délicatesse ne font pas partie des qualités premières. Il n’y a pas que du bon dans le mauvais, et inversement. Un jour, un ami de sa mère qui s’est pris d’affection pour lui, apprend au petit Henry, alors âgé de dix ans, à égorger les animaux avant de leur faire l’amour. Non seulement les dernières convulsions sont source de plaisir, mais ça permet d’éviter de prendre un mauvais coup au cas où l’animal récalcitrant se débattrait plus que de raison. C’est ce genre de petits conseils qui forment la jeunesse et l’aident à mettre un pied serein dans l’âge adulte. À vingt-quatre ans, Henry, qui a déjà passé une bonne partie de sa vie en prison pour vol, vol en récidive et tentatives d’évasion, se dispute avec sa mère, se retrouve sans trop savoir comment avec un couteau à la main, lequel couteau se retrouve sans trop savoir comment dans le ventre de sa mère, laquelle mère se retrouve sans trop savoir comment refroidie pour de bon. Il fallait bien que ça arrive un jour, elle ne l’avait pas volé, diront certains auxquels je laisse l’entière responsabilité de leurs propos. Henry plaide la légitime défense, écope de trente ans de prison, est déclaré trop con et secoué du bulbe pour effectuer sa peine, atterrit à l’hôpital psychiatrique du coin, tente de mettre fin à ses jours à plusieurs reprises, suit un traitement de choc à base de courant électrique et antidépresseurs surpuissants, est libéré dix ans plus tard pour bonne conduite (il est tellement shooté qu’il tient à peine sur ses jambes et ne risque pas de faire chier grand monde), peine quelque peu à se réadapter à la vie normale, retourne en taule un an après pour avoir tenté d’enlever deux écolières dans l’intention manifeste de leur faire subir des sévices sexuels inappropriés, et en ressort quatre ans plus tard à condition de se tenir à carreau, arrêter de faire la sortie des écoles et cesser immédiatement de tenter de s’approprier sous la menace des choses qui ne lui appartiennent pas. C’est alors qu’il fait la connaissance d’Ottis Toole, charmant garçon avec lequel, même s’il conserve une certaine attirance pour les adolescentes, il ne s’interdit pas d’avoir des relations dépassant de strict cadre de l’amitié consensuelle. Toole est con comme un balai, c’est vrai, mais aussi d’une touchante naïveté qui l’émeut au plus profond de lui-même. Il est comme le petit frère (ils ont dix ans d’écart) qu’il n’a jamais eu et pourra enculer all night long sans avoir de comptes à rendre à la justice divine et encore moins celle des hommes, même s’il ne faisait pas toujours bon être gay dans l’Amérique profonde des années 80. Lui-même, d’ailleurs, n’était pas vraiment gay, mais plutôt bisexuel, à voile et à vapeur, comme on disait à l’époque en étouffant un ricanement gêné. Quand on a appris à enculer des chèvres, qu’on a des besoins sexuels importants et qu’on n’a rien d’autre sous la main que son petit frère pour les soulager, pourquoi ne pas en profiter, même si le petit frère en question affiche des pratiques douteuses avec lesquelles on n’est pas toujours en accord, par exemple manger de la chair humaine cuite au barbecue. Mais bon, on sait que la vie de couple n’est pas toujours parfaite ni idéale. Il y a des hauts et des bas, des points d’achoppement sur lesquels on doit éviter de se focaliser si on veut avoir une chance de s’en sortir. Mais cette belle amitié prend fin le jour où Lucas enlève Becky Powell, la nièce de Toole dont il est secrètement amoureux. Putain de destinée, qui finit toujours par triompher quels que soient les trésors d’énergie qu’on dilapide à la combattre. Quelle perte de temps ! Ce qui devait arriver arriva, les deux hommes se séparent : Toole rentre à Jacksonville, la queue entre les jambes et le cœur meurtri (la queue aussi, du reste, à force de la fourrer n’importe où), tandis que Lucas et Powell décident d’aller couler le parfait amour au Texas (où ils vont couler, en effet, mais pas le parfait amour).

Fin d’une des plus belles, tragiques et subversives histoires d’amour du vingtième siècle (bon évidemment pas très glamour du fait qu’elle ne se passe pas chez les riches, dans un château avec des toiles de maîtres, des meubles de prestige et les portraits des ancêtres accrochés aux murs, mais met~-- mémé était une sorcière~-- en scène des personnages issus de milieux modestes, sinon franchement défavorisés, d’immondes cloaques infestés de vermine, autrement dit des gens qui n’ont jamais mastiqué un toast au caviar, chié une langouste flambée au rhum ni bu du Cristal au goulot, des cons arriérés, avec des QI de mouche de merde, formatés dans le seul et unique but de faire le mal, nuire à leurs concitoyens, mais animés malgré tout d’une farouche volonté de s’en sortir, du vibrant désir d’échapper à une condition qu’ils n’ont pas choisie et rejettent de toute la force de leurs âmes corrompues, flétries par vice), dont les deux protagonistes écoperont de peines de prison à vie et finiront leurs jours derrière les barreaux.

D’après Zaahid, qui connaissait évidemment le cas de Toole, on pouvait se trouver en présence d’un gros tas de merde du même genre, un attardé sexuellement perturbé qui prenait son pied en foutant le feu à des gens et regardant les flammes danser sous la lune. Ou alors, il l’avait fait griller comme un méchoui dans l’idée de se tailler quelques côtelettes pour son repas du soir, hypothèse cependant peu probable compte tenu de l’état de calcination avancée du cadavre, même pour un amateur de viande bien cuite. L’analyse détaillée de la scène de crime, notamment les débris organiques et végétaux situés sous la dépouille, avait révélé la présence de trois prothèses dentaires, une boucle d’oreille sertie de diamant et deux piercings de tétons en forme de serpent qui se mord la queue. D’autre part, il était établi avec certitude que la victime était un individu de sexe mâle âgé d’une trentaine d’années, descendant très probable des Nambiquara\nf{Les Nambiquara sont un peuple autochtone du Brésil établi sur les hauts plateaux du Mato Grosso, étudiés par Claude Lévi-Strauss lors de son expédition de 1938. Leur population a été drastiquement réduite par la colonisation et les épidémies ; ils ne sont plus aujourd’hui que quelques milliers. \textit{Source :} \textup{fr.wikipedia.org/wiki/Nambikwara}}, une tribu amérindienne des hauts plateaux du Mato Grosso. Comme partout en Amérique du Sud, la colonisation a signé leur perte, et ils ne sont plus aujourd’hui que quelques milliers disséminés dans des villages le long des fleuves, menacés en permanence par la cupidité des chercheurs d’or et la rapacité des promoteurs. Naturellement, les missionnaires portugais ne faisaient pas qu’apporter la bonne parole à des sauvages qui se baladaient à poil dans la forêt. Force était de reconnaître que pour des sauvages, à mi-chemin entre la femme et la guenon, les femelles, surtout les plus jeunes, étaient quand même très attirantes, même si elles dégageaient des odeurs pas toujours très catholiques, et les missionnaires, au-delà de la parole, se devaient de vaincre leurs appréhensions afin d’introduire en elles les gènes salvateurs de la civilisation. Quelques siècles plus tard, on ne peut que rendre hommage à leur clairvoyance et leur dévouement : les femmes portent des robes à fleurs et ont troqué leurs colliers de perles, bracelets en cul de tatou et autres plumes dans le nez contre des bijoux made in China à trois euros le kilo qui leur irritent la peau. Quant aux hommes, ils ont cessé de faire joujou avec leurs flèches empoisonnées pour bosser dans les plantations ou à la centrale hydroélectrique du coin. Et pour ce qui est de l’or présent un peu partout sur le territoire, qu’ils ne s’inquiètent de rien, des gens très compétents venus d’ailleurs se chargent d’en tirer le meilleur parti. Car enfin, si on veut que la civilisation continue son irrésistible progression à travers les contrées désolées de la barbarie et l’absence totale de culture occidentale de gens qui n’ont pas le permis de conduire et n’ont même, pour la plupart, jamais vu de voiture, il faut bien que tout le monde y mette un peu du sien.

Un beau soir, Greg m’a appelé pour savoir si j’avais entendu parler de certaines choses concernant la communauté LGBT.

J’ai répondu : Non, pas vraiment.

Le fait est que les retours concernant les difficultés de la communauté LGBT n’étaient pas monnaie courante dans la police.

\textsc{Lui} : Apparemment, il y en a plein qui disparaissent.

\textsc{Moi} : Ah bon ?

\textsc{Lui} : Oui. Crois-le ou non, mais je travaille en ce moment pour le sosie de Danny DeVito.

\textsc{Moi} : Sans blague ?

\textsc{Lui} : Oui. Sauf que c’est une femme avec une paire de nichons digne d’un film de Russ Meyer !

Russ Meyer\nf{Russ Meyer (1922--2004), réalisateur américain, pionnier du film érotique indépendant dit «~sexploitation~». Auteur de \textit{Faster, Pussycat! Kill! Kill!} (1965) et d'une œuvre caractérisée par l'humour burlesque et la mise en scène de femmes à la poitrine opulente. \textit{Source :} \textup{fr.wikipedia.org/wiki/Russ\_Meyer}}, le roi de la sexploitation (avec Tinto Brass\nf{Tinto Brass (né en 1933), réalisateur et monteur italien, connu pour des films érotiques provocateurs dont \textit{Caligula} (1979) et \textit{Salon Kitty} (1976). \textit{Source :} \textup{fr.wikipedia.org/wiki/Tinto\_Brass}} et Jess Franco\nf{Jess Franco (1930--2013), réalisateur espagnol prolifique, auteur de plus de deux cents films d'exploitation mêlant horreur et érotisme. Figure culte du cinéma de genre européen des années 1960--1980. \textit{Source :} \textup{fr.wikipedia.org/wiki/Jess\_Franco}}), ennemi juré de la morale judéo-chrétienne et du code Hays\nf{Le code Hays, ou Code de production, est un ensemble de règles de censure morale imposées à Hollywood de 1934 à 1968. Il interdisait notamment la nudité, les relations sexuelles explicites et toute représentation jugée contraire aux bonnes mœurs. \textit{Source :} \textup{fr.wikipedia.org/wiki/Code\_Hays}}, à qui l’on doit une foultitude de chefs-d’œuvre presque tous aussi inoubliables les uns que les autres, parmi lesquels Le Désir dans les tripes, Mondo Topless, La Vallée des Plaisirs et la série des Vixens. Spéciale dédicace à Kitten Natividad, alias Lola Langusta.

\textsc{Moi} : On n’arrête pas le progrès. Si demain des gens ont envie de se faire greffer un troisième sein, une langue de serpent ou un deuxième trou de balle, il y aura toujours quelqu’un pour leur donner satisfaction. Ça leur coûtera la peau des fesses, mais ils pourront toujours, moyennant une petite rallonge, se la faire remplacer par du cuir d’hippopotame, nettement plus résistant quand on passe sa vie assis devant un ordinateur ou une console de jeu.

\textsc{Lui} : T’as pas tort.

\textsc{Moi} : Rarement, en effet.

\textsc{Lui} : T’exagères un peu, comme d’habitude, mais t’as pas tort. Donc, si je comprends, t’as rien d’intéressant à me dire sur les LGBT.

\textsc{Moi} : Non, pas vraiment. Je sais qu’ils se crêpent régulièrement le chignon, se tirent les cheveux et se donnent des coups de pieds dans les tibias, mais je t’avouerai franchement que ça ne fait les gros titres de la gazette de la police. Après tout, jusqu’à preuve du contraire, ce sont des adultes responsables et on n’a pas à se mêler de leurs affaires. S’ils veulent disparaître, s’évaporer dans la nature comme des pets, c’est leur problème. Par contre, j’ai un cadavre sur le dos qui pèse une tonne et dont j’aimerais assez me débarrasser au plus vite. On l’a retrouvé dans la forêt, carbonisé jusqu’à l’os. Il ressemble à quoi, ton mec qui a disparu ?

\textsc{Lui} : Brun, la trentaine, d’origine brésilienne, plutôt mignon.

Une petite musique s’est fait entendre dans le fond de mon crâne, une petite musique genre flûte de pan dans la pampa, comme dans \textit{Mission} de Joffé\nf{Roland Joffé (né en 1945), réalisateur britannique, auteur de \textit{Mission} (1986), film sur les missions jésuites guaranis en Amérique du Sud au \textsc{xviii}\textsuperscript{e}~siècle, Palme d'or à Cannes. La bande originale, signée Ennio Morricone, inclut le célèbre \textit{Gabriel's Oboe}. \textit{Source :} \textup{fr.wikipedia.org/wiki/Mission\_(film,\_1986)}} ou \textit{Apocalypto} de Mel Gibson\nf{Mel Gibson (né en 1956), acteur et réalisateur américano-australien. Récompensé de l'Oscar du meilleur film pour \textit{Braveheart} (1995), il réalise également \textit{La Passion du Christ} (2004) et \textit{Apocalypto} (2006). Ses déclarations antisémites (2006) et ses soutiens politiques controversés ont alimenté les polémiques. \textit{Source :} \textup{fr.wikipedia.org/wiki/Mel\_Gibson}}, ce crétin réactionnaire qui vote Trump au même titre que Kanye West\nf{Kanye West (né en 1977), rappeur, producteur et entrepreneur américain, auteur de \textit{The College Dropout} (2004) et \textit{My Beautiful Dark Twisted Fantasy} (2010). Connu pour ses prises de position erratiques, ses déclarations antisémites (2022) et son soutien affiché à Donald Trump. \textit{Source :} \textup{fr.wikipedia.org/wiki/Kanye\_West}}, Jon Voight\nf{Jon Voight (né en 1938), acteur américain, Oscar du meilleur acteur pour \textit{Le Retour} (1978) et père d'Angelina Jolie. Républicain militant, il soutient Donald Trump depuis 2016. \textit{Source :} \textup{fr.wikipedia.org/wiki/Jon\_Voight}}, Dana White et Buzz Aldrin\nf{Buzz Aldrin (né en 1930), astronaute et pilote militaire américain. Le 21~juillet 1969, il est le deuxième homme à marcher sur la Lune lors de la mission \textit{Apollo~11}, après Neil Armstrong. \textit{Source :} \textup{fr.wikipedia.org/wiki/Buzz\_Aldrin}}, calviniste de mes deux qui aurait mieux fait de rester sur la Lune le jour où il y a posé le pied, le 21 juillet 1969. Make America Beauf Again, les rednecks ont encore de beaux jours devant eux.

\textsc{Moi} : Il n’avait pas des piercings aux tétons, par hasard ?

\textsc{Lui} : Si, des anneaux.

\textsc{Moi} : En forme de serpent qui se mord la queue ?

\textsc{Lui} : Tout juste, Auguste.

\textsc{Moi} : On les a retrouvés au milieu des cendres. Je suppose qu’il y a des tas gens qui ont des piercings de ce genre, mais il y a quand même de fortes chances pour ton client et le mien soient une seule et même personne. D’autant que si on en croit Zaahid, qui se trompe rarement sur le sujet, le mien descend en droite ligne des Nambiquara, une tribu amérindienne du Mato Grosso dont seuls quelques rares spécimens s’accrochent encore aux branches de la forêt amazonienne. Je pense que ça commence à faire beaucoup, tu ne crois pas.

\textsc{Lui} : C’est clair, Albert. Il s’appelle Tiago Alvarez et bosse comme ambulancier SMUR au CHU Désiré Trudeau.

\textsc{Moi} : C’est énorme !

\textsc{Lui} : Oui, c’est super excitant !

\textsc{Moi} : T’as son adresse ?

\textsc{Lui} : Bien sûr, que je l’ai ! Tu me prends pour qui ? Un amateur ?

\textsc{Moi} : Tu peux me la donner, s’il te plaît. On va en avoir besoin pour récupérer de l’ADN et le comparer à celui de notre grand brûlé.

\textsc{Lui} : 73 rue Valentin Abou, dans le 14\ieme{}. Je te cache pas que je suis déjà allé y faire un tour, histoire de m’assurer qu’il n’y était pas. En même temps, s’il était mort depuis trois semaines, l’odeur de charogne aurait alerté le voisinage.

\textsc{Moi}, en train de siroter tendrement un verre de Chevalières (une dizaine d’hectares de chardonnay entre les Rougeots et Meix-Chavaux, à Meursault) dans ma cuisine en compagnie de Zarina Brizzi, une fille dont j’ai déjà eu l’occasion de vous parler et vous laisser entendre à quel point sa présence était loin de me laisser dans l’état d’indifférence totale que générait ordinairement chez moi la présence de ses congénères : Ça dépend du voisinage. On a bien retrouvé une petite vieille morte depuis deux ans dans son appartement de Saint-Brieuc. Elle ne devait pas recevoir souvent de la visite.

\textsc{Lui} : Pauvre vieille.

Petite mise au point en passant : c’est au Narcisse Rose que j’avais, quelque temps plus tôt, fait la connaissance de Zarina Brizzi, trente-deux ans et toutes ses dents, archétype du missile supersonique made in Italy (comme le balistique Alfa au propergol solide développé par Leonardo dans les années 70, soi-disant resté à quai après l’adoption du traité rédigé par l’ONU dans le but de contrôler et limiter au maximum la prolifération des armes nucléaires à travers le monde), aussi bouleversante qu’une assiette de pappardelle au sanglier ou de lapin au chou et raisins secs, longue chevelure brune ramassée en un savant entrelacs (ne me demandez pas pourquoi il y a un S à la fin, ça fait partie du charme de la langue française, comme contrebasse, architecture, daim, ornithorynque, Seldjoukides, groupe sanguin, orang-outan et Vincent van Gogh pour n’en citer que quelques-uns) de mèches tressées avec une maestria confondante, regard profond comme les eaux de l’Arno quand le soleil se couche sur Florence et que vous êtes en train de vous goinfrer de crostini neri aux foie de volaille, câpres et anchois dans une antique trattoria de la via della Scala, etc., etc. … et tout cela, oui tout cela est très émotionnant, c’est certain, et je pourrais continuer pendant des centaines et des centaines de milliers de pages à vous décrire au millimètre près le moindre centimètre-carré de son incomparable épiderme à la coloration parfaite, mais l’essentiel de ce que vous avez à savoir tient en quelques mots : oui, elle et moi formions maintenant ce qu’il est convenu d’appeler un couple, et j’avais, à mon grand regret, dû renoncer au vœu de chasteté qui m’avait jusqu’alors tenu éloigné de la promiscuité charnelle inhérente à ce genre d’activité. Autrement dit il nous arrivait, Zarina et moi, de niquer comme des bêtes jusque tard dans la nuit. Et vous devez savoir aussi, par la même occasion, que Zarina avait une sœur, Tosca, copie conforme d’elle-même à quelques détails près, et que cette même sœur formait avec mon ami Zaahid Shirani ce qu’il est également convenu d’appeler un couple, autrement qu’il arrivait aussi à Zaahid et Tosca de niquer jusque tard dans la nuit, et je dirais même, les concernant, que cette activité avait pris une place prépondérante dans leur existence. J’ajoute que toutes deux avaient rendu les clés de leur suite du Jade Mountain Hôtel pour venir s’installer chez l’un et l’autre de leurs conjoints respectifs, ce qui ne m’arrangeait qu’à moitié dans la mesure où j’avais pris des habitudes de vieux garçon auxquelles je n’entendais pas renoncer aussi facilement.

\textsc{Moi} : Tu sais qu’on n’a pas le droit d’aller fouiller comme ça chez les gens ?

\textsc{Lui} : Ah bon ?

\textsc{Moi} : C’est de la violation de domicile. Pour faire ce genre de choses, il faut avoir une autorisation. T’as touché à rien, j’espère ?

\textsc{Lui} : Non, je voulais juste m’assurer qu’il n’était pas en train de pourrir dans un coin.

\textsc{Moi} : Quoi d’autre ?

\textsc{Lui} : Rien de spécial. Ah si, Alvarez fréquente le Sugar \& Spice, un cabaret de la rue Théo Cazenave. Je te conseille d’aller y faire un tour, ça vaut le coup d’œil !

\textsc{Moi} : Je n’y manquerai pas, dès que j’aurai la preuve qu’on parle bien du même type.

\textsc{Lui} : Si c’est le cas, je vais perdre mon job.

\textsc{Moi} : Pourquoi ? T’es pas obligé de le dire à ta cliente. Tu continues à palper tes honoraires comme si de rien n’était, et un beau jour, quand tu penses que la comédie a assez duré, tu lui balances l’info. De mon côté, je me suis arrangé pour que l’affaire ne fuite pas dans les journaux. Tu peux dormir sur tes deux oreilles, jamais ta cliente ne fera le lien entre les deux affaires. Elle s’appelle comment, d’ailleurs ?

\textsc{Lui} : Et le secret professionnel, t’en fais quoi ?

\textsc{Moi} : Je m’assois dessus. Et l’amitié, t’en fais quoi ?

\textsc{Lui} : Je m’assois dessus. Non, sans blague, c’est pas que je veux pas te le dire, mais je ne vois pas très bien à quoi ça te servirait dans l’état actuel des choses. Il sera toujours temps de voir quand tu auras les résultats de l’analyse ADN. J’ajoute que ma cliente, du haut de son mètre cinquante-sept, n’a pas vraiment le physique de l’emploi.

\textsc{Moi} : C’est vrai que c’est pas souvent que des nains assassinent des gens. Je sais même pas s’il existe un seul cas de tueur nain dans les annales du crime. Le nain, bizarrement, tue assez peu, alors qu’il aurait toutes les raisons d’en vouloir à l’existence et la société qui ne cesse de lui renvoyer son infirmité au visage. Cela dit, il ne faut pas se fier aux apparences. Il n’est pas rare, par exemple, que l’assassin vienne signaler lui-même la disparition de sa victime. Si ça se trouve, c’est ton sosie de Danny DeVito qui a refroidi son petit copain Alvarez. Enfin, refroidi, c’est une façon de parler. Je dirais plutôt un peu trop réchauffé. Certains criminels se croient intouchables, invisibles, ça les fait bander de venir se fourrer dans la gueule du loup pour jouer avec ses dents. Mais ils finissent toujours par se faire croquer.

\textsc{Lui} : Je crois que je t’en ai assez dit comme ça. Tu sais que ma cliente mesure moins d’un mètre soixante, qu’elle ressemble à Danny DeVito avec des nichons et fréquente le Sugar \& Spice. Je pense que même un flic aussi peu doué que toi ne devrait pas avoir trop de mal à l’identifier. Mais je suis certain qu’elle n’a rien à voir là-dedans. Je pense juste qu’elle est raide-dingue d’Alvarez et aimerait bien savoir ce qui lui est arrivé. Je peux même te dire qu’elle et ses potes du Sugar \& Spice ont fait une cagnotte pour retrouver leur mascotte. Par contre, s’il est arrivé quelque chose à Alvarez et si tu retrouves le tueur, on pourrait le livrer à la communauté LGBT qui se fera une joie d’organiser un simulacre de procès et le condamner à la peine capitale. J’ai évoqué le sujet avec ma cliente, très à cheval sur les principes, et elle ne m’a pas caché que le cas échéant, elle serait ravie de procéder elle-même à l’exécution.

\textsc{Moi} : On vit quand même dans un drôle de monde.

\textsc{Lui} : C’est bien vrai, ma bonne dame.

\textsc{Moi} : Je fais ma petite enquête et je te tiens au courant. De ton côté, préviens-moi si t’as du nouveau.

\textsc{Lui} : Je n’y manquerai pas.

J’ai raccroché et me suis précipité comme un sauvage sur la bouteille de Meursault qui attendait sagement dans le réfrigérateur, toute perlante de gouttelettes de buée rafraîchissante (même si, techniquement parlant, ce n’est la buée qui rafraîchit, on est bien d’accord, sachant que c’est seulement la vision de celle-ci qui présage de la fraîcheur du liquide qui se trouve dans la bouteille, en l’occurrence un Meursault Chevalières signé Jean-Charles Pichard, considéré par beaucoup et non des moindres comme le pape de l’appellation, le très saint père de la Côte de Beaune devant lequel les apôtres du chardonnay venaient se prosterner et lécher la terre sacrée du vignoble en poussant des petits couinements de plaisir). Nos verres étaient vides mais nos cœurs pleins d’amour et nos sous-vêtements débordants de victuailles (du boudin, de la saucisse, des noix, de la moule, du poireau, de la courgette, de la frisée et j’en passe) comme des caddies de supermarché en période de fêtes. Enfin, surtout Zarina, parce que moi, comme je vous l’ai dit, j’étais plus proche du moine cistercien que de la star du X dopée à la testostérone. Mon truc, c’était plutôt la contemplation, la méditation, au fond d’une cellule ou assis sur un rocher surplombant le vide, la bite à l’air de préférence, dressée comme une antenne pour capter la tiédeur buccale du soir et les fréquences spirituelles de l’univers. J’allais par monts et par vaux, simplement vêtu, la bourse vide et la queue entre les jambes, prêcher la bonne parole et secourir la veuve et l’orpheline. J’avais fait don de mon existence au service de l’autre, renonçant à tous les plaisirs, honneurs et privilèges dus à mon rang. Néanmoins, en prenant sur moi et avec l’aide de notre seigneur tout-puissant, grâce lui soit rendue, j’arrivais tant bien que mal à un niveau de performance qui n’était pas sans rappeler les riches heures de mon étincelante jeunesse, les plus brillants faits d’armes du séducteur impénitent que j’étais alors, ce prédateur à la démarche souple et féline et au sourire carnassier auquel aucune adolescente boutonneuse et dentairement appareillée ne pouvait résister. Cela dit, j’avais trouvé auprès de Zarina, devant laquelle je me sentais tel un enfant qui s’émerveille chaque jour un peu plus de la beauté du monde, une oreille bienveillante, attentive à mes tourments et sensibles à mes exigences parfois retorses. Bon, je ne veux pas entrer dans les détails, mais le fait est que mon approche des plaisirs physiques n’était pas toujours très orthodoxe. Le mot «~chair~», pour moi, avait une consonance qui résonnait parfois bien au-delà de ce que des oreilles dites normales sont en capacité de percevoir, et encore moins, si elles y parviennent, en capacité d’accepter. Oui, je crois pouvoir dire que j’avais trouvé en Zarina une âme sœur des plus compatissantes, et les retours que j’avais de Zaahid sur sa liaison avec Tosca me confortaient dans l’idée que nous avions l’un et l’autre mis la main sur des perles d’une grande rareté, des joyaux d’autant plus précieux qu’ils appartenaient à une vieille lignée florentine (comment ne pas imaginer les palais somptueux, les jardins merveilleux, les fontaines jaillissantes et les villas de rêve sur la Riviera) dont l’immense fortune faisait l’admiration et l’envie de bien des pèlerins désargentés. Je m’empresse (je sais à quel point les gens sont méchants) de préciser que tel n’était pas exactement notre cas à Zaahid et moi-même, puisque le fric que je détournais copieusement dès que j’en avais l’occasion, les perquisitions-acquisitions-disparitions inexpliquées de liasses de billets de banque, ainsi que les saisies de drogue que je me permettais de réinjecter discrètement sur le marché afin de satisfaire une demande en constante expansion, me garantissaient un train de vie plus que raisonnable, train dans lequel je me gardais bien de voyager en dehors de ma plus stricte intimité, heureusement constituée de personnages aussi peu recommandables que moi. C’est ainsi que je persistais à rouler ostensiblement dans ma vieille Kangoo déglinguée et porter des fringues de confection d’un goût douteux, lesquelles me valaient régulièrement les sarcasmes de la profession. Zaahid, de son côté, pouvait se prévaloir de son salaire de légiste, un salaire plutôt modeste qu’il optimisait officieusement avec un certain nombre de prestations médicales réservées à ce qu’il appelait lui-même une «~clientèle particulière~», c’est-à-dire des gens (des grands timides pour la plupart, des mélancoliques ou des agoraphobes) qui préféraient rester en dehors de la filière hospitalière classique. Cette clientèle particulière, qui débarquait souvent avec des trous de balles situés à des endroits inhabituels, savait se montrer reconnaissante.



\noindent Croyez-le ou non, mais Aymeric Jégou et Tacito Cerqueira étaient deux des pires enfoirés qui aient jamais posé le pied sur notre bonne vieille terre, laquelle en avait pourtant vu des vertes et des pas mûres depuis les quelques milliards d’années qu’elle s’échinait à tourner autour du soleil. Ils travaillaient tous deux pour la même entreprise de déménagement, entreprise qu’ils avaient fondées à deux et dont ils étaient les deux principaux actionnaires et pour ainsi dire les deux seuls employés, hormis une secrétaire répondant au nom de Jessica Millet, sans qu’il soit établi de près ou de loin que son patronyme ait un quelconque rapport avec l’auteur de l’angélus, des glaneuses, de la sieste, du bouquet de marguerites, de l’homme à la houe et du retour du troupeau, autant de valeurs simples dont on appréhende aujourd’hui pleinement l’évidente nécessité.

Jégou et Cerqueira étaient donc, chacun dans leur genre (assez diamétralement opposé, on va le voir), des individus auxquels il n’était pas spécialement conseillé de s’amuser à chercher des poux dans la tête (de toute façon il aurait été difficile d’en trouver vu qu’ils avaient tous les deux la boule à zéro). Sans être à proprement parler une étoile brillant au firmament de l’intelligence, un phare dans la nuit, Jégou était loin d’être bête et pouvait même se montrer retors, tandis que Cerqueira était pourvu d’une capacité de réflexion aussi restreinte que les droits de la femme (aujourd’hui à peine supérieurs à ceux du hérisson) en Afghanistan depuis la prise de Kaboul par les talibans, ou encore la quantité de tissu sur le corps d’une danseuse du Crazy Horse.

Quand nos deux compères n’étaient pas en train de soulever des caisses et trimballer des meubles dans les escaliers, ils officiaient gracieusement en tant que membres du service d’ordre d’un certain parti politique d’extrême-droite dont j’aimerais autant, pour des raisons de dignité personnelle, éviter de dire le nom. Sachez seulement qu’il y est question de la partie haute du visage, située entre les yeux et le haut du crâne, et de quelque chose qui renvoie à l’idée de naissance en général, étymologiquement parlant, mais plus précisément de naissance sur un territoire donné, régi par des règles similaires, des lois auxquelles sont subordonnés tous les habitants dudit territoire, lesquels sont censés parler la même langue, partager des valeurs similaires fondées sur l’expérience d’une histoire commune plus ou moins ancienne, et éprouver une certaine fierté, sinon une réelle satisfaction, à vivre les uns avec les autres. Mais si vous voulez réellement mon avis, eh bien je pense que tout ça c’est surtout de la merde en barre, et que ce genre de structure tient essentiellement sur des faux-semblants et des malentendus.

Sur le plan physique, Jégou et Cerqueira étaient aux antipodes l’un de l’autre. Autant le premier, de nature (cela ne se voyait plus trop maintenant du fait qu’il avait le crâne rasé, au même titre que son comparse), était blond aux yeux bleus, correspondant en tout point au type aryen tel que défini par Ernst Brestrich dans son «~Traité sur les races~», ouvrage controversé aujourd’hui introuvable en librairie (mais vous pourrez le trouver sur Onionland, si vous tenez absolument à vous plonger dans ce torchon), autant le second était brun aux yeux noirs, doté d’une pilosité portugaisoïde assez aux antipodes des canons de la beauté nordique. Cela dit, un physique de ce type, avec monosourcil des plus broussailleux qui plus est, n’avait pas empêché un Rudolf Hess, pourtant notoirement paranoïaque, d’accéder aux plus hautes fonctions du Reich. Pas pour longtemps, il est vrai : en 41, ce triste sire décide de s’envoler vers l’Angleterre pour soi-disant négocier un traité de paix avec Churchill, le roi, la reine et compagnie, personnalités qu’il ne rencontrera bien entendu jamais et traité qui ne sera tout aussi bien évidemment jamais signé. Par contre, il aura tout le loisir de faire connaissance avec le sens de l’hospitalité anglaise, notamment la qualité d’accueil des établissements psychiatriques dans lesquels il sera invité à résider jusqu’à la fin de la guerre, avant d’être expédié à Nuremberg par le premier charter, jugé, condamné à perpète et incarcéré à la prison de Spandau en compagnie de six autres joyeux drilles de son acabit, j’ai nommé les tristement célèbres Karl Dönitz (désigné par Hitler lui-même alors à l’agonie comme son successeur officiel à la tête du Saint-Empire, à la place d’Himmler qui s’était fait la malle et du gros Göring complètement défoncé à la codéine), Erich Raeder (commandant en chef de la Kriegsmarine, libéré en 55 pour raisons de santé), Konstantin von Neurath (ministre des Affaires étrangères, SS-Obergruppenführer et gouverneur de Bohême-Moravie, libéré en 54 pour raisons de santé), Walther Funk (président de la Reichsbank et blanchisseur en chef de l’or et l’argent piqués aux Juifs, libéré en 57 pour raisons de santé lui aussi), Albert Speer (grand architecte du Reich et concepteur du Reichsparteitagsgelände de Nuremberg, libéré en 66 à l’issue de sa peine), et Baldur von Schirach (chef des Jeunesses hitlériennes, amateur d’art et poète à ses heures, je n’en dirai pas davantage, libéré en 66 à l’issue de sa peine). En 1989, à l’âge de 93 ans, Hess se décide enfin à mettre un point final au roman de gare de sa pitoyable existence. Après la libération de Speer et von Schirach en 66, il reste le seul et unique pensionnaire de la prison de Spandau, laquelle sera rasée après sa mort pour éviter d’en faire un lieu de culte pour les nostalgiques du Troisième Reich. Le fils de Hess et filleul de Hitler, Wolf Rüdiger, au même titre que Gudrun Himmler et Edda Göring, a tout fait pour réhabiliter la mémoire de son crétin de père, allant jusqu’à prétendre qu’il avait été assassiné par la CIA pour des raisons politiques aussi obscures que farfelues. Si les autorités ouest-allemandes avaient voulu s’en débarrasser, ce qui aurait pu se comprendre vu qu’elles payaient ses frais de séjour (soit le prix d’une single room au Lutetia pour une cellule de 6 m², multipliez ça par 40 ans à 365 jours et vous aurez une idée de ce que cet enfoiré a coûté à la communauté~-- plus de 20 millions d’euros, au cas où vous n’auriez pas de calculette sous la main), elles n’auraient pas attendu vingt ans pour le suspendre au bout d’un fil électrique. Quant à la CIA (ou les SAS suivant les versions), je pense qu’elle avait autre chose à foutre qu’éliminer un vieillard sénile dont tout le monde se fichait comme de l’an 40. Et malgré tout ça, figurez-vous que le petit-fils, Wolf Andreas, négationniste convaincu, n’a rien trouvé de mieux à faire que de reprendre le flambeau à la mort de son père. Bref, il se passe des choses bizarres dans le cerveau humain, et je pense qu’on n’a pas fini de se creuser les méninges pour essayer de comprendre ce qui se trame à l’intérieur de cette machine infernale.

Et, pour en revenir à ce que je racontais en début de paragraphe, non seulement Jégou s’inscrivait parfaitement dans la catégorie des blonds aux yeux bleus, un blond presque blanc avec des yeux d’un bleu plus bleu que bleu, mais il était aussi sec et nerveux que Cerqueira était humide et détendu. Par «~humide~» j’entends perpétuellement moite, sinon trempé, car il transpirait à grosses gouttes sous l’épaisse couche de poil qui tapissait la majeure partie de son corps. Il était aussi beaucoup plus grand et corpulent que son acolyte, de sorte que tandis que l’autre passait son temps à s’exciter, toujours prêt à mordre comme le roquet hargneux qu’il était, lui conservait au contraire un calme à toute épreuve. Autrement dit, quand il démontait quelqu’un, le dépouillait os par os de chacun de ses membres, jouait de la guitare avec ses tendons et du xylophone avec ses dents, c’était pour ainsi dire sans haine ni violence, se contentant de l’écrabouiller méticuleusement avec les armes de destruction massive qui lui tenaient lieu de mains. Il ne paniquait jamais, ni même ne montrait, au plus fort des affrontements (à coups de barres de fer, manches de pioche, battes de baseball et parfois même couteaux et armes à feu) qui les opposaient régulièrement aux antifas, poseurs d’affiches trotskistes et autres anarcho-syndicalistes à la petite semaine, le moindre signe d’inquiétude. Des fois, on aurait presque dit qu’il s’emmerdait, n’éprouvait plus aucune joie à tuer, massacrer ses adversaires, aucune espèce d’intérêt à les réduire en purée, en faire de la pâtée pour chien. Accessoirement, il servait aussi de garde du corps à Aymeric Jégou, lequel, même s’il était particulièrement vicieux et ne s’embarrassait d’aucun scrupule dès lors qu’il s’agissait d’arracher des cris de souffrance à son prochain (et on pouvait alors lire sur son visage toute l’étendue de la satisfaction de ces cris lui apportaient), pouvait se montrer parfois physiquement un peu léger face à des montagnes de muscles qui faisaient deux fois sa taille et trois fois son poids, même s’il parvenait la plupart du temps à en venir à bout. Dans le cas contraire, Tacito intervenait pour remettre un peu d’ordre dans la mêlée, briser quelques mâchoires et fracasser quelques membres pour inciter les belligérants à plus de retenue.

Un soir qu’ils étaient en train d’écluser des chopes de Karlsbrau au Bouclier, bar tenu par un ancien skinhead du nom de Ronny Bell et connu pour être un repaire de fachos, Jégou, qui commençait à voir tout en double, avait dit à Cerqueira : Faut que je te dise un truc…

Cerqueira, qui pouvait avaler trois tonneaux de bière sans sourciller : Ouais, quoi ?

\textsc{Jégou} : On a décidé de casser du pédé.

\textsc{Cerqueira} : Qui ça, on ?

\textsc{Jégou} : Milo, Noé et moi.

Pour info, Milo Monteil et Noé Desmarais étaient deux abrutis de la même trempe que Jégou et Cerqueira.

\textsc{Cerqueira} : C’est pas ce qu’on fait déjà ?

\textsc{Jégou} : Si, mais on a décidé de passer à la vitesse supérieure.

\textsc{Cerqueira} : Ah ouais ?

\textsc{Jégou} : Ouais. On a formé un petit groupe d’action, les Disciples de la Colère. Parce que nous on est vachement en colère, on a carrément la haine. T’es pas en colère, toi ?

\textsc{Cerqueira} : Si, si, bien sûr, je suis grave en colère.

\textsc{Jégou} : Non, t’es pas vraiment en colère parce que t’es qu’un gros pédé de suceur de queues !

\textsc{Cerqueira} : Non, je suis pas pédé !

\textsc{Jégou} : Si, t’es pédé, tu le sais, je le sais, mais t’es mon pote et tout le monde n’est pas obligé de le savoir. Vaudrait mieux pas, d’ailleurs, sinon je suis pas sûr qu’on pourrait te sortir le cul des ronces. Mais c’est pas grave, je t’en veux pas, t’es l’exception qui confirme la règle, la preuve que le Mouvement est ouvert à tout et qu’on n’est pas les enfoirés sectaires que tout le monde raconte.

\textsc{Cerqueira} : Et vous allez faire quoi ?

\textsc{Jégou} : Rien, ma grosse. On en a juste plein le cul de voir toutes ces tafioles se bécoter en pleine rue et se balader en se tenant par la main. On est en France, merde, on a des valeurs !

\textsc{Cerqueira} : Oui, mais moi aussi je suis pédé.

\textsc{Jégou} : Toi, c’est pas pareil. T’es pédé, okay, mais c’est la société qui t’a obligé à le devenir.

\textsc{Cerqueira} : C’est vrai que je me dégoûte moi-même.

\textsc{Jégou} : Bien sûr, t’as honte de toi, t’es la victime de ce qu’on t’a obligé à devenir. Si tu fais tout pour arrêter d’être pédé, redevenir un mec normal, tu te sentiras beaucoup mieux et tout rentrera dans l’ordre.

\textsc{Cerqueira} : Tu crois ?

\textsc{Jégou} : J’en suis sûr, mon pote.

\textsc{Cerqueira} : Ouais, t’as sans doute raison. Comment t’as dit que ça s’appelait, votre truc ?

\textsc{Jégou} : Les Disciples de la Colère.

\textsc{Cerqueira} : Ça sonne bien.

\textsc{Jégou} : Grave ! C’est moi qui l’ai trouvé. Tu veux en être ?

\textsc{Cerqueira} : Ça dépend. Vous faites quoi, au juste ?

\textsc{Jégou} : On a décidé de s’attaquer aux trans, de leur faire bouffer leurs prothèses mammaires !

\textsc{Cerqueira} : J’ai rien contre les trans.

\textsc{Jégou} : T’as rien contre les trans ?

\textsc{Cerqueira} : Non. Un jour, mon père m’a dit qu’il ne pouvait plus supporter d’être un homme et qu’il avait décidé de changer de sexe. C’était pas du vice ou quoi, c’était juste qu’il se sentait comme une femme à l’intérieur.

Jégou, en faisant rapidement le signe de croix comme s’il avait vu le diable en personne : Seigneur Jésus !

Cerqueira, fixant sa chope avec son regard vide de ruminant dépressif : Le pire c’est que ma mère avait toujours rêvé d’être un homme. Ils s’aimaient toujours, mais ils voulaient inverser les rôles. Mon père est devenu ma mère et ma mère mon père.

\textsc{Jégou} : Quelle horreur !

\textsc{Cerqueira} : Pas tant que ça, on est resté une famille unie. Au début j’ai été un peu perturbé, j’ai fait quelques tentatives de suicide, et puis j’ai lu Mein Kampf et j’ai vu la lumière au bout du long tunnel noir dans lequel j’étais en train de ramper et étouffer.

\textsc{Jégou} : Et t’as pas envie de te venger ?

\textsc{Cerqueira} : Me venger de quoi ?

\textsc{Jégou} : De ce que t’ont fait subir tes parents. C’est à cause d’eux si t’es devenu pédé. C’est à cause d’eux que tu t’aimes pas et que tu te traites comme de la merde. Tu te rabaisses plus bas que terre.

\textsc{Cerqueira} : Ouais, je sais.

\textsc{Jégou} : Bon écoute, si t’as pas envie de casser du trans, c’est pas grave on s’en chargera Milo et moi. Je comprends que t’aies pas envie d’y toucher, les parents, c'est sacré. Sauf que nous, on est quand même obligés de s’en occuper, parce que si t’as bien lu Mein Kampf, la bible du national socialisme, tu sais comme moi que les trans sont pas des gens comme tout le monde, et que tous les gens qui sont pas comme tout le monde sont des dégénérés. Dieu a fait les hommes et les femmes avec une bite, des nichons et tout ce qui s’ensuit, et vouloir changer de sexe, ça revient à dire qu’il a mal fait son boulot. Et les gens qui osent dire que Dieu a mal fait son boulot sont forcément des dégénérés, des agents des forces du mal. Tiens, vendredi soir, on fait une petite virée avec les Disciples de la Colère. T’es le bienvenu si ça te dit.

\textsc{Cerqueira} : Okay, mais je touche pas aux trans.

\textsc{Jégou} : Je te ferai un mot, tu seras exempté pour raisons familiales.

\textsc{Cerqueira} : Je veux bien tabasser tout ce que vous voulez, les juifs, les cocos, les arabes et les pédés, mais pas les trans.

\textsc{Jégou} : Tu sais ce qu’on leur fait, nous, aux pervers et aux drogués qui salissent nos rues et violent nos gosses ?

\textsc{Cerqueira} : Non ?

Jégou, à voix basse : Tu le répéteras pas ?

\textsc{Cerqueira} : Dis pas de conneries !

\textsc{Jégou} : On les crame.

Une nuit, vers trois heures du matin, on a chopé Cerqueira en train de tapiner rue Armand Brunelle, habillé en femme.

Il y avait de la matière mentale qui pourrissait dans les plis de son cerveau et le poussait à faire des choses que sa propre morale réprouvait à cor et à cri.

Histoire de marquer le coup, j’ai fait quelques photos de notre homme en pleine action, la robe relevée jusqu’au nombril, le trou de balle bien en évidence, le genre de documents que les gens n’aiment pas trop savoir en liberté dans la nature.

Naturellement, on savait que quand il ne racolait pas le micheton dans les pissotières, il jouait les gros durs dans les services d’ordre d’extrême-droite avec une bande de détraqués qu’on avait depuis longtemps dans le collimateur. Jusque-là on laissait faire sans trop d’appréhension, dans le souci du respect des traditions et la liberté d’expression, mais on avait eu vent de certaines choses pas très reluisantes concernant leurs activités subalternes, la façon dont ils occupaient leur temps libre, et on avait, après mûre réflexion, décidé que le moment était venu de hausser le ton. Aussi lui ai-je clairement laissé entendre que ses petits copains ne seraient peut-être pas ravis d’apprendre que le meilleur d’entre eux menait ce qu’il est convenu d’appeler une double vie, nazi le jour, tapette la nuit.

Tout ça fleurait bon la schizo des familles, et quand je lui ai demandé s’il avait l’habitude de se flageller pour expier ses fautes, il a dégrafé sa robe à fleurs et m’a montré les cicatrices qui zébraient son dos. Je me suis dit (mon bon cœur me perdra) que ce type au physique d’ours des cavernes, de plantigrade prématurément arraché à la préhistoire, n’était peut-être pas totalement irrécupérable. Traduit dans le langage commun de la police nationale, cela signifiait qu’il n’était pas inévitable que ses petites combines sexuelles s’ébruitent à tout va, au risque de semer un vent de panique générale de nature à lui porter fortement préjudice. Je pouvais rester discret, aussi muet qu’une limace aphone sur une feuille de salade verte en plein soleil, si en échange il me concédait quelques informations utiles. Par exemple, on savait que son collègue de travail, Aymeric Jégou, n’était pas franc du collier, et le voir s’acoquiner avec deux ordures de compète comme Milo Monteil et Noé Desmarais n’était pas fait pour rassurer le staff technique de la kommandantur.

Avait-il, lui qui était dans le cénacle des chemises brunes et casques à pointe, des antisémites racistes, homophobes et fiers de l’être, entendu bruisser quelque rumeur suspecte à son sujet, reniflé l’odeur fétide de quelque flatulence le concernant ?

Il a hésité, minaudé, joué les pucelles effarouchées (ce qui était un peu fort de café pour un garage à bites de son espèce), mais quand je lui ai annoncé qu’un travelo brésilien avait été retrouvé cramé au milieu des bois, une vague lueur d’intérêt s’est allumée dans le fond de son regard de méduse anorexique.

Je précise que Zaahid, après avoir effectué la synthèse experte des éléments en sa possession, en était arrivé à l’irréfutable conclusion que le grand brûlé de la forêt de Pleimelding n’était autre que Tiago Alvarez. De plus, quelques jours plus tôt, on avait retrouvé un corps dans une fosse à purin au cours d’une intervention chez un éleveur de chevaux pédophile et cannibale. Titus et moi avions copieusement tabassé le pédophile, une espèce de connard à particule qui ne se prenait pas pour la moitié d’une merde, pour lui faire cracher le morceau. Cet enculé abusait sexuellement d’enfants du voisinage, avant de les tuer à coups de fourche et les découper en morceaux. Il avait trois congélateurs dans son sous-sol, tous pleins à ras bord de barbaque (sanglier, chevreuil, marcassin, perdrix, lièvre, bécasse et compagnie, le gars était chasseur et gastronome), dont des morceaux de gosses soigneusement emballés et étiquetés qui attendaient de passer à la casserole (les pauvres y passaient deux fois, en somme). Une fois qu’on lui a eu explosé la gueule dans les grandes largeurs (même sa marie-couche-toi-là de mère, complètement folle sur ses vieux jours, si elle n’était pas morte noyée quelques années plus tôt dans cette même fosse à purin, aurait été incapable de le reconnaître), de la Maîtraie (c’était son nom, Louis-Marie de la Maîtraie, du genre je te la mettrais bien où je pense, mignon petit page au sourire d’ange) a fini par reconnaître une certaine attirance pour la chair fraîche, mais a juré ses grands dieux qu’il n’avait rien à voir dans la mort du type qui se trouvait dans sa putain de fosse à purin. Depuis le décès de sa mère, il n’y mettait jamais plus les pieds, et jamais il ne lui viendrait à l’idée de mettre quelqu’un d’autre à sa place. Depuis le décès de sa très sainte mère, cette fosse à purin était un mausolée, un lieu de culte, un endroit sacré qu’il aurait mille fois préféré mourir que profaner. Apparemment, des ordures sans foi ni loi étaient venues noyer ce type dans sa fosse à purin à son insu, sans doute la nuit pendant qu’il dormait à poings fermés, l’estomac bien rempli après un bon repas de chair humaine. Car sans se vanter, il était un cuisinier hors pair. Une fois, il avait fait goûter sa terrine à des voisins dans l’affliction suite à la disparition d’une personne proche, des gens qu’il assistait dans leur détresse, sans leur dire que leur gamin de huit ans était l’ingrédient principal de la terrine en question, et ils avaient été unanimes pour dire que de toutes les terrines qu’ils avaient eu l’occasion d’ingurgiter au cours de leur vie, et dieu sait qu’ils en avaient ingurgité des tonnes et non des moindres, celle-ci était la meilleure et de loin. Après l’avoir croisé vingt secondes dans les couloirs de la PJ, passablement esquinté mais avec toujours cette même lueur de démence dans le regard, le psy avait conclu qu’il était dingue et qu’il n’y avait à priori aucune raison de ne pas le croire s’il disait n’être pour rien dans le cadavre de la fosse. Placement d’office en HP de haute sécurité avec camisole de force et chambre capitonnée, plus neuroleptiques à haute dose et séance quotidienne d’électrochocs, affaire classée.

Pour en revenir au cadavre en question, on n’a pas eu beaucoup de mal à l’identifier. Et pour cause, il avait encore ses papiers sur lui. Certains avaient souffert de leur séjour prolongé dans le purin, mais les cartes en plastique, après un brin de toilette, étaient pleinement exploitables. Elles appartenaient à un certain Abraham Botrel, cinquante-trois ans, folle notoire disparue deux mois plus tôt sans laisser d’adresse. Le mode opératoire n’était pas le même que pour Alvarez, mais le traitement peu gratifiant qui lui avait été réservé semblait indiquer qu’une même équipe de détraqués était à l’œuvre.

Moi, assis en face de Cerqueira dans le bureau réservé aux interrogatoires musclés (celui où les caméras de surveillance tombent régulièrement en panne), Titus debout près de la porte : Va falloir nous en dire un peu plus, mon petit Tacito.

\textsc{Cerqueira} : Pas devant le nègre.

\textsc{Moi} : Pardon ?

\textsc{Lui} : Depuis quand ils engagent des nègres dans la police ?

\textsc{Moi} : Je serais de toi, j’éviterais de la ramener. Mon collègue a parfois des réactions imprévisibles, surtout en présence de têtes de cons dans ton genre.

\textsc{Lui} : M’en fous ! Je veux bien parler, mais pas devant le nègre.

\textsc{Titus} : Qu’est-ce qu’elle dit, la pédale ?

\textsc{Cerqueira} : Je t’emmerde, négro.

Titus a fait un pas en avant, l’œil chargé de mauvaises intentions, et il a dit : Comment on t’appelle, chez les tapettes ? La velue ?

Cerqueira a poussé un grognement de sanglier auquel on vient de planter une fourchette dans le cul sans lui demander son avis, après quoi il a tenté de se jeter sur Titus qui le toisait avec un petit sourire narquois, l’air content de lui. Dans sa précipitation, Cerqueira avait zappé un menu détail : il avait les mains attachées dans le dos, derrière le dossier de sa chaise. Si vous avez des doutes, faites l’expérience chez vous : asseyez-vous sur une chaise, joignez vos mains derrière le dossier et essayez de vous lever, c’est sans danger mais instructif sur les lois de la physique et les limites du corps humain.

Il a sifflé entre ses dents : Tu me paieras ça, négro.

Je me suis dit que le moment était venu de détendre un peu l’atmosphère : T’as entendu parler du 27 avril 1848 ?

\textsc{Lui} : Je dirai rien tant que cette saleté de gorille sera dans la pièce !

L’instant d’après, la main droite de Titus se dirigeait à vive allure vers la joue gauche du prisonnier, tel un missile à tête chercheuse. Une fraction de seconde plus tard, elle s’écrasait sur sa cible avec une violence telle que l’ensemble du dispositif (je parle de la joue, son propriétaire et la chaise sur laquelle il était assis) effectuait un tour complet sur lui-même dans le sens des aiguilles d’une montre.

\textsc{Titus} : Je t’ai dit de la fermer, grosse fiotte.

Cerqueira, secouant la tête (on entendait tinter les glaçons) pour se remettre les idées en place après son voyage improvisé : Bande de fumiers, vous avez pas le droit de faire ça ! Je veux parler à mon avocat, tout de suite !

\textsc{Titus} : T’as pas d’avocat, pauvre con.

\textsc{Moi} : Eh bien, le 27 avril 1848, c’est le jour où le gouvernement provisoire de la République française, qui compte parmi ses membres des personnalités aussi éminentes que Louis Blanc, Ledru-Rollin, Arago et Lamartine, signe le second décret d’abolition de l’esclavage, celui de 1794 n’ayant pas donné les résultats escomptés.

\textsc{Cerqueira} : Et alors, qu’est-ce que tu veux que ça me foute !

\textsc{Moi} : Et alors, depuis cette date, traiter un Noir de nègre est un crime passible de prendre des grandes claques dans sa sale gueule d’enfoiré de suceur de queues !

Cerqueira, massant sa joue endolorie, sur un ton geignard : Pas la peine de continuer à me parler, je dirai plus rien.

Je lui ai collé une photo sous le nez : Tu vois, ça ?

Il a jeté un œil sur le document et détourné rapidement le regard, manifestement peu enthousiasmé par ce qu’il venait de voir.

\textsc{Moi} : T’es plutôt mignonne, en jupe, à condition d’avoir un penchant pour les femmes qui ont de la moustache et du poil aux pattes. Perso, je suis pas fan. J’en ai tout un tas du même genre, dont certaines en gros plan qui font leur petit effet. Je vais peut-être envoyer ça à la téloche, je sais qu’ils adorent les docus animaliers pour meubler les longues soirées d’hiver. Tu vas faire sensation ! Je verrais bien un titre un peu accrocheur du style «~Rue Armand Brunelle, le ballet des phoques, plongée dans les chiottes de l’extrême-droite~», un reportage-choc avec une musique dramatique qui scotche le spectateur devant son écran ou le réveille quand il pique du nez dans son paquet de chips. Non, sans blague, je peux viser le Pulitzer avec un truc comme ça. Par contre, je suis pas certain que tes petits copains nazis vont trouver ça à leur goût.

Cerqueira, une larmichette au coin de l’œil : C’est pas bien de se moquer des gens, commissaire.

\textsc{Moi} : Je me moque pas, je dis juste que t’es quand même une belle grosse salope comme on n’en fait plus !

\textsc{Lui} : Vous avez pas le droit.

Moi, lui soufflant la fumée de mon Rocky Patel Disciple dans les naseaux (on n’était pas censé fumer dans les locaux de la police nationale, sauf chez le préfet, mais je m’autorisais quelques libertés dans l’enceinte de ce que j’appelais mon «~cabinet privé~», à savoir la salle d’interrogatoire spéciale sujets récalcitrants, celle où on transformait le loup en agneau, l’aigle en taupe, le grand requin blanc en sardine à l’huile) : Excuse-moi, mon chaton, mais quand des individus se permettent d’en cramer d’autres au lance-flammes, j’ai tendance à oublier la Déclaration des droits de l’homme et du citoyen. D’ailleurs, la Déclaration parle des droits de l’homme, pas des droits de l’enculé. Je me doute qu’un gros nounours comme toi n’a rien à voir dans des saloperies de ce genre, raison pour laquelle j’aimerais t’éviter des ennuis. On t’a vu au Bouclier avec Jégou, Monteil et Desmarais. Ces bouffons sont des ordures de la pire espèce et je suis certain que tu sais des choses à leur sujet. Et toutes ces vilaines choses, je veux les savoir aussi, sinon je te colle en taule pour exhibitionnisme et racolage sur la voie publique, plus deux ou trois bricoles que je n’aurai aucun mal à dénicher dans ton CV. Le tarif officiel, c’est deux mois derrière les barreaux et trois mille cinq cent balles d’amende. Le fric, passe encore, mais les deux mois vont te paraître une éternité. Tes photos vont faire le tour du Net en trois secondes et demie, tu vas devenir une star du X et les détenus vont faire la queue pour te rendre visite.

\textsc{Lui} : Vous n’avez pas le droit de faire ça.

\textsc{Moi} : Je sais, tu l’as déjà dit. En même temps, je vais pas prendre de pincettes avec un vieux kleenex comme toi, Tacito. J’ai envie qu’on se parle d’homme à homme, comme des adultes responsables.

\textsc{Lui} : J’ai droit à un avocat.

\textsc{Titus} : Ici t’as droit à rien du tout, sac à merde ! Je peux te couper les couilles et te les faire bouffer, personne ne viendra à ton secours.

Cerqueira disposait de tout un catalogue de grossièretés dans lequel il pouvait puiser à volonté pour insulter son interlocuteur en fonction de ses origines raciales. J’ai lu dans ses yeux qu’il était en train de le feuilleter pour faire son choix. Mais au moment de passer à l’offensive, alors même que sa bouche d’égout venait de s’ouvrir pour déverser le raz de marée d’immondices qu’il destinait à Titus, de quoi ensevelir à tout jamais sa dignité et le pousser dans ses ultimes retranchements, il s’est ravisé et contenté de serrer les dents à s’en faire exploser les mâchoires, tout en le fusillant du regard et soufflant bruyamment tel un taureau sur le point de charger. Bien que d’une stupidité à toute épreuve, il venait quand même de prendre conscience que chacune de ses interventions ordurières ne faisait que l’enfoncer un peu plus dans la mouise.

Moi, d’une voix aussi suave qu’un coulis de framboise sur une boule de glace à la vanille : Allons allons, je suis certain que monsieur sait parfaitement où se trouve son intérêt et ne fera aucune difficulté pour nous dire tout ce qu’il sait. N’est-ce pas, mon cher Tacito ?

Pas de réponse.

Votre serviteur, qui n’est pas du genre à se laisser décontenancer par si peu : Dans l’Antiquité, Tacite était connu pour sa grande sagesse, comme en témoignent ses écrits et sa correspondance avec Pline le Jeune. Tu connais Tacite, Tacito ?

Toujours pas de réponse, mais bref regard en coin de l’intéressé dans ma direction, un regard chargé de méfiance dans lequel j’ai cru déceler également, contre toute attente (intervention divine, Allahu akbar ! comme diraient nos amis musulmans), une vague lueur d’intelligence, au sens d’intelligence avec l’ennemi, c'est-à-dire la volonté de nouer des relations circonstanciées avec l’adversaire dans le but de satisfaire un intérêt commun.

D’où la réplique suivante : Tacite est un historien de la Rome antique. Tu devrais lire les Annales, je suis sûr que ça te plairait.

\textsc{Cerqueira} : Très drôle !

\textsc{Moi} : Bon, pour en revenir à notre petite affaire, je te cache pas que j’ai un peu de mal à te croire quand tu me dis que t’es au courant de rien.

\textsc{Lui} : C’est pourtant vrai.

\textsc{Moi} : L’ennui, vois-tu, c’est qu’on t’a vu traîner au Sugar \& Spice.

\textsc{Lui} : Au quoi ?

\textsc{Moi} : Au Sugar \& Spice, rue Théo Cazenave.

\textsc{Lui} : Ah bon ?

\textsc{Moi} : Mais oui. J’ai des petites vidéos très sympas sur lesquelles on te reconnaît très bien.

\textsc{Lui} : Jamais mis les pieds.

\textsc{Moi} : Bien sûr que si. Mais on est dans un pays libre, tu as parfaitement le droit de préférer les garçon aux filles et de fréquenter les établissements de ton choix, y compris les plus extravagants. Non, vois-tu, ce qui m’embête le plus, dans cette histoire, c’est qu’on t’a vu à plusieurs reprises en grande conversation avec un certain Tiago Alvarez, lequel nous intéresse tout particulièrement. Tu vois de qui je parle ?

Pour la première fois, j’ai senti que la carapace de Cerqueira était en train de se fissurer.

Lui, à voix basse : Oui.

\textsc{Moi} : Pardon, j’ai pas bien entendu ?

Lui, nettement plus fort : Oui !

\textsc{Moi} : Excellent, on avance à grands pas. Et tu sais où il se trouve, en ce moment-même ?

\textsc{Lui} : Non.

\textsc{Moi} : Dans un tiroir, à la morgue. Et tu sais pourquoi il est à la morgue ?

\textsc{Lui} : Non.

\textsc{Moi} : Eh bien je vais te le dire. Il se trouve à la morgue parce que des gens très mal intentionnés l’ont fait griller comme une vulgaire saucisse. Et on a toutes les raisons de penser que ces gens très mal intentionnés font partie de la bande de nazillons avec laquelle tu traînes habituellement. Je me trompe ?

\textsc{Lui} : J’ai rien fait.

\textsc{Moi} : Raison de plus pour dire la vérité.

Lui, transpirant à grosses gouttes : Ils vont savoir que c’est moi.

\textsc{Moi} : Qui ça, «~ils~» ?

\textsc{Lui} : Ben eux !

\textsc{Moi} : Jégou et les autres ?

\textsc{Lui} : Oui, Aymeric, Monteil et Desmarais !

\textsc{Moi} : Ils ont buté Alvarez, pas vrai ?

\textsc{Lui} : Il ne m’arrivera rien ?

Moi, un sourire carnassier aux lèvres : Pas si tu dis la vérité. Et pas si tu dis à Titus que tu regrettes de l’avoir traité de nègre.

Lui, jetant un regard torve à Titus, qui se tenait debout à quelques centimètres de lui, le surplombant de toute la puissance musculeuse de son anatomie hors norme (et pour tout dire assez flippante quand on la voyait débouler par une nuit sans lune au détour d’une ruelle mal éclairée) : Quoi ?

\textsc{Moi} : Dis à Titus que tu regrettes de l’avoir traité de nègre.

Titus, d’une voix faussement douce : Tu sais, les Noirs sont des gens comme les autres, avec une tête, des bras, des jambes et un cœur qui bat. C’est juste que les gens naissent dans des endroits différents, avec des couleurs différentes, adaptées à leur environnement.

Cerqueira, la larme à l’œil, le nez reniflant : Oui, je sais bien.

\textsc{Titus} : T’es pas complètement con, quand même ?

\textsc{Lui} : Non.

\textsc{Moi} : Bien sûr qu’il y a encore de l’espoir de le sauver. Certains sont irrécupérables, comme Jégou, mais d’autres peuvent encore rejoindre les rangs de l’humanité. Et je suis persuadé que notre ami Tacito fait partie de ceux-là. Hein Tacito ?

Tacito, reniflant de plus en plus fort, tel un sanglier qui a flairé une truffe bien mûre et juteuse sous le tapis de feuilles mortes de l’automne : Oui. J’en ai marre de toutes ces conneries.

\textsc{Titus} : Alors dis que t’aimes les Noirs et que tu regrettes de m’avoir traité de nègre.

\textsc{Moi} : Tu veux un kleenex, Tacito ?

Lui, les yeux remplis de reconnaissance : Oui, je veux bien.

\textsc{Titus} : Dis que tu regrettes.

\textsc{Lui} : Je suis désolé, monsieur, je recommencerai plus.

\textsc{Titus} : Dis «~je regrette, monsieur Titus, de vous avoir traité de sale nègre~».

\textsc{Lui} : Je regrette, monsieur Titus, de vous avoir traité de sale nègre.

\textsc{Moi} : C’est vrai, ça, Tacito, je ne comprends pas comment on peut en vouloir à quelqu’un parce qu’il est physiquement différent. Tu aimerais, toi, qu’on te traite de sanglier parce que tu es couvert de poils de la tête aux pieds ?

\textsc{Lui} : Non, m’sieur. Je peux avoir mon mouchoir, maintenant ?

\textsc{Moi} : Alors écoute-moi bien, mon petit Tacito : est-ce que, si je te détache les mains pour te prouver ma confiance, tu seras respectueux de l’honneur que je te fais et ne tenteras pas de te servir de tes mains pour autre chose que moucher ce vilain nez qui pisse comme une fontaine ?

Lui, ravalant un filet de morve : Ça veut dire quoi, m’sieur ?

\textsc{Moi} : Ça veut dire : est-ce que tu promets de pas faire le con si je te détache ? Je peux te faire confiance ?

\textsc{Lui} : Oh oui, m’sieur le commissaire, bien sûr !

\textsc{Moi} : Je suis pas commissaire mais c’est pas grave, l’important c’est de savoir que je peux te faire confiance.

\textsc{Lui} : Mais alors vous êtes quoi, au juste ?

\textsc{Moi} : Titus, détache-le, s’il te plaît. Eh bien disons, pour répondre à ta question, que j’occupe une place un peu particulière au sein de la police nationale. Titus et moi opérons dans des conditions particulières, un peu hors-champ, si tu vois ce que je veux dire. Nous sommes, en quelque sorte, des hommes de l’ombre. Tu as vu le film de Lee Tamahori ? Titre original : Mulholland Falls, parce que ça se passe à Los Angeles, entre Hollywood et la vallée de San Fernando. T’es déjà allé à Los Angeles, Tacito ?

Tacito, soufflant comme un phoque (sans mauvais jeu de mots) dans la poignée de mouchoirs que je venais de lui refiler : Non. Et vous ?

\textsc{Moi} : J’essaie d’y aller au moins une fois par an. Non, je rigole ! Comment veux-tu qu’un pauvre flic comme moi ait les moyens d’aller à Los Angeles. Les Hommes de l’ombre, de Lee Tamahori, parle d’une unité spéciale de la police de Los Angeles dont la fonction est de nettoyer les rues de la ville en toute discrétion. Je fais un peu la même chose. À mon modeste niveau, bien sûr ! Ça y est, t’as fini ? Je ne pensais pas qu’un nez humain, même gros, pouvait contenir autant de morve !

Lui, s’essuyant le nez et les régions alentours, touchées elles aussi par le sinistre : J’ai fini, inspecteur.

\textsc{Moi} : Je suis pas vraiment inspecteur, mais c’est pas grave. Moins t’en sais, mieux ça vaut pour toi.

Titus, refilant une autre poignée de kleenex à Tacito : Tiens, essuie-toi, t’en as partout. T’es vraiment un gros porc !

\textsc{Tacito} : Dites pas ça, m’sieur le Noir. C’est juste que mon père était au chômage et que ma mère faisait des ménages pour gagner notre pain. J’ai pas été dans les grandes écoles avec les gens de la haute.

\textsc{Titus} : Tu peux m’appeler Inspecteur, si ça te défrise pas trop les bigoudis. Tacito, par contre, c’est pas facile à dire. Trop long. Je vais t’appeler Tata, si ça te dérange pas trop.

\textsc{Tacito} : C’était mon surnom à l’école communale.

\textsc{Titus} : T’aimes bien qu’on t’appelle Tata, alors. Et puis ça correspond bien à ce que t’es : une grosse tata.

Tacito, s’épongeant le pif : J’ai pas fait exprès, chef.

\textsc{Titus} : Pas grave, je t’aime bien quand même. On a tous nos petites différences, c’est pas une raison pour se faire la gueule, tu crois pas ?

\textsc{Tata} : Bien sûr, inspecteur-chef.

Titus, lui tendant la poubelle : Tiens, mets tes détritus là-dedans. Tu me fais de la peine avec ton gros nez tout mouillé et tes doigts pleins de morve. Nous, en Afrique, on est tellement pauvres qu’on n’a pas de kleenex pour s’essuyer. On fait ça avec des touffes d’herbe sèche ramassées dans la savane. Tu connais la Sierra Leone ?

\textsc{Tata} : Non, m’sieur. C’est où ? En Espagne ?

\textsc{Titus} : En Afrique, mon pote ! Qu’est-ce que tu veux que j’aille foutre en Espagne ! Aussi bizarre que ça puisse paraître, c’est de là que viennent mes ancêtres, mon petit Tata. C’est comme qui dirait le berceau de l’humanité. Même toi, qui détestait les Noirs il n’y a encore pas cinq minutes, eh ben il se trouve que t’as du sang noir dans les veines. Peut-être pas beaucoup, c’est vrai, mais assez quand même pour que t’en aies un peu. Qu’est-ce que tu dis de ça ?

\textsc{Tata}, écarquillant des yeux larges comme des roues de vélo : Je savais pas !

\textsc{Titus}, tout sourire : Ben tu le sais, maintenant.

Je vous l’avoue franchement, j’avais profité de l’intermède pour fumer quelques taffes de mon Rocky Patel Disciple (pas de la colère) en toute tranquillité, après quoi j’ai estimé que leur petite conversation, objectivement nulle à chier, n’avait aucune raison de s’éterniser.

C’est donc avec tact mais fermeté, toujours parfait dans le rôle de l’animateur charismatique, du meneur d’hommes déterminé mais sensible, d’une telle intelligence supérieure, certes, mais toujours au service de l’intérêt collectif, le mieux-être de la communauté (c’est fou ce que j’adore parler de moi, surtout en bien, sachant que je suis tenu par le secret professionnel sur bien des sujets que j’aimerais aborder de façon plus substantielle si la société n’était pas ce qu’elle est, avec ses grandes oreilles qui trainent partout et ses censeurs toujours prêts à bondir sur le dissident) : Tu vois, Tacito, je pourrais te rattacher les mains dans le dos et reprendre l’interrogatoire à l’endroit où on en était resté, comme si de rien n’était. Mais je ne vais pas le faire, parce que j’ai décidé de te faire confiance. Oui, je pense qu’on a fait un bon bout de chemin ensemble et que tu seras plus à l’aise pour parler si tu conserves ta liberté de mouvement. Naturellement, s’il te venait à l’idée de faire des bêtises, tu comprendras que je pourrai difficilement empêcher Titus, mon fidèle adjoint ici présent, de faire usage de son arme.

Plus rapide que le faucon qui s’abat sur le mulot insouciant, Titus, sourire de tueur aux lèvres, a sorti le Glock qu’il planquait dans son dos, lui a fait faire une série de moulinets façon Jamie Foxx dans Django Unchained, puis l’a remis à sa place.

\textsc{Re-moi} : Mais avoue que ce serait dommage, parce que vous êtes maintenant devenus des amis, des gens qui se comprennent et se respectent, et que je sais que ce n’est pas sans un pincement au cœur que Titus se verrait dans l’obligation de faire sauter ta cervelle de moineau.

\textsc{Tata} : Je serai sage comme une image, chef.

\textsc{Votre serviteur} : Bien. On en était où, déjà ?

\textsc{Titus} : Aux Disciples de la Colère, si ma mémoire est bonne.

\textsc{Moi} : Et elle est bonne, mon cher Titus, et je dirais même excellente ! Oui, c’est bien ça, les Disciples de la Colère de mes deux, autrement dit Jégou, Monteil et Desmarais ! Tu me corriges si je me trompe, Tata.

\textsc{Tacito} : Non, c’est bien ça.

\textsc{Moi} : Tu veux un cigare ? Non, parce que si tu veux un cigare, c’est avec plaisir que je t’en offre un. Je suis comme ça, moi, j’aime que mes amis soient bien traités.

\textsc{Tata} : J’sais pas, m’sieur, j’ai pas l’habitude de fumer.

\textsc{Moi} : Okay, comme tu veux ! T’en veux pas, t’en veux pas, je vais pas te forcer ! Je veux juste que tu te sentes à l’aise, détendu pour répondre à mes questions. Tu as soif, tu veux peut-être un verre d’eau ?

\textsc{Lui} : Oui, j’veux bien.

\textsc{Moi} : C’est con, on n’en a pas. Par contre, j’ai ça.

J’ai sorti une flasque de la poche de ma veste : C’est peut-être pas l’idéal pour étancher la soif, mais ça va te donner un bon coup de fouet si tu te sens un peu mou du genou. T’en veux ?

\textsc{Lui} : C’est quoi, chef ?

Titus, d’humeur badine : Du schnaps.

\textsc{Moi} : Jamais de la vie ! Non, du whisky, bien sûr, du vrai, pas du whisky de tap… de ped… enfin… du vrai whisky, quoi, en provenance directe des Highlands. Aberfeldy 12 ans d’âge, pas piqué des hannetons !

Je m’en suis jeté une rasade et j’ai demandé à Titus, qui avait l’air de bien se marrer : T’en veux ?

\textsc{Lui} : Non, jamais pendant le sévice.

Puis j’ai tendu la flasque à notre ami Tata, recroquevillé sur sa chaise comme une crotte de nez dans le fond d’un mouchoir : Une petite goutte ?

\textsc{Lui} : C’est pas de refus.

Je pensais qu’il allait se contenter d’un petit coup comme ça, vite fait, en passant, une petite gorgée pour la route, mais eu lieu de ça, cette espèce de pithécanthrope en phase terminale de putréfaction mentale n’a rien trouvé de mieux à faire que de lessiver la quasi-entièreté de ma flasque en une seule aspiration, comme s’il avait une pompe à whisky à la place du tube digestif.

Votre humble serviteur, après avoir récupéré et remis ses petites affaires en place : Bon, Tacitouille, maintenant que tu t’es bien rincé la dalle, j’espère que t’es prêt à répondre à mes questions.

\textsc{Tacito}, après un claquement de langue significatif : Fin prêt, commissaire.

\textsc{Moi} : À dire la vérité, toute la vérité, rien que la vérité ?

\textsc{Tacito}, main droite levée comme pour un petit salut nazi mi-coude à la Hitler : \textit{\foreignlanguage{portuguese}{Nada além da verdade, eu juro!}}

\textsc{Titus} : C’est quoi, ce charabia ?

\textsc{Tacito} : Du portugais, chef.

\textsc{Moi} : Mouais. Bien, donc, pour en revenir à tes petits camarades Jégou, Monteil et Desmarais, on est bien d’accord que ces enculés ont buté Botrel et Alvarez.

\textsc{Tata} : Botrel, je sais pas, mais Alvarez oui.

\textsc{Moi} : Botrel tu sais pas ?

\textsc{Tata} : Non, je sais même pas qui c’est.

\textsc{Moi} : Abraham Botrel, antiquaire, grosse fortune, folle notoire qui se prenait pour la réincarnation de Charles de Beaumont, alias le chevalier d’Éon, retrouvé noyé dans la fosse à purin d’un aristo pédophile et cannibale, ça te dit rien ?

\textsc{Lui} : Rien du tout.

\textsc{Moi} : Faut que je te le dise en portugais, ou quoi ?

\textsc{Lui} : J’ai juré de dire toute la vérité, rien que la vérité, chef.

Titus, faisant craquer les jointures de ses doigts : Je sais pas toi, mais moi j’ai comme l’impression qui se fout de notre gueule.

\textsc{Tacito} : Je jure sur la tête de ma mère que je connais pas de Botrel !

\textsc{Titus} : Parce que t’as une mère, toi ?

\textsc{Tacito} : Ben oui, comme tout le monde. Sauf que ma mère à moi, c’est mon père.

\textsc{Titus} : Quand je te disais qu’il se fout de notre gueule !

\textsc{Moi} : Oui, bon, on s’en fout de sa mère ! Okay, Tacito, tu ne connais pas de Botrel, mais Tiago Alvarez, tu le connais lui ?

\textsc{Lui}, sa bouche se mettant à trembloter comme s’il allait éclater en sanglots : Oui. Pour tout vous dire, j’étais raide-dingue de lui ! Je l’ai rencontré au Sugar \& Spice pendant une soirée déguisée, on a couché ensemble une ou deux fois, et puis après je me suis mis à flipper et j’ai voulu tout arrêter. J’aime les garçons, je l’avoue, mais j’ai pas intérêt à ce que ça se sache si je veux pas finir dans une benne à ordures avec le signe d’infamie tatoué sur le front.

\textsc{Moi} : Le signe d’infamie ?

\textsc{Lui} : Ouais, une croix dans un cercle, la marque de Caïn.

\textsc{Moi} : Parce qu’une croix dans un cercle, c’est la marque de Caïn ?

\textsc{Lui} : J’en sais rien, moi !

\textsc{Moi} : T’en sais rien, t’en sais rien ! Faudrait peut-être voir à savoir un peu quelque chose, de temps en temps !

\textsc{Lui} : Ce que je sais, c’est que j’ai annoncé à Tiago que tout était fini entre nous et qu’il l’a très mal pris. Il a commencé à me harceler, venir faire du grabuge en bas de chez moi, rappliquer sans prévenir sur mon lieu de travail avec des fringues provocantes sur le dos. Aymeric lui a dit de se barrer, a menacé de lui casser la gueule, mais Tiago n’était pas du genre à se laisser impressionner. À l’entendre, il était champion de capoeira et n’avait pas peur de se battre. Un jour, Aymeric a décrété que, champion de capoeira ou pas, le moment était venu de lui donner une bonne leçon, et que j’avais intérêt à filer doux si je voulais pas qu’il m’arrive des bricoles. Je devais lui tendre un piège, sous un prétexte quelconque, et les autres, les Disciples de la Colère, devaient lui tomber dessus et lui flanquer la frousse de sa vie. Je voulais pas qu’ils lui fassent de mal, pas trop en tout cas, juste lui foutre la trouille pour qu’il arrête ses conneries. Donc je l’ai appelé, disant que j’étais toujours amoureux de lui, que j’avais changé d’avis et voulais continuer à sortir avec lui. Ce qui n’était pas totalement faux, d’ailleurs, mais je pouvais difficilement le dire aux autres. Je lui ai donné rendez-vous chez moi vers minuit, et on lui est tombé dessus au moment où il s’apprêtait à sonner à l’interphone. Aymeric lui a braqué un flingue sous le nez, un Luger P08 hérité de son arrière-grand-père qui avait bossé pour la Gestapo, au 11 rue des Saussaies, et on l’a obligé à monter dans le Sprinter (utilitaire léger de chez Mercedes, ndlr) de Noé.

\textsc{Moi} : Donc t’étais présent au moment des faits.

\textsc{Lui} : Ouais, pas chez moi mais dans le Sprinter avec Aymeric, Milo et Noé au volant. On guettait son arrivée, et dès qu’il a pointé le bout de nez, Aymeric et moi on est sorti et on l’a fait monter de force dans le Sprinter. Noé a démarré en trombe et on a foutu le camp. Pendant que Milo le tenait en joue, je lui ai attaché les mains dans le dos avec du ruban adhésif.

\textsc{Moi} : Quel genre de ruban adhésif ?

\textsc{Lui} : Le genre dont on se sert pour masquer les murs quand on fait des travaux de peinture. Noé en a toujours dans son fourgon, en plus de ses accessoires habituels.

\textsc{Moi}, pianotant sur le bureau, signe que je commençais à trouver le temps long (je suppose d’ailleurs que vous aussi) : À savoir ? Dis donc, mon vieux, va falloir être un peu plus loquace. Je vais pas te sortir les vers du nez un par un !

\textsc{Lui}, aussi à l’aise que s’il barbotait les fesses à l’air dans un marigot infesté de sangsues, alligators et piranhas, le combo parfait pour des vacances réussies : Marteau, perceuse, barre de fer, manche de pioche, des outils de ce genre. Rien d’étonnant à ça, il travaille dans le bâtiment.

\textsc{Moi} : Je t’en foutrai, moi, du bâtiment !

\textsc{Lui}, essayant vainement de remonter sur la berge : Bon, c’est vrai qu’il s’en sert aussi pour autre chose que monter des murs ou ravaler des façades.

\textsc{Moi}, jetant un œil sur ma Rousselot P06 Ultramatic 720 (c’est une montre, si vous voulez tout savoir, que la somptueuse Zarina Brizzi, mystérieuse orientale au regard de braise~-- la Toscane est bien en Orient, non~-- somptueuse, disais-je, Zarina Brizzi, m’avait offerte pour mon anniversaire, lequel je ne vous le dirai pas, ça ne vous regarde pas, sachez seulement que c’était la première fois que quelqu’un, une femme en particulier, somptueuse qui plus est ce qui ne gâte rien (c’est quand même pas pareil si une femme moche, un infâme boudin dont on ne voudrait même pas pour récurer ses chiottes, vous refourgue une tocante toute pourrie achetée une bouchée de pain à un vendeur à la sauvette), m’offrait une montre, raison pour laquelle la chose m’était allée d’autant plus droit au cœur, telle une flèche décochée par quelque angelot grassouillet, me laissant pour ainsi dire sans voix, dans un état d’ébriété émotionnelle que je n’aurais jamais cru pouvoir atteindre un jour, et je ne parle pas seulement de la qualité de l’objet, indéniable, mais aussi des circonstances et la façon dont il m’avait été offert) : À Cordoue, on ne mange jamais de queue de taureau de combat. Il n’est pas rare que les gens me posent la question, aussi tenais-je à vous le dire avant de passer à autre chose.

Les autres m’ont regardé bizarrement, comme si j’étais de couleur verte, ou bleue, avec des yeux de lémurien, une trompe à la place du nez, des antennes télescopiques sur le front, des tentacules à la place des bras et des turboréacteurs à la place des jambes.

\textsc{Titus} : Je sais pas ce qu’ils mettent dans le whisky, mais je crois que tu devrais te reposer un peu.

\textsc{Tacito} : Il est toujours comme ça ?

\textsc{Titus} : T’emballe pas, Toto, il est juste un peu surmené en ce moment.

\textsc{Moi} : Ça va très bien, persil. On en était où ?

\textsc{Titus} : Nos quatre joyeux drilles venaient d’embarquer Tiago Alvarez sous la menace d’une arme.

\textsc{Moi} : Ah oui, c’est ça. Et ensuite ?

\textsc{Toto} : Ensuite on a roulé, on est sortis de la ville, et on s’est retrouvés quelque part dans la forêt, vers une heure du matin. Je pensais qu’ils allaient juste le tabasser un peu, mais quand j’ai vu Noé sortir du Sprinter avec un lance-flammes à la main, j’ai vu le bad trip arriver à plein nez !

\textsc{Moi} : Ça veut rien dire, ça, «~j’ai vu le bad trip arriver à plein nez~» !

\textsc{Lui} : Ah bon ?

\textsc{Moi} : Non, rien du tout.

\textsc{Lui} : Ben je me suis dit comme ça, à l’intérieur de moi-même, que les choses risquaient de tourner au vinaigre.

\textsc{Moi} : Et alors, qu’est-ce que t’as fait ?

\textsc{Lui} : J’ai refusé d’aller plus loin, les suivre dans les profondeurs de la nuit et sacrifier un innocent pour des conneries de runes nordiques et autres légendes celtiques tatouées sur de la peau humaine dans des vieux grimoires poussiéreux. J’ai dit que s’ils faisaient du mal à Tiago, j’arrêtais tout, je plaquais définitivement le Mouvement. Mais ils en avaient rien à foutre, ces cons ! On s’est pris le chou, une bagarre a éclaté, je me suis pris un coup de matraque télescopique derrière les oreilles et je me suis retrouvé le nez dans les feuilles mortes à ronfler comme un bienheureux. Et je peux dire que j’ai eu de la chance, en effet, parce que grâce à ça j’ai échappé au carnage qui s’en est suivi. Noé est un grand malade qui prend son pied à foutre le feu et regarder les poubelles et les bagnoles brûler, mais jamais j’aurais pensé qu’il était capable de foutre le feu à quelqu’un ! Franchement, c’est l’horreur ! Quand je me suis réveillé, une heure plus tard, les autres n’étaient toujours pas là. J’ai attendu, rongé par l’anxiété comme vous pouvez l’imaginer, en lançant des appels pour essayer de savoir où ils étaient. Je les ai vus revenir une heure plus tard, mais ils n’étaient plus que trois. Quand je leur ai demandé où était Tiago, ils m’ont répondu que c’était pas la peine de l’attendre, qu’il rentrerait plus tard par ses propres moyens. J’ai bien vu qu’Aymeric était tout retourné, mais il n’a rien voulu me dire de plus. Pas pour l’instant, en tout cas. Il était incapable de parler, totalement tétanisé par ce qu’il venait de subir. Aymeric est un nazi de la première heure, c’est vrai, un nostalgique du grand Reich aryen qui devait dominer le monde et régner en maître sur la Terre, mais c’est pas un assassin. Je dis pas que c’est un saint, loin de là, mais il a des valeurs et pense que tout être humain a droit à une chance de rédemption.

\textsc{Moi} : Arrête, tu vas me faire chialer. Donc, si je comprends bien, tu dormais tranquillement le nez dans la mousse et tu ne sais pas ce qui s’est passé dans la forêt ?

\textsc{Lui} : Non, rien sur le moment. Je vous le jure, commissaire ! Je l’ai su par la suite, quand Aymeric, qui en avait gros sur la patate, m’a tout raconté. Franchement, commissaire, je suis dégoûté !

\textsc{Titus} : Et dégoûtant, surtout.

L’entretien s’était poursuivi de la sorte pendant un certain temps, après quoi, estimant qu’on avait pressé le citron jusqu’à la dernière goutte de jus aigre et indigeste (et bu le calice de la misère humaine jusqu’à la lie, une vraie purge), on avait décidé de le foutre en cabane sous un certain nombre de chefs d’accusation, tous plus d’hiver et avariés les uns que les autres, parmi lesquels association de malfaiteurs, non-assistance à personne en danger et incitation à la haine raciale. En fait, il s’agissait surtout de s’assurer que ce sinistre crétin n’entrerait pas en contact avec ses ex-comparses pour les avertir que le bras séculier de la justice était sur le point de s’abattre sur eux. Ce bras séculier, vêtu d’un long manteau noir et armé d’une faux au tranchant ébréché à force de s’échiner sur des nuques rétives, se ferait une joie de leur séparer la tête du tronc et les envoyer croupir à tout jamais dans les culs-de-basse-fosse de la connerie humaine, précipice insondable dont on voyait chaque surgir des créatures plus monstrueuses les unes que les autres. Car l’espèce humaine, nous le savons, en dépit de certaines qualités qu’on ne saurait lui dénier (phrase toute faite, j’en conviens, parce que j’ai beau chercher, avec toute la force de mes petites cellules grises embuées par les vapeurs de Barolo, je ne suis pas certain de voir de quelles qualités on parle), reste quand même, comment dirais-je… une entité problématique dont on ne sait pas très bien quoi faire à court ou moyen terme, sachant qu’elle n’est pas là depuis très longtemps mais a déjà accompli un certain nombre de performances dont on ne sait trop quoi penser, sinon qu’elles convergent toutes dans un seul et unique sens, un peu embarrassant il faut bien le dire, à savoir assurer à ladite engeance une espèce de suprématie naturelle censée légitimer, envers et contre tout, la consubstance quasi divine de sa nature intrinsèque, son approche mystique et dévoyée de l’existence, sa vision totalement égocentrée du monde, de l’univers et tout ce qui s’ensuit, et, au-delà même de l’univers et tout ce qui se trouverait en excéder, par quelque malversation cosmique d’obscure nature, les sublimes contours, l’épicentre tant géostratégique que métaphysique de sa petite personne.

Vous avez compris quelque chose ? Tant mieux pour vous, parce que moi je vous avouerai humblement que je nage en plein brouillard. Il y avait finalement assez peu de chances pour que, dans l’infinie solitude de l’univers, éclose sur une planète perdue au fin fond de l’espace une forme de matière inhabituelle, interactive et redondante, susceptible de s’envisager elle-même, et capable d’incarner à elle seule toutes les tensions et forces contraires qui s’agitent au sein du creuset originel.

Par exemple, ce soir-là, alors même que les combats fratricides faisaient rage aux quatre coins de la planète, laissant dans leur sillage des femmes en pleurs serrant dans leurs bras la dépouille ensanglantée de leurs enfants, j’avais convié un modeste couple de mes plus proches amis, en l’occurrence Zaahid Shirani et Tosca Brizzi, à une petite collation sans prétention à mon humble domicile, au 157 rue des Anus en fleurs (adresse fantaisiste, vous l’aurez compris et je vous remercie d’avance pour votre sollicitude, mais je ne puis, pour des raisons de sécurité évidentes, dévoiler ici l’adresse exacte dudit domicile), domicile, vous ne l’ignorez plus maintenant, que j’avais l’honneur et le privilège de partager avec Zarina Brizzi, la propre sœur jumelle de cette chère Tosca. Je précise à ce sujet que le fait d’avoir sensiblement la même femme avait considérablement renforcé les liens qui nous unissaient déjà, Zaahid et moi, au point, sans aller tout à fait jusque-là, que nous nous considérions maintenant presque comme des frères, unis par une forme de gémellité profane, certes, et dizygote à souhait, mais néanmoins bien réelle.

Zarina, pour l’occasion, avait préparé avec amour un \textit{\foreignlanguage{italian}{bollito misto}} puissamment goûtu, sorte de pot-au-feu à base de bœuf, poule et \textit{\foreignlanguage{italian}{cotechino di puledro della provincia di Belluno}} (viande de poulain et couenne de porc, plus mélange d’épices en proportions variables), le tout servi avec une sauce verte (cresson, cerfeuil, corne de cerf et pimprenelle) du plus bel effet, mais suffisamment douce pour ne pas interférer avec la finesse extrême du breuvage élégiaque que j’avais choisi pour l’accompagner.

Ce breuvage, dont tous les mots, y compris ceux du plus brillant poète ou œnologue le plus averti, seraient bien impuissants à exprimer la nature profonde, n’était autre qu’un Volnay Santenots Cuvée Gauvain 1984 des Hospices de Beaune qui s’est révélé boxer très au-dessus de sa catégorie. Pour info, il provenait de la petite ponction que j’avais eu le bon goût d’effectuer sur la cave de feu Mathéo Riqueti, évêque du Sanctuaire de Ddarr, ami personnel de Prospero Cangelosi (le chef de chœur du Vatican) et ordure patentée avec laquelle, on s’en souvient, j’avais entretenu des relations pour le moins électriques.

S’il peut sembler assez inhabituel de servir ce genre de chose avec un bollito misto, un vin de moindre pédigrée (qu’on peut aussi écrire pedigree, puisqu’il vient de la déformation de «~pied de grue~» par les Anglais, «~pied de grue~» désignant à l’origine un arbre généalogique), plus jeune et fruité, lui étant généralement préféré, j’insiste pour dire et répéter que personne, autour de la table, n’a eu la mauvaise grâce de s’en plaindre. Je pense, moi, que servir un grand vin, même avec de la cervelle de singe flambée au calva ou des couilles de taureau farcies à la confiture de fraise, n’est jamais une perte de temps, même s’il convient bien évidemment de se rincer abondamment la bouche à l’eau tiède entre chaque gorgée.

Pour la conversation qui va suivre, je tiens à préciser que Tosca, au même titre que sa sœur, s’exprimait avec un accent italien particulièrement charmant et délicieux, une merveille de suavité que je ne tenterai en aucune façon de retranscrire par le truchement de quelque prose lourde et disgracieuse, en flagrante contradiction avec la nature même du modèle. Ne comptez pas sur moi pour commettre une telle hérésie, bande de lapereaux maléfiques !

\textsc{Zaahid} : Je reprendrais bien un peu de bollito misto, moi !

Il n’en fallait pas davantage pour que Zarina le resserve copieusement, avec toute l’élégance qu’elle savait mettre dans le moindre de ses gestes, comme si une force supérieure de beauté et d’intelligence l’animait en permanence.

\textsc{Moi} : Tu es sûr que tu seras des nôtres, ce soir ?

On avait prévu, Titus, Greg et moi, de mettre un terme définitif aux agissements des Disciples de la Colère, cette bande de fanatiques d’extrême-droite qui semaient la mort et la destruction sur leur passage. On savait qu’ils devaient se réunir à minuit dans un endroit secret, secret que Cerqueira avait eu la bonne idée d’éventer pour nous. Il n’avait pas trop le choix, c’est vrai, mais sachant qu’il devait lui-même assister à cette réunion, il prenait un risque non négligeable en crachant le morceau.

Zaahid, la bouche pleine : Et comment !

\textsc{Tosca} : Non, c’est bien trop dangereux.

\textsc{Zaahid} : Mais j’ai besoin d’action, moi ! J’en ai marre de passer mon temps à tripatouiller des cadavres.

Son verre était vide. Et pour cause, il ne cessait de le vider.

Moi, toujours au petit soin avec les amis : Un peu de Volnay ?

\textsc{Lui} : Oui, volontiers.

Tosca, qui était assise en face de moi (le plan de table, susceptible de varier au gré des circonstances, était ainsi composé ce soir-là, de Tosca en face de moi, Zarina à ma gauche, et Zaahid en face de Zarina), m’a refilé un coup de pied sous la table, non parce qu’elle me détestait et prenait un plaisir sadique à me faire souffrir, mais parce qu’elle entendait avec ce geste aussi discret que percutant me rallier subrepticement à sa cause, qui était de tout tenter pour décourager Zaahid de prendre part à notre petite virée digestive qui risquait de tourner à Règlements de comptes à OK Corral, avec votre serviteur dans le rôle de Wyatt Earp et l’honorable Greg Lussier dans celui de Doc Holliday.

Moi, le resservant : Mon cher Zaahid, tu sais que je t’aime beaucoup et que je ferais tout pour t’être agréable, mais je ne suis vraiment pas certain que ce soit une bonne idée de t’emmener avec nous ce soir.

Zarina, abondant dans notre sens : Il a raison, tu es complètement ivre.

Techniquement parlant, on s’était enfilé une bouteille de Corton Charlemagne à l’apéro (cuvée François de Salins 2014 des Hospices de Beaune, même provenance que le Volnay, une bouteille tout à fait remarquable dont le niveau avait baissé tellement vite qu’on s’était demandé si on n’avait pas ouvert une bouteille vide), plus une bouteille et demie de Volnay puisqu’on était en train de lessiver la deuxième, ce qui, pour quatre adultes en pleine possession de leurs moyens, ne représentait pas une charge d’alcool invraisemblable dans le sang, même si ça commençait tout de même à devenir légèrement problématique pour des choses à priori aussi anodines que prendre le volant ou se tenir en équilibre sur une jambe.

\textsc{Zaahid} : Non, je ne suis pas ivre. D’ailleurs, j’ai amené ça.

On s’était demandé pourquoi il était venu avec un sac de sport et ce qu’il pouvait bien y avoir à l’intérieur. Vous connaissez beaucoup de gens que vous invitez à dîner et qui débarquent avec un sac de sport ? Moi pas, et c’était d’autant plus interpellant que Zaahid, intellectuel exotique à la sensibilité réelle même si pas toujours explicite, avait toujours témoigné de la plus océanique répulsion pour tout ce qui touchait de près ou de loin à la cause sportive, son intérêt pour la discipline se limitant accessoirement à l’observation attentive des musculatures en action pendant l’effort, féminines notamment. Par exemple, et je peux en témoigner parce qu’il m’a obligé à me farcir des meetings en sa compagnie à de nombreuses reprises, faisant fi de mes dénégations les plus énergiques, l’athlétisme avait ses faveurs, surtout quand il s’agissait de voir courir ou encore sauter en hauteur ou à la perche des jeunes filles (souvent de couleur) à la plastique impressionnante moulées dans des tenues quasi inexistantes.

C’est ainsi, sous les yeux ébahis d’une assistance au bord de la congestion cérébrale, qu’il a ouvert son sac et, après avoir fouillé dedans pendant quelques instants, en a sorti un jouet en plastique qui ressemblait vaguement à une arme de poing.

\textsc{Moi} : C’est quoi, ça ?

\textsc{Lui} : Un pistolet 9 mm, que j’ai fabriqué moi-même avec mon imprimante 3D.

\textsc{Moi} : Tu m’avais caché ça. Et tu comptes t’en servir pour quoi ? Aller à la chasse aux papillons ?

\textsc{Lui} : Tu peux toujours rigoler. Il se trouve que cet engin fonctionne parfaitement, et offre une puissance de feu identique à celle d’un pistolet conventionnel tout en étant beaucoup plus léger et pratiquement indétectable.

\textsc{Moi} : T’as pas peur qu’il t’explose entre les pattes ?

\textsc{Zarina}, d’une voix fraîche et chantante comme les eaux claires du Tronto à Ascoli Piceno : Quelqu’un veut encore du bollito misto ?

\textsc{Moi}, d’une voix aussi douce et satinée qu’une comptine pour enfant en bas âge : Non merci, ma chérie.

\textsc{Tosca} : Même chose pour moi, je suis prête à exploser.

\textsc{Zarina} : Zaahid ?

\textsc{Zaahid} : Ce serait avec plaisir, mais je ne tiens pas à sortir de table sur une civière.

\textsc{Moi} : Rassure-moi, tu n’as quand même pas sérieusement l’intention de venir chasser le nazi avec cet équipement dérisoire ?

\textsc{Lui} : Dérisoire mon cul ! On fait des trucs très bien, de nos jours, avec une imprimante 3D. Est-ce que tu sais, par exemple, que les Sentinelles de la Révolution du Pakistan sont équipées de HGGP-9 ?

\textsc{Moi} : De quoi ?

\textsc{Lui} : Des HGGP-9, pour Homemade Ghost Gun Pistol 9 mm, un pistolet 3D dont le modèle a été publié en open source sur le Dark Net par Colton Murray, un crypto-anarchiste d’extrême-droite, fondateur de la Division du Chaos, récemment condamné pour trafic d’armes, détention d’images pédopornographiques et agression sexuelle sur mineure.

\textsc{Moi} : Et tu as imprimé cette merde avec une imprimante 3D ?

\textsc{Lui}, arborant un visage rayonnant de joie et de fierté : Oui, et elle marche du feu de Dieu !

\textsc{Zarina}, dont je dois reconnaître qu’à ma demande elle était quasiment nue, au même titre que sa sœur, et ce, conformément aux usages en vigueur au sein de notre petite communauté : Dans ce cas, je débarrasse et on passe au fromage.

\textsc{Tosca} : Je viens avec toi.

Et pourquoi, me direz-vous, parce que j’entends déjà des voix s’élever dans la pénombre pour crier au loup et exiger que toute la lumière soit fait sur de tels agissements, pourquoi obliger ces malheureuses créatures à évoluer en tenue d’Ève alors que leurs conjoints ne sont en aucun cas soumis à une telle injonction ?

Eh bien, mes chers amis, permettez-moi de vous détromper amplement à ce sujet, car il se trouve que nous-mêmes ici présents, à savoir le docteur Zaahid Shirani et votre humble serviteur, étions tous deux vêtus avec une légèreté comparable, c'est-à-dire, en tout et pour tout, un modeste slip de bain qui nous moulait le paquet avec insistance. Ben quoi ? Vous trouvez ça ridicule, pour ne pas dire plus ? Eh bien sachez, quand la température s’y prête, que manger dans de telles conditions, déchargé du poids inique de la contrainte vestimentaire, et ce sans pour autant se revendiquer de la mouvance naturiste stricto sensu, est une des choses les plus agréables qui soient. Et puis, il y a quand même une sacrée différence entre manger du bollito misto et boire du Volnay 84 des Hospices en petite tenue, activité d’une noblesse évidente, et se balader les fesses à l’air sur une plage du Cap d’Agde, entouré de viande périmée qui marine dans l’huile solaire, activité d’une désolante vulgarité. Car même si ça vous semble difficile à avaler, sachez que nous étions tous d’une infinie pudeur. Aucun d’entre nous, même en échange d’une forte somme d’argent (jusqu’à un certain point tout de même, c’est bien d’avoir des convictions, des règles de conduite, mais il faut quand même savoir lâcher un peu de lest de temps à autre), n’aurait accepté d’aller se balader les fesses à l’air sur une plage du Cap d’Agde, alors que dîner tous les quatre dans la tenue la plus naturelle qui soit ne nous semblait en aucun cas contraire à nos principes. Il s’agissait, comme cela se pratique dans certes entreprises, de renforcer la cohésion du groupe en le plaçant dans des situations extrêmes, même s’il devenait évident que le fond de la deuxième bouteille de Volnay ne serait en aucun cas suffisant pour nous permettre de mener l’expérience à son terme.

À toute cause il faut un carburant digne de ce nom, et mon choix pour remplacer le défunt Volnay s’était porté sur un Beaune 1er Cru Cuvée Rousseau Deslandes 69 des Hospices, lui aussi prélevé par mes soins dans la cave de Riqueti.

\textsc{Moi} : Mesdemoiselles, s’il vous plaît, ne bougez plus !

Tosca et Zarina nous tournaient le dos, prêtes à repartir à la cuisine.

Elles, stoppées net dans leur élan : Qu’est-ce qui se passe ?

\textsc{Moi} : Ne bougez surtout pas, je vous en conjure !

\textsc{Zaahid} : Quoi, qu’est-ce qui se passe ?

\textsc{Moi} : Tu ne remarques rien ?

\textsc{Lui} : Non, quoi ?

\textsc{Moi} : Regarde mieux.

\textsc{Lui} : Désolé, je ne vois rien.

\textsc{Moi} : C’est extraordinaire !

\textsc{Lui} : Quoi, qu’est-ce qui est extraordinaire ?

\textsc{Moi} : Je suppose que tu as remarqué que Tosca a un grain de beauté sur la fesse gauche ?

\textsc{Lui} : Évidemment, pour qui me prends-tu !

\textsc{Moi} : Et maintenant si tu observes attentivement la fesse droite de Zarina, chose que je t’autorise exceptionnellement à faire sans la moindre retenue, qu’est-ce que tu vois ?

\textsc{Lui} : Nom de Dieu !

\textsc{Elles} : Ah, c’est ça !

Que dire, sinon que, sur la fesse droite de Zarina, par un de ces effets de symétrie reproductive qui caractérise l’essentiel des créations de la nature (et, par extension, celles de l’homme dont l’inspiration est la même, n’en doutons pas, même s’il a souvent l’impression de s’affranchir des règles qui le gouvernent à son insu), se trouvait un grain en tout point similaire à celui qui ornait la fesse gauche de sa jumelle !

\textsc{Zarina} : Oui, Tosca a un grain de beauté sur la fesse gauche et moi sur la droite. On peut y aller, maintenant ?

\textsc{Moi} : Je suis sincèrement désolé, mais vous comprendrez que je ne pouvais décemment pas laisser Zaahid dans l’ignorance d’un tel miracle de la nature. Il était de mon devoir de lui ouvrir les yeux, je l’ai fait et je ne regrette rien.

\textsc{Zarina}, soulevant légèrement l’organe en question : J’ai aussi un grain de beauté sous le sein gauche.

\textsc{Tosca}, de la même façon : Et moi sous le sein droit.

\textsc{Zaahid} : Et moi j’en ai un sur la couille gauche, mais je pense que tout le monde s’en fout !

\textsc{Moi} : Non ?

\textsc{Lui} : Si, pourquoi ? Attends, ne me dis pas que…

\textsc{Moi}, titubant d’émotion : Si, j’en ai un sur la couille droite !

\textsc{Zaahid}, jaillissant de sa chaise pour bondir dans mes bras : Embrasse-moi, vieux frère !

\textsc{Moi} : Oui, oui. Bon, va te rassoir, maintenant, il faut que j’aille chercher une bouteille de vin.

\textsc{Lui} : Je viens avec toi !

\textsc{Moi} : Sûrement pas, tu tiens à peine debout. Je te ramène jusqu’à ta chaise.

\textsc{Lui} : Je tiens parfaitement debout, et j’aimerais assez que tu arrêtes de me traiter comme un adolescent boutonneux !

\textsc{Moi}, le forçant à s’assoir : Pose tes fesses là-dessus et tâche de rester tranquille !

\textsc{Lui} : Je suis un adulte parfaitement responsable, une personnalité reconnue du monde de la science !

\textsc{Moi} : Oui, enfin, reconnu surtout par moi, ce qui n’est pas un mince honneur, je te l’accord. Bon allez, presque tranquille pendant que je vais chercher du vin.

\textsc{Lui} : C’est inadmissible !

\textsc{Moi} : Quoi, encore ?

\textsc{Lui} : La façon dont on me traite dans cette maison. Je ne suis quand même pas n’importe qui, merde !

\textsc{Moi} : Mais bien sûr que non. Sans toi, on n’aurait jamais mis la main sur ces enfoirés de Disciples de la Colère.

\textsc{Lui} : Exact ! Voilà pourquoi j’exige non seulement de la considération mais aussi de participer à leur extermination !

\textsc{Moi} : Je te rappelle qu’il y a des lois dans ce pays, et qu’il n’est en principe pas prévu de procéder à leur extermination. Ou alors très peu. Non, le but du jeu est de leur passer les bracelets pour les empêcher de nuire et les traduire devant une cour de justice qui décidera de leur sort.

Tosca et Zarina, de retour de cuisine avec les bras chargés de claquos (fromage à pâte molle et croûte fleurie, typiquement associé au camembert, dans le langage populaire, dixit lalanguefrançaise.com qui n’est pas la moitié d’un repaire de branleurs analphabètes) : Vous êtes vraiment des fachos !

\textsc{Zaahid} : Un de ces bons vieux jurys populaires qui font la grandeur de la démocratie !

\textsc{Moi} : Oui, cette démocratie que le monde entier nous envie. Bon, je vais chercher du vin.

Le temps de :

1. enfiler mon peignoir Aescwig Paige collection printemps-hiver 2017 en laine de soie peignée et microfibre de bambou 100\% bio du Suriname (cadeau de Zaahid qui possédait exactement le même),

2. croiser la mère Ouvrard qui, comme par hasard, était en train d’errer tel un ectoplasme maléfique dans les couloirs à la recherche de cette pourriture rousse de Korax (un de ces trois enfoirés de chats qui se faisaient un malin plaisir de chier et pisser partout, en particulier sur mon paillasson en fibre de coco WELCOME TO THE JUNGLE, triple référence au chef-d’œuvre de Peter Berg avec Dwayne Johnson, alias The Rock, à son remake tout aussi inoubliable de Rob Meltzer avec le grand poète et penseur belge Jean-Claude Van Damme, et bien entendu au film de Jonathan Hensleigh (historien, avocat, scénariste et réalisateur méconnu) qui narre les aventures cannibalocaustiques de quatre sympathiques jeunes gens partis enquêter sur la disparition~-- histoire vraie~-- de Michael Rockefeller, richissime héritier de 23 ans, en Papouasie-Nouvelle-Guinée),

3. échanger quelques rapides paroles avec cette même mère Ouvrard qui n’a pas manqué de s’étonner de me voir en robe de chambre sur la palier,

4. sauter dans l’ascenseur (je rappelle aux moins attentifs d’entre vous que je logeais au sixième et dernier étage d’un immeuble tout confort situé dans les beaux quartiers de la toute proche périphérie urbaine dans ce qu’elle avait de plus humain et progressiste, même si le bâtiment n’était pas de la toute première fraîcheur, chose qui, je parle du fait d’habiter au sixième étage, ne facilite pas vraiment l’approvisionnement quand on se trouve subitement à court de vin et qu’il faut descendre au sous-sol en quatrième vitesse),

5. fouiller un peu et écarquiller des yeux d’enfant émerveillé devant les casiers en fer forgé sur lesquels j’avais empilé les innombrables bouteilles volées à Riqueti (paix à son âme, il n’aurait plus jamais soif, maintenant, et j’avais sauvé d’un avenir incertain le meilleur de son héritage),

6. trouver le Rousseau Deslandes 69, superbe flacon dont chaque grain de poussière et tache de moisi sur l’étiquette avait été religieusement conservés par mes soins, au point que j’osais à peine le prendre en main de peur de l’altérer,

7. méticuleusement refermer à quintuple tour la porte blindée de ma cave adorée qui ressemblait davantage à Fort Knox ou l’ancien bunker du Gothard qu’à un simple lieu de stockage et affinage domestique de produits de bouche et autres denrées de première nécessité,

8. reprendre l’ascenseur en sens inverse,

9. ré-échanger quelques paroles insipides avec cette vieille harpie de mère Ouvrard qui était toujours sur le pied de guerre dans le couloir, la bave aux lèvres, avec son crâne déplumé et son épiderme fripé et parcheminé maculé de taches suspectes, sa moustache et son poil au menton, éternellement chaussée de ses abominables pantoufles à pompon, en robe de chambre élimée de jour comme de nuit, mue par une irrésistible et démoniaque envie de pourrir la vie des gens au point de les pousser à déménager ou se lancer dans une longue et coûteuse psychanalyse, comme c’était le cas de ce pauvre Marc-Antoine Jacquinot, prof de philo au bord du suicide par pendaison dans la cage d’escalier, homme d’une solitude extrême qui faisait quotidiennement les frais de son manque de charisme et d’autorité naturelle (les gens sont cruels),

10. parvenir enfin à échapper à ses griffes et me précipiter chez moi le cœur battant,

et je pouvais de nouveau envisager l’existence non pas comme un long chemin de croix pavé des mauvaises intentions d’une cohorte de nuisibles perpétuellement à l’affût (ne me demandez pas pourquoi, d’où vient cette étrange répulsion, mais il y a des mots sur lesquels je n’ai pas la moindre envie de mettre un accent circonflexe, et affût en fait partie) d’un mauvais coup, mais de longues vacances enchanteresses dans des contrées merveilleuses peuplées de gens éminemment sympathiques et accueillants n’ayant qu’une seule idée en tête : se couper en quatre pour faire votre bonheur.

Pendant mon absence, Tosca et Zarina avaient fait le ménage et déposé au centre de la table un plateau de fromages digne des Mille et Une Nuits, lequel exerçait sur un Zaahid pourtant accoutumé une fascination sans cesse renouvelée, tant il contrastait avec l’indigence fromagère de son Bangladesh natal (même s’il n’y était pas né personnellement, il le considérait comme la terre de ses ancêtres, et, par voie de conséquence, sa terre natale par procuration), réduite à la production plus élémentaire qu’alimentaire de quelques vagues tonnes de fromages de yak (essentiellement pour les chiens, qui sont les seuls à pouvoir le mastiquer), vache maigre et buffle d’eau, plus ou moins frais, quelquefois fumés ou frits, d’une comestibilité douteuse quoi qu’il en soit.

Les deux créatures, qui je le rappelle évoluaient seins nus telles d’innocentes naïades entourées de requins affamés, avaient, en plus des couverts, changé les assiettes, chose qui, je ne vous le cache pas, avait le don de m’exaspérer au plus haut point, et ce d’autant plus qu’il fallait à nouveau les changer, de même que les couverts, pour manger le dessert, terminaison sucrière qui n’était heureusement pas prévue ce soir, que j’estimais redondante, inutilement alourdissante, et retardant d’autant le plaisir ultime et aboutissement dînatoire d’allumer des barreaux de chaise et s’envoyer des digestifs dans une ambiance à mi-chemin entre le gentlemen’s club et le tripot. Je reconnais que bien sûr, de nos jours, les données techniques concernant la vaisselle ont changé du tout au tout. Avant la femme s’y collait avec véhémence, ou les domestiques dans les familles les plus aisées, mais l’exercice était à la fois ingrat et hautement préjudiciable à la douceur des mains, ces mêmes mains qui devaient par la suite servir à des pratiques plus intimes nécessitant, précisément, un épiderme à l’opposé du papier de verre ou du gant de crin. Après la somptueuse Josephine Cochrane et son lave-vaisselle à manivelle en 1886, les frères Walkers (ils s’y sont mis à deux) ont inventé en 1911 le premier lave-vaisselle entièrement automatique, d’abord équipé d’un moteur à gasoil moyennement agréable en espace clos, avant de passer au confort de l’électrique quelques années plus tard. C’est bien évidemment très émouvant, et je ne saurais remettre en question l’utilité de cet appareil, mais je constate qu’aujourd’hui les gens s’en servent à tort et à travers et l’utilisent comme prétexte pour changer les couverts à tout bout de champ, ce qui est éthiquement irresponsable et intellectuellement discutable, même si je n’ignore pas que les tâches ménagères ne sont plus en odeur de sainteté, en admettant qu’elles l’aient jamais été. Il est clair que le fait de pouvoir laver sans effort entraîne ce que j’appellerai une sorte de frénésie dépensière énergétique, d’illusion hygiéniste du luxe comme si on pouvait changer la vaisselle entre chaque bouchée sous prétexte qu’on dispose d’une armée de domestiques pour la faire incessamment. Déjà qu’on ne mange plus avec ses doigts, ce qui me paraît hautement dommageable dans la plupart des cas, je trouve regrettable qu’on ne fasse même plus la vaisselle, même si, je l’admets, j’ai moi-même une sainte horreur de la faire, et plus particulièrement de la rincer et l’essuyer.

Voilà, c’était juste un petit aparté sans grand intérêt ni conséquence, je vous le concède, mais je tenais tout de même à soulever, ne serait-ce qu’un court instant, le sujet de cette épineuse question.

Revenons-en maintenant, si vous le voulez bien, à notre plantureux plateau de fromages.

Il y avait là de la Couronne Lochoise, merveille au lait de chèvre (cru, bien sûr) en provenance de la ferme de La Biche, à Betz-le-Château ; du neufchâtel du pays de Bray, adorable petite chose en forme de cœur fondant ; deux saint-marcellin de toute beauté, débordants d’amour ; un brie de Nangis résolument fermier dans le même état de coulaison superlative, exhalant un fumet d’une puissance à couper le souffle ; et enfin une portion conséquente de l’incomparable tomme de vache de l’abbaye Notre-Dame de Donezan, perchée à près de mille cinq cent mètres dans les Pyrénées ariégeoises.

Tu la sens (je te tutoie, nous sommes tous frères) la douceur de vivre, tu l’entends le petit ruisseau qui coule, fleurant bon le lait cru et l’herbe tendre des vertes prairies hexagonales ?

Vous savez, sur une bouteille de vin de 69, il n’est pas rare que le bouchon, imbibé sur la totalité de sa longueur et rongé par l’acidité, se montre excessivement friable. On ne saurait lui en vouloir, bien sûr, mais cela impose de le manipuler avec des précautions certaines. Pas question de se jeter sur lui et le trousser à la hussarde, l’empaler à grands coups de tire-bouchon comme si on ouvrait une bouteille de beaujolais nouveau (je ne conseille de toute façon à personne de faire, laissons cela à nos amis Japonais qui raffolent de cette infâme piquette, comme d’ailleurs à peu près tout ce qui vient de France, avec une nette prédilection pour les produits les plus ringards et franchouillards, beaujolais nouveau, béret basque, saucisson sec et quiche lorraine, Mireille Mathieu, David Guetta et Joe Dassin, bref tout ce qui leur rappelle de près ou de loin l’odeur de petite culotte de leurs lolitas gothico-victoriennes aux yeux bridés).

Je conseille, dans ces cas-là, d’utiliser un tire-bouchon au pas de vis suffisamment important pour assurer une bonne prise (évitez le Charles-de-Gaulle à ailettes et le tire-bouchon à levier, tout ce qui peut exercer une quelconque forme de torsion sur le bouchon, et veillez à rester bien droit pendant toute la durée de l’opération), de traverser le bouchon de part en part avec ladite vis (avec une extrême délicatesse pour ne pas risquer d’enfoncer l’obturateur dans le goulot), puis, toujours avec la même plus extrême délicatesse, d’extraire l’objet liégeux de son logement millimètre par millimètre, sans jamais le brusquer. Si vous appliquez scrupuleusement cette méthode, vous avez une bonne chance de réussir, sachant que malgré tout le bouchon peut à tout instant se casser en deux ou tomber en miettes. C’est ainsi, on n’y peut rien, mais quand on fait les choses dans les règles de l’art au moins on n’a rien à se reprocher. Selon le millésime, le vin peut concentrer plus ou moins de dépôt dans le fond de la bouteille. Dans le cas d’un dépôt important, qui n’est pas nécessairement mauvais signe, laissez reposer la bouteille à la verticale pendant au moins vingt-quatre heures avant de procéder à l’ouverture, que vous effectuerez dans le même esprit de verticalité et en limitant autant que possible la redispersion des particules dans le breuvage, celles-ci étant généralement chargées d’une amertume tout à fait dispensable à la dégustation.

Sachant qu’il avait affaire à des amateurs de premier choix, et un déboucheur hors pair, rompu à toutes les techniques d’extraction, le bouchon de notre Beaune 69, d’une qualité de liège exceptionnelle, n’a fait aucune difficulté pour quitter le logement qu’il occupait depuis tant d’années. On aurait dit qu’il n’attendait que ça, estimant qu’il avait accompli sa tâche à la perfection et pouvait maintenant rejoindre en toute sérénité le paradis des bouchons, en l’occurrence la boîte dans laquelle je remisais les bouchons les plus mémorables dont j’avais eu le privilège de croiser la route, tous du meilleur liège, et pour la plupart estampillés des millésimes, crus, appellations et châteaux parmi les plus respectables.

Bref, ma joie était intense et sans limite, et celle-ci est encore montée d’un cran quand les premières effluves du vin sont parvenues à mes narines, ouvertes à tous les vents telles des antennes paraboliques pour en capter les plus infimes fréquences.

Souvent, à l’ouverture, un vin de cet âge, qui a passé des décennies à méditer dans le fond d’une cave, tel un moine dans sa cellule, se montre aussi mutique et fermé à double tour que le cul d’une nonne endurcie par la prière et l’abstinence, un trou de balle que seul le doigt de Dieu a eu l’insigne honneur de visiter. Mais cette fois, tel Peter Pan et son Petit Oiseau Blanc (cui-cui), je me suis retrouvé catapulté sans préavis dans les jardins de l’hôpital pour enfants malades de Great Ormond Street, à Londres, peuplé de créatures magiques, d’arbres vivants et de champignons hallucinogènes. J’avais été, moi aussi, cet enfant malade qui pleurait seul dans son coin en attendant la Mort. Celle-ci, sans doute trop occupée par ailleurs, n’était jamais venue, mais elle avait laissé une empreinte indélébile dans le fond de mon cœur, un cœur brisé auquel seule l’absorption d’un authentique philtre d’amour comme le Beaune 1er Cru Cuvée Rousseau Deslandes 69 des Hospices (je pense que c’est de là que venait la référence à Peter Pan et aux enfants malades de Great Ormond Street) pouvait rendre le sourire.

L’heure était venue, pour moi, de passer d’un état de quasi-nudité à quelque chose de plus habillé. Pour servir le vin d’une part, le Rousseau Deslandes 69 (bon, alors, pour information et parce que j’en ai marre qu’on me pose sans cesse la question, qu’on m’abreuve de lettres recommandées et d’injures à caractère raciste sur les réseaux sociaux parce que je n’ai pas tout dit à son sujet, sachez que la cuvée en question, mélange de Cent Vignes, Montrevenots et Clos de la Mignotte, trois premiers crus de Beaune, porte les noms des bienheureux Antoine Rousseau et Barbe Deslandes, sa charmante épouse, fondateurs et généreux donateurs, en 1645, de l’Hôpital de la Sainte-Trinité destiné à accueillir les orphelins de la peste et la Guerre de Trente Ans, rattaché aux Hospices de Beaune sous la Révolution et aujourd’hui dévolu aux personnes âgées) exigeant tout de même une certaine élégance pour être servi avec la déférence due à son rang, et ensuite parce que j’avais rendez-vous dans une heure avec mes potes chasseurs de nazis pour aller nettoyer un nid.

Et moi, quand je nettoie des nids, j’aime assez être physiquement à mon avantage. Certains préfèrent la tenue de camouflage style raid dans la jungle, d’autres l’ensemble tactique façon GIGN avec gilet pare-balles et pantalon cargo, personnellement mon choix se porte plutôt sur le costume trois pièces nœud pap’ genre James Bond, le costume colonial pour les interventions par temps chaud ou encore, même si je la trouve un peu moule-burnes à mon goût, la combinaison de plongée tropicale avec kit anti-requin, fusée de détresse et micropropulseurs intégrés.

Après avoir rempli les verres (du genre aquarium, spécialement conçus pour permettre aux grands vins de Bourgogne et d’ailleurs de barboter en toute liberté et livrer aux narines exigeantes la quintessence de leur palette olfactive) de Tosca et Zarina, j’ai rempli le mien et reposé la bouteille.

\textsc{Zaahid} : Et moi, je n’y ai pas droit ?

\textsc{Moi} : Tu as assez bu comme ça.

\textsc{Zaahid} : C’est une blague !

Il a tenté d’attraper la bouteille, mais j’ai été plus rapide.

\textsc{Lui} : Donne-moi immédiatement cette bouteille !

\textsc{Moi} : Désolé, mais je pense que tu n’es plus en état d’apprécier un tel nectar.

\textsc{Tosca} : On ne va pas donner de la confiture à un cochon.

\textsc{Zaahid}, essayant péniblement de se lever : Donne-moi cette bouteille !

\textsc{Zarina} : Arrêtez, il va faire un malaise !

\textsc{Lui} : Je suis en parfaite santé, et j’exige de goûter ce vin !

\textsc{Tosca} : Tu veux du fromage, mon chéri ?

\textsc{Lui}, furax : Oui, je veux du fromage. Et du vin !

\textsc{Tosca} : Tu veux un peu de ça ?

\textsc{Lui} : C’est quoi ?

\textsc{Elle} : De la Couronne Lochoise.

\textsc{Moi} : Du fromage de chèvre.

\textsc{Lui} : Jamais de chèvre avec le Beaune.

\textsc{Moi} : Ah bon ?

\textsc{Lui} : Avec le bourgogne en général.

\textsc{Tosca}, désignant le brie de Nangis : Alors ça.

\textsc{Lui} : C’est quoi ?

\textsc{Elle} : De la vache.

\textsc{Moi} : Excellent avec le Beaune. Tu veux un peu de vin ?

\textsc{Lui} : Tu te fous de moi ?

\textsc{Moi} : Non. Je te demande juste si tu veux un peu de vin.

\textsc{Lui} : Bien sûr, que j’en veux. Je pense même être le seul expert digne de ce nom autour de cette table !

\textsc{Moi} : Le seul alcoolique, tu veux dire.

\textsc{Lui} : Je ne comprends pas comment on peut boire du bourgogne rouge avec du fromage de chèvre. C’est contraire à tous les principes.

\textsc{Moi}, tapant violemment du poing sur la table, chose qui n’est pas dans mes habitudes tant j’ai toujours prôné la non-violence et la résolution des conflits par le dialogue et l’ouverture d’esprit : Et moi j’emmerde les principes, et je n’autorise personne à me donner des leçons en matière d’accord vin et fromage. Je prétends, non, je ne prétends pas, j’affirme que le Beaune Rousseau Deslandes 69, à égalité peut-être avec le Nuits Les Lavières et Bas De Combe des Hospices du même nom, est le meilleur vin qui existe sur cette putain de terre ravagée par les guerres et le mildiou pour accompagner la Couronne Lochoise, surtout quand elle commence à héberger toute une colonie d’asticots bien vifs et grassouillets, alors que le persillé de Tignes, par exemple, sera plus à l’aise avec un Pommard cuvée Suzanne Chaudron ou même, pourquoi pas, un Corton Charlemagne cuvée François de Salins dans la force de l’âge.

\textsc{Lui} : N’importe quoi ! Du Corton Charlemagne sur du persillé de Tignes ! Pourquoi pas du Chambolle Musigny avec de la chevrette de Novel ou du bleu de Termignon, tant qu’on y est ! Non, si c’est pour entendre ça, je préfère encore aller me coucher sans manger.

\textsc{Moi} : C’est très bon, le Charlemagne avec le persillé.

\textsc{Lui} : Le jambon persillé, à la rigueur.

Et puis soudain, sans crier gare, alors qu’il avait le visage profondément enfoncé dans son aquarium de Rousseau Deslandes et semblait vivre une expérience fascinante de plongée viticole au cœur même de la quintessence du grain, une grosse larme a roulé le long de sa joue avant de tomber dans le liquide, sans doute pour lui apporter une petite touche de salinité supplémentaire.

\textsc{Moi} : Allons bon, qu’est-ce qui se passe, encore ?

\textsc{Lui}, reniflant comme un gamin de cinq ans qui vient de se prendre la fessée de sa vie (pratique aujourd’hui interdite, car on s’est rendu compte que ça pouvait faire naître des idées bizarres dans l’esprit des jeunes enfants, voire déclencher des vocations sado-masos préjudiciables au bon déroulement de leur future vie sexuelle) : Rien.

Tosca l’a pris dans ses bras et lui a caressé légèrement les boules pour tenter de le détendre. Cette technique, bien connue des chamanes bouriates de Mongolie intérieure, consiste à masser doucement les testicules du patient en grande détresse psychologique, généralement avec les mains enduites de lait de renne mélangé à des excréments d’ours polaire, le tout accompagné de chants traditionnels rythmés ou non par des tambours. Tosca, qui ne connaissait pas de chants traditionnels bouriates, les remplaçait avantageusement par des airs du grand poète Livournais Piero Ciampi.

\textsc{Moi} : Ne me raconte pas de conneries, je vois bien que tu fais une gueule de cent pieds de long.

\textsc{Zarina} : C’est l’alcool.

\textsc{Moi} : Sans doute, mais pas que.

\textsc{Tosca} : Il ne tient pas l’alcool. Hein, mon chéri, que tu ne tiens pas l’alcool ?

\textsc{Lui} : Je tiens très bien l’alcool, merci. Aussi bien que n’importe lequel d’entre vous !

\textsc{Moi} : Il y a des gens que ça rend gais, toi ça te rend triste comme un vieux pot de chambre abandonné avec plein de merde au fond.

\textsc{Lui} : Je suis pas triste. C’est juste que… que…

\textsc{Moi} : Que quoi ? Tu peux arrêter de chanter une seconde, Tosca ?

\textsc{Elle} : Non.

\textsc{Moi} : Comment ça, non ?

\textsc{Elle} : Non, je ne peux pas arrêter de chanter. Zaza (c’était le surnom ridicule qu’elle avait donné à Zaahid) ne va pas bien, et je sais mieux que personne ce qui peut l’aider à aller mieux.

\textsc{Zarina} : Je suis d’accord avec elle.

\textsc{Moi} : T’es toujours d’accord avec elle.

\textsc{Zarina} : C’est ma sœur et on est toujours d’accord sur tout.

\textsc{Moi} : N’empêche que si elle pouvait arrêter de couiner et lui masser les couilles pendant que je lui parle, ça pourrait peut-être nous permettre d’avancer un peu dans le débat.

\textsc{Zaahid} : C’est juste que je pensais à Jaya, voilà ! Pas la peine de s’engueuler pour si peu.

Dès qu’il en avait un coup dans le nez, le sujet de Jaya, sa fille adorée, revenait sur le tapis. Et pour cause, il ne la voyait plus. Elle l’avait désavoué, rejeté, le tenant pour responsable de la mort de sa mère, Faustina Barreira, dont elle avait retrouvé le corps sans vie en rentrant du collège. Expérience ô combien traumatisante pour une ado de douze ans, qui avait coïncidé avec le début d’une interminable descente aux enfers, un spectacle d’une désolation telle que même les plus charitables avaient fini par détourner pudiquement les yeux et prendre discrètement la tangente. Après un parcours exemplaire dans l’autodestruction et le cumul frénétique de tout ce qu’il ne faut pas faire si on veut avoir une chance de réussir dans l’existence, elle avait croisé la route de Simon Keskula, le fondateur de l’Alliance de la Révélation, une communauté survivaliste installée au pied de la Montagne de Lure, dans les Alpes-de-Haute-Provence, dont la spécialité était précisément de recueillir les âmes perdues pour les remettre dans le droit chemin et redonner un sens à leur vie. Le sens en question passait bien évidemment par la conscience aiguë du fait que les problèmes ne viennent jamais de vous, innocente créature qui n’avez rien demandé à personne et ne demandez qu’à vivre en paix avec elle-même et les autres, mais des autres, et tout particulièrement d’une bande de politiciens véreux qui vous imposent des règles absurdes dans le seul but de se remplir les poches à vos dépens. Et si, par malheur, vous vous retrouvez dans une situation telle que vous ne leur êtes plus d’aucune utilité, ils s’empressent de vous laisser tomber et vous désigner aux yeux de la société comme un résidu de fausse couche indigne de toute espèce de considération (sachant qu’un con, sidéré ou pas, reste un con quoi qu’il arrive), un déchet de l’humanité tout juste bon à crever la gueule ouverte sans que personne ne lève le petit doigt pour lui venir en aide (ou alors si, l’abbé Pierre, mais si c’est pour se retrouver avec le petit doigt où je pense, je ne suis pas certain que le jeu en vaille la chandelle).

Aujourd’hui âgée de vingt-deux ans, enfermée à double tour dans la communauté de Keskula (frère Simon, comme l’appelaient ses disciples avec des yeux remplis d’un amour et d’une reconnaissance éternelle qui faisaient plaisir à voir, et aussi un peu froid dans le dos), Jaya avait coupé tout lien non seulement avec son père, qui ne parvenait toujours pas à recoller les mille et un morceaux de son cœur éparpillés sur le froid carrelage de l’existence, mais aussi avec tous les gens qu’elle avait pu connaître de près ou de loin par le passé. Keskula prônait une vie en autarcie, dans le genre amish ou mennonite, totalement autosuffisante et sans contact avec le monde extérieur, considéré comme éminemment malsain et corrupteur, source de tous les maux et fossoyeur d’une humanité moribonde au bord de l’extinction. À l’Alliance de la Révélation, on cultivait son potager, on élevait ses poules, on gobait ses œufs, on se laissait pousser les cheveux, on ne se rasait plus les poils nulle part, on se baladait les fesses à l’air et on faisait caca dans les toilettes sèches prévues à cet effet. C’était comme une giclée de fraîcheur (même si ça ne sentait pas toujours le muguet) dans le gosier malodorant de la civilisation, un doigt dans le cul du consumérisme, un bras d’honneur à la science et la technologie. Pour ses disciples, encore assez peu nombreux mais d’une dévotion sans faille, proche du fanatisme religieux, frère Simon était un visionnaire, un nouveau Noé envoyé sur Terre pour construire une nouvelle arche et guider les heureux élus vers un monde meilleur, à l’image d’un Elon Musk qui construit des arches supersoniques pour guider les heureux élus milliardaires vers la planète rouge, terre promise des survivants de la guerre atomique qui se profile à l’horizon. La déflagration sera totale et tous les pauvres seront exterminés comme des cafards, rayés de la carte et condamnés à servir d’engrais radioactif pour les mutants nécrophages qui vivent dans les galeries souterraines et se nourrissent de déchets organiques. Mais ceci n’est qu’une théorie, bien sûr, car il n’existe à ce jour aucune preuve tangible de l’existence de ces créatures, créatures dont la forme exacte (quelque part entre le ver de terre, la sangsue et le dragon de Komodo, charognard et cannibale de sinistre réputation), reste purement spéculative, même si quelques traces suspectes semblent avoir été aperçues ici et là dans certaines zones parmi les plus désertiques et inhospitalières de la planète. Comme Musk, Keskula incarnait une nouvelle forme de messie 2.0 à équidistance entre le national socialisme, la scolastique médiévale et le néo-darwinisme cyberpunk, survivaliste et homophobe de l’incroyable HOC, l’Homme à l’Oreille Coupée, surnommé, en raison de la couleur pour le moins inhabituelle de son épiderme et la toxicité de sa personne, l’Agent Orange par Busta Rhymes, rappeur East Coast sujet à l’embonpoint, et même, au choix, la Limace ou l’Anus Orange par Rosie O’Donnell, actrice elle-même traitée de «~grosse plouc moche~» par l’intéressé, piètre ornithologue mais grand amateur de noms d’oiseaux. Selon eux (et il m’arrivait parfois, quand le long manteau noir du découragement s’étendait tel un linceul sur le peu d’optimisme qui restait en moi, d’être tenté de souscrire à leurs thèses frelatées), il fallait se rendre à l’évidence : les bons sentiments ne menaient nulle part et seul le pragmatisme le plus échevelé pourrait permettre à l’espèce humaine de survivre à sa mort annoncée depuis des lustres par des gens ayant le pouvoir, sinon de lire l’avenir, au moins d’en évoquer les contours sinueux à travers d’obscures métaphores semblables à des carreaux poussiéreux ne laissant transparaître qu’une infime partie de la lumière du jour. Ainsi, le mode de vie prôné par frère Simon privilégiait la reconnexion au monde et la fusion dans un espace-temps autorisant les projections les plus folles, optimistes et ambitieuses. Un avenir serein se dessinait sous les yeux ébahis des fidèles, représenté au cours des ateliers d’expression plastique dirigés par le Maître sous la forme de mandalas fortement inspirés par les kalams anthropomorphes en poudre de riz du Kérala.

Vous allez me dire : tout cela est ridicule, et je vous répondrai : oui, sans aucun doute, mais le respect des croyances n’est-elle pas la garantie d’une vie harmonieuse en société.

Aussi, ami lecteur dont je salue une fois de plus la résilience et la fidélité, il se peut fort bien que ton cerveau de qualité supérieure (même si vraisemblablement pollué aux pesticides) ait le plus grand mal à digérer favorablement le flot d’informations que je viens de déverser dans tes synapses surexcitées. Par contre, ce que tu pourras comprendre sans difficulté, c’est que mon ami Zaahid, en dépit d’une sensibilité élevée, une appétence toute particulière pour la spiritualité et une tendance naturelle à la bienveillance, ne pouvait se résoudre à voir la chair de sa chair s’évaporer dans la nature, aussi idyllique soit-elle.

Raison pour laquelle je lui ai dit, tandis que son gros nez plein de morve s’épanchait joyeusement dans le liquide prestigieux que j’avais commis l’erreur de lui servir (je me consolais en me disant que je ne l’avais pas payé cher, même si, en y réfléchissant, j’aurais aussi bien pu le payer de ma vie) : T’inquiète, je vais aller te la chercher, ta fille.

\textsc{Lui}, des sanglots dans sa voix qui n’était plus qu’une éponge imbibée de sonorités humides et incertaines : Mais elle ne veut plus me voir !

\textsc{Moi} : La première chose à faire est de la sortir des griffes de ce taré. J’ai fait ma petite enquête, Greg aussi, et il est arrivé aux mêmes conclusions que moi, à savoir que Keskula n’est pas à proprement parler un perdreau ou plutôt un vautour de l’année. Il partage de nombreux points communs avec Matthias Schuster, le curé de Montaulogne, ou encore Charles Manson, même s’il n’envoie pas encore ses disciples défoncés massacrer des stars de cinoche dans leurs villas de luxe de Benedict Canyon. Pas encore, en tout cas. Il faut dire que le Ravin des Avelines, le trou à rats où ils ont élu domicile, n’a pas grand-chose à voir avec Beverly Hills. Par contre, ça ressemble un peu au Barker Ranch de Manson, dans la vallée de la Mort. Comme lui, et Schuster le faux pasteur, il est né d’une mère alcoolique et prostituée, avec un père absent qu’il n’a fait qu’entrevoir au gré de ses rares visites au domicile conjugal, bref un grand classique de la misère humaine, fleuron de la tragédie populaire qui fait pleurer dans les chaumières et émeut même les cœurs les plus endurcis. Quand sa pute de mère s’est retrouvée en taule pour avoir assassiné de cent cinquante coups de couteau, déchaînement de violence, qu’ils ont dit, un client un peu plus entreprenant que les autres, il a été ballotté de famille d’accueil en famille d’accueil, toutes plus pourries les unes que les autres, où il a subi son lot de violences morales, physiques et sexuelles, de quoi bien lui faire comprendre que la vie n’était sans doute pas le camp de vacances qu’on lui avait vendu à la naissance. Du coup grosse déception, tristesse, dépression, et surtout forte tendance, comme il est d’usage dans ce cas de figure, à collectionner les actes de délinquance, allant du vol à l’étalage au viol en réunion, avant de réaliser, au détour d’un séjour en prison pendant lequel il a eu tout le loisir de méditer sur ses péchés et faire le point sur sa carrière professionnelle, qu’il était la réincarnation de Jésus et que sa mission, si toutefois il l’acceptait, était de fonder sa propre église 
% LTeX: language=off
e que s'apelerio
% LTeX: language=fr
l’Alliance de la Révélation, havre de paix pour les âmes en peine et terreau de renaissance glorieuse d’une civilisation qui courait à toutes jambes à sa perte. Une bien belle histoire, n’est-ce pas, pleine de fraîcheur et d’eau de source, parfaite carte postale qui cache, j’en ai bien peur, une tout autre réalité.

\textsc{Zaahid}, qui n’en démordait pas, au point qu’il en devenait parfois assez pénible à supporter : N’empêche que Jaya ne veut plus entendre parler de moi.

\textsc{Moi} : Chaque chose en son temps. D’abord on la sort des griffes de Keskula, après on l’envoie en cure de désintox avec suivi psychologique et tout le tintouin. Ça prendra le temps que ça prendra, mais elle finira par te revenir, et ensemble, main dans la main, vous pourrez faire le deuil de Faustina.

\textsc{Lui} : Si seulement !

\textsc{Moi} : Pas question qu’on laisse ta fille pourrir dans ce cloaque ! T’as vu Expendables, avec Sylvester Stallone, Jason Statham, Mickey Rourke, Dolph Lundgren et Jet Li ?

\textsc{Lui} : Non. Excuse-moi, mais j’ai autre chose à faire que regarder ce genre de conneries.

\textsc{Moi} : Ben t’aurais dû, au lieu de faire ton gros intello qui se la pète, parce qu’il se trouve que Titus, Greg, Sam, Maël et moi-même, on forme une équipe tout à fait comparable, un peu de muscle en moins mais beaucoup de cervelle en plus. S’il faut extraire Jaya manu militari de ce repaire de troglodytes d’extrême-droite, on n’hésitera pas à mouiller la chemise et je te garantis que tu reverras ta fille en un seul et unique morceau.

\textsc{Zarina} : Et c’est qui, Stallone ?

\textsc{Moi} : Stallone, ma petite chérie d’amour que j’aime plus que le chianti, le jambon de Parme, Mozart et la mozzarella réunis, c'est-à-dire globalement plus que tout au monde y compris moi-même qui ne suis qu’un pâle trublion de l’existence, un saltimbanque sans consistance, c’est un petit gars d’origine italo-bretonne qui a mijoté dans les casseroles de la Cuisine de l’Enfer, à Manhattan, passablement handicapé après une naissance au forceps, cette pince à bébé qui avait un peu trop tendance à le réduire en bouillie ou achever la mère une fois sur deux. Mais c’est d’abord et avant tout Rocky Balboa, alias l’Étalon italien, perso je dirais plutôt l’Étalon sur les talons (NDLR : bide total, le jeu de mots n’ayant été capté par personne sauf Zaahid, lequel, déjà difficile à décoincer en temps normal, s’est contenté d’un vague ricanement teinté de condescendance affligée), boxeur amateur dur au mal, con comme un balai et à peu près aussi séduisant qu’un rat crevé dans le caniveau, qui tente de gravir un par un et sur les genoux les échelons de la gloire et conjurer le sort qui s’acharne sur lui.

\textsc{Elle} : Je ne suis pas une spécialiste du cinéma comme toi et encore moins des films d’action, mais je connais Stallone, merci. Non, ce que je te demandais, c’est qui joue le rôle de Stallone dans votre petite équipe ?

\textsc{Moi} : Je sais pas, mais en tout cas c’est pas moi.

\textsc{Zaahid}, jamais à court de méchanceté : Je dirais le plus con et le plus costaud de la bande.

\textsc{Zarina} : Titus ?

\textsc{Moi}, révolté : Stallone n’est pas noir ! Et puis il est costaud, c’est vrai, mais loin d’être con. Je me demande s’il n’y a pas un petit fond de racisme refoulé, chez vous.

\textsc{Tosca} : Il plaisantait. Hein, Zaza, que tu plaisantais ?

\textsc{Zaza}, du fromage plein la bouche : Bien sûr, que je plaisantais. J’adore ce garçon.

\textsc{Moi}, d’une voix légèrement ébréchée par l’émotion que le seul fait d’évoquer de nom de mon fidèle lieutenant suscitait en moi : C’est le type le plus droit et honnête que je connaisse. Tu peux l’appeler au milieu de la nuit, il bondira de son lit pour voler à ton secours. Il a un sens de l’amitié qui défie l’entendement, et ça, pour moi, c’est une des choses les plus précieuses qui soient.

\textsc{Zaahid} : On se demande pourquoi.

\textsc{Moi} : Pourquoi quoi ?

\textsc{Lui} : Pourquoi c’est une des choses les plus précieuses qui soient.

\textsc{Moi} : Tu ne trouves pas que c’est agréable et rassurant de pouvoir compter sur quelqu’un en toute circonstance ?

\textsc{Lui} : Si, c’est agréable et rassurant. Enfin, surtout pour toi. Parce que pour lui, je ne sais pas si c’est agréable et rassurant de savoir que tu peux l’appeler à tout moment pour lui demander de faire des choses pas forcément très agréables ni rassurantes. Tu vois ce que je veux dire ?

\textsc{Moi}, serrant les poings en signe de victoire : Mes amis, mes amours, mes emmerdes, pour citer le grand poète franco-arménien Charles Aznavourian, je crois que notre Zaahid va mieux ! Quand il commence à couper les cheveux en quatre pour chercher la petite bête, c’est que la vie reprend ses droits et qu’on ne va pas tarder à le retrouver au meilleur de sa forme.

\textsc{Lui}, définitivement bougon : Non, je ne vais pas mieux, je ne vais pas bien et je me sens horriblement mal. J’en ai marre de tout, de Stallone, des Gardiens de la Révolution, du fromage et du reste ! Sans Jaya, ma vie n’a plus aucun sens.

\textsc{Moi} : C’est pas les Gardiens de la Révolution, c’est l’Alliance de la Révélation. Keskula est une ordure, je te l’accorde, mais je le préfère encore à Salami.

\textsc{Tosca} : AU salami.

\textsc{Moi} : Non, à Salami. Hussein Salami, le chef des Gardiens de la Révolution.

\textsc{Lui} : Jambon, salami, pâté de foie, tête de veau, Gardiens de la Révélation, Alliance de la Révolution, gardiens de mes couilles, alliance de mes deux, qu’est-ce que j’en ai à foutre ! Tout ce que je veux, c’est que ma fille revienne à la maison !

\textsc{Moi} : Et elle va revenir, je t’en donne ma parole. Mais d’abord, si ça tu n’y vois pas d’inconvénient, je vais aller rendre une petite visite aux Disciples de la Colère et leur faire passer l’envie de défiler au pas de l’oie. Le Stechschritt comme disent nos amis allemands, dans cette langue si douce et mélodieuse qui est la leur. Je me demande s’ils vont arrêter de nous faire chier un jour, ceux-là. Quand je pense qu’il n’y a même pas cent ans ils étaient encore en train de gazer des juifs par millions, et qu’ils ont déjà le culot de nous ressortir un putain de parti nazi en bonne et due forme, comme si de rien n’était, en plus de leurs bagnoles hors de prix et leur bouffe de merde ! Sans parler des Ritals, des Hongrois, des Belges qui suivent le même chemin, comme un seul homme, tous unis dans la connerie et marchant d’un pas décidé vers le néant. La France tente de résister, tant bien que mal, au souffle nationaliste qui fait tanguer l’Europe, mais la gangrène s’installe et la mort fera son œuvre si on ne se décide pas à trancher dans le vif. L’Homme est un cancre, le dernier de la classe. Non seulement il n’apprend rien, mais il prend un malin plaisir à répéter ses erreurs, encore et encore, à faire ses gammes sur le piano de l’horreur, écrire la symphonie du chaos, le concerto pour le pied gauche, celui qui porte bonheur quand on marche dans la merde. On nous bassine avec les images de camps de concentration, les SS qui font sauter des cervelles pharisiennes en rigolant, cadavres ambulants en pyjama rayé, spectres buréniens (relatif à Buren, l’artiste conceptuel, et accessoirement anagramme de rubénien, relatif à Rubens, auteur des Trois Grâces, souvent parodié en Trois Grosses, ou Trois Grasses, qui ne donnent pas forcément une image très flatteuse de la femme du XVIIe siècle) aux yeux hagards, entassés comme des carcasses de bestiaux dans des charniers à ciel ouvert, montagne de cheveux et de dents en or, le trésor de guerre de la glorieuse armée allemande, le devoir de mémoire par-ci, plus jamais ça par-là, et dans le même temps les mêmes relents nauséabonds reviennent au pas de charge empuantir l’atmosphère, avec la complaisance et la complicité active de tous les profiteurs de la Cinquième République, cette bande de tiques accrochées au cul du Pouvoir qui se pavanent sur les plateaux de télévision et jouissent impunément d’une telle pléthore d’avantages et privilèges que le seul fait d’en parler me donne envie de gerber ! Mais bon, je m’emporte, je m’emporte, alors que je ferais mieux de rester focus sur l’objet de la mission. Donc, voilà ce que je vous propose : on va tranquillement se siffler un ballon de fine au salon, musique douce, lumière tamisée et plus si affinités, fumer un bon cigare en parlant de tout et n’importe quoi, après quoi je prends mes cliques et mes claques et vais faire ce que j’ai à faire, même si je vous avouerai franchement que j’aimerais autant rester ici avec vous.

Un jour, ma chère et tendre Zarina, splendeur maléfique qui avait incarcéré mon cœur dans la prison de l’amour, avait décidé comme ça, l’air de rien, que mon salon était indigne de sa présence. Pas spécialement la déco, réduite à quelques toiles de maîtres volées ici et là, ni la pièce en elle-même, d’une surface certes modeste mais amplement suffisante dès lors qu’il s’agissait de passer l’aspi et qu’on n’avait pas les moyens de s’offrir une femme de ménage (fonction avec laquelle, comme nombre de mes semblables qui n’avaient pas été prévenus que les femmes étaient des hommes comme les autres et inversement, je n’avais strictement aucune affinité, mais que Zarina, jusqu’à présent et mon petit doigt me murmurait dans le creux de l’oreille que ça n’allait pas durer éternellement, assumait avec une bonne humeur relative, sachant qu’elle n’avait de toute façon pas le choix si elle ne voulait pas cohabiter avec des tonnes de poussière abritant des légions d’acariens tous plus moches les uns que les autres), mais plutôt le mobilier dans son ensemble, et tout spécialement les fauteuils et le canapé censés permettre aux invités de poser agréablement leurs fesses. Certes, je les avais depuis des lustres et le risque de passer à travers devenait chaque jour plus menaçant, telle une araignée grossissant à vue d’œil jusqu’à atteindre des dimensions éléphantesques, mais j’avoue que le célibataire endurci que j’étais et pensais rester (jusqu’à la date fatidique de notre rencontre, raz de marée qui avait totalement bouleversé le paysage de mon existence) n’y prêtait pas la moindre attention. Depuis qu’elle s’était installée chez moi, elle avait bien évidemment entrepris de refaire l’appartement à son goût, sans se soucier le moins du monde de mon opinion, au motif que les femmes qui, je le rappelle, sont des hommes comme les autres, sont néanmoins bien plus compétentes en la matière, et au point qu’il m’arrivait parfois de penser m’être gouré d’appartement en poussant la porte de ce qui avait jadis été chez moi. J’avais reconnu (il aurait fallu être complètement con et de mauvaise foi pour prétendre le contraire) qu’ils n’étaient pas de la toute première fraîcheur, faisant valoir dans le même temps qu’il n’était cependant pas dans mes intentions de les changer, pour des raisons diverses, plus ou moins complexes et avant tout financières, argumentaire qui avait été jugé irrecevable par mon interlocutrice, laquelle avait immédiatement décidé de prendre l’ensemble des transformations à sa charge, ses moyens le lui permettant aisément d’une part, le plaisir de le faire n’ayant pas de prix de l’autre. Difficile de refuser dans ces conditions atmosphériques.

Son choix s’était porté, et le mien aussi je ne vous le cache pas, sur un ensemble chesterfield de haute lignée, fabriqué à partir des meilleurs cuirs, d’occasion mais en parfait état, vintage comme on dit, issu d’un club de gentlemen anglais qui avait récemment mis la clé sous la porte. Autrement dit, fauteuils et canapé avaient été patinés par des générations de culs prestigieux, de sorte qu’on ne pouvait pas ne pas se sentir investi du poids écrasant de l’Histoire quand on avait le privilège de s’y enfoncer jusqu’au nombril.

C’est dans ces monuments historiques que nous avons pris place, Tosca et Zarina dans le canapé, Zaahid et moi-même dans les deux somptueux fauteuils qui se trouvaient de chaque côté.

J’ai sorti quatre énormes verres ventrus et courts sur patte, une bouteille de vieille fine Napoléon des années 50 (provenance Riqueti) et une boîte de Fuente Hemingway Work Of Art, considéré par nombre d’amateurs comme un des meilleurs de la gamme (qui comprend aussi les vitoles Short Story, Best Seller, Masterpiece, Between the Lines et Untold Story) crée par la maison Fuente (sise comme chacun sait à Santiago de los Caballeros, en République dominicaine) en hommage à Ernest Hemingway, écrivain bipolaire, paranoïaque et alcoolique, grand amateur de cigares et de chats polydactyles (il en avait près d’une centaine dans sa propriété de Key West, en Floride, une bien étrange passion dont on a toujours le plus grand mal à percevoir la finalité initiale).

La plupart des femmes détestent le cigare, et il faut bien reconnaître que son odeur, comme celle de l’époisses ou du munster, ces deux fromages à pâte molle bien connu des tables françaises, ou encore celle de la merde ou l’andouillette, et des abats en général (pour lesquels je n’ai jamais fait mystère de ma passion dévorante), n’est pas toujours des plus agréables, ce qui bien évidemment n’altère en aucune façon, bien au contraire serais-je tenté de dire, la valeur gustative de l’objet du délit. De nombreuses personnes y sont sensibles (y compris des hommes, soyons honnête), plus ou moins, mais parmi celles-ci, tout en haut du podium, se trouve cette créature à la fois mythique et bien réelle qu’on appelle une femme, pour qui toute odeur suspecte représente une agression caractérisée passible des pires représailles, à commencer par un tirage de gueule en bonne et due forme pouvant s’étaler sur plusieurs semaines (je conseille, pour rentrer en grâce, une attitude de soumission absolue accompagnée d’offrandes somptueuses humblement déposées à ses pieds). Je suppose que le fait de s’asperger de parfum et se tartiner de cosmétiques à longueur de journée n’est pas pour rien dans cette hypersensibilité olfactive, raison pour laquelle nous autres hommes, si nous voulons un jour (ou peut-être une nuit) pouvoir lui souffler l’épaisse fumée de nos cigares dans les naseaux sans l’entendre hurler à la mort et nous vouer aux gémonies, ferions sans doute bien de lui rappeler que nous n’aimons rien tant que la nature dans son expression la plus pure et dépourvue d’artifice. Non, nous n’avons pas besoin qu’elle se barbouille le visage de couleurs criardes pour la trouver belle, ni qu’elle s’impose diverses tortures physiques pour continuer à susciter notre intérêt. Nous l’aimons telle qu’elle est, jeune de préférence, c’est vrai, plutôt que molle et fripée, imago dépliant ses ailes au sortir de la chrysalide plutôt que vieille larve desséchée, mais sommes tout disposés à contempler le spectacle de sa décrépitude avec d’autant plus de bienveillance que nous conservons précieusement le souvenir de sa splendeur passée.

Par chance mon adorée, en plus des innombrables qualités qu’elle ne cessait de révéler avec une régularité constante qui ne tarissait guère de m’émerveiller, était grande amatrice de cigares, en particulier cubains, qu’elle (ainsi que sa sœur, qui ne crachait pas sur un barreau de chaise de temps à autre) avait commencé à fumer dès son plus jeune âge sous l’impulsion de son père, le très charismatique comte Stefano Fragale Di Brizzi, hélas tragiquement décédé trois ans plus tôt après avoir percuté un sanglier sur la route de Scarperia, non loin de Florence. Le comte, pourtant fin pilote, n’avait rien pu faire pour l’éviter. Sa 250 GT California Spyder, somptueux cabriolet estimé à plusieurs millions d’euros, avait été pulvérisée sur le coup, ce qui représentait quand même une grosse perte pour l’histoire de la carrosserie automobile et aussi pour les collectionneurs de Ferrari qui n’auraient pas hésité à s’entretuer à grands coups de club de golf, épingle à cravate ou pic à glace pour l’acquérir, sachant qu’il n’en restait guère plus d’une cinquantaine dans le monde. Le comte, qui comptait de nombreux amis parmi les moines et les ecclésiastiques en général, venait d’effectuer une rapide retraite au fond des bois de chênes chevelus, bien au frais dans le couvent de Bosco ai Frati, dans le but avoué de se recentrer, reconnecter ses chakras avec les forces cosmiques et énergies positives de l’univers, retrouver le chemin d’une foi qu’il avait parfois tendance à égarer dans les excès en tous genres, à commencer par l’alcool, les femmes et les mondanités, et accessoirement, ainsi que les réflexes acquis au long de sa longue et brillante carrière d’antiquaire lui commandaient instamment de le faire, tenter de s’approprier par tous les moyens, y compris les plus vils, certains tableaux et objets d’art présent dans l’enceinte de l’établissement. Ce soir-là, le comte avait méchamment fêté son départ avec ses potes moines qui n’étaient pas les derniers à rigoler, en dehors des heures de service bien entendu, et on peut avancer qu’il était sérieusement éméché au moment de prendre le volant, fait dont il était malheureusement coutumier. Le sanglier, lui, est un animal qui ne s’embarrasse pas de formules de politesse. Il fonce tête baissée, et malheur à celui qui à l’imprudence ou la malchance de se trouver sur son chemin. Sauf que le sanglier, il a beau être très très costaud, quand il se mange une 250 GT lancée à vive allure, avec au volant un type en état d’ébriété qui mâchouille un havane et écoute Lucia di Lammermoor de Donizetti à fond sur son lecteur CD dernier cri (pas d’origine, c’est vrai, mais en 1961 il n’y avait pas de lecteur CD), eh ben le sanglier, tout costaud qu’il est, il explose comme un fruit pourri et n’a plus qu’à aller ramasser ses défenses sur le bitume. Les secours ont retrouvé le sanglier les quatre fers en l’air sur le bord de la chaussée, la Ferrari en vrac dans le fossé, et le comte en vrac dans la Ferrari. Ils ont mis des heures à l’extraire de la tôle froissée, rassembler tous les morceaux et les remettre dans le bon ordre. Le comte ne voyageait pas seul : dans le coffre de sa voiture, ils ont trouvé, outre quelques objets ayant appartenu à des résidents célèbres du couvent comme Jean de Pérouse et Giuliano della Cavallina, une petite huile sur bois de très belle facture signée Vittorio Brancaccio, bras droit de Fra Angelico et assurément son élève le plus doué. Sur le marché noir américain, ce genre de pièce pouvait se négocier facilement dans les quarante ou cinquante mille dollars. Les Ricains, dans leurs villas ultramodernes de New York, Los Angeles, Dallas, San Francisco, Houston, Miami, Boston ou Chicago, aiment s’entourer de reliques en provenance de la vieille Europe, cet endroit bizarre que leurs ancêtres ont décidé, quatre siècles plus tôt, de quitter pour venir s’installer sur une terre vierge peuplée de sauvages (qui, soit dit en passant, ne l’entendaient pas du tout de cette oreille, et qu’il a fallu par conséquent méthodiquement massacrer pour arrêter de se faire scalper et planter des flèches dans le cul). Cette nostalgie du vieux continent, somme toute bien compréhensible de la part de gens en quête de leurs racines véritables, autres que la très sainte bible et les 400 millions d’armes à feu présentes sur le territoire (et je parle des civils, pas des militaires lourdement armés), soit au moins une par habitant (même si certains n’en possèdent pas alors que d’autres en ont trois ou quatre, en fonction de la passion qui les anime et du sentiment d’insécurité qui les habite), encourage l’OPA des Américains sur le patrimoine européen. La France et l’Italie, qui jouissent d’une réputation particulière en matière de richesse culturelle, sont des cibles privilégiées, et il se peut fort bien qu’on assiste un jour ou l’autre au spectacle hallucinant de châteaux, manoirs et cathédrales emportés d’une seule pièce par la voie des airs d’un continent à l’autre. On en est encore, pour l’instant, au stade de la reconstruction le plus souvent grotesque, caricaturale et kitschissime des édifices en question, du genre gros gâteau à la crème posé au milieu d’un écrin de verdure aseptisée, mais il faut s’attendre à tout, surtout si les Ricains s’allient avec la Corée du Nord, les Russes, les Indiens (le premier sinistre Modi aimerait bien jouer dans la cour des grands, lui aussi, et commence à s’agacer de voir qu’on ne lui prête pas plus d’attention qu’à un intouchable dans un bidonville de New Delhi) et les Chinois pour nous réduire à néant, raser le Moyen-Orient dans la foulée, faire de la bande de Gaza la plaque tournante du speed, de la cocaïne et des amines sympathicomimétiques, du Japon un centre d’essais nucléaires et du sanctuaire shinto d’Izumo un centre d’amaigrissement et de remise en forme pour oligarques dépressifs, du Canada et du Groenland des stations de ski trois étoiles avec tire-fesses en or et chalets tout confort, et de l’Afrique (enfin débarrassée de ses dictateurs ridicules boudinés dans leurs uniformes de carnaval et coiffés de bérets en peau de léopard) une mine à ciel ouvert, avant bien sûr de s’entretuer joyeusement entre amis jusqu’à ce qu’il n’en reste qu’un (à noter que les Américains, au cas où les choses tourneraient au vinaigre, ont prévu de se replier sur Mars, nouvel eldorado de l’espace et terre promise pour fonder la colonie de l’avenir).


\chapter{Acte 6}

\noindent J’ai tout juste eu le temps de siffler deux verres de fine (une merveille dont je pourrais vous parler des heures durant sans faire la moindre pause, mais malheureusement le temps presse), fumer la moitié de mon cigare, enfiler un sweat à capuche, une paire de baskets et un gilet pare-balles (j’avais finalement renoncé au costume trois pièces et à la chemise à jabot) avant que mon téléphone se mette à sonner furieusement pour attirer mon attention sur le fait suivant : Titus m’attendait en bas, en compagnie d’un certain nombre de personnes mal intentionnées et armées jusqu’aux dents, et le contrat moral passé avec eux stipulait que je devais descendre les rejoindre au plus vite, sachant qu’il me faudrait des heures pour arriver jusqu’à l’ascenseur si je tombais sur la mère Ouvrard et son âme damnée en forme de chat.

C’est donc la mort dans l’âme que je me suis arraché au confort domestique qui m’enlaçait tendrement et me serrait contre lui avec toute la force de ses petits bras potelés, non sans avoir auparavant jeté un coup d’œil à ma Rousselot P06 Ultramatic 720 (un très bel objet, vraiment, la montre du mec qui a réussi dans la vie, tout à fait le genre de gadget qu’il m’aurait fallu économiser pendant plusieurs générations avant de parvenir à m’offrir, pour info il était zéro heure et vingt-sept minutes) et déposé sur les lèvres charnues de l’étoile de mes nuits (lady Zarina Fragale Di Brizzi, joyau de l’aristocratie italienne, par ailleurs excellente cuisinière et pianiste de grand talent qui aurait pu, avec un peu de travail, prétendre à une carrière de concertiste internationale) un long baiser mouillé censé exprimer toute l’affection que je lui portais d’une part (c’était elle qui m’avait fait cadeau de la Rousselot P06, ça crée des liens, et, comme je vous l’ai dit, elle était parée de toutes les qualités et beautés du monde), et d’autre part la crainte que je nourrissais tel un rat affamé planqué dans le fond de mes tripes de ne peut-être jamais revoir son visage angélique, rayon de soleil qui enchantait ma misérable existence, laquelle misérable existence pouvait fort bien s’achever brutalement durant les heures fatidiques qui allaient suivre, fauchée par une décharge de fusil de chasse en pleine poitrine ou une attaque sauvage à la clé de 12 ou la tronçonneuse.

Il m’a fallu un certain temps pour me débarrasser de Zarina qui refusait obstinément de me laisser partir, usant honteusement de toutes les armes en sa possession, charme, chantage et tentative de corruption de la plus extrême bassesse, arguant que les autres étaient bien assez nombreux pour faire le boulot sans moi, qu’elle ne s’en remettrait jamais s’il devait m’arriver quelque chose, que j’aurais sa mort sur la conscience, ce à quoi j’ai répondu qu’en ma qualité de guide suprême, fin stratège et chef d’unité, il était bien évidemment hors de question que je plante mes hommes pour me réfugier dans les jupons d’une femme, aussi belle et désirable soit-elle. Même si ma pauvre mère, paix à ses cendres (elles se trouvaient ici même, chez moi, bien en évidence sur une copie de buffet Louis XV made in Taïwan, à l’intérieur d’un vase canope à tête de faucon, l’effigie de Kébehsénouf, la divinité chargée de veiller sur les intestins du défunt, touchante attention car le voyage vers l’au-delà peut s’avérer plus long que prévu et on n’est jamais à l’abri de choper une bonne chiasse pendant le trajet), était encore de ce monde et m’implorait sur son lit de mort de renoncer à cette folie suicidaire, je devrais détacher l’un après l’autre ses doigts crochus profondément implantés dans mes chairs afin de prendre le large et voguer toutes voiles dehors vers mon rendez-vous avec l’Histoire.

Morceaux choisis de la joute verbale nullissime qui nous a opposés quelques instants (elle n’a pas dépassé ce stade de joute verbale nullissime, même si j’ai senti plusieurs fois des fourmis dans les doigts et rêver de serrer son petit cou jusqu’à le faire craquer comme du bois mort) :

\textsc{Elle} : Reste ici !

\textsc{Moi} : C’est hélas hors de question.

\textsc{Elle} : Reste ici ou tu le paieras très cher !

\textsc{Moi} : Je casserai mon PEL.

\textsc{Elle} : Je te préviens, je ne serais plus là quand tu rentreras. Si tu rentres.

\textsc{Moi} : Je mettrai tout en œuvre pour y parvenir, et je t’en fais le serment ici même, d’une voix ferme dans laquelle il est impossible de déceler la moindre trace de tremblement ni hésitation : je rentrerai, plus fort et couvert de gloire que jamais !

\textsc{Elle} : Je m’en fiche, je ne serai plus là pour voir ça.

\textsc{Moi} : Dans ce cas, je passerai le restant de ma vie seul comme un chien à pleurer jour et nuit ton absence, noyer mon chagrin dans l’alcool.

\textsc{Elle} : Oui, dans la fine Napoélon de 50 ans d’âge.

\textsc{Moi} : On n’est pas obligé de noyer son chagrin dans de la piquette.

\textsc{Elle} : Je te fais confiance pour picoler, mais je doute fort que tu passes le restant de ta vie à pleurer jour et nuit.

\textsc{Moi} : Je ferai une grève de la faim pour t’obliger à revenir.

\textsc{Elle} : Je ne reviendrai pas.

\textsc{Moi} : Tu me laisserais crever de faim ?

Elle, s’agrippant à mon bras avec l’énergie du désespoir (elle aurait mérité un prix de comédie) : Avec la plus grande joie !

Moi (je n’étais pas mal non plus, entre la tragédie grecque et le théâtre de boulevard) : Dans ce cas, je vais me jeter à corps perdu dans la bataille, et tant pis si je vole au devant d’une mort certaine !

\textsc{Elle} : Non, ne fais pas ça !

\textsc{Moi} : Si, mon amour, il le faut, car sans toi je ne vois plus l’intérêt de continuer à respirer.

\textsc{Elle} : Je ne le permettrai pas !

\textsc{Moi} : C’est gentil, merci.

\textsc{Elle} : Quoi, qu’est-ce qui est gentil ?

\textsc{Moi} : De te préoccuper de moi.

\textsc{Elle} : Tu sais que je t’aime.

\textsc{Moi} : Et moi aussi, je t’aime, plus que tout au monde. Mais je suis un homme, un homme d’honneur, et je me dois d’accompagner mes hommes au combat. Il n’en est pas un, parmi eux, qui ne donnerait sa vie pour moi.

\textsc{Elle} : Et toi ?

\textsc{Moi} : Quoi, moi ? Si je donnerais ma vie pour eux ?

\textsc{Elle} : Oui…

\textsc{Moi} : Euh… Oui, enfin, moi c’est différent. Je suis le chef, et un chef se doit de rester en vie quoi qu’il arrive, ne serait-ce que pour témoigner et présenter ses plus sincères condoléances aux veuves et aux orphelins. Je me présenterai tête basse devant mes juges et assumerai courageusement toute la responsabilité du carnage. Je suppose que tu n’aimerais pas épouser un lâche ?

\textsc{Elle} : C’est une demande en mariage ?

\textsc{Moi} : C’est une façon de parler. Cela dit, si je m’en sors, et je dois m’en sortir pour les raisons que je viens de t’expliquer, il n’est pas exclu que j’envisage la question avec un certain intérêt.

\textsc{Elle} : Parce que pour l’instant, ça ne t’intéresse pas.

\textsc{Moi} : Si, bien sûr, mais il se trouve que j’ai d’autres chats à fouetter. Tu sais ce que c’est, on est pris dans les flots tumultueux de l’existence, et dans le feu de l’action on ne sait plus trop où donner de la tête.

\textsc{Elle} : Mais tu vas en rester en vie.

\textsc{Moi} : Je ne te cache pas que c’est assez dans mes intentions. Oui, je vais rester en vie, et aussi longtemps que possible, parce que je veux continuer à te serrer dans mes bras et m’abreuver à la source fraîche de tes lèvres purpurines jusqu’à la fin des temps.

\textsc{Elle} : Donc tu me promets de revenir sain et sauf ?

\textsc{Moi} : Je te le promets sur la tête de ce que j’ai de plus cher au monde, toi en l’occurrence. Et tu me promets que tu seras là pour m’accueillir ?

\textsc{Elle} : Bien sûr, que je serai là. Les traits tirés, rongée par l’inquiétude, mais je serai là, fidèle au poste !

Moi, lui dévorant goulûment les lèvres : Mon amour !

\textsc{Elle} : Allez, fiche le camp, maintenant !

J’ai entrouvert la porte et embrassé le périmètre du regard pour m’assurer que la voie était libre.

Entendons-nous bien : j’étais prêt à affronter les légions de l’enfer, accepter mes super-pouvoirs et intégrer l’équipe des Secret Warriors de Daisy Johnson pour sauver le monde des métamorphes, filer des baffes à cette andouille de Kim Jong-un jusqu’à ce qu’il ait l’empreinte de mes doigts tatouée à tout jamais dans la peau de ses foutues joues de poupon mégalo, siroter à la paille des hectolitres de CAJOHN’S Sauce Extrême Black Mamba (2,5 millions sur l’échelle de Scoville), me coltiner l’intégrale des 25 saisons de l’Inspecteur Derrick, démembrer des T.rex à mains nues pour les beaux yeux d’une actrice de seconde zone (vous aurez deviné que je parle d’Ann Darrow dans King Kong, film qui, au-delà d’une imagerie coloniale bon enfant à la limite du ridicule, explore sans ambiguïté les pulsions zoophiles de son héroïne, à l’instar d’une Charlotte Rampling dans Max mon amour de Nagisa Oshima, réalisateur obsédé par les déviance sexuelles à qui l’on doigt, entre autres productions tout aussi sulfureuses, les Plaisirs de la chair et l’Empire des sens), TOUT MAIS PAS LA MERE OUVRARD, SA ROBE DE CHAMBRE ELIMEE ET SES PANTOUFLES DE LA MORT !!!

Je me suis permis de glisser une oreille inquiète dans l’espace avoisinant, aussi tendue que la peau sur le visage de cette pauvre (mais très riche) Donatella Versace : rien à signaler, j’avais peut-être des chances de m’en sortir sans dommage. La vie est une lutte sans merci où la plus infime fraction de seconde de relâchement peut infléchir le destin dans un sens ou dans l’autre. Pile vous êtes encore en vie, face la mort s’abat sur vous comme une mouche à merde sur un étron fraîchement pondu. La très vieille et quasi momifiée Maria Ouvrard, et avec elle Korax, le démon à quatre pattes qui la suivait comme son ombre, deux entités auxquelles le premier crétin venu n’aurait pas hésité à donner le bon dieu sans confession, pouvaient à tout moment faire basculer votre petite vie paisible de justicier masqué dans un cauchemar existentiel tel qu’il semblait que les Forces du Mal au grand complet avaient conjugué leurs efforts pour le concevoir. Heureusement pour moi, la vieille taupe avalait des cargaisons de médocs, et parmi eux des doses conséquentes de somnifères ultra puissants qui semblaient bien, pour une fois, avoir produit leurs effets. J’implorais quotidiennement le Malin pour qu’il les rappelle à lui, elle et sa pourriture de chat qui passait son temps à mitrailler mon paillasson de déjections malodorantes, mais le Seigneur des Ténèbres, ancien ange déchu je le rappelle (et même ange de qualité supérieure au rayonnement suprême, à tel point que certaines mauvaises langues prétendent que Dieu n’aurait pas hésité à le virer du Paradis parce qu’il lui faisait de l’ombre, hypothèse hideuse et blasphématoire que j’aime autant ne pas envisager), mais le Seigneur des ténèbres, disais-je, n’avait aucune envie de se les coltiner jusqu’à la fin des Temps, raison pour laquelle il restait sourd à mes supplications (en dépit des messes noires que j’organisais deux ou trois fois par semaine dans les sous-sols de l’immeuble et des nourrissons innocents volés dans les maternités que je sacrifiais à tour de bras pour lui être agréable). J’étais même persuadé que le jour où ses deux fléaux casseraient enfin leurs pipes, chose qu’ils feraient probablement de concert, le Diable les renverrait aussitôt sur Terre, à l’endroit précis d’où ils venaient, pour continuer à faire chier le monde jusqu’à ce que l’immeuble soit officiellement déclaré hanté, avant d’être rasé et transformé en espace vert. Et même alors, il y avait de fortes chances pour que cet espace vert, victimes de rumeurs insolites concernant certaines disparitions ou apparitions inquiétantes, se vide peu à peu de ses visiteurs et finisse par être purement et simplement interdit au public, puis laissé à l’abandon jusqu’à devenir un terrain vague réputé pour ses mauvaises ondes et son atmosphère délétère, un repaire de satanistes, zombies toxicomanes et adolescents en mal de sensations fortes.

Bon, je ne crois pas inutile de rappeler que si je n’étais pas rond comme une queue de pelle, peu s’en fallait. À vrai dire, il est souhaitable de s’alcooliser un tantinet quand on se prépare pour une baston qui risque de s’avérer un poil plus saignante que prévu. Pas trop, bien sûr, sinon on risque de viser à côté et se faire descendre sur un malentendu, ce qui serait quand même de la dernière incivilité, mais assez pour se sentir pousser des ailes et courir au casse-pipe le sourire aux lèvres (plutôt que la peur au ventre, laquelle, comme le soulignait une Elisabeth Badinter au sommet de son art, est mauvaise conseillère).

J’ai le plaisir de vous annoncer que je suis arrivé sans encombre sur le trottoir devant chez moi.

Titus m’attendait, s’est autorisé quelques réflexions dispensables sur mon état général, puis il m’a demandé, s’étonnant sans doute de me voir aussi peu chargé pour une mission d’une telle importance (c’était un peu comme se pointer en maillot de bain avec des claquettes aux pieds pour traverser la Sibérie) : T’as pensé à prendre l’artillerie lourde ?

\textsc{Moi} : T’inquiète, j’ai Manu, mon fidèle 6.35.

Il m’a regardé comme s’il me voyait pour la première fois : Tu te fiches de moi, là ?

\textsc{Moi} : Non, pourquoi ?

\textsc{Lui} : Manu ne ferait pas de mal à une mouche.

\textsc{Moi} : Détrompe-toi, il fait des ravages à bout portant.

\textsc{Lui} : Je voudrais pas te vexer, mais le principe d’une arme à feu c’est pas de faire des ravages à bout portant, mais à distance.

\textsc{Moi} : Ouais, n’empêche que t’as peu de chance de rater ta cible à bout portant.

Lui, découragé : Vu comme ça. Bon, c’est pas grave, de toute façon Sam a pris du rab.

\textsc{Moi} : Sam est là ?

\textsc{Lui} : Ouais, il vient de rentrer de vacances.

\textsc{Moi} : À Zanzibar, oui, je sais.

\textsc{Lui} : Oui. La pêche sous-marine, c’est bien, mais ça ne remplace pas la chasse au gros gibier. Il avait besoin de se dégourdir un peu les jambes, retrouver le feu de l’action, l’odeur de la poudre et les montées d’adrénaline.

\textsc{Moi} : Ça ne m’étonne pas de lui. Et il y a qui d’autre, si ce n’est pas indiscret ?

\textsc{Lui} : Tu verras bien.

\textsc{Moi} : Où est ta voiture ?

\textsc{Lui} : C’est pas ma voiture.

Moi (dans le genre mec chiant qui répète tout ce qu’on dit, pose un milliard de fois la même question et ne comprend rien à rien) : T’as pas pris ta voiture ?

\textsc{Lui} : Non, trop petite.

\textsc{Moi} : Trop petit, ton break 307 ?

Lui, d’une patience d’ange : Oui, trop petit.

\textsc{Moi} : Vous auriez pu me le dire, quand même.

\textsc{Lui} : Pourquoi faire ?

\textsc{Moi} : J’aimerais bien qu’on me tienne un peu un courant, c’est tout. Je te rappelle que c’est quand même un peu moi le chef des opérations spéciales. C’est encore loin ?

\textsc{Lui} : Là-bas, au coin de la rue.

On est arrivés là-bas, au coin de la rue, et mon regard a été aussitôt aimanté par une espèce de gros truc noir (version automobile du monolithe de 2001), le genre de monstre à quatre roues motrices blindé jusqu’aux ouïes utilisé par les forces de l’ordre pour trimballer un certain nombre de choses ou personnalités dites sensibles telles que :

sacs de sport remplis de fric piqué aux trafiquants et autres honnêtes pères de famille partis planquer leurs économies en Suisse ;

tonnes de coke promises à la destruction (après quelques modestes retenues pour le petit personnel et plus si affinités, le reliquat étant éventuellement discrètement repositionné sur le marché pour alimenter l’effort de guerre, on ne peut pas se permettre de gaspiller quand la patrie est au bord de l’invasion, et j’ajoute à titre personnel que je ne vois pas ce qu’il y a de mal à dépouiller des méchants qui utilisent leurs revenus illégaux pour financer des activités criminelles hautement toxiques pour l’humanité, en plus de s’acheter des voitures de sport et des montres de luxe par treize à la douzaine);

repentis de la mafia sous très haute protection qui ont accepté de témoigner contre leur ancien employeur avant de changer de nom, se faire refaire le portrait et disparaître sans laisser de traces ;

tueurs en série cannibales équipés de muselières pour les empêcher de bouffer tout le monde ;

dictateurs sud-américains moustachus ou nord-coréens maniaco-dépressifs qui ont plus d’ennemis que de poils sur le corps (ça vaut surtout pour les nord-coréens maniaco-dépressifs);

ces mêmes véhicules pouvant également être utilisés pour : procéder à des interventions sous testostérone dans des zones de non-droit infestées de junkies, dealers, forcenés et repris de justice en cavale ;

défoncer des repaires de terroristes eux-mêmes défoncés à la fénétylline, armés de fusils d’assaut communistes et ceinturés d’explosif artisanal à base de peroxyde d’acétone ;

traverser des champs de mines en essuyant des tirs nourris d’armes de gros calibre (genre lance-roquettes antichar, mitrailleuse lourde et obus de mortier);

s’inscrire aux Cup Series de la National Association for Stock Car Auto Racing pour s’offrir quelques tours de pistes sur le speedway de Daytona ;

et enfin, aussi étrange que cela puisse paraître le cas a déjà été observé ici et là, aux États-Unis notamment, mais aussi dans la banlieue de Moscou, faire ses courses (à vitesse réduite) au supermarché si on n’a vraiment rien d’autre à se mettre sous la main ni de mieux à faire.

Contre toute attente, le véhicule en question n’était autre qu’un van Chevy G30 de 1992 entièrement reconditionné par les bons soins de Nathan, le frère de Greg, lequel Nathan vivait sous l’emprise d’un démon de la mécanique qui le poussait à s’attaquer aux tas de ferraille les plus insignifiants pour les transformer en machines de guerre crachant le feu et inspirant la terreur sur leur passage.

Dans le cas présent, le pacifique G30 n’avait plus rien à voir avec le véhicule sympathique et débonnaire du bon père de famille qui baise (missionnaire, levrette les jours de fête) sa femme tous les week-end, baise ses go…. euh… non… passe du temps avec ses gosses tous les week-end, fait cramer de la barbaque au barbecue avec ses potes bourrés qui rotent, pètent et fument comme des pompiers pyromanes en lâchant des blagues salaces, du jogging en rentrant du boulot pour ne pas ressembler à un gros tas de graisse ambulant, taille les haies, bine les patates et bouture les bégonias, passe l’aspi une fois par an pour se donner bonne conscience, boursicote avec l’argent du ménage, se fait construire (ou la construit lui-même avec des copains bricolos) une piscine même s’il neige 364 jours par an dans le secteur, se paluche comme un voleur sur des sites de boules quand tout le monde dort à poings fermés, vote à droite parce qu’il y en a ras le bol de tous ces assistés qui n’en foutent pas la rame (quand ils ne violent pas nos femmes et ne nous tabassent pas pour nous piquer notre carte bleue), paie ses impôts rubis sur l’ongle, s’insurge contre la rapacité fiscale qui tue les petits et protège les gros, peste contre le laxisme étatique qui favorise l’invasion migratoire de parasites multicolores et la dégradation islamo-gauchiste du tissu républicain, se lave consciencieusement les mains après avoir pissé, et veille à ce que ses enfants n’aient rien à débourser le jour où le moment sera venu de le mettre en terre ou le carboniser.

Fidèle à son habitude, Nathan avait mis le paquet.

En toute modestie, il considérait ce G30 comme un des chefs-d’œuvre de sa carrière de magicien spécialiste de la transformation auto. Selon lui, plutôt que de se retrouver sur la route à rouler bêtement comme n’importe quel tas de boue grossièrement monté sur quatre roues sans intérêt, l’objet, ou plutôt ce qu’il convenait d’appeler la sculpture mobile, entre art conceptuel et abstraction pure, aurait dû être exposé dans un musée, en bonne place au milieu des fleurons de la créativité humaine. Il suffisait, pour se rendre compte du chemin parcouru entre sa sortie d’usine et le stade ultime de perfection qu’elle affichait aujourd’hui, d’écouter la symphonie fantastique émise par son moteur au premier tour de clé. Ce moteur, un V8 de six litres deux d’origine, avait été entièrement remanié et magnifié dans le sens bien évidemment d’une puissance accrue au-delà du raisonnable, autorisant accélérations décoiffantes et pointes de vitesse dignes d’un avion de chasse, mais aussi pour qu’on ait le sentiment, confortablement installé dans les banquettes en cuir de buffle qui garnissaient le véhicule, d’être à la Scala de Milan en 1950, en train d’assister à l’enregistrement en live de la Tétralogie de Wagner sous la direction incomparable (quoiqu’un peu rigide et exagérément théâtrale, aux dires de certains dont l’opinion m’importe finalement assez peu) du très germanique Wilhelm Furtwängler, directeur de génie que sa relative indolence (euphémisme) pendant la guerre ne saurait en aucun cas désigner comme un sympathisant nazi, j’en veux pour preuves les tentatives restées vaines d’Hitler et Goebbels pour le corrompre, ainsi que son soutien avéré pour la communauté juive. Je rappelle que grâce à lui, de nombreux musiciens et chefs allemands comme Josef Krips, Otto Klemperer, Bruno Walter, Hans Knappertsbusch et même l’étrange Arnold Schönberg, peintre assez médiocre il faut bien le dire mais compositeur de génie, inventeur du dodécaphonisme et cofondateur, avec Berg et Webern, de la Seconde école de Vienne (la Première regroupant des gens comme Haydn, Mozart et ce bon vieux Ludwig van), ont été sauvés de la déportation, ainsi qu’en témoignent certains documents transmis par Curt Riess au général Robert McClure, l’officier américain chargé d’instruire le procès de Furtwängler dans le cadre du projet de dénazification tel que ratifié par les accords de Potsdam en juillet 45. Cette année-là, il faisait un temps superbe à Postdam, un ciel plus bleu que bleu irradiant une luminosité intense, et Staline, Truman et Churchill semblaient très à l’aise dans les jardins du château de Cecilienhof, ancienne résidence de Guillaume de Prusse et sa charmante épouse Cécile de Mecklembourg-Schwerin. Tellement à l’aise que c’est depuis le fond de son transat avec vue imprenable sur la Havel, un cigare offert par Churchill au bec, que Truman donna l’ordre au général Spaatz, futur chef d’état-major de l’US Air Force, de larguer Fat Man et Little Boy sur la gueule des Japs : plus de deux cent mille morts et des milliers de grands brûlés condamnés à endurer les pires souffrances pendant des décennies. On a parfois tendance à l’oublier, mais c’est dans un déluge de feu que la «guerre froide» (d’après George Orwell dans You and the Atomic Bomb) a commencé.

Mais quel que soit mon amour pour la musique sérielle et mon ardent désir que toute la lumière soit faite sur les activités de Furtwängler pendant la période nazie, au-delà des zones d’ombre liées à certaines déclarations pas toujours très heureuses du maître berlinois, je préfère en rester là. Je me contenterai de rappeler que Berta Geissmar, sa fidèle secrétaire de confession juive, n’a jamais cessé d’affirmer et continue à clamer depuis sa tombe que Furtwängler, en dépit de certaines allégations proférées par des esprits chagrins, n’a jamais été l’ami du gros Göring, médiocre mélomane mais excellent mégalomane et toxicomane notoire, lequel semble lui avoir toujours préféré le fils de notables salzbourgeois Herbert von Karajan. Ce dernier, qui ne crachait pas sur le fric, les honneurs et les joies de la vie en (bonne) société, avait eu la présence d’esprit, après des épousailles peu fructueuses - erreur de jeunesse - avec Elmy Holgerloef, une chanteuse d’opérette proche de Goebbels, de se remarier avec Anita Gütermann, riche héritière dont il divorcera à son tour, une fois au sommet de sa gloire, pour convoler avec la succulente Éliette Mouret, mannequin de 17 ans croisé à Saint-Tropez, la Côte d’Azur étant une des autres grandes passions de Karajan. Tel était bien évidemment son droit le plus strict, raison pour laquelle je ne me permettrai en aucune manière d’émettre ne serait-ce que la moindre critique sur sa vie et ses motivations amoureuses. Par contre, si l’on en croit certains rapports diversement circonstanciés, il semblerait que notre homme ait été nettement plus conciliant envers le régime nazi, en tout cas moins regardant, mais en conservant toutefois suffisamment de recul pour ne jamais se commettre totalement. Mais cette fois encore, j’espère que vous ne m’en voudrez pas, le moment me semble assez mal choisi pour rouvrir une polémique sur ce descendant d’un Valaque de Kozani émigré en Saxe et anobli par le prince-électeur Frédéric-Auguste le Juste pour services rendus à la Nation.

En clair, je ne vais pas vous refaire toute l’histoire de la seconde guerre mondiale, ni remonter aux Croisades et encore moins à l’extinction du crétacé ou au Big Bang, pour tenter de vous faire comprendre à quel point Nathan était un génie de la mécanique (à l’égal d’un Furtwängler pour la direction d’orchestre ou un Harmony Korine pour le cinéma d’auteur américain), un alchimiste capable de transmuter la pire poubelle asthmatique perdant ses boulons en chemin en missile de croisière inaltérable.

Il m’a fallu quelques instants pour encaisser le choc de cette rencontre inattendue, puis je me suis retrouvé dans le van en compagnie de Titus, qui m’avait escorté jusqu’ici, de Greg bien sûr, de son frère Nathan, guest-star dont personne n’avait jugé utile de m’avertir de la présence, de mon cher Manu, qu’on ne présente plus, et du capitaine Samuel Girard, ancien des Forces Spéciales reconverti dans la sécurité civile, le gardiennage, le recouvrement de dette et accessoirement l’extorsion de fonds au profit de particuliers ou sociétés aux activités pour le moins douteuses, cuisinier hors pair capable de vous tirer des larmes avec quelques ingrédients savammment disposés dans le fond d’une assiette (cette capacité à émouvoir, en complète contradiction avec l’essence même de son être, restait pour moi un des plus grands mystères de l’univers, à l’égal de l’existence d’un système suffisamment foireux pour avoir conduit à l’émergence d’une espèce aussi nuisible que la nôtre), phalériste détenteur d’une remarquable collection de décorations anciennes (chose qui, en dépit des explications savantes qu’il manquait rarement de déverser à chaque présentation, n’éveillait pas en moi le moindre embryon des prémices du plus infime début de commencement d’intérêt), ceinture noire cinquième dan d’aïkido (il avait suivi l’enseignement de Yoshimura Masayoshi, grand maître de l’école Katsuhiko à Kobe, sur l’île de Honshu, connu pour sa capacité à projeter ou tétaniser ses adversaires par la seule force de la pensée, décédé dans sa cent-onzième année d’une chute malencontreuse dans les escaliers de son dojo), et enfin tireur d’élite capable de défoncer le trou de balle d’un oiseau-mouche à cinq cents mètres de distance.

Il régnait dans l’habitacle une douce ambiance de fin du monde, en plus de quelque chose de parfaitement ridicule et dérisoire que je ne parvenais pas à m’expliquer réellement, si ce n’était qu’il fallait être complètement ravagé pour monter dans un van transformé en char d’assaut avec une poignée de tarés pour aller investir en pleine nuit un repaire de nazillons nazes connus dans le milieu sous le vocable fleuri (des fleurs vénéneuses et malodorantes) de Disciples de la Colère. Et comme si ça ne suffisait pas, Nathan, dont les goût musicaux contrastaient assez violemment avec le profil général, avait choisi, pour nous accompagner dans cette aventure, l’Air de la folie de Lucia de Lammermoor (aria de la scène 1 de l’acte III, quand Lucia perd la raison et poignarde son mari Arturo Bucklaw pendant leur nuit de noces avant de se donner la mort), opéra seria de Donizetti inspiré de La Fiancée de Lammermoor de Walter Scott et créé le 26 septembre 1835 sur la grande scène du théâtre San Carlo de Naples. Je n’ai rien contre l’auteur de Lucrèce Borgia, Don Pasquale et La Fille du régiment, même si je ne suis pas spécialement fan de bel canto, un peu primaire à mon goût, pour ne pas dire chiant, tout comme je trouve très chiantes ces pénibles histoire d’amour impossible sur fond tempétueux de grands espaces battus par les vents, châteaux hantés, marais insalubres, vieilles rivalités familiales et envolées lyriques semblables à des vols de corbeaux planant sur les destinées d’innocentes victimes de malédictions ancestrales, mais un tel choix musical, qui plus est dans une interprétation grésillante de Maria Callas datée de 1953 (entendons-nous bien, ce n’était pas la voix de Maria qui était grésillante mais l’enregistrement, même s’il a pu arriver à la voix de Maria de grésiller un peu en fin de carrière, ce qui n’enlève bien entendu rien à l’immensité de son talent, génie, même, dans certaines interprétations restées légendaires, de Puccini à Delibes en passant par Gounod et Verdi), un tel choix musical, disais-je, que d’aucuns auraient pu juger audacieux (voire disruptif, pour reprendre un concept cher à Jean-Marie Dru et popularisé par un transfuge prépubère et chronophile de la banque Rothschild infiltré au plus haut niveau de l’État), me paraissait à moi totalement incongru. D’autant que ce même air tournait en boucle avec une coupable insistance, chose qui à la fois vous mettait les nerfs en pelote, indiquait on ne plus clairement que Nathan souffrait d’une pathologie mentale sur laquelle il faudrait bien se résoudre à statuer un jour ou l’autre, et in fine vous plaçait dans une espèce d’état de transe suspect dont j’ai vite compris que le but ultime n’était pas de chanter les louanges de l’harmonie mais d’exalter jusqu’à l’euphorie la pulsion de mort qui habite chacun de nous, et ce faisant le pousser à commettre les exactions les plus répréhensibles sans se soucier le moins du monde des conséquences de ses actes, prendre conscience ne serait-ce qu’un seul instant de leurs implications désastreuses dans le cadre de la vie en société et des valeurs fondamentales de la République.

Oui, c’est exactement ce qui se produit si vous écoutez l’Air de la folie de Lucia de Lammermoor en boucle, et tout spécialement dans la version grésillante de Maria Callas enregistrée en 1953 à la Scala de Milan sous la direction de Tullio Serafin, excellent violoniste au demeurant, ancien assistant de Toscanini et grand amateur de (tournedos) Rossini, qui, outre la Callas, a eu le privilège de diriger au cours de sa longue et prestigieuse carrière (le bougre est mort en pleine possession de ses moyens ou presque à quatre-vingt-dix ans, cf la biographie Tullio Serafin, le patriarche du mélodrame par Teodoro Celli \& Giuseppe Pugliese) des cantatrices aussi admirables, inoubliables et bouleversantes que la sulfureuse Elisabeth Schwarzkopf, la langoureuse Victoria de los Angeles et la sémillante Christa Ludwig.

Bref, comme on dit chez nous, ça craignait méchamment du boudin, pour reprendre une expression populaire qui fait clairement référence au caca (et son odeur déplaisante, étant entendu que les expressions populaires, parfois non dénuées de bon sens, sont rarement d’une finesse exemplaire), matérialisé ici par le boudin noir, et par extension à l’anus et la sodomie, le boudin faisant alors office de représentation du pénis qui s’introduit dans le fion de la victime pour y effectuer un certain nombre d’allées et venues plus ou moins dévastatrices, d’où cette fois encore la notion de désagrément afférente à ladite expression. Oui, ça boudinait grave, à tel point que je me suis demandé un instant ce que je foutais dans ce van, par quelle aberration je m’étais retrouvé à monter dedans, dans un état d’ébriété avancée qui plus est (ou était, suivant le niveau de cohérence temporelle que l’on choisit d’adopter), et si je n’allais pas tout bonnement sauter en marche et me précipiter dans le premier taxi venu pour regagner mes pénates. Et tenez, à propos de Pénates, avez-vous seulement la moindre idée de qui ils sont, ou étaient, suivant le niveau de cohé… etc… etc… vertige de la syntaxe (rien à voir avec la sainte taxe de Trump, fallait oser la faire, celle-là) ? D’abord, ce sont des hommes et non des femmes, contrairement à ce qu’on pourrait penser à l’ouïe de la consonance du mot (petit NB vite fait en passant : à propos de la locution «à l’ouïe de», le Littré, hélas tombé en désuétude, nous confirme qu’elle «est bonne, qu’on la dit à Genève et qu’elle appartient au style réfugié, autrement dit celui des protestants français chassés par la révocation de l’édit de Nantes», voilà qui ne manque pas de charme ni de piquant), chose étrange, du reste, car on ne dit pas une mainate, par exemple, mais bel et bien UN mainate, religieux ou autre (il en existe de nombreuses espèces comme le Martin triste, le Mino de Dumont, le Scissirostre des Célèbes ou mainate dubitatif, le Streptocitte à cou blanc et le Basilorne de Céram ou mainate des Moluques, sans oublier l’étourneau de Rothschild et le mainate des mangroves de Floride), sturnidé du genre Gracula qui en plus de bouffer comme un cochon a la capacité inattendue, au même titre que le perroquet sinon mieux, de reproduire à la perfection les sons qui l’entourent, à commencer par les intonations de la voix humaine. Ensuite, figurez-vous que les Pénates n’étaient autres que les dieux romains, probables descendants des Dioscures de Zeus et Léda, chargés de veiller sur le confort du foyer et le bien-être de ses occupants. Oui, à ceci près que mon bien-être à moi tout le monde s’en foutait comme de l’an quarante, à commencer par le ou les dieux censés résider dans les plus hautes sphères de l’atmosphère, et que dans le cas présent ce n’était pas Léda mais Laideron, sa sœur jumelle moche, qui avait enfanté de la situation nauséabonde dans laquelle j’étais allé me fourrer avec la naïveté d’un enfant de chœur tournant le dos au curé de la paroisse.

Au bout d’un quart d’heure à tourner dans la ville, j’ai commencé à me poser des questions sur la nature exacte de notre destination. En d’autres termes, les Disciples de la Colère se réunissaient habituellement dans un pavillon de la banlieue nord, rue Jordan Peshkov (un agent du KGB qui bossait pour la France pendant la guerre froide, exfiltré de justesse avant de tomber aux mains de l’ennemi, liquidé des années plus tard dans cette même rue alors qu’il coulait des jours paisibles dans notre beau pays et pensait bien en avoir fini avec cette période difficile de son existence, mais c’était compter sans la rancune tenace de certains de ses ex-employeurs), et sans avoir un GPS dans le cul ni être doté d’un sens de l’orientation particulièrement performant, j’avais quand même le très nette impression que l’itinéraire choisi n’était pas le plus court chemin pour s’y rendre, même si, j’en conviens, ma perception des choses était peut-être très légèrement altérée par les quelques dizaines d’hectolitres de boissons alcoolisées ingurgités pendant la soirée.

D’où ma question : On va où, au juste ? Allo, y a quelqu’un, une présence, une âme qui vive, une loupiote dans la nuit ? ou dois-je, tel le poète, errer sans fin dans les grottes du destin à la lueur d’une bougie au bord de l’extinction ?

Greg, manifestement agacé par le fait que je n’avais pas respecté le deal, lequel était de se préparer physiquement et mentalement à affronter une bande de tarés qui n’en avaient plus rien à foutre de quoi que ce soit, de véritables boules de haine, bonbonnes de gaz remplies de clous qui n’attendaient qu’une occasion de vous exploser à la gueule : On passe prendre quelqu’un.

\textsc{Moi} : Ah bon ? Qui ça ?

\textsc{Lui} : Sally Robinson, au Sugar \& Spice.

Votre serviteur, cueilli à froid par une déferlante d’interrogation : C’est une blague ?

\textsc{Lui} : Non.

\textsc{Sam} : C’est qui, ce type ?

\textsc{Greg} : Un client à moi.

\textsc{Sam} : T’emmènes tes clients en vadrouille, maintenant ?

\textsc{Greg} : Il est personnellement impliqué dans une sale affaire avec ces types. Pout tout te dire, le gars est gay et ils ont carbonisé au lance-flammes une personne qui lui était chère.

\textsc{Sam} : J’étais pas au courant, j’aime pas ça. Je pars pas en expé avec n’importe qui. Tu le savais, Nath ?

\textsc{Nathan} : J’étais vaguement au courant, oui.

\textsc{Sam} : Vaguement ?

\textsc{Nath} : Greg m’a juste dit qu’on passerait récupérer quelqu’un en route.

\textsc{Sam} : Et toi, Titi ?

\textsc{Titus} : Si on avait cramé ta femme au lance-flammes, je crois que toi aussi tu aurais à cœur de te venger.

\textsc{Sam} : Sauf que j’ai pas de femme, et qu’en l’occurrence, si j’ai tout bien compris, la femme est un homme.

\textsc{Titus} : Et alors ?

\textsc{Greg} : Alors il faudrait voir à ne pas se tromper de camp. Les homophobes c’est eux, pas nous.

\textsc{Sam} : Je suis pas homophobe.

\textsc{Nat} : Allons donc. Tout le monde sait qu’il y a plein d’homophobes dans l’armée.

\textsc{Sam} : Je suis plus dans l’armée. Et puis, si tu vas par là, il y en a plein aussi chez les garagistes.

\textsc{Titus} : Vous disputez pas, les gars. C’est comme les racistes, il y en a plein partout.

\textsc{Nat} : Oui, mais moi je suis un garagiste progressiste qui vit avec son temps. C’est pas parce qu’on passe sa vie les mains dans le cambouis qu’on est une brute pour autant.

\textsc{Sam} : Et moi je suis ni raciste ni homophobe, je vous rappelle que j’ai quand même passé quelques années dans la légion. C’est formateur, la légion, on apprend à vivre avec plein de gens d’origines et de confessions différentes. Par contre, comme tous les pros, j’aime bien savoir avec qui je pars en opé. Risquer ma peau avec un trave qui se trémousse sur une scène de music-hall, très peu pour moi, même si je reconnais qu’il peut y avoir des traves très costauds. J’en ai connu un en Afghanistan, d’origine néerlandaise, qui se produisait dans un clandé de Kandahar, eh bien je peux vous garantir que ce type était une véritable boule de muscles. Et quand je dis boule, c’est pas seulement une image : il mesurait moins d’un mètre soixante et était aussi large que haut, de sorte qu’on pouvait le faire rouler dans les lignes ennemies comme une boule dans un jeu de quilles.

\textsc{Greg} : Ben c’est marrant que tu dises ça, parce Sally Robinson, la femme dont je te parle, enfin le mec dont je te parle, a exactement la même taille et le même profil sphérique que ton pote de Kandahar.

\textsc{Moi} : Sauf que lui, si on le fait rouler, j’ai peur que ses cinquante kilos de nichon l’empêchent d’aller bien loin !

\textsc{Nathan} : Perso, je suis un garagiste ouvert d’esprit et ça me dérange pas d’aller au casse-pipe avec un travelo.

\textsc{Titus} : Et avec un Black ?

\textsc{Nathan} : Avec un Black non plus. Y a des garagistes travelos, vous savez.

\textsc{Titus} : Et black aussi.

\textsc{Nathan} : Oui, et aussi des travelos black qui bossent dans des garages.

\textsc{Moi} : Oui enfin, je sais pas pourquoi, mais j’ai le plus grand mal à imaginer un type passer ses journées les mains dans la graisse et s’habiller en femme à la nuit tombée pour aller faire le guignol sur une scène de music-hall.

\textsc{Greg} : Tout est possible, en ce bas monde.

\textsc{Moi} : Très bas, même, et plus on descend plus on tombe sur des trucs pas très catholiques.

\textsc{Titus} : Catholiques ou musulmans, on s’en fout, du moment qu’ils ne viennent pas nous chier dans les pattes.

\textsc{Greg} : Tous ces trucs bizarres qui vivent dans les abysses.

\textsc{Nathan} : Qui ? les musulmans ?

\textsc{Sam} : Qui vivent dans les abysses de la civilisation.

\textsc{Moi} : Faut pas généraliser. Il y a des musulmans très bien, qui mangent avec une fourchette et un couteau.

\textsc{Greg} : Ouais, à Dubaï.

\textsc{Titus} : Ils ont même du pétrole, une montre suisse et des voitures de sport avec des sièges en cuir de dromadaire.

\textsc{Greg} : Et des autoroutes flambant neuves pour rouler dessus avec leurs voitures de sport.

\textsc{Moi} : Et une cravate en soie sous la djellaba. Si ça c’est pas un signe de civilisation, alors je serais curieux qu’on me dise ce que c’est.

\textsc{Sam} : N’empêche, cravate ou pas, je maintiens que les trucs bizarres qui vivent dans les abysses ne sont pas censés remonter à la surface.

\textsc{Titus} : Peut-être pas, mais ça leur arrive quand même de remonter.

\textsc{Sam} : Le fait est qu’on retrouve souvent des créatures venues du fond des âges venues s’échouer ici et là.

\textsc{Moi} : Vomies par les entrailles de l’océan.

\textsc{Greg} : Ouais, comme ce calmar de quinze mètres de long retrouvé une plage de Californie la semaine dernière.

\textsc{Moi} : Un calmar avec une cravate ?

\textsc{Greg} : Ouais, et une djellaba.

\textsc{Titus} : Un calmar musulman, sans doute.

Nathan, l’œil vif du conducteur toujours sur le qui-vive, au ralenti dans une ruelle à sens unique : Et qu’est-ce qu’il est devenu, ton calmar ?

\textsc{Greg} : Je sais pas. Ils ont dû le mettre dans un grand bocal rempli de formol pour l’exposer au musée d’histoire naturelle de Santa Barbara.

\textsc{Nathan} : On arrive bientôt, les gars.

\textsc{Sam} : Vous êtes toujours sûrs qu’on embarque ce clown avec nous ? Il ne faut pas confondre chasse aux nazis et vacances au club Med. En tout cas, je vous préviens tout de suite qu’il ne faudra pas compter sur moi pour assurer ses arrières. Il vient, il se démerde, et il ne faudra pas qu’il vienne se plaindre s’il se prend une balle dans le cul.

\textsc{Moi} : On n’a qu’à mettre ça au vote. Qui est pour que Sally Robinson vienne avec nous ?

Un doigt s’est levé, celui de Greg, suivi de celui de Nathan, plus hésitant mais se sentant obligé de suivre son grand frère, et enfin celui de Titus, qui en tant que représentant officiel des minorités opprimées et autres espèces en voie disparition (contrairement au garagiste qu’on ne peut décemment pas classer dans cette catégorie, tant les problèmes récurrents qui affectent nos véhicules, sa rapacité notoire et sa déontologie approximative le rangent sans ambiguïté du côté des oppresseurs, voire des presseurs tout court, pressurateurs, pressoirs ou presses hydrauliques qui extraient jusqu’à la dernière goutte de suc de leurs victimes) pouvait difficilement faire autrement, même si on sentait bien qu’il ne barbotait pas dans un lagon d’eau turquoise d’enthousiasme à toute épreuve.

\textsc{Moi} : Proposition retenue à trois voix contre deux. Sally sera des nôtres ce soir, et Greg, qui a eu la bonne idée de nous la fourrer dans les pattes, sera personnellement chargé d’assurer sa protection.

Greg, jetant sur moi un œil chargé de ce qu’il faut bien appeler un doux mélange de réprobation certaine et d’acrimonie à peine dissimulée : Pas de problème.

\textsc{Nathan} : En parlant de calmar….

\textsc{Greg} : On ne parle pas de calmar…

\textsc{Nat} : Non, mais on en parlait y a pas longtemps.

\textsc{Moi} : On a parlé de calmar, c’est vrai.

\textsc{Greg} : Peut-être, mais on ne va pas en parler toute la soirée.

\textsc{Nathan} : Je voulais juste préciser que justement, j’en ai mangé ce soir, des calmars.

\textsc{Moi} : T’as mangé des calmars ?

\textsc{Lui} : Ouais, en boîte.

\textsc{Sam} : Je suis pas fan de calmar.

\textsc{Greg} : Surtout en boîte, c’est sec et ça n’a pas de goût.

\textsc{Moi} : Par contre, frits à la romaine, c’est pas dégueulasse. Surtout avec de la mayonnaise.

\textsc{Titus} : Vous avez vraiment des conversations de merde, les gars.

\textsc{Moi} : Pourquoi, on ne mange pas de calmar en Sierra Leone ?

\textsc{Titus} : Non. On mange des plantains frits, de la soupe de gombo et de la sauce palabre aux feuilles de taro. Quand on mange, parce que la plupart du temps on ne mange pas. La plupart des gens crèvent de faim dans l’indifférence quasi générale, alors tu penses bien qu’on a autre chose à foutre que de s’amuser à découper des calmars en rondelles.

\textsc{Sam} : C’est quoi la soupe de gombo ?

Nathan, passant un coup d’essuie-glace sur le pare-brise crasseux du G30 (dont l’éclairage n’était pas la principale qualité) pour s’assurer de la réalité de sa vision : C’est quoi, ça ?

Greg, qui se trouvait à ses côtés (à la place dite «du mort», du temps où la ceinture de sécurité n’avait pas encore été inventée et où la personne assise à cet endroit était celle qui avait le plus de chances de passer comme une balle à travers le pare-brise), n’avait manifestement aucune réponse significative à apporter à la question : Aucune idée.

Quant à nous, Sam et Titus derrière et votre serviteur relégué tout au fond tel un pestiféré, notre visibilité était loin d’être optimale et ne nous permettait par conséquent pas d’apporter des informations susceptibles d’éclairer le problème d’un jour nouveau.

Globalement, la situation était la suivante : au détour d’une ruelle, aussi sombre et étroite que l’anus d’un ver de terre, une forme venait d’apparaître dans les phares du van.

Cette forme, qui se dressait au milieu de la chaussée tel un spectre décharné de retour sur terre après un long séjour six feet under, avançait lentement vers nous, ayant manifestement du mal à tenir sur ses jambes ramollies par l’inactivité, mais ne semblait pas, en dépit d’une infériorité numérique évidente à tout point de vue, manifester la moindre appréhension concernant le sort funeste qui l’attendait si elle venait à entrer en collision avec le monstre de puissance et d’acier dont Nathan s’efforçait tant bien que mal de conserver le contrôle. Car n’en doutons pas, si une telle chose venait à se produire, ce pantin désarticulé serait renvoyé en pièces détachées dans les flammes de l’enfer et n’aurait aucune chance de survivre une seconde fois au traumatisme de sa destruction, intégrale cette fois, définitive et sans appel, et tous les tubes de colle et tours de passe-passe des magiciens de l’au-delà n’y pourraient rien changer.

Court vêtue d’oripeaux malodorants qu’elle semblait avoir empruntés à un épouvantail et entassés les uns sur les autres sans se soucier du résultat, la chose avançait vers nous en se dedandinant. Je dois maintenant vous révéler qu’il s’agissait d’une femme d’origine manifestement africaine, d’une beauté assez confondante quand on se donnait la peine de l’examiner un peu mieux sous toutes les coutures. La peau très noire tendue sous les muscles saillants, les membres longs et fins, la poitrine ferme et menue, la taille de guêpe, le cul rebondi, le cou interminable et les dents d’une blancheur éclatante dissimulées derrière le rideau pulpeux de lèvres écarlates, tout semblait indiquer, au-delà des apparences qui ne plaidaient pas en sa faveur, qu’elle était le rejeton de quelque prestigieuse lignée. Je pourrais aussi vous parler de ses yeux, son regard, mais des centaines de pages seraient nécessaires pour en faire le tour. Une telle interruption, pour intéressante qu’elle soit, ne manquerait pas de plomber le rythme de l’action en cours, particulièrement dynamique en cet instant vous n’êtes pas sans l’avoir remarqué. Je pourrais aussi, durant quelques centaines de pages supplémentaires, lui tartiner le cul des éloges les plus dithyrambiques qui soient, tant cette partie de son anatomie était ronde et charnue, moulée à la louche dans le lait entier du désir et la sexualité la plus débridée, sculptée et polie de main de maître dans le diamant brut de l’extase et la sensualité, mais le résultat, pour terriblement excitant qu’il soit, serait tout aussi préjudiciable au bon déroulement de l’action.

Greg, à son frère : Qu’est-ce que tu fabriques ?

\textsc{Nathan} : Ben rien, je roule.

\textsc{Greg} : Justement, il faudrait peut-être songer à t’arrêter.

\textsc{Nathan} : Je la connais pas, moi, cette bonne femme.

\textsc{Greg} : C’est pas une raison pour l’écraser.

\textsc{Nathan} : J’aime pas ça.

\textsc{Greg} : Arrête-toi, je te dis !

Nathan, sautant à contrecœur sur la pédale de frein : Ça va, c’est bon, je m’arrête.

\textsc{Sam} : Elle est canon !

\textsc{Moi} : Je confirme : bizarrement fringuée, mais canon. T’en penses quoi, Titi ?

\textsc{Titus} : Bof.

\textsc{Moi} : Comment ça, bof ?

\textsc{Lui} : Je te rappelle que j’ai une femme et des gosses, Djef.

\textsc{Moi} : Et alors ? Moi aussi j’ai une femme, ça ne m’empêche pas d’apprécier la beauté des autres.

\textsc{Lui} : C’est pas vraiment une femme.

\textsc{Moi} : Je te demande pardon ?

\textsc{Lui} : C’est une femme, bien sûr, mais c’est pas comme si vous étiez mariés et aviez des gosses.

\textsc{Sam} : Je sais pas d’où elle sort, mais j’avoue que ça ne me déplairait pas de faire plus ample connaissance, apprendre à mieux se connaître.

\textsc{Titus} : Vous êtes vraiment des porcs, les gars. Surtout toi, Djef.

\textsc{Moi} : On n’est pas des porcs, on est des amateurs de belles choses, et contrairement à toi on est particulièrement sensibles au charme des articles exotiques.

\textsc{Sam} : Exact.

\textsc{Titus} : Je vous rappelle qu’on a une mission en cours, on n’est pas là pour faire du tourisme sexuel.

\textsc{Nathan} : Je suis d’accord que c’est pas forcément une bonne idée de s’arrêter.

\textsc{Greg} : Oui ben arrête-toi quand même !

\textsc{Nathan} : C’est ce que je suis en train de faire, figure-toi.

\textsc{Greg} : On s’en débarrasse en vitesse et on file rue Théo Cazenave récupérer mon client.

\textsc{Sam} : Ça non plus c’est pas une bonne idée.

On n’arrête pas aussi facilement un char d’assaut qu’une planche à roulettes. Avec le G30 de Nathan, il fallait commencer à freiner la veille pour espérer s’arrêter le lendemain. Pour un type comme lui, qui aimait pulvériser les chronos, s’arrêter n’était pas la préoccupation première. Il voyait l’existence comme une longue ligne droite sur laquelle on aurait pu accélérer sans fin jusqu’à atteindre le septième ciel. En fait, il aurait dû être pilote de chasse. Autrement dit, le V8 du G30 avait été gonflé au point de frôler la désintégration à chaque instant, mais il fallait prendre de l’élan et sauter à pieds joints sur la pédale de frein pour réussir son coup.

Par miracle, le van s’est arrêté à temps, à quelques centimètres de sa proie, en l’occurrence la sublime princesse-zombie court vêtue d’oripeaux malodorants qui déambulait en pleine nuit au milieu de la rue sans se soucier du qu’en-dira-t-on, et après tout on ne pouvait pas lui en vouloir car quand on est aussi magiquement belle il n’y a aucune raison de se faire chier à se soucier de ce que pensent les autres, ce ramassis de minables qui ne vous arrivent pas à la cheville et n’ont rien de mieux à faire que de vous cracher à la gueule toute la bile accumulée au cours de leur misérable existence. Bon, elle semblait quand même avoir pris quelques substances qui ne justifiaient peut-être pas d’un état sanitaire irréprochable sur le plan mental, mais malgré tout ça, en dépit de ces approximations difficilement compréhensibles pour le commun des mortels, il existait au plus profond d’elle-même des choses que rien ni personne ne pourrait jamais atteindre ni altérer.

Nathan, qui détestait s’arrêter, a aussitôt descendu sa vitre pour balancer une bordée d’injures à l’intéressée (bordée dont je préfère, afin de ménager la sensibilité du lecteur, m’abstenir de répéter les termes exacts, particulièrement grossiers il faut bien le dire, Nathan, je le rappelle à sa décharge, étant garagiste, et le garagiste dans son ensemble n’étant pas spécialement réputé pour la tenue de son langage). Eh bien vous allez rire, ou pas je n’en sais rien, mais les imprécations en question n’ont pas eu l’heur de l’émouvoir le moins du monde.

Strictement rien à foutre, à peine un regard pour ce pauvre Nathan qui s’époumonait la tête à la fenêtre.

Tant et si bien qu’il a fini par sortir de la voiture, suivi de Greg, Sam et moi-même, qui tous (à part Titus qui, respectueux des valeurs et traditions en vigueur dans son pays, savait rester digne en toute circonstance) souhaitions nous assurer que la mystérieuse apparition n’était pas le fruit d’une hallucination collective (bon j’avoue qu’en ce qui me concerne je faisais surtout une petite fixette sur sa plastique enchanteresse, et caressais secrètement le projet de l’enlever pour l’attacher nue au fond d’une cave et abuser d’elle à longueur de journée jusqu’à ce que mort s’ensuive, projet ignoble s’il en est, et rêve inaccessible condamné à rester enchaîné à tout jamais dans les bas fonds de mes plus viles pulsions, la boue fantasmatique d’étreintes telles que l’humanité n’en a plus connu depuis que le dernier rhinocéros laineux s’est éteint dans les steppes de Mongolie).

La mystérieuse apparition n’était pas le fruit d’une hallucination collective, je vous le confirme avec émotion, mais ne semblait pas excessivement perméable aux informations que nous tentions de lui faire passer aussi élégamment que possible, à savoir qu’il aurait très aimable de sa part de dégager le passage afin que nous puissions continuer notre route sans avoir à faire usage de la force, chose qui, contrairement à ce que l’arsenal que nous trimballions pouvait laisser penser, aurait été pour nous (au moins pour Greg et moi-même, qui étions d’une nature plutôt affectueuse, Nathan et surtout Sam pouvant se montrer assez belliqueux à l’occasion) la source de la plus vive contrariété.

Aussi lui ai-je dit, de ma voix la plus douce, en la prenant par le bras avec une avidité certaine : Venez, mon enfant, je vais vous reconduire sur le trottoir.

Sa peau était douce et soyeuse comme de la mousse, et la sensation de son contact si délicieuse que mon cœur a bien failli s’arrêter de battre. Cela dit, j’avais beau la pousser fermement, tout en faisant preuve de la plus extrême délicatesse bien entendu, du plus grand soin pour éviter d’infliger ne serait-ce que la plus légère flétrissure à son épiderme, l’étrange créature ne bougeait pas d’un millimètre.

\textsc{Greg} : Il ne faut pas rester là, mademoiselle, c’est dangereux.

Sam, dans un autre registre : Bouge ton cul de là, si tu ne veux pas qu’il t’arrive des bricoles.

\textsc{Nathan} : Laisse tomber, tu vois bien qu’elle est complètement défoncée.

\textsc{Moi} : Messieurs, s’il vous plaît ! Essayez de faire preuve d’un peu de compassion, au moins une fois dans votre vie.

Finalement, la fille s’est remise en marche et dirigée droit vers l’arrière du véhicule, à l’endroit où se trouvait Titus, lequel était en train de se curer les ongles avec un couteau de chasse.

Il a levé la tête et la fille a planté ses yeux droit dans les siens, accrochant son regard pour ne plus le lâcher.

Quelques minutes plus tard, alors qu’ils étaient toujours en train de se fixer intensément sans dire un mot, j’ai commencé à trouver le temps long et me suis dit que le moment était venu de rompre le charme. Il s’agissait peut-être d’un de ces coups de foudre dont les gens parlent avec admiration et que tout un chacun rêve de connaître au moins une fois dans sa vie, cette espèce d’éclair qui fait que deux êtres qui se rencontrent pour la première fois deviennent inséparables en une fraction de seconde, irrésistiblement attirés l’un vers l’autre comme les deux moitiés d’un même corps qui se retrouveraient enfin après s’être cherchés durant des siècles aux quatre coins de la planète. Et dans ce cas, même quelqu’un comme Titus, qui était la droiture même, n’hésiterait pas à bazarder la femme et les enfants qu’il adorait par dessus tout pour partir à l’aventure avec cette inconnue. Oui, l’être humain fait parfois preuve de comportements irrationnels, en totale contradiction avec ses convictions les plus affirmées. Par quelle mystérieuse alchimie deux êtres qui ne se connaissent ni d’Eve ni d’Adam ont soudain le sentiment d’être une seule et même personne, éprouvent l’un pour l’autre une attraction aussi instantanée que fusionnelle, au point de perdre tout repère et se laisser aller corps et âme dans la tourmente ? S’agit-il de la rencontre inopinée de deux profils fantasmatiques en parfaite adéquation, lesquels, relégués depuis des lustres dans les oubliettes du refoulement, explosent littéralement au contact l’un de l’autre, produisant un feu d’artifice d’étincelles contradictoires qui cloue le sujet sur place et le déstabilise au point de tout laisser tomber pour s’y consacrer sans réserve ? C’est complexe, et il se peut aussi que les phéromones copieusement émises par les protagonistes sous le coup de l’émotion jouent un rôle non négligeable dans cette affaire. Ou alors, il s’agit de quelque chose qui échappe totalement à notre entendement, auquel cas il n’est d’aucune utilité de continuer à se gratter les méninges jusqu’au sang. On n’y comprendra jamais rien, et le mieux, dans ce cas-là, est de prendre le parti de s’en foutre et se laisser porter sans résistance par les flots écumeux du destin.

Quoi qu’il en soit, ne serait-ce que par égard pour Bérénice et les enfants, que je connaissais depuis toujours et considérais un peu comme les miens, je ne pouvais décemment pas laisser Titus s’abîmer dans les affres de l’amour fou, assister sans réaction au spectacle désolant de la passion dévastant ce qu’il avait mis tant d’années à construire, je veux bien sûr parler de cette réussite familiale exemplaire qui constituait pour nous une inépuisable source d’admiration et d’inspiration.

\textsc{Moi} : Titus ?

Lui, sans quitter la fille des yeux : Quoi ?

\textsc{Moi} : Tu fais quoi, là ?

\textsc{Lui} : Rien.

\textsc{Moi} : Okay. Faut y aller, maintenant.

\textsc{La fille} : Non, lui pas aller.

\textsc{Sam} : Comment ça, pas lui ?

\textsc{La fille} : Lui pas aller.

\textsc{Nathan} : C’est quoi ces conneries ?

\textsc{La fille} : Lui grand danger pour.

\textsc{Sam} : Elle est complètement cinglée ! Fichons le camp d’ici, les gars.

La fille, à Titus : Toi pas aller.

\textsc{Titus} : Pas aller où ?

\textsc{La fille} : Pas aller, courir grand danger.

\textsc{Titus} : Je peux savoir qui vous êtes ?

\textsc{La fille} : Toi pas aller, grand danger.

\textsc{Greg} : Elle commence vraiment à me faire flipper.

\textsc{Sam} : Oui, moi aussi.

\textsc{Nathan} : Même chose. Je serais assez d’avis qu’on se barre d’ici vite fait. Je vous rappelle qu’on doit encore aller chercher l’autre folle.

\textsc{Moi} : On y va, on y va. T’en penses quoi, Titus ?

\textsc{La fille} : Titus pas aller.

\textsc{Titus} : Pas aller où ?

\textsc{La fille} : Grand danger !

\textsc{Greg} : Je crois qu’on n’en tirera rien de plus.

\textsc{Titus} : Où est-ce que je ne dois pas aller ?

\textsc{La fille} : Toi sortir voiture, maintenant, tout de suite, et rentrer chez toi !

\textsc{Sam} : Pas aller, pas aller, commence à me les briser, celle-là, avec ses pas aller, pas aller !

La fille, en agitant ses ongles de trente centimètres de long sous le nez de Titus : Toi sortir voiture et rentrer chez toi. Pas aller avec les autres, ou toi mourir ce soir.

\textsc{Nathan} : Laisse tomber, elle est folle !

\textsc{Titus} : N’empêche que j’ai pas envie de mourir ce soir, moi. J’ai une femme et des gosses.

\textsc{Sam} : Tu ne vas pas me dire que tu crois à ces conneries !

\textsc{La fille} : Pas conneries ! Si lui aller, lui mourir. Grand danger !

\textsc{Moi} : Tu t’appelles comment, ma chérie ?

\textsc{La fille} : Moi pas chérie ! Moi faire rêves, voir et entendre choses. Moi voir lui dans rêve. Lui pas aller, ou mourir ce soir.

\textsc{Moi} : Et tu as vu quoi d’autre, mon petit lapin ?

\textsc{La fille} : Moi pas lapin. Moi voir gens avec armes tirer partout et lui mourir.

Moi, en montrant Sam : Et lui, tu l’as vu ?

\textsc{La fille} : Pas vu lui. Juste Titus, tête exploser et cervelle gicler partout. Lui pas aller, ou jamais plus revoir femme et enfants.

\textsc{Moi} : Euh… bon, d’accord. Tu fais quoi, sinon ?

\textsc{La fille} : Moi rien faire. Juste voir et entendre choses.

Puis, s’agrippant à Titus pour essayer de le faire sortir de la voiture : Toi sortir voiture, maintenant. Toi pas aller ou mourir ce soir.

Nathan, envoyant des coups d’accélérateur pour signifier son impatience : Bon, on y va. J’ai pas que ça à faire, moi.

\textsc{Greg} : Qu’est-ce que t’as à faire ?

\textsc{Nathan} : Ben… rien. Enfin si, j’ai mes plantes à arroser.

\textsc{Greg} : Tu te fous de ma gueule ?

\textsc{Nathan} : Non, c’est des plantes spéciales que j’ai fait venir d’Amérique du Sud. On ne peut les arroser qu’à une heure précise de la nuit, sans quoi elles dépérissent et dégagent des vapeurs toxiques.

\textsc{Sam} : N’importe quoi ! Je crois que j’ai eu ma dose de conneries pour ce soir !

\textsc{Nathan} : Et puis ils passent Bagne de femmes avec Zarah Leander à trois heures du mat.

\textsc{Sam} : Bagne de femmes ? c’est quoi, ce truc ?

\textsc{Nathan} : Un vieux film de Douglas Sirk. À l’époque, en 36, Sirk ne s’appelait pas encore Sirk mais Sierck. Il bossait pour la principale société de films allemande mais s’est retrouvé embringué dans une sale histoire. Quand sa première femme a appris qu’il s’était remarié avec une Juive, elle l’a balancé aux Boches et il s’est retrouvé obligé de quitter le pays en quatrième vitesse, sans avoir pu revoir son fils de dix ans, Klaus, qu’il avait eu avec la précédente femme en question dont j’ai oublié le nom.

\textsc{Sam} : Ça tombe bien on s’en fout.

\textsc{Nathan} : N’empêche que ce type a eu une vie extraordinaire. Il a quitté l’Allemagne nazie, s’est retrouvé à élever des poulets dans la banlieue de San Francisco, avant de se remettre au cinoche et tourner les films qui l’ont rendu célèbre, comme Désir de femme, Le Secret magnifique, Tout ce que le ciel permet, ou encore Écrit sur du vent avec Rock Hudson, Lauren Bacall et Dorothy Malone. Excusez du peu.

\textsc{Sam} : Super ! Bon, on y va, maintenant ?

\textsc{Nathan} : Mais le drame de sa vie, c’est que…

\textsc{Sam} : Le drame de ma vie, c’est de bosser avec des amateurs qui s’intéressent à tout sauf la mission en cours. Je comprends les mecs qui préférèrent bosser en solo, au moins ils ne sont pas obligés de composer avec une bande de bras cassés dont on ne sait jamais ce qu’ils vont faire ou pas.

La fille, toujours en train de tirer sur Titus qui refusait de sortir de la voiture : Toi venir, venir avec moi.

\textsc{Moi} : Mais enfin, vous ne m’enlèverez pas de l’idée qu’il se passe quand même des trucs bizarres sur cette planète ! Allez, ma petite caille, il faut le laisser tranquille, maintenant.

\textsc{La fille} : Moi pas petite caille. Toi laisser lui partir ou lui mourir ce soir.

\textsc{Nathan} : Je sais que tout le monde s’en fout, mais ça doit quand même être terrible d’apprendre que son fils est mort sur le front russe sans qu’on n’ait jamais pu le revoir ni le serrer une dernière fois dans ses bras. Tout ça à cause de sa salope d’ex-femme nazie !

\textsc{Greg} : Les drames de la séparation.

\textsc{La fille} : Gentil Titus. Toi laisser autres et venir avec moi.

\textsc{Moi} : Je crois que t’as un ticket, Titus. Pense à ta femme et tes gosses, tu ne vas pas tout envoyer balader sur un coup de tête.

\textsc{Titus} : Arrête tes conneries ! Non, je voudrais juste savoir qui est cette fille et ce qu’elle attend de moi.

\textsc{La fille} : Toi sortir voiture et venir avec moi.

\textsc{Titus} : Je ne sais même pas comment tu t’appelles.

\textsc{La fille} : Moi Atiena, gardienne de la nuit. Toi venir.

\textsc{Titus} : Venir ? Mais où ça ?

\textsc{Atiena} : Suivre moi dans chambre.

\textsc{Titus} : Quelle chambre ?

\textsc{Atiena} : Chambre hôtel, tout près d’ici.

\textsc{Sam} : Et voilà, j’en étais sûr !

\textsc{Titus} : Quoi ?

\textsc{Sam} : C’est une pute, j’en étais sûr !

\textsc{Atiena} : Moi pas pute. Moi Atiena, gardienne de la nuit.

\textsc{Nathan} : Gardienne de mon cul, oui !

\textsc{Greg} : Nathan, s’il te plaît, tu sais que j’ai horreur de la grossièreté.

\textsc{Atiena} : Moi Atiena, gardienne de la nuit.

\textsc{Moi} : Oui oui. Alors écoute, Atiena, ma petite caille en sucre…

\textsc{Atiena} : Moi pas caille sucre. Moi Atiena, gardienne de la nuit.

\textsc{Moi} : Oui, bien sûr, il n’y a aucun doute là-dessus et tu ne trouveras personne ici pour prétendre le contraire. N’est-ce pas, les gars ?

\textsc{Greg} : Personne.

\textsc{Sam} : Absolument personne.

\textsc{Moi} : Non, vois-tu, ce que je voudrais te faire comprendre, ô Atiena gardienne de la nuit, c’est que nous avons des choses à faire et que Titus ne va pas pouvoir aller avec toi dans chambre hôtel tout près d’ici. Toi comprendre ou moi pas parler français ?

\textsc{Atiena} : Toi méchant. Titus venir avec moi dans chambre et lui pas mourir ce soir.

\textsc{Moi} : Et il en pense quoi, Titus ? Lui vouloir venir dans chambre hôtel tout près d’ici ?

\textsc{Titus} : Lui pas vouloir mourir ce soir.

\textsc{Sam} : Je le crois pas ! Tu ne vas quand même pas te laisser influencer par cet épouvantail sortir de nulle part !

\textsc{La fille} : Moi pas épouvantail. Moi Atiena, gar…

\textsc{Sam} : … dienne de la nuit, oui, je sais !

\textsc{Nathan} : Il a raison, Titus. Tu ne vas tout de même pas croire les balivernes de cette… cette…

\textsc{Sam} : Pute, tu peux le dire.

\textsc{Atiena} : Moi pas pute, moi gardienne de la nuit. Marcher dans rue pour sauver gens qui vont mourir.

On en était là de nos aventures fascinante avec Atiena, la gardienne de la nuit, qui était d’ailleurs elle-même une créature fascinante dont les yeux étrangement bridés (caractéristique des Khoïsan du Kalahari, proches des San de Namibie, longtemps persécutés tant par ces enfoirés de Bantous, leurs ennemis naturels, que par ces pourritures d’Afrikaners, à tel point qu’ils mériteraient aujourd’hui d’être classés dans la catégorie des espèces en voie de disparition) exerçaient sur nous, et Titus en particulier, une emprise irrésistible, et dont la plastique assez sidérante (désolé de vous le dire mais il fallait bien que quelqu’un le fasse, et ce quelqu’un, en tant qu’auteur responsable, ne pouvait être que moi) faisait éclore en nous un florilège de sentiments dont le désir n’était pas totalement absent, et je dirais même, si je devais être parfaitement honnête (et, sur la vie de ma mère, Dieu m’est témoin que je m’y efforce autant que faire se peut), occupait une place de plus en plus prépondérante parmi nos préoccupations du moment (au moins les miennes, et, je pense pouvoir l’affirmer sans trop m’avancer, celles de ce cher Titus).

On en était donc là, disais-je, passablement englués dans les miasmes de l’expectative, quand le téléphone de Greg s’est mis à retentir avec une violence inouïe, nous faisant tous sursauter comme des lapereaux effarouchés (sauf Atiena, bien sûr, la gardienne de la nuit, entièrement focalisée sur l’objectif qu’elle s’était fixé de faire sortir Titus de la voiture pour l’emporter dans sa tanière et le sauver d’une mort certaine, que j’imaginais sans peine longue et douloureuse).

Greg, parti à la recherche de l’objet qui éructait dans le fond d’une de ses nombreuses poches (Greg adorait les poches, il en avait toujours plein sur lui, à commencer par un de ces affreux gilets multipoches style grand reporter ou aventurier qu’il portait jour et nuit hiver comme été, dans lequel on trouvait toujours au moins une brosse à dents, un paquet de chewing-gums sans sucre, quelques élastiques et bouts de ficelle, des capotes, de la menue monnaie, une boîte d’allumettes et un couteau suisse alors qu’il ne fumait pas et n’avait jamais mis les pieds à Lausanne, même pour aller planquer du fric à la Banque Cantonale Vaudoise) : Qu’est-ce que c’est encore que ça ?

Ça, c’était Sally Robinson, qui commençait à en avoir marre de faire le pied de grue sur le trottoir comme une vulgaire tapineuse devant le Sugar \& Spice.

\textsc{Greg} : Salut Sally. Ça va ?

\textsc{Sally} : Non, ça ne va pas. Qu’est-ce que vous foutez, bordel ? Ça fait une demi-heure que je vous attends !

\textsc{Greg} : Oui, je sais, je suis vraiment désolé.

\textsc{Sally} : J’en ai marre, moi ! Je n’arrête pas de faire aborder par des pervers qui essaient de m’embarquer.

\textsc{Greg} : Oui, en effet.

\textsc{Sally} : Quoi, en effet ?

\textsc{Greg} : En effet, ce sont vraiment des pervers.

\textsc{Sally} : Ça veut dire quoi, ça ?

\textsc{Greg} : Rien, juste que ça ne doit pas être marrant de se faire aborder sans arrêt par des pervers qui essaient de vous embarquer.

\textsc{Sally} : Non, en effet, c’est vraiment pas marrant. Vous faites quoi, en ce moment ?

\textsc{Greg} : On discute avec la gardienne de la nuit.

\textsc{Sally} : La quoi ?

\textsc{Greg} : La gardienne de la nuit, une fille qu’on a rencontré en chemin et dont on a un peu de mal à se débarrasser.

\textsc{Sally} : Vous vous foutez de moi ?

\textsc{Greg} : Pas du tout. Croyez bien que je suis le premier surpris de me retrouver dans une telle situation.

\textsc{Sally} : C’est qui, cette gardienne de la nuit ?

\textsc{Greg} : Une Black dans les vingt-trente ans, superbe, qui parle français comme une vache espagnole. On n’en sait pas plus, sinon qu’elle refuse de laisser partir Titus.

Atiena, gardienne de la nuit : TITUS PAS ALLER !!!

\textsc{Sally} : C’est qui, Titus ?

\textsc{Greg} : Un collègue, black lui aussi, costaud, venu nous filer un coup de main pour nettoyer le trou à rats que vous savez.

\textsc{Sally} : Vous êtes où ?

\textsc{Greg} : Pas très loin, rue des Nénuphars.

Ça faisait des lustres qu’on n’avait pas vu un nénuphar pousser dans le secteur, mais sans doute qu’il y en avait eu du temps où le quartier n’était qu’un vaste marécage infesté de malandrins et femmes de petite vertu. Aujourd’hui la rue n’était guère plus avenante mais nettement moins humide, même si les chats, les chiens et les noctambules avinés ne se privaient pas de pisser sous les portes cochères. Peut-être que la gardienne de la nuit était déjà là des siècles auparavant, se dressant devant le voyageur égaré pour lui interdire le passage et le sauver ainsi d’une mort certaine. Sa peau noire, ses yeux bridés et ses cheveux hirsutes devaient leur flanquer la trouille de leur vie, à une époque où on buvait de l’eau croupie et brûlait des sorcières à tous les coins de rues. La pauvre aurait fini sur le bûcher, sans aucun doute, pour apparence délictuelle et commerce avec le Diable. Comment avoir la peau aussi noire si on n’avait pas passé une bonne partie de sa vie dans les flammes de l’enfer ? Cette carbonisation et ces tatouages rituels qu’elle arborait un peu partout sur son corps magnifique, sans parler de ces breloques et autres amulettes suspectes qui tintaient à chacun de ses pas, étaient la preuve irréfutable d’une existence vouée au blasphème et la dépravation. Et si elle avait survécu jusqu’ici, renaissant sans cesse de ses cendres, c’était bien la preuve supplémentaire que Lucifer la considérait comme un de ses émissaires les plus efficaces sur terre et veillait sur elle comme sur la prunelle de ses yeux maléfiques.

\textsc{Sally} : Je connais, c’est à dix minutes à pied. Je viens vous rejoindre, j’en ai marre de faire du sur place.

\textsc{Greg} : Vous êtes sûr ?

\textsc{Sally} : Oui. Attendez-moi là-bas, j’arrive.

Greg, avant de raccrocher : Bon, comme vous voudrez.

\textsc{Moi} : Qu’est-ce qui se passe ?

\textsc{Greg} : Sally Robinson en a marre de poireauter, elle vient nous rejoindre ici.

Atiena, toujours en train de tirer sur Titus qui avait déjà un pied hors du van : TOI PAS ALLER, VENIR AVEC MOI !

J’avais le sentiment diffus, mais quand même assez preignant, que la situation était en train de se barrer en couille. Vous savez, et je dis ça en toute modestie, quand on a une certaine expérience du terrain comme c’est mon cas, on finit par sentir ces choses-là. Il y a une petite sonnette d’alarme qui est toujours à l’affût dans votre cerveau et se met en branle à la moindre anomalie. Et, depuis quelque temps, même si elles n’étaient pas aussi énormes qu’un troupeau de brontosaures en train de brouter (broutosaures) dans les vastes prairies du Crétacé, les anomalies semblaient s’accumuler avec un regain de vigueur inquiétant. Certains appellent ça l’instinct. Pour expliciter le concept, je vais reprendre l’exemple du brontosaure sus-cité. L’animal, énorme, donc, pait (et pète aussi, ce qui produit à chaque fois une déflagration de tremblement de terre) tranquillement dans les vastes prairies du Jurassique supérieur, indifférent à ce qui se passe autour de lui tellement son énormité lui tient lieu de protection, de garantie que nul ne viendra lui casser les couilles sous quelque prétexte aussi fallacieux que dérisoire que ce soit. Il pait, pète, et le temps s’écoule ainsi avec une lenteur toute préhistorique, tandis qu’il arrache des grosses touffes d’herbe bien grasse et juteuse avec sa minuscule tête accrochée au bout d’un cou interminable. On voit bien qu’il n’est pas d’une intelligence folle, qu’il ne sort pas de Cambridge, Oxford ou Harvard (on ne voit d’ailleurs pas très bien comment il aurait réussi à franchir la porte sans faire s’écrouler le bâtiment), mais il s’en branle, le brontosaure, il s’en secoue la nouille énergiquement, parce que quand on est aussi énorme que lui on n’a pas besoin d’être un fils de bonne famille ni de faire de longues études pour s’en sortir. Non, tout ce qu’on a à faire, c’est brouter et péter tranquillement dans les vastes prairies du Tithonien (d’après Tithon, le fils de Laomédon, enlevé par la déesse de l’Aurore avant de finir transformé en cigale), en se déplaçant le moins possible pour éviter de se fatiguer (parce que c’est pas forcément facile, même quand on est très très costaud, de véhiculer quinze tonnes et vingt mètres de barbaque sur quatre courtes pattes de tabouret), brouter encore et encore jusqu’à ce qu’il ne reste plus un seul brin d’herbe dans les vastes prairies du Jurassique supérieur, je vous parle de ça il y a au moins cent cinquante millions d’années. Donc le brontosaure il est là, en train de brouter tranquillement, se servant de son très long cou pour aller arracher des grosses touffes d’herbe au sommet des arbres ou au fond des précipices, lâchant des caisses monumentales qui font des trous énormes dans la couche d’ozone, et pis vla ti pas que soudain une petite cloche se met à tinter au fond de sa minuscule cervelle de sauropode de la formation de Morrison, bien avant Little Big Horn, le siège de Fort Alamo et la ruée vers l’or. À l’époque, il n’y avait pas l’ombre d’une plume de Sioux ou de Cheyenne dans les Grandes Plaines de l’Ouest, qui n’étaient d’ailleurs encore pas les Grandes Plaines de l’Ouest mais juste un trou à rats infesté d’une flopée de dinosaures dont certains avaient des dents aussi longues et tranchantes que des couteaux de chasse. Et parmi eux, il y avait toute une tripotée de grands théropodes carnivores, comme l’Allosaure ou le Mégalosaure, qui n’étaient pas des plus commodes, même s’ils n’étaient pas aussi énormes que notre pote brontosaure qui passait le plus clair de son temps à s’empiffrer dans ce qui deviendrait un jour, un bon paquet de millions d’années plus tard, le théâtre de l’une des plus fantastiques épopées de l’histoire coloniale du XIXe siècle, je veux bien sûr parler de la légendaire Conquête de l’Ouest. Alors certes, il était loin de se douter qu’un bipède du nom de Steven Spielberg tournerait un jour Jurassic Park dans ce qui n’était pas encore la Californie mais juste un trou à rats… enfin bref vous connaissez la suite, mais il était tout à fait capable de sentir que quelque chose ne tournait pas rond quand un de ces putains de grands théropodes carnivores, genre T-Rex, se pointait dans le secteur avec la ferme intention de se tailler un bon steak dans un gros cul de bronto. À ce moment-là, une petite sonnette se mettait à tinter dans son crâne de piaf, et il s’arrêtait aussitôt de brouter pour jeter un coup d’œil aux alentours, se servant de son cou immense comme d’une perche pour effectuer des moulinets avec sa tête et ratisser la zone à trois cent quatre-vingts degrés. Bon, évidemment, sa vitesse de déplacement ne lui permettait pas de prendre la fuite efficacement, d’autant que le Rex, tout en restant lui-même assez lourdingue, était capable d’accélérations non négligeables à défaut d’être foudroyantes. Pourquoi ? Eh bien mais tout simplement parce que dame Nature, toujours pétrie de bonnes intentions, avait jugé cocasse de le doter, en sa qualité de prédateur ultime, de pattes arrière extrêmement puissantes lui autorisant une certaine approche de la bipédie, en même temps que ses pattes avant, ainsi dégagées des contraintes de la quadrupédie, s’étaient vues recyclées en organes préhensiles armés de griffes tranchantes comme des lames de rasoir. Enfin, comme si ça ne suffisait pas, lorsqu’il s’agissait de s’attaquer à des proies disproportionnées, type bronto, Rex savait faire preuve d’un esprit de corps inhabituel, autrement dit faire appel à quelques uns de ses congénères les plus déterminés pour ne laisser aucune chance à la victime, quitte à devoir ensuite partager le butin à peu près équitablement (chose qui pouvait à l’occasion, ne nous voilons pas la face, occasionner quelques légères prises de bec). Bronto, face à une telle adversité, était condamné d’avance, sauf s’il parvenait, sur un malentendu, à transformer son agresseur en amas de viande sanguinolente en le piétinant sans ménagement. Mais la nature, même si elle le destinait à servir de pâture à ses plus sanguinaires rejetons (d’authentiques machines de guerre dépourvue de la moindre once de compassion), l’avait hypocritement équipé d’un avertisseur interne censé le prévenir du danger et lui offrir une chance infime de sauver sa peau.

Cet avertisseur, c’était l’instinct, et tous les animaux, y compris l’animal humain, en étaient pourvus. Chez nous, je pense qu’il est légèrement dévoyé, comme tout le reste, au point qu’il n’est pas rare que la victime imaginaire se transforme, à titre préventif dirons-nous, en agresseur bien réel. Le revers de la médaille, en quelque sorte.

Toujours est-il que quelques cent cinquante millions d’années plus tard, j’étais là, digne et alcoolisé (sinon alcoolique) représentant d’une espèce certes controversée mais ayant tout de même, au fil des millénaires, accompli quelques prouesses techniques qui laissaient loin derrière la concurrence, réduite à un rôle de figuration sur la scène du grand théâtre de l’évolution. J’étais là, et mon avertisseur interne me hurlait dans le creux de l’oreille que le costume que je croyais tiré à quatre épingle de la situation était en train de se détricoter lentement sous mes yeux incrédules.

Un des principes mêmes de la notion de naissance, voyez-vous, et c’est particulièrement vrai pour l’être humain (d’autres s’en sortent beaucoup mieux à ce niveau-là, même s’ils restent passablement démunis face à l’adversité), c’est de venir au monde dans un état de fragilité et de dépendance absolues. La nature, qui ne laisse au hasard que le strict nécessaire, et encore n’est-on pas sûr qu’il s’agisse réellement de hasard, a mis sur point un système de filiation qui protège le nouveau-né des appétits de sa mère, laquelle pourrait fort bien l’ingurgiter sur le champ (ce qu’elle s’abstient généralement de faire, à quelques rares exceptions près chez les coraux, les coquillages et les rongeurs), comme elle le fait, de nombreux cas en témoignent (mantes, araignées, scorpions, grenouilles, crabes, poissons, insectes), du mâle qui vient d’accomplir périlleuse besogne. En fait, au sein de cette noble maison, tout le monde est plus ou moins suceptible de dévorer tout le monde, qu’il s’agisse de son mari, ses parents ou ses enfants (et, pourquoi pas, se dévorer soi-même dans les cas les plus extrêmes, on parle alors de tendance suicidaires observées, comme par hasard, exclusivement chez l’être humain, en dehors de quelques fourmis et autres pucerons qui explosent spontanément dans le but de protéger la communauté, ce qui n’est pas le cas de l’être humain qui cherche le plus souvent à entraîner le plus de monde possible dans se chute). Certes, il existe des raisons de nature alimentaire ou sanitaire censée légitimer ces pratiques barbares, mais il apparaît clairement que la nature ne connaît aucune limite en matière de reproduction et n’hésite pas à ses livrer aux pires expérimentations (à l’image des «médecins» nazis de Dachau, Auschwitz, Ravensbrück et Buchenwald) pour optimiser son fonctionnement. En l’occurrence, l’expression aristotélicienne «la nature a horreur du vide» se vérifie une fois de plus par son absence totale de compassion envers ses sujets. Un petit animal aussi charmant que le Hamster doré, par exemple, met volontairement bas des portées excessives de rejetons à seule fin qu’une partie d’entre eux serve de casse-croûte à la mère, laquelle peut ainsi se nourrir tranquillement d’une partie de sa descendance tout en allaitant l’autre, ce qui tout de même assez abject quand on y pense (et je pense globalement que la nature est une des entreprises les plus abjectes qui aient jamais été conçues, laquelle, si elle n’est pas la sinistre conséquence de quelque fâcheux concours de circonstances, ne peut être que l’œuvre d’un dément, l’ultime éructation d’un cerveau malade dont on ne peut imaginer un seul instant qu’il appartienne à un quelconque parangon d’amour et de vertu). Comment, devant le spectacle pitoyable de ces mâles minuscules condamnés à s’approcher en catimini de l’objet de leur désir (en espérant qu’il dorme et prenant toutes les précautions pour ne surtout pas le réveiller), faire leur petite affaire en quatrième vitesse et le plus discrètement que possible (pas question de se mettre à hurler des insanités du genre «oh oui c’est bon», «t’aimes ça, salope» ou encore «tu la sens ma grosse queue» pour s’encourager à la manœuvre), avant de se barrer à toutes jambes (pattes) la goutte au gland et le slip sur les talons, comment, disais-je, devant un spectacle d’une telle affliction, ne pas évoquer les techniques de soumission chimique utilisées par les violeurs patentés. Ce n’est que face à une femme inconsciente (voire morte pour les plus méfiants, qui sacrifient du même coup tout espoir de descendance), qu’ils se sentent suffisamment à l’aise pour s’accoupler. Nul doute qu’ils appartiennent à cette catégorie de mâles minuscules (ici au sens psychologique du terme) contraints d’user des plus vils expédients pour arriver à leurs fins. On se doit bien évidemment, sur un plan juridique, de leur faire porter l’entière responsabilité de leurs actes (il s’agit de fixer des limites pour éviter au monde des perspectives post-apocalyptique peu réjouissantes, d’établir des règles pour cartographier les vastes territoires de la perversité humaine), mais il ne trompe personne qu’ils ne sont que des marionnettes dont la nature, confortablement installée au plus profond du génome, tire sournoisement les ficelles.

Naturellement, je vous livre toutes ces pensées après coup, car vous pensez bien que dans le feu de l’action j’avais d’autres chattes à fouetter que me livrer à ce genre de considérations sur la triste réalité de la condition humaine.

En tant qu’organisateur de l’événement, je me devais de prendre de toute urgence la décision qui s’imposait, décision qui, je dois bien le dire, ne s’imposait pas clairement à mon esprit, au point que j’envisageais de plus en plus de rentrer chez moi et déléguer mes pouvoirs à quelqu’un de mieux informé.

«TITUS PAS ALLER, VENIR AVEC MOI !!!!!!!!!»

Ces paroles, répétées à l’envi par la sexuellement très pertinente Atiena, sorte de clocharde céleste surgie des profondeurs de la nuit, érodaient inexorablement le peu de volonté et les maigres ressources intellectuelles qui s’accrochaient encore à mes neurones en perdition.

Pendant ce temps, Sam et Nathan commençaient à s’exciter. Ces deux va-t-en-guerre, pressés d’en découdre, trouvaient que la plaisanterie n’avait que trop duré. J’étais, en mon for intérieur, tenté de leur donner raison, tant la situation était en train de tourner au vaudeville le plus ahurissant. Je sentais également, en ce même for intérieur, que mon collègue et indéfectible ami Titus Beaugendre, ébranlé par l’intervention de cette sorcière aux yeux vert céladon (nom du berger forézien héros de L’Astrée d’Honoré d’Urfé, comte de Châteauneuf, marquis du Valromey et seigneur de Virieu-le-Grand, écrivain à ses heures à qui l’on doit également quelques épîtres morales et paraphrases sur les cantiques de Salomon de moindre intérêt), ne trouvait plus aucun charme à notre compagnie. Greg, quant à lui, n’était pas du genre à foncer tête baissée dans les emmerdements. Il prenait tout son temps pour observer les choses à distance, et ne s’engageait pleinement dans la bataille que lorsqu’il avait acquis la certitude d’obtenir gain de cause. L’épisode Yiorgos Panayiotou (la balle dans le bras, le bijoutier reconnaissant et les vacances à Zanzibar) n’avait fait que le renforcer dans l’idée qu’il valait mieux tourner dix fois sa langue dans sa bouche avant de l’ouvrir pour raconter n’importe quoi. Contrairement à la plupart d’entre nous, il ne parlait que quand il avait quelque chose à dire, c’est à dire pratiquement jamais, ce qui était très reposant pour tout le monde.

Cent cinquante millions d’années plus tard, donc, soit bien après la grande extinction du Crétacé, laquelle, en plus d’avoir exterminé la quasi totalité des dinosaures (il semble que seules quelques espèces de théropodes à plumes aient survécu), avait également causé la perte des ammonites après trois cent cinquante millions d’années de bons et loyaux services au sein des océans (les pauvres, victimes elles aussi des expérimentations stylistiques de dame Nature, sont passées par toutes les configurations possibles au niveau de leur coquille, la spirale classique restant quand même la plus stable et populaire), je me trouvais là, au beau milieu de la rue, en pleine nuit, sans trop savoir sur quel pied danser (je danse très mal, du reste, ce qui fait que je ne danse pas du tout parce que je n’ai pas envie de passer pour un clown).

Depuis un bon quart d’heure maintenant, la Gardienne de la Nuit essayait vainement de tirer Titus hors du van, mais je ne doutais pas un instant qu’elle finirait par arriver à ses fins. C’était écrit dans le Grand Livre Noir de la Nuit éternelle et autres étendues infinies du Cosmos. Titus était mon bras droit, ma couille gauche, mon troisième œil ou tout ce que vous voudrez, de sorte que, rempli d’une détresse poisseuse qui m’engluait jusqu’aux ouïes, je ne me voyais aucunement aller au combat sans son appui technique et logistique. Non seulement je ne me voyais pas y aller sans lui, mais je me voyais de moins en moins y aller tout court, tant la soirée avait pris un tour inhabituel qui ne présageait rien de bon.

Je vous passe les détails, mais Titus, comme je m’y attendais avec son cerveau de limace et sa volonté de poisson rouge, a fini par répondre aux injonctions de la Gardienne de la Nuit.

Le malheureux avait les yeux exorbités et ne parvenait plus à détacher son regard de celui d’Atiena, particulièrement fascinant il faut bien le dire. J’avais moi-même, à titre d’indication, les plus grandes difficultés à détacher mon regard de son anatomie callypige de Vénus hottentote, sans commune mesure toutefois avec le postérieur monumental de Saartje «Swatchie» Baartman, cette Khoïsan exhibée sur toutes les scènes du monde à l’époque des grandes expositions coloniales, assimilée par Cuvier et ses contemporains à un singe humain proche de la monstruosité, un phénomène de foire qui attisait le sentiment de supériorité et la sexualité trouble des visiteurs. Ce brave Cuvier, disciple de la malédiction de Canaan et la Table des peuples (Genèse 9:18-29) mais guidé par un indomptable esprit scientifique et un sens de l’éthique inébranlable, avait même jugé pertinent de conserver dans le formol quelques morceaux choisis de son anatomie, notamment son cerveau et ses organes génitaux. Deux siècles plus tard, au début des années 2000 et sous la pression de Mendela, ces «pièces de musée» et le squelette de Swatchie seront restituées à l’Afrique du Sud dans le cadre de la vaste campagne de repentir colonial orchestrée par l’UNESCO, au grand dam des pilleurs de tombes et autres receleurs étatiques du butin esclavagiste. Depuis, tout un tas de reliques, crânes, momies et organes en tout genre ont repris le chemin de leurs pays d’origine, à commencer par El Negro de Banyoles, ce guerrier du Botswana empaillé en toute discrétion par les frères Verreaux (et véreux) en 1831 et revendu quelques années plus tard à un certain Francesc Darder i Llimona, vétérinaire et médecin catalan grand amateur d’objets scabreux et auteur d’un très recherché Manuel pratique pour l’élevage des oies (à noter qu’il s’intéressait aussi à la reproduction des truites, j’en veux pour preuve sa conférence lors de la fête du poisson de Ripoll restée dans toutes les mémoires), et longtemps exposé au Museu Darder de Banyoles en compagnie d’autres articles de choix tels que fœtus, têtes réduites et peaux humaines.

Dieu sait comment, à moins que ce ne soit le diable en personne, Atiena avait réussi à prendre le contrôle de l’esprit de Titus (ce qui, je vous le concède, était plus ou moins à la portée du premier venu). Certains parasites microscopiques, des vers notamment, qui se développent à l’intérieur d’un hôte spécifique, sont capables de ce genre de prouesse. Ainsi le spinochorde, qui ne peut se reproduire que dans l’eau, prend le contrôle du cerveau de la sauterelle et la force à se noyer le moment venu. De façon tout aussi sournoise, le toxoplasme obéit à un cycle reproductif qui nécessite la mobilisation d’un hôte intermédiaire, tel que l’oiseau ou le rongeur, avant de passer à l’hôte définitif qui est le chat. C’est ainsi, le moment venu, qu’il prend le contrôle du cerveau de son hôte intermédiaire pour le pousser à finir entre les griffes du chat, au lieu de le fuir comme son instinct lui commande de le faire. Des chimpanzés, infectés par ce même cheval de Troie cellulaire, ont vu leur libido perturbée au point d’être attirés sexuellement par les léopards, chose qui réduit évidemment drastiquement leur espérance de vie, le léopard ne mangeant pas de ce pain-là (mais ayant par ailleurs un goût prononcé pour la viande de chimpanzé). On n’a, à ma connaissance, pas encore observé de cas probant de membre d’une quelconque tribu africaine qui serait allé volontairement à la rencontre d’un lion pour lui rouler une pelle. Cela dit, les contaminations humaines ne sont pas rares et peuvent conduire à de sérieuses complications telles que malaises, céphalées, myalgies, désordre mental, hallucinations, convulsions et parfois même coma. Pour la petite histoire, il semblerait également qu’un accroissement substantiel des sécrétions hormonales rendent les sujets infectés nettement plus attirants que les autres, mais les études conduites en ce sens, de nature à renvoyer définitivement les inhibiteurs de la phosphodiestérase de type 5 au rang de remède de bonne femme, en sont encore au stade embryonnaire).

De toute évidence, Atiena appartenait à cette catégorie de parasite capable de manipuler mentalement sa proie.

J’ai pensé un instant à lui tirer une ou deux balles dans la tête, je ne vous le cache pas, et puis je me suis dit que ce serait peut-être un peu disproportionné, un manque évident de savoir-vivre (mais le savoir-vivre des uns passe par le savoir-crever-la-gueule-ouverte des autres, non, vous ne croyez pas ?), et surtout qu’il serait quand même dommage de faire exploser bêtement la cervelle d’une créature en laquelle la nature avait manifestement investi sans compter une bonne partie de ses ressources.

\textsc{Sam} : Il faut faire quelque chose !

\textsc{Moi} : Je suis bien d’accord. Mais quoi ? Titus ?

Pas de réponse.

\textsc{Greg} : Je sais pas vous, mais moi je la sens de moins en moins, cette petite virée.

\textsc{Nathan} : On n’a qu’à le laisser là et continuer sans lui.

\textsc{Moi} : Sans Titus ? Hors de question !

\textsc{Sam} : M’est avis que cette chose le tient en son pouvoir.

\textsc{Atiena} : Moi pas chose. Moi Gardienne de la Nuit.

Je suppose que rares sont celles et ceux parmi vous qui ont entendu parler de ce bon vieux Jim Bowie, un type auquel il valait mieux éviter de chercher des crosses. Après une brillante carrière de trafiquant d’esclaves en Louisiane, il a rencontré Dieu au détour d’un verre de bourbon et décidé de se racheter une conduite en s’engageant dans l’armée pour lutter contre le général Antonio Lopez de Santa Anna y Perez de Lebron, le Napoléon du Nouveau Monde, héros immortel de Zempoala et gouverneur de Veracruz. Courageuse attitude qui lui a valu de trouver la mort pendant le siège de Fort Alamo, en compagnie d’autres valeureux combattants comme Bill Travis et surtout un certain Davy Crockett, ancien trappeur du Tennessee connu pour être un type cool épris de justice et proche des Amérindiens. Mais ce qui a fait la réputation de Jim Bowie, c’est qu’il adorait jouer du couteau. Et pas n’importe lequel. Lui, ce qu’il appréciait particulièrement, c’était le genre de bibelot avec lequel vous pouvez facilement décapiter un sanglier d’une main tout en vous grattant les couilles de l’autre. Il n’est pas, techniquement, l’inventeur du couteau qui porte son nom, le couteau Bowie, mais s’est découvert une telle passion pour lui, un amour à ce point fusionnel qu’il est désormais impossible de les dissocier l’un de l’autre. Jim en avait toujours un sur lui, et s’en servait aussi bien pour couper sa viande, vider un ours que découper ses ennemis en rondelles. C’est d’ailleurs aussi parce qu’il avait l’habitude de se curer les dents avec que l’objet a hérité de son surnom chantant de «cure-dent de l’Arkansas», même s’il fallait un sacré coup de main pour ne pas se couper la langue par la même occasion. Le Bowie est ce qu’on appelle un couteau de survie, la bonne à tout faire des objets tranchants, le compagnon idéal des virées en solitaire dans la jungle ou les montagnes infestées de Peaux-rouges (je parle d’avant les guerres indiennes, le génocide et les réserves, bien sûr, parce que maintenant, à part Donald Trump, je ne pense pas qu’il reste beaucoup de Peaux-rouges réellement dangereux aux USA), le chaînon manquant entre le couteau de poche et le sabre d’abordage. Il suffit qu’il voie sa lame affûtée étinceler sous le chaud soleil des Tropiques pour que le jaguar qui s’apprêtait à vous sauter dessus regagne sa tanière la queue entre les pattes en poussant des petits couinements de chaton effarouché. Si tu aimes tuer des gens et adores par dessus tout les voir se tordre de douleur pendant que tu leur tournes et leur retournes la lame dans le bide, alors c’est l’engin qu’il te faut. Il se glisse dans la main avec facilité, je dirais presque gourmandise, et ne fait aucune difficulté pour traverser le cuir épais d’un tapir ou un caïman. L’essayer c’est l’adopter, et une fois que vous aurez passé votre pouce sur le tranchant de sa lame, taillé votre premier morceau de bois et éviscéré votre premier arapaïma avec lui, après avoir pris un bain dans les eaux troubles de l’Amazone et échappé de justesse aux dents acérées du poisson vampire et du piranha à ventre rouge, alors vous ne pourrez plus jamais vous en passer. Vivre sans lui n’aura plus de sens, et toutes vos facultés intellectuelles seront mobilisées par un seul et unique objectif : vous en servir sans cesse pour couper, trancher, élaguer, tailler, sectionner, tronçonner, débiter, amputer, ouvrir, inciser, mutiler tout ce qui vous passe à portée de main.

C’est exactement ce qui arrivé à Bobby Beausoleil, en ce dimanche 27 juillet 1969, quand il a étripé Gary Hinman à Old Topanga Canyon, dans le comté de Los Angeles, avec son bowie mexicain. L’endroit, à l’époque, grouillait de chevelus défoncés à la mescaline et au LSD. Les produits naturels, issus de pratiques ancestrales basées sur une connaissance millénaire de l’environnement, étaient privilégiés. Les hippies n’avaient pas envie de bosser. Ils se foutaient du capitalisme qui avilissait l’homme et le rendait hermétique à sa propre existence. Ils étaient jeunes, beaux (pas tous, mais on s’en foutait parce que la beauté n’était pas un critère de sélection, même les moches avaient le droit de vivre), ils avaient envie de se balader à poil, sans entrave, parce que toutes ces conneries de fringues c’était bon pour les bourgeois qui passaient leur vie à grelotter de trouille et se planquer derrière des écrans de fumée. Les bourgeois corrompus vivaient dans le mensonge et n’hésitaient pas à offrir leurs propres enfants en sacrifice à Mammon. Les hippies avaient envie de profiter de la vie, cramer leur jeunesse par les deux bouts, mouiller, bander sans arrêt, éjaculer à tout-va, danser sous la lune, s’accoupler et faire leurs besoins sans honte ni retenue devant les autres. Ils n’avaient pas envie de vieillir pour ressembler à ces vieux cons pétris de certitudes et rongés par l’amertume qui se claquemuraient derrière leurs portes fermées à double tour et dormaient avec un flingue sous l’oreiller. Ces vieux singes vérolés avaient tout donné au Fric et le Fric ne leur avait rendu que des miettes de bonheur tiédasse noyées au sein d’un océan de servitude et de désillusions. Les hippies voulaient vivre aux quatre vents, sans barrières ni limites, revendiquaient le libre usufruit d’une terre qui appartenait à tout le monde mais n’était la propriété de personne, et surtout pas d’une poignée d’autocrates névrosés qui s’arrogeaient le droit de vie et de mort sur des populations entières. Les gens n’avaient pas de comptes à rendre, de papiers d’identité à présenter, de garanties à fournir pour être acceptés à la table commune. Ils n’avaient pas besoin de s’excuser d’être en vie, prêter allégeance à un quelconque pouvoir, être validés par un comité de censeurs grisonnants au cœur sec.

Bobby Beausoleil était un de ces hippies défoncés à la mescaline qui se baladaient à moitié à poil, une guitare en bandoulière, dans les montagnes de Santa Monica. Sans domicile fixe, il avait trouvé refuge chez Gary Hinman, un prof de musique à la cool qui aimait dépanner les petits jeunes dans le besoin. Bobby était du genre beau gosse pas farouche, on pouvait lui mettre la main au paquet sans qu’il fonce porter plainte au commissariat du coin, qui l’aurait de toute façon envoyé bouler comme une merde. Le soleil tapait fort, ce jour-là, sur sa petite tête de piaf, et son Bowie chicano commençait à se sentir à l’étroit dans son étui en cuir de buffle. Quelque temps plus tôt, alors qu’il se baladait pieds nus dans les montagnes de Santa Monica (ce qui n’est pas très malin parce que les crotales ne sont pas rares dans le secteur), sa guitare en bandoulière et son petit paquet de couilles moulé à la louche dans son microshort en lycra vert fluo, Bobby s’était retrouvé du côté de Santa Susana Pass Road, au-dessus de Chatsworth, dans ce qu’on a coutume d’appeler la Vallée de San Fernando depuis que les Espagnols l’ont piquée aux indiens. C’est là que se trouvait le Spahn Movie Ranch, un site de tournage désaffecté squatté par un certain Charles Manson et sa petite famille, des filles surtout, dont la plupart se prostituaient pour faire tourner la boutique, quand elles n’étaient pas occupées à voler ou faire les poubelles de supermarchés. Les bourgeois étaient tous des porcs qui puaient la pisse et ne pensaient qu’à se taper des petits culs roses et frais. Ils profitaient de la misère humaine pour assouvir leurs plus bas instincts, avec la bénédiction des autorités qui leur mangeaient dans la main. Mais un jour ou l’autre, peut-être plus proche qu’il y paraissait, le moment serait venu de régler la note, et celle-ci risquait d’être un peu plus salée que ce qu’ils avaient imaginé. Comme Bobby, Charlie jouait de la gratte et poussait la chansonnette pour dénoncer les dérives de la société et mettre en garde l’humanité - et notamment ces enfoirés de rednecks consanguins qui tiraient sur tout ce qui bouge - sur les conséquences de ses actes. Un jour ça allait péter, de la viande froide allait être expédiée aux quatre coins de l’univers, les flammes allaient ravager le monde, les oiseaux prendre feu dans le ciel, et seuls une poignée de disciples triés sur le volet survivraient pour reconstruire un monde nouveau, meilleur et plus juste. Charlie avait un regard intense, avec des yeux comme des braises qui vous foutaient la cervelle en ébullition. Quand il parlait, les gens l’écoutaient religieusement et se sentaient aussitôt transportés vers des horizons insoupçonnés. Il a proposé à Bobby de venir s’installer chez lui, Bobby a accepté. Bobby «belle gueule» Beausoleil s’est tout de suite très bien entendu avec tout le monde, les filles en particulier, Mary, Linda, Gipsy, Brenda, Susan, Squeaky (ancienne du Girls Athletic Club de Santa Monica et chasse gardée de Charlie, c’est elle qui avait le privilège de bichonner le vieux George Spahn, le propriétaire du ranch, alors âgé de quatre-vints piges, pratiquement aveugle et physiquement en état de décomposition avancée, mais toujours prêt à accorder des réductions de crédit quand on lui tripotait la nouille, autant dire que vivre entouré de ces petites chattes étaient pour lui une véritable bénédiction), Leslie, Snake, Sandy, Sherry, Pat et les autres. C’est qu’il y en avait du monde, là-dedans, et c’était que des gens jeunes et beaux animés des meilleurs intentions, une vraie petite famille baignant dans l’amour et la compassion comme on aimerait en voir plus souvent. De l’avis général, Bobby n’était pas seulement beau gosse, c’était aussi un super musicien, un authentique artiste, et tous les producteurs auraient dû se bousculer au portillon pour lui faire signer des contrats mirobolants. Même chose pour Charlie, bien sûr, dont les textes étaient d’une profondeur inégalée, disant tout le mal-être d’une jeunesse en perdition dans un monde gangrené par l’injustice, la bêtise et la cupidité. Et justement, en parlant de fric, il se trouve que Gary Hinman, le prof de musique à la cool qui dealait de la drogue chez qui Bobby avait séjourné un certain temps, avait semble-t-il un joli bas de laine planqué sous son matelas. En plus, Bobby n’était pas content parce que Gary l’avait arnaqué en lui vendant ce qui était soi-disant de la mescaline de première bourre et n’était en fait que de la strychnine bidouillée, le genre de cocktail qui pouvait vous exploser à la gueule à tout moment. Il l’a dit à Charlie, et Charlie, qui détestait se faire arnaquer, et ce d’autant plus qu’il avait l’impression de se faire arnaquer depuis son plus jeune âge (fils d’un père qu’il n’a jamais connu et d’une mère alcoolique et prostituée qui passe la moitié de son temps en taule, il est élevé par un oncle et une tante qui le tabassent et abusent de lui sexuellement, alors que d’autres sont choyés par des parents aimants, pètent dans la soie et bouffent avec des couverts en argent), est parti en sucette sur les chapeaux de roues. Bobby prêchait l’amour et la non-violence, comme tout bon hippie qui se respecte. Il voulait juste qu’on le laisse faire de la musique, vivre d’amour et d’eau fraîche, de chasse et de cueillette, comme ses ancêtres avant lui, et leurs ancêtres avant eux aussi loin qu’on puisse remonter dans les méandres de l’histoire et la préhistoire. Une vie simple, basique, axée principalement sur les ressources naturelles et l’exercice de ses droits les plus élémentaires dans le strict respect de ceux d’autrui. Charlie aussi, mais il prêchait surtout (à son corps défendant, car il n’était au fond de lui-même qu’amour et bonté, tendresse filiale détournée du droit chemin par des années de mauvais traitements) la violence et la haine comme des maux nécessaires pour se frayer un chemin jusqu’à Zion, la terre promise, pendant lumineux de la sombre Babylone. Ils sont allés voir Gary, qui jouait de la flûte dans son salon en fumant un joint (pas en même temps), et Charlie, après avoir menacé de lui enfoncer sa putain de flûte dans le fion jusqu’à la luette, lui a flanqué un coup de sabre de samouraï qui lui a arraché la moitié d’une oreille. Gary s’est mis à couiner comme un goret, disant qu’il n’était pour rien dans cette histoire de came, et Charlie lui a dit qu’il ferait bien de les rembourser vite fait s’il ne tenait pas à se faire hacher menu. Il avait appris de source sûre qu’il planquait un petit héritage d’au moins vingt mille dollars chez lui, et Charlie exigeait le pactole en échange du préjudice subi, tant sur le plan moral que physique, cette came frelatée ayant très bien pu les envoyer ad patres, chose qui ne s’était heureusement pas produite mais laissait quand même des traces dans le corps et l’esprit de celles et ceux qui étaient passés aussi près de l’annihilation pure et simple de leur identité, la négation de leur être. Bobby, qui n’était pas le mauvais bougre, a suggéré à Charlie de rentrer au ranch pour essayer de se calmer un peu, la colère n’étant pas bonne conseillère, tandis que lui-même, Susan (alias Sadie Mae Glutz, ancienne danseuse nue à San Francisco, future meurtrière de Sharon Tate enceinte de huit mois et demi) et Mary (Brunner, alias Mother Mary, la taulière, la première victime de Charlie, son âme damnée, la pierre fondatrice de son édifice criminel qui n’aura de cesse, par la suite, de l’aider à agrandir sa «famille» et pousser des jeunes femmes dans son lit, l’équivalent, si vous voulez, d’une Monique Olivier pour un Michel Fourniret), resteraient avec Gary pour discuter tranquillement le coup et tenter de trouver une issue à cette affaire qui soit honorable et satisfaisante pour tout un chacun. En définitive, la chose ne serait pas si grave si Bobby n’avait revendu une partie de la dope frelatée de Gary aux Straight Satans, un gang de bikers déjantés avec lesquels il venait mieux entretenir des relations de bon voisinage. Maintenant ils en avaient après lui et il allait passer un sale quart d’heure s’il ne les remboursait pas illico de l’argent indûment investi. Les Straight Satans avaient le crime dans le sang et ils en avaient dessoudés pour moins que ça. En plus ils travaillaient salement, à l’ancienne, et les souffrances endurées étaient proportionnelles à l’embonpoint de la créance.

Finalement, la discussion a tourné au vinaigre. Pendant trois jours, Gary a été torturé par la joyeuse bande pour savoir où il cachait son fric. Mais, d’après ses dires, il était fauché comme les blés. Gary pissait le sang et voulait voir un médecin. Quand il a vu que Gary ne craquerait pas, Bobby a changé son fusil d’épaule. Il avait remarqué les deux bagnoles garées dans la cour, et il a exigé que Gary les lui donne en échange de sa dette. Quand l’affaire a été réglée, sachant pertinemment que Gary irait le balancer à la première occasion, Bobby l’a planté deux fois dans le thorax avec son Bowie mexicain. Après quoi la petite bande l’a regardé agoniser en sirotant des bières avant d’écrire des cochoncetés sur les murs avec son sang, du genre MORT AUX PORCS, et de laisser des indices foireux pour faire croire que les Black Panthers étaient dans le coup.

Bon, eh bien quelques années plus tard, on retrouve le même genre de coupe-chou entre les mains de John Rambo (interprété par Stallone après avoir été refusé par Dustin Hoffman, Al Pacino, Robert de Niro, Steve McQueen et Clint Eastwood, entre autres), ancien béret vert et héros de la guerre du Vietnam qui éprouve quelques difficultés à se réinsérer dans la vie civile. Non de son fait, car il est plein de bonne volonté, mais à cause de son regard de tueur et son allure générale qui ne plaident pas en sa faveur. Il suffit qu’il débarque dans un trou perdu au milieu des montagnes pour qu’un gros con de shérif redneck lui tombe dessus et le prie de quitter les lieux sans demander son reste. Rambo estime qu’on n’a pas à le traiter de cette façon, surtout après qu’il ait risqué sa vie dans les rizières pour sauver le pays des griffes du Vietcong et porter au plus haut les valeurs du capitalisme et de l’impérialisme américain. Rambo a vu ses compagnons d’armes mourir les uns après les autres, dans des conditions atroces, rongés par la vermine et la pourriture dans la chaleur étouffante de la forêt primaire, et s’il s’en est sorti, c’est grâce à son courage, bien sûr, mais aussi aux techniques de survie et de combat que lui a enseignées le colonel Trautman, son mentor à Fort Bragg, en Caroline du Nord (ainsi nommé en l’honneur du général de division Braxton Bragg, brute esclavagiste et commandant de l’armée du Tennessee, défaite par Ulysses Grant lors de la troisième bataille de Chattanooga). Bref, Rambo fait de la résistance, le shérif le boucle pour vagabondage et détention d’une arme de catégorie D de grande taille, et les flics beaufs et racistes du coin redoublent d’efforts pour lui faire subir les pires humiliations. Sauf qu’ils ne savent pas qui est Rambo, et surtout ils ne savent pas ce qui se passe quand Rambo pète un plomb. Parce que Rambo, c’est pas qu’il est complètement taré ou quoi, mais quand même, après ce qu’il a enduré au Vietnam, les horreurs qu’il a vues (ses potes les tripes à l’air, la langue arrachée, le visage en bouillie, les vers qui grouillent dans les plaies, les araignées et les serpents qui tombent des arbres, la puanteur et la crasse, les Viets qui s’amusent à vous couper les couilles, vous arracher les ongles et vous crever les yeux, les corps déchiquetés par les mines), disons que tout ne tourne pas toujours très rond dans sa petite tête et qu’il vaut mieux éviter de le pousser dans ses derniers retranchements. Sinon, l’animal traqué se transforme en prédateur impitoyable, prêt à tout pour sauver sa peau, le chasseur devient la proie et ses chances de survie en milieu hostile sont pratiquement inexistantes. C’est ce que le shérif Will Teasle (excellent Brian Dennehy) et sa bande de ploucs en uniforme ne vont pas tarder à apprendre à leurs dépens. Ils vont se lancer à la poursuite de Rambo dans la forêt et en prendre plein la gueule pour pas un rond. Le sergent Arthur Galt, à bord d’un hélico qui survole la montagne, entreprend de dézinguer Rambo accroché à la paroi à coups de carabine, ignorant les ordres de Teasle qui exige qu’on le prenne vivant. Il le rate à plusieurs reprises, mais Rambo finit par dégringoler dans le ravin et s’ouvre le bras en tentant de se raccrocher à un sapin. Déjà qu’il n’était pas très satisfait de la tournure prise par les événements, ce fâcheux contretemps est l’élément déclencheur d’un accès de fureur difficilement contrôlable. N’oublions pas que Rambo est un être frustre (aux antipodes de l’intellectuel de gauche qui écrit des essais, donne des conférences dans les grandes universités et passe à la télé pour débattre sur les grands sujets de société de son époque, sous la houlette de journalistes onctueux confortablement installés dans des fauteuils de velours) dont les nerfs ont été mis à rude épreuve par les années passées sous les drapeaux. Je ne dis pas qu’il est totalement stupide, non, loin de là, mais simplement que ses réactions ne sont pas celles d’un homme cultivé qui pèse longuement le pour et le contre avant de se lancer dans la destruction méthodique de son prochain. Sous le coup d’une émotion qu’il n’arrive pas à contenir, Rambo se saisit d’une grosse pierre et la balance sur l’hélico qui continue à lui tourner autour comme une saleté de moustique géant. La pierre atterrit dans le pare-brise de l’hélico qui fait une embardée. Galt, qui tirait depuis le bord, perd l’équilibre, tombe à son tour dans le ravin et se fracasse mortellement la tronche sur les rochers tranchants comme des éclats de verre. Teasle, du haut de la falaise, assiste impuissant à la mort de son collègue et néanmoins ami, même si Galt était quand même une sacrée tête de nœud (vous savez ce que c’est, les gens ont beau être des abrutis de la pire espèce, il reste toujours une part d’humanité en eux, même factice, qui fait qu’on hésite à les exterminer comme des cafards). Suite à cet incident désastreux, une chasse à l’homme sans merci s’engage dans la montagne. Rambo, habitué à évoluer dans la jungle et tirer parti des éléments, n’éprouve aucune difficulté à se débarrasser un par un des hommes du shérif. Désireux de ne pas envenimer la situation, il ne les tue pas mais se contente de leur infliger quelques blessures et les avertir qu’ils s’exposent au pire s’ils continuent à lui coller au train. Lui, tout ce qu’il veut, c’est qu’on lui foute la paix. Mais les flics ne l’entendent pas de cette oreille, même quand le colonel Trautman débarque pour leur dire que c’est lui, Samuel Trautman, colonel de son état et formateur de bérets verts à Fort Bragg, qui a formé, conçu Rambo, qu’il en a fait une véritable machine de guerre, une arme de destruction massive rompue aux techniques les plus extrêmes en matière de guérilla et de combat rapproché, et qu’ils n’ont aucune chance de s’en sortir en l’affrontant sur son terrain. Le mieux est de faire semblant de lâcher le morceau et lui tomber dessus en douceur quand il se décidera à sortir de son trou. Naturellement, Teasle ne veut rien entendre. Il continue à s’acharner sur Rambo, lequel entre dans une fureur noire et met la ville à feu et à sang. À la fin, Trautman récupère son poulain et ils partent ensemble sous le soleil couchant pour aller effectuer d’autres missions périlleuses au service de la nation. Rambo, avec son regard de chien mouillé, encore traumatisé par les horreurs de la guerre, pleure sur l’épaule paternelle de Trautman comme un gros bébé musclé avec un cerveau de poule, d’une naïveté déconcertante qui émeut même les cœurs les plus secs. Le spectateur médusé comprend alors que Rambo n’est qu’un gosse des rues perdu dans un monde trop grand pour lui, exploité par un gouvernement cruel et sans pitié, et finalement rendu à ses semblables qui le recrachent comme un vieux bout de viande périmée coincé entre deux chicots pourris. Ses seuls amis s’appellent Randall 18, Hoyt Spectra, Martin Cougar, Saco M60, Heckler \& Koch MP5A3, Dragunov SVD-63, Gil Hibben IV, Browning M2 et Winchester modèle 1892, une référence absolue dans le monde merveilleux des armes à feu qui fascine petits et grands depuis toujours (à titre d’exemple, John Wayne, membre de la Marion McDaniel Lodge 56 de Tucson, médaille d’or du Congrès et autre référence absolue de l’american way of life dans tout ce qu’elle a de plus frétillant, l’utilise abondamment et avec une jouissance manifeste dans La Prisonnière du désert, notamment pour tirer Debbie Edwards - Natalie Wood dans un de ses rôles les plus sexy - des griffes de Relampago, le chef des Comanches qui l’a enlevée alors qu’elle n’était encore qu’une enfant et élevée à la sauce indienne).

Si je prends la peine de vous raconter tout ça, c’est parce que c’est précisément une réplique exacte du Randall 18 de Rambo que Sam trimballait dans un étui en cuir de buffle accroché à sa ceinture, réplique qu’il a sortie d’un geste brusque et pointée en direction d’Atiena en prononçant ces paroles dont l’extrême indignité me révulse aujourd’hui encore, bien longtemps après les faits, au plus profond de moi-même : GARDIENNE DE MON CUL, OUI !!!!!!

C’était d’une vulgarité insupportable, laquelle, je dois le dire, m’a profondément choqué sur le coup. Et puis, finalement, après quelques semaines de soins intensifs dans une unité psychiatrique spécialisée dans la remise à niveau des militaires polytraumatisés par les horreurs de la guerre (je parle de gars des forces spéciales ayant écumé les pires zones de combat de la planète et affronté des terroristes prêts à toutes les ignominies pour se débarrasser d’eux), j’ai réussi à sortir du marasme et me réadapter peu ou prou à la vie civile, même s’il m’arrivait encore de sentir monter en moi des pulsions destructrices difficilement contrôlables.

Moi, sur un ton extrêmement réprobatif : Sam, non !

\textsc{Lui} : Quoi, non ?

\textsc{Moi} : Range-ça tout de suite, tu veux !

Lui, les yeux injectés de sang, les traits déformés par la haine au point que je ne suis même pas certain que je l’aurais reconnu si je l’avais croisé par hasard dans la rue : M’en vais lui tailler les oreilles en pointe, moi, à cette salope !

\textsc{Atiena} : Moi pas salope. Toi gros porc.

Greg, s’allumant une Alain Delon (The Taste of France, il s’en faisait régulièrement expédier de Phnom Penh par un membre de sa famille qui tenait un restaurant huppé en plein cœur de celle qu’on appelait autrefois La Perle de l’Asie, chose que je désapprouvais au plus haut point pour au moins deux raisons : d’une part je trouve que la clope c’est de la merde, d’autre part j’ai toujours pensé qu’Alain Delon était un acteur médiocre qui ne devait sa carrière qu’à son physique avantageux) à la flamme d’un briquet jetable d’une marque bien connue que je ne citerai pas (sauf, bien sûr, en échange d’une rétribution conséquente) : Quelque chose me dit qu’on n’est pas sorti de l’auberge.

Sam (qui, maintenant que j’y pense, avait à peu près la même tête de basset que Stallone dans First Blood, cette même tête ayant, avec l’âge et le botox, changé radicalement de physionomie pour ressembler de plus en plus à celle d’un bulldog amateur de cigares) : Vous avez entendu ça, les gars ?

\textsc{Greg} : C’est vrai que t’es un gros porc.

\textsc{Sam} : Moi, je suis un gros porc ?

\textsc{Greg} : Oui. Tu passes ton temps à roter, péter, dire des gros mots, et en plus tu manges comme un cochon.

Sam, se tournant vers moi : Djef, je te rappelle que c’est toi le chef de cette opération.

\textsc{Moi} : Et alors ?

\textsc{Sam} : Alors c’est à toi de remettre de l’ordre dans la discussion.

\textsc{Moi} : Quelle discussion ? Je t’ai demandé de ranger ce couteau, je constate que tu l’as toujours dans la main.

\textsc{Atiena} : Lui gros porc.

\textsc{Moi} : N’en rajoute pas, mon poussin. Je ne suis pas certain de pouvoir le tenir en laisse bien longtemps.

Sam, brandissant son couteau : Je vais la tuer !

\textsc{Atiena} : Moi pas poussin.

\textsc{Nathan} : Non, toi casse-couilles !

\textsc{Elle} : Moi pas casse-couilles. Moi Atiena, Gardienne de la Nuit.

Greg, étonnamment en verve à une heure aussi avancée de la nuit : Et des ennuis.

Vous me connaissez assez pour savoir que je ne suis pas spécialement adepte de tous ces trucs mystiques qu’on nous balance à travers la gueule à longueur de journée. Je sais qu’il arrive parfois qu’une vieille bique à l’article de la mort débarque en fauteuil roulant à Lourdes et en reparte en trottinant sur ses jambes tel un lapereau découvrant la vie avec émerveillement, mais je ne crois pas pour autant au miracle de la grotte enchantée (au propre comme au figuré). Même chose pour les senteurs idylliques qui envahissent les naseaux frémissants du pèlerin tout pantelant de ferveur religieuse, les gouttes de sang qui suintent des statues de plâtre et les apparitions mariales à Medjugorje. Non, la Vierge (dont on attend toujours les résultats de l’examen gynécologique censé confirmer ses dires) a certainement d’autres chattes à fouetter que se fendre d’apparitions quotidiennes aux yeux de trois adolescentes analphabètes (Ivanka, Vicka et Mirjana, pour ne pas les nommer) de Bosnie-Herzégovine (à moins de bosser pour l’office de tourisme du coin), ou se pointer de la même façon à Fatima pour s’entretenir d’eschatologie cosmique avec des bergers prépubères pour qui jouer à touche-pipi constitue l’essentiel des préoccupations existentielles. Je suis prêt à entendre beaucoup de choses, y compris les plus débiles qui soient, mais il ne faut pas non plus me prendre pour un perdreau de l’année. Pour être tout à fait franc, je ne crois pas non plus une seule seconde que Jésus soit sorti de sa tombe par ses propres moyens, la bouche en cœur et frais comme un gardon, après avoir été cloué sur une croix. Je sais qu’il est de bon ton de dire que des tas de choses échappent à notre entendement, et je ne doute pas que ce soit le cas. Pour moi, si miracle il y a, la Nature seule en est responsable, et elle fait tellement chier le monde par ailleurs que c’est bien la moindre des choses qu’elle se rende un peu utile de temps à autre. Si des gens développent des maladies rares et meurent du jour au lendemain, je ne vois pas pourquoi d’autres ne guériraient pas de la même façon. En un mot comme en cent, il faut s’attendre à tout dans l’existence, à commencer par le pire dont on peut être certain qu’il finira tôt ou tard par arriver (la technique principale de survie, ou culture du déni, est de tout mettre en œuvre pour ne surtout jamais y penser). Bien sûr, il y aura toujours des gens pour me dire que la Nature n’est pas venue au monde toute seule, n’a pas surgi de nulle part, sous l’effet de quelque génération spontanée, mais qu’elle est le fruit d’une intelligence supérieure qui, tout en lui laissant l’illusion d’une certaine liberté, continue à contrôler discrètement ses moindres faits et gestes. Tout notre modèle humain, social et sociétal, est construit sur cette idée de manipulation occulte, laquelle, sous éteignoir la plupart du temps, refait parfois surface avec une vigueur accrue, quand elle n’alimente pas les délires paranoïaques et complotistes d’une certaine catégorie de la population.

À la lueur chevrotante des lignes qui précèdent, je gage que vous n’aurez aucun mal à imaginer ma surprise quand j’ai été témoin de la scène suivante : Sam, qui brandissait son couteau sous le nez d’Atiena en exprimant clairement son intention de la désosser comme un vulgaire jambon, s’est soudain immobilisé dans sa gestuelle belliqueuse tel un Troll des montagnes changé en statue de pierre par les premiers rayons du soleil (CF Le Hobbit, de Tolkien, quand Bilbon et les treize Nains de la Compagnie de Thorin sont faits prisonniers par Tom, Bert et Bill, trois Trolls des montagnes bien décidés à les bouffer mais incapables de se mettre d’accord sur la façon de les accommoder, Tom préférant les consommer crus après s’être assis dessus pour en faire de la gelée, Bill au barbecue avec juste un peu de sel et de poivre, et Bert, seul véritable gastronome et cuisinier de la bande, longuement mijotés avec des aromates).

Titus, quant à lui, s’il n’était pas à proprement parler plongé dans un état de minéralisation avancée, n’en subissait pas moins les effets d’une sorte de transe hypnotique interdisant tout contact avec le néocortex qui présidait habituellement à se destinée (à noter que son système limbique présentait lui aussi des signes d’indisponibilité temporaire).

\textsc{Nathan} : Sam ?

\textsc{Sam} : …

\textsc{Nathan} : Tu m’entends, Sam ?

\textsc{Sam} : …

Nathan, hésitant à le toucher comme s’il craignait de le voir tomber en poussière : Sam ?

\textsc{Sam} : …

\textsc{Atiena} : Lui pas entendre.

\textsc{Nathan} : Qu’est-ce que tu lui as fait, sorcière ?

\textsc{Atiena} : Moi pas sorcière. Moi Atiena, gar…..

\textsc{Nathan} : … dienne de la Nuit, oui, je sais. Je te préviens que Sam est ceinture noire cinquième dan d’aïkido. Il a été l’élève de maître Yoshimura Masayoshi, à Kobe, qui était lui aussi capable de paralyser …

\textsc{Atiena} : Adversaire par pensée, oui, moi savoir. Connaître lui.

\textsc{Nathan} : Toi connaître euh… tu connais maître Yoshimura Masayoshi ?

\textsc{Atiena} : Moi voir lui dans télévision.

\textsc{Nathan} : Ah bon.

\textsc{Atiena} : Moi venir très loin, Afrique. Faire partie société secrète capable de paralyser gens par pensée, comme maître Machin Chose.

Nathan, consultant sa Casio G-Shock Master of G Rangeman GW-9500-3ER à boussole numérique et altimètre barométrique : Tiens, ma montre est en panne. C’est bizarre.

\textsc{Greg} : Faut penser à changer les piles.

\textsc{Nathan} : Quelle heure est-il, por favor ?

Greg, consultant sa Baume \& Mercier Classima 10415, un modèle classique et discret en acier inoxydable étanche à cinquante mètres et bénéficiant de deux ans de garantie : Une heure du mat. Pourquoi ?

\textsc{Lui} : J’en ai ma claque. Je crois que je vais rentrer arroser mes plantes et regarder Bagne de femmes de Douglas Sirk.

\textsc{Greg} : T’es pas bien, avec nous ?

\textsc{Nathan} : Si, frangin. C’est juste que si Sam et Titus sont hors jeu, je crois qu’on n’a aucune chance de mener cette mission à bien.

\textsc{Greg} : Aucune, en effet.

\textsc{Atiena} : Titus pas aller.

\textsc{Moi} : Si Titus pas aller, nous pas aller non plus.

\textsc{Atiena} : Titus pas aller.

\textsc{Nathan} : Tu l’as déjà dit, poupée.

\textsc{Atiena} : Moi pas poupée.

\textsc{Greg} : Qu’est-ce qu’on fait ?

\textsc{Moi} : J’avoue que je suis légèrement dépassé par la situation.

\textsc{Nathan} : Elle est chouette, ta montre.

\textsc{Greg} : Merci.

\textsc{Moi} : T’as pas vu la mienne.

\textsc{Nathan} : Non. C’est quoi ?

\textsc{Moi} : Une Rousselot P06 Ultramatic 720, avec bracelet en cuir d’alligator fumé au bois de hêtre.

\textsc{Nathan} : Pas dégueu.

\textsc{Greg} : Juste un léger parfum de scandale.

\textsc{Moi} : C’est de l’alligator d’élevage, je précise.

\textsc{Nathan} : C’est qui, celui-là ?

Il faisait allusion au type qui venait d’apparaître dans la lumière des phares du van.

\textsc{Greg} : Celle-là.

\textsc{Nathan} : Pardon ?

\textsc{Greg} : Celle-là. Elle s’appelle Sally Robinson.

\textsc{Sally Robinson} : Bonjour, je suis Sally Robinson, celle par qui tous les scandales arrivent. Je peux savoir ce qui se passe, ici ?

\textsc{Moi} : Il se passe que rien ne se passe comme il faut.

\textsc{Sally} : Ah bon ? Et c’est qui, cette charmante jeune personne ?

\textsc{Atiena} : Moi Atiena, Gardienne de la Nuit.

\textsc{Sally} : Bonjour, charmante gardienne de la nuit. Dois-je comprendre que vous serez des nôtres, ce soir ?

\textsc{Moi} : Peau de balle ! Je suis désolé de vous décevoir, ma vieille Sally, mais je crois qu’on va être obligé de reporter l’opération aux calendes grecques.

\textsc{Sally} : Vous ne feriez pas ça ?

\textsc{Moi} : Si c’était nécessaire.

\textsc{Greg} : Nous avons eu quelques petits imprévus.

Sally, avançant la main pour toucher les dreads de la Gardienne : Très chouettes, vos dreadlocks, chère amie.

Atiena, avec un geste de recul : Toi pas toucher ! Toi homme ou femme ?

\textsc{Sally} : Les deux à la fois, mon petit chat.

\textsc{Atiena} : Moi pas petit chat.

\textsc{Sally} : Je suis meneuse de revue au Sugar \& Spice. Tu connais le Sugar \& Spice ?

\textsc{Atiena} : Pas connaître.

\textsc{Sally} : Tu devrais faire du music-hall, chérie. Je connais plein de monde dans le milieu. Je peux te présenter, si tu veux.

\textsc{Atiena} : Toi pervers sexuel ?

\textsc{Sally} : Mais non, voyons, pas du tout. C’est juste que tu as un physique à monter sur une scène. Je m’y connais, tu sais, en matière de spectacle vivant.

\textsc{Nathan} : Elle est folle, celle-là !

\textsc{Greg} : Tiago Rodriguez, lâchement assassiné par une bande de nostalgiques du troisième reich, était un excellent ami de Sally. N’est-ce pas, Sally ?

\textsc{Sally} : Alvarez, pas Rodriguez. Oui, c’était un excellent ami à moi, raison pour laquelle je tiens tout particulièrement à venger sa mort. Je me suis acheté une tenue de combat spécialement pour l’occasion.

\textsc{Moi} : Eh bien il faudra la plier et la ranger dans un coin en attendant de la ressortir. Je suis désolé de vous le dire, ma chère Sally, mais Atiena, Gardienne de la Nuit ici présente, affirme que les planètes ne sont pas idéalement alignées pour garantir le plein succès de l’opération.

\textsc{Greg} : Pire : elle affirme que nous entêter dans cette voie serait courir à une mort certaine.

\textsc{Atiena} : Titus pas aller.

\textsc{Greg} : Vous voyez.

Sally, sortant un bout de carton de carton imprimé d’une des très nombreuses poches de son gilet de camouflage : Tenez, ma chère, voici ma carte. N’hésitez pas à m’appeler en cas de besoin.

Atiena, dédaignant le bout de carton : Moi pas besoin.

\textsc{Sally} : On ne sait jamais. J’ai le bras long, vous savez.

\textsc{Atiena} : Pas besoin !

Sally, remballant son bout de carton : Comme vous voudrez. De toute façon, en cas d’absolue nécessité, vous pourrez toujours passer par monsieur (il parlait de Greg). Il a mes coordonnées. Ce n’est pas à proprement parler mon agent, mais il travaille pour moi.

\textsc{Greg} : Oui, d’ailleurs, à ce propos, j’attends toujours le règlement de mes honoraires.

\textsc{Sally} : Nous étions convenu d’une opération de nettoyage ethnique, ce soir.

\textsc{Greg} : Ethnique ?

\textsc{Sally} : Nazique, si vous préférez.

\textsc{Moi} : Chère madame, je suis le commandant Beauvais, de la police judiciaire. Je suis aussi le chef de cette petite assemblée, et si je décide que l’opération est annulée, c’est que j’ai de bonnes raisons de le faire. Je me dois bien évidemment de faire régner l’ordre et mettre les méchants face à leurs responsabilités, mais je dois aussi garantir la sécurité de mes hommes. Vous, par exemple, n’êtes absolument pas une spécialiste de la question. J’ai accepté de vous recevoir parce que Greg, qui est un ami, a lourdement insisté, mais plus je vous regarde et plus j’ai la désagréable sensation d’avoir commis une grave erreur.

\textsc{Sally} : Oui, eh bien dites-vous bien que j’ai exactement la même désagréable sensation. D’abord on me fait poireauter des heures, et ensuite quand j’arrive ici qu’est-ce que je trouve, une équipe totalement désorganisée, en proie au doute, et manifestement sous l’emprise d’une tierce personne dont personne ne m’avait averti de la présence. Il me semble que je paye tout de même assez cher pour qu’on me traite avec un minimum de déférence.

\textsc{Greg} : Justement, puisqu’on en parle, j’attends toujours de palper le solde de mes émoluments.

\textsc{Nathan} : Je peux dire quelque chose ?

\textsc{Moi} : Quoi ?

\textsc{Nathan} : Je sais pas si vous l’avez remarqué, mais la situation est totalement ridicule.

\textsc{Moi} : Et alors ?

\textsc{Nathan} : Alors je propose d’y mettre un terme. Pour commencer, Atiena pourrait rendre sa liberté à Sam.

\textsc{Moi} : Oui, c’est vrai. Atiena ?

\textsc{Atiena} : Quoi ?

Moi (je rappelle que Sam était toujours figé en statue de pierre, le bras levé, son Randall 18 Attack Survival à la main) : Pourriez-vous, s’il vous plaît, rendre sa liberté à Sam ?

\textsc{Atiena} : Lui gros porc.

\textsc{Moi} : Oui, lui gros porc, mais je pense qu’il a compris la leçon, maintenant.

\textsc{Atiena} : Lui menacer moi avec couteau.

\textsc{Moi} : C’est juste, mais je pense qu’il ne le refera plus.

Nathan, essayant sans succès de récupérer le couteau dans la main de Sam : Je n’arrive pas à le retirer.

\textsc{Moi} : Nathan, que vous connaissez bien maintenant, aimerait récupérer le couteau dans la main de Sam. Peut-être que vous pourriez l’aider à le faire.

Atiena, après un temps de réflexion : D’accord.

Aussitôt, les doigts de Sam se sont relâchés et Nathan a pu récupérer le couteau sans effort.

\textsc{Moi} : Très bien. Nathan, range ce couteau, tu veux.

\textsc{Nathan} : Tout de suite, chef.

\textsc{Moi} : Merveilleux. Maintenant que tout danger semble écarté, voulez-vous bien rendre sa liberté à Sam, ô somptueuse et vénérable déesse de la nuit ?

\textsc{Atiena} : Moi pas déesse, moi Gardienne de la Nuit.

\textsc{Moi} : Oui, bien sûr, c’est ce que je voulais dire.

\textsc{Atiena} : D’accord.

Quelques instants plus tard, Sam était de retour parmi nous. Quand je dis de retour, c’est une façon de parler, car j’avais la très nette et subtilement désagréable impression qu’une partie de lui-même était restée en rade pendant l’éclipse qu’il venait de traverser. Son regard, aussi vague qu’un paysage de neige vu à travers une épaisse couche de brume, semblait glisser sur les êtres et les choses comme la rosée du matin sur les plumes d’un canard endormi. Un sourire béat illuminait son visage, à peine reconnaissable tant l’expression de brutalité viscérale qui l’animait ordinairement avait disparu au profit de quelque chose qui s’apparentait étrangement à de la douceur ou de la bienveillance, deux mots qui en temps normal auraient suffi à le plonger dans un état de fureur incontrôlable.

\textsc{Nathan} : Mon dieu, qu’est-ce que vous lui avez fait !

\textsc{Atiena} : Sam revenu.

\textsc{Nathan} : Sam ?

\textsc{Sam} : Oui ?

\textsc{Nathan} : Ça va ?

\textsc{Sam} : Très bien, merci.

\textsc{Nathan} : C’est toi, Sam ?

\textsc{Sam} : Ben oui, qui veux-tu que ce soit.

\textsc{Nathan} : Désolé, mais cette chose n’est pas Sam.

\textsc{Atiena} : Lui Sam.

\textsc{Nathan} : Greg, tu ne vas quand même pas me dire que cette chose est Sam !

\textsc{Greg} : Physiquement parlant, c’est Sam, il n’y a aucun doute là-dessus. Par contre, je te concède qu’il semble avoir changé moralement.

\textsc{Nathan} : Merde !

\textsc{Moi} : Oui, merde.

\textsc{Lui} : Non, merde ! Remballez les couteaux, l’artillerie et tout le toutim, les gars, on a de la visite !

Les phares d’une voiture venaient d’apparaître au coin de la rue.

\textsc{Greg} : On dirait…

\textsc{Nathan} : Oui, c’est une voiture de flics !

La voiture s’est approchée, au ralenti, et arrêtée à quelques centimètres de nous.

Deux flics en uniforme en sont sortis : un petit, râblé, costaud, ramassé sur lui-même comme un pitbull prêt à passer à l’attaque, qui devait avoir dans les vingt-cinq/trente ans, et un autre nettement plus grand, osseux, froid comme un bac à glace, les traits taillés à la hache, d’une cinquantaine d’années.

Tous deux avaient des parfaites têtes d’abrutis comme on les aime, des lunettes de soleil sur le nez alors qu’on était en pleine nuit, la main sur le revolver et un sourire mauvais au coin du bec. C’était le genre sadique, cowboy mâchouilleur de chewin-gum comme on en croise sur les routes désertes du Texas, au volant d’une Dodge Charger gonflée à bloc, et qui n’hésitent pas à vous farcir de pruneaux au moindre pet de travers. Il faut dire qu’ils s’emmerdent tellement que le fait de tomber sur un contrevenant représente pour eux une véritable aubaine. Ils seraient prêts à flinguer un coyote en train de traverser hors des clous.

Le petit trapu a dit : Messieurs-dames bonsoir. Contrôle des papiers, s’il vous plaît.

J’ai dit : Je suis de la maison, les gars.

Le grand noueux a dit : Dans ce cas, vous n’êtes pas sans savoir que les rassemblements sur la voie publique de nature à troubler l’ordre public sont interdits.

\textsc{Moi} : On ne trouble rien du tout. Et oui, merci, je suis au courant, article 431 du code de procédure pénale.

J’ai fait le geste de glisser ma main dans ma poche pour sortir ma plaque, ce qui a eu pour effet de faire apparaître comme par magie un flingue dans la main des deux flics qui nous faisaient face.

\textsc{Le grand noueux} : J’éviterais de faire ça, si j’étais vous.

\textsc{Moi} : Commandant Beauvais, de la PJ. Je veux juste vous montrer ma carte.

Le petit trapu a dit, le flingue braqué sur moi : Alors allez-y. Mais en douceur, s’il vous plaît.

\textsc{Le grand noueux} : Lentement, très lentement.

\textsc{Le petit trapu} : J’ai des fourmis dans la gâchette, surtout à une heure aussi avancée de la nuit.

J’ai sorti ma carte.

Le grand noueux a dit, après avoir examiné le document sous toutes les coutures : Toutes mes excuses, mon commandant. Mais avec les trucs bizarres qu’on voit en ce moment, on est obligé de prendre toutes les précautions.

\textsc{Le petit trapu} : Les gens n’ont plus aucun respect pour l’uniforme, vous savez.

Ce que je savais, c’était que ces deux bouffons tombaient comme des poils de cul sur la soupe. Déjà qu’en temps normal personne n’avait envie de voir leurs tronches, en temps pas normal, comme c’était le cas maintenant, leur présence constituait à elle seule une insulte au bon goût passible des pires sanctions administratives et judiciaires. Cela dit, vu qu’il était difficile de les traduire devant la cour martiale pour un motif aussi inconsistant, monstrueux, certes, mais assez peu susceptible d’intéresser les autorités concernées dont on connaît, soit dit en passant, le laxisme systémique, j’ai préféré jouer la carte de l’apaisement.

\textsc{Moi} : Pas de problème, les gars. Ce n’est pas moi qui vous reprocherai de faire votre boulot.

\textsc{Le grand noueux} : Vous comprendrez que quand on voit des gens s’agiter au milieu de la rue, on ne peut pas laisser passer ça.

\textsc{Le petit trapu} : Sinon c’est tout le système qui s’écroule.

\textsc{Le grand noueux} : C’est la porte ouverte à l’anarchie, les hordes sauvages qui déferlent dans les cités et cassent tout sur leur passage.

\textsc{Le petit trapu} : Le chaos s’installe, la guerre civile.

\textsc{Moi} : Vous avez mille fois raison, les gars. On ne serait pas dans cette merde s’il y avait plus de jeunes comme vous. Hélas, la plupart ne pense plus qu’à faire la fête, boire de l’alcool et consommer de la drogue. Les zones de non-droit se multiplient et les honnêtes gens n’osent plus sortir de chez eux de peur de se faire agresser ou prendre une balle perdue.

\textsc{Le grand noueux} : Ouais, c’est la guerre des gangs, comme aux States. On trouve des armes à tous les coins de rues, et les gamins de douze ans n’hésitent plus à vous tirer dessus.

\textsc{Le petit trapu} : Des siècles de civilisation pour en arriver là, avouez que c’est bien triste. Perso, j’ai jamais fumé un joint de ma vie. Mon père était flic, il m’a transmis des valeurs que j’essaie de transmettre à mon tour, à coups de pompe dans le cul si nécessaire. Vous pensez que j’en fais trop ?

\textsc{Moi} : Pas du tout, mon garçon. Je suis content de voir qu’on peut s’appuyer sur des gars comme vous pour faire régner l’ordre dans cette putain de ville. On n’est pas encore à Gotham City, mais on ne va pas tarder à y arriver si personne ne prend les choses en main.

\textsc{Le grand noueux} : Pour sûr. On peut vous demander ce que vous faites là à une heure aussi avancée de la nuit, mon commandant ?

\textsc{Moi} : Il s’agit d’une petite réunion privée avec quelques amis. Il faut bien se détendre un peu de temps en temps.

\textsc{Le petit trapu} : Pour sûr. Surtout qu’on ne fait pas des métiers faciles, hein, mon commandant.

\textsc{Moi} : Et comment ! Des fois je me dis que j’aurais mieux fait d’être vétérinaire, ingénieur ou photographe de mode.

Greg, qui avait toujours un peu de mal à fermer sa gueule : Ou professeur. Même cuisinier ou chauffeur routier. Ou écrivain.

Le grand, soupçonneux en plus d’être noueux : Vous êtes flic aussi ?

\textsc{Greg} : Non, enquêteur privé.

\textsc{Le petit trapu} : Ah ouais ?

\textsc{Le grand noueux} : C’est marrant, ça.

\textsc{Le petit trapu} : Vous devez aussi en voir des vertes et des pas mûres, dans votre profession.

\textsc{Greg} : Le fait est que c’est une plongée vertigineuse dans la noirceur de l’âme humaine.

Le grand, toujours aussi noueux et soupçonneux : Pardon ?

\textsc{Le petit trapu} : Il cause bien, le monsieur.

\textsc{Le grand noueux} : Ce serait ti pas que vous auriez fait des études, par hasard ?

\textsc{Greg} : Pas vraiment. En vérité, je ne connais pas de meilleure école que celle de la vie. Je me suis formé sur le tas, comme on dit.

\textsc{Le petit trapu} : Et ça gagne bien, comme boulot ?

\textsc{Greg} : Il y a des hauts et des bas, comme partout.

\textsc{Le grand noueux} : Peut-être bien que vous avez des clients pleins aux as qui vous demandent d’enquêter sur leur femme ou d’espionner leur voisin ?

\textsc{Greg} : Ça peut arriver, s’ils pensent que leur voisin se tape leur femme. Ou alors s’ils pensent que leur voisin est un type bizarre et qu’ils veulent en savoir un peu plus à son sujet. On est là pour rendre service avant tout.

\textsc{Le petit trapu} : N’empêche qu’il faut quand même avoir les moyens pour se payer les services d’un privé. C’est comme qui dirait pas à la portée de toutes les bourses.

\textsc{Le grand noueux} : Ils vont où, les deux, là ?

Il parlait de Titus et Atiena qui étaient en train de se faire la malle discrètement, bras dessus bras dessous, manifestement assez peu désireux de participer à une discussion dont je me serais volontiers dispensé moi aussi, je dois bien le reconnaître, même si j’ai toujours plaisir à échanger avec mes concitoyens, aussi stupides, bornés et atteints de crétinerie aiguë qu’ils soient, ne serait-ce que parce que la présence de l’être humain, y compris la mienne, reste pour moi une source d’interrogation constante (et de consternation par la même occasion, moins en ce qui me concerne, mais quand même).

\textsc{Moi} : Ils rentrent chez eux, je pense.

\textsc{Le grand noueux} : Ils habitent ensemble ?

\textsc{Moi} : Non, ils rentrent chez eux chacun de leur côté.

\textsc{Lui} : Oui, mais ils sont ensemble, là.

\textsc{Moi} : Ils marchent côte à côte, ça ne veut pas dire qu’ils sont ensemble.

\textsc{Lui} : Je n’entends rien, mon commandant. C’est juste que j’ai l’habitude de me poser des questions quand j’assiste à des trucs bizarres.

\textsc{Moi} : Je ne vois pas ce qu’il y a de bizarre là-dedans.

\textsc{Le petit trapu} : Vous les connaissez bien ?

\textsc{Moi} : Très bien.

\textsc{Lui} : C’est qui, le grand black ?

\textsc{Moi} : Titus Beaugendre, un collègue de la PJ.

\textsc{Lui} : Et la fille ?

\textsc{Moi} : Une amie à lui.

\textsc{Le grand noueux} : Elle est pas un peu bizarre, la fille ?

\textsc{Moi} : Comment ça, bizarre ?

\textsc{Lui} : Le genre qui prendrait un peu de drogue, voyez, des trucs de ce genre.

\textsc{Moi} : Donc, si je comprends bien, vous insinuez que moi, Djeferson Beauvais de la PJ, je fréquente des drogués ?

\textsc{Lui} : J’ai pas dit ça.

\textsc{Moi} : Mais vous l’avez pensé très fort.

\textsc{Lui} : Vous savez ce que c’est, mon commandant. Avec tout le respect que je vous dois, on croit connaître les gens et il arrive parfois qu’on découvre des choses pas très jolies à leur sujet. Je dis pas que c’est systématique, je dis que ça arrive parfois. Et en ce qui concerne votre amie, disons que j’ai trouvé qu’elle avait une façon assez bizarre de nous regarder.

Moi, l’air détaché du type qui connaît la musique et n’apprécie pas plus que ça qu’on vienne le bassiner avec des histoires futiles de gens qui auraient soi-disant une façon assez bizarre de regarder les autres : Vous faites erreur sur toute la ligne, agent… ?

\textsc{Lui} : Brigadier-chef Darian Lisnic, mon commandant. À vos ordres !

\textsc{Moi} : C’est pas français, ça.

\textsc{Lui} : Moldave, mon commandant.

\textsc{Moi} : Ah ! la Moldavie ! ses paysages vallonnés, ses forêts luxuriantes, ses ruisseaux poissonneux, ses vignobles ancestraux, ses tumulus de l’âge du bronze, un bien beau pays s’il en est ! Eh bien sachez qu’elle non plus, cette admirable créature dont vous avez l’outrecuidance de prétendre qu’elle regarde les gens de façon assez bizarre, n’est pas française, comme vous l’aurez sans doute remarqué à son allure générale, la finesse de ses traits et la couleur de son épiderme. Sachez, mon cher Darian… vous permettez que je vous appelle Darian ?

\textsc{Lui} : À vos ordres, chef !

\textsc{Moi} : C’est pas un ordre, c’est une question.

\textsc{Lui} : Vous faites comme vous voulez, c’est vous le chef.

\textsc{Moi} : Mais ça ne te dérange pas ?

\textsc{Lui} : Quoi, que vous soyez le chef ?

\textsc{Moi} : Non, que je t’appelle par ton prénom.

\textsc{Lui} : Vous faites comme vous voulez, chef. Si vous avez envie de m’appeler par mon prénom, vous m’appelez par mon prénom. Je n’y vois aucun inconvénient. D’ailleurs, même si j’en voyais un, ça ne servirait à rien que je le dise, puisque vous êtes le chef et que vous pouvez m’appeler comme vous voulez.

\textsc{Moi} : Mais vous n’en voyez pas.

\textsc{Lui} : D’inconvénient ?

\textsc{Moi} : Oui.

\textsc{Lui} : Aucun, mon commandant.

\textsc{Moi} : Dans ce cas, mon cher Darian, sachez que cette jeune personne appartient à un peuple très ancien, les Khoïsan, qui chassent l’antilope et font griller des sauterelles sur des barbecue de fortune depuis des temps immémoriaux pour nourrir leur très nombreuse famille, absence de contraception oblige. Depuis le Paléolithique supérieur, pour être exact. Ça vous dit quelque chose, le Paléolithique supérieur ?

Il a entrouvert la bouche dans l’intention manifeste de s’exprimer sur le sujet, mais je ne lui ai pas laissé le temps de le faire, certain que la réponse, constituée principalement de balbutiements inintelligibles, ne présenterait pas grand intérêt : Je suppose que non, à en juger par le regard bovin et la lippe tombante que vous affichez en ce moment-même.

Il s’est rétracté dans le fond de sa coquille comme un escargot ébouillanté.

J’ai enchaîné, sans me soucier des bâillements et autres signes d’ennui dispensés par mon auditoire : Quoi qu’il en soit, mon cher Darian, Darwin lui-même, dans sa Filiation de l’homme et la sélection liée au sexe, en parle de façon très élogieuse. Les San, noble peuple s’il en est, descendent d’une population fantôme vieille de plusieurs centaines de milliers d’années. Leur ADN contient des haplogroupes mitochondriaux appartenant à la Mère de toutes les mères, celle dont le ventre primordial a enfanté l’espèce humaine au sens noble du terme. Le généticien japonais Kudo Naonori, professeur à l’université de Sendai, membre honoraire de la Royal Society de Londres et spécialiste de l’évolution des espèces, auteur d’ouvrages de référence sur la théorie de la coalescence et la dérive génétique, a dressé un arbre généalogique complet de tous les locus polymorphes connus. Ce dernier indique on ne peut plus clairement que les origines de l’Homme moderne ne remontent pas à quelques deux cent mille ans, comme on le pensait depuis la découverte des squelettes de la basse vallée de l’Omo, mais à plus de trois cent mille. Quelques rares ossements, dont un crâne quasiment complet, ont été retrouvés dans une mine de barytine de la province de Youssoufia, au Maroc, en compagnie de nombreux outils en pierre taillée de facture très avancée. Si vous aviez traversé autant d’épreuves au cours de votre existence, peut-être que vous aussi auriez une façon un peu «bizarre» de considérer vos semblables.

Lisnic, à la fois ému par le fait que pour une des rares fois de sa vie quelqu’un s’adressait à lui comme à autre chose que l’abruti de première qu’il n’avait jamais cessé d’être, allant même jusqu’à pousser le vice d’en concevoir une certaine fierté de classe, et agacé par celui d’être une fois de plus confronté à l’immensité océanique du désastre culturel dont il était l’un des plus fidèles représentants : Z’êtes un vrai puits de science, mon commandant.

\textsc{Moi} : Tu l’as dit, bouffi.

Lui, tête baissée tel un collégien qui vient de se faire méchamment remonter les bretelles par son paternel après avoir récolté un zéro en maths assorti de commentaires peu élogieux sur sa conduite au sein de l’établissement : Désolé, je voulais pas vous froisser. C’est juste que quand je les ai vu se barrer en loucedé, elle et son compagnon, je me suis dit qu’ils avaient peut-être des choses à cacher.

\textsc{Moi} : Comme je vous l’ai déjà dit, c’est juste un ami, pas son compagnon.

\textsc{Le petit trapu} : N’empêche que j’ai trouvé que lui aussi avait l’air un peu bizarre, si je puis me permettre.

Greg, qui rongeait son frein depuis un moment et n’attendait qu’une occasion de monter au créneau : Dites-moi, mon petit vieux, tout le monde a l’air bizarre, avec vous.

L’autre l’a regardé de travers, comme s’il s’agissait d’un chien qui venait de pisser sur ses bas de pantalon, et je me suis dit que la conversation, qui se déroulait jusqu’ici sur un ton relativement badin, pouvait dégénérer rapidement si je n’intervenais pas pour calmer le jeu.

\textsc{Moi} : Je peux connaître vos nom, adresse et matricule, soldat ?

\textsc{Lui} : Vous voulez aussi mon numéro de sécurité sociale ?

\textsc{Moi} : Votre nom suffira.

\textsc{Lui} : Barkad Achaari, chef.

\textsc{Moi} : Pas très français non plus, tout ça.

\textsc{Lui} : Marocain, chef. Tout à l’heure, vous parliez d’un crâne retrouvé dans une mine de barytine de la province de Youssoufia. On trouve toutes sortes de minerais, dans la région, argent, cuivre, cobalt et manganèse, mais elle est surtout connue pour ses gisements de phosphates, qui sont parmi les plus riches du monde. Il y a aussi, à Oued Zem et Khourigba, des gisements de fossiles de reptiles marins du Mésostoïque qui attirent les amateurs du monde entier.

\textsc{Moi} : Du Mésostoïque ?

\textsc{Achaari} : Oui, chef, du Mésostoïque.

\textsc{Moi} : Tu veux dire du Mésozoïque.

\textsc{Lui} : Oui, peut-être. Toujours est-il que mon père est originaire du coin. Après avoir été longtemps chercheur d’or à Tichla, dans le Sahara occidental, et vendu des dents des fossiles aux touristes pour arrondir ses fins de mois, il a fini agent de sécurité à l’OCP, l’Office Chérifien des Phosphates de Safi, plus gros producteur de phosphates du monde.

Moi, de la docte voix de l’enseignant qui s’adresse à un amphithéâtre plein à craquer d’étudiants avides de boire son jus vocal jusqu’à la dernière gorgée syllabique : Ton histoire ému aux larmes, mon petit Barkad, et je crois que je ne suis pas le seul. Maintenant, si tu veux mon sentiment sur le sujet qui nous occupe, à savoir est-ce que l’homme de loi jouit d’une hypersensibilité particulière pour tout ce qui est bizarre, insolite, inhabituel, eh bien sache que ma réponse est oui, sans aucun doute. C’est dans sa nature. Dès qu’un truc sort de l’ordinaire, il le renifle à des lieues à la ronde, comme un requin qui a flairé l’odeur du sang.

\textsc{Achaari} : Ouais, c’est comme un sixième sens.

\textsc{Lisnic} : On anticipe les situations délicates, toujours prêt à réagir au quart de tour.

\textsc{Achaari} : On sent tout de suite quand quelque chose cloche, même le plus infime détail. C’est un truc de dingue !

\textsc{Lisnic} : Ouais, et on fait en sorte que le citoyen lambda continue à vivre sa petite vie tranquille sans jamais se douter qu’il risque sa peau à tous les coins de rues. C’est ça, notre job, et c’est pour ça qu’on passe notre temps à arpenter les rues de la cité au lieu de s’occuper de notre petite famille. Pas vrai, mon commandant ?

\textsc{Moi} : Pour sûr.

\textsc{Lisnic} : Et lui, c’est qui ? Un ami à vous, aussi ?

Il parlait de Sam, toujours en train de bâiller aux corneilles avec un air bienheureux d’idiot du village pour qui chaque brin de muguet, oisillon tombé du nid ou hérisson écrasé au bord de la route représente une source d’émerveillement digne des Mille et une nuits, Vingt mille lieues sous ta mère et tous les autres trucs avec mille dans le titre.

\textsc{Moi} : Affirmatif. C’est le capitaine Samuel Girard, un ancien des Forces Spéciales.

\textsc{Achaari} : Il n’a pas l’air au mieux de sa forme, lui non plus.

Sam, d’une voix monocorde de robot humanoïde : Je suis le capitaine Samuel Girard, ancien des Forces Spéciales, et je vais très bien, merci.

\textsc{Lisnic} : Dites, il aurait pas un peu sauté sur une mine ou un truc dans le genre, le capitaine. Il a l’air un peu… comment dire… diminué intellectuellement.

\textsc{Moi} : Il a eu un petit accident du travail.

\textsc{Sam} : Accident du travail.

Achaari, s’adressant cette fois à Sally Robinson : Et vous, vous êtes qui ?

\textsc{Sally Robinson} : Sally Robinson, meneuse de revue au Sugar \& Spice, à quelques rues d’ici.

\textsc{Achaari} : Je connais, mon cousin y a travaillé comme serveur. Sally : Et il s’appelle comment, votre cousin ?

\textsc{Achaari} : Moustapha Nedali. C’est le fils de ma tante Rachida.

Sally, une lueur égrillarde dans le fond de l’œil : Oui, Moustapha, je me rappelle très bien. Beau comme un dieu, le gamin. Mousse, comme on l’appelait, le petit Moumousse. Mousse la peau douce comme de la mousse, qui sentait bon le houmous. Qu’est-ce qu’il devient, le petit chéri ?

\textsc{Achaari} : Aux dernières nouvelles, il est coach sportif au Marouazi Palace de Casablanca, dans le Triangle d’or. Il s’occupe des riches clientes qui ont besoin de se remettre en forme entre deux séances de shopping.

\textsc{Sally} : Je vois. Il faut dire qu’il sait se servir de ses mains, le petit coquinou !

\textsc{Achaari} : Je vous en prie.

\textsc{Sally} : Et pas que de ses mains, vous pouvez me croire !

Achaari, affichant les signes d’une certaine nervosité : Attention, vous parlez de mon cousin, là !

\textsc{Sally} : Oui, votre cousin, le petit Moumousse. On était toutes amoureuses de lui.

\textsc{Achaari} : Quoi ???!!!! Vous n’êtes tout de même en train de me dire que…

\textsc{Sally} : Que quoi ?

\textsc{Achaari} : Que Moustapha, le fils de ma tante Rachida, est…

Sally, le sourire d’une oreille à l’autre : Gay ? Non, je n’ai pas dit ça. Plutôt à voile et à vapeur, si vous voyez ce que je veux dire. Il faut dire qu’un corps pareil, des yeux de braise et des tablettes de chocolat de ce calibre, ce serait dommage de ne pas en faire profiter le plus grand monde. C’est une attitude responsable qui force l’admiration. Non, vous ne croyez pas ?

\textsc{Achaari} : Vous n’avez pas à parler de mon cousin comme ça.

\textsc{Sally} : Et je ne parle du reste.

\textsc{Achaari} : Quoi ? Quel reste ?

Sally, se pourléchant ostensiblement les babines tartinées de gloss repulpant à l’acide hyaluronique : Eh bien, son petit équipement personnel. Son petit héritage familial, quoi.

Achaari, grattant du pied tel un taureau prêt à charger : Vous vous foutez de moi ?

\textsc{Sally} : Pas du tout, mon mignon. Je dis juste que votre cousin trimballe dans son cabas un service trois pièces de star du X ! Je ne vois pas ce qu’il y a de mal à ça.

Achaari, le souffle coupé par l’indignation : Non mais… vous l’entendez, mon commandant ?

Moi, sentant venir le drame : Oui, je me paluche pas mal mais je ne suis pas encore sourd. Le problème, voyez-vous, c’est qu’on est dans un pays libre. Chacun à le droit de s’exprimer comme il l’entend, avec les mots qui sont les siens.

\textsc{Achaari} : Désolé, mais je ne peux pas le laisser dire ça.

Sally, remontée à bloc : LA laisser. Je suis une femme, au cas où vous ne l’auriez pas remarqué.

\textsc{Achaari} : Non, effectivement, je n’avais pas remarqué. Vous, une femme ? C’est une blague !

\textsc{Sally} : Pas du tout. Ça ne se voit peut-être pas extérieurement, autant qu’il faudrait en tout cas, mais intérieurement je suis une femme de toute la force de mon âme.

\textsc{Achaari} : Homme, femme, chèvre ou table basse, je me fiche de savoir ce que vous êtes ! Tout ce que je sais, c’est que ça ne vous donne pas le droit de manquer de respect à ma famille !

\textsc{Sally} : Je ne manque de respect à personne. Au contraire, je dis que votre cousin Moustapha est un des plus beaux gosses qu’il m’ait été donné de sucer… croiser, pardon…

Achaari, au bord de l’éruption : Quoi, qu’est-ce que vous venez de dire ?

\textsc{Moi} : Calmez-vous, mon vieux. Greg, s’il te plaît, tu ne veux pas dire à ta cliente de fermer un peu sa grande gueule ?

\textsc{Greg} : Oui, en effet, je pense que certaines limites ne sont pas loin d’avoir été franchies. Essayons de rester digne, je vous en prie.

\textsc{Nathan} : Quelle soirée de merde !

\textsc{Moi} : Je crois qu’on a tous besoin de se détendre un peu.

\textsc{Sally} : Je suis parfaitement détendue. Ou plutôt je l’étais, avant que tout le monde décide de me casser les couilles !

\textsc{Greg} : C’est quoi, votre problème ? Vous en voulez à la terre entière, c’est ça ?

\textsc{Sally} : Pas le moins du monde. Mais comme l’a très bien dit votre ami, on est dans un pays libre et j’ai le droit de m’exprimer comme n’importe qui d’autre. Et, n’en déplaise à monsieur, je ne pense manquer de respect à sa famille en disant que Moustapha est un des plus beaux garçons qu’il m’ait été donné de rencontrer.

\textsc{Achaari} : Vous n’avez pas dit ça.

\textsc{Sally} : Si, je l’ai dit.

Lisnic, écartant son subordonné d’un revers de la main afin de prendre les rênes (et pas les rennes, comme j’ai pu le voir écrit ici et là, on n’est pas dans le Grand Nord) de l’interrogatoire : Non. Vous avez prétendu vous êtes livré à des actes de nature sexuelle avec le cousin de mon collègue. De tels propos, adressés à un agent dans l’exercice de ses fonctions, constituent un outrage caractérisé.

\textsc{Achaari} : C’est de la diffamation pure et simple !

\textsc{Sally} : Ma langue a fourché, voilà tout.

\textsc{Lisnic} : Vous reconnaissez donc n’avoir entretenu aucune relation de nature sexuelle, orale ou autre, avec monsieur Moustapha Nedali, le cousin de monsieur Barkad Achaari, mon collègue ici présent ?

\textsc{Sally} : Oui, si vous voulez.

\textsc{Lisnic} : Ce n’est pas si je veux, c’est oui ou c’est non.

\textsc{Sally} : Oui, je le reconnais, et croyez bien que je le déplore.

\textsc{Lisnic} : Vous avez également affirmé, en parlant de ce que vous avez qualifié assez légèrement de «petit héritage familial» de monsieur Moustapha Nedali, qu’il, je cite «trimballe dans son cabas un service trois pièces de star du X». Vous comprenez bien qu’une telle affirmation, hors cadre, constitue une grave atteinte à la dignité de l’intéressé.

Sally, tirant nerveusement sur une Vogue Superslims Bleue qu’on voyait trembloter entre ses doigts aux ongles interminables recouverts d’une épaisse couche de vernis rose fuchsia : Je ne vois pas en quoi.

\textsc{Lisnic} : Donc vous maintenez vos propos ?

\textsc{Sally} : Et comment, que je les maintiens !

\textsc{Greg} : Vous feriez mieux de lâcher l’affaire.

Nathan, entre ses dents : Commence à me courir sur le haricot, la grosse.

Lisnic, une vague lueur de plaisir sadique déformant ses lèvres minces et cruelles de jeune loup aux dents longues : Dans ce cas, Mme Robinson, vous me permettrez de vous poser la question suivante : Qu’est-ce que vous permet d’affirmer que, je cite, «monsieur Moustapha Nedali trimballe dans son cabas un service trois pièces de star du X» ? Vous l’avez déjà vu ?

\textsc{Sally} : Mousse se baladait souvent en slip dans les loges.

\textsc{Lisnic} : En slip ?

\textsc{Sally} : En caleçon, si vous préférez.

\textsc{Lisnic} : Quel genre de caleçon ?

\textsc{Sally} : Je ne vais pas citer de marque, mais c’est le genre boxer qui moule bien le paquet. Poutre apparente, comme on dit.

\textsc{Lisnic} : Je vois. D’où ma question suivante : Pourquoi monsieur Moustapha Nedali évoluait-il dans cette tenue dans les loges ?

\textsc{Sally} : Pour une raison très simple, monsieur l’agent…

\textsc{Lisnic} : Brigadier-chef, s’il vous plaît.

\textsc{Sally} : … monsieur le brigadier-chef, pardon : parce qui’il avait besoin d’aller dans les loges pour se changer, comme tout le monde.

\textsc{Lisnic} : Se changer ?

\textsc{Sally} : Oui. Je vous rappelle que le Sugar \& Spice est ce qu’on appelle un cabaret, autrement dit un établissement dans lequel se produisent des artistes de music-hall qui sont amenés à changer régulièrement de tenue.

\textsc{Lisnic} : J’avais cru comprendre que monsieur Nedali travaillait comme serveur dans cet établissement ?

Sally, d’une voix exprimant clairement son exaspération : Oui, mais dans notre établissement même les serveurs ont droit à une tenue particulière. C’est comme dans les clubs Playboy : les Bunnies portent un costume de lapin sexy avec corset, nœud pap, oreilles et queue de lapin. Eh bien chez nous c’est pareil : les serveurs portent une tenue de service.

\textsc{Lisnic} : Quel genre de tenue ?

\textsc{Sally} : Légère.

\textsc{Lisnic} : Vous voulez dire que Mr Nedali travaillait en slip ?

\textsc{Sally} : Les employés portent une tenue de service adaptée à leurs besoins, spécialement conçue pour ne pas entraver leur liberté de mouvement tout en étant agréable à regarder.

\textsc{Lisnic} : C’est à dire ?

\textsc{Sally} : Un boxer en latex avec étui pénien.

\textsc{Lisnic} : C’est une plaisanterie ?

\textsc{Sally} : Pas du tout, non. Tous nos serveurs portent un boxer en latex avec étui pénien, très apprécié de la clientèle. Pour répondre à votre question, encore que rien ne m’oblige à le faire, c’est grâce à ça, et aussi au fait que j’ai à de très nombreuses reprises eu le privilège d’entrevoir Moumousse sous la douche, que je suis en mesure d’affirmer qu’il était plutôt bien équipé de ce côté-là.

\textsc{Lisnic} : Donc, vous reconnaissez à demi-mot qu’il vous arrive de mater les gens sous la douche.

\textsc{Sally} : Ecoutez, monsieur l’agent brigadier-machin-chose, les mœurs sont très libres au Sugar \& Spice, et il n’est pas rare que nous prenions nos douches en commun. Les sportifs font ça régulièrement, on n’en fait pas tout un plat.

\textsc{Lisnic} : Mais les sous-entendus vont bon train.

\textsc{Sally} : Il y a longtemps que je ne me soucie plus de ce que racontent ou pensent les gens. Quand on a, comme c’est mon cas, une personnalité que je qualifierai de clivante, on apprend à passer outre ce genre de considérations. Si on n’en est pas capable, mieux vaut rester terré chez soi.

\textsc{Moi} : Je ne voudrais surtout pas me mêler de ce qui ne me regarde pas, chacun, je le répète, étant libre de se balader dans la tenue de son choix et se tripoter sous la douche avec qui bon lui semble, mais il commence à se faire tard.

\textsc{Sally} : Je ne vois effectivement pas où est le problème entre personnes majeures et consentantes.

\textsc{Achaari} : C’est pas la question !

\textsc{Sally} : Si. Et j’ajoute que rien ne m’oblige à répondre à vos questions débiles concernant ma vie privée. Maintenant, si vous voulez me menotter et me placer en garde à vue parce que votre cousin se balade à moitié à poil sur son lieu de travail, je vous en prie, ne vous gênez surtout pas !

\textsc{Greg} : Si vous pouviez éviter d’en rajouter.

\textsc{Nathan} : Je crois en effet qu’on a déjà perdu assez de temps comme ça.

\textsc{Sally} : Désolé, mais je n’apprécie pas tellement d’être stigmatisé en raison de mes orientations sexuelles.

\textsc{Achaari} : Il ne s’agit pas de ça !

\textsc{Sally} : Bien sûr que si. Vous vous croyez autorisé à me traiter comme une moins que rien parce que je suis différente des autres. Si quelqu’un peut se sentir blessé ici, c’est moi et pas vous. J’ai des témoins, je pourrais fort bien porter plainte.

\textsc{Lisnic} : Je ne suis certain que votre stratégie de défense soit la bonne.

\textsc{Sally} : Quelle stratégie de défense ? Je ne vois pas pourquoi je devrais me défendre de quoi que ce soit ! C’est plutôt vous qui aurez à répondre des accusations indignes que vous portez contre moi.

\textsc{Greg} : Vous ne pouvez pas la boucler un peu ?

\textsc{Sally} : Oui, bien sûr, on m’insulte ouvertement, tout juste si on ne me traite pas de pervers, de délinquant sexuel, et je devrais me taire, tendre gentiment la joue gauche ! Eh bien non, monsieur le détective, personne ne fait taire Sally Robinson !

\textsc{Nathan} : Quelle soirée de merde !

\textsc{Moi} : Messieurs, je vous en prie !

\textsc{Nathan} : Quoi, c’est vrai, on n’a rien fait de mal ! Pourquoi est-ce que ces soi-disant gardiens de la paix nous traitent comme de vulgaires malfrats ? C’est de l’abus de pouvoir, ni plus ni moins ! Vous n’avez pas mieux à foutre que d’emmerder les honnêtes gens ?

\textsc{Lisnic} : On ne fait que notre travail, monsieur. La loi nous oblige à intervenir quand on repère un attroupement au milieu de la chaussée. C’est ce que nous avons fait, et tout se passait pour mieux avant que cet individu ne se mette à proférer des insanités au sujet du cousin de mon collègue ici présent.

\textsc{Sally} : Je n’ai fait que rapporter l’exacte vérité des faits.

\textsc{Lisnic} : Reconnaissez que vous vous êtes livré à des actes de provocation gratuite.

\textsc{Achaari} : Dont je ne serais pas étonné qu’ils soient directement liés aux origines ethniques de mon cousin et moi-même.

\textsc{Sally} : Si je comprends bien, vous me traitez de raciste, maintenant, alors que c’est vous qui avez fait preuve de discrimination à mon encontre.

\textsc{Lisnic} : Pas du tout.

\textsc{Achaari} : J’ai rarement vu une telle mauvaise foi !

\textsc{Moi} : Messieurs, je ne voudrais pas abuser de mes prérogatives, mais en temps que votre supérieur hiérarchique, je pense en effet que la comédie a assez duré. Vous admettrez que vos agissements sont assez peu en rapport avec la procédure habituelle. En clair, si vous continuez à nous casser les couilles au lieu de faire le boulot pour lequel le contribuable vous rémunère, je me verrai dans l’obligation d’en référer à qui de droit. Suis-je clair ?

\textsc{Lisnic} : Très clair, mon commandant.

\textsc{Moi} : J’ai été ravi de faire votre connaissance.

\textsc{Lisnic} : Et réciproquement, mon commandant.

\textsc{Moi} : Hélas, il n’est de bonne compagnie qui ne se quitte.

\textsc{Greg} : Eh oui, toutes les bonnes choses ont une fin.

\textsc{Sally} : Contrairement aux mauvaises qui ne s’arrêtent jamais.

\textsc{Lisnic} : Mes respects, mon commandant. J’espère que vous ne garderez pas un trop mauvais souvenir de notre rencontre.

\textsc{Moi} : Nullement. Je n’irais peut-être pas jusqu’à dire que vous faites honneur à la police française, mais vous êtes sur la bonne voie. Si l’occasion se présente, je rendrai compte à vos supérieurs de l’excellente qualité de votre intervention.

\textsc{Lisnic} : Merci, mon commandant. Et veuillez excuser mon subordonné, l’agent Achaari. Il fait parfois preuve d’excès de zèle, mais dans le fond c’est quelqu’un de tout à fait fiable et généreux.

\textsc{Sally} : Fiable et généreux ?

\textsc{Lisnic} : Fiable et généreux. Il ne laisse jamais un collègue en plan et privilégie le plus souvent la prévention à la répression.

\textsc{Achaari} : Toutes mes excuses, mon commandant, si je vous ai paru un peu excessif. Je reconnais que j’ai tendance à m’emporter dès qu’on touche à la famille. Je suppose que ça ne me regarde pas, mais le capitaine n’a pas l’air bien du tout.

Sam, en effet, était allongé sur le trottoir, les bras en croix.

\textsc{Moi} : Ne vous en faites pas, il fait ça tout le temps. C’est sa façon à lui de se reposer après une dure journée de travail.

\textsc{Nathan} : Ça va, Sam ?

\textsc{Sam} : Ça va, Sam ?

\textsc{Nathan} : Je te demande si ça va, Sam ?

\textsc{Sam} : Je te demande si ça va, Sam ?

\textsc{Nathan} : Je suis embêté, quand même. Je me demande si on ne devrait pas l’emmener faire un petit tour aux urgences.

\textsc{Lisnic} : C’est vrai qu’il n’a pas l’air bien.

\textsc{Achaari} : C’est bizarre cette façon de répéter ce qu’on lui dit.

\textsc{Moi} : Messieurs, votre sollicitude me touche. Mais je connais Sam depuis des années, je sais parfaitement comment m’y prendre avec lui. Il souffre d’une forme assez rare de tétanie passagère. La crise dure quelques minutes, puis il revient à lui comme si de rien n’était.

\textsc{Greg} : Il a longtemps travaillé pour les forces spéciales, en Syrie et en Afghanistan notamment, et a reçu de nombreuses décorations pour services rendus à la Nation. Un jour, près de Kandahar, il participait à une opération de nettoyage d’un trou à rats infesté de terroristes. Il s’est pris une balle dans la tête, et c’est un miracle qu’il soit encore en vie. Wilfrid Chauveau, professeur de neurochirurgie à la Pitié-Salpêtrière, n’en revient toujours pas. Il a fallu toute une série d’interventions extrêmement pointues, dont certaines ont duré plusieurs heures, pour le sortir de là. Seulement, depuis, il est sujet à des accès de mélancolie et des troubles du comportement plus ou moins spectaculaires. De temps en temps, comme c’est le cas ce soir, il s’allonge sur le sol, les bras en croix, la bouche ouverte et les yeux révulsés, et séjourne dans cette position un temps indéterminé.

Un gargouillis, style cuvette de chiottes qui se débouche d’un seul coup, est sorti de la gorge de Sam, après quoi ses yeux ont effectué quelques tours complets dans leurs orbites, il a remué un bras, puis l’autre, et posé sur ce qui l’entourait un regard empreint d’une certaine forme de curiosité brumeuse.

Greg s’est aussitôt porté à son chevet : Sam ? Ça va ?

\textsc{Sam} : Euh… oui, je crois… On est où, là ?

\textsc{Greg} : Rue des Nénuphars.

\textsc{Sam} : Merde, alors ! Qu’est-ce qu’on fout là ?

\textsc{Greg} : Tu ne te rappelles pas ?

\textsc{Sam} : Non.

\textsc{Greg} : Mais lui (il parlait de moi), tu le reconnais, quand même ?

\textsc{Sam} : Ben oui, c’est Djef.

\textsc{Greg} : Et moi, je suis qui ?

\textsc{Sam} : Greg. Et lui c’est Nathan. Mais ça ne dit toujours pas ce qu’on fait là.

\textsc{Moi} : Tu ne te rappelles de rien, donc ?

\textsc{Lui} : De rien du tout. Qu’est-ce que je fais allongé par terre ?

\textsc{Greg} : C’est rien, juste un petit malaise. Suite à ton opération du cerveau.

\textsc{Sam} : J’ai été opéré du cerveau ?

\textsc{Moi} : Oui. Même que tu as bien failli ne jamais récupérer la totalité de tes facultés mentales.

Greg, entre ses dents : Je ne suis pas certain qu’il les ait totalement récupérées.

\textsc{Sam} : Merde, alors !

\textsc{Greg} : Oui, hein. Mais tout ça c’est du passé, rassure-toi. Tout va bien, maintenant.

\textsc{Sam} : Je peux savoir ce que la police fait là ?

\textsc{Greg} : Rien. Elle allait partir, justement. N’est-ce pas, messieurs ?

\textsc{Lisnic} : Oui, tout à fait.

\textsc{Achaari} : On ne faisait que passer.

\textsc{Lisnic} : On a vu de la lumière, on s’est arrêté pour jeter un œil.

\textsc{Achaari} : Réflexe professionnel.

\textsc{Lisnic} : C’est notre devoir de prêter assistance aux gens dans la détresse. Content de voir que tout va bien, mon capitaine.

\textsc{Sam} : Et la grosse, là, c’est qui ?

\textsc{Sally} : Il parle de moi, là ?

\textsc{Moi} : Je crois bien, oui.

\textsc{Greg} : C’est Sally Robinson, une cliente à moi.

Sam, avec une moue de dégoût : Rarement vu un boudin pareil !

\textsc{Sally} : Non mais je vous en prie ! Pour qui il se prend, cet abruti !

\textsc{Moi} : Vous voyez bien qu’il n’est pas dans son état normal.

\textsc{Sally} : Tout de même, ce n’est pas une façon de parler aux gens !

Nathan, tout en aidant Sam à se relever : Ça va, tu tiens debout ?

\textsc{Sam} : Ça va, merci.

\textsc{Nathan} : Viens, on aller s’assoir un peu dans le van.

\textsc{Sam} : Pas envie de m’assoir.

\textsc{Nathan} : Faut te reposer.

\textsc{Sam} : Pourquoi faire ? Je me sens tout à fait bien, maintenant. Une absence passagère, rien de plus.

\textsc{Greg} : Tu ne te rappelles vraiment de rien ?

\textsc{Sam} : De quoi je devrais me rappeler ?

\textsc{Greg} : Tu ne te rappelles pas qu’il y avait d’autres gens avec nous ?

\textsc{Sam} : Des gens ? Non. Quels gens ? Et elle, c’est qui ?

\textsc{Greg} : Je viens de te le dire : c’est Sally Robinson, une cliente à moi.

\textsc{Sam} : Ah bon. Tu pourrais les choisir un peu mieux, tes clientes. J’ai rarement vu une bonne femme aussi moche. On dirait un homme.

\textsc{Sally} : Je vais lui en coller une !

\textsc{Greg} : Ne faites pas attention à lui.

\textsc{Sally} : Je ne suis pas venue ici pour me faire insulter !

\textsc{Sam} : Et qu’est-ce qu’elle fait là, cette mocheté ?

\textsc{Sally} : J’exige que cet imbécile me fasse des excuses. Tout de suite !

\textsc{Greg} : Calmez-vous, je vous prie.

\textsc{Sally} : Je me calmerai si je veux ! Vous me décevez beaucoup, monsieur Lussier. Beaucoup ! Je pensais que vous étiez quelqu’un de sérieux et raisonnable, et je me rends compte que vous n’êtes qu’une ordure comme les autres !

\textsc{Greg} : Excusez-moi, mais je ne vois pas très bien le rapport.

\textsc{Sally} : Tous ces gens que vous présentez comme étant vos amis sont des beaufs racistes et homophobes !

\textsc{Moi} : Pardon ?

\textsc{Nathan} : Qu’est-ce qu’elle dit, cette morue ?

\textsc{Sally} : Là, vous voyez !

\textsc{Greg} : Nathan, s’il te plaît !

\textsc{Nathan} : Désolé, mais je n’ai pas l’intention de me laisser insulter par ce monstre de foire !

\textsc{Sam} : Haha ! Elle fait moins sa maline, la grosse !

\textsc{Sally} : C’est inadmissible, insupportable !

\textsc{Greg} : Ne nous emballons pas.

\textsc{Sally} : Je ne resterai pas une minute de plus avec des gens aussi grossiers ! Je ne vous félicite pas, monsieur Lussier.

\textsc{Greg} : Je suis désolé. La fatigue, le stress, les événements qui s’entrechoquent…

\textsc{Sam} : Dans ton cul, oui.

\textsc{Sally} : Ça suffit, je m’en vais !

\textsc{Nathan} : C’est ça, barre-toi, Elephant Man !

\textsc{Sam} : Retourne dans ta cage, King Kong !

\textsc{Sally} : Tu peux parler, toi, le débile !

\textsc{Sam} : C’est à moi que tu parles, Moby Dick ?

\textsc{Sally} : Demeuré !

\textsc{Sam} : Je vais te défoncer, le trave !

\textsc{Lisnic} : Allons, allons, je vous en prie ! Tout se passait bien jusqu’à présent, ne m’obligez à intervenir.

\textsc{Nathan} : On t’a rien demandé, à toi, le Moldave pourrave !

\textsc{Greg} : Nathan, s’il te plaît.

\textsc{Lisnic} : Je vais faire comme si je n’avais rien entendu.

\textsc{Nathan} : Bien sûr, que t’as rien entendu. De toute façon, t’es sourd comme un pot. Un pot de chambre moldave !

\textsc{Achaari} : Chef !

\textsc{Lisnic} : Quoi ?

\textsc{Achaari} : On nous signale une tentative de viol à quelques rues d’ici. Apparemment la fille rentrait chez elle en maillot de bain, après une soirée bien arrosée, et elle s’est fait agresser par un type déguisé en sanglier.

\textsc{Lisnic} : Comment ça ? Avec les sabots et tout et tout ?

\textsc{Achaari} : Non, juste la tête. Un masque de sanglier, si vous préférez.

Sally, à Sam : HOMOPHOBE !

\textsc{Sam} : TA GUEULE, LE TRAVELO !

\textsc{Sally} : TRISOMIQUE !

\textsc{Greg} : Tout ça va trop loin. Djef, fais quelque chose !

\textsc{Lisnic} : Oui, faites quelque chose, mon commandant. La situation est en train de déraper.

\textsc{Achaari} : On y va, chef ? On a assez perdu de temps avec cette bande de cinglés.

\textsc{Sam} : Dis donc, toi, le bouffeur de couscous, remballe ta merguez et va faire joujou dans la semoule !

Moi, avec la voix de Pierre Thau, inoubliable interprète des plus grands rôles de basse du répertoire (Méphisto, Don Quichiotte, la statue du Commandeur, le Comte des Grieux et Don Pedro, pour ne citer que les plus fameux) : VOUS ALLEZ LES FERMER VOS GRANDES GUEULES, OUI OU MERDE !!!!!!!!!!!!!! NOM DE DIEU DE BORDEL DE MERDE !!!!!!!!!! JE COMMENCE VRAIMENT À EN AVOIR PLEIN LE CUL DE VOS CONNERIES !!!!!!!!!!!!!!

\textsc{Greg} : Il a les raison, les gars, vous faites chier !

Moi, à Sally : Et vous, allez-vous-en, votre présence ne fait qu’envenimer la situation. (Aux flics :) Ça vaut aussi pour vous, bande de nazes, avec tout le respect que je vous dois.

\textsc{Lisnic} : Ne vous en faites pas, mon commandant, on est sur le départ.

\textsc{Achaari} : On a cette affaire de fille en maillot de bain violée par un sanglier rue Timothée Carbonneau qui requiert toute notre attention.

\textsc{Moi} : C’est tout à votre honneur !

\textsc{Lisnic} : Oui. On serait bien resté à bavarder avec vous, mais le devoir nous appelle. Un viol en soi c’est pas si grave, surtout quand les risques de grossesse sont limités par l’hybridation, mais le problème c’est qu’on observe souvent un effet «trainée de poudre» où tout le monde se met à violer tout le monde dans une espèce de frénésie sexuelle incontrôlable. On a vu des ménagères de moins de cinquante ans se faire violer par des chihuahuas, des majorettes par des étalons en rut, des écolières par des éléphants de mer échappés du zoo, heureusement la plupart du temps trop lents pour les poursuivre efficacement, et même des macaques roux abusés sexuellement par des langurs de Hanuman - à Shimla, notamment, où cette vermine pullule -, des racailles de banlieue par des fils de bonnes familles aux yeux injectés de sang et aux bourses anormalement dilatées, j’en passe et des meilleurs, les champs du possible étant quasiment illimités dans ce domaine de compétence. L’amour seul peut nous sauver, j’en sais quelque chose en tant qu’oracolophile distingué.

\textsc{Achaari} : Hamdullah !

\textsc{Lisnic} : J’ajoute que l’animal en question, décrit par les témoins comme une bête de forte corpulence aux réactions imprévisibles, apparemment dotée d’un membre énorme, surdimensionné tant par la longueur que le diamètre, capable d’assommer un bœuf aussi sûrement d’une barre de fer, est en fuite, en train d’errer quelque part dans le quartier en laissant derrière lui une traînée de liquide spermatique, chose qui ne va pas sans poser problème pour la sécurité de nos concitoyens.

\textsc{Moi} : Je dois reconnaître que vous avez idéalement narré la chose, mon ami.

\textsc{Greg} : Quel talent !

\textsc{Sarah} : C’est pas tous les jours qu’on croise un flic moldave qui maîtrise aussi bien la langue de Molière.

\textsc{Lisnic} : Je vous remercie tous du fond du cul.

\textsc{Achaari} : Hamdullah !

\textsc{Lisnic} : Mais le temps presse. J’ai peur, si on tarde à intervenir, qu’une milice populaire armée jusqu’aux dents ne tente de se substituer aux forces de l’ordre.

\textsc{Moi} : Assurément.

\textsc{Lisnic} : Quelle joie de se lever aux aurores pour exercer un aussi beau métier que le nôtre, et se coucher à point d’heure avec la satisfaction du devoir accompli.

\textsc{Moi} : Bien peu de gens, dans cette société qui a perdu ses valeurs et se délite à vue d’œil, peuvent se vanter d’avoir une existence aussi palpitante que celle du fonctionnaire de police.

\textsc{Achaari} : Protéger et servir, telle est notre mission.

\textsc{Greg} : Je reconnais, en tant que privé, qu’il m’arrive parfois d’envier la fonction publique dans ce qu’elle a de plus noble et désintéressé.

\textsc{Moi} : Au revoir, messieurs.

Lisnic, au garde à vous : Force et honneur, mon commandant.

Achaari, dans la même position : Protéger et servir, telle est notre devise.

\textsc{Moi} : Amen.

Après avoir claqué des talons et exécuté un demi-tour quasi parfait, les deux guignols sont remontés dans leur caisse et partis sur les chapeaux de roues vers de nouvelles aventures. Quelle vie palpitante que celle du gardien de la paix, toujours sur la brèche pour secourir la veuve et l’orphelin, l’honnête homme persécuté par une vermine tenace, la joggeuse importunée par des pervers sexuels et autres sociopathes avides de sexe et de pouvoir, les deux étant souvent les pis d’une même mamelle tuméfiée d’affirmation obsessionnelle de sa personnalité inexistante, de revanche sur une vie ingrate qui traite ses rejetons comme de la merde et les expose sans cesse aux pires vicissitudes, poussant le vice jusqu’à leur faire miroiter une récompense bien méritée s’ils souffrent en silence jusqu’à leur dernier souffle.

Sally, remettant ses nichons en place dans son soutien-gorge débordé : Eh bien puisque c’est comme ça, que tout le monde se fiche comme d’une guigne du respect de mes droits les plus élémentaires à l’égalité et la différence, je m’en vais moi aussi. Mais soyez certain, Mr Lussier, que je suis extrêmement déçue par votre comportement et surtout celui de vos amis. Je pensais avoir affaire à des gentlemen, je me suis bien trompée.

\textsc{Greg} : Je vous rappelle, chère madame, que j’attends toujours le règlement du solde de mes honoraires.

\textsc{Sally} : Tu peux t’assoir dessus, mon grand.

\textsc{Greg} : Pardon ?

\textsc{Sally} : C’est comme Tiago Alvarez. Il s’agissait soi-disant de venger sa mort, mais ça aussi tout le monde s’en fout !

Greg, perdant quelque peu le contrôle de ses nerfs d’acier, au point d’employer des mots d’une certaine grossièreté, ordinairement absents de son vocabulaire : Tu vas me payer ce que tu me dois, morue !

Moi, émergeant peu à peu des vapeurs brumeuses de l’alcool : Greg, s’il te plaît. Tu ne vas pas t’y mettre, toi aussi.

Greg, manifestement en proie à cette forte excitation qui le dominait dès lors qu’il s’agissait d’évoquer l’aspect financier de ses activités : Je n’ai pas pour habitude de bosser pour des prunes.

Sally, après un ricanement d’une méchanceté absolue, ce genre de ricanement sardonique d’où toute humanité semble avoir effacée par la main même de Satan : Bosser !!! Laissez-moi rire !

Greg, prêt à fondre sur sa proie telle la harpie féroce sur le paresseux tridactyle assoupi dans la jungle guyanaise (ou - pour reprendre une expression chère à Simone de Bavoir, alors qu’elle n’était pas encore le castor de Jean-Paul Tartre - le jaguar richement orné sur le stylistiquement insignifiant pécari à lèvres blanches, bien trop occupé à retourner bruyamment le terre avec son groin humide pour se rendre compte du danger qui le menace) : Je vais buter cette salope !

Nathan, peinant à cacher sa joie de voir la situation se dégrader encore un peu plus : C’est ce qu’on aurait dû faire dès le début.

Sam, se frottant les mains et se pourléchant les babines, des éclairs de haine joyeuse plein les yeux (j’emploie à dessein le mot «joyeuse» car la perspective de faire subir à son prochain les pires atrocités était synonyme pour lui d’ambiance festive et hautement récréative, tendance qui préexistait chez lui depuis la tendre enfance, s’était très largement accrue au cours des longues années en territoire hostile, et pouvait à présent s’épanouir en toute liberté dans ses activités sécuritaires à caractère privé) : Laissez, je m’en charge !

Votre serviteur, qui, après avoir un temps formé le projet d’éradiquer les Disciples de la Colère (Jégou, Monteil et Desmarais) de la surface de la terre, ne rêvait plus à présent que d’une seule chose : regagner prestement ses pénates, se glisser subrepticement sous la couette - tel un serpent à sonnette ayant passé la journée à ramper sous le soleil brûlant de l’Arizona, serrer son corps recru (adjectif peu usité auquel j’aimerais redonner un semblant de vie, comme ça, même si je m’en fous royalement par ailleurs, des choses autrement plus graves ayant lieu actuellement dans le monde, ce que je veux dire par là c’est qu’on n’en est pas forcément à un adjectif près dans la langue française, riche par nature, peut-être trop, même si, on le sait, des choses en apparence infimes, des détails de l’Histoire, peuvent engendrer de durables turbulences, lesquelles gagnent en intensité au fil du temps, des siècles si nécessaire, voire des millénaires, et finissent par conduire à des catastrophes majeures qui elles-mêmes, par une étrange distorsion de la perception humaine, peuvent apparaître comme des déflagrations épiphénoménologiques sans réelle consistance, des flatulences civilisationnelles) contre celui de sa dulcinée, et se laisser lentement glisser dans la tiédeur réparatrice d’un sommeil bien mérité : Non, Sam, personne ne va se charger de personne ! Pas pour l’instant, en tout cas.

\textsc{Sam} : Faut que je tue quelqu’un, tout de suite !

\textsc{Moi} : Un peu de patience, mon garçon. Chère madame Robinson, je comprends votre déception. Moi-même, si j’étais à votre place, je crois que je serais extrêmement déçue. Enfin, déçu. Tiago Alvarez, votre amant…

\textsc{Sally} : L’amour de ma vie, vous voulez dire !

\textsc{Moi} : L’amour de votre vie, si vous voulez, a été refroidi par une bande de brutes épaisses qui ne méritent aucune clémence.

\textsc{Greg} : Refroidi, c’est une façon de parler.

\textsc{Moi} : Okay. Je retire refroidi, et je remplace par euh… réchauffé…

\textsc{Greg} : Je dirais plutôt carbonisé.

\textsc{Moi} : Réchauffé à mort, si tu préfères. Excuse-moi, mais on ne va pas passer la soirée à jouer sur les mots !

\textsc{Greg} : Assurément. Ne serait-ce que par égard pour Sally qui est toujours sous le coup de la plus vive émotion.

\textsc{Moi} : La plus vive.

\textsc{Sally} : Je vous déteste !

\textsc{Moi} : Désolé, Sally, mais je crois que vous avez compris le fond de ma pensée. Au cas où ce ne serait pas le cas, je vous donne ma parole que Tiago Alvarez sera vengé comme il se doit. À ce propos, j’aimerais, si vous le permettez, faire part à l’assemblée d’une petite idée qui m’est venue pour solutionner le problème.

\textsc{Sam} : Moi aussi !

\textsc{Moi} : Quoi, toi aussi ?

\textsc{Sam} : J’ai une idée pour solutionner le problème.

\textsc{Moi} : Génial. Et quelle est-elle ?

\textsc{Sam} : On bute la grosse et on rentre chez nous !

\textsc{Moi} : Négatif. Ton idée c’est de la merde, et tu commences à me casser sérieusement les burnes ! Je ne te cache pas que je suis assez déçu par ton comportement.

\textsc{Nathan} : Il a été malade quand il était petit.

\textsc{Moi} : Genre les oreillons ?

\textsc{Nathan} : Pire. Son père le battait et sa mère faisait des passes pour arrondir les fins de mois.

\textsc{Sam} : Qu’est-ce que tu racontes ?

\textsc{Nathan} : Quoi, c’est pas vrai ?

Sam, après qu’un vol de corbeaux ait traversé le ciel nuageux de son cerveau : Euh… si, plus ou moins, mais je tiens pas forcément à ce que tout le monde le sache.

\textsc{Greg} : Il faut nommer les choses pour avoir une idée claire de ce qu’elles sont.

\textsc{Sam} : Quoi ?

\textsc{Greg} : Rien, laisse tomber.

\textsc{Nathan} : Sam, je voulais juste dire que t’es un mec pas toujours facile à gérer.

\textsc{Moi} : C’est sans doute doute pour ça qu’il n’a pas fait carrière dans l’armée.

\textsc{Sam} : Quoi ? J’ai pas fait carrière dans l’armée ?

\textsc{Moi} : Pas plus que ça, non.

\textsc{Sam} : Je suis capitaine, je te rappelle.

\textsc{Greg} : Croquemitaine, à la rigueur.

\textsc{Sam} : Très drôle ! Si je suis parti de l’armée, c’est uniquement parce que j’estimais que mes qualités n’étaient pas récompensées à leur juste valeur.

\textsc{Greg} : Ben voyons !

\textsc{Moi} : Non, Sam. Si t’es parti de l’armée, comme tu dis, c’est parce qu’on t’a foutu dehors pour alcoolisme et comportement déplacé envers les jeunes recrues.

\textsc{Nathan} : Et en plus, tu t’amusais à clouer des animaux morts sur des troncs d’arbre !

\textsc{Greg} : Ouais, comme Jeffrey Dahmer !

\textsc{Sam} : Vous insinuez quoi, là ?

\textsc{Greg} : Rien.

Nathan, remonté : T’as jamais tué des chats avec la 22 de ton père quand t’étais petit ?

Sam, gêné : Peut-être une fois ou deux, je me rappelle plus.

\textsc{Nathan} : Après tu les ramenais discrètement dans ta chambre et tu les disséquais.

\textsc{Sam} : Oui, peut-être. C’est loin, tout ça.

\textsc{Nathan} : Avant de les plonger dans l’acide sulfurique pour voir combien de temps il fallait pour les dissoudre complètement.

\textsc{Sam} : Tous les gosses font ça. On leur apprend à disséquer des rats et des grenouilles en cours de sciences naturelles.

\textsc{Greg} : Non, tous les gosses ne butent pas des chats à la 22 pour les dissoudre dans de l’acide.

\textsc{Sam} : Les gosses sont cruels avec les animaux. T’as jamais arraché les ailes des mouches, toi ?

\textsc{Greg} : Arracher les ailes des mouches, passe encore, même s’il faut être franchement débile pour s’amuser à ça. Mais tuer des chats à la 22 pour les disséquer et les dissoudre dans de l’acide, on change clairement de catégorie !

\textsc{Moi} : Le bien et le mal sont des concepts qui n’existent pas dans la nature. La seule chose qui compte, c’est d’assurer la survie et le développement de l’espèce.

\textsc{Greg} : D’accord, mais l’espèce humaine est un cas particulier. Il a fallu créer des garde-fous pour limiter la casse. Et même avec ça, on a du mal à empêcher les gens de s’entretuer.

\textsc{Moi} : L’homme n’est pas prêt pour la liberté. Trop dangereux. Ce serait comme laisser un loup affamé errer au milieu d’un troupeau de moutons bien gras et dodus. Il faut lui mettre une muselière, le trimballer en laisse et passer derrière lui pour ramasser son caca.

Tout en parlant, et m’écoutant parler avec une certaine délectation (la teneur de mes propos me ravissait autant que le son de ma voix, aussi chaude et enveloppante qu’une couche de pâte feuilletée autour d’un rôti de bœuf), j’ai avisé Sally qui tentait de profiter de l’occasion pour se faire discrètement la malle.

J’aurais pu la laisser partir, trop content de m’en débarrasser, mais le sens de l’autre et l’amour du travail bien fait qui m’habitent en permanence me permettent rarement de céder à la facilité.

Je l’ai donc stoppée dans son élan : Vous partez déjà ?

Elle s’est arrêtée, retournée, et a rétorqué d’une voix étouffée par la rage et la déception : Je crois que je n’ai plus rien à faire ici.

J’ai dit : Je comprends que vous soyez dépitée. Mais je vous en prie, revenez ici, j’ai encore des choses à vous dire.

\textsc{Sally} : Je ne vois pas quoi. Mais si vous insistez.

\textsc{Moi} : Non seulement j’insiste, mais je vous donne ma parole que votre ami sera vengé.

Sally, au bord des larmes : C’était bien plus qu’un simple ami.

\textsc{Moi} : Je le sais, et vous présente une fois de plus, en mon nom et ceux de mes camarades ici présents, mes plus sincères condoléances. Mais j’aimerais maintenant, si vous le permettez, revenir à l’idée que je souhaitais soumettre à l’assistance. Comme vous n’êtes pas sans le savoir, les principaux suspects dans cette sombre affaire sont les sieurs Jégou, Monteil et Desmarais, tristes sires s’il en est, qui forment le noyau dur d’un groupuscule d’extrême-droite connu sous le nom de Disciples de la Colère. Je dirais plutôt Disciples de la Connerie. Quand ils ne sont pas en train de se saouler la gueule au Bouclier, un bar facho de la rue de Gobineau, ils se retrouvent rue Jordan Peshkov, chez Jégou qui fait office de cerveau de la bande, ou plutôt chez sa grand-mère puisque c’est là qu’il habite, dans le sous-sol transformé en bunker nazi, pour préparer la troisième guerre mondiale et le retour triomphal du führer qui attendait patiemment son heure dans un caisson de cryogénisation. En attendant ce jour de gloire, ils bricolent des complots racistes, antisémites et homophobes pour déstabiliser la société et saper les fondements de la démocratie. La vieille, sourde comme un pot et toute acquise aux thèses négationnistes de son petit-fils adoré, qu’elle a élevé depuis sa plus tendre enfance, ses parents s’étant barrés en vitesse dès qu’ils ont compris qu’ils avaient engendré un monstre, leur mitonne des petits plats pour qu’ils ne crévent pas de faim pendant les longues soirées d’hiver passées à refaire le monde, ou plutôt imaginer le plus sûr moyen de le réduire en cendres. Pour en revenir à l’affaire qui nous occupe, Cerqueira, qui était présent au moment des faits, affirme avoir tout tenté pour empêcher Desmarais de commettre le pire. Mais autant essayer de retirer un os de la gueule d’un dogue allemand ! Plombier de métier, cette sous-merde est aussi pompier volontaire. Pas pour sauver des gens, je vous rassure tout de suite, mais pour assouvir sa fascination pour le feu. Voir cramer des choses le met en joie, lui déclenche des érections d’anthologie, lui procure des orgasmes apocalyptiques, choses d’autant plus rares, exceptionnelles et profitables qu’il souffre d’impuissance chronique. C’est au contact des flammes qu’il s’épanouit, et sa grande passion consiste à faire semblant d’éteindre des incendies qu’il a lui-même allumés. Si on en croit Cerqueira, et je suis assez enclin à le faire, c’est Desmarais qui a carbonisé Alvarez au lance-flammes. Ils ont tous, de près ou de loin, participé à cette abomination, mais c’est à Desmarais que revient la palme du pire taré de service. Je vais donc m’en occuper personnellement, à mon rythme, et corriger le tir de dame Nature qui s’est une fois de plus fourvoyée sur la longue route semée d’embûches de l’évolution. Je propose que Sam s’occupe de Jégou et Greg et Nathan de Monteil.

\textsc{Nathan} : Je vais me le faire au Cougar MT-6, avec des pointes de flèches explosives.

\textsc{Greg} : Je ramasserai les morceaux.

\textsc{Sam} : Moi je vais me le faire à l’ancienne, à mains nues. Je vais lui déboiter les articulations une par une, et ensuite je lui briserai gentiment la nuque pour mettre un terme à ses souffrances.

\textsc{Moi} : Excellent. Pour ma part, je n’ai pas encore de modus operandi précis en tête, mais je vais essayer de trouver quelque chose de sympa, si possible en rapport avec sa passion pour Hitler, le suprémacisme et la destruction méthodique de son prochain.

\textsc{Greg} : Et Titus ?

\textsc{Moi} : Ça fait déjà trois fois que j’essaie de l’appeler.

\textsc{Greg} : Il fait quoi, à ton avis.

\textsc{Moi} : Il est prisonnier de la Gardienne de la Nuit, une déesse surgie des entrailles de la terre pour lui sauver la vie.

\textsc{Greg} : Tu crois ?

\textsc{Moi} : C’est comme ça que je vois les choses. En attendant, s’il n’est rentré à la maison à la première heure, Bérénice va me tomber dessus et je vais avoir toutes les peines du monde à lui expliquer ce qui s’est réellement passé.

\textsc{Greg} : Tu sais ce qui s’est réellement passé ?

\textsc{Moi} : Pas vraiment, non.

\textsc{Sam} : Si on allait le chercher ?

\textsc{Moi} : Où ça ?

\textsc{Nathan} : La fille a parlé d’un hôtel tout près d’ici.

En tant qu’homme moderne parfaitement connecté au monde réel et toujours au fait des dernières avancées en matière de technologies de pointe, Greg a aussitôt sorti son smartphone, appareil sur lequel il disposait de toutes les applis nécessaires pour quadriller le périmètre au dixième de millimètre près : Attends, je regarde s’il y en a un dans le coin.

Moi, estimant que le moment d’allumer un Gurkha Ghost Spook était amplement arrivé : Alors ?

\textsc{Greg} : Il y en a un à deux pas d’ici, le Caribbean Hôtel, rue des Maléfices.

Moi, savourant avec une délectation indécente l’épaisse fumée de mon cigare : Ça ne me dit rien qui vaille.

\textsc{Sam} : Sans doute un hôtel de passe.

\textsc{Nathan} : On devrait aller y faire un tour.

\textsc{Moi} : Je ne suis pas chaud. Après tout, Titus est majeur et vacciné. Il a parfaitement le droit de faire des folies de son corps si bon lui semble.

\textsc{Greg} : Je te rappelle qu’il a une femme et des enfants.

\textsc{Moi} : Merci, je suis au courant.

\textsc{Greg} : Et qu’il n’avait l’air dans son état normal quand il s’est fait embarquer par cette fille.

\textsc{Nathan} : Très bizarre, cette fille.

\textsc{Sam} : Ça ne coûte rien d’aller y jeter un œil.

\textsc{Moi} : Vite fait, alors.

\textsc{Sally} : Ce sera sans moi. Bonne nuit. Et n’oubliez pas, monsieur Lussier, que nous avons une affaire en cours.

\textsc{Greg} : Vous vouliez savoir qui a tué Alvarez, vous le savez. Le reste ne me concerne en aucune manière.

\textsc{Sally} : Vous n’aurez pas un centime de plus tant que cette ordure sera en vie. Je suis même prête à doubler la mise si vous me donnez satisfaction.

\textsc{Greg} : C’est du chantage pur et simple.

\textsc{Sally} : Appelez ça comme vous voudrez. Bonne nuit, messieurs.

\textsc{Sam} : Bon, on fait quoi ? On va à l’hôtel ou pas ?

Moi, après un temps d’hésitation : Okay, on y va !


\chapter{Acte 7}

\noindent Et on y est allé, figurez-vous, car il nous arrivait parfois, quand on n’avait rien de mieux ni de plus pressant à faire, de faire ce qu’on avait dit qu’on allait faire, chose qui, dans le cas présent, était loin de tomber sous le sens, car j’avais personnellement beaucoup mieux et plus pressant à faire que courir à la recherche d’un ami d’enfance enlevé sous mes yeux par une mystérieuse créature, très belle au demeurant, pour ne pas dire fascinante, mais dont les intentions véritables étaient loin d’être d’une clarté aveuglante. Ce que j’avais de mieux et autrement plus pressant à faire, c’était, vous l’aurez compris, de foncer chez moi ventre à terre pour retrouver enfin Zarina, belle Italienne aux contours immortels et avec des braises à la place des yeux qui avait saisi mon cœur — et le reste — entre ses griffes puissantes et ne semblait aucunement décidée à les lâcher. D’autant moins, je l’avoue à ma grande honte, que je ne faisais pas le moindre effort pour tenter de m’en affranchir. Bien au contraire, c’était avec une lascivité répugnante que je m’abandonnais totalement à l’emprise sexuelle et affective qu’elle exerçait sur mon humble et pathétique personne. Sa longue chevelure moirée exerçait sur moi un attrait puissant, et c’est telle une limace ruisselante de bave que je rampais à ses pieds menus, léchant goulûment ses chevilles d’une finesse exquise, lapant jalousement la moindre de ses sécrétions, reniflant désespérément l’air parfumé de ses fragrances enchanteresses, ronronnant tel un chaton orphelin à la moindre de ses caresses, étalon en rut hennissant dès que le bout de ses doigts effleurait mon épiderme hypersensible, agneau tondu de frais offrant sa gorge au sacrifice en bêlant de joie comme le dernier des imbéciles, mortifiant mes chairs luxurieuses pour tenter d’en extirper le vice et sauver mon âme faisandée des barbecues de l’enfer. J’étais, en un mot comme en cent, réduit à un état de loque humaine ayant sciemment renoncé à toute espèce de respect de soi-même, sens critique et semblant de dignité. Songez que je pouvais, à des lieues à la ronde, tel un limier surentraîné, sentir son odeur et percevoir le souffle régulier de sa respiration nocturne, deviner, moulées par le fin tissu des draps, les courbes affolantes de son corps de rêve, fraîche oasis au sein de laquelle il me tardait d’aller me ressourcer après une longue et quotidienne traversée du désert de l’existence.

Oui, au cas où vous ne l’auriez pas remarqué, il se trouve que je suis aussi un très grand écrivain de la passion amoureuse, descripteur minutieux des suggestions voluptueuses qui vous sucent les neurones à la paille et pondent leurs œufs dans le cerveau ramolli des amants telles des mouches à viande dans la dépouille d’un rat crevé ou un chat écrasé, et je tenais à le préciser, en toute modestie. Voilà, c’est fait. Et rassurez-vous, je ne compte pas m’étendre sur le sujet plus que nécessaire.

Le Caribbean Hôtel ne se trouvait pas à plus de cinq cents mètres de l’endroit où nous étions. On aurait pu y aller à pied, bien sûr, comme des écologistes responsables alliant saine pratique sportive et conscience aigüe de l’environnement, mais non seulement on n’en avait strictement rien à foutre (la majeure partie d’entre nous estimant que la présence de l’être humain sur terre touchait à son terme et que rien ne pourrait plus empêcher l’inéluctable, même si on décidait de faire machine arrière et sautait à pieds joints sur la pédale de frein), mais surtout on en avait plein les pattes d’une journée qui s’éternisait au-delà du raisonnable, raison pour laquelle nous avons jugé que le mieux était encore de remonter dans le G30 pour nous y transporter.

L’établissement, niché au creux d’un quartier connu pour abriter une forte population noire (mais je parle ici d’une population noire de qualité supérieure, pas de zombies accros à la xylazine, non, je parle de gens éduqués à forte valeur ajoutée, anesthésistes-réanimateurs, profs d’université, cadres supérieurs, assistants de production, directeurs financiers, experts-comptables, chefs de produit, community managers, web designers, traders, architectes, plus évidemment quelques dealers en costard-cravate et putes haut de gamme pour permettre à tout ce petit monde de s’encanailler un peu), était la réplique à quelques détails près du couvent des Ursulines de la Nouvelle-Orléans, œuvre du capitaine Broutin, ingénieur militaire, cartographe et architecte officiel de Louisiane dans les années 1700. À l’origine destiné à l’éducation des jeunes filles de bonnes familles, lesquelles étaient pour la plupart des petites garces échevelées qui ne pensaient qu’à jouer à touche-pipi, il s’est peu à peu transformé en lupanar fréquenté par les esclavagistes du coin, aussi nombreux que les morpions dans le slip d’un légionnaire.

Le Caribbean, d’inspiration latino-créole revue et corrigée par les châteaux de la Loire, le gothique flamand et Ludwig Mies van der Rohe, ne s’étalait pas aussi largement dans l’espace que son glorieux modèle de Pelican State (le pélican brun, je le rappelle au passage pour celles et ceux qui s’attendraient à trouver des pélicans en train de barboter dans la piscine de l’Eden Roc, est un volatile endémique de cette région du globe). En compensation, il s’était vu adjoindre quelques étages de plus, ce qui lui conférait une forme de majesté supplémentaire. Cela dit, une bonne séance de réfection n’aurait pas été superflue pour lui redonner son lustre d’antan, même si les petites imperfections et lézardes qui se manifestaient ici et là contribuaient indéniablement à son charme suranné.

En tant que chef d’expédition, c’était à moi que revenait la charge de pénétrer en premier dans la place.

Il fallait appuyer longuement sur un bouton en forme de furoncle et se positionner idéalement sous l’œil de verre de la caméra de surveillance pour espérer obtenir une réponse, l’hôtel étant à priori fermé à cette heure avancée de la nuit. Seule la clientèle, dûment équipée pour aller et venir à sa guise, pouvait s’abstenir de ces contraintes sécuritaires.

Quelques longues minutes plus tard, une voix s’est fait entendre dans l’interphone : Qui êtes-vous ? Vous avez perdu votre clé ?

À laquelle j’ai répondu, en exhibant ma carte : Commissaire Beauvais, police judiciaire.

\textsc{La voix} : Désolé, nous n’avons plus de chambre.

\textsc{Moi} : J’ai de bonnes raisons de penser qu’il se passe des choses bizarres dans votre hôtel.

\textsc{La voix} : Qui sont ces gens qui vous accompagnent ?

\textsc{Moi} : Mes assistants. Je vous demanderai d’ouvrir cette porte et de répondre aux quelques questions que j’aimerais vous poser.

\textsc{La voix} : Il est tard. Revenez demain.

\textsc{Moi} : Dois-je en conclure que vous faites volontairement entrave à l’action de la justice ?

\textsc{La voix} : Vous avez un mandat, monsieur le policier ?

\textsc{Sam} : Ouvre la porte, connard ! Sinon je la défonce, et je te défonce la gueule après !

\textsc{La voix} : Il n’est pas question que ce grossier personnage mette un pied dans mon établissement.

\textsc{Moi} : Sam, s’il te plaît, ferme-la. Je suis désolé, cher monsieur, mais nous avons eu une journée extrêmement fatigante, pour ne pas dire harassante et pleine de rebondissements inattendus. Je vous demande juste d’ouvrir cette porte et de bien vouloir répondre à quelques-unes de mes questions, rien de plus.

\textsc{La voix} : Je comprends. Mais nous avons une procédure très stricte, ici. Il nous est interdit de laisser entrer n’importe qui à une heure aussi avancée de la nuit. Avec les agressions, sexuelles et autres, qui se multiplient dans le quartier depuis un certain temps, nous sommes obligés de nous montrer extrêmement prudents. Nous avons la responsabilité de nos clients, nous nous devons d’en assurer la sécurité. C’est aussi pour ce genre de service qu’ils acceptent de payer un prix aussi élevé pour avoir le privilège de résider dans nos murs. J’ajoute, au cas où vous ne le sauriez pas, que cet établissement est interdit aux Blancs.

\textsc{Moi} : C’est une blague ?

\textsc{La voix} : Pas du tout. Nous sommes officiellement rattachés à l’ambassade de Namibie, dont nous sommes en quelque sorte une annexe résidentielle. Nous jouissons d’un certain nombre de privilèges parmi lesquels celui de choisir notre clientèle. Cet hôtel est interdit aux Blancs.

\textsc{Sam} : Interdit aux Blancs ! On aura tout vu !

\textsc{La voix} : Exclusivement réservé aux gens de couleur, si vous préférez. Voici les faits : un jour, en 1884, un explorateur prussien du nom de Gustav Nachtigal, représentant de Bismarck, débarque dans le golfe de Guinée pour annexer des territoires. Un an plus tard, son successeur, Heinrich Göring, le père d’Hermann, arrive en Namibie avec la ferme intention de faire main basse sur ses richesses et réduire ses habitants à l’esclavage. Depuis, vos semblables, car cela vaut aussi pour les Français et les Européens en général, n’ont cessé de nous persécuter, nous dépouiller, nous humilier. Vous comprendrez que nous ayons aujourd’hui, en dépit de vos efforts pour faire table rase du passé et des quelques vagues compensations que vous daignez nous accorder de temps à autre, quelques réticences à votre égard. Je n’irais pas jusqu’à dire que nous vous détestons profondément, mais nous ne vous aimons pas beaucoup.

\textsc{Moi} : J’entends bien, cher monsieur, et vous présente mes plus plates excuses concernant les erreurs de mes ancêtres.

\textsc{La voix} : Les horreurs, vous voulez dire !

\textsc{Moi} : Oui, si vous voulez. Reste que nous sommes ici dans le cadre d’une enquête policière de la toute première importance, en lien étroit avec la lutte contre le racisme et l’homophobie.

\textsc{La voix} : Vraiment ?

\textsc{Moi} : Mais oui.

\textsc{La voix} : Dans ce cas, revenez avec un mandat en bonne et due forme. Je verrai alors ce que je peux faire, même si légalement rien ne m’oblige à donner suite à vos exigences. Ici, vous êtes officiellement sur le territoire de Namibie et vos lois ne s’appliquent pas.

\textsc{Moi} : Il se trouve que nous sommes à la recherche d’un homme de couleur, originaire de Sierra Leone, descendant d’un loyaliste des King’s Dragoons de Freetown, dont nous pensons qu’il pourrait avoir été enlevé et se trouver à l’instant même retenu contre son gré dans votre établissement.

\textsc{La voix} : Rien que ça ?

\textsc{Moi} : C’est déjà pas mal.

\textsc{La voix} : Disons que c’est un bon début. Et qu’est-ce qui vous fait croire que ce personnage serait retenu contre son gré au Caribbean Hôtel, un des établissements les plus sélects de la ville, réputé pour sa clientèle triée sur le volet et la qualité de ses prestations ?

\textsc{Moi} : Le ravisseur est une femme.

\textsc{La voix} : Vraiment ?

\textsc{Moi} : Oui. Une femme de couleur, qui plus est, d’une très grande beauté.

\textsc{La voix} : Tout s’explique, en effet.

\textsc{Moi} : Atiena, Gardienne de la Nuit, ça vous dit quelque chose ?

\textsc{La voix} : Rien du tout.

J’ai sorti mon smartphone et affiché une photo de l’intéressée que j’avais prise discrètement à son insu, dans le plus strict irrespect de son prétendu droit à l’image (ça me fait d’autant plus rigoler que les gens passent leur temps à se prendre en photo sous toutes les coutures), photo que j’ai ensuite exposée à l’œil inquisiteur de la caméra de surveillance : Tenez, c’est elle.

\textsc{La voix} : Très jolie femme, en effet.

\textsc{Moi} : N’est-ce pas.

\textsc{La voix} : Tout à fait remarquable.

\textsc{Moi} : Oui, je dois bien reconnaître que j’ai rarement vu une femme aussi belle. Une femme ou quoi que ce soit d’autre, d’ailleurs.

\textsc{La voix} : Nous autres, hommes de couleur, sommes particulièrement sensibles à la beauté des femmes.

\textsc{Moi} : Je pense que nous sommes particulièrement sensibles à la beauté des choses que nous désirons le plus.

\textsc{La voix} : Vous voulez dire que plus nous les désirons et plus elles sont belles, et non l’inverse ?

\textsc{Moi} : Oui. Le désir rend aveugle, ou sinon aveugle au moins fortement myope, ce qui fausse complètement la perception des choses.

\textsc{La voix} : C’est une théorie comme une autre.

\textsc{Moi} : En tout cas, c’est cette créature qui a enlevé Titus Beaugendre, lequel se trouve être un collègue de travail et un ami d’enfance. Bien qu’il soit de couleur, je le considère comme mon propre frère. Je ne sais pas où j’en serais sans lui. Sans doute au même point, mais la vie n’aurait pas du tout la même saveur. Ensemble, nous luttons contre le vice et la corruption.

\textsc{La voix} : Je vois. Malheureusement, je crains fort de ne pas pouvoir faire grand-chose pour vous.

\textsc{Moi} : Vous n’avez jamais vu cette femme traîner dans les couloirs de l’hôtel ?

\textsc{La voix} : Comment ça, traîner dans les couloirs de l’hôtel ? Vous n’êtes tout de même pas en train d’insinuer que…

\textsc{Moi} : Je n’insinue rien du tout. Elle a parlé d’un hôtel dans lequel elle résidait, et j’ai tout lieu de penser qu’il s’agit du Caribbean.

\textsc{La voix} : Qu’est-ce qui vous fait croire ça ?

Moi, hésitant : Un faisceau d’éléments concordants.

\textsc{La voix} : Mais naturellement vous n’avez aucune preuve de ce que vous avancez ?

\textsc{Moi} : Naturellement. Si j’en avais, je serais venu avec un mandat de perquisition en bonne et due forme. Alors que là j’arrive les mains dans les poches, en comptant sur votre bonne volonté pour m’aider à résoudre cette affaire.

\textsc{La voix} : Désolé, je ne peux rien vous dire de plus.

\textsc{Moi} : Vous ne connaissez pas cette femme ?

\textsc{La voix} : Bien sûr, que je la connais. Tout le monde la connaît dans notre petite communauté.

Sam, incapable de tenir sa langue : Qui, la clocharde ?

\textsc{Moi} : La ferme, Sam !

\textsc{La voix} : Ne vous en faites pas, je ne prête pas plus d’attention à cet individu que s’il s’agissait d’un moucheron écrasé sur la pare-brise de ma voiture.

\textsc{Sam} : Vous avez une voiture ?

\textsc{La voix} : Eh oui. Ça vous étonne, pas vrai ?

\textsc{Sam} : Un peu, oui. Je ne savais même pas que les Noirs avaient le droit de conduire.

\textsc{La voix} : Mais je ne suis pas Noir.

\textsc{Sam} : Je croyais que c’était interdit aux Blancs, ici ?

\textsc{La voix} : Je n’ai pas dit que j’étais Blanc. Si vous voulez tout savoir, j’appartiens au groupe ethnique des métis du Cap. Ma mère était française et mon père hottentot, descendant d’un héros de la bataille de Salt River, durant laquelle nombre de nos frères sont tombés pour rejeter les Portugais à la mer. Mais je ne vois pas pourquoi je perds mon temps à vous parler de ça. Pour en revenir à la Gardienne de la Nuit, celle que vous traitez de clocharde, il se trouve qu’il s’agit d’une personne extrêmement influente, très connue dans notre communauté, appréciée, respectée et crainte, car elle dispose de pouvoirs surnaturels hérités de ses ancêtres. Mais au lieu d’utiliser ces pouvoirs dans le seul but de s’enrichir et manipuler son prochain, elle s’efforce de les mettre au service du Bien. C’est une référence pour nous, un bien précieux que nous vénérons et protégeons en permanence. Quoi qu’il en soit, monsieur le commissaire, si comme vous le prétendez elle a «enlevé» votre ami, je peux vous assurer que vous ne tarderez pas à le retrouver en meilleure forme que jamais. Je suppose qu’elle s’est efforcée de le mettre à l’abri face à un grave danger qui le menaçait, voilà tout. Vous vous faites du souci pour rien, il ne peut être en de meilleures mains. Sur ce, je vous souhaite une excellente nuit.

\textsc{Moi} : Soit. Mais je vous préviens que si Titus n’est pas rentré chez lui à la première heure, je reviendrai avec un mandat et passerai votre coupe-gorge au peigne fin.

\textsc{La voix} : Il vous faudrait pour ça obtenir une autorisation préalable et je doute fort que vous y arriviez, tout policier que vous soyez. Nous avons des relations dans les plus hautes sphères du Pouvoir, et il nous suffirait d’un claquement de doigts pour mettre un terme à votre carrière pour une raison quelconque, tracasseries policières, par exemple. Je vous déconseille tout excès de zèle si vous ne voulez pas vous retrouver à faire la sortie des écoles avec un uniforme de gardien de la paix sur le dos. Cela dit, notre politique est de ne pas faire de vagues et nous œuvrons autant que possible dans la discrétion. Soyez tranquille, votre ami sera rentré chez lui à la première heure, en pleine forme, et tout cela ne sera plus qu’un mauvais souvenir que vous oublierez au plus vite. Bonne nuit, commissaire.

Il y a eu un déclic suivi d’un grésillement dont il n’était pas nécessaire d’être doctorant à Polytechnique pour comprendre le sens : foutez le camp en vitesse si vous ne souhaitez pas vous exposer à de gaves sanctions et réduire drastiquement votre espérance de vie.

J’aurais pu résonner, appuyer comme un taré sur le bouton pendant des heures afin de tarauder la cervelle de mon adversaire jusqu’à la folie et le pousser à une reddition sans condition, mais ma petite voix intérieure, celle qui me renseignait habituellement sur les choses à faire ou ne pas faire en cas de difficulté inattendue, qui me disait, par exemple, de détaler comme un lapin si un cordon de CRS arrivait au pas charge dans ma direction au cours d’une manif non déclarée place de la Nation, ma petite voix intérieure m’a vivement conseillé de n’en rien faire.

Il fallait se rendre à l’évidence : le Caribbean Hôtel était un bunker bien gardé, les négociations étaient terminées et il ne servait à rien d’insister, sauf si je tenais absolument à me retrouver à la circulation et passer le restant de mes jours à aider des petites vieilles à traverser la rue, dresser des PV pour stationnement gênant et faire souffler des gens dans des éthylomètres.

Autant dire que c’est le pas traînant et la mine déconfite qu’on est tous gentiment remontés dans le van, lequel, sans doute lui-même dans un état de dépression mécanique avancée, bielles et pistons baignant dans l’huile rance de la déconfiture, a finalement, après une longue série de rots, pets, râles déchirants, éructations sinistres et autres manifestations sonores aussi intempestives que déplaisantes, consenti à démarrer.


\chapter{Acte 8}

\noindent Trois quarts d’heure plus tard, après avoir déposé Sam chez lui (humainement parlant, c’était à peu près l’équivalent d’une grenade dégoupillée en permanence, raison pour laquelle j’étais~-- et je n’étais pas le seul, Greg partageant très largement mon opinion sur le sujet, et Nathan aussi dans une moindre mesure sans doute liée à son quotient intellectuel réduit et sa façon plus terre-à-terre de voir les choses~-- assez pressé de m’en débarrasser), je me suis retrouvé au 157 rue des Anus en fleur (adresse fantaisiste je le rappelle, étant tenu à la plus extrême discrétion pour tout ce qui touche à ma vie privée), sur le trottoir devant mon domicile, en train de m’échiner sur le digicode de la porte d’entrée de l’immeuble. Je connaissais par cœur le numéro gagnant, bien entendu, mais il m’arrivait de plus en plus souvent de l’oublier, tout ou partie, au même titre d’ailleurs que mon numéro de carte de crédit et tous les numéros en règle générale, y compris les dates de naissance de mes proches et plus proches amis (heureusement en nombre limité). Ce triste état de fait m’avait conduit à les noter, lui et quelques autres, sur un petit morceau de papier que je trimballais discrètement dans le fond de ma poche, soigneusement plié dans le portefeuille flambant neuf que Zarina, désespérée de me voir utiliser envers et contre tout la vieille relique qui m’avait accompagné jusqu’à présent, avait jugé bon de m’offrir. Cette vieille relique avait traversé avec moi maintes épreuves, dont certaines parmi les plus douloureuses de mon existence, et je n’ai pas honte de dire, au risque de passer pour un demeuré, que j’y étais aussi viscéralement attaché qu’un enfant à son doudou. Zarina, bien décidée à couper le cordon quasi ombilical qui me reliait à mon vieux portefeuille, m’avait placé face à un ultimatum sans équivoque : c’était lui ou moi. Choc, séisme, avis de tempête, alerte cyclonique, vents à trois cents kilomètres/heure, pluie diluvienne de grêlons gros comme des œufs d’autruche, la terre s’effondrait sous mes pieds et je plongeais dans un abîme de tumulte et de magma incandescent !!! Face à un tel dilemme, je m’étais accordé un délai de réflexion de quelques semaines avant de donner ma réponse. Pendant ce laps de temps, tel Forest Whitaker dans Ghost Dog, Corey «Raekwon» Woods gravissant un par un les échelons menant à la salle des Cinq Dragons ou encore San Te affrontant le général Sien Ta dans la 36e Chambre de Shaolin, je suis parti au sommet de la montagne afin de méditer sur le sens de la vie, seul, la tête dans les nuages, vêtu d’un simple pagne en raphia dans la plus pure tradition hawaïenne, où j’ai vécu entouré de quelques rares animaux qui sortaient de leurs repaires à la nuit tombante pour m’assister dans ma quête de l’absolu. Elles avaient bien compris, les pauvres bestioles sans avenir, minables créatures entièrement soumises au joug impitoyable de la nature, à sa brutalité bestiale, que j’étais le phare autour duquel il convenait de se rassembler pour entrevoir une vague lueur d’espoir dans les ténèbres du néant. Pour moi la question était simple : étais-je oui ou non un homme de bien, d’amour et de bonté, gorgé de tolérance et boursouflé par l’empathie, bouffi de générosité, tumescent d’affection, abritant un cœur obèse dans sa poitrine gonflée d’orgueil, prêt à toutes les concessions, à renier l’essence même de son être, la substantifique moelle de sa nature profonde, pour faire de sa vie en couple une de ces réussites qui inspirent les générations futures, leur insufflent la conviction que tout n’est pas perdu alors même que les bombes pleuvent sur leurs têtes et que les flammes ravagent le monde. C’est ainsi qu’on a pu, à une époque pas si lointaine, me voir errer dans les forêts hostiles, parler avec les bêtes, danser au clair de lune avec les chouettes et les blaireaux, m’abreuver à la source des ravins, glaner des baies et creuser la terre avec mes ongles crasseux à la recherche de larves et vers de terre pour assouvir ma faim, dormir à même le sol sur un lit de feuilles mortes. Au terme de cet exil, je suis rentré chez moi, ai solennellement aspergé d’essence mon vieux larfeuille adoré, puis j’y ai mis le feu et l’ai regardé brûler sans regret, avec, au plus, une légère touche d’humidité dans le fond de l’œil. Je savais qu’une page du grand livre de ma vie, best-seller de mon histoire personnelle, chronique de la désespérance et recueil de poèmes aux strophes atrophiées, vers solitaires à la rime mal arrimée, rafiot littéraire en perdition, venait de se tourner, et que mon portefeuille-doudou élimé jusqu’à la corde faisait partie de ces objets sacrificiels dont la transsubstantiation cathartique précipite la renaissance de l’initié. J’ai ouvert grand les bras, laissé venir à moi les petits enfants~-- et les simples d’esprit~-- de l’Amour et dit à Zarina, qui était tombée à genoux et pleurait toutes les larmes de son corps en léchant goulûment mes sandales Valentino studshield fisherman en cuir de veau : tu vas tomber enceinte, et un fils tu enfanteras ; et maintenant ne bois ni vin ni liqueur forte (tu parles, Charles !), ni ne mange rien d’impur (ah bon, même pas des beignets de poisson-pénis ou des yeux de mouton à la kazakhe ?), car cet enfant sera consacré à Dieu dès le ventre de sa mère jusqu’au jour de sa mort. Rassurez-vous, je plaisantais, il n’était nullement dans mes intentions de me reproduire, et surtout pas à l’identique. À elle seule, ma présence constituait déjà une insulte au bon goût, un scandale naturel, une aberration évolutionnelle. Malgré l’esprit pervers qui régissait la plupart de mes actes, je ne voulais pas qu’une bande de petits moi-mêmes se répande comme une trainée de poudre à travers le monde, semant la mort et la destruction sur son passage. Bon, je ne dis pas que Zarina n’avait pas quelques petites arrières-pensées en tête, mais il n’était pas question pour moi, au moins pour l’instant, de prêter la moindre oreille à ses supplications. Vous m’imaginez, moi, Djeferson Beauvais, commandant de police dont les décorations prestigieuses s’entassent sur le revers de la veste, unanimement reconnu et salué par sa hiérarchie comme un leader de premier plan, un meneur d’hommes comme on n’en fait plus depuis Alexandre le Grand, Simon Bolivar, Churchill, Gengis Khan, Clemenceau, Hannibal et Charles Quint, vous m’imaginez en train de changer des couches et trimballer une poussette dans les rues de la ville, l’air idiot, un sourire béat sur mon visage horriblement déformé par les joies de la paternité ? De raconter des histoires idiotes à un gamin qui refuse obstinément de fermer l’œil à trois heures du mat ? Me déguiser en père Noël et rester coincé dans la cheminée avec ma hotte sur le dos ? Me ronger les ongles jusqu’au sang à la moindre poussée de fièvre ? Me taper les goûters d’anniversaires, les spectacles de fin d’année ? Me faire virer de chez moi comme un malpropre pour les booms d’ados pendant lesquelles ces petits cons vident le bar, vomissent et chient partout ? Me faire traiter de tous les noms parce que j’ai eu le malheur de faire une réflexion sur la tenue d’untel ou les horaires de sortie de tel autre ? Voir mes gosses raser les murs quand je les emmène à l’école parce qu’il ont les boules de se trimballer avec un vieux con dans mon genre ? Me saigner aux quatre veines pour eux et les voir me planter comme une merde à la première occasion ? Leur enseigner principes et valeurs pour les retrouver en train de dealer du shit à la sortie du collège ? Leur acheter un nouveau smartphone hors de prix tous les deux ou trois mois parce qu’ils se sont assis dessus ou l’ont laissé tomber dans la cuvette des chiottes ? Flipper ma race et tourner en rond comme un fauve en cage parce qu’il est une minuit deux et que ma fille devait rentrer à minuit ? Etre obligé de rencontrer les parents de son petit copain et faire comme si on était les meilleurs amis du monde alors que je sais pertinemment qu’ils ne seront plus ensemble dans trois jours (ma fille et son petit copain, ses parents je ne sais pas) ? Rencontrer les parents du petit copain de ma fille, découvrir que sa mère est une bombe atomique, une arme de destruction massive ultra radioactive, le genre qui porte des sous-vêtements à l’uranium enrichi et vous fait exploser les globes oculaires si vous avez le malheur de poser les yeux sur certaines parties charnues de son anatomie, que naturellement elle n’est pas insensible à mon charme (et comment le pourrait-elle, la pauvresse, loin de moi l’idée de lui jeter la pierre), et qu’il va par conséquent être assez difficile de ne pas sombrer dans le vice et la dépravation dignes des plus vils animaux (au risque de me faire démonter par le mari qui travaille dans le bâtiment, mesure deux mètres, pèse cent trente kilos et n’a jamais lu une ligne de Hume, Locke, Popper ou Aristote) ? M’apercevoir avec effroi que le petit copain de ma fille a quarante-cinq ans de plus qu’elle (et avec satisfaction qu’il est extrêmement riche, fils unique, célibataire et sans enfant) ? Mais aussi, plus généralement, se voir vieillir dans les yeux de ses gosses, les voir grandir, changer et vous pousser lentement dans la tombe pour prendre votre place, se reproduire à leur tour et comprendre enfin les souffrances que vous avez endurées en silence (plus ou moins) durant toutes ces années passées à les chérir et les choyer, lesquelles ont passé à la vitesse de l’éclair, comme une balle qui vous transperce le cœur, comme si le temps s’était accéléré rien que pour vous faire chier, vous empêcher de profiter pleinement de moments rares et précieux que vous ne retrouverez plus jamais, même pas quand vos gosses vous refileront les leurs pour aller au resto ou partir en week-end, sous prétexte qu’il faut aussi que vous profitiez d’eux alors qu’ils sont complètement pourris-gâtés, caractériels, insupportables, et que leurs parents sont trop contents de s’en débarrasser pour souffler un peu, alors que vous, toujours à la traine, à la ramasse, à côté de la plaque, vous les aviez sur le dos sept jours sur sept et vingt-quatre heures sur vingt-quatre, sans personne à qui les refiler pour éviter de sombrer corps et âme dans les affres de la folie, ou alors juste un papy ou un tonton pédophile qui avait la trique en les faisant sauter sur ses genoux et les tripotait dans votre dos (et je ne vous compte pas les dommages collatéraux du type tentatives de suicide et années de psychanalyse à cinquante balles de l’heure). Vous m’imaginez sérieusement vivre tout ça, ces années de cauchemar, uniquement pour faire plaisir à une femme que j’adore, certes, vénère par-dessus tout, pour l’instant, mais dont je serai peut-être en train de disperser les morceaux aux quatre vents dans quelques années ? Non, si je faisais ça, mes gosses ne me le pardonneraient jamais.

J’ai finalement réussi à taper le code, entrer, monter dans l’ascenseur et gagner le sixième sans encombre. J’entends par «sans encombre» que ledit ascenseur, d’une vétusté à toute épreuve, ne s’est pas subitement décroché pour s’enfoncer dans les profondeurs de la terre telle une bombe anti-bunker GBU-57 de l’US Air Force, fascinant concentré de technologie développé par Lockheed Martin, Boeing et la Northrop Grumman Corporation. Si vous êtes un psychopathe qui veut devenir le maître du monde et pense aller s’installer à six pieds sous terre pour se livrer en toute tranquillité à ses petites activités illicites, n’y pensez plus. Avec la GBU-57, les dictateurs en herbe, génies du mal mégalos et autres tarés du même genre ne sont plus en sécurité nulle part. N’imaginez surtout pas que cinquante mètres de béton armé vous protégeront de la justice divine exercée par le bras vengeur et doré à l’or fin de l’HPPM, l’Homme le Plus Puissant du Monde. Elon Musk, qui a choisi d’aller s’exiler dans l’espace pour préparer son grand retour sur Terre, l’a bien compris, ce qui fait de lui l’apprenti-dictateur 2.0 neuroatypique le plus hype de l’univers, même si ses affinités politiques discutables et sa propension psychotique à repeupler le monde avec ses propres rejetons (une quinzaine à ce jour, qui portent tous des noms à coucher dehors) ne plaident pas nécessairement en sa faveur. À propos de la GBU-57, je rappelle au passage que douze de ces sympathiques engins ont été largués sur Fordo dans la nuit du 21 au 22 juin 2025, l’objectif revendiqué par un Donald Trump ivre de puissance et à haute valeur psychiatrique ajoutée étant d’empêcher Ali Khamenei, le guide suprême iranien (dont les dernières mises à jour en matière de droits de l’homme et de la femme en particulier remontent à la civilisation de Jiroft et aux premiers Aryens, je ne parle bien évidemment pas de ces cons de Nazis mais des Aryens de la période védique de l’âge du bronze), de fabriquer des bombes atomiques pour les balancer sur la gueule de Benyamin Netanyahou, autre guide suprême… enfin, premier ministre israélien d’extrême-droite visé par un mandat d’arrêt de la CPI pour crimes de guerre. Au même titre, bien sûr, que les terroristes du Hamas et une bonne partie des dirigeants de cette planète, l’autre partie, tenue par des impératifs commerciaux et une certaine image de marque à préserver, ne pouvant malheureusement pas s’en donner autant à cœur-joie qu’elle le souhaiterait, dans l’attente d’une guerre mondiale qui permettrait enfin à tout le monde de faire joujou avec ses têtes nucléaires et se tripoter la nouille atomique jusqu’à l’explosion finale, l’orgasme actinique fukushimiquement pur, l’ultime éjaculation qui rayerait définitivement toute forme de vie de la surface de la Terre, hormis peut-être quelques vagues espèces primitives à l’avenir incertain.

Je ne pensais pas (non, vraiment pas, tant j’étais certain d’avoir connu mon lot de souffrance pour la journée, une journée qu’il me tardait de voir s’achever comme rarement il m’avait tardé de voir une journée s’achever, alors que vous savez comme moi qu’il existe d’autres journées dont on aimerait qu’elles durent éternellement, ces journées qui font dire que le temps passe trop vite et réfléchir amèrement sur la vacuité de l’existence et la finitude des choses), en arrivant au sixième, qu’il me faudrait encore affronter de terribles épreuves avant de pouvoir enfin me glisser dans la chaleur tiède et réparatrice de mes draps.

J’ai vu une ombre glisser furtivement dans la pénombre du couloir, tel un requin fantôme dans les eaux les plus noires, profondes et mystérieuses de l’océan Pacifique, des eaux dans lesquelles glissent encore, et sans doute depuis la nuit des temps, des créatures dont l’existence défie l’imagination et qu’on croirait tout droit sorties du cerveau détraqué d’un Lovecraft, un Sheridan Le Fanu ou un Montague Rhodes James pour n’en citer que quelques uns.

Cette ombre venait de l’autre bout du couloir, un endroit où peu de gens osaient s’aventurer tant circulaient à son sujet d’étranges légendes (l’immeuble ne datait pas d’hier) et rumeurs de phénomènes surnaturels ayant entraîné un certain nombre de décès inexpliqués, et qui était précisément celui dans lequel Marc-Antoine Jacquinot, le prof de philo neurasthénique au physique de planche à pain maladive et myope, avait choisi de s’établir. Jacquinot lui-même, avec son éternelle serviette en cuir patiné par des décennies de bons et loyaux services, prêtait le flanc avec indifférence aux racontars les plus insidieux tant de la part de ses collègues que de la population estudiantine du lycée dans lequel il exerçait. Profondément philosophe dans l’âme, Jacquinot savait que son séjour sur terre serait de courte durée et n’avait pas de temps à perdre à se prendre le chou avec ses concitoyens. N’étant moi-même pas dépourvu de ce que j’appellerai une certaine forme de pessimisme raisonnable, j’avais tenté à de nombreuses reprises, sans réel (aucun, si vous préférez) succès, de nouer contact avec lui, ne serait-ce qu’au titre des bonnes relations de voisinage. Il avait su, en quelques mots d’une parfaite courtoisie, me faire comprendre que, malgré les apparences, il n’était en aucune façon à la recherche d’un ami, une âme-sœur ou quoi que ce soit d’autre qui s’apparente de près ou de loin à un être humain. Même s’il arpentait d’un pas traînant les couloirs sombres et glacés d’une existence avec laquelle il ne se sentait aucune affinité particulière, il entendait bien continuer à profiter jalousement d’une solitude chèrement acquise à la force du poignet, poignet à l’extrémité duquel se trouvait une main qui, aux dires des mauvaises langues, était le seul et unique partenaire avec lequel il ait jamais entretenu une quelconque relation à caractère sexuel.

J’ai essayé d’allumer mais la lumière ne marchait pas.

La lumière ne marche jamais quand une scène dramatique se profile à l’horizon.

C’est en principe à ce moment-là qu’une musique sinistre fait son entrée en scène pour mettre le spectateur sous pression. Cette musique allie le plus souvent sons d’une gravité extrême, plus bas que bas, comme sortis d’outre-tombe, et couinements suraigus qui mettent les nerfs en pelote. Pour parfaire le tout, des effets percussifs soudains et largement amplifiés s’abattent sur vous à l’improviste.

Comme on a coutume de dire dans ces cas-là : je n’y voyais pas plus loin que le bout de mon nez, assez court du reste.

Sans une aide extérieure, j’aurais été rigoureusement incapable de trouver le trou de la serrure de la porte d’entrée de mon appartement, ou alors il m’aurait fallu tâtonner des heures durant pour y parvenir. C’est sans doute ce qui se serait passé si j’avais vécu au dix-sept ou dix-huitième siècle, mais j’avais la chance de vivre au vingt-et-unième, à la glorieuse époque de la téléphonie mobile, et d’avoir dans ma poche ce qu’on appelle un smartphone, littéralement un téléphone intelligent, fin et racé, connu aussi sous les noms de mobile multifonction et terminal de poche, véritable couteau suisse des temps modernes, capable de satisfaire toutes vos exigences en un temps record. Prendre des photos de qualité professionnelle alors que vous êtes complètement nul et disposez au mieux de la sensibilité artistique d’une moule à marée basse ? C’est possible. Envoyer des messages chiffrés alors que vos connaissances en mathématiques se limitent à la table de multiplication par 2, et encore par temps clair quand Neptune est en conjoncture favorable avec Vénus ? C’est possible. Avoir des tas d’amis qu’on n’a jamais vu, qui affichent des photos de profil bidons, postent des gros plans de leur bite en érection et débitent des insanités par treize à la douzaine ? C’est possible, bien sur, et même très à la mode, grâce à nos réseaux sociaux dernier cri, offrant à prix cassé toutes les avancées en matière de navigation en eaux troubles et hypertextualisation des relations quotidiennes. Grâce à nous, non seulement vous ne serez plus jamais seul, mais vous ferez partie d’une cybersociété mondiale toujours prête à répondre à vos attentes et vous niquer dans les grandes largeurs. On attend avec impatience le modèle «plancha» permettant de se faire griller des saucisses sans fumée dans les salles d’attente et les transports en commun.

L’ombre en question, celle qui glissait furtivement dans le couloir en provenance de l’appartement de Marc-Antoine Jacquinot, «l’ermite du sixième» comme on l’appelait, était celle d’une créature que je ne connaissais que trop bien : Korax, le chat de la mère Ouvrard, animal sournois et malfaisant qui ne perdait jamais une occasion de nuire à son prochain. Il avait fait de l’immeuble son territoire et entendait bien y faire régner une seule et unique loi : la sienne.

Il est passé devant moi, m’a jeté un regard mauvais, puis s’est dirigé nonchalamment, trop nonchalamment, en faisant exprès de prendre tout son temps pour bien me faire chier, me narguer, vers l’autre bout du couloir, celui où se trouvait l’appartement de la mère Ouvrard, l’ignoble vieille bonne femme qui sentait le rance et empestait la méchanceté à plein nez, la Voisin, la Giulia Tofana, l’Hélène Jégado, la Marie Besnard, la Chisako Kakehi du sixième qui empoisonnait ses victimes au porto frelaté, une infâme piquette qu’elle achetait à vil prix à la supérette du coin. S’il existait un semblant de justice dans ce pays, il y a belle lurette qu’elle aurait dû être condamnée à la réclusion criminelle à perpétuité, à l’isolement pour éviter de contaminer le reste de la colonie. Soit maudite jusqu’à l’os, mère Ouvrard, et puissent les flammes de l’enfer te calciner les miches à petit feu jusqu’à la fin des temps ! Oui, je sais ce que tu vas me dire : je suis seule au monde, jamais personne ne vient me voir, je n’ai ni famille, descendance ni ami, de moula non plus et y a longtemps que j’ai arrêté de bédave, ma vie est si laide et inhumaine que les roses fanent à ma vue et les rivières se tarissent. Sur la vie de ma mère, même les serpents, les rats et les cafards me fuient ! Je suis la plus immonde des larves qui aient jamais été pondues par le cloaque putréfié d’une femme de mauvaise vie, un tel concentré d’abjection que la pourriture elle-même fait figure de nectar à côté de moi. Ouais, je sais tout ça, et pourtant j’estime que j’ai droit moi aussi à minimum sinon d’amour, je demande pas la lune, au moins de considération et de respect.

Bon, okay, j’exagère un peu. D’accord, Maria Ouvrard était une de ces femmes comme on n’aimerait pas en croiser une au fond d’un placard, un bois ou un cimetière, notamment la nuit, et on se demande encore à quoi pouvaient bien ressembler ses parents pour engendrer une horreur pareille. Quelle mouche les avait piqués ? Néanmoins, qu’on le veuille ou non, elle avait bien dû être jeune, elle aussi, à un moment ou à un autre. Ce n’est pas une garantie en soi, mais enfin tout de même, on est souvent surpris de voir à quel point une adorable fillette peut se transformer, en l’espace de quelques décennies, en chose plus ou moins monstrueuse, improbable croisement entre l’éléphant de mer et le poisson chauve-souris à lèvres rouges (sale gueule, celui-là), puis en vieille sorcière hideuse et maléfique. Ça vaut aussi pour les hommes, bien entendu, je ne voudrais pas qu’on me traite de phallocrate. C’est juste, si on veut, que la marge de manœuvre est plus étroite pour l’homme en raison du niveau de mocheté qu’il parvient à entretenir tout au long de son existence. Le moment critique, à savoir la puberté, impacte durablement son apparence. Il est alors tellement laid, sans queue ni tête (surtout sans tête, d’ailleurs, parce que la queue tourne à plein régime), que le temps ne peut que lui être bénéfique. Si, vous serez d’accord avec moi que pas mal de types sont moins laids vieux que jeunes, même s’ils restent de toute façon tellement immondes qu’on se demande comment une femme peut avoir envie d’y toucher. Le désespoir, sans doute, ou la déformation professionnelle. Bref, quels que soient les griefs qu’on pouvait avoir à son égard, la mère Ouvrard restait une créature de Dieu. Ou du diable, peu importe, elle restait une créature tout court, et, à ce titre honorifique, ne méritait pas qu’on l’écrase comme une vulgaire punaise malodorante ou la cloue comme une vieille chouette (moche) à la porte de sa chambre. D’autant qu’ils n’en avaient plus pour bien longtemps, elle son suppôt à quatre pattes, à empuantir l’atmosphère du sixième étage. Un jour on la retrouverait morte dans son appartement, de même que Korax qui serait mort empoisonné en essayant de la becqueter. Fin de l’histoire, au diable les varices !

Par chance, à cette heure avancée de la nuit, elle dormait à poings fermés, car rien n’aurait été plus désagréable, pour ne pas dire traumatisant, que de voir surgir sa vieille tête parcheminée dans l’embrasure de sa porte grinçante.

J’aurais dû aller me coucher, mais au lieu de ça j’ai trouvé bizarre de voir Korax sortir de là et me suis dirigé toute affaire cessante vers le fond du couloir, scrutant les alentours à la lumière de mon téléphone.

Une fois devant la porte de Jacquinot, je me suis rendu compte qu’elle était entrebâillée.

Aussitôt, une giclée de sueur froide m’a parcouru la nuque et glissé le long du dos jusqu’à la raie du cul.

Sans être tout à fait désagréable, la sensation trahissait quand même une certaine anxiété de ma part, savamment entretenue par les ténèbres environnantes et le silence pesant qui planait sur les lieux (oui, on peut être pesant et planer, la chose n’est point contraire aux règles de la physique).

Il n’était bien évidemment pas normal que la porte de Jacquinot soit entrebâillée à trois heures du matin, alors que, de jour comme de nuit, elle ne l’était jamais, sauf bien entendu quand il entrait ou sortait de chez lui, moments particuliers où il entrebâillait alors sa porte juste assez pour pouvoir se glisser dans l’ouverture. De cette façon, si un observateur se trouvait, par hasard ou non, dans les parages, il lui était impossible de distinguer ce qui se trouvait à l’intérieur. À force de s’adonner à ce petit exercice, Jacquinot était devenu presque aussi plat que la sacoche en cuir vintage qu’il trimballait en permanence au bout de son bras. À propos de ses bras (petite anecdote en passant pour détendre l’atmosphère, assez irrespirable il faut bien le dire), j’avais remarqué qu’il les avait nettement plus longs que la moyenne, ceux d’un homme normal lui arrivant peu ou prou à mi-cuisse, à l’endroit où se trouvent les poches de son pantalon s’il en porte un, alors que les siens lui arrivaient pratiquement aux genoux, d’un côté comme de l’autre. Tant et si bien que cette fameuse sacoche en cuir qu’il trimballait en permanence ne se trouvait jamais très loin de racler le sol, raison pour laquelle il avait choisi, ou s’était fait fabriquer sur mesure (ou avait hérité de son père qui avait le même genre de bras que lui), un modèle d’une confection assez inhabituelle, nettement plus long que haut.

J’ai poussé la porte et demandé d’une voix mal assurée, conscient que le fait qu’une porte soit entrouverte ne donnait pas pour autant le droit à quelqu’un de s’introduire chez quelqu’un d’autre sans y être invité : Il y a quelqu’un ?

Apparemment non.

J’ai néanmoins insisté : Monsieur Jacquinot, vous êtes là ?

Finalement, après avoir posé plusieurs fois la même question et avoir obtenu à chaque fois la même absence de réponse, je me suis décidé à entrer pour de bon.

Mon premier travail a été de mettre la main sur l’interrupteur le plus proche, tâche plutôt aisée qu’il ne m’a pas fallu plus de quelques secondes pour mener à bien.

Jacquinot aurait pu se trouver dans l’entrée, étalé de tout son long, face contre terre, baignant dans une mare de sang s’échappant de son crâne fracturé de part en part, ou encore des multiples plaies par arme blanche présentes la quasi totalité de son corps sans vie. Pire encore, seule sa tête aurait pu se trouver là, les yeux hagards et la langue pendante, les restes de son corps affreusement mutilé gisant un peu partout dans l’appartement, un bras par ci, une jambe par là, la bite dans le congélo et les couilles dans le vide-ordures, ces deux derniers points constituant la signature tristement célèbre du tueur en série connu sous le nom de «tueur du sixième», monstre insaisissable dont la principale caractéristique est de commettre ses forfaits uniquement au sixième étage des immeubles dans lesquels il s’introduit. Une fois, un jour où il devait être particulièrement en forme et remonté à bloc contre la société (espérons que les experts nous en apprendront un peu plus sur lui quand on aura réussi à le coincer), il ne s’est pas contenté d’un seul appartement mais a décimé la totalité de l’étage, enfants et animaux compris (le visage des enfants peint en jaune citron, comme s’il baignait dans un océan de lumière estivale, et les animaux coupés en deux dans le sens de la longueur).

Si un tel carnage s’était produit, vous pensez bien que Korax aurait été le premier sur les lieux pour se goinfrer des restes de la victime.

Mais tel n’était pas le cas.

D’interrupteur en interrupteur, j’ai poursuivi mon inspection et constaté que l’endroit était bien tel que je me l’étais imaginé : rempli de poussière et de toiles d’araignées, de la vaisselle sale s’entassant dans l’évier de la cuisine, des restes alimentaires trainant ici et là, abandonnés au détour d’un placard ou une commode et tartinés d’une généreuse couche de moisi, des bougies spectrales dégoulinantes de cire fondue, des meubles branlants dont on osait à peine effleurer portes et tiroirs de peur de les voir s’effondrer sous ses doigts. Même chose pour les chaises sur lesquelles seul un fou furieux suicidaire se serait risqué à poser ses fesses, les fauteuils défoncés et le canapé qui crachait ses tripes et servait vraisemblablement de repaire à toute une faune à laquelle il valait mieux éviter de songer si on ne voulait pas finir à l’asile le plus proche. Ce mobilier préhistorique, que Jacquinot avait dû hériter de quelques lointains ancêtres vivant à l’âge de pierre, il ne lui était manifestement jamais venu à l’idée de s’en débarrasser, comme tout être sain d’esprit l’aurait fait. J’imagine qu’il ne voulait pas froisser la susceptibilité des défunts, lesquels l’observaient attentivement depuis l’enfer où leurs méfaits les avaient conduits, et n’auraient pas manqué, en le voyant bazarder toute cette merde, d’exprimer leur désaccord en déplaçant des objets, faisant s’ouvrir et se refermer les portes de façon intempestive, entendre des bruits bizarres et souffler un vent glacial dans les couloirs de l’appartement. Ensuite, ils auraient pris possession de Jacquinot qui se serait mis à pisser partout et débiter des obscénités indignes de sa condition. Il aurait alors fallu faire venir un vieux prêtre cardiaque et un jeune prêtre en pleine crise de foi pour tenter de mettre un terme à l’infestation. Naturellement, les choses se seraient mal passées, le vieux prêtre serait mort en pleine action, le crucifix à la main, et le jeune aurait vendu son âme au diable avant de se jeter dans la cage d’escalier, dévaler les six étages sur le dos et se retrouver en bas avec la nuque brisée, tout ça pour une vague histoire de mobilier périmé.

Je savais Jacquinot grand amateur de littérature, mais ce que je découvrais au fur et à mesure de ma visite (je devrais dire mon exploration, car je me sentais de plus en plus dans la peau de Raymond Maufrais se frayant un chemin à coups de machette dans la jungle du Mato Grosso) dépassait de loin mes espérances les plus folles : tous les murs étaient recouverts de livres du sol au plafond, fascinant spectacle digne du plus majestueux des édifices religieux ! Car quand je dis tous les murs, c’est bien tous les murs, y compris, chose assez inhabituelle vous en conviendrez, ceux des toilettes et de la salle de bain. Je veux bien qu’on trouve quelques bouquins dans les toilettes, les gens ayant tendance à se faire chier quand ils chient et profiter de l’occasion pour parfaire leur culture, mais de là à ce que les murs soient couverts de livres du sol au plafond, il y a un pas que je n’avais encore jamais vu franchi. D’ailleurs, la plupart du temps, il ne s’agit pas, dans les toilettes des gens, de livres mais de revues, magazines, journaux plus distrayants, avec des images et moins de mots, moins à même de contrarier l’extrême concentration nécessaire à l’expulsion de matières qui ont parfois tendance à se faire désirer, faire preuve d’une certaine mauvaise volonté pour quitter le nid douillet dans lequel elles reposaient jusqu’ici.

Pour autant que l’on sache, l’être humain a toujours entretenu des relations particulières avec son caca. Il faut dire que là où l’animal peut se permettre de déféquer en toute tranquillité sans se soucier de rien, se torcher par exemple, parce que tout le monde, à commencer par lui, se fiche comme d’une guigne qu’il se promène le nez au vent avec de la merde au cul, suivi en permanence par un essaim de mouches irrésistiblement attirées par l’odeur infecte qu’il dégage, l’être humain, lui, se doit de sacrifier à des règles d’hygiène très strictes. Dès son plus jeune âge, alors qu’il n’est encore qu’un nourrisson totalement dépendant du bon vouloir de ses parents, ces derniers lui enseignent avec autorité qu’il est hors de question de chier dans son froc comme si de rien n’était. Comme d’une part il est incapable de se rendre aux toilettes par ses propres moyens, et d’autre part ne voit aucune raison valable de le faire, ils l’obligent à porter des couches et se relaient à son chevet pour les changer aussi souvent que nécessaire, chose qu’ils font de plus ou moins bonne grâce suivant leurs appétences en la matière et leurs disponibilités respectives. Avant on lavait les couches, ce qui représentait une charge de travail supplémentaire pas toujours très agréable, maintenant on les jette, le progrès se caractérisant par une accumulation de plus en plus massive de déchets dont on sait de moins en moins quoi faire, ce qui oblige au mieux à se creuser la tête pour trouver un moyen de les éliminer, au pire (qui est toujours certain) à les enterrer discrètement dans des endroits déserts en croisant les doigts pour qu’ils ne refassent jamais surface et ne contaminent pas le sous-sol avec leurs émanations malsaines. C’est ainsi que l’enfant, dès sa naissance, prend l’habitude de se chier dessus et baigner de longues heures dans la tiédeur rassurante et légèrement poisseuse de ses excréments, chose qui l’amène tout naturellement à les considérer avec une certaine bienveillance, d’autant que l’odeur forte et familière qui s’en dégage ne l’incommode nullement, contrairement à ses parents et son entourage qui témoignent d’une certaine gêne dès qu’ils la détectent. Bébé, au contraire, est ravi de se retrouver les quatre fers en l’air dès que sa couche est pleine, de gesticuler les pattes écartées en attendant que papa-maman le délestent de son panier bien garni pour le remplacer par un autre qu’il se fera une joie de remplir à la première occasion. Il exhibe fièrement ses organes génitaux et son petit trou de balle tout crotté, sachant que ses parents leur témoignent le plus vif intérêt. Ce caca qu’ils aiment tant, c’est avec la plus vive satisfaction qu’il le leur offre, en quantité illimitée et sous une forme diversement liquide et parfumée. Et puis, un beau jour, le ton change, on lui fait comprendre qu’il ne pourra pas continuer toute sa vie à chier dans des couches parce que ses parents en ont ras le cul (façon de parler) de trimballer des sacs de merde à longueur de journée. Quelle déception, pour ne pas dire trahison. Lui qui pensait pouvoir encore, à trente-cinq ou quarante ans, exhiber fièrement ses organes génitaux et son cul plein de merde sous le nez de ses parents émerveillés, tombe de haut. On a beau lui expliquer que ça reviendra peut-être quand il sera vieux, tout pourri et sur le point de casser sa pipe, mais que ce ne sera plus papa-maman qui feront le job parce qu’ils seront morts et enterrés depuis longtemps (et puis que de toute façon même s’ils étaient encore en vie ils n’auraient aucune envie de faire), il n’en tire aucune consolation. Non seulement on lui dit qu’il est grand temps d’arrêter de se chier dessus, mais on insiste lourdement sur le fait que ce qu’il tenait pour quelque chose de précieux, le meilleur de lui-même issu des fondements les plus intimes de sa personne, un authentique trésor qu’il offrait de bon cœur à ses parents pour les remercier de lui avoir donné la vie, n’est en réalité qu’une chose parmi les plus infectes et répugnantes qui soient, une pure abomination que ses parents ont pris courageusement sur eux pout tenter d’éliminer jusque dans ses moindres recoins, traquer dans ses plus infimes retranchements. Lui qui pensait que ses parents avaient soigneusement conservé les tonnes d’excréments qu’il leur offrait avec joie depuis des lustres, apprend avec stupéfaction qu’ils n’avaient de cesse de s’en débarrasser avec dégoût sitôt la récolte effectuée. Quelle désillusion, quel temps perdu, et surtout quelle duplicité de la part de gens qu’il pensait proches de lui et soucieux de préserver jalousement les témoignages d’affection qu’il leur dispensait en toute innocence. Maintenant qu’il est en âge de marcher, on lui explique que le caca c’est caca et qu’il n’est pas question de continuer à emmerder le monde avec ça. C’est tellement caca qu’il devra, pour éviter tout contact avec ses vêtements, porter des slips pendant le restant de ses jours, le slip étant une sorte de couche-culotte extrafine qui n’a pas vocation à recueillir le gros des troupes mais seulement limiter les dégâts en cas de fuite (à noter que le délit de fuite urinaire ou fécale ne donne pas lieu à des poursuites judiciaires). S’il semble avéré que les rois de France se soulageaient en public, devant une foule compacte qui se pressait pour assister au spectacle nauséabond et pétaradant de son seigneur et maître, de telles pratiques sont aujourd’hui révolues et sévèrement réprimées par la Loi, sauf peut-être dans le cadre de certaines réunions privées à la moralité douteuse. De nos jours, il existe, pour se soulager, des endroits prévus à cet effet, doté d’un système de verrouillage intérieur permettant à l’utilisateur de ne pas se faire surprendre en pleine action, la bite à la main ou la crotte au cul. De tels endroits portent des noms aussi divers et évocateurs (et souvent au pluriel) que toilettes, cabinets, latrines, sanitaires, commodités, petit coin et lieu d’aisance, plus quelques noms argotiques tels que goguenots, chiottes ou tartisses, ce dernier étant aujourd’hui sorti de l’usage courant. On parle aussi, à propos d’un tel endroit dépourvu de siège, de chiottes à la turque, ce qui semblerait indiquer que nos amis turcs ne sont pas très à cheval sur le confort et ne craignent pas de se chier (du latin cacare, «évacuer des excréments») sur les bottes. Chez nous, gens civilisés, on prend le plus grand soin de son postérieur et met toutes les chances de son côté pour joindre l’utile à l’agréable (d’où la fréquente présence de livres, revues, et, pourquoi pas si on dispose de l’espace nécessaire, un mini-bar et un téléviseur dernier cri, même s’il ne faut pas non plus que ce soit trop confortable sans quoi les gens risqueraient de plus jamais remettre le nez dehors, préférant rester bien au chaud dans leur merde plutôt que de retourner affronter les frimas de la vie quotidienne). Et surtout, quand on a fini, on ne se contente pas de se rhabiller et retourner vaquer à ses occupations comme si de rien n’était, un sourire de contentement sur le visage, sans faire le moindre cas des gens qui se pincent le nez et grimacent autour de soi. Non, avant de sortir, on s’essuie soigneusement le fion avec un papier lui aussi spécialement prévu à cet effet, le plus souvent de couleur rose ou blanche, orné ou non de motifs assez sommaires, plus ou moins rugueux, présent dans les toilettes sous la forme de rouleaux de feuilles pré-coupées aisément détachables. Reste, vous l’aurez compris, que s’enfermer à double tour dans un placard pour satisfaire un besoin aussi naturel que légitime pose un problème de fond (et même de fondement) sur lequel il ne serait peut-être inutile de s’interroger de temps à autre, histoire de ne pas entretenir outre mesure ce qui s’apparente clairement à un déni systémique aux conséquences potentiellement dévastatrices. Car enfin, je vous le demande à vous, gens de bonne volonté à l’esprit clair et lucide, quelle honte y a-t-il à chier, et pourquoi le fait de satisfaire un besoin aussi élémentaire se retrouve-t-il soudain frappé d’opprobre, au même titre d’ailleurs que l’orifice qui sert à mener l’action ? Chez beaucoup, cette interrogation ne trouve pas réponse, ce qui les conduit, hagards et dépités, à s’enfermer dans une sorte de névrose obsessionnelle (ou névrose de contrainte, en l’occurrence le fait d’être obligé de se cacher pour chier, et, par extension, s’attacher à faire disparaître toute mauvaise odeur de sa personne, voire disparaître tout court, tirer la chasse sur son existence vaine et inutile) et développer des troubles comportementaux plus ou moins graves. La chose est d’autant plus sensible que sa configuration elle-même prête à confusion. La nature, pour laquelle j’ai beaucoup de respect par ailleurs, n’a-t-elle pas fait preuve d’un peu trop de légèreté en plaçant l’anus dans le voisinage immédiat des organes génitaux ? Quelle mouche (à merde) l’a donc piquée pour qu’elle aille, elle d’ordinaire si subtile et mesurée, se fourvoyer dans un tel salmigondis anatomique ? Le cas de la vulve est encore plus prégnant, dans la mesure où bite et couilles sont des organes bénéficiant d’une certaine forme d’externalité. Mais la vulve, même si son architecture habile et fonctionnelle fait d’elle un organe tout à fait digne d’intérêt, n’en reste pas moins, qu’on le veuille ou non, un orifice, certes moins étroit que le fion, et aussi plus accessible car contrairement à lui elle ne dispose pas de la protection joufflue des fesses, mais un orifice tout de même, une excavation, un creux, un trou, une dépression, voire un gouffre, un précipice, une tombe, appelez ça comme vous voudrez. Au même titre que la bouche, me direz-vous, les trous de nez aussi bien que ceux des oreilles, car si on le veut on peut effectivement voir des orifices partout (et certains ne se privent pas d’en faire usage dans le feu de l’action, qu’il s’agisse des aisselles, d’un simple entrecuisse ou tout autre dispositif vaguement concave utilisé en guise de substitut), mais vous remarquerez comme moi que cette fois, la nature n’a pas commis l’impair de placer la bouche à côté du fion. Ce serait, à mon sens, tout aussi pratique pour manger, mais peut-être pas pour faire d’autres choses comme jouer de la flûte ou chanter des cantiques à tue-tête, l’expression «à tue-tête» n’ayant du même coup plus aucune raison d’être. Il faudrait alors la remplacer par «à tue-fion» ou «à tue-fesses». On notera aussi, dans ce contexte de sexualité débridée, que bouche et orifice anal entretiennent des relations dont l’étroitesse ne le cède en rien à l’incongruité, Mais, pour en revenir à nos moutons, vous ne m’enlèverez pas de l’idée que cette coupable proximité entre deux organes que tout oppose, je veux bien sûr parler de la vulve et le fion, est l’archétype même de ce qui s’appelle une erreur de la nature, d’autant plus manifeste que l’une et l’autre jouissent d’un statut très particulier au sein de la sphère anatomique. Tous deux, en effet, ont une importance capitale sur le plan vital, non dépourvue de symbolique qui plus est. L’un sert en quelque sorte à se vider, évacuer les scories de l’appareil digestif, tandis que l’autre sert à faire le plein de vie, régénérer l’espèce. Pourtant, qu’il s’agisse de l’étron ou du nouveau-né, tous deux sont éjectés de la même manière, à ceci près que les affres exprimées pendant l’accouchement sont sans commune mesure avec celles correspondant à l’expulsion délicate de quelque boudin récalcitrant, à un point tel que seule une farouche volonté de donner la vie peut permettre d’y souscrire en toute sérénité.

Mais comment, pour le dire vulgairement (chose dont j’ai horreur mais qui autorise parfois des raccourcis saisissants), par quelle aberration cosmique un trou qui produit de la merde au kilomètre a-t-il pu se retrouver à côté d’un trou qui produit de la vie, autrement dit la chose la plus belle et noble qui soit ? Reconnaissez que de là à penser que la vie c’est de la merde, il n’y a qu’un pas. Je veux bien que la nature ait le sens du contraste, mais tout de même ! On ne s’étonne plus, dès lors, que certains individus, déjà largement traumatisés par le fait d’être obligés de s’enfermer dans un placard pour chier, présentent des troubles du comportement qui confinent à la démence pure et simple, et Dieu sait pourtant que j’ai les idées larges en la matière. On trouve par exemple, sous le nom de code 302.9 de la section des paraphilies non spécifiées du DSM (l’encyclopédie des troubles mentaux soigneusement tenue à jour par l’Association Américaine de Psychiatrie, tant il est vrai que les pauvres ont fort à faire de ce côté-là, j’en veux pour preuve le cinglé à peau rouge et cheveux jaunes qu’ils ont trouvé le moyen de placer à la tête de l’État et qui n’arrête pas de taxer tout le monde, jouer au golf, envoyer des bombes sur l’Iran, se prendre la tête avec des milliardaires néonazis sud-africains et construire des centres de détention pour migrants dans des trous paumés infestés d’alligators), la coprophilie, pratique qui consiste à placer ou faire entrer des excréments dans sa bouche, les mastiquer plus ou moins longuement avant de les ingérer ou les recracher suivant la gravité des cas. Certains, parmi les scatophiles les plus modérés, se contentent de jouer avec, les manipuler avec délectation, pouvant aller jusqu’à s’en barbouiller le visage ou s’en enduire le corps. Le Marquis de Sade, dans ses Cent Vingt Journées de Sodome ou l’École du Libertinage, s’est longuement (avec délectation) penché sur la question. Cet ouvrage, rédigé sur trente-trois feuillets de papier vergé crème et bleu collés bout à bout jusqu’à constituer un rouleau de près de douze mètres de long, alors que Sade se trouvait en résidence à la Bastille pour faits de débauche extrême à la limite de la criminalité la plus débridée, est aujourd’hui considéré comme le catalogue de perversions sexuelles le plus abject et complet jamais réalisé par la main de l’Homme. Véritable compilation des pires abominations en la matière, ce rouleau de PQ satanique a longtemps été dissimulé aux yeux du monde, dans le but de protéger l’espèce humaine, essayer de sauver son âme dévoyée des flammes de l’enfer, empêcher coûte que coûte que de telles horreurs tombent entre des mains innocentes. Sans doute rendu fou par un sentiment de toute-puissance lié à son rang, à moins que la consanguinité n’ait quelque chose à voir là-dedans, Sade s’est rendu coupable des pires atrocités, se livrant sans retenue à des passions telles que même son génie littéraire ne parvenait plus à les transcender. Son style, proche de l’écriture automatique, déversant un flot continu d’immondices à la face du lecteur tétanisé (ébahi, et parfois même, il faut malheureusement bien le dire, excité et inspiré), trahissait une démence que les mots n’arrivaient plus à contenir. Et parmi ces immondices se trouvaient bien évidemment, de façon récurrente jusqu’à l’écœurement, outre le sperme, le sang et même le pus, les matières fécales pondues par les anus les moins ragoûtants de la planète. Selon Freud et moi-même, car je le rejoins volontiers sur ce point (avec peut-être, quand même, en ce qui me concerne, un léger surcroît de finesse dans la vision holistique des choses), l’apprentissage de la propreté anale n’est pas sans conséquence sur la vie de l’enfant. L’enfant en bas âge, dépourvu de dents, doit se contenter de boire du lait et avaler des aliments liquides. Et quand il chie, non seulement ça ne sent pas le muguet, mais il en fout partout dans sa couche parce que ses selles manquent de consistance. Autrement dit, quand il chie, et selon qu’il est une fille ou un garçon, il s’en fout plein la chatte, la bite et les couilles. Normalement, quand on grandit, on s’essuie le trou de balle et on passe à autre chose. Dans le cas présent, quand l’heureux élu, le père ou la mère en l’occurrence, la nourrice le cas échéant ou toute autre personne autorisée, ouvre le paquet, il ou elle est contraint de procéder à des travaux de nettoyage de grande ampleur, d’autant plus techniques que les surfaces concernées offrent de nombreux coins et recoins pas toujours faciles à atteindre. En d’autres termes, la merde s’accumule dans les plis et il faut frotter sec pour en venir à bout. On imagine facilement que ce genre de traitement occasionne des sentiments confus chez l’enfant, qui associe tout naturellement le fait de chier, abondamment de préférence (le nettoyage dure plus longtemps), au fait de se faire tripoter les organes génitaux, chose qui lui procure bien évidemment un certain plaisir. Pourquoi ? Eh bien mais tout simplement parce la nature, certainement l’entité la plus manipulatrice qui soit, a mis au point ce stratagème diabolique qui consiste à faire en sorte que la reproduction soit synonyme de plaisir extrême, de jouissance absolue. Quoi qu’il en soit, même si ces notions de propreté et reproduction sont encore bien abstraites pour lui, l’enfant en bas âge comprend très vite que plus il chie copieusement et plus il sera récompensé par papa-maman qui passeront du temps à lui récurer les parties génitales. Avant, c’était le plus souvent maman qui s’y collait, de sorte que la plupart des filles étaient lesbiennes et des garçons rêvaient de coucher avec leur mère. Maintenant, les papas s’y mettent aussi, de sorte que la plupart des filles sont hétéros et des garçons homos. Si les deux participent à relative égalité, ce qui est préconisé par la nouvelle charte de la vie en couple, on peut espérer avoir affaire à des enfants raisonnablement bisexuels (à voile et à vapeur comme on disait avant, quand on avait le sens de la métaphore maritime et ferroviaire), ce qui est encore le moyen le plus sûr de prendre son pied dans toutes les circonstances. cela semble d’une logique imparable, frappé violemment au coin du bon sens le plus inaltérable, et pourtant vous devez garder solidement en mémoire que je ne dispose d’aucune information permettant de le confirmer ou l’infirmer officiellement. Il ne s’agit, j’en ai bien peur, que de spéculations dont la gratuité ne saurait en aucun cas légitimer l’abus.

Cela dit, à l’heure de la désinformation ouvrant la voie à toutes les formes de paranoïa complotiste, je ne vois aucune raison de s’en priver.

Mais ce n’est pas de ça que je voulais vous parler initialement.

Non, ce sur quoi je voulais m’arrêter, c’est sur le fait que je me trimballais une terrible envie de chier depuis des lustres, et que celle-ci, qui était plus ou moins passée au second plan depuis une dizaine de minutes, s’était brutalement réveillée à la vue des toilettes de Jacquinot, endroit, comme je vous l’ai dit (ou pas, je ne sais plus), lui aussi recouvert de livres du sol au plafond.

Parmi ceux-ci, outre de nombreux ouvrages consacrés à la philosophie, la psychiatrie et la psychanalyse, parmi lesquels les œuvres complètes de Freud, Fliess (je recommande à tous la lecture des «relations entre le nez et les organes génitaux féminins présentés selon leur signification biologique», indispensable) et Breuer (ses remarquables travaux sur l’hystérie, notamment le cas de la militante féministe et polyglotte Bertha Pappenheim, alias Anna O.), se trouvait aussi tout ce que la littérature érotique avait produit de plus excitant depuis la plus haute Antiquité.

Comme je n’avais rien remarqué de suspect pendant mon expédition, je me suis dit que rien ne m’empêchait, puisque j’étais sur place, de couler un bronze dans les toilettes de Jacquinot. J’aurais aussi bien pu le faire chez moi, je vous l’accorde, mais l’envie était devenue tout à coup si impérieuse que je courais le risque de perdre une partie du chargement en cours de route. Ç’eut été dommage, vous en conviendrez. J’ajoute qu’il y avait là un tas d’ouvrages que je ne connaissais pas, des auteurs dont je n’avais même~-- en dépit de ma culture encyclopédique~-- pour ainsi dire jamais entendu parler, une foultitude de livres rares à n’en pas douter, et, cerise sur le gâteau, un certain nombre d’opuscules poussiéreux et autres grimoires antiques qui ressemblaient furieusement à des éditions originales d’une valeur plus ou moins inestimable. Qu’est-ce qu’ils foutaient dans les chiottes, mystère et boule de gomme, mais toujours est-il que mon niveau d’excitation intellectuelle avait atteint des sommets que seul ma dramatique envie de chier pouvait lui disputer.

C’est donc mu par une avidité certaine que je me suis introduit dans la place, ai dégrafé mon pantalon, fait tomber mon slip (le modèle Brando en jersey de coton bi-stretch à imprimé léopard de chez Dolce \& Gabbana, pas du meilleur goût je vous l’accorde, mais c’était Zarina qui me l’avait offert et il coûtait une couille, j’étais donc obligé de le porter de temps en temps pour amortir l’investissement), me suis assis sur la cuvette des chiottes et ai aussitôt balancé un flot de chiasse digne des chutes du Niagara.

Dès lors, il ne me restait plus qu’à tendre le bras pour attraper le volume de mon choix.

J’ai hésité un instant entre le Roman de Violette de la marquise Henriette de Mannoury d’Ectot (alias la Vicomtesse de Cœur Brûlant) et Le Diable au corps du chevalier Andréa de Nerciat (aka «docteur Cazzoné»), personnage obscur dont j’avais maintes fois croisé le nom mais n’avais jamais lu la moindre ligne. Le moment était venu de corriger cet oubli, ce qui ne m’empêcherait bien évidemment pas, si la séance s’éternisait, de jeter un œil sur le roman de cette chère Violette, créature à laquelle il n’était pas difficile d’imaginer qu’il arrivait tout un tas d’aventures parmi les plus affriolantes.

Un quart d’heure plus tard, j’étais toujours dans le Diable au corps (à ne pas confondre avec l’opéra-bouffe d’Ernest Blum et Raoul Toché ou le chef-d’œuvre de Radiguet), dans les toilettes de Jacquinot. J’avais fini de décharger mon abominable cargaison depuis un certain temps, je me sentais nettement mieux et ne voyais aucune raison valable d’interrompre ma lecture. J’avais, je dois le reconnaître, la fâcheuse habitude, que j’imagine partagée par beaucoup de gens dans mon genre, même si je reste un exemplaire assez unique de ce que l’humanité peut produire de meilleur, ceci dit en toute modestie bien sûr, j’avais, disais-je, la fâcheuse habitude de m’éterniser sur le trône au point d’avoir tellement de fourmis dans les jambes qu’il me devenait pratiquement impossible de marcher. Les bestioles me dévoraient les pattes et je risquais de me foutre par terre à chaque fois que je posais le pied au sol. Toute personne m’ayant vu sortir des toilettes dans cet état n’aurait pas manqué d’éclater de rire d’une part, tant le spectacle est ridicule, et d’autre part, au vu du contexte, s’interroger sérieusement sur l’état de ma santé mentale. Dieu merci, tout ne tardait pas à rentrer dans l’ordre.

«Dans cette attitude, il a le superbe cul sur les yeux et sa bouche est croisée de cette entaille magique où la Nature a fixé le siège des voluptés. En même-temps, l’intéressant et fier boute-joie se dresse contre les yeux de Nicole, déjà provoquée par une langue qui n’est pas la gauche et peu complaisante langue de l’automate Hilarion.»

Voilà très exactement ce que j’étais en train de lire, rédigé dans le style inimitable du XVIIIe, quand un bruit suspect a attiré mon attention, faisant immédiatement retomber la tension palpable qui était en train de s’installer discrètement au niveau de la partie la plus bas-ventrale de mon humble personne.

Je suis comme tout le monde, je n’aime pas tellement que des bruits suspects attirent mon attention, surtout quand je suis en train de chier dans les chiottes de quelqu’un d’autre, que ce quelqu’un d’autre, que je connais à peine (bonjour, au revoir, joyeux Noël et bonne année), ne m’a pas invité à lui rendre visite, lire ses livres rares, et encore moins utiliser ses chiottes sans autorisation préalable. J’aurais aussi bien pu passer un coup de fil, en m’excusant de l’heure tardive, pour l’avertir que j’étais sur le point de débarquer pour utiliser ses toilettes, sous le prétexte fallacieux que les miennes étaient bouchées, remplies de merde jusqu’aux ouïes avec une chasse d’eau hors service, ou encore occupées par quelqu’un d’autre, ma compagne, par exemple, créature de rêve qui préparait idéalement le bollito misto à la viande de poulain et la couenne de porc, ou encore encore, pourquoi pas, une entité démoniaque et grimaçante avec des yeux injectés de sang et la bouche remplie d’asticots, un extraterrestre (genre pas commode, Alien ou Predator, et je vous prie de croire que quand vous tombez nez-à-nez avec un de ces deux-là en train de chier dans vos toilettes à trois heures du matin, vous vous excusez du dérangement, refermez doucement la porte et vous éloignez aussi vite et aussi loin que vos jambes flageolantes peuvent vous porter), ou tout simplement un inconnu qui les avait prises en otage et exigeait une rançon exorbitante pour les libérer.

Ô crotte ! suspends ta chute, comme aurait dit Lamartine dans Le lac, un lac de merde en l’occurrence, un océan de matière fécale dont les effluves fleuraient bon le poisson pourri et la crevette en rupture de chaîne du froid, afin que je puisse tendre l’oreille et mettre un nom sur le bruit que j’entends !

Le bruit se rapprochait, et il ne fallait pas avoir fait cinq millions d’années d’études ou être titulaire d’un diplôme de MOF (Meilleure Oreille de France) pour se rendre compte que ce bruit était celui que des pieds humains chaussés de chaussures à semelles de cuir produisent habituellement lorsqu’ils se déplacent sur des lattes de plancher.

Dans quelques instants, la personne à qui appartenaient ces pieds, personne dont il y avait tout lieu de penser qu’elle n’était autre que Marc-Antoine Jacquinot, prof de philo notoirement dépressif qui foutait la trouille à ses élèves~-- et pas seulement eux, il me la foutait à moi aussi et à la plupart des gens qui avaient le malheur de croiser sa route au détour d’un cimetière, un terrain vague, une allée de bibliothèque ou une voie sans issue~-- et ne se déplaçait jamais sans une sacoche en cuir d’apparence antédiluvienne, cette personne allait débarquer et me découvrir en fâcheuse posture dans ses chiottes, à trois heures du matin, la crotte au cul et le Diable au corps entre les mains.

J’imaginais sans peine son visage (au nez aquilin, au regard terne chaussé d’horribles lunettes vintage à monture dorée qui lui donnaient des airs de médecin-chef de camp de la mort adepte de la vivisection et la stérilisation de masse, au menton tapissé de barbe semblable à de la moquette élimée et au crâne aussi dégarni qu’un buffet de mariage après la fête) déformé par la stupeur et la consternation de me découvrir moi, son voisin de palier avec lequel il n’entretenait que des relations de la plus stricte convenance, assis sur la cuvette de ses chiottes à trois heures du matin. Le choc pouvait, s’il était cardiaque, l’emporter directement dans la tombe. L’avantage, vu qu’il n’avait apparemment ni ami ni relation d’aucune sorte, c’était que personne ne le regretterait. Il n’y aurait, comme pour Mozart, pour suivre son cercueil qu’une poignée de chiens galeux et de clodos endimanchés, plus, éventuellement, quelques adolescents boutonneux, chevelus et binoclards, fripés à la va-comme-je-te-pousse avec des vestes trop grandes et des pantalons trop courts, qui voyaient en lui un maître absolu du nihilisme relativiste au sens le plus démocritien, kafkaïen, polysémique et destouchien du terme.

Dans la situation pour le moins délicate qui était la mienne, et compte tenu du temps extrêmement court dont je disposais pour mettre sur pied une stratégie plus créative, j’en suis arrivé à la conclusion que le mieux était encore de m’enfermer dans les toilettes, ce que j’ai fait sans hésiter, courageusement.

Les chaussures à semelle de cuir sont arrivées et se sont arrêtées à ma hauteur.

De mon poste d’observation, je pouvais entendre la respiration de la personne à qui appartenaient les pieds qui se trouvaient dans ces chaussures, personne qui se tenait juste derrière la porte, ou devant si on se plaçait de son point de vue, et qui, à en juger par le temps qu’elle prenait pour faire entendre le son de sa voix, exprimer ses interrogations et faire valoir ses droits, devait se poser pas mal de questions sur la conduite à adopter.

J’ai vu la poignée bouger et j’ai prié pour que la porte, construite dans un matériau de piètre qualité et à peu près aussi épaisse qu’une feuille de papier à cigarette, résiste à la pression.

La poignée a bougé à nouveau, avec une vigueur accrue trahissant un certain énervement de la part de mon visiteur. Mais peut-être faisait-il tout simplement lui aussi l’objet d’une envie pressante, et pouvait-on alors comprendre, en admettant qu’il s’agisse bien du propriétaire des lieux, que le fait de trouver la porte de ses chiottes fermées de l’intérieur lui occasionne une certaine forme de contrariété, laquelle pouvait aussi, si la situation ne se réglait pas au plus vite, se transformer en fureur noire elle-même susceptible d’entraîner un déchaînement de violence sans précédent. Même si, de la part d’une personne aussi flasque, absente et dépourvue de relief que Jacquinot, toute idée de déchaînement de violence sans précédent semblait relever davantage du thriller horrifique au scénario inexistant (voir exemple ci-dessous) que d’une vision claire et détaillée de la triste réalité des choses.

Exemple ci-dessous, qui s’apparente à une note en bas de page mais ne se trouve pas en bas de page (une petite coquetterie que je m’autorise de temps à autre, quand la Lune est en conjonction avec Saturne, période qui favorise la concentration et la confiance en soi, de préférence à l’heure où la Sittelle de Krüper rejoint à petits coups d’ailes pressés son nid dans les noires forêts de pins de l’Azerbaïdjan), et ce pour la bonne et simple raison que je sais pertinemment que personne ou presque ne lit les notes en bas de page, ce qui signifie que si on veut qu’elles soient lues il ne faut pas les mettre en bas de page mais en plein milieu, ne pas chercher à les ostraciser, les reléguer au second plan, mais au contraire les inclure de plein pied dans un récit auquel non seulement elles appartiennent, mais dont elles viennent également préciser onctueusement le sens et renforcer en douceur la pertinence : le professeur Roman Bozhko, un virologue d’origine ukrainienne d’une petite cinquantaine d’années (il a une jambe plus courte que l’autre, des hémorroïdes, et surtout un syndrome de Moersch et Woltman qui l’oblige à se bourrer de benzodiazépines), qui bosse pour un labo P4 top secret du genre Unité 731 (quand les Japs violaient des femmes à la chaîne et se livraient à des expériences horribles sur des êtres humains pour développer des armes bactériologiques) ou Zagorsk-6 (même chose pour les Russes, mais avec des babouins~-- attention, je n’ai pas dit que les Russes violaient des babouins, même s’il faut s’attendre à tout avec eux), découvre avec effroi que sa femme le trompe avec un prof de philo dépressif et totalement dépourvu de charisme. Bozhko pète un câble et met au point, à l’insu de ses employeurs dont on ne sait pas très bien qui ils sont ni d’où ils viennent, un virus qui s’attaque exclusivement aux profs de philo dépressifs et les transforme en tueurs psychotiques soumis à des accès de violence incontrôlable, accès qui les poussent, notamment, à s’en prendre à des femmes enceintes pour leur ouvrir le ventre et dévorer ce qui se trouve à l’intérieur.

Autant dire que les chances de voir Jacquinot se transformer en tueur psychotique et cannibale étaient aussi minces que celles de le voir se transformer en loup-garou, vampire, mouche géante, créature du marais ou entité maléfique venue d’ailleurs, même si, dans le noir et vu de dos, il pouvait aisément flanquer la trouille à n’importe qui. D’autant, je le rappelle, qu’il avait en permanence avec lui cette sacoche bizarre dont personne ne connaissait le contenu et qui ressemblait étrangement à la mallette de chirurgien de sir William Gull, proche de la reine Victoria un temps suspecté d’être Jack l’Éventreur (voir From Hell des frères Hughes, inspiré des écrits de Stephen Knight, obscur complot mêlant franc-maçonnerie et famille royale d’Angleterre sur fond de misère sociale et prostitution).

La poignée a cessé de bouger et une petite voix s’est fait entendre derrière la porte : Il y a quelqu’un ?

Ce à quoi j’ai répondu au débotté : C’est occupé !

\textsc{La voix} : Ah, pardon !

\textsc{Moi} : Je vous en prie.

Quelques instants de silence s’en sont suivis, aussi lourds et chargés d’électricité qu’un ciel d’orage, à l’issue desquels la voix s’est fait entendre de nouveau : Excusez-moi de vous déranger, mais… je peux savoir qui vous êtes ?

\textsc{Moi} : Et vous-même ?

\textsc{La voix} : Marc-Antoine Jacquinot, j’habite ici.

\textsc{Moi} : Ah, très bien. Je suis Djeferson Beauvais, votre voisin de palier. On s’est déjà croisé dans l’ascenseur.

\textsc{La voix} : Le commissaire de police ?

\textsc{Moi} : C’est ça.

\textsc{La voix} : Enchanté. Mais dites-moi, je peux savoir ce que vous faites dans mes toilettes à cette heure-ci ?

\textsc{Moi} : Une envie pressante, je suis vraiment désolé.

\textsc{La voix} : Vous n’avez pas de toilettes chez vous ?

\textsc{Moi} : Si, bien sûr, mais elles sont malencontreusement bouchées de chez bouchées. Vous savez ce que c’est, à force de chier dedans, encore et encore, la merde s’accumule et les ennuis commencent.

\textsc{La voix} : Non, je ne sais pas. Je mange très peu et vais très rarement à la selle, pour tout vous dire.

\textsc{Moi} : Je vois. En ce qui me concerne, j’adore chier et ne m’en prive guère, un véritable tube digestif sur pattes. Il y a des gens qui se mordent la queue, c’est bien connu, à défaut d’être assez souple pour se la sucer. Mais moi, si je pouvais avoir la bouche directement reliée au trou de balle, comme dans The Human Centipede, je serais le plus heureux des hommes. Je ne vous choque pas, j’espère ?

\textsc{La voix} : Du tout.

\textsc{Moi} : Vous avez vu le film ?

\textsc{La voix} : Du tout non plus, mais le concept est intéressant philosophiquement parlant.

\textsc{Moi} : Vous avez une très jolie collection de livres rares et anciens, en tout cas.

La voix, traversée de bizarres inflexions aquatiques comme s’il y avait une fuite d’eau à l’intérieur : Oui, j’en suis assez fier.

\textsc{Moi} : J’ajoute que vos toilettes sont très spacieuses et agréables pour s’adonner à la lecture.

\textsc{La voix} : Oui, il m’arrive souvent d’y aller uniquement pour ça. Je peux vous poser une question un peu indiscrète ?

\textsc{Moi} : je vous en prie, faites comme chez vous.

\textsc{La voix} : Justement. Je peux savoir comment vous avez fait pour entrer chez moi ?

\textsc{Moi} : Question légitime à laquelle je vais me faire un devoir de répondre dans les meilleurs délais.

\textsc{La voix} : C’est très aimable à vous. J’ose espérer que rien de grave ne vous amène ici.

\textsc{Moi} : Rassurez-vous. Si je me suis permis d’entrer, c’est uniquement parce que la porte était ouverte. Tenaillé par une violente envie de chier, je ne me voyais pas aller sonner à la porte de cette chère madame Ouvrard, Maria de son prénom, pour me soulager. D’abord il est tard, ou tôt, ensuite son chat me déteste et se ferait une joie de me crever les yeux, et enfin la seule idée de visiter les toilettes de cette charmante vieille dame me soulève le cœur aussi sûrement qu’une bourriche d’huîtres avariées. Je crois que je préférerais encore me retenir jusqu’à la fin des temps, quitte à exploser et engloutir la Terre sous un déluge de merde !

\textsc{La voix} : Comme je vous comprends. Vous dites que ma porte était ouverte, c’est bien ça ?

\textsc{Moi} : En effet. La chose a attiré mon attention, du reste. J’ai beau être un être humain avec une bouche, un tube digestif et un trou de balle, je n’en reste pas moins, avant toute considération d’ordre scientifique ou rationnel, un flic jusqu’au plus profond de mon âme, un limier que son flair infaillible mène par le bout du nez sur les chemins les plus étroits et les pistes les plus escarpées de la vérité.

\textsc{La voix} : Ce que vous venez de dire est assez joli.

\textsc{Moi} : Je suis poète à mes heures. Non, ce que je voulais dire, c’est que j’ai vu votre porte ouverte, que j’ai trouvé ça bizarre et me suis dit qu’il vous était peut-être arrivé quelque chose. Une fois sur place, j’ai été saisi par une violente envie de chier. Vos toilettes me tendaient les bras, ou la cuvette, et j’ai pensé que vous ne verriez pas d’inconvénient à ce que je les utilise, vu que les miennes bouchées. J’ai eu tort ?

\textsc{La voix} : Non, bien sûr. Quel genre d’homme serais-je si j’interdisais l’accès à mes toilettes à un des mes semblables au cœur de la tourmente.

\textsc{Moi} : C’est bien ce que je me suis dis, et je suis rassuré de voir qu’il ne vous est rien arrivé. Vous étiez sorti ?

\textsc{La voix} : Oui, comme tous les premiers vendredis soir du mois. J’anime un café philo au Monocle, rue Marcel Duclos. Laurent Jaubert, le patron, est un garçon très sympathique et ouvert à la conversation. C’est aussi le sosie exact du grand acteur français Paul Meurisse, jusqu’au timbre de la voix, particulièrement savoureux. Vous avez vu la série des Monocle ?

Moi, cédant à l’envie de mentir sans raison, par pur vice, alors que j’avais vu et revu la série des Monocle un nombre incalculable de fois : Non.

\textsc{La voix} : C’est bien dommage.

\textsc{Moi} : Oui.

\textsc{La voix} : Donc vous ne connaissez pas Paul Meurisse.

\textsc{Moi} : Si, je l’ai vu dans L’Armée des ombres, de Melville.

\textsc{La voix} : Ah, très bien.

\textsc{Moi} : Et aussi dans Le Cri du cormoran le soir au-dessus des jonques, de Michel Audiard, avec Michel Serrault, Jean Carmet et Bernard Blier.

\textsc{La voix} : Excellent. Ce n’est pas son meilleur film, mais Meurisse joue le rôle d’un chef de gang assez savoureux, il faut bien le dire, inspiré de Jean-Pierre Melville, précisément, avec ses lunettes noires et son chapeau. C’est aussi la première apparition de Gérard Depardieu dans un long-métrage.

\textsc{Moi} : Si vous le dites. Ecoutez, tout cela est très intéressant, mais je n’aime pas trop bavarder à travers une porte. Laissez-moi cinq petites minutes, le temps de me refaire une beauté, et je suis à vous.

\textsc{La voix} : Oui, bien sûr.

Je me suis torché et rhabillé en vitesse. Bien inspiré par une longue expérience et une connaissance profonde de la nature fécale des choses, j’avais pris soin d’ouvrir la fenêtre en grand avant de procéder à la vidange complète des huit mètres d’intestins qui garnissaient mon abdomen, en plus du foie, de la rate et l’estomac. L’odeur, même si elle était de nature à terrasser des personnes fragiles ou des animaux de petite taille tels que rongeurs, insectes, symphyles et autres pauropodes, restait supportable pour un individu normalement constitué, sain d’esprit (condition indispensable pour affronter une telle épreuve, et je dois reconnaître que j’avais quelques doutes à ce sujet concernant mon interlocuteur) et ne souffrant d’aucune pathologie respiratoire chronique.

J’ai ouvert la porte et me suis retrouvé nez-à-nez avec Jacquinot, sa mallette pourrie de tueur en série londonien, sa petite bouche cruelle agitée de rictus incessants, comme s’il cherchait en permanence à détacher des trucs coincés entre les dents, son nez d’oiseau de proie, et, last but not least, son regard vitreux scintillant mollement derrière ses lunettes vintage à monture dorée. N’ayant pas eu l’opportunité de me laver les mains, j’ai fait fi des usages en m’abstenant de lui tendre l’une d’entre elles pour qu’il la serre chaleureusement, chose qu’il n’aurait de toute façon vraisemblablement pas faite.

Ses yeux se sont rapidement promenés sur moi, comme des mouches bleues sur un étron fraîchement pondu, puis sont allés se fixer sur un point précis à l’intérieur des chiottes.

Il a alors pointé son doigt en direction du point en question, poussé un cri étouffé, et finalement réussi à vaguement articuler, d’une voix si déchirante qu’elle aurait pu tirer des larmes au plus retors, chauve, ondiniste et nécrophile des huissiers de justice : NOM DE DIEU DE BORDEL DE MERDE !!!!!!!!!

Moi, tournant la tête vers l’endroit en question, couvert de livres comme le reste de la pièce, sans remarquer quoi que ce soit de particulier (je précise au passage que je ne l’aurais jamais cru capable d’une telle vulgarité) : Quoi ?

\textsc{Lui} : VOUS NE VOYEZ PAS ?

\textsc{Moi} : Non, quoi donc ?

\textsc{Lui} : IL Y A UN TROU !!!!!!!

\textsc{Moi} : Ah bon, où ça ?

Lui, se dirigeant vivement vers l’emplacement désigné, moi à ses trousses : Mais là, voyons, entre La Métaphysique du strip-tease de Denys Chevalier et le Satiricon de Pétrone !!!!!!

\textsc{Moi} : Eh bien ?

\textsc{Lui} : Il y a un trou, vous ne voyez pas !

\textsc{Moi} : Vous voulez parler de ce léger espace entre les deux ?

\textsc{Lui} : Pas si léger que ça ! C’est l’endroit où se trouvait Glicère, ou la Philosophie de l’Amour de Nicolas Cammaille-de-Saint-Aubin, une édition originale de 1796, tirée à une centaine d’exemplaires dont la plupart ont aujourd’hui disparu.

\textsc{Moi} : Ça vaut cher ?

\textsc{Lui} : Dans les deux mille euros.

\textsc{Moi} : Ah oui, en effet, c’est pas donné. Vous êtes bien certain qu’il était là ?

\textsc{Lui} : Vous me prenez pour qui, un amateur ? Bien évidemment, que j’en suis certain ! Je connais par cœur l’emplacement de chacun de mes livres, et il y en a plus d’un millier dans cette pièce. ET LÀ, TENEZ !!!!!!!!!!!!!!

\textsc{Moi} : Quoi encore ? Ne me dites pas que….

\textsc{Lui} : Si, regardez, là, entre les Dialogues des courtisanes de Lucien de Samosate et Margot la ravaudeuse de Fougeret de Monbron, vous ne remarquez rien ?

\textsc{Moi} : Il y a un léger espace ?

\textsc{Lui} : Oui, comme vous dites ! C’est là que se trouvait une très rare édition originale de la Lettre à la Présidente de Théophile Gautier, une tenancière de bordel connue sous le nom d’Apollonie Sabatier qui a beaucoup inspiré Baudelaire et dont on dit qu’elle aurait servi de modèle à L’Origine du monde, le fameux tableau de Courbet. Il n’en existe qu’une cinquantaine d’exemplaires dans le monde, édités sous le manteau dans les années 1850, dont la valeur se situe aujourd’hui entre cinq et six mille euros. C’est une catastrophe !

\textsc{Moi} : Je comprends mieux.

\textsc{Lui} : Quoi ?

\textsc{Moi} : Pourquoi la porte était ouverte. Il est évident qu’un ou plusieurs individus ont profité de votre absence pour faire main basse sur vos livres rares. J’espère que vous êtes bien assuré.

\textsc{Lui} : Vous savez aussi bien que moi comment fonctionnent les assurances. Vous payez plein pot pendant des lustres, et quand c’est à eux de le faire ils font tout pour minimiser les frais. Je vais toucher un prix de gros pour l’ensemble, un lot de consolation tout au plus. Car voyez-vous, monsieur…

\textsc{Moi} : Beauvais. Djeferson Beauvais, avec un D comme désir ou doryphore, et un seul F. Mon père était un fervent admirateur de Thomas…

\textsc{Lui} : Car voyez-vous, monsieur Beauvais, il y a la valeur marchande des choses, qu’on peut évaluer avec une certaine précision en fonction de données matérielles et statistiques, et celle qu’on leur accorde pour des raisons plus personnelles, existentielles, métaphysiques et autres.

\textsc{Moi} : Sans doute, oui. Ce qu’on appelle le préjudice moral.

\textsc{Lui} : En termes de justice, oui. Mais il n’y a pas que la justice, dans la vie. Il y a des choses qui ne sont pas de son ressort, même si elle entend se mêler de tout et régler tous les problèmes à coups de peines de prison et compensations financières. Je vous en foutrai, moi, des dommages et intérêts ! Rien ne remplacera jamais les trésors qui m’ont été volés !

\textsc{Moi} : Il faudra penser à faire une liste complète des objets dérobés.

\textsc{Lui} : Ne vous en faites pas pour ça, je connais tout par cœur.

\textsc{Moi} : Il y a des livres partout, et je pense que les voleurs ont fait le tour de l’appartement.

\textsc{Lui} : Sans aucun doute. Par exemple, je suis prêt à parier que mon édition originale des Passions de l’âme de Descartes, datée de 1649, soit un an avant la mort de l’auteur à Stockholm, estimée aujourd’hui entre quinze et vingt mille euros, a disparu.

\textsc{Moi} : C’est probable, oui.

\textsc{Lui} : Certain

\textsc{Moi} : En principe, quand on a des objets de valeur chez soi, on fait installer un système d’alarme.

\textsc{Lui} : Est-ce que j’ai une tête à faire installer des systèmes d’alarme ?

\textsc{Moi} : Non, pas vraiment. Mais les assurances risquent de se faire tirer l’oreille pour rembourser.

\textsc{Lui} : L’argent n’a aucune valeur pour moi. J’ai hérité la plupart de ces livres de mon père, mon grand-père et mon arrière-grand-père, qui étaient tous des collectionneurs avertis, c’est comme si je les avais perdus une seconde fois.

\textsc{Moi} : Il y a quand même chose qui me turlupine.

\textsc{Lui} : Quoi ?

\textsc{Moi} : Vous n’avez aucun ami, n’est-ce pas ?

\textsc{Lui} : Je ne vois pas à quoi ils me serviraient. Mes seuls amis sont mes livres. C’est à eux que je me confie, et c’est eux qui m’apportent le réconfort dont j’ai besoin.

\textsc{Moi} : Pas de famille non plus, je suppose ?

\textsc{Lui} : Quelques vagues oncles et tantes que je vois une fois tous les dix ans ou vingt ans, principalement aux enterrements, le leur ou celui des autres.

\textsc{Moi} : Vous allez aux enterrements ?

\textsc{Lui} : Ça m’est arrivé une fois ou deux. Contrairement à ce qu’on pourrait penser, je n’ai aucune appétence particulière pour la mort et la religion. Les gens vivent, meurent, apparaissent et disparaissent, pas de quoi en faire tout un plat.

\textsc{Moi} : Ni femme ni enfant, bien évidemment.

\textsc{Lui} : Surtout pas ! Ils vous pompent toute votre énergie et vous vous retrouvez au fond du trou avant d’avoir eu le temps de comprendre ce qui vous arrivait ! Non, la solitude est un mal nécessaire si on veut approfondir un peu les raisons de sa présence sur terre, comprendre les tenants et aboutissants ce cette chose absurde qu’on appelle la vie. Je dois reconnaître que ça fait des années que je bosse comme un chien et que je ne suis toujours pas plus avancé, ou à peine. Pour ce qui est des enfants, j’en ai bien assez comme ça dans mes salles de classe.

\textsc{Moi} : Ça vous dérange si je fume ?

\textsc{Lui} : Oui.

\textsc{Moi} : C’est sans importance. De toute façon, je n’avais aucune envie de fumer. Non, voyez-vous, ce qui me turlupine le plus, dans cette histoire, et c’est le flic qui parle et non plus le simple voisin, c’est comment, alors que vous avez une vie sociale proche du zéro absolu, ni femme ni amis qui pourraient fourrer leur nez dans vos affaires, des malfrats ont pu savoir que vous gardiez chez vous, sans aucune protection qui plus est, tous ces livres rares et autres éditions originales inestimables ?

\textsc{Lui} : On se le demande, en effet.

\textsc{Moi} : Car à voir les livres qui ont été volés, on voit que ces bandits connaissaient leur affaire. Ils sont venus ici en sachant pertinemment ce qu’ils allaient y trouver.

\textsc{Lui} : À n’en pas douter. J’irai porter plainte dès demain au commissariat du quartier.

\textsc{Moi} : Le plus tôt sera le mieux. Il se trouve que j’ai quelques relations au sein de l’OCBC, l’Office central de lutte contre le trafic des biens culturels, je tâcherai de faire en sorte que votre dossier se retrouve en haut de la pile. La chose peut sembler anodine de prime abord, mais les voleurs travaillent le plus souvent pour le compte d’antiquaires et marchands d’art véreux qui financent indirectement le grand banditisme et le terrorisme. À moins, bien sûr, qu’on ait affaire à un type qui travaille en solo, un amoureux des belles-lettres qui cherche à agrandir sa collection sans bourse délier. C’est moins grave, si on veut, mais ça n’en reste pas moins du vol.

\textsc{Lui} : Vous en connaissez ?

\textsc{Moi} : Quoi ?

\textsc{Lui} : Des gens de ce genre.

\textsc{Moi} : Moi non, mais mes collègues certainement. C’est leur métier de suivre toutes les pistes, traquer les criminels, notamment sur Internet, le Dark Web en particulier. C’est parfois des gens comme vous, qui n’ont l’air de rien, des citoyens exemplaires, au-dessus de tout soupçon, qui brûlent d’une passion dévorante pour l’art et n’ont pas d’autre moyen de la satisfaire que de s’approprier des choses qui ne leur appartiennent pas. Si vous les interrogez à ce sujet, ils vous répondront qu’ils n’ont rien fait de mal, dans la mesure où leur intérêt exclusif et désintéressé pour la chose les rend seuls dignes de sa possession. Vous avez vu Les Pleins Pouvoirs, de et avec Clint Eastwood ?

\textsc{Lui} : Non.

\textsc{Moi} : Le héros de ce film est précisément un type de ce genre, qui vole des tableaux de maîtres pour avoir le plaisir et le privilège d’être le seul à les admirer.

\textsc{Lui} : Vous m’en direz tant.

\textsc{Moi} : C’est un peu le cas avec vos livres, non ?

Lui, posant sur moi un regard bleu délavé chargé de mépris et de consternation : Je ne les ai pas volé, à ce que je sache.

\textsc{Moi} : Non, bien sûr, mais les aimez profondément.

\textsc{Lui} : J’y suis viscéralement attaché, c’est vrai, et leur valeur marchande n’a rien à voir là-dedans. Comme je vous l’ai dit, j’ai hérité de la plupart de ces ouvrages, en tout cas tout ceux que mon salaire de prof ne m’aurait jamais permis de m’offrir. Je me fiche que vous preniez pour un cinglé, un maniaque, un ermite qui passe son temps enfermé chez lui dans la pénombre, entouré de grimoires poussiéreux et de déchets alimentaires en voie de putréfaction. J’aimerais qu’on les retrouve, mais je survivrai à leur disparition. Je ne dis pas que ce sera facile, que je ne serai pas la proie d’une puissante et atrocement douloureuse sensation de manque, mais je survivrai. J’ai survécu à bien pire que ça, vous pouvez me croire.

\textsc{Moi} : Ah bon ?

\textsc{Lui} : Oui. Mais inutile d’insister, vous n’en saurez pas davantage.

\textsc{Moi} : Ne vous en faites pas, je n’ai aucunement l’intention d’insister. Non que ça ne m’intéresse pas, encore que, mais la journée a été longue, interminable, même, et plus vite je serai dans mon lit, mieux je me porterai. Pour ce qui est de vos bouquins, j’essayerai de voir ce que je peux faire. Pas grand-chose, je le crains, le rayon culturel n’étant pas le mieux pourvu de la police nationale, d’ailleurs assez dépourvue en général. Cela dit, si j’étais vous, je sécuriserais un minimum les lieux. Pour quelques dizaines d’euros, on trouve des caméras de surveillance connectées qui permettent de voir et entendre ce qui se passe chez soi à distance et de recevoir des messages d’alerte en cas d’intrusion. Vous avez un téléphone portable ?

\textsc{Lui} : Bien sûr, oui. Vous me prenez pour quoi ? Un homme préhistorique ?

\textsc{Moi} : Je ne sais pas, vu que n’avez pas la télévision.

\textsc{Lui} : Pourquoi faire ? Regarder des jeux et des feuilletons débiles, m’enquiller des kilomètres de spots publicitaires avilissants et m’abrutir devant des chaînes d’info en continu qui servent d’organes de propagande à des milliardaires d’extrême-droite ? Très peu pour moi, merci.

\textsc{Moi} : Je suis tout à fait de votre avis.

\textsc{Lui} : Etre vautré devant un écran et regarder défiler des images en bouffant du pop-porn ne correspond pas vraiment à l’idée que je me fais de l’existence.

\textsc{Moi} : Et vous avez parfaitement raison. Bon, c’est pas que je m’ennuie, mais il faut vraiment que j’aille me coucher, maintenant. Je suis éreinté.

Jacquinot, d’une voix, qui jusqu’ici procurait une sensation à peu près aussi agréable que de se faire râper vigoureusement la couenne avec du papier de verre, devenue soudain aussi suave et onctueuse qu’une épaisse couche de crème pâtissière : Vous voulez boire un verre ? Je crois qu’il me reste une bouteille de Bisset.

\textsc{Moi} : De quoi ?

\textsc{Lui} : De Bisset, un apéritif au quinquina, un arbuste originaire de la cordillère des Andes. C’est un peu comme Byrrh, Dubonnet ou le Cap Corse de Mattei, une mistelle de cépages locaux dans laquelle on fait macérer de l’écorce de quinquina et des plantes en proportions variables. Très populaire à la Belle Époque, aujourd’hui totalement passée de mode. La quinine est un antidouleur et un antipaludéen naturel, mais c’est aussi un excitant dont l’usage intempestif peut avoir des conséquences graves. Le président Félix Faure, par exemple, est mort à l’Élysée dans les bras de sa maîtresse, le demi-mondaine Marguerite Japy. La coquine était goulue et le président avait besoin de fortifiant pour survivre à ses assauts. Sauf que cette fois il en est mort, manifestement. Ça vous dit ?

Moi, me dirigeant ostensiblement vers la sortie : Merci, sans façon.

\textsc{Lui} : Il est un peu tard, c’est vrai.

Moi, alors qu’il se tenait devant moi et entravait ma progression : Un peu, oui. Pardon, excusez-moi.

\textsc{Lui} : Ce sera pour une autre fois, alors.

Je n’en demandais pas tant, la distance respectueuse qui prévalait jusqu’ici dans nos relations me convenant en réalité tout à fait, même s’il m’arrivait parfois, comme tout le monde et pour me donner bonne conscience, de dénoncer la solitude grandissante dont mes contemporains faisaient l’objet en dépit d’une proximité accrue jusqu’à l’excès, l’overdose sociale, la surcharge pondérale existentielle, de regretter la convivialité tribale des temps anciens, déplorer l’absence de communication entre voisins, absence de communication dont on a la plupart du temps toutes les raisons de se féliciter compte tenu de la relative absence d’intérêt, pour ne pas dire la nullité stratosphérique du voisinage en question, et absence de communication sournoisement encouragée par l’hyperconnectivité ambiante, le fantasme communautaire : Oui, nous aurons certainement l’occasion de nous recroiser dans l’ascenseur ou l’escalier. Pardon…

\textsc{Lui} : Que nenni, je me ferai une joie de vous inviter chez moi ! (Et d’ajouter tout sourire, apparemment satisfait de son trait d’humour :) Comme ça vous ne serez pas obligé de pénétrer par effraction.

Moi, courant presque pour échapper aux griffes de Jacquinot : La porte était ouverte. Jamais je ne me serais permis autrement.

Lui, posant sur moi son regard délavé de reptile bigleux, d’une voix dégoulinante de chantilly périmée : Et vous avez bien fait ! Sans cela, je n’aurais sans doute jamais su qu’un voisin aussi instruit et passionnant résidait à deux pas de chez moi.

Je me suis abstenu de lui dire que mon rythme de lecture n’excédait pas les deux ou trois livres par an, tout comme j’ai omis de préciser qu’il m’arrivait fréquemment, même pas à ma grande honte, de ne lire que le début et la fin des ouvrages en question, voire les acheter, les remiser dans un coin et ne les ouvrir que six ou sept ans plus tard, en tombant dessus par hasard après avoir oublié leur existence. Non, tous ces détails sans importance devaient rester à tout jamais enfouis sous une épaisse couche de vernis protecteur si je ne voulais pas que mon aura de détective esthète et cultivé (et parfois un tantinet pédant, à la façon d’un Sherlock Holmes ou un Poirot de la grande époque) ne tombe en poussière tel un vampire exposé aux premières lueurs de l’aube.

J’ai donc glissé un pied dans l’embrasure de la porte, que j’étais enfin parvenu à entrouvrir, et laissé négligemment tomber aux pieds de mon interlocuteur énamouré : Vous êtes trop aimable. Je possède en effet un niveau de culture générale qui n’est pas complètement négligeable, je vous l’accorde, mais je ne me considère en aucune façon comme quelqu’un de particulièrement instruit ni passionnant.

Lui, tout sourire, dévoilant une dentition approximative à tout point de vue, tant par la disposition, la forme et les manquements que la couleur, d’un jaune pisseux du plus mauvais effet, très universitaire, à des années-lumière de celle d’un Tom Croûte, un Bras de Bite ou même un George Clownesque refait à neuf (jeux de mots dont il est assez difficile de rendre toute la finesse dans la langue de Shakespeare, Fitzgerald, Poe, Salinger ou Harriet Beecher Stowe, et sans doute plus encore dans le dialecte du New Hampshire tel qu’on le parle encore~-- de moins en moins, hélas~-- dans le comté de Sullivan, je m’en excuse d’avance auprès du traducteur qui aura la lourde charge de rendre cet ouvrage accessible aux anglophones) : Et vous, cher voisin, vous êtes bien trop modeste ! Si vous avez besoin d’un livre, parler à quelqu’un ou quoi que ce soit d’autre, n’hésitez pas à venir me voir.

Moi, prenant le fuite : Je n’y manquerai pas. (Traduction : Tu peux toujours courir, face de pet ! Plutôt mille fois crever dans d’atroces souffrances que de remettre les pieds dans ton gourbi !)

\textsc{Lui} : Au revoir, monsieur Beauvais. À très bientôt, j’espère.

«L’espoir fait vivre, pauvre con !»

Voilà en substance ce que je me suis dit en prenant congé de Jacquinot, ce qui n’était pas une façon très polie de le remercier de son accueil, je vous le concède, mais n’en traduisait pas moins avec une troublante exactitude les sentiments étranges et pénétrants que sa présence faisait pousser tels des champignons vénéneux sur le terreau fertile de mon imagination débordante. Je dis «étranges», et pour tout dire tout à fait inexplicables, de l’ordre du mystère le plus insondable, aussi épais que la couche de fond de teint sur la tronche d’une rombière de Neuilly-sur-Seine, parce que le fait est que (formule pas très gracieuse littérairement parlant, mais ça le fait si on a un bon timing de lecture), excepté un certain état de délabrement plus ou moins généralisé qui pouvait, à la longue, susciter l’aversion de la plupart de ses semblables (notamment un physique ingrat dont il n’était, sinon par négligence et désintérêt total pour ce qui touchait aux apparences, aucunement responsable), le pauvre vieux n’avait pas fait grand-chose pour mériter autant d’acrimonie de la part de quelqu’un comme moi, à priori sympathique, ouvert sur la différence et parfaitement à même de faire preuve de tolérance et de compréhension dans les situations les plus désespérées.

Le trajet de retour, quoique bref, fut éprouvant, car je sentais peser sur moi, telle la hache du bourreau prête à s’abattre sur ma nuque, le regard tranchant de Jacquinot qui m’observait depuis le pas de sa porte, tout à la joie de s’être fait un nouvel ami, lequel nouvel ami était d’ailleurs le seul puisqu’il n’en avait à ma connaissance pas d’autre. Et cette lourde responsabilité, celle d’être le seul et unique ami de Jacquinot, quitte à passer pour une ordure de premier choix, un cloporte au cœur sec, une larve immonde baignant dans une mare de pus sanguinolent, je vous avouerai franchement que je n’étais pas prêt à l’assumer. Au fond de lui-même, le pauvre vieux se félicitait que ce regrettable incident (le vol de ses précieux bouquins) lui ait permis de faire plus ample connaissance avec son voisin de palier. En effet, même s’il était à priori difficile de ranger ce dernier dans la catégorie des intellectuels de haut vol, il n’en cachait pas moins, sous des dehors rugueux, un esprit sinon brillant, n’exagérons rien, au moins affûté et ouvert à des perspectives dépassant de très loin de cadre de ses attributions. Il ne doutait pas un instant que ce nouvel ami, tout acquis à sa cause, userait de toutes les attributions en question pour l’aider à retrouver ses trésors disparus. Il va sans dire que pour ma part, en dépit des belles paroles prononcées dans le feu de l’action, je n’avais pas la moindre intention de lever le petit doigt pour lui, pas plus que je n’avais l’intention d’aller boire un verre chez lui et encore moins de l’inviter à le faire chez moi. Au contraire, comme nous avions parfaitement réussi à le faire jusqu’à présent, j’avais la ferme intention que nous continuions à vivre comme des étrangers et ne nous adresser la parole qu’en de très rares occasions, par exemple quand nous nous retrouvions coincés ensemble dans l’ascenseur et n’avions par conséquent pas le moyen de faire autrement (enfin si, on aurait pu, mais c’est quand même plus difficile de ne pas parler à quelqu’un quand vous vous retrouvez coincé avec lui dans un espace restreint pour un temps indéterminé ; l’étroitesse des lieux et l’atmosphère irrespirable font que vous êtes rapidement amenés à échanger quelques mots, ne serait-ce que pour discuter de la meilleure stratégie pour échapper à ce cauchemar). Les quelques mots que nous échangions annuellement suffisaient à mon bonheur, et il m’avait maintes fois fait comprendre que cela suffisait au sien. Pourquoi changer une équipe qui gagne, risquer sur un malentendu, une occurrence hasardeuse et sans avenir, de briser une absence totale d’amitié qui avait fait ses preuves et nous apportait à l’un et l’autre la plus entière satisfaction ? C’est peu dire que je me maudissais la curiosité, ou plutôt la déformation professionnelle qui m’avait poussé à m’introduire dans l’antre de cette créature visqueuse à l’haleine de putois, et plus encore cette intempestive envie de chier qui m’avait poussé à utiliser ses toilettes. J’aurais mille fois mieux fait de m’abstenir, quitte à chier dans mon froc et regagner piteusement mon logis, sachant que je devrais fournir des explications à ce sujet. Je mettais cette boulette (et ce paquet de merde) sur le compte de l’alcool, dont j’avais quelque peu abusé au cours de la soirée, et jurais qu’on ne m’y reprendrait pas.

C’est donc en rasant les murs, tel un affreux mille-pattes se déplaçant furtivement dans l’obscurité, que j’ai longé le couloir jusqu’à chez moi, me suis retrouvé devant ma porte, ai introduit (non sans difficulté, encore sous le coup de l’alcool et l’émotion) ma clé dans la serrure, et fait effectuer à ladite clé un tour complet dans le sens inverse des aiguilles d’une montre. Cet acte, dont l’apparente simplicité n’en mobilise pas moins une certaine technicité (j’en veux pour preuve qu’il n’est pas si facile de crocheter une serrure multipoints sans savoir-faire ni équipement adéquat), a eu pour effet de déverrouiller la situation et me permettre de regagner enfin mes chères pénates. Celles-ci, vous le savez maintenant, m’étaient d’autant plus chères que m’attendait à l’intérieur cette authentique et intrépide aventurière des temps modernes qui avait su, à force de patience et de ténacité, se frayer un chemin à travers la jungle de ma vie sentimentale. On pourrait aussi parler de désert, bien sûr, mais je préfère parler de jungle, parce que dans le désert on se fait chier, un peu comme sur un radeau de fortune au milieu de l’océan, alors que dans la jungle on se fait chier aussi, c’est vrai, à essayer de sauver sa peau par tous les moyens, mais au moins on n’a pas le temps de s’ennuyer (vous allez me dire qu’en mer il y a toujours moyen de survivre à des tempêtes ou des attaques de requins, mais quand même c’est pas pareil, c’est plus plat). C’est ainsi, sortie tout droit de The Lost City of Z, que l’implacable héroïne, après avoir déjoué les pièges les plus sournois et manqué mille fois de se faire dévorer par les cannibales du coin, avait fini par découvrir, tapi dans un écrin de verdure soigneusement dissimulé au milieu d’un inextricable enchevêtrement de plantes carnivores et autres lianes toxiques, ce temple inviolé que toute femme digne de ce nom rêve secrètement de profaner au moins une fois dans sa vie, je veux bien sûr parler de mon cœur.

Et elle avait parfaitement réussi son coup, avec une efficacité redoutable, au point que le célibataire endurci que j’étais s’était ramolli aussi sûrement une vieille tranche de pain trempée dans du lait (ou tout autre sorte de liquide alcoolisé ou non, ça marche aussi).

Je suis rentré sans faire de bruit, allé avaler un grand verre d’eau à la cuisine, me laver les dents à la salle de bain, après quoi je me suis dirigé vers la chambre en essayant tant bien que mal de ne pas faire grincer le plancher, lequel n’était pas de la première jeunesse et prenait un malin plaisir à hurler dès qu’on posait le pied dessus.

Et c’est là, croyez-le ou non, qu’elle m’attendait au pied du lit, dans le plus simple appareil, hormis ce gode-ceinture long et effilé, de couleur rose pâle, dont il nous arrivait d’user dans nos ébats. Je m’étais promis de ne donner aucun détail sur ma vie privée, ses aspects les plus intimes notamment, et voilà que je me retrouve au pied du mur. J’aimerais par conséquent que tout ce que je vais dire ici, ou fortement sous-entendre, ne sorte jamais de ces pages. Ce sera notre petit secret, si vous le voulez bien, et n’allez surtout pas vous faire des idées : il ne s’agissait que d’un petit jeu entre nous, somme toute bien innocent, à mille lieues des turpitudes sacrilèges que vos esprits mal tournés vont s’empresser d’imaginer.

Toujours est-il que quand je l’ai vue dans cet accoutrement, les premiers mots qui sont sortis de ma bouche, dans un souffle, ont été les suivants : Oh non, chérie, pas ce soir !

Ce à quoi elle a répondu : C’est à cette heure-ci que tu rentres ?

J’avais à peine la force de parler : Cette journée a été la plus longue et éreintante de toute mon existence.

Elle, l’air sévère : Ça fait des heures que je t’attends !

\textsc{Moi} : Je t’avais dit que je rentrerais tard, si toutefois je revenais vivant de cette aventure.

Elle, intraitable : Tout de même, tu mérites une punition.

\textsc{Moi} : Tu n’es pas contente que je rentre vivant de cette aventure ?

\textsc{Elle} : Si, ça va me permettre de te donner la punition que tu mérites.

Moi, aussi déterminé que mon état de fatigue générale, proche du dépôt de bilan et la liquidation totale, me le permettait : Désolé, mais ce sera pour une autre fois.

\textsc{Elle} : Tu sors d’où, d’abord ?

\textsc{Moi} : De chez le voisin.

\textsc{Elle} : Jacquinot ?

\textsc{Moi} : Oui, Jacquinot.

\textsc{Elle} : Je peux savoir ce que tu foutais chez Jacquinot ?

La chère âme, d’ordinaire si tendre et délicate, pouvait se transformer en véritable harpie. Sur une échelle de zéro à dix, j’estimais mes chances de survie à deux ou trois.

Moi, tout en me désapant : C’est une longue histoire. Je te la raconterai demain, si tu veux bien. Enlève ce truc et viens me rejoindre au lit.

\textsc{Elle} : Tout s’est bien passé ?

\textsc{Moi} : Non, pas vraiment. Mais je t’en supplie, laisse-moi me coucher et dormir quelques heures. Je te raconterai tout ça demain.

\textsc{Elle} : Tu ne veux vraiment pas que t’en mette un petit coup pour t’aider à t’endormir ?

\textsc{Moi} : Merci, ça ira. En plus, je ne suis pas tout à fait à l’aise de ce côté-là.

\textsc{Elle} : Ah bon ? Comment ça ?

\textsc{Moi} : Je suis barbouillé, ballonné. Et j’ai la chiasse, si tu veux tout savoir, à tel point que j’ai dû aller me soulager toute affaire cessante chez Jacquinot.

\textsc{Elle} : Tu n’as pas digéré le bollito misto à la viande de poulain et couenne de porc ?

\textsc{Moi} : Faut croire que non, mon amour. C’est bien la première fois que ça m’arrive. Je pense que c’est dû au stress, à la fatigue.

\textsc{Elle} : Tu travailles trop, voilà tout. Et tu ne pouvais pas venir te soulager ici ?

\textsc{Moi} : Tu te doutes bien que si je suis allé me soulager chez Jacquinot, c’est que je n’avais pas le choix de faire autrement. Personne ne s’amuse à aller chier chez Jacquinot si une autre solution s’offre à lui. Il a une superbe collection de livres anciens, soit dit en passant.

\textsc{Elle} : C’est vrai ?

Moi, nu : Et comment, que c’est vrai ! Et on lui en a piqué une bonne partie ce soir, raison pour laquelle j’étais chez lui. On se couche ?

Elle, plus douce que la mousse qui pousse en douce dans la cambrousse : On se couche.

\textsc{Moi} : Enlève ce truc, tu veux. je n’ai vraiment pas la tête à ça.

\textsc{Elle} : Oui, mon chéri. J’avais pensé te faire une petite surprise, mais je vois bien que ce n’est pas le bon moment.

Moi, assis sur le bord du lit : Non, vraiment pas. Cela dit, t’es vraiment un amour. Approche.

Elle a enlevé le… enfin, ce que vous savez, et s’est approchée en ondulant telle une charmeuse de serpent (j’entendais, dans le fond de mon crâne transformé en caisse de résonance improvisée, façon calebasse, cucurbitacée, balafon, tambour d’eau et nébuleuse de l’Œuf pourri, un petit air de pipeau qui me vrillait les synapses et me faisait des nœuds dans la cervelle).

J’ai posé mes mains sur ses fesses, qu’elle avait aussi rondes et fermes que celles de Melanie Nunes Fronckowiak (élue Miss-Plus-Beau-Cul-du-Monde en 2008, après la somptueuse Kristina Dimitrova en 2007, pour info je rappelle que ce grand concours international, organisé par la marque Sloggi, réunissait le gratin de la fesse mondiale), et j’ai commencé à lui bouffer le nombril. J’adorais lui bouffer le nombril, c’était quelque chose que je n’aurais pas hésité un seul instant à placer en tête de liste de mes activités favorites, bien avant la pêche à la truite, l’homicide avec préméditation et l’écoute en boucle de l’adagio assai du Concerto en sol de Ravel (lequel, peu de temps après la composition de ce chef-d’œuvre intemporel, allait malheureusement devoir mettre un terme à ses activités pour cause de paralysie supranucléaire progressive). Il y a toute sorte de nombril, plus ou moins appétissant, mais le sien, véritable petite perle nacrée nichée au creux d’un écrin soyeux, représentait pour moi ce qui se faisait de mieux en la matière. En temps normal, il suffisait que je m’attaque à ce délicieux amuse-bouche pour devenir aussitôt la proie d’un appétit dévorant. Mais pas ce soir. Non, ce soir son nombril me faisait l’effet d’un vieux chewing-gum insipide à force d’avoir été mâché et remâché. Je me suis demandé ce que je foutais là, en train de léchouiller le nombril d’une femme que je connaissais à peine, ce que tout le monde foutait là, pourquoi tous ces gens continuaient à s’agiter vainement à la surface d’un caillou perdu au fin fond de l’univers, pourquoi on avait des bras, des jambes, des fesses, des langues, des yeux, des nez et des oreilles, à quoi rimaient toutes ces conneries qu’on était obligé de supporter à longueur de journée, pourquoi on se cassait le cul à naître si c’était pour disparaître quelques années plus tard, pourquoi on ne pouvait pas naître une bonne fois pour toutes et vivre jusqu’à la fin des temps sans avoir à se reproduire à tout bout de champ, pourquoi on n’arrêtait pas de se poser idiotes, et surtout pourquoi je n’étais pas déjà au lit en train de ronfler comme un bienheureux en attendant la fin du monde, la fin de tout en admettant qu’il y ait eu un début à quoi que ce soit. Tous ces concepts foireux m’insupportaient et ne m’inspiraient qu’une longue série de bâillements à m’en décrocher la mâchoire, tandis que ma langue, semblable à une limace anémique, une larve exsangue engluée dans sa propre bave, s’épuisait à récurer le bouton d’or de ma fleur des champs préférée.



\noindent Cette nuit-là, j’ai fait des rêves étranges dont seuls les épisodes les plus marquants surnagent encore dans ma mémoire.

Nathan, Sam, Greg, Titus et moi, accompagnés de Molawa VIII (voir note en bas de page) et ce bon vieux Jim Bowie, ancien trafiquant d’esclaves en Louisiane, caracolions dans la vallée de San Fernando à la recherche d’un temple maléfique dédié à Atiena, la Gardienne de la Nuit au corps de rêve et aux yeux vert céladon. Sous la forme d’un centaure entouré d’un harem de juments de Thessalie, le général Antonio Lopez de Santa Anna y Perez de Lebron, alias l’Aigle du Mexique, le Napoléon du Nouveau Monde ou encore le Héros Immortel de Cempoala, leur filait le train en tirant des coups de feu dans tous les sens avec ses Colt 1860 Army calibre 44 PN (il en avait un dans chaque main et chevauchait à toute berzingue en s’agrippant à son canasson à la seule force de ses cuisses trapues couverte d’une pilosité abondante).

En équilibre instable sur le dos d’une mule pas très accommodante, les couilles en vrac à force de rebondir sur le dos de la bête, je ne vous cache pas que j’avais toutes les peines du monde à suivre le rythme endiablé de cette folle cavalcade.

À noter aussi, chose qui ne va pas nécessairement de soi compte tenu du contexte susdécrit, que je ne portais ni cache-poussière, pantalon de cuir, santiags ni sombrero, mais un peignoir Aescwig Paige collection printemps-hiver 2017 en laine de soie peignée et microfibre de bambou 100\% bio du Suriname (cadeau de mon ami Zaahid Shirani, légiste de génie et beau-frère potentiel qui possédait exactement le même et prenait le plus grand plaisir à l’enfiler~-- entre autre~-- sitôt rentré chez lui après une dure journée de labeur) et des pantoufles en peau de Cottontail du désert (Sylvilagus auduboni, une espèce qu’on ne rencontre guère que dans les steppes semi-arides du désert de Sonora).

Pas non plus de six-coups tonitruant pour moi, mais juste Manu, mon fidèle 6.35 Manufrance, une arme de collection à laquelle j’étais attaché comme à la prunelle de mes yeux, vous le savez maintenant (il avait appartenu à mon grand-père Philibert, résistant de la première heure dont l’esprit taquin veillait en permanence sur moi depuis les hautes sphères où il résidait, en paix avec lui-même, sans doute occupé à tremper ses vers dans l’eau claire de quelque céleste ruisseau éternellement poissonneux, entouré de naïades au physique de reine de beauté attentives à satisfaire ses moindres désirs). À côté de la mitraille ambiante, les détonations de Manu ressemblaient aux jappements d’un de ces foutus roquets qui ne perdent jamais une occasion de brailler même lorsqu’ils se retrouvent face à un molosse qui fait cent fois leur taille, lequel molosse les regarde généralement de haut, l’œil morne, un filet de bave au coin du bec, sans y prêter plus d’attention qu’à une feuille morte balayée par le vent, dédain qui ne fait que pousser lesdits roquets à hurler de plus belle jusqu’à l’extinction de voix.

Note de bas de page, toujours pas en bas de page pour les mêmes raisons pratiques et esthétiques que précédemment : Molawa VIII, chef hottentot de la région du Cap, s’est éteint en 1830. Récupéré puis empaillé comme une vulgaire charogne exotique par les frères Verreaux, antiquaires peu portés sur le respect des morts mais beaucoup sur l’argent, il finit, début vingtième, par échouer entre les mains de Francesc d’Asis Darder et Llimona, vétérinaire catalan secoué du bocal et naturaliste pervers officiant en tant que directeur du zoo de Barcelone. Au crépuscule de sa vie, passée à amonceler jalousement tout ce que la nature a produit de plus insolite et terrifiant depuis les origines du monde, Darder lègue sa collection de reliques horrifiques au musée municipal de Banyoles, sa ville natale. Lance à la main et revêtu de son seul pagne bouffé aux mites, sa Majesté MOLAWA trône en bonne place dans la salle principale, au milieu des momies, têtes réduites, peaux de serpents, monstres en tous genres et autres joyeusetés zoologiques à l’avenant. Au début des années 1990, quand le fait d’exposer des pygmées dans des vitrines ne fait plus rigoler grand monde, hormis quelques nostalgiques du colonialisme et autres suprémacistes blancs, El Negro de Banyoles fait l’objet de ce que n’hésiterai pas à appeler une vive polémique. Les visiteurs de couleur se sentent quelque peu mal à l’aise face à ce compatriote empaillé comme un vulgaire babouin, témoin d’un passé pas si lointain, voire toujours d’actualité, où l’Homme Blanc régnait en maître absolu sur la Terre, traitant tout ce qui divergeait peu ou prou de son idéal occidental en bête de somme corvéable à merci. Alphonse Arcelin, médecin haïtien et néanmoins socialiste installé à Cambrils, sympathique station balnéaire du Baix Camp, près de Tarragone, entend parler de l’affaire, envoie une missive retentissante au maire de Banyoles, exigeant le retrait immédiat de cette horreur et sa restitution à qui de droit dans les meilleurs délais. Mais le maire en question, qu’une telle infamie n’émeut nullement, n’entend pas se laisser déposséder aussi aisément de l’une des principales sources de revenus de sa commune. Grâce à El Negro, son musée des horreurs ne désemplit pas. Il a augmenté les tarifs et aimerait bien continuer à se remplir les poches aussi longtemps que possible. Après tout, il ne s’agit que d’une enveloppe vide avec de la paille à l’intérieur, une enveloppe en forme de personne réelle, certes, mais qui permet aussi, au-delà de son aspect mortuaire discutable, de prendre la pleine mesure des fondements historiques de notre civilisation. Quelques siècles plus tôt, qu’on le veuille ou non, nos ancêtres découvraient avec stupeur qu’il existait des créatures plus proches d’eux qu’ils ne l’avaient imaginé jusqu’alors, d’apparence bien plus humaine que le singe, capables de communiquer entre elles avec un langage rudimentaire et d’effectuer des actions non dépourvues de logique et d’efficacité. Si la chose était avérée, l’histoire de notre filiation en serait grandement bouleversée. Alphonse Arcelin, fort de ses origines caribéennes, affronte alors une cabale néo-colonialiste qui nie haut et fort l’aspect éthique de la chose au profit de l’histoire et la science, seules valeurs acceptables dans ce cas de figure frappé de péremption. On ne va quand même pas se laisser casser les burnes par une poignée de primates endimanchés ! Si les descendants de ce glorieux personnage souhaitent se recueillir sur sa dépouille (comme neuve ou presque grâce aux bons soins des conservateurs attentifs qui se sont relayés à son chevet, soit dit en passant), ils devront faire le déplacement jusqu’à Banyoles, où l’entrée du musée leur sera bien évidemment gracieusement offerte, contrairement aux frais du voyage qui resteront à leur charge, faut pas non plus pousser mémé dans les orties. Après une lutte acharnée qui causera la ruine du valeureux médecin et son exil à Cuba, Molawa sera rapatrié en grande pompe sur ses terres natales et se verra, en présence des édiles locaux en tenue d’apparat frétillant d’aise sous l’œil embué des caméras internationales, offrir enfin une sépulture digne de sa condition.

C’est donc passablement ensuqué que j’ai ouvert les yeux sur un monde dans lequel je ne tenais pas plus que ça à me retrouver, un univers dont l’étroitesse m’apparaissait chaque jour un peu plus manifeste, odieuse et insoutenable, une cellule dont les murs se rapprochaient chaque jour davantage, réduisant mon espace vital à une bulle inconfortable, un trou à rat dont l’atmosphère confinée devenait chaque jour un peu plus suffocante. J’avais même, l’espace d’un instant, espéré qu’il avait enfin disparu pour laisser la place à une île perdue au fin fond d’une mer imaginaire, une île paradisiaque bénéficiant d’un climat idéal, de paysages enchanteurs sans cesse renouvelés, d’une beauté inépuisable, psychédélique, un coin de paradis entièrement peuplé de créatures de rêve totalement dépourvues d’inhibition, comme quand Fletcher Christian débarque à Tahiti après avoir jeté cette pourriture de capitaine Bligh à la mer, des créatures de rêve fascinées par ma chevelure dorée et l’azur magnétique de mon regard de grand fauve mélancolique. J’avais espéré, même si je n’étais pas exactement le sosie de Marlon Brando (un Marlon tellement investi dans son rôle d’aventurier du Nouveau Monde qu’il repartira du tournage avec Tarita Teriipaia, l’interprète de la belle et sauvage Maïmiti, avec laquelle il aura deux gosses dont une fille, Cheyenne, qui se suicidera vingt-cinq ans plus tard après que son demi-frère Christian aura abattu Dag Drollet, son compagnon violent, de deux balles dans le dos, sachant que Brando père, l’inoubliable interprète, entre autres, du Bal des maudits, Reflets dans un œil d’or, Missouri Breaks, la Comtesse de Hong-Kong et la Poursuite impitoyable, était aussi un bel enfoiré prétentieux et Cheyenne une gamine pourrie-gâtée toxico qui souffrait de troubles mentaux, comme quoi tout ne va pas forcément pour le mieux dans l’existence, même à Mulholland Drive, chose qui n’avait sans doute pas échappé à Lee Tamahori et David Lynch), tirer un trait sinon sur le fardeau de mon ancienne vie, au moins sur le fiasco de la soirée précédente, et repartir sur de meilleures bases vers un avenir souriant au ciel dégagé des noirs nuages du passé.

Et au lieu de ça, de cette perspective gorgée du nectar sucré de l’espérance, je me retrouvais une fois de plus dans mon lit, sans la moindre envie d’en sortir, avec la sensation toujours aussi désagréable d’avoir passé la nuit à mastiquer des copeaux de chêne du Limousin au fond d’une cave sombre et humide (ce qui est le cas de la plupart des caves quand elles sont bonnes).

Au prix d’un effort surhumain, du type de ceux que fournissaient les héros de l’Antiquité pour déplacer des montagnes ou terrasser des dragons, j’ai réussi à me laisser tomber du lit en douceur et sortir de la chambre sans réveiller Zarina. En même temps, réussir à réveiller Zarina quand elle dormait à poings fermés constituait un exploit autrement majeur que tous ceux accomplis par les héros et autres demi-dieux de la Mythologie. Vous avez déjà vu un troupeau d’éléphants traverser la jungle au pas de charge, ou assisté à l’effondrement d’une tour de cent cinquante étages ? Eh bien, si ce même troupeau avait traversé la chambre au pas de charge ou cette même tour s’était effondrée au pied du lit, Zarina n’aurait même pas soulevé une paupière pour s’enquérir de ce qui se passait. Qu’un séisme, un ouragan s’abatte sur elle, que des rafales de vent d’une violence extrême arrachent les fenêtres de la chambre et que des pluies diluviennes la remplissent d’eau jusqu’au plafond ? Peu importe, elle aurait continué à dormir comme si de rien n’était. Vous pensez peut-être qu’une guerre atomique, déclenchée par les milliardaires caractériels qui dirigent ce monde, aurait pu venir à bout de sa résistance. Eh bien je vous fiche mon billet qu’on l’aurait retrouvée endormie au milieu des décombres fumants de l’immeuble, sans une égratignure, aussi fraîche que la rosée du matin. Sauf qu’il n’y aurait plus de rosée du matin, ni de matin tout court, du reste, mais seulement la nuit sans fin de l’Apocalypse, le silence assourdissant de la mort et les vapeurs toxiques qui recouvrent tout d’une épaisse couche de moisissure radioactive. Et quand elle ouvrirait enfin les yeux, ce serait pour se rendre compte que tout avait disparu. Enfin, presque tout, parce que j’aurais survécu moi aussi, bien évidemment. ADAM \& EVE 2, l’éternel retour, mais sans le jardin d’Eden, les arbres chargés de fruits défendus et les serpents maléfiques qui sifflent sur nos têtes pour nous pousser à croquer dedans. Tel le Phénix d’Héliopolis, le Fenghuang de la dynastie Han, le Minka aborigène ou encore le Wakinyan Tanka des Sioux, je renaîtrais de mes cendres, nu comme un asticot blafard et grassouillet, aussi intact et glorieux qu’au premier jour, après quoi je m’empresserais d’aller réveiller ma princesse endormie d’un baiser langoureux sur ses lèvres brûlantes de désirs inassouvis. C’est à nous que reviendrait le privilège de repeupler le monde de petits condensés de nous-mêmes et relancer la folle histoire de l’espèce humaine, indigne s’il en est, aussi médiocre et imparfaite que terriblement prétentieuse, imbue d’elle-même jusqu’à l’overdose existentielle, le trauma crypto-sensoriel de forte magnitude, mais exceptionnellement autorisée, à travers nous, à tenter une nouvelle fois sa chance dans ce coin perdu au fin fond de l’univers qu’on appelle la Terre.

Et d’ailleurs, à propos de repeupler l’univers, je ne sais pas si vous avez remarqué, mais les lendemains de cuite, outre un solide état de déprime arrimé au cuir chevelu, se signalent souvent par un niveau d’excitation sexuelle anormalement élevé. La plupart du temps, il faut recourir à une activité masturbatoire proche de la frénésie pour y remédier. Je suppose que le sexe est un antidépresseur naturel, comme le safran, la camomille, l’orpin et le millepertuis, ce qui est encore une façon très habile (et habituelle de la part d’une entité aussi sournoise et rouée que dame Nature, rompue à tous les artifices pour arnaquer son monde), compte tenu du marasme existentiel qui le nôtre, d’encourager l’espèce à se reproduire malgré tout, ce qui, sur le plan de la pure objectivité scientifique, ferait clairement de l’alcoolisme un rouage essentiel de l’évolution des espèces, la nôtre en l’occurrence, sachant que notre avenir est de toute façon largement compromis. Un peu plus un peu moins, on n’en est pas à une cirrhose près. Dans le cas présent, j’aurais pu me soulager dans l’arrière-train de Zarina sans qu’elle y trouve à redire, d’autant moins, comme je viens de vous l’expliquer, que rien ne pouvait la réveiller quand elle dormait à poings fermés. Entendez par là que le troupeau d’éléphants (pachydermes dont on imagine sans peine la taille colossale du membre et le poids abyssal des burnes) susévoqué aurait pu lui passer dessus à plusieurs reprises, un par un ou tous ensemble, et je parle cette fois au sens de l’expression dans ce qu’elle a de plus bestial et ordurier, sans susciter la moindre réaction de sa part. J’ai néanmoins, dans ma grande sagesse, jugé préférable de n’en rien faire. Quant à l’érection monumentale (tout est relatif, bien sûr, rien à voir avec celles d’un Jonah Falcon ou un Roberto Cabrera) qui me précédait dans tous mes déplacements (fidèle animal de compagnie tenu fermement en laisse sans quoi il n’aurait pas manqué de se ruer sur tout ce qui bouge), je lui ferais rendre gorge dès que mon emploi du temps éminemment chargé me le permettrait. Il était de toute façon hors de question de me présenter au bureau dans cet état, le personnel féminin n’ayant alors pas manqué de me traîner en justice sous les chefs d’inculpation les plus divers, tels que inculpation ou inculpalpation, pourquoi pas, tandis que le personnel masculin, moins procédurier, n’en aurait pas moins conçu à mon égard une rancune tenace pour avoir osé exhiber en toute impudeur des attributs d’une telle magnificence.

Mon périple matinal, loin d’être terminé, m’a ensuite conduit à la cuisine, où j’ai entrepris de mettre en branle la machine à café.

Pendant qu’il coulait, j’avais tout le loisir de me traîner jusqu’à la salle de bain afin de prendre une bonne douche et me laver les dents, autant d’activités d’un ennui mortel auxquelles je me devais néanmoins de sacrifier avant de rejoindre les rangs de la civilisation. Il se trouve que moi, Djeferson Beauvais avec un D et un seul F, commandant de police judiciaire de Saint-Nom-la-Branlette, j’étais soumis à quelques obligations difficilement dispensables, même s’il m’arrivait parfois d’officier dans un état de semi-ébriété coupable, avec des trous dans les chaussettes au niveau du gros orteil et un slip d’une propreté douteuse (le modèle Brando en jersey de coton de chez Dolce \& Gabbana, d’accord, mais tout de même).

Un quart d’heure plus tard, après m’être consciencieusement récuré la couenne, généreusement vidé les bourses et dentifricé le ratichier, j’étais, sinon l’éphèbe éminemment désirable de ma prime jeunesse, au moins en état de faire illusion auprès des gens que je serais amené à rencontrer pendant la journée.

C’est donc plein d’espoir que je me suis rendu à la cuisine pour siroter mon demi-litre de café matinal, dégusté seul, pour lui-même, sans l’adjonction de quelque autre denrée alimentaire que ce soit.

J’ai jeté un œil à mon téléphone et constaté que Bérénice avait essayé de m’appeler plusieurs fois aux premières lueurs de l’aube, alors que j’en écrasais encore sévèrement sous les quelques centimètres carrés de drap que Zarina avait consenti à me laisser, sachant que chaque dixième de millimètre de chacun d’eux avait été conquis de haute lutte, à la force du poignet, au cours de la nuit assez peu réparatrice qui avait été la mienne (Zarina, quoique dépourvue d’origines asiatiques à ma connaissance, avait en effet la fâcheuse manie de s’enrouler dans les draps comme une farce au porc, champignons noirs, vermicelle et pousses de soja dans une fine galette de riz).

Je m’apprêtais à la rappeler, aussi rapidement que possible mais sans excès de vitesse inconsidéré (cet appel ne me disait rien qui vaille et je n’étais par conséquent pas particulièrement pressé de donner suite), quand la sonnerie de mon téléphone (une version électro de la Chevauchée des Walkyries de Richard W, suffisamment immonde et irritante pour forcer le plus réfractaire à la téléphonie mobile et au progrès en général à décrocher dans la seconde) s’est fait entendre.

J’ai pris une grande respiration, décroché et tenté de présenter mes salutations respectueuses à l’intéressée.

Elle ne m’en a pas laissé le temps : Je peux savoir ce que vous avez fait hier soir ?

\textsc{Moi} : Hein ? Quoi ? Qui ça ?

\textsc{Elle} : Titus et toi.

\textsc{Moi}, l’air aussi dégagé que possible : Eh bien… pas grand-chose, à vrai dire… Pourquoi tu me demandes ça ?

\textsc{Elle}, hurlant presque : TITUS N’EST PAS RENTRÉ DE LA NUIT !

\textsc{Moi}, qui avais refusé de croire à cette éventualité et devais me rendre à l’évidence que j’avais sans doute fait preuve d’un peu trop de légèreté dans cette histoire : Ah !… hum hum… oui, en effet, c’est assez embêtant… Alors comme ça, tu dis que… qu’il…

\textsc{Elle} : N’est pas rentré de la nuit, oui, c’est ce que je viens de dire !

\textsc{Moi} : C’est bizarre, quand même.

\textsc{Bérénice} : Oui, je trouve aussi. Est-ce que, par le plus grand des hasards, tu aurais des informations à ce sujet ?

\textsc{Moi} : Eh bien, écoute, euh, là j’avoue que tu me prends un peu de court… Il est tôt, je viens de sortir du lit, je ne suis pas encore très bien réveillé… Alors… oui… effectivement… maintenant que tu le dis… je crois me souvenir qu’il nous a quittés à une heure assez avancée de la nuit… aux alentours de… enfin… je ne me rappelle plus exactement… mais… assez tard, en effet… ou tôt, suivant la façon dont on… enfin, tu vois ce que je veux dire…

\textsc{Elle} : Et ?

\textsc{Moi} : Eh bien je me suis dit qu’il rentrait chez lui, voilà tout.

\textsc{Elle} : Il a dit qu’il rentrait ?

Votre serviteur, au comble de l’inconfort : Euh… non, il n’a rien dit de particulier à ce sujet… mais je ne vois pas ce que je pouvais penser d’autre à ce moment-là.

\textsc{Elle} : Eh bien, il se trouve qu’il n’est pas rentré.

\textsc{Moi} : T’es sûre ?

\textsc{Elle} : Certaine, oui.

\textsc{Moi} : Peut-être qu’il est rentré et… et qu’il est ressorti…

\textsc{Elle} : J’ai le sommeil assez léger, je crois que je m’en serais rendu compte. Vous faisiez quoi, quand il est parti ?

\textsc{Moi} : Mon Dieu… pas grand-chose, comme je te l’ai dit… On était en opération spéciale… strictement confidentielle… je ne peux malheureusement pas t’en dire davantage à ce sujet, question de sécurité, la tienne et celle de tes proches… mais il se trouve que… comment dire… eh bien il se trouve que les choses ne se sont pas exactement déroulées comme prévu… Les impondérables, tu sais ce que c’est.

\textsc{Elle} : Non, pas trop. JE CROIS SURTOUT QUE T’ES EN TRAIN DE BIEN TE FOUTRE DE MA GUEULE !!!!!!!!!

\textsc{Moi}, écarquillant des yeux ronds comme des culs de babouins : COMMM-MMMENT ???????????!!!!!! Mais enfin comment oses-tu imaginer une chose pareille ! Non, je suis offusqué, là, limite vexé. L’affaire n’avançait pas, s’étirait inutilement, en un mot comme en cent on se trainait comme des merdes dans le purin, des limaces dans la bave, des…

\textsc{Elle} : Oui, bon, ça va, épargne-moi tes comparaisons douteuses !

\textsc{Moi} : Bref, on s’est dit que ça ne servait à rien de continuer à patauger dans la semoule comme des mamelouks en détresse. Quand Titus a décidé de partir, j’en ai conclu tout naturellement qu’il rentrait chez lui.

\textsc{Elle} : Ben voyons. Je l’ai appelé plusieurs fois, j’ai laissé des messages, et il ne répond toujours pas ! Tu peux essayer de faire quelque chose, s’il te plaît ?

\textsc{Moi} : Mais bien sûr, Bérénice, tu sais aussi bien que moi que tu peux compter sur… enfin sur moi, quoi… Je vais faire tout ce qui est en mon pouvoir (autant dire pas grand-chose, note de la Rédaction) pour le retrouver. Cela dit, je ne suis pas sûr qu’il réponde davantage si c’est moi qui appelle…

\textsc{Elle} : Mais tu peux localiser son téléphone, non ?

\textsc{Moi} : Oui, en principe.

\textsc{Elle} : Comment ça, en principe ?

\textsc{Moi} : Eh bien… il y a des fois où on peut, et d’autres fois où on ne peut pas…

\textsc{Elle} : Ah bon ? Vous n’avez pas des spécialistes pour faire ce genre de chose ?

\textsc{Moi} : Si, mais peut-être qu’il n’a plus de batterie.

\textsc{Elle} : On ne peut pas le localiser s’il n’a plus de batterie ?

\textsc{Moi} : C’est plus difficile. Peut-être aussi que son téléphone a été détruit pour une raison ou pour une autre…

\textsc{Elle} : Et alors ? Il pourrait… je ne sais pas, moi, m’appeler d’une cabine ! Il sait pertinemment que je me fais un sang d’encre quand je suis sans nouvelles ! Tu n’as vraiment aucune idée de l’endroit où il peut être ?

\textsc{Moi}, tel Judas Iscariote (le Judas «~sikariot~» de la Peshitta, traduction syriaque de la Bible chère aux Maronites) claquant la bise au Roi des Juifs dans les jardins de Gethsémani : Ma foi… comme ça… non, je ne vois pas… Je vais me renseigner pour savoir s’il ne lui est pas arrivé quelque chose… Je ne pense pas, mais on ne sait jamais… un accident… que sais-je…

\textsc{Elle} : Je ne sais pas pourquoi, mais j’ai la désagréable impression que tu me caches quelque chose.

\textsc{Moi}, pour une poignée de deniers : Hein ??? Quoi ??? Comment ???

\textsc{Elle} : Tu as très bien entendu.

Et pour quelques deniers de plus : Mais jamais de la vie, voyons ! Qu’est-ce que tu vas imaginer !

\textsc{Elle} : Je connais les hommes.

\textsc{Moi}, plus roublard que Tuco : Je ne vois pas de quoi tu veux parler.

\textsc{Elle} : Ils sont prêts à se couvrir les uns les autres jusqu’à la mort, au nom de je ne sais quel sens débile de l’amitié virile et l’esprit de corps !

\textsc{Moi} : Ce n’est pas pire que les femmes qui passent leur temps à se tirer dans les pattes !

\textsc{Elle} : Tu me jures que tu ne sais pas où est Titus ?

\textsc{Moi}, plus répugnant qu’une punaise de lit sortant furtivement de sa tanière pour sucer le sang de sa victime endormie : Tu penses bien que je te le dirais si je savais quelque chose ! Titus est mon meilleur ami, c’est vrai, mais toi et les enfants comptez aussi beaucoup pour moi.

\textsc{Elle} : N’empêche que tu n’as pas répondu à la question.

\textsc{Moi} : Quelle question ?

\textsc{Elle} : Est-ce que tu me jures que tu ne sais pas où est Titus ?

\textsc{Moi}, après un raclement de gorge suffisamment prononcé pour laisser planer le vautour déplumé du doute au-dessus de la charogne faisandée de ma duplicité grouillante des asticots replets de la bêtise crasse et la lâcheté triomphante (oui, bon, ça va, personne n’est parfait) : Bien sûr, que je te le jure, mon amour !

\textsc{Bérénice} : Quoi ???!!!!

\textsc{Moi}, confus : Excuse-moi, ça m’a échappé dans le feu de l’action.

\textsc{Elle} : Tu m’aimes ?

\textsc{Moi} : Bien sûr, que je t’aime, comme la femme de mon meilleur ami, comme une sœur, une personne chère, hors de prix, à laquelle il ne me viendrait jamais à l’idée de mentir.

\textsc{Elle} : Ah bon, tu m’as fait peur !

\textsc{Moi} : Tu veux dire que si je te disais que je t’aime, non pas seulement comme la femme de mon meilleur ami, mais comme une femme qui fait naître en moi les désirs les plus… comment dire… intenses, au sens érotique du terme, tu en concevrais une certaine… rancœur, amertume ?

\textsc{Elle} : Disons que je serais un peu perturbée, oui.

\textsc{Moi} : Eh bien rassure-toi, il n’en est rien.

\textsc{Elle} : Encore heureux !

\textsc{Moi}, évoluant avec une grâce discutable sur le fil d’un rasoir qui commençait à m’entamer sérieusement la voûte plantaire : Écoute, Bérénice, ma douce et tendre Bérénice, Bébé mon bébé, je ne sais pas au juste ce que tu vas imaginer, mais dis-toi bien que…

\textsc{Elle}, d’une voix dont la surface se voulait d’un calme olympien mais n’en laissait pas moins transparaître une agitation comparable à celle d’un groupe d’orques en train de mettre en pièces un requin-baleine dans les eaux troubles du golfe du Mexique : Je n’imagine rien du tout. Je note juste que mon mari a découché, ce qui ne lui était encore jamais arrivé sans un mot du médecin, qu’il ne répond pas au téléphone quand je l’appelle, et que son meilleur ami, un type plutôt bizarre avec lequel il a passé la nuit, prétend ne rien savoir de ce qui lui est arrivé. Quant aux enfants, qui ont l’habitude de déjeuner avec leur père avant de partir à l’école, ils me posent des questions auxquelles je suis incapable de répondre.

\textsc{Moi} : Dis-leur qu’il est parti au boulot un peu plus tôt que prévu.

\textsc{Elle} : Certainement pas ! Je ne leur ai jamais menti, ce n’est pas maintenant que je vais commencer.

\textsc{Moi} : C’est tout à ton honneur.

\textsc{Elle} : C’est jamais bon de mentir aux enfants.

\textsc{Moi} : C’est évident. Ils comptent sur nous, nous font une totale confiance. Si on commence à leur raconter n’importe quoi, ils perdent leurs repères et partent en vrille.

\textsc{Elle} : Quand la confiance est altérée, tout s’écroule.

\textsc{Moi} : Ouais. Cette connerie de Père Noël, par exemple, ce vieux barbu en pyjama rouge, qui se la joue papy sympa mais sent le pédophile à plein nez, le «~bad Santa~» obsédé du cul, se trimbalant en permanence une érection digne d’un cheval de trait, une gaule à casser des noix, censé se balader en traineau dans le ciel et passer par les cheminées avec sa hotte pleine de jouets… Ben voyons ! Vu son tour de taille, ce bouffon n’arriverait même pas à passer dans le tunnel sous la Manche sans toucher les bords ! C’est l’exemple-type des conneries qu’on raconte aux gosses pendant des années, jusqu’au jour où ils apprennent brutalement qu’on leur a menti soi-disant pour la bonne cause, essayer de mettre un peu de rêve dans la bouillie insipide de la vie quotidienne, fêter dignement le jour béni où Jésus a ouvert les yeux dans une étable de Bethléem, le plus simplement du monde, sans se soucier du fait que l’étable en question, remplie de paille et éclairée à la bougie, pouvait prendre feu à tout moment.

\textsc{Elle}, très remontée, au point d’utiliser une formulation qui n’était pas habituellement la sienne (je n’irais pas jusqu’à dire que j’en étais choqué, mais tout de même, je découvrais une facette de sa personnalité dont j’avais à peine soupçonné l’existence jusqu’ici) : Rien à secouer de l’âne, du bœuf et des Rois mages !

\textsc{Moi} : Exact, qu’ils aillent tous se faire foutre ! À commencer par ce prétendu Messie qui aurait mieux fait de nous prévenir gentiment qu’on allait passer le restant de nos jours à mentir, voler, violer et assassiner son prochain ! Lui aussi nous a bien roulés dans la farine comme des boulettes de viande avant de passer à la casserole ! En même temps, comment faire confiance à un type qui prétend être le fils de Dieu, rien que ça, débarque pour sauver le monde et n’est même pas fichu de se sauver lui-même ? Mais excuse-moi, je crois que je m’égare un peu.

\textsc{Elle} : Non, t’as raison. Je ne comprends toujours pas pourquoi on continue à fêter la naissance de ce type deux mille ans après. Regarde, moi, par exemple, on n’a pas fait tout ce ramdam le jour où je suis venue au monde !

\textsc{Moi} : Pareil pour moi, à commencer par mes parents qui s’en foutaient comme de l’an quarante !

\textsc{Elle} : Ah bon ?

\textsc{Moi} : Bien sûr. Je te l’ai déjà raconté, non ?

\textsc{Elle} : Je crois pas, non.

\textsc{Moi}, pas avare de confidences aux premières heures du jour, quand les derniers fêtards rentrent chez eux ivres morts, que les travailleuses du sexe sucent leurs derniers clients, que les agents de propreté urbaine collectent leurs dernières ordures sur les trottoirs de la ville, que les rats d’égout regagnent furtivement leur tanière la dent jaune et le ventre plein, que les collégiens matent fébrilement une dernière vidéo de boule sur les réseaux sociaux avant d’aller en classe, que le soleil se lève paresseusement entre deux nuages endormis, que les dealers goûtent quelques heures de repos bien mérité avant de retourner au taf, que les politiciens véreux, hommes d’affaires corrompus, chanteurs de charme affiliés à la mafia et autres marchands de mort spécialisés dans la vente d’armes aux pays sous embargo sillonnent le ciel dans leurs jets privés pour aller planquer leur pognon dans les paradis fiscaux, que les milliardaires paranoïaques et mégalos, accros aux drogues dures, aux partouzes et à la pêche au gros, lancent leurs dernières OPA pour devenir enfin les maîtres du monde, se prendre des cuites au Clos D’Ambonnay, se rouler dans les œufs d’esturgeon et se taper des putes mineures sur des yachts de 50 mètres de long au large des côtes de Tanzanie : Ma mère ne voulait pas d’enfant, mais elle était super canon et mon père ne pensait qu’à la niquer par tous les orifices. La plupart du temps, il se contentait de l’enculer vite fait sur la table de la cuisine ou dans le local à poubelles, chose qui, je ne m’explique toujours pas pourquoi, avait le don de le mettre dans un état de transe sexuelle irrépressible, et comme ma mère oubliait régulièrement de prendre la pilule et que mon père n’était pas un adepte du retrait d’urgence au moment fatidique, ce qui ne devait pas arriver arriva néanmoins. Je naquis donc, beau bébé replet promis au plus brillant avenir, à ceci près que mon père voulait une fille et que le service trois pièces de toute beauté que j’affichais au sortir du ventre maternel ne laissait planer aucune équivoque sur la nature de mon sexe. Sa tronche s’est décomposée quand il a posé les yeux dessus, il a claqué les talons, comme tout bon militaire qui se respecte, s’est fendu d’un demi-tour-droite irréprochable techniquement, a quitté la chambre sans dire un mot, droit comme un I, puis est allé se saouler la gueule avec ses potes de régiment pendant une durée indéterminée qui a duré exactement trois jours et trois nuits. Ensuite, il s’est repointé à la maison comme si de rien n’était, et ne s’est pas gêné pour m’en faire baver des ronds de chapeaux jusqu’à ce que je sois enfin en âge de me barrer sans demander mon reste.

\textsc{Elle} : Ton père était dans l’armée ?

\textsc{Moi} : Oui, un de ces tueurs professionnels qui sillonnent la planète pour faire un peu de ménage. Sur ce, je te souhaite une bonne journée.

\textsc{Elle} : Tu te fous de ma gueule ?

\textsc{Moi} : Non… excuse-moi… ce n’est pas ce que je voulais dire… enfin si… passe une bonne journée dans le sens où ne t’en fais pas, je vais retrouver Titus et tout va rentrer dans l’ordre.

J’ai prévenu le bureau que je risquais d’être en retard, mais comme j’étais le chef je pouvais faire à peu près ce que je voulais et me permettre d’être en retard quand j’avais des affaires plus importantes à traiter que ce qui constituait l’ordinaire des activités d’un commissariat de quartier, à savoir faire des rondes dans les quartiers chauds, contrôler des gens de couleur qui n’ont rien demandé à personne, faire souffler des automobilistes dans des éthylotests, verbaliser des pervers à la sortie des écoles, et enregistrer des plaintes de vieilles dames qui se sont fait braquer leurs économies par des escrocs souriants et bien habillés (il avait l’air si gentil).

Je me suis habillé en vitesse, et j’ai ouvert délicatement ma porte pour jeter un œil sur le palier. Pas de mère Ouvrard ni de Marc-Antoine Jacquinot en vue, tapis dans un angle mort pour fondre sur moi comme l’aigle sur sa proie. Juste Korax avachi sur le paillasson de sa porte d’entrée, plus sournois et dédaigneux que jamais, bien entendu, mais apparemment peu disposé à tenter une de ces attaques surprises dont il avait le secret. C’est donc l’esprit relativement dégagé que j’ai pu prendre l’ascenseur, dévaler (à vitesse réduite, l’engin n’étant pas de la toute première jeunesse, ni même de la seconde) les six étages qui me séparaient de la terre ferme, sortir de l’immeuble et respirer enfin les fraîches senteurs de l’aube si nécessaires à mon équilibre mental, même si raisonnablement polluées par le monoxyde de carbone, le dioxyde d’azote et les particules fines chargées d’hydrocarbures polycycliques hautement cancérigènes contenus dans les gaz d’échappement.

Contrairement à son habitude, le 1,6 16v de ma Kangoo, boosté à près de deux cent cinquante chevaux de course par les bons soins de Nathan, le frère de Greg, a consenti à s’ébrouer après seulement deux tentatives infructueuses, ce qui n’était pas loin de constituer un record personnel. Le modeste quatre cylindres produisait maintenant un feulement rauque de grand fauve en rut, subtilement mis en valeur par une ligne d’échappement optimisée qui m’avait coûté la peau des fesses. La prochaine mission de Nathan, si toutefois il l’acceptait, consisterait à doter le paisible utilitaire de tous les aménagements nécessaires pour se transformer à loisir en voiture-bélier, aéronef ou sous-marin de poche. Enfin, un exosquelette à cristaux liquides permettrait de rendre le véhicule invisible à la demande, par simple action sur un commutateur dédié. On n’en était pas encore là, mais je disposais tout de même d’un certain nombre de gadgets intéressants, tels que blindage intégral, quatre roues motrices avec pneus increvables, autoradio à écran tactile Bluetooth, système audio haute performance avec ampli de 900 watts, mais aussi et je dirai même surtout boîte à gants transformée en humidor pour une parfaite conservation de mes cigares préférés.

Et justement, à propos de cigares, j’ai ouvert la boîte à gants, me suis glissé un Fuente Short Story dans le bec, et l’ai allumé à la flamme de mon Guevara Vintage double jet anti-tempête (utile en mer par gros temps quand on souhaite s’en griller un sous des trombes d’eau, ce qui n’est pas chose facile, ou encore quand on se retrouve pris au piège par une tornade qui avance inexorablement en ravageant tout sur son passage et qu’on s’apprête à allumer ce qui sera sans doute son dernier cigare).

Un peu tôt pour commencer à fumer, je vous l’accorde, mais les mauvaises habitudes sont les plus difficiles à perdre et la journée s’annonçait aussi chargée que l’haleine d’une hyène qui vient d’arracher ses derniers lambeaux de chair à une carcasse de gnou faisandée depuis des lustres sous le soleil impitoyable de Namibie. Le Kalahari, par exemple, en dépit de l’aridité redoutable qu’on lui connaît, n’en offre pas moins des paysages d’une beauté à couper le souffle, et ce ne sont certainement pas les amateurs de safari en Land Rover et de bivouac sous les étoiles qui me contrediront. Car tous savent bien qu’il n’est pas donné à tout le monde de connaître ces moments inoubliables où le lion rugit à quelques mètres de vous, le vautour vous tourne autour dans le ciel chauffé à blanc, le guépard vous regarde droit dans les yeux et le suricate vous mange dans le creux de la main. Tout comme il n’est pas donné à tout le monde~-- et heureusement, du reste, car ils n’hésiteront pas à vous tirer une flèche empoisonnée dans le cul si vous vous avisez de leur manquer de respect~-- de croiser la route des fiers Bochimans qui arpentent les pistes du désert depuis des dizaines de milliers d’années, parlent à la Lune comme à une vieille amie et pratiquent aujourd’hui encore les étranges rituels hérités de leurs plus lointains ancêtres. Bien sûr, ils ne fument pas la pipe ou le cigare, comme vous et moi, et n’imaginent même pas qu’on puisse résider à plein temps dans un manoir des Highlands, mettre des glaçons dans un verre de Bruichladdich de 30 ans d’âge, porter des costumes sur mesure, piloter des avions de chasse et tirer des missiles balistiques sur des sites stratégiques d’Europe de l’Est ou du Moyen-Orient. Non, tout ça leur passe largement au-dessus de la tête. Mais ils n’en ont pas moins, dans le dénuement le plus total, vêtus de simples peaux de bêtes et exposés sans cesse à la cuisante morsure du feu solaire, réussi à survivre et s’intégrer dans l’environnement hostile qui a toujours été le leur. Jamais, certes, vous ne les verrez défiler en kilt dans les rues de Fort William, s’époumoner dans des cornemuses ni admirer les moutons de Rosa Bonheur dans les musées de Londres, Hambourg ou Washington, mais soyez certain qu’ils sauront vous défendre au péril de leur vie si le babouin vous menace, le phacochère vous charge ou l’éléphant furibard tente de vous écrabouiller comme la grosse merde de capitaliste nuisible et envahissant que vous êtes (le pachyderme en question étant tout particulièrement motivé du fait que vous n’avez cessé de le persécuter pour lui voler ses défenses et lui couper les pattes pour en faire des cendriers, ce qui montre bien l’extrême perversité et la frivolité inqualifiable de vos agissements).

Greg, réveillé aux aurores, m’attendait sur le pas de sa porte, rasé de près, l’œil vif et le jarret fringant, impatient de prendre part à ce qui s’annonçait sinon comme l’aventure la plus palpitante de sa vie, au moins de ces dix dernières années. C’était, à priori, le genre de chose qu’on prend plaisir à raconter à ses petits-enfants au coin du feu quand on est vieux, malade et sur le point de crever, à condition bien sûr d’avoir des petits-enfants et une cheminée à portée de main, et que le fait d’être sur le point de crever nous permette encore de prononcer quelques paroles vaguement intelligibles sans foutre la trouilles aux gosses.

À vrai dire, n’ayant ni enfants et encore moins petits-enfants, je crois surtout qu’il était impatient de prendre la fuite, Lou (ou Loulou suivant les cas, rarement Louloulou ou Loulouloulou, de son vrai nom Louise De La Croix, créature pour tout dire assez fascinante issue d’une vieille famille d’aristos désargentés, dont le père était un escroc notoire ayant fait plusieurs fois le tour de la planète pour échapper à ses créanciers, tant et si bien que plus personne~-- et sans doute pas même lui~-- ne savait précisément où il était ni même s’il était encore en vie, cette dernière option paraissant hautement improbable vu le nombre de gens prêts à sacrifier tout ou partie de leur fortune pour avoir sa peau), sa nouvelle copine nymphomane, ne lui laissant pas une seconde de répit. C’était une fille dont il avait fait la connaissance du temps où il bossait quinze heures par jour comme analyste financier chez Reckless \& Knot, avec laquelle il avait… comment dire… partagé quelques affinités électives dans les endroits et positions les plus divers et incongrus, avant de lui faire part de sa décision longuement mûrie en ses âme et conscience de mettre un terme à toute activité autre que strictement professionnelle les concernant, considérant que cette incursion dans le domaine du privé, en dépit de son caractère physiquement satisfaisant, là n’était pas la question, s’avérait néanmoins incompatible avec la bonne marche de l’entreprise en général et de sa santé mentale en particulier. Mais ne voilà-t-il pas, au moment où il s’y attendait le moins (il ne faut jamais dire jamais, c’est bien connu, et surtout se tenir prêt en permanence à toute éventualité), que l’imprévisible Lou venait de refaire surface et lui mettre à nouveau le grappin dessus (assez facilement, il faut bien le dire), plus chaude qu’un brasier dévastant des centaines d’hectares de garrigue dans l’Hérault. Fragile psychologiquement, autant que sexuellement fortement attiré par cet incendie que nulle compagnie de soldats du feu n’était jamais parvenue à maîtriser, Greg avait commis l’erreur de remettre un doigt dedans et s’était aussitôt retrouvé pris au piège (je pense alors au fameux fingertrap de la famille Addams, cadeau de dixième anniversaire de Fester, le frère de Gomez, qui a dû apprendre à manger avec les pieds parce qu’il est resté coincé dedans pendant deux ans).

Et pour vous dire toute la vérité, eh bien sachez que le Greg que j’évoquais précédemment, rasé de près, à l’œil vif et au jarret fringant, tel un Bilbon Sacquet prêt à prendre part aux aventures les plus rocambolesques avec une bande de Nains sortis de nulle part, des magiciens et des Elfes sylvains pas toujours très bien disposés, affronter des hordes d’Orques hideux et terrasser des dragons gardiens de trésors sous des montagnes solitaires, ce Greg-là n’était hélas rien d’autre qu’une vue de l’esprit, un mirage, un fol espoir, un fantasme à mille lieues de la réalité affligeante qui s’offrait à ma vue dépitée.

Quand je l’ai vu arriver en traînant les pieds, tel un petit vieux accablé par le poids des ans et pressé d’en finir avec une existence qui lui a procuré en gros quatre-vingt-dix et quelques pour cent d’emmerdes pour dix tout petits pour cent de moments de vague satisfaction aussi rares que fugaces (autrement dit pas du tout assez pour faire pencher favorablement la balance), monter dans la bagnole avec l’entrain d’une vache qu’on fait entrer à coups de pied dans le cul dans une bétaillère pour la conduire à l’abattoir, me tendre une main tellement molle que j’ai eu l’impression de serrer la pince à une flaque de vomi, attacher sa ceinture avec résignation, poser les mains bien à plat sur ses genoux et se mettre à regarder fixement devant lui sans desserrer les dents, j’ai compris que je n’étais pas au bout de mes peines.

J’ai cherché son regard pendant quelques instants, sans la moindre réaction de sa part, et lui ai posé la première question qui m’est venue à l’esprit, assez basique il est vrai, et d’autant plus superflue que j’en connaissais déjà la réponse : Ça va ?

\textsc{Greg}, sans tourner la tête : Non, pas vraiment.

\textsc{Moi} : Mal dormi, peut-être ?

\textsc{Lui} : Pas fermé l’œil de la nuit.

\textsc{Moi} : Lou ?

\textsc{Lui} : Lou.

\textsc{Moi} : Elle t’attendait ?

\textsc{Lui} : Derrière la porte, dans sa tenue la plus suggestive.

\textsc{Moi}, la tête noyée dans un nuage de fumée : Lumière tamisée, senteurs orientales, lingerie fine, résille et balconnet, difficile de résister.

\textsc{Lui} : Pas exactement, mais elle a des moyens de persuasion très efficaces. Même un eunuque n’y résisterait pas.

\textsc{Moi} : Un moine bénédictin, peut-être ?

\textsc{Lui} : Pas davantage.

\textsc{Moi} : Tu veux dire que saint Benoît de Nursie lui-même n’aurait pas résisté longtemps à ses avances ?

\textsc{Lui} : Le pauvre vieux n’aurait pas tenu trente secondes.

\textsc{Moi} : Que si Augustin d’Hippone l’avait croisée dans les rues de Carthage ou les travées de la basilique Saint-Pierre de Rome, ses Confessions compteraient bon nombre de pages en plus ?

\textsc{Lui} : Bon nombre, et je ne suis pas certain qu’il aurait osé les écrire toutes.

\textsc{Moi} : J’en conclus qu’elle ferait bander un mort.

\textsc{Lui} : Un cimetière tout entier !

\textsc{Moi} : Avec elle, les cadavres sortent de terre avec une trique d’enfer ! Lève-toi et gicle !

\textsc{Lui} : Tout à fait. Tel Lazare recroquevillé dans le fond de sa tombe, ma bite croupissait dans le fond de mon slip, aussi inerte et engluée dans sa propre bave qu’une limace à l’agonie.

\textsc{Moi} : Quelle vision lugubre et déprimante. Ce membre éminent du genre humain à jamais perdu pour la France, le monde, l’univers tout entier ! La théorie de l’évolution bouleversée dans ses fondements mêmes, ébranlée dans ses fondations, durement secouée.

\textsc{Lui} : Une tragédie, oui, on peut dire ça.

\textsc{Moi} : Digne des riches heures de l’Antiquité, les champs de bataille dévastés, les ruines jonchées d’ossements, les rats courant ici et là, à la recherche d’un lambeau de chair à grignoter.

\textsc{Lui} : Oui, abominable. Je ne sais pas comment elle s’y est prise, quel stratagème elle a utilisé, mais toujours est-il qu’elle a réussi à la ressusciter.

\textsc{Moi} : Un vrai miracle !

\textsc{Lui} : Et ça a duré des heures et des heures, tel un torrent de sexe qui a déferlé sur moi sans que je puisse rien faire pour échapper à son emprise ! Aucun homme ne devrait avoir à endurer pareille torture.

\textsc{Moi} : Toutes mes condoléances, vieux. Malheureusement, tu sais comme moi qu’on a du pain sur la planche. Titus a disparu, Bérénice m’est tombée dessus aux premières lueurs du jour, alors que j’ai moi-même passé une très mauvaise nuit et que si ça ne tenait qu’à moi je retournerais me coucher jusqu’à après-demain soir, et dans un moment d’aberration, de pure folie, j’ai juré de tout mettre en œuvre pour le ramener vivant à la maison.

\textsc{Lui} : Tu ne crois pas que tu en fais un peu trop ?

\textsc{Moi}, tirant nerveusement sur mon cigare : Je crois pas, non.

\textsc{Lui} : Et puis c’est nouveau, ça ?

\textsc{Moi} : Quoi ?

\textsc{Lui} : Tu fumes le matin, maintenant ?

\textsc{Moi} : C’est dire dans quel état de nerfs je suis ! Tendu comme un string, mon vieux, prêt à exploser à la moindre pression sur la ficelle. Je comptais sur toi pour me remonter le moral, j’ai bien peur de m’être trompé.

\textsc{Lui} : Désolé, mais j’ai les couilles en purée, comme si un rouleau-compresseur avait passé la nuit à passer et repasser dessus. Je ne sais même pas comment je fais pour tenir encore debout. J’ai eu droit à tout : le flipper, le papillon, la balançoire, le soixante-neuf, le cavalier pendu, la liane ensorcelée, le triangle lumineux, l’étoile mystérieuse, le papillon…

\textsc{Moi} : Tu l’as déjà dit.

\textsc{Lui} : Ah bon ?

\textsc{Moi} : Oui.

\textsc{Lui} : J’ai eu droit à plusieurs espèces de papillons, en fait. Sans oublier les ciseaux, le cadenas, la brouette enchantée, la bête à deux dos, l’aurore boréale, le cheval d’Hector, la cravate de notaire, le bateau ivre, j’en passe et des meilleurs. Tout, je te dis !

\textsc{Moi} : La mouche à merde, aussi ?

\textsc{Lui} : Non, pas la mouche à merde.

\textsc{Moi} : Un vrai cauchemar, en tout cas.

\textsc{Lui} : Tu l’as dit ! Il faut une condition physique de sportif de haut niveau pour tenir le coup.

\textsc{Moi} : Et c’est loin d’être ton cas.

\textsc{Lui} : J’ai passé l’âge.

\textsc{Moi} : Pour info, c’est quoi le triangle lumineux ?

\textsc{Lui} : Sensiblement la même chose que le missionnaire, sauf que la femme est un peu plus active.

\textsc{Moi} : Tu veux mon avis ?

\textsc{Lui}, dans un soupir : Non.

\textsc{Moi} : Tant pis, je te le donne quand même : tu ferais bien de te débarrasser de cette pute avant qu’il soit trop tard. À ce rythme-là, elle aura ta peau dans pas longtemps.

\textsc{Lui} : Je sais.

\textsc{Moi} : C’est du suicide, reconnais-le.

\textsc{Lui} : Je le reconnais, mais c’est pas une raison pour la traiter de pute. Elle est malade, tu comprends ? Malade !

\textsc{Moi} : Je suis sûr qu’elle a tué plein de mecs en les obligeant à baiser comme des dingues jusqu’au bout de la nuit.

\textsc{Lui} : Elle te fais des trucs que t’as même pas idée ! C’est une sorte de génie du sexe, dont la créativité semble inépuisable. Je pense qu’elle a signé un pacte avec le diable !

\textsc{Moi} : C’est une tueuse en série, mon pote, un vampire qui vide les hommes de leur substance jusqu’à la dernière goutte. Tu ne t’en rends pas compte, mais tu viens de prendre un aller simple pour l’enfer. T’en as marre de la vie, ou quoi ?

\textsc{Lui} : Je me pose parfois la question.

\textsc{Moi} : Fous-la dehors et reprends le cours normal de ton existence, ça vaudra mieux. Sauve ce qui peut encore être sauvé. Je suis là, moi. Si tu as besoin de quelque chose, n’hésite pas à me demander.

\textsc{Lui} : Je sais, vieux, je sais.

\textsc{Moi} : Cette femme est un démon, un succube de la pire espèce, une sorcière qui erre nue dans la forêt et s’accouple avec les bêtes sauvages.

\textsc{Lui} : Faut peut-être pas exagérer non plus. C’est une grosse chaudasse, il n’y a aucun doute là-dessus, mais ça reste un être humain fait de chair et de sang, avec un cœur qui bat sous sa poitrine avantageuse.

\textsc{Moi} : Et une chatte qui miaule en permanence, toujours affamée, jamais repue, les crocs affûtés, prêts à se planter dans le moindre morceau de viande qui passe à leur portée. Je me fais du souci pour toi, voilà tout. Je n’ai pas envie qu’on retrouve ton cadavre méconnaissable au fond de l’océan ou dans le lit d’une rivière à sec, à moitié dévoré par les sangsues.

\textsc{Lui} : Honnêtement, je ne vois pas très bien ce que mon cadavre ferait dans le lit d’une rivière à sec.

\textsc{Moi}, appuyant délicatement (je rappelle qu’avec le troupeau d’étalons survitaminés qui piaffaient sous le capot, tout excès d’enthousiasme se traduisait aussitôt par la sensation de décoller aux commandes d’un avion de chasse, ce qui, à moins d’être titulaire d’un brevet de pilote en bonne et due forme, pouvait avoir des conséquences désastreuses pour l’environnement et la sécurité des passagers) sur le champignon après avoir enclenché la première et desserré le frein à main : Moi non plus, mais on ne sait jamais. Je tenais à te mettre en garde, te rappeler une dernière fois que si l’activité physique est bonne pour la santé, largement recommandée par la sphère médicale dans sa plus grande majorité, il n’en reste pas moins que certaines choses sont réservées à des professionnels aguerris et ne devraient en aucun cas être reproduites sans assistance par des amateurs présomptueux. C’est tout ce que j’avais à dire. Maintenant, en voiture Simone et en route pour de nouvelles aventures !

\textsc{Lui}, manifestement vexé : Merci pour ta sollicitude. Mais sache, pour ta gouverne, que je ne manque jamais de m’échauffer avant de passer à l’acte.

\textsc{Moi}, évitant de justesse une joggeuse moulée dans une tenue rose fluo de nature à distraire dangereusement l’attention de tout conducteur un tant soit peu sensible aux attraits de la plastique féminine (et amateur de belles choses, vins fins, rivières de diamants, tonnes d’or, voitures de sport, croisières sur le Nil, couchers de soleil sur le Bosphore, plaisirs simples d’une bonne bouillabaisse partagée entre amis, gambas, naturisme, poules de luxe, bel canto et œuvres d’art en général) : C’est tout à ton honneur. Il n’empêche, et je me permets de te le dire en toute gentillesse, que tu arrives tous les matins sur ton lieu de travail dans un état physique déplorable. J’ajoute que ta santé mentale semble également affectée par les excès de ta vie sexuelle débridée, raison pour laquelle je te demande de lever le pied dans les plus brefs délais. Je crains, si tu ne suis pas mon conseil, que l’érotomanie te guette.

\textsc{Lui} : Et moi, je me demande comment tu peux savoir dans quel état je me trouve en arrivant sur mon lieu de travail puisque tu n’y es pas. Ce n’est pas parce que je suis un peu fatigué ce matin que c’est tous les jours pareil.

\textsc{Moi} : Désolé de te le dire, mais tu n’es plus le même depuis que tu es retombé entre les griffes de cette nymphe des temps modernes, ce suppôt de Lilith.

\textsc{Lui}, perfide : Je ne suis pas certain que tu sois en mesure de me donner des leçons sur le sujet.

\textsc{Moi}, écrasant la pédale de frein pour éviter de griller un feu rouge : Ah oui ? Je peux savoir ce que tu entends par là ?

\textsc{Lui}, avec un petit sourire méchant esquissé au coin des lèvres : J’ai bien vu ton petit manège, hier soir.

\textsc{Moi} : Pardon ?

\textsc{Lui} : Ton petit manège avec la Gardienne de la Nuit. Tu m’excuseras, mais dans le genre succube, elle se pose un peu là !

\textsc{Moi}, faisant ronfler le moteur pour démarrer sur les chapeaux de roues sitôt le feu passé au vert : Sois à tout jamais maudit jusqu’à la trente-sixième génération ! Que les vers te picorent, les asticots te butinent, la moisissure recouvre entièrement ta peau vérolée et les légions de l’enfer viennent danser la salsa sur le tas de terre retournée de ta sépulture anonyme ! Comment toi, mon ami, fidèle parmi les fidèles, oses-tu proférer de pareilles inepties !!!!!!! Jamais, tu m’entends, jamais je n’ai considéré Atiena comme autre chose qu’une créature magique sortie tout droit d’un recueil de contes pour enfants !

\textsc{Lui}, ricanant presque : Oui, enfin, ta façon de la regarder s’apparentait davantage à celle du grand méchant loup en train de reluquer le Petit Chaperon rouge et sa petite motte de beurre frais, ou encore d’Émile Louis en train de jeter un coup d’œil dans le rétro de son bus rempli de gamines handicapées de la DDASS. Tu te pourléchais clairement les babines, n’en déplaise à ta susceptibilité outragée !

\textsc{Moi}, écrasant la pédale d’accélérateur avec une violence telle qu’on s’est retrouvés instantanément de l’autre côté de la route : C’est honteux ! Je ne sais pas ce qui me retient de m’arrêter et te jeter hors de la voiture !

\textsc{Lui} : Ta réaction prouve que j’ai mis dans le mille.

\textsc{Moi} : T’as rien mis dans quoi que ce soit ! À part ta bite dans le cul de cette pute, peut-être !

\textsc{Lui} : Tu vois, tu deviens grossier quand t’es énervé. Et puis c’est pas une pute, je te l’ai déjà dit.

\textsc{Moi} : Je suis pas énervé, je suis offusqué !

\textsc{Lui} : Reconnais que tu la trouves à ton goût.

\textsc{Moi} : Là n’est pas la question.

\textsc{Lui} : Ben si, un peu quand même.

\textsc{Moi} : Je ne mélange pas le travail et les sentiments, moi.

\textsc{Lui} : Moi non plus.

\textsc{Moi} : Tes sentiments interfèrent avec ton travail, ça revient au même.

\textsc{Lui} : C’est pas mes sentiments, c’est ma libido.

\textsc{Moi} : Oui ben ta libido, tu vas la laisser un peu de côté et te concentrer sur le boulot.

\textsc{Lui} : Métro, boulot, libidodo.

\textsc{Moi} : Très drôle !

\textsc{Lui} : Tu ne vas pas faire la gueule pendant tout le trajet, j’espère.

\textsc{Moi} : Je fais pas la gueule, je suis fatigué d’entendre des conneries à longueur de journée. Ça commence tôt le matin et ça ne s’arrête plus jusqu’au milieu de la nuit. J’essaie de t’aider, te sortir un peu de l’ornière dans laquelle tu te trouves, et toi tu ne trouves rien de mieux à faire que d’insinuer des choses à mon sujet, comme si je n’étais pas le preux chevalier blanc épris de justice et d’équité que je prétends être.

\textsc{Lui} : D’équitation, tu veux dire.

\textsc{Moi} : Pardon ?

\textsc{Lui} : Le preux chevalier blanc épris d’équitation.

\textsc{Moi} : Et voilà, je te parle de choses sérieuses, et toi tu continues à me bassiner avec tes blagues à deux balles ! Comment veux-tu que je ne sois pas au bout du rouleau avec des gens comme toi.

\textsc{Lui} : Moi aussi, je suis au bout du rouleau.

\textsc{Moi}, évitant de justesse une collision après avoir refusé une priorité à droite : Pas pour les mêmes raisons que moi.

\textsc{Lui} : Ralentis, tu veux.

\textsc{Moi} : Je ne roule pas vite.

\textsc{Lui}, s’accrochant tant bien que mal à tout ce qui lui tombait sous la main, y compris mon bras ou ma jambe à l’occasion, renforçant au passage les risques de perte de contrôle inhérent à la conduite pour le moins sportive que j’avais choisi d’adopter : Si. Tu as déjà failli provoquer au moins une demi-douzaine d’accidents.

\textsc{Moi} : Je suis un peu tendu, excuse-moi. Et enlève ta main de ma cuisse, tu veux bien.

\textsc{Lui} : Ralentis ou je descends de la voiture.

\textsc{Moi} : Je ne vois pas très bien comment.

\textsc{Lui} : Je vais sauter en marche, au risque de me casser une jambe ou atterrir sur un piéton qui n’a rien demandé à personne et va finir à l’hôpital avec une double fracture du crâne et des lésions irréversibles au niveau de la moelle épinière.

\textsc{Moi} : Je consens à lever le pied si tu retires ce que tu as dit à propos de moi et la Gardienne de la Nuit.

\textsc{Lui} : ATTENTION !!!!!!!

\textsc{Moi} : Quoi encore ?

\textsc{Lui} : Il y a une femme enceinte qui vient de s’engager sur le trottoir !!!!!!

\textsc{Moi} : Elle n’est pas enceinte, elle est juste grosse. Énorme, même.

\textsc{Lui} : Ce n’est une raison pour l’écraser.

J’ai pilé juste à temps, sous le regard horrifié de la grosse bonne femme qui s’est arrêtée net dans sa trajectoire.

Elle s’est mise à gesticuler des bras et des jambes, surtout des bras parce que ses jambes lourdes et cylindriques ne lui permettaient guère de se livrer à des facéties musculaires et autres prouesses articulaires, tout en m’insultant copieusement et me menaçant de représailles judiciaires dignes d’un caïd de la pègre ou un tueur cannibale. Ça a duré un certain temps, pendant lequel je me suis prudemment abstenu de toute déclaration intempestive ou objection inappropriée, après quoi elle a paru soulagée et s’est enfin décidée à reprendre sa route vers la destination qui était la sienne (qui n’était pas celle du cimetière, manifestement, en tout cas pas encore, car compte tenu de sa surcharge pondérale et des difficultés circulatoires afférentes, il y avait gros parier qu’elle ne tarderait pas à s’y retrouver).

C’est le moment choisi par Greg pour détacher sa ceinture et ouvrir la portière dans l’intention manifeste de s’extraire du véhicule.

\textsc{Moi} : Qu’est-ce que tu fais ?

\textsc{Lui} : Je te l’ai dit, je sors de la voiture.

\textsc{Moi} : C’est bon, je vais ralentir.

\textsc{Lui}, qui n’avait aucune envie de parcourir in pede (traduction latine de «~à pied~», laquelle ne fait aucunement référence à quelque orientation sexuelle réelle ou supposée concernant l’individu en question, lequel, par le plus grand des hasards, se trouvait également figurer en bonne place sur la liste~-- certes succincte mais tout de même pas totalement insignifiante~-- de mes meilleurs amis) les deux ou trois bons kilomètres qui nous séparaient encore de la rue des Maléfices, en plein cœur d’un quartier sensible où il ne faisait pas nécessairement bon se balader les mains dans les poches en sifflotant Dixie de Dan Emmett : Tu promets ?

\textsc{Moi} : Je promets. Mais toi, tu promets de ne plus faire d’insinuations gratuites à mon sujet.

\textsc{Lui} : J’ai bien vu comment tu la regardais.

\textsc{Moi} : Tu recommences ?

\textsc{Lui}, une fesse dans la voiture : Non non, j’arrête. Mais reconnais que tu n’aurais rien contre lui faire un brin de causette au bord de l’eau.

\textsc{Moi} : Au bord de l’eau ?

\textsc{Lui}, deux fesses dans la voiture, refermant la portière : Ou ailleurs.

\textsc{Moi} : Non. Au bord de l’eau, c’est bien.

\textsc{Lui} : De l’eau de mer, par exemple.

\textsc{Moi} : Sur la plage, quoi.

\textsc{Lui} : Au coucher du soleil.

\textsc{Moi} : Je ne suis pas fan des clichés, genre compter fleurette à une fille sublime sur la plage au coucher du soleil alors que le temps est d’une douceur élégiaque et que la fille porte une robe blanche translucide avec rien en dessous à part son corps de rêve tapissé d’un épiderme aussi fin et soyeux que la peau d’une pêche de vigne gorgée de nectar sucré. C’est un peu de la merde, tout ça, de la carte postale pour blaireau élevé à la malbouffe et la musique d’ascenseur.

\textsc{Lui}, attachant sa ceinture : Un peu, oui. Mais pas totalement.

\textsc{Moi} : Tu me fatigues, tu sais.

\textsc{Lui} : Je sais.

Je suis reparti, mais comme j’avais oublié de faire deux choses que tout conducteur se doit absolument de faire quand il redémarre après s’être arrêté sur le bas-côté de façon plus ou moins intempestive, à savoir mettre son clignotant et regarder son rétro pour s’assurer que personne n’arrive en trombe au même moment, j’ai évité de justesse une Mini Cooper qui précisément n’avait rien trouvé de mieux à faire que d’arriver en trombe à ce moment-là. Sinon en trombe, en tout cas à une vitesse (pour autant que je puisse en juger, et sans me vanter je suis assez fort pour déterminer avec une marge d’erreur quasi insignifiante la vitesse des véhicules qui se déplacent autour de moi) largement supérieure aux trente kilomètres-heure autorisés. Cela dit, en cas de choc, même si la conductrice en question (il s’agissait d’une femme, je le précise sans la moindre arrière-pensée sexiste concernant la prétendue dangerosité des femmes au volant, lesquelles, si l’on en croit certains individus dont la misogynie avérée et les blagues graveleuses ne plaident guère en faveur de l’intelligence, seraient moins occupées à conduire qu’à bavasser au téléphone ou se repoudrer le nez dans le rétro de courtoisie, quand elles ne seraient pas en train de fouiller dans la boîte à gants ou vagabonder dans leurs pensées au point de perdre tout contact avec la réalité) n’était pas totalement en règle avec la limitation de vitesse en vigueur dans le quartier, j’aurais été tenu pour seul responsable du sinistre, ayant déboîté sans mettre mon clignotant ni m’assurer que la voie était libre. J’aurais bien sûr prétendu le contraire, comme toute ordure qui se respecte, avec un aplomb sans faille, une morgue abjecte digne du politicien le plus corrompu, et Greg se serait fait un devoir d’abonder dans mon sens, en toute mauvaise foi, répugnant de duplicité servile, mais il m’aurait fallu un certain temps (au moins deux ou trois heures) avant de pouvoir à nouveau contempler mon doux visage dans la glace sans être aussitôt pris d’une violente envie de gerber. Comme vous n’êtes pas sans le savoir, du moins je l’espère, l’être humain est ainsi fait qu’il s’accommode assez facilement des bassesses ordinaires de sa moralité douteuse, trouvant sans cesse des arrangements avec sa conscience, laquelle semble ne lui avoir été donnée que pour servir de caution à ses exactions (en plus, tout de même, d’être un rempart naturel à la frénésie destructrice qui l’agite habituellement, à mon sens largement supérieure aux capacités créatrices qui lui ont été dévolues, sans cesse dévoyées par la purulence égotique qui préside à l’essentiel de ses activités).

Surprise, et heureuse de s’en tirer à bon compte, la conductrice s’est éloignée en vociférant tant et plus, jetant des coup d’œil furieux dans son rétroviseur et agitant les bras de façon désordonnée, au risque de perdre le contrôle de son moyen de locomotion, griller le stop qui l’attendait au bout de la rue et emplafonner le bus qui arrivait à vive (très, sans doute même trop, le chauffeur, en plein divorce et affligé de problèmes de santé aussi divers que douloureux, affichant une nervosité assez préjudiciable à la souplesse de sa conduite) allure sur sa droite.

Greg, qui, sans doute par manque de vigilance au moment des faits, ne semblait pas avoir pris la pleine mesure de la catastrophe que nous venions de frôler : Tu crois qu’il lui est arrivé quoi, à Titus ?

\textsc{Moi} : Je pense que la Gardienne de la Nuit l’a entraîné dans son repaire pour l’initier à des activités sexuelles dont nous n’avons même pas idée.

\textsc{Lui} : Dans ce cas, on ferait mieux de le laisser où il est.

\textsc{Moi} : L’ennui, c’est qu’on ne sait pas où il est.

\textsc{Lui} : Au Caribbean Hôtel, non ?

\textsc{Moi} : J’ai tout lieu de le penser, en effet. Je pense qu’elle en a fait son jouet sexuel et le séquestre dans une chambre sans numéro située quelque part dans les bas-fonds de l’établissement, une sorte de nid d’amour, d’alcôve secrète réservée aux proies de cette harpie en jupon.

\textsc{Lui} : Oui, enfin, je crois surtout qu’elle lui a tapé dans l’œil et qu’il a complètement oublié qu’il avait une femme, des gosses et un boulot. Il va réapparaître en fin de matinée, le bec enfariné, et prétendre qu’il ne se souvient de rien.

\textsc{Moi} : Peut-être qu’elle lui a fait boire un philtre d’amour à base de belladone, jusquiame noire, miel vert et vin de Pramnos, telle Circé à Ulysse.

\textsc{Lui} : On appelle ça du GHB de nos jours. C’est moins glamour mais tout aussi efficace.

\textsc{Moi} : Je préfère miel vert et vin de Pramnos. Toujours est-il que Bérénice va être folle de rage si elle apprend qu’il a passé la nuit avec une femme. Elle saura aussi qu’on a essayé d’étouffer l’affaire, et notre belle amitié finira dans la benne à ordures.

\textsc{Lui} : Si on lui sort que Titus a été l’innocente victime d’une créature de rêve dotée de pouvoirs surnaturels, elle nous clouera au pilori sans l’ombre d’une hésitation.

\textsc{Moi} : Oui, à grands coups de marteau. Après quoi elle nous crèvera les yeux et nous arrachera les entrailles pour les donner à bouffer aux corbeaux !

\textsc{Lui} : Elle retrouvera la fille, l’arrosera d’essence et se fera un plaisir de la réduire en cendres.

\textsc{Moi} : Comme les sorcières au Moyen Âge.

\textsc{Lui} : Quant à Titus, elle lui coupera les couilles avec des ciseaux rouillés et l’obligera à les bouffer pour qu’il s’étouffe avec.

\textsc{Moi} : Une mort atroce, et on ne pourra rien faire pour l’empêcher.

\textsc{Lui} : Rien.

\textsc{Moi}, appuyant sur le champignon : C’est pour ça qu’il faut se dépêcher si on veut éviter le pire.

\textsc{Greg} : Je te signale que tu viens de griller un feu rouge.

\textsc{Moi} : Non, je suis passé à l’orange.

\textsc{Greg} : Un orange très très mûr, alors.

\textsc{Moi} : Tu veux sauver Titus, oui ou merde ?

\textsc{Lui} : Bien sûr que je veux sauver Titus. Je donnerais tout pour sauver Titus. Tout sauf ma vie, parce que même si j’adore Titus, j’ai quand même une certaine affection pour ma propre existence. Je dirais même une certaine addiction, contractée au fil des années passées à rouler ma bosse sur cette terre. C’est sans doute de la faiblesse de ma part, mais si quelqu’un devait mourir, j’aimerais autant que ce soit lui. J’aurais du mal à m’en remettre, bien sûr, et son souvenir resterait gravé dans ma mémoire jusqu’à la fin de mes jours, mais je pense que j’aurais assez de force de caractère pour arriver à vivre avec.

\textsc{Moi} : Tu devrais avoir honte de dire des choses pareilles !

\textsc{Lui} : J’ai honte, mais j’ai assez de force de caractère pour arriver à vivre avec.

\textsc{Moi} : Et moi ?

\textsc{Lui}, Quoi, toi ?

\textsc{Moi} : Si ma vie était en jeu ?

\textsc{Lui} : Tu veux savoir si je te laisserais crever comme un chien ?

\textsc{Moi} : Je ne me fais aucune illusion.

\textsc{Lui} : Toi c’est pas pareil, t’es comme un frère pour moi. Je ne dis pas que j’irais jusqu’à donner ma vie pour toi, mais je serais prêt à sacrifier quelques morceaux de mon anatomie.

\textsc{Moi} : Non ?

\textsc{Lui}, posant une main sur ma cuisse : Si.

\textsc{Moi} : Okay, je te remercie. Tu peux enlever ta main, s’il te plaît ?

\textsc{Lui} : Ah oui, pardon.

\textsc{Moi} : On arrive bientôt. Je vais tâcher de trouver une place stratégique pour me garer, histoire qu’on puisse se barrer en vitesse si les choses tournent mal.

Personne n’ayant eu l’idée saugrenue de le déplacer pendant la nuit, le Caribbean Hôtel se trouvait exactement au même endroit que la veille.


Vous rigolez, mais ça s’est déjà vu que des gens déplacent des choses pendant la nuit pour vous faire croire que vous perdez la boule. Dans les vieux films policiers, par exemple, il n’est pas rare qu’une blonde pulpeuse avec des yeux en velours bleu, des dents de vampire, des lèvres de sangsue, une taille de guêpe et deux obus à la place des seins, engage, pour une raison X ou Y (le plus souvent parce que la pauvre chérie est victime d’un odieux chantage qui risque de faire capoter son mariage avec son mari richissime et vieillissant, ce qui aurait pour conséquence désastreuse de la priver de toute ressource et l’obliger à retourner racoler le micheton dans les bars des hôtels de luxe), un détective privé alcoolique au charme ravageur qui ne se départit jamais de son humour caustique et sa décontraction à toute épreuve. Un beau soir, la créature arrive en courant dans le bureau de ce dernier, toujours impeccablement coiffée en dépit d’une course effrénée sous une pluie battante, pour lui annoncer qu’elle vient d’être témoin d’un meurtre horrible et qu’elle préfère venir le trouver lui plutôt que la police car elle ne tient aucunement à être mêlée officiellement à cette sombre histoire. Après quelques hésitations, durant lesquelles le privé pose avec insistance son regard acéré sur les courbes vertigineuses de la fille, il consent à enfiler son imper (faute de mieux), allumer une clope, poser son chapeau sur sa tête, et la suivre sur les lieux du drame afin de vérifier l’authenticité de ses dires. Tous deux prennent place dans sa grosse voiture cabossée, traversent la ville en sens inverse, toujours sous la pluie et sans desserrer les dents (la tension est palpable, l’atmosphère en prise directe sur le triphasé, on sent bien qu’il ne faudrait pas grand-chose pour qu’ils se jettent l’un sur l’autre et que le détective lui roule une de ces putains de pelles d’anthologie dont il a le secret sans même enlever sa clope de sa bouche et tout en gardant un œil avisé sur la route), et quand ils arrivent sur place, le cadavre a disparu sans laisser la moindre trace. Le sang a été nettoyé et les meubles et objets renversés remis en place comme si de rien n’était. Du coup la fille passe pour une affabulatrice bonne pour l’hôpital psychiatrique, se lance dans des explications à n’en plus finir, toutes aussi dépourvues de sens les unes que les autres, et finalement, histoire de ne pas s’être déplacé pour rien, le privé lui roule une pelle d’anthologie avec la clope au bec et le chapeau sur la tête, avant, grand seigneur, de la ramener discrètement chez son mari cocu (parce qu’elle le trompe, bien entendu, et plutôt deux fois qu’une, la salope, avec tout ce qui lui tombe sous la chatte, du jardinier à tablette de chocolat au chauffeur ancien boxeur en passant par le puceau de service et le comptable érotomane).

Mais ce qui arrive assez fréquemment avec des cadavres ou des objets de valeur dans des appartements haussmanniens du huitième arrondissement de Paris, des villas en bord de mer, des cabanes au fond des bois ou même de simples chambres de bonne situées au dernier étage d’immeubles insalubres, se produit nettement plus rarement avec des hôtels entiers, particuliers ou non, car même si vous disposez de moyens exceptionnels pour déplacer un tel établissement dans son entièreté, il vous faudra impérativement opérer à une heure très avancée de la nuit. Et même de nuit, si l’hôtel, comme c’est généralement le cas des hôtels, particuliers ou non, est situé dans ce qu’on appelle le centre-ville ou sa périphérie immédiate, je doute fort que vous arriviez à le déplacer par la voie des airs sans attirer l’attention de personne. Ne serait-ce que celle des forces de l’ordre, par exemple, qui effectuent des rondes régulières dans les centres-villes, ou encore celle des fêtards attardés qui, même déchirés au point d’éprouver les plus vives difficultés à mettre un pied devant l’autre, seraient les premiers surpris de voir un tel édifice passer au-dessus de leurs têtes échevelées.

J’ai fait le tour du pâté de maisons un certain nombre de fois, que je n’ai pas jugé nécessaire de compter, avant que quelqu’un se décide enfin à monter dans sa putain de bagnole, glisser la clef de contact dans la fente prévue à cet effet, démarrer le moteur et se déplacer d’un lieu à un autre pour des raisons qui lui appartenaient et ne m’intéressaient en aucune façon.

Se déplacer a toujours fait partie des préoccupations majeures de l’espèce humaine, et elle le fait maintenant de plus en plus vite pour occuper le maximum d’espace en un minimum de temps, rêvant d’ubiquité, téléportation et autres colonies interstellaires pour assouvir ses coupables penchants. La question qui se pose est la suivante : que faire quand on est une espèce éminemment invasive qui ne dispose d’aucun autre prédateur qu’elle-même ? L’autorégulation est-elle une option viable ? Peut-on légitimement compter sur la tempérance d’une espèce dont la voracité légendaire ne laisse aucune place à l’expectative ou l’introspection, sinon pour une poignée de marginaux au crâne rasé qui vivent reclus dans des ruines vieilles de plusieurs siècles (pour vivre heureux vivons caché, c’est bien cul nu) ? Arrêtons de manger de la viande, les animaux sont nos amis pour la vie et en plus c’est mauvais pour la santé, s’époumone la jeunesse 2.0 qui se teint les cheveux en rose fluo, flotte dans des fringues XXL et revendique le droit de pouvoir jouer au foot quand on est une fille et à la poupée quand on est un garçon, ainsi que la liberté de choisir tel ou tel sexe si l’on estime, notamment à l’adolescence quand on est en plein quête de soi et recherche de la vérité existentielle, que le sien n’est pas conforme. On se demande bien au nom de quoi une fille ne pourrait pas venir au monde avec des attributs masculins, et inversement un garçon avec ce qui fait habituellement les charmes de la féminité. La nature fait des erreurs, c’est à nous qu’il revient de les corriger. Elle en fait même un peu trop, raison pour laquelle il serait peut-être temps, alors que la cinquième génération de téléphone sans fil vient de voir le jour et que nous n’avons aucunement l’intention de nous arrêter en si bon chemin, de songer à la sortir du jeu une bonne fois pour toutes. Après des centaines de millions d’années de règne sans partage, l’heure de la retraite a sonné. Je bande donc je suis, peut-être, mais pas forcément un garçon. Et oui, c’est vrai, j’ai une jolie petite paire de loches bien rebondies qui prend le frais dans mon Lise Charmel à balconnet, mais ce n’est pas pour autant que je suis une fille. Merde, il est temps de mettre un terme à cette vision étriquée du genre humain ! La nature est vieille, dépassée, ses injonctions n’ont plus aucun sens dans le contexte actuel, elle a perdu le contact avec les nouvelles générations, l’élève a dépassé le maître, Dieu est mort, vive Dieu ! La tyrannie des sexes, qui veut qu’on soit maçon ou garagiste quand on est un garçon et danseuse nue ou secrétaire de direction quand on est une fille, a fait son temps. Quant à cette pratique d’un autre âge que j’évoquais précédemment, qui consiste à élever des animaux dans des conditions déplorables et les assassiner pour se repaître de leur chair en toute impunité, elle encourage la bestialité qui est en nous et nous rabaisse au rang des plus vils représentants d’un passé à jamais révolu. Devenons de paisibles herbivores, abandonnons tout esprit de compétitivité, toute trace d’ego malsain, d’individualisme forcené, éclairons-nous à la bougie et arrêtons de boire du sang comme des vampires poussiéreux. Nous sommes des milliards sur terre, tous animés des meilleures intentions, et je ne doute pas qu’avec un peu de bonne volonté il soit possible de faire en sorte que tout se passe pour le mieux dans le meilleur des mondes. Nous venons en paix, amis Terriens, et apportons dans nos valises les nouvelles technologies de la félicité éternelle. Il y a trop, bien trop longtemps que vous vous entretuez bêtement (pour des raisons d’une telle frivolité que nous peinons encore à comprendre la nature réelle de vos motivations), il vous faut maintenant apprendre à vivre en bonne intelligence. Enculez-vous les uns les autres, a dit en substance notre Sauveur bien-aimé avant de finir cloué sur une croix entre deux malfrats et de repartir au Ciel la queue entre les jambes, et ne faites pas aux truies ce que vous n’aimeriez point qu’on vous fasse, bande de putois malodorants ! Vous n’aimeriez pas qu’on vous mange, n’est-ce pas, même si certains d’entre vous l’ont fait pendant un certain temps avant de prendre conscience que cette pratique n’était pas sans doute la mieux adaptée à l’établissement d’une paix durable entre les peuples. Alors laissez ces pauvres bêtes tranquilles, laissez-les se gaver de glands, gambader joyeusement dans la luzerne, et chassez de votre tête cette obsession de vouloir à tout prix les transformer en jambon, saucisson, pâté de tête et andouillette.

Mais je m’égare, une fois de plus (et je ne vous cache pas que parler de jambon, saucisson, pâté de tête et andouillette, m’a donné une solide envie de me restaurer sans plus attendre).

Pour en revenir à ce que je disais précédemment, avant de m’embarquer dans cette diatribe futuriste qui, je l’imagine, risque de faire grincer quelques dents aussi bien dans les clapets réactionnaires de l’extrême-droite décomplexée que la bouche en cœur des bobos de la Rive Droite, un de nos plus gros problèmes sur terre est qu’il y a tellement de voitures partout qu’il n’y a plus moyen de se garer nulle part. S’il y a encore moyen de trouver une place à la cambrousse, entre deux troupeaux de vaches garés en double file, la chose est devenue quasiment impossible dans les centres-villes saturés du 21e siècle.

C’est ainsi que nous étions des dizaines, que dis-je des dizaines, des centaines à tourner autour du Caribbean Hôtel, dont certains qui tournaient jour et nuit depuis des semaines et avaient fini par développer un tel degré d’exaspération qu’il aurait suffi de la plus minuscule étincelle pour mettre le feu aux poudres et déclencher une guerre civile comme on n’en avait pas connu depuis la Fronde et le régime de Vichy. Aussi opportuniste que la hyène qui profite d’un instant d’inattention du guépard pour lui subtiliser la gazelle qu’il vient de passer des heures à traquer dans la brousse, j’ai profité de ce qu’une personne handicapée (il s’agissait en fait d’un individu parvenu à un tel degré d’obésité que ses jambes disparaissaient presque entièrement sous une cascade de plis graisseux) remonte péniblement dans son SUV pour me substituer à elle en toute illégalité. Particulièrement bien équipé, je disposais en effet, en plus d’une affichette sur laquelle était inscrit en lettres majuscules «~INTERVENTION POLICE~» que je plaçais bien en évidence sur le tableau de bord, d’un macaron HANDICAPÉ qui me permettait d’outrepasser régulièrement mes droits, y compris, je l’avoue humblement, quand je n’étais pas spécialement en intervention. Après tout, le maintien de l’ordre est une chose essentielle si on veut espérer que la société ait une chance de survivre aux dissensions internes et autres dérèglements intestinaux responsables des flatulences qu’elle émet en permanence. J’ajoute que les handicapés, physiques et mentaux, du reste, avec tout le respect que j’ai pour eux (j’ai moi-même d’innombrables amis handicapés avec lesquels je passe d’excellents moments, à tel point que c’est tout juste si je vois la différence avec mes amis valides, pour la plupart à peine plus intéressants, même si c’est tout de même plus facile de s’adonner aux joies de l’escalade ou faire un footing en forêt de Rambouillet avec une personne valide qu’un cul-de-jatte), doivent prendre conscience que leurs problèmes personnels, aussi cruels et injustes soient-ils, je suis le premier à le reconnaître, ne doivent pas pour autant entraver l’action du bras séculier de la justice. Je suis tout à fait pour qu’on leur réserve des places ici et là, mette tout en œuvre pour leur faciliter la tâche au maximum, ne les priver d’aucunes des réjouissances auxquelles ont droit les gens normaux, mais je ne tiens pas à les avoir dans les pattes à tout bout de champ. Imaginez, par exemple, que vous êtes en train de courser un pickpocket ou un vendeur à la sauvette dans les rues de la cité, chose qui nous arrive malheureusement plus souvent qu’à notre tour à nous autres gens d’armes et de police, et qu’un handicapé vous barre la route en se traînant comme une limace en plein milieu du trottoir. Vous faites quoi ? Vous l’évitez soigneusement, au risque de laisser filer le contrevenant, ou le traitez sans discrimination, comme n’importe quel citoyen, autrement dit lui foncez dessus et le percutez violemment sans vous soucier un seul instant des conséquences ? Il y a des moments, dans l’existence, où il faut savoir faire des choix qui ne sont pas toujours agréables et faciles à assumer, et si un jour on doit me couper les deux bras ou les deux jambes (ou les deux, enfin les quatre, plus la bite et tout ce qui dépasse), eh bien j’essaierai de me faire aussi discret que possible pour ne pas emmerder le monde, tel Raymond Burr dans L’Homme de fer (pour le cas où je devrais continuer à enquêter cloué dans un fauteuil roulant, au risque que tous les criminels se foutent de ma gueule et rêvent de me pousser dans les escaliers).

Bon, blague à part (parce que je blaguais, bien sûr, vous n’avez tout de même pas cru un seul instant que j’étais à ce point dépourvu de sens civique que je n’hésitais pas à me garer sur les places réservées aux handicapés, des gens qui n’ont pas eu de chance dans la vie et méritent bien un minimum de compassion de la part de celles et ceux qui jouissent de la totalité de leurs facultés), je n’ai pas eu à user ou abuser de mes prérogatives pour réussir à me garer. Car en effet, non loin de l’entrée principale du Caribbean, un espace entre deux citadines d’entrée de gamme attendait qu’on vienne l’occuper, chose que je me suis empressé de faire séance tenante, exécutant avec maestria un créneau à montrer en boucle dans toutes les écoles de conduite. Ce que je veux dire par là, c’est que s’il existait un Nobel du créneau, récompense que malheureusement aucune autorité compétente n’a encore songé sérieusement à attribuer, les gens étant tous des abrutis qui ne voient pas plus loin que le bout de leur nez, j’aurais pu empocher les onze millions de couronnes suédoises sans la moindre difficulté, soit environ un million d’euros qui m’auraient permis de mettre un peu de beurre dans les épinards desséchés de mon ordinaire. Mais s’il arrivait, par extraordinaire, que le comité Nobel norvégien soit assez con pour accorder à Donald Trump le prix tant convoité (mais c’est avec un immense soulagement que j’apprends à l’instant même, preuve qu’il y a encore un vague semblant de justice en ce bas monde, que le prix Nobel de la Paix vient d’être décerné à Maria Corina Machado, femme politique qui tente courageusement, au péril de sa liberté et sans doute sa vie, de faire obstacle aux sordides manigances de l’ignoble Nicolas Maduro, l’actuel président du Venezuela, prêt à toutes les exactions pour se maintenir au pouvoir et continuer à s’enrichir honteusement sur le dos de ses concitoyens), s’il arrivait, disais-je, qu’une telle aberration (à peu près aussi absurde, si vous voulez mon avis, que l’idée de voir des requins tomber du ciel ou des parasites extraterrestres mal intentionnés atterrir dans une forêt de Caroline du Sud) se produise, je foncerais aussitôt à Oslo (en voiture bien sûr, au volant de ma fidèle et puissante Kangoo, et il va de soi que j’effectuerais un créneau parfait devant le numéro 51 de la rue Henrik Ibsen) afin d’exiger que le Nobel du Créneau soit créé immédiatement pour m’être remis dans la foulée.

C’est avec un ensemble parfait, au millimètre près, comme si on avait répété la scène pendant des mois sous la direction d’un des plus grands chorégraphes de tous les temps (même si je ne suis pas certain que Balanchine, Cunningham, Béjart ou Pina Bausch auraient accepté de faire la choré de deux types insignifiants et mal réveillés en train de sortir d’une Kangoo à peu près aussi glamour qu’un crapaud barbotant dans une flaque d’eau croupie), que Greg et votre serviteur, tels deux Titans nés des amours coupables d’Ouranos et Gaïa (je rappelle quand même que Gaïa est plus ou moins la mère d’Ouranos, ce qui fait de cette union l’inceste fondateur de la mythologie grecque, inceste d’où sont issus, outre les Titans, une série de monstres comme les Cyclopes et les Hécatonchires, et que c’est encore du ventre de Rhéa, fécondée par son propre frère Cronos, que sortiront les dieux de l’Olympe, somme toute une belle bande de dégénérés qui ne font pas franchement honneur à la profession), sommes sortis du véhicule sus-mentionné. Au cinéma, la scène aurait été tournée au ralenti, en slow motion, comme disent nos amis américains qui ont toujours une longueur d’avance sur tout (ou de retard, suivant l’endroit où on se trouve), et servie accompagnée d’un morceau de choix comme L’entrée des dieux au Walhalla, scène finale de L’Or du Rhin de Richard Wagner, lui-même connu pour être une assez belle ordure prête à tout pour assouvir ses plus bas instincts, à commencer coucher avec la fille de Franz Liszt, Cosima, âgée de vingt-quatre ans de moins que lui, qui est aussi, accessoirement, la femme de son meilleur ami, le pianiste, compositeur (assez peu doué il est vrai) et chef d’orchestre Hans von Bülow. Comme quoi les plus sombres histoires d’inceste et de trahison n’empêche pas le génie de s’exprimer, pour le meilleur, le pire, le pire du meilleur et le meilleur du pire.

Greg (qui s’était mis au tango depuis quelques semaines, histoire de rencontrer des femmes superbes, bien sûr, au regard de braise, à la taille de guêpe et la croupe incendiaire, mais aussi de faire un peu d’exercice pendant ses rares heures de loisir, ce qui ne serait pas du luxe car il avait pris pas mal de bide ces derniers temps), s’était offert, en plus d’une paire de chaussures de danse en cuir souple avec talons de 22 mm à absorption de chocs et semelles antidérapantes, un Bersa Thunder 380 CC, spécialement conçu pour assurer une protection discrète en toute circonstance sans pour autant renoncer à une redoutable puissance de feu. Connu aussi aussi sous le nom de 9 mm court, avec une douille de 17,3 mm (l’une des nombreuses munitions créées dans les années 1910 par le regretté John Moses Browning, un petit gars de l’Utah, membre de l’Eglise de Jésus-Christ des saints des derniers jours, qui avait un foutu sens des affaires et n’ignorait pas que l’homme est un loup pour l’homme, et que malgré tout l’amour que notre Seigneur exige qu’on lui porte il vaut quand même mieux éviter de tourner le dos à son prochain), le calibre 380 ACP vous permettra d’exploser en toute fraternité le crâne d’un individu qui aurait la mauvaise idée de s’en prendre à votre intégrité physique ou au contenu de votre portefeuille, que je vous souhaite aussi dodu et rembourré que les fesses de Kim Kardashian, Nicki Minaj, Jennifer Lopez et Kylie Jenner réunies. J’ajoute que sa poignée ergonomique en polymère texturé viendra se réfugier dans le creux de votre main tel un chaton craintif et ronronnant. Quand on sait à quel point les narcotrafiquants sont des gens qui aiment consommer local, et ont une telle foi dans la qualité de leurs produits qu’ils n’hésitent pas à les exporter aux quatre coins du monde, on comprend mieux que les membres des cartels sud-américains soient de fervents adeptes de la marque Bersa, originaire de Buenos Aires, de même que nos amis allemands ne jurent que par le Walther PPK, et italiens par le 80X Cheetah de Beretta. Dieu que les gens peuvent être chauvins !

Et chauvin, je l’étais sans doute aussi (même si peut-être pas autant que nos amis allemands et italiens, lesquels n’ont d’ailleurs pas toujours été nos amis, il faut bien le dire, voir NBPPBP, Note en Bas de Page Pas en Bas de Page), puisque c’était entre les mains expertes d’une manufacture d’armes et cycles française, sise en la bonne ville de Saint-Etienne et jadis réputée pour la robustesse et la fiabilité de ses fusils de chasse, que j’avais choisi de remettre ma vie.

\textsc{NBPPBP} : Je pense surtout à nos amis allemands qui sont allés jusqu’à s’installer chez nous sans nous demander notre avis, dans le Nord d’abord, puis l’ensemble du pays, hormis la Corse et les départements du Sud-Est réservés à Mussolini. Fils de militant socialiste révolutionnaire, puis ancien instituteur devenu dictateur sous le nom de Guide Suprême de la République Sociale Italienne (Duce en italien), ce dernier sera désavoué par le Grand Conseil (avec la bénédiction du roi qui s’irrite de son omniprésence), arrêté et emprisonné dans les Abruzzes, au Campo Imperatore. Par chance, Hitler a vent de ses emmerdements, et comme il n’est pas du genre à laisser tomber ses amis dans la débine, il appelle son vieux pote Otto Skorzeny, SS-Hauptsturmführer farouchement anticommuniste de son état, et envoie un commando des forces spéciales du tristement célèbre Sonder Lehrgang Oranienburg pour libérer Musso (ou Mumu, comme l’appellent ses rares amis ploutocrates, tous sexuellement déviants, Mumu lui-même ne cachant pas ou peu une certaine appétence pour les très jeunes filles).

De retour aux affaires, Mumu s’autoproclame Président Directeur Général en Chef de la République de Salò (ou des Salauds, suivant l’orthographe retenue), régime vaguement cryptocommuniste sous tutelle nazie qui est loin de faire l’unanimité dans le pays.

En avril 45, sous la pression des Alliés qui entendent bien récupérer les vins de Rinaldi, Conterno, Mascarello et Giacosa, la mozarella di bufala, le moliterno truffé, le jambon de Parme, la mortadelle et le guanciale, sans parler des trésors artistiques proprement dit présents dans tous les coins et recoins de cette région bénie des dieux, il décide de fuir en emportant deux mallettes dont le mystérieux contenu reste aujourd’hui encore marqué d’un point d’interrogation (peut-être de la truffe blanche d’Alba ou du Storico Ribelle de 10 ans d’âge). Arrêté par la Résistance italienne, condamné à mort par le Comité de libération nationale, il est exécuté, en même temps que sa maîtresse et âme damnée Clara Petacci, par des partisans dans une ferme des environs de Dongo, non loin du lac de Côme, endroit idyllique s’il en est mais funeste pour les amants maudits du fascisme, dont les corps sans vie finiront tristement pendus par les pieds sur la piazza Loreto de Milan (avec ceux de Nicola Bombacci, Alessandro Pavolini et Achile Starace).

La mallette n’a jamais été retrouvée, mais on se demande bien ce que le Duce trimballait avec tant d’acharnement dans son ultime cavale. La célébrité de celui ou celle qui retrouvera ce trésor, sans doute escamoté par les partisans (à moins que Musso, comprenant que tout était fini, ne l’ait enfoui quelque part avant de sombrer dans les abîmes de l’Histoire), est assurée.

Pendant des années, la Manufacture d’Armes de Saint-Etienne a été une référence dans le monde de la Mort, tant sur le plan humain qu’animal. Ses fusils de chasse, par exemple (Falcor, Robust, Simplex et Idéal, des noms qui font rêver et traduisent assez bien une certaine idée de la destruction), jouissaient d’une excellente réputation de fiabilité et solidité, et ses armes de guerre faisaient le bonheur des courageux jeunes gens qui avaient mis leur vie au service de la Nation. En échange de leurs bons et loyaux services, la Patrie reconnaissante mettait à leur disposition du matériel de qualité pour exterminer son prochain dans les meilleures conditions, avec le maximum de confort et d’efficacité. En plus de ses modèles propres, issus du génie créatif de ses ingénieurs (des artistes de la mort, virtuoses de l’homicide, disciples de Zénon d’Élée, Anaxagore et Mélissos, pour qui le tir à balle réelle était un art que l’on se devait de porter à son plus haut degré de finitude dénazifiée, l’expression la plus ontologiquement pure de la vérité au sens présocratique et phénoménologique du terme), Manufrance (nom commercial de la Manufacture d’armes et cycles de Saint-Etienne, pionnière de la vente par correspondance) restera à jamais dans les mémoires pour avoir assemblé des armes aussi légendaires que le Beretta M12, le G3 de Heckler \& Koch et le lance-roquettes antichar LRAC F1 de la Luchaire Défense SA, société anonyme au capital de 4 millions de francs. Mais pour moi, outre le Chassepot de 1866 et le canon de 75 de 1897, véritables fleurons d’une approche plus moderne, progressiste, pour ne pas dire humaniste de la guerre, le chef-d’œuvre absolu de la Manufacture d’Armes de Saint-Etienne reste incontestablement celui que j’appelle affectueusement Manu, le petit Manu, autrement dit le pistolet automatique Le Français dans sa version de poche, calibre 6.35, jouet que le commissaire Ottavioli gardera précieusement sur lui jusqu’à la fin des années 70 (avant de passer à des modèles plus consistants pour s’adapter à la puissance de feu croissante des nouvelles générations de malfrats). Je sais que les jeunes d’aujourd’hui pensent que toutes les choses qui se sont passées avant le jour de leur naissance sont les reliques nauséabondes d’une époque révolue, mais je tenais à leur faire savoir, quitte à passer pour un vieux con dépravé, un vieux débris obsolète, qui était vraiment Pierre Ottavioli. Figure du 36, c’était d’abord un homme d’honneur et un de ces flics à l’ancienne comme on n’en fait plus, à l’image d’un Charles Pellegrini, un Roger Marion, un Robert Broussard ou encore un Marcel Guillaume (le modèle du Maigret de Simenon). En ce temps là, somme toute pas si lointain, flics et voyous se donnaient la réplique dans une ambiance, sinon d’admiration réciproque et d’estime à proprement parler, au moins de respect mutuel, ce qui poussait chacun à se surpasser au profit d’une cause commune qui dépassait largement le cadre exigu de la simple individualité : celle du grand banditisme, grandeur qui ne définissait pas la violence extrême et aveugle qui s’y exerçait, mais la qualité supérieure (et parfois volontiers franchouillarde, un tantinet nationaliste, je vous le concède, au sens pur porc du terme, les expressions «~à l’ancienne~» et «~qualité supérieure~» désignant aussi bien des produits du terroir tels que le saucisson sec ou la blanquette de veau) de ses intervenants.

Voilà comment Greg Lussier, Bersa Thunder, le petit Manu et moi-même nous sommes retrouvés devant la porte du Caribbean Hôtel, une entrée majestueuse trônant au centre d’une de ces majestueuses façades à colonnades qu’un Louis Le Vau, un Robert de Cotte, un Charles Le Brun, un François Mansart, un Claude Perrault ou encore un Jacques Gondouin de Folleville, pourquoi pas (le «~bon Gondouin~», comme l’appelait Louis XV, d’abord jardinier du château de Choisy avant de se lancer dans l’architecture sous l’égide du Roi qui semblait le tenir en haute estime), auraient été fiers d’inscrire au catalogue de leurs réalisations les plus significatives, même s’ils n’avaient encore qu’une très vague idée de ce que serait un jour le style colonial dans toute sa rigidité phallique et son paternalisme débonnaire. Force, hélas, et croyez bien que je suis le premier à le déplorer, était de constater que l’édifice avait perdu une bonne partie de sa superbe. La façade, sans un ravalement d’urgence, risquait de s’effondrer à tout moment, ensevelissant au passage d’innocentes victimes dont le seul tort aurait été de se trouver là au mauvais moment. S’ensuivrait alors, avec une implacable dynamique, le cours normal des choses, en partant du postulat que la catastrophe se produirait de nuit plutôt que de jour, l’obscurité étant un excellent adjuvant de l’angoisse, la terreur et la dramaturgie : cacophonie des sirènes hurlant à tout va, féérie lumineuse des gyrophares multicolores, présence massive des forces de l’ordre pour sécuriser le périmètre et permettre aux personnels de santé de s’acquitter au mieux de leur mission, arrivée tonitruante des vautours surexcités de l’info en continu, forêt de micros tendus aux rares témoins oculaires et survivants de ce qui pourrait bien rester dans les anus comme une des pires tragédies de l’histoire de l’humanité (après le tremblement de terre de Shaanxi en 1556, l’explosion de La Valette en 1634, le naufrage du Scipion dans la baie de Samana en 1782, la catastrophe du Victoria Hall en 1883, le vol 123 de la Japan Airlines en 1985, l’effondrement du toit de la patinoire de Bad Reichenhall en 2006 et la terrible bousculade d’Antananarivo avant un concert de Paul Bert Rahasimanana~-- alias Rossy~-- en 2019), corps sans vie évacués sur des civières, membres épars récupérés ici et là et aussitôt congelés dans le vain espoir d’être un jour restitués à leurs propriétaires, badauds en pantoufles et robes de chambre prêts à vendre père et mère pour apercevoir ne serait-ce qu’une goutte de sang ou un morceau de cervelle sur la chaussée, téléphones portables en surchauffe et vidéos de l’événement faisant le tour de la planète en une fraction de seconde. Tristesse envahissante du monde, nullité cosmique et décomplexée de l’espèce humaine en voie de décomposition, sidération intersidérale, folle envie d’appeler Dieu en PCV (laid moins de vain temps ne pleuvent pas qu’au naître, homme aime titre qu’un nombre inca le cul glabre de choses dont ils mourirons singe ah mais avoir an tendu pas relais, l’orthographe, par exemple, sachant que nous aussi casserons nos pipes sans jamais avoir entendu parler d’un nombre d’autant plus incalculable de choses qu’elles ne cesseront de s’accumuler pendant les siècles et les siècles qui suivront notre mort, siècles dont j’ai malheureusement toutes les raisons de penser qu’ils ne seront peut-être pas aussi nombreux que prévu à suivre le cortège du temps) pour le questionner une nouvelle fois sur la nature exacte de ses motivations, tenter une dernière fois de comprendre par quelle aberration il s’est mis en tête de créer, à son image paraît-il (Genèse 1:26-28 LSG), ce qui n’est soit dit en passant pas très flatteur pour lui, une espèce aussi débile et dérisoire que la nôtre. Si le but était de nous faire passer par toutes les étapes de la médiocrité pour arriver enfin à quelque chose de présentable, alors on peut dire que nous n’en sommes encore qu’au tout début de notre évolution. Par contre, si le but était d’expier à travers nous quelque faute originelle qu’il aurait lui-même commise, alors on peut dire que l’objectif est entièrement atteint et qu’il serait peut-être temps de songer à mettre un terme à nos souffrances, chose que nous sommes par ailleurs tout à fait en mesure de réaliser par nos propres moyens. J’ai toujours dit, et je le maintiens avec la plus extrême vigueur, que tout ce que nous faisons et accomplissons avec tant de fierté n’est finalement qu’une maigre resucée à visage humain des œuvres de la nature, dont nous ne faisons que reproduire les faits et gestes en les adaptant à nos besoins, lesquels sont d’autant plus importants que s’impose chaque jour davantage l’évidence de notre inaptitude à vivre en harmonie avec le monde. Chacune de nos actions est soumise à l’utilisation d’une prothèse correspondante, une voiture pour rouler, un bateau pour naviguer, une fusée pour aller dans l’espace, des couverts pour manger, des verres pour boire, des armes pour tuer, des écrans pour voir ce qui se passe autour de nous, des chambres pour dormir, des salles de bain pour se laver, des sex-toys pour forniquer (même si c’est haram de s’en servir et si le guide suprême iranien Ali Khamenei s’est fendu d’une fatwa à leur encontre), des tables pour poser des trucs dessus, des chaises pour s’assoir, des chaussures pour marcher, des claviers pour écrire, des mots pour le dire, etc, etc, etc. Nous sommes tous des infirmes de naissance qu’on équipe de prothèses sans cesse plus sophistiquées pour les transformer en sportifs de haut niveau. Les performances sont remarquables, si on veut, mais le prix à payer bien trop élevé pour la majeure partie d’entre nous. En nous dotant des moyens intellectuels nécessaires pour surpasser notre condition, la nature s’est tiré une balle dans le pied. Sa légitimité a pris du plomb dans l’aile, elle a vu ses prérogatives contestées et son champ d’action se transformer lentement en peau de chagrin. Elle a pris conscience que si l’Homme en avait un jour les moyens, il n’hésiterait pas à la détruire, comme il n’hésite pas à détruire tout ce qui fait obstacle à ses ambitions démesurées. Lui, dans le même temps, s’est rendu compte que la nature ne lui était plus d’aucune utilité. Non seulement elle n’avait cessé de lui mettre des bâtons dans les roues, l’obligeant à surmonter des épreuves qui menaçaient jusqu’à la survie de son espèce, mais les rares satisfactions qu’elle lui procurait en échange ne faisaient que renforcer sa servitude et exacerber sa frustration. À défaut de vivre caché, sur une île déserte ou reclus entre les hauts murs de quelque monastère situé au sommet d’une montagne inaccessible, il lui fallait, pour vivre heureux et donner la pleine mesure de ses capacités, construire un monde à son image, entièrement artificiel, dont il pourrait contrôler le fonctionnement jusque dans ses moindres rouages. Quant à cette belle intelligence, cette conscience exceptionnelle qui lui servait soi-disant à accomplir des miracles, elle-même devait renoncer au naturel pour se lancer à corps perdu dans les délices de l’artifice. La machine, à terme (même si elle le fait déjà dans de nombreuses situations), est vouée à remplacer l’être humain, bien trop fragile et approximatif dans tous les secteurs d’activité. Car enfin, si le rêve de l’Homme a toujours été de vaincre la mort, il est évident qu’il lui faut d’abord vaincre la vie pour y parvenir. Ce n’est que lorsqu’il aura percé les mystères de l’obsolescence programmée qu’il pourra enfin s’affranchir des limites du temps. Il pourra alors vivre le cœur léger, même si ce cœur n’est qu’une machine, et envisager l’avenir sans cette épée de Damoclès de la Mort suspendue en permanence au-dessus de la tête. Comment, je vous le demande, se consacrer sereinement à une tâche si vous savez que tout peut s’arrêter d’un instant à l’autre ? La nature, en nous condamnant à vivre dans cette incertitude, dans l’urgence d’une fin aussi certaine qu’imprévisible, a fait acte de cruauté absolue. Car même si elle constitue parfois une source de motivation, cette urgence est d’abord et avant tout un instrument de torture diabolique. C’est ainsi que nous mettons au monde des enfants, que nous sommes censés aimer plus que tout au monde, en sachant pertinemment qu’il nous faudra les abandonner à leur triste sort. Il ne faut pas s’étonner que le principe même de la reproduction, véritable rouleau-compresseur de l’existence, machine à broyer du vivant qui nourrit les enfants du sang de leurs parents, soit aujourd’hui dans le collimateur des nouvelles générations. Là encore, cette fatalité qui nous condamne à engendrer notre propre succession, si elle peut sembler flatteuse à première vue, revient en réalité à signer notre arrêt de mort et entériner le fait que nous allons passer le restant de nos jours à assister au spectacle pitoyable de notre déchéance. Et personne, croyez-le bien, ne se privera de vous faire sentir à quel point vous ne servez plus à rien, si tant est que vous ayez jamais servi à quelque chose, sinon vous plier en quatre sans jamais vous plaindre pour que votre progéniture ne manque de rien. Et si un jour votre enfant vient vous trouver et vous dit qu’il n’a pas demandé à naître, répondez-lui que vous non plus n’avez rien demandé et vous seriez volontiers passé de sa présence si vous aviez su à quoi il allait ressembler. Il se peut alors, pour se venger bêtement d’une adversité qu’il ne soupçonnait pas, que, dans un geste exagérément théâtral, le rejeton en question décide de mettre fin à ses jours. Dites-vous bien, dans ce cas-là, que vous n’êtes pas davantage responsable de sa mort que vous ne l’étiez de son existence, et que toutes ces considérations oiseuses ne seront bientôt plus de mise dans le nouveau monde merveilleusement virtuel et orgasmiquement artificiel, véritable feu d’artifice de joie de vivre cybernétique sur fond de neurosciences sexy dopées à la myéline homéostatique, que nous nous proposons de créer. Plus personne, alors que tout le monde savait pertinemment que vous n’étiez pas en mesure de les assumer, ne viendra vous reprocher d’avoir failli à vos devoirs parentaux au profit de vos ambitions personnelles. Les psychologues autoproclamés de la connerie institutionnelle vous le répètent assez comme ça, se gargarisant à l’envi d’éléments de langage auxquels eux-mêmes ne comprennent pas un traître mot : vos enfants ne sont pas les vôtres, ne vous appartiennent pas, alors personne ne viendra vous demander de les fabriquer vous-même (souvent au prix d’atroces souffrances, la nature, que son principe d’économie, sinon de radinerie, pousse à entasser le maximum d’accessoires dans un minimum d’espace, n’ayant pas jugé nécessaire de doter la femme d’un orifice digne de ce nom pour mettre son enfant au monde, ce qui signifie que donner la vie a longtemps été synonyme de perdre la sienne) et vous en sentir responsable jusqu’à la fin de vos jours. La nature, qui n’a eu de cesse de nous harceler depuis des millénaires, sera domestiquée jusqu’au moindre brin d’herbe, la moindre touffe de poil, réduite au silence le plus abyssal, et il nous sera alors possible de promener virtuellement des microbes en laisse sur les trottoirs de l’infinitésimal. Et quand on en aura marre d’être éternel, que le temps aura totalement disparu et que Dieu lui-même sera venu publiquement reconnaître qu’il n’existe pas et n’a jamais existé ailleurs que notre imagination dévoyée, il nous suffira d’exercer une légère pression sur notre nombril pour mettre un terme à notre inexistence.

Le hall du Caribbean Hôtel était tellement vaste qu’un McDonnell-Douglas C-17 Globemaster III de l’US Air Force aurait pu s’y poser sans problème si la porte d’entrée avait été ne serait-ce qu’un poil plus large. De la même façon, avec une aisance comparable, un troupeau de buffles d’Afrique au grand complet aurait pu y séjourner en toute quiétude si le carrelage et la moquette avaient été remplacés par de l’herbe, n’ayant aucunement à redouter les balles des riches chasseurs occidentaux qui rêvent d’accrocher des têtes coupées de Big Five sur les murs de leurs résidences hors de prix. Et ceci pour une raison très simple : le Caribbean Hôtel était, sinon interdit aux Blancs, au moins réservé aux gens de couleur (au sens large du terme, c’est à dire que les Na’vi, Yondu, le Dr Manhattan~-- à ne bien évidemment pas confondre avec le Mr Manatane de Benoît Poelvoorde~-- et même l’ignoble Yellow Bastard de Sin City pouvaient y être admis en montrant patte blanche, au même titre qu’un Hellboy ou encore un Géant Vert, que son épiderme verdâtre et l’abominable odeur de maïs en boîte qui se dégage de sa personne ont définitivement mis au ban de la société), ce qui revenait sensiblement au même. Ce statut particulier, nettement discriminatoire, n’était finalement pas pire que d’interdire aux pauvres l’accès à nombre de manifestations culturelles ou écoles privées sous le prétexte qu’ils n’ont pas les moyens de se les offrir (ce qui entretient incidemment l’idée que ce sont tous des crétins incultes et sans avenir, proposition inacceptable même si non totalement dépourvue de fondement). On fait semblant de s’en émouvoir, d’en dénoncer l’injustice, mais la vérité c’est que la discrimination par l’argent ne choque plus personne depuis belle lurette, à commencer par les pauvres eux-mêmes qui trouvent tout naturel de claquer des fortunes pour aller applaudir des gens qui gagnent en une fraction de seconde ce qu’il leur faut des mois de labeur intensif, ingrat et notoirement sous-payé pour acquérir.

À cette heure matinale, l’endroit était désert.

Quelques vagues grooms en costume folklorique s’agitaient ici et là, sans doute pour faire croire qu’il se passait quelque chose alors qu’il ne se passait strictement rien, je peux en témoigner sur la vie de feu ma grand-mère maternelle adorée Alexandrine Chéron, née Lemaître, décédée en juin 2022 alors qu’elle survolait la cordillière des Andes à bord du Cessna Skylane 182 jaune canari qu’elle s’était offert pour son quatre-vingt-septième anniversaire. Cette femme, une femme de tête qui avait toujours placé l’indépendance au-dessus de tout et ne s’en était jamais laissé compter par tous les beaux-parleurs qui avaient croisé sa route (exception faite de mon grand-père Philibert, dont le charme ravageur, le sens aigu de la probité et les moyens financiers assez conséquents avaient eu raison de sa résistance), cette femme, disais-je, restait pour moi l’archétype absolu de l’aventurière intrépide au physique de reine de beauté. Les photos d’elle que j’avais vu quand elle n’était encore qu’une adolescente frondeuse ou une splendide jeune femme dont le regard bleu d’acier ne laissait planer aucun doute sur le caractère farouche et la soif de liberté, m’avaient fait forte impression. Aujourd’hui encore, parvenu à un certain degré de maturité dans l’existence, il m’est difficile de regarder ces photos, de qualité très médiocre pour la plupart, sans faire aussitôt l’objet de turbulences intérieures d’une violence inexplicable. Certes, c’était ma grand-mère et je l’adorais, mais de là à me mettre à chialer comme un gosse dès que je tombe sur une photo d’elle en maillot de bain, je pense qu’il y a tout de même un pas qui ne devrait pas être franchi avec une telle allégresse. Je me souviens, quand j’étais petit, qu’elle était encore un très belle femme pour son âge. Je dirais même anormalement, comme si un philtre de jeunesse éternelle la protégeait des ravages du temps. Bon, je reconnais que celui-ci avait fini par la rattraper, car à bientôt cent ans elle ressemblait quand même davantage à une vieille momie édentée qu’à une biche au teint frais comme la rosée du matin. Cela dit, au milieu des ruines subsistaient encore quelques reliques des splendeurs du passé, aussi scintillantes que des pépites dans le lit boueux d’une rivière.

Greg a dit, la voix traversée par un vieux frisson de peur ancestrale telle que l’homme n’en avait plus connu depuis que le dernier spécimen d’ours de Deninger s’est éteint dans une grotte de la sierra d’Atapuerca, près de Burgos : Je déteste cet endroit.

Ce à quoi j’ai répondu immédiatement, sans lui laisser le temps de s’enfoncer davantage dans les profondeurs sombres et humides de l’effroi : Pas moi.

C’était vrai, du reste, je le trouvais plutôt sympathique, cet endroit. Sauf peut-être la décoration, qui laissait quelque peu à désirer. En effet, le propriétaire des lieux, sans doute frappé de démence, n’avait rien trouvé de mieux à faire que de transformer l’endroit en une espèce de vieux musée de province rempli d’objets poussiéreux tout droit sortis d’un film d’horreur des années 50. Les animaux empaillés, par exemple, faisaient un peu désordre dans un établissement de cette catégorie, revendiquant un niveau de standing à priori incompatible avec la présence d’un groupe de hyènes au milieu du salon. L’hippopotame non plus n’avait rien à faire là, pas plus que les antilopes, le léopard en train de déchiqueter un phacochère (les entrailles étaient particulièrement bien imitées), ou encore le vautour qui déployait ses ailes au-dessus de la Réception, prêt à se jeter sur le client venu réclamer sa clé.

\textsc{Greg} : On se croirait dans Psychose.

\textsc{Moi} : En plus exotique.

Je sais qu’il existe, de nos jours enténébrés, des jeunes gens qui n’ont jamais vu un film en noir et blanc, ni entendu parler d’Alfred Hitchcock et encore moins de Robert Bloch.

Grand admirateur de Lovecraft avec lequel il entretient une longue relation épistolaire, Bloch est pourtant une des figures majeures de la littérature fantastique et horrifique américaine. Passionné par les histoires de monstres en tout genre, il s’intéresse de près à une certaine catégorie de prédateurs sexuels qui n’hésitent pas à tuer pour assouvir leurs fantasmes déviants. Un certain Edward Theodore Gein, par exemple, vient de défrayer la chronique. Il semblerait que le décès de sa mère, une fanatique protestante qui détestait les hommes, ait eu un effet désastreux sur sa personnalité. Après sa mort, le fiston, alors âgé de trente-neuf ans, commence à péter très sérieusement les plombs. Les gens normaux, au moins ceux qui croient en Dieu et une vie après la mort, se rendent au cimetière pour fleurir les tombes et prier pour le salut des âmes de leurs défunts. Pas lui. Quand il a constaté, malgré ses suppliques répétées et ses incantations au clair de lune, que sa mère ne semblait pas décidée à refaire surface, aux grands maux les grands remèdes, il est revenu avec sa plus belle pelle pour la sortir de terre. Il a ramené son butin à la ferme et s’est livré sur lui a des pratiques que la morale réprouve. Et le jour où il s’est lassé de son jouet, mu par des pulsions dans le détail desquelles je préfère ne pas entrer (je m’en voudrais qu’un enfant innocent, tombé par hasard sur cet ouvrage, se retrouve traumatisé par sa lecture), il est retourné au cimetière pour s’approvisionner. Que des cadavres de femmes, bien sûr, qu’il rapportait jalousement chez lui pour se fabriquer des trophées tous plus macabres les uns que les autres. Si vous aviez des envies bizarres, comme équiper votre salon avec un canapé en cuir de femme ou vous balader dans les rues de la ville avec une veste du même matériau sur le dos, c’est Eddie qu’il fallait aller voir. Nul doute, avec tous les cinglés en liberté, qu’il aurait pu se faire pas mal de fric en vendant ses créations au lieu de les garder pour lui. Mais s’il était doué pour la couture (peut-être pas autant que Paul Poiret ou Jean Patou, mais il avait son petit savoir-faire), il n’avait aucun sens du commerce et ne tenait aucunement à ce que ses activités s’ébruitent. Certains, du fait de la nature quelque peu discutable de leurs activités, sont condamnés à la clandestinité, ce qui est un moindre mal comparé aux risques qu’ils encourent. Toujours est-il que ce qui n’était au début qu’un passe-temps bien innocent, censé lui changer les idées et l’aider à supporter les affres de la solitude, est rapidement devenu une quête obsessionnelle. Il lui en fallait toujours plus, et les ressources que la nature met généreusement à notre disposition se révèlent parfois largement insuffisantes. En clair, les gens ne mouraient pas assez vite pour suivre le rythme effréné de sa créativité. Eddie, qui n’avait jusqu’ici connu que les plaisirs solitaires en essayant d’échapper au regard accusateur des crucifix disséminés un peu partout dans la baraque, venait de découvrir avec émerveillement les joies de l’amour physique avec une vraie femme parfaitement consentante et entièrement soumise à ses désirs, qu’il pouvait profaner par tous les orifices sans que sa mère, désormais transformée en abat-jour, descente de lit et autre rideau de douche, vienne le menacer des foudres de l’enfer. Il pouvait désormais faire tout ce qu’il voulait, même si les chairs faisandées qu’il malaxait avaient parfois tendance à se déliter sous ses doigts, tout comme il n’était pas toujours très agréable de fourrer sa langue dans des cavités buccales débordant d’asticots. Pour toutes ces raisons (pénurie de matière première et besoin de chaleur humaine), Eddie s’est mis à rêver de faire l’amour à des femmes encore tièdes qu’il aurait lui même choisies, au lieu de se contenter d’articles de récupération ayant depuis longtemps dépassé leur date de péremption. Il se fait la main sur Mary Hogan, de Pine Grove, dont il garde la tête pour se rappeler des bons moments passés en sa compagnie, avant de jeter son dévolu sur Bernice Worden, une sexagénaire qu’il croise régulièrement dans les rues de Plainfield et à laquelle il n’ose déclarer sa flamme, ne disposant pas du bagage technique nécessaire pour s’exprimer clairement et avoir la moindre chance de retenir l’attention d’une femme aussi belle et raffinée. En désespoir de cause, il met un terme brutal à son existence, la ramène dans sa tanière et peut enfin, à l’abri des regards et sans craindre le ridicule, lui témoigner toute l’étendue de la passion qui le consume jour et nuit. Mais quand ils s’étonnent de ne plus la voir et apprennent l’étrange disparition de Bernice, des voisins signalent à la police avoir vu à plusieurs reprises un type bizarre rôder autour de chez elle. Ce type bizarre, c’est Ed Gein, un gars du coin qui vit seul dans une ferme pourrie des environs de la ville. C’est le fils de George et Augusta Gein, le petit dernier d’une fratrie de deux garçons. Ils sont tous morts sauf lui. Il n’est plus tout à fait le même depuis le décès de sa mère, à laquelle il vouait une adoration sans bornes. C’était une maîtresse femme qui menait son monde à la baguette, une protestante rigoriste qui ne tolérait pas les écarts de conduite. Eddie avait la réputation d’un gars à la limite du handicap mental, mais toujours bien poli et prêt à rendre service. Quand les flics ont débarqué chez lui, ils ont été saisis par une pestilence telle que le contenu de leur estomac leur est aussitôt remonté dans le fond de la gorge. Ensuite, ils ont eu droit à une visite guidée du petit musée des horreurs qu’Eddie s’était aménagé à domicile. Il ne leur a pas fallu longtemps pour comprendre qu’ils venaient de tirer le gros lot.

C’est ainsi que Robert Bloch, quand il a eu vent de l’affaire, s’est attelé à la rédaction que ce qui allait devenir son roman le plus célèbre : PSYCHOSE. Alfred Hitchcock, le petit gros qui adorait foutre la trouille aux gens et fantasmer sur les créatures de rêve, blondes pour la plupart, qu’il engageait pour tourner dans ses films, lit le livre, et, en bon maniaque sexuel féru de psychanalyse qu’il est, décide aussitôt de l’adapter à l’écran. Gein n’est plus le péquenaud attardé du Wisconsin qui a fait frissonner le pays tout entier, mais le jeune et timide Norman Bates, célibataire endurci qui tient un motel pas très réjouissant en bordure de nationale. Ténébreux à souhait, posant sur les êtres et les choses un regard d’une étrange fixité, Norman prend soin de sa vieille mère malade et empaille des oiseaux pendant son temps libre. On aurait tendance à lui donner le bon dieu sans confession, mais il a aussi des petites manies qui pourraient facilement le faire passer pour un vilain garçon. Par exemple, et cela n’a rien à voir avec une quelconque passion contrariée pour le bricolage, il adore faire des trous dans les murs pour mater les clientes en petite tenue. Il faut dire que sa libido et ses conditions de vie difficiles lui occasionnent de sérieux troubles du comportement. Sévère et très à cheval sur les principes (à défaut d’autre chose), sa mère se transforme instantanément en redoutable bras armé de la justice divine pour châtier les Messaline qui viennent agiter leurs appas sous les yeux exorbités de son petit chéri d’amour. Lui-même, conscient des agissements de la vieille femme qui n’a hélas plus toute sa tête, se doit de tout mettre en œuvre pour la protéger. Il n’est donc pas rare que des visiteurs un peu trop curieux disparaissent sans laisser de trace.

Je citerai aussi, dans un style nettement plus gore et pétaradant, le cultissime Massacre à la tronçonneuse de Tobe Hooper : une famille de cannibales vit dans la nostalgie du passé, le bon vieux temps où l’abattoir du coin faisait vivre tout le comté. Ils ont un sens de la décoration assez particulier, en rapport avec leur ancienne profession de boucher. Le papy, qui doit avoir dans les cent cinquante ans et survit au grenier dans des conditions d’hygiène déplorables, n’avait pas son pareil pour assommer un bœuf à coups de marteau. Il passe le plus clair de son temps à somnoler dans son fauteuil, mais la vue d’un bon plat de tripes ou une belle andouillette suffit à le ramener à la vie et lui redonner la pêche de ses vingt ans. Cinq potes en vacances, deux filles et deux garçons plus un troisième qui n’est autre que le frère handicapé de l’une d’entre elles, débarquent dans les environs de Round Rock, au Texas. Sur la route, ils font la connaissance de divers membres de la famille, dont le père à la station-service, un type bizarre avec une tête de rat, et un des fils débiles qu’ils font l’erreur de prendre en stop. Ce dernier, qui n’a manifestement pas toute sa tête, s’intéresse de très près à Franklin, le frère handicapé de Sally Hardesty, la copine de Jerry. Quand le débile, après avoir tenu des propos inquiétants, sort un couteau pour se tailler un steak dans le cul de Franklin, les cinq gens lui demandent de prendre congé. En guise de réponse, il s’entaille le creux de la main en rigolant et menace de tuer tout le monde. Ils finissent par réussir tant bien que mal à le faire sortir, mais le débile, qui se balade avec des animaux morts dans son sac, les maudit ouvertement et imprime l’empreinte sanglante de sa main en guise de signe cabalistique sur la carrosserie du van. Pendant ce temps, les autorités alertent l’opinion sur le fait que les profanations de sépultures se multiplient dans le secteur, accompagnées d’actes blasphématoires particulièrement scandaleux et répugnants. En effet, après avoir été extraits de leurs tombes, ce qui représente déjà une grave atteinte à leur intégrité, les cadavres sont ensuite installés comme des épouvantails dans le cimetière. Seuls des gens souffrant de graves troubles mentaux, sans doute membres d’une secte s’adonnant à des orgies sexuelles et des sacrifices de nourrissons, sont capables de telles horreurs. Les cinq jeunes gens, partis à la recherche de la maison où Sally et Franklin ont passé une partie de leur enfance, finissent par mettre la main dessus. Grande est leur déception quand ils constatent qu’il n’en reste que des ruines. Mais quand on est jeune, beau, qu’on a toute la vie devant soi, qu’on n’a pas un gramme de graisse sur le corps, qu’on a des tablettes de chocolat quand on est un garçon et un petit cul rond comme un ballon moulé dans un micro-short en jean quand on est une fille, il faut un peu plus que ce genre de contretemps pour entamer votre bonne humeur. C’est ainsi que Kirk et Pam, l’autre sympathique petit couple de notre fine équipe de randonneurs estudiantins, décident d’aller faire trempette dans le doux ruisseau clapotant aperçu en cours de route. Ce faisant, ils aperçoivent, au détour d’un sentier tortueux, une propriété isolée dont le moins qu’on puisse dire est qu’elle n’inspire pas franchement confiance. Ils l’ignorent encore, et nous aussi même si on commence à entrevoir la sinistre vérité à travers le voile trompeur de la joie de vivre et l’insouciance, mais cette bicoque délabrée n’a rien de la petite maison dans la prairie. On a tous en tête cet endroit idyllique où des gens charmants vous accueillent à bras ouverts comme si vous faisiez partie de la famille depuis toujours. Vous tombez en panne en rase campagne, frappez à la porte, une mère de famille sexuellement attirante (en anglais Mother I’d Like to Fuck, NDLR) vous ouvre la porte, les cheveux en bataille et le décolleté largement ouvert sur des perspectives vertigineuses dignes de la Vallée des plaisirs, et vous offre aussitôt une énorme part de tarte à la citrouille que vous n’avez pas intérêt à refuser si vous ne voulez pas qu’elle vous arrache les yeux avec ses ongles de trente centimètres de long taillés en pointe. Vous vous dites «~Pourquoi moi, pourquoi tant de bonheur, est-ce que la chance serait enfin en train de tourner en ma faveur ?~», et elle vous annonce que le type encadré dans le photo, celui-là même que vous regardez avec insistance, les yeux remplis de crainte parce qu’il a l’air d’une brute épaisse et que vous ne pouvez vous empêcher de penser qu’il ne sera peut-être pas enchanté de vous trouver en compagnie de sa femme en rentrant d’une dure journée de labeur, elle vous annonce qu’il est décédé l’année dernière, pulvérisé par un poids lourd sur l’interstate 35, une des routes les plus meurtrières de tout le Texas. Vous aimeriez sauter de joie jusqu’au plafond, sombrer sans retenue dans la plus infâme concupiscence, mais, Dieu merci, la bienséance et la volonté de ne pas passer pour un sale con et un gros porc lubrique vous empêchent de décoller. Quoi qu’il en soit, il semblerait en effet que la chance soit en train de tourner en votre faveur. Et attendez, ce n’est pas tout. Vous lui posez prudemment la question de savoir s’il elle lui a trouvé un remplaçant, et elle vous répond, tout en posant sur vous un regard tropical qui vous transforme le bas-ventre en fourmilière, que non, elle n’a pas la tête à ça, que c’est encore trop tôt pour songer à la bagatelle, alors que des vieux bouts de vêtements et des lambeaux de chair pourrie s’accrochent encore aux os de ce pauvre Edward qui s’agite dans le fond de sa tombe. Bien sûr, vous ne comprenez que trop bien, il faut laisser du temps au temps, répondez-vous en baissant pudiquement les yeux sur les profondeurs de son décolleté et vous fustigeant intérieurement de l’absence totale de moralité qui vous caractérise, ajoutant d’une voix de fausset qu’il commence à se faire tard et que vous n’allez pas tarder à prendre congé, même si vous n’avez nulle part où aller et courez le risque d’atterrir dans un motel pourri tenu un jeune homme féru de taxidermie. Que nenni, s’insurge-t-elle en projetant sur vous une onde de choc parfumée qui vous transporte au septième ciel, ce n’est pas parce que ce pauvre Edward est mort dans des circonstances tragiques qu’elle doit pour autant renoncer aux valeurs de charité chrétienne qui ont toujours été les siennes. Vous trépignez de joie intérieurement en entendant ces paroles de réconfort, aussi douces qu’un tapis de mousse sous vos pieds nus. Un bonheur n’arrive jamais seul, écrivait Marie de Rabutin-Chantal, marquise de Sévigné, dans une lettre à sa fille datée du 20 novembre 1676, et vous n’allez pas tarder à en faire l’expérience. En effet, la porte s’ouvre et une jeune fille, plus belle que le jour et la nuit réunis, fait sont entrée dans la pièce. Et cette charmante jeune fille, sommairement emballée dans une robe à fleurs qui ne cache pas grand-chose de son anatomie dévastatrice, c’est tout simplement Candy, la fille de Janet, la jeune femme qui a eu la gentillesse de vous offrir l’hospitalité en plus d’une généreuse part de tarte à la citrouille (vous détestez ça mais vous forcez quand même à l’avaler jusqu’à la dernière miette, conscient que cette mise en bouche pourrait bien constituer les prémisses d’un festin autrement réjouissant).

Voilà comment les choses pourraient se passer si nous vivions dans un monde meilleur, et surtout si quelqu’un d’autre que Tobe Hooper avait écrit et réalisé le film. Dans le cas présent, vous le savez parce que vous avez déjà vu le film ou n’êtes tout simplement pas tombé de la dernière averse, la résidence en question, à peu près aussi accueillante qu’une décharge à ciel ouvert infestée de rats et de cafards gros comme des rats, n’est autre que celle des cannibales de service, la famille Sawyer. Kirk va être le premier à y passer, suivi de près par Pam qui finit dans le congélo après avoir été empalée sur un croc de boucher. Je vous passe les détails, mais je vous promets que la scène ne manque pas de piquant. La curiosité est un vilain défaut, et si vous en doutez encore, faites confiance à Bubba Sawyer pour vous en faire la démonstration aussi sonore que tranchante. Géant consanguin doté d’un cerveau de batracien, Bubba sait qu’il

est moche comme un pou trisomique et ne supporte pas son reflet dans un miroir. Il porte un masque en peau humaine, arraché au crâne de l’une de ses nombreuses victimes. Bûcheron dans l’âme, il découpe les voyageurs de passage à la tronçonneuse avant de les refiler à son frère Drayton, le cuisinier de la famille, qui se charge de les transformer en fromage de tête (bon, ça, le fromage de tête, avec des pickles, des cornichons et un bon verre de vin blanc sec) et chili con carne (un des meilleurs de la région, si l’on en croit les voyageurs qui ont eu le privilège de le déguster, et surtout de survivre à la dégustation). Après Kirk et Pam, c’est au tour de Jerry, le petit copain de Sally parti à leur recherche, de se faire défoncer la gueule à coups de marteau par Bubba Sawyer, plus connu dans le milieu SM sous le nom de Leatherface. Sally, restée avec son frère handicapé pour veiller sur lui, n’a aucune nouvelle de ses amis et voit la nuit tomber avec une anxiété grandissante. Je suppose que Hooper, quand il a écrit le scénario, s’est dit que ce serait sympa de voir un type en fauteuil roulant se faire découper à la tronçonneuse par un géant demeuré avec un masque de cuir sanguinolent sur le visage. S’en prendre avec une telle violence à une créature sans défense ne pourrait que faire date dans l’histoire du septième art et du film d’horreur en particulier, genre mineur que ses plus ardents défenseurs considèrent comme le nec plus ultra de la création artistique. Il a également compris que le film serait encore plus crédible si l’on insinuait perfidement qu’il n’était en aucun cas une fable horrifique issue du cerveau malade d’un réalisateur indépendant plus ou moins détraqué, mais la relation fidèle et sans complaisance d’événements s’étant réellement déroulés quelque part au fin fond du Texas. Oui, braves gens, tandis que vous ronflez paisiblement dans vos pavillons de banlieue, bien à l’abri derrière vos portes blindées et vos système d’alarme dernier cri, le 357 Magnum sous l’oreiller, il existe encore des endroits reculés où la sauvagerie la plus extrême s’exerce en toute liberté. Pendant que vos sirotez vos cocktails au bord de vos piscines à quatre-vingt mille dollars, insensibles à la misère du monde, des jeunes gens dans la fleur de l’âge se font découper à la tronçonneuse par des mongoliens de deux mètres de haut au visage masqué. Pensez-y, quand vous vous gaverez de pop corn devant la télé, et n’oubliez pas de vous renseigner à deux fois avant de partir à l’aventure. Vous vous croyez plus malin que les autres, mais pourriez vous aussi vous exposer à de très graves dangers. Et vous, parents, tremblez quand vos enfants décident de partir en vacances à l’autre bout du monde, certains que la fraîcheur de la jeunesse leur ouvrira toutes les portes et les préservera du mauvais sort. Elle pourrait tout aussi bien leur ouvrir grand les portes de l’enfer.

Dans la même veine, je me ferai également une joie de mentionner le très glauque et méconnu Maniac de William Lustig : Franck Zito ne fait pas partie de ces gens qui ont eu la chance de vivre dans un foyer harmonieux, avec des parents aimants qui ne passent pas leur temps à picoler et s’engueuler, et élèvent leurs enfants dans le respect des saines valeurs du travail et l’amour du prochain. Non. Quand le père de Franckie n’était pas en train de s’arsouiller au bistrot, il rentrait à la maison pour tabasser sa femme et ses gosses. Un jour pas fait comme un autre, il s’est barré et on ne l’a jamais revu. Quant à la mère de Franckie, elle ne valait guère mieux que son père. Elle-même largement alcoolique et dépendante de nombreuses substances prohibées, elle se vendait au plus offrant pour arrondir ses fins de mois. Contrairement à d’autres prostituées dans son genre, elle bossait à domicile et le petit Franckie assistait fréquemment aux ébats de sa mère et ses amants de passage, une «~clientèle~» dont je vous laisse imaginer l’élégance naturelle et la distinction. En dépit de son comportement inadapté et totalement dépourvu de chaleur humaine, Franckie avait élevé sa mère au rang d’idole absolue et rêvait du jour où les gros porcs qui lui passaient sur le corps trouveraient leur juste châtiment. Ce n’était pas elle la coupable, mais ces ordures qui profitaient de sa faiblesse pour assouvir leurs coupables penchants. Et lui aussi était coupable de ne pas pouvoir la protéger. Sa mère reçoit des hommes chez elle, mais tous n’ont pas les mêmes intentions. Certains, par exemple, s’intéressent davantage à Franckie qu’à sa mère. Ils vont même jusqu’à faire des offres très alléchantes pour s’octroyer le droit de faire de lui ce que bon leur semble. Sa mère, toujours à court de fric pour payer ses doses, explique à Franckie qu’il va devoir faire à ces messieurs certaines des choses qu’elle-même fait avec eux, notamment avec sa bouche. Franckie hésite, mais sa mère insiste en disant qu’ils risquent de lui faire du mal s’il ne le fait pas. Il ne voudrait pas que les vilains messieurs fassent du mal à sa maman, n’est-ce pas ? S’il est bien sage, maman lui fera un gros câlin quand ils seront partis. Des années plus tard, longtemps après la mort de sa mère, Franck a développé de sérieux troubles de la personnalité. En clair, il est complètement barré, incapable de se comporter normalement en société. Comme Hugh Hefer, il vit entouré de mannequins. Sauf qu’il ne vit pas à Beverly Hills, dans un manoir de deux mille mètres-carrés avec (entre autres) salle de sport, piscine, tennis, zoo privé et parc paysager avec grotte comme dans les contes de fées, et que ses amis ne s’appellent pas John Lennon, Elvis Presley, Alec Baldwin, Sylvester Stallone, Pamela Anderson, Cameron Diaz, Leonardo DiCaprio ou Kim Kardashian. Oh non. Franckie n’a aucun ami, sa vie est loin d’être un conte de fées, et ses mannequins à lui n’ont rien à voir avec les créatures plantureuses au sourire éclatant qui s’agglutinent comme des mouches autour du patron de Playboy. Ses mannequins à lui ne marchent pas, ne parlent pas, ne respirent pas. Ils ne sont pas morts, non, c’est juste qu’ils n’ont jamais été en vie, même s’ils ont toutes les apparences de la réalité. Ce sont juste des objets, des poupées à taille humaine qu’on habille pour exposer dans les boutiques de fringues. Plastiquement parfaites, certes, mais surtout parfaitement en plastique. C’est précisément cette inertie, alliée à cette troublante ressemblance, qui peut donner des idées à certains individus sexuellement perturbés. Et même à d’autres, qui n’ont à priori rien de commun avec les pervers en question, mais sont juste fatigués d’avoir à composer avec les exigences d’un ou une partenaire en chair et en os (vous le voyez, j’essaie d’être aussi inclusif que possible, conscient que les fanatiques de la non-binarité intersexuelle et transidentitaire seraient trop contents de planter ma tête au bout d’une pique au moindre faux pas). Pour beaucoup d’hommes, confrontés à la difficulté croissante de vivre en couple (depuis, pour faire court, que la femme n’est plus cette petite chose fragile et soumise qu’on pouvait engrosser à loisir et asservir en toute tranquillité), la femme parfaite serait une poupée sexuelle ou un androïde hyperréaliste offrant tous les avantages d’un être humain sans en avoir les inconvénients. Notons, au passage, qu’une telle évolution des mœurs éviterait bon nombre de féminicides perpétrés par des conjoints alcooliques et violents. Car si un être vivant ne vous appartient pas, il n’en va pas de même pour un humanoïde en silicone. Même les prédateurs sexuels les plus abjects, comme l’ont prouvé les récentes affaires de ventes de poupées pédopornographiques sur Shein, AliExpress et Amazon, pourraient y trouver leur compte. Non seulement de nombreux enfants ne seraient plus traumatisés par leurs agissements, et des familles brisées dans la foulée, mais de nombreuses vies seraient épargnées, sachant que les pires d’entre eux n’hésitent pas à tuer pour s’assurer du silence de leurs victimes. Sur le plan de la déontologie, je reconnais que la pilule est assez difficile à avaler. C’est déjà assez moche de savoir qu’on vit entouré de porcs qui ne songent qu’à abuser de nos enfants (certains vous diront que c’est encore plus moche de ne pas savoir qui ils sont, et ils n’auront sans doute pas complètement tort), on ne va pas en plus leur fournir de quoi assouvir leurs fantasmes en toute légalité. Cela reviendrait, en quelque sorte, à officialiser leur existence et accepter de vivre avec eux en bonne intelligence. C’est d’autant plus difficile à imaginer qu’on sait pertinemment qu’il y aura toujours des brebis encore plus galeuses que les autres pour s’affranchir des règles en vigueur, jetant le discrédit sur toute la communauté de gentils pédophiles respectueux qui se contentent de tripoter des poupées en silicone.

Lustig, après de rapides débuts dans le porno, s’est tourné vers le cinéma de genre, le psycho-killer movie en l’occurrence. Un beau jour de l’année 1979, il est allé trouver son pote Joe Spinell, abonné aux seconds rôles de flics véreux et truands sans envergure, pour lui proposer de se glisser dans la défroque peu avenante de Franck Zito, propriétaire d’une boutique de mannequins et tueur de femmes psychotique obsédé par le souvenir de sa mère prostituée. Spinell, qui n’avait sans doute rien de mieux à foutre, a accepté, et s’est impliqué dans le film au point de participer à l’écriture du scénario. Vivre avec des femmes en plastique, c’est bien, mais ça ne suffit pas à faire un film d’horreur digne de ce nom. Pas mal de trucs ont déjà été faits, il va falloir trouver quelque chose d’un peu original si on veut avoir une chance de sortir du lot et décrocher un Oscar. À l’époque, la plupart des tueurs écumaient les magasins de bricolage pour s’équiper. Agrafeuse, burin, clé à molette, disqueuse, faucille, fourche, hache, machette, marteau, masse, meuleuse, pelle, perceuse, pied à coulisse, pioche, pistolet à clous, poinçon, tournevis, tronçonneuse, scie sauteuse, spatule, tout y passait dans la joie et la bonne humeur. Si on avait pu tuer quelqu’un à coups de tire-bouchon, ouvre-boîte ou lime à ongles, je vous assure qu’on ne se serait pas gêné pour le faire. J’estime toutefois que le couteau, qui peut sembler désuet, sinon vulgaire à première vue, reste une valeur sûre, à condition bien évidemment qu’il soit de taille conséquente et manipulé par quelqu’un qui connaît son affaire. Et c’est exactement ce que se sont dit nos deux compères : Zito, qui n’a aucun goût particulier pour le bricolage, va se servir d’un bon vieux couteau de chasse des familles. Et à quoi peut bien servir un bon vieux couteau de chasse des familles, hein, je vous le demande ? Eh bien à dépecer des animaux, par exemple, ou tailler des morceaux de bois pour en faire des armes redoutables, comme Stallone dans Rambo. Mais Zito crèche à New York, pas dans la jungle, et ce n’est pas dans les rues de la Grosse Pomme qu’on risque de croiser une biche ou tapir. Il fallait donc trouver quelque chose de plus en rapport avec le profil de sociopathe schizophrène de l’intéressé.

Spinell, avec sa gueule de type qu’on n’avait pas envie de croiser de nuit au détour d’une ruelle sombre et humide, au pavé glissant et aux trottoirs envahis de sacs poubelles éventrés, était un grand fan de western. Comme vous le savez, les westerns sont ces films qui évoquent les différents aspects de la conquête de l’Ouest. Les Rosbifs ont débuté avec treize colonies sur la côte Est, tandis que les Français étaient plutôt bien installés au centre du pays. Ils s’étaient même fait quelques potes indiens pas trop regardants avec lesquels ils commerçaient pour tenter s’intégrer harmonieusement dans le paysage, même s’il était clair que ça risquait de péter un jour ou l’autre. Plus au Sud, on trouvait les Espagnols qui continuaient à creuser des trous un peu partout pour trouver de l’or. Il y avait aussi quelques Hollandais qui gravitaient dans le secteur, arrivé dans le sillage de Peter Stuyvesant, mais ils ne faudrait pas longtemps pour le rejeter à la mer en cas de conflit. On trouvait un peu de tout, en fait, mais pas en quantité suffisante pour représenter une réelle menace. Tous ces gens bricolaient dans l’espoir de tirer un jour leur épingle du jeu. Un jour, les treize colonies en ont eu marre des exigences de la Couronne et décidé de s’affranchir une bonne fois pour toutes de ses directives. Ils avaient pris tous les risques et entendaient bien créer un nouvel empire dont ils seraient les seuls maîtres après Dieu. La Mère Patrie resterait à tout jamais dans le fond de leur cœur, bien entendu, tatouée à l’encre rouge de la conquête normande et la révolte des barons, mais ils avaient aujourd’hui d’autres chattes à fouetter, et non des moindres. Dans la foulée, ils ont viré les Français et le reste, laissant les Espagnols se dépatouiller avec l’Amérique du Sud. Je vous la fais courte, historiquement approximative, mais c’est grosso modo comme ça que les choses se sont passées. Comme ils n’occupaient qu’une toute petite partie du très vaste territoire sur lequel ils venaient de poser le pied, ils ont repris leur bâton de pèlerin et décidé d’aller voir ce qui se passait un peu plus loin. Ils n’ont pas tardé à se rendre compte que, en dehors des colons blancs et esclaves noirs qu’ils avaient amenés dans leurs valises, il existait une race d’autochtones dont il n’allait peut-être pas être aussi facile de se débarrasser. Même s’il s’agissait de sauvages peinturlurés tels qu’on en avait maintes fois croisés dans les expéditions passées, ils risquaient de sérieusement entraver la marche triomphale de la civilisation occidentale vers l’avenir radieux du libéralisme économique. Dès qu’un brave type de colon sans histoire s’installait quelque part, tranquille, avec sa petite famille et ses quelques têtes de bétail qui broutaient paisiblement dans le champ voisin, ces abrutis lui tombaient dessus en poussant des hurlements stridents et saccageaient tout sur leur passage, allant jusqu’à violer les femmes et enlever les enfants pour les élever à la sauce indienne. Déjà qu’il fallait se coltiner les hors-la-loi qui foutaient la merde dans les saloons, se battaient en duel à tous les coins de rues et détroussaient les voyageurs (il leur arrivait aussi de violer les femmes), ça allait devenir difficile de faire son beurre s’il fallait avoir en permanence cette bande d’emplumés sur le dos. La tâche était d’autant plus ardue qu’ils n’étaient pas constitués en une seule armée, qu’on aurait pu vaincre en une seule fois, mais une multitude de tribus éparpillées aux quatre coins du pays. L’avantage, c’était que les tribus en question n’étaient pas toujours en très bons termes, ce qui permettait, avant bien sûr de se retourner contre eux, de pactiser avec les uns pour affaiblir les autres. Et comme c’était des sauvages, c’est à dire des gens qui vivaient sous des huttes en peau de lapin, se trimballaient les couilles, lâchaient des caisses dans la plus totale insouciance, fumaient des pipes de trois mètres de long et ne disposaient que d’un armement rudimentaire pour faire valoir leurs droits à la terre de leurs ancêtres, il ne devrait pas être trop difficile de leur faire fermer leur grande gueule une bonne fois pour toutes. Car dis-toi bien ceci, ô vil suppôt de l’impérialisme américain dont l’existence même représente un grave danger pour la survie de l’espèce à laquelle j’ai la malchance d’appartenir : il existait, dans les antiques contrées de la lointaine Amérique, des kyrielles de tribus qui se partageaient les vastes plaines de l’Ouest, du Texas à l’Etat de Washington en passant par la Lorraine, l’Arizona, l’Idaho, le Colorado, le Montana, l’Iowa, le Nouveau-Mexique, le Nevada, l’Oregon, l’Utah, l’Arkansas, le Kansas tout court, le Wyoming et l’entrée de service. En plus ou moins bonne intelligence, il est vrai, car il leur arrivait fréquemment de se taper sur la gueule comme des gamins de cinq ans qui se disputent un jouet. Ces grands enfants avaient de longues conversations avec la lune, les pierres et les ruisseaux, dansaient avec les loups, et respectaient chaque chose, même la plus insignifiante, comme un membre à part entière de leur famille. Chaque année, ils attendaient le retour des bisons pour remplir les frigos, faire le plein de peau et viande séchée, et les bisons étaient tellement nombreux qu’il y en avait largement assez pour tout le monde. Leur vie s’écoulait ainsi, au gré du vent et des saisons. Ils n’en demandaient pas davantage et pensaient que les choses dureraient ainsi jusqu’à la fin des temps. Quand il a vu débarquer l’homme blanc, avec ses chemises carreaux, ses jeans et ses santiags, l’Indien a tout de suite compris que les jours heureux étaient terminés. Il faut dire que l’homme blanc s’est vite montré envahissant, totalement irrespectueux des usages, coutumes et règles en vigueur, arrogant, violent, dominateur, certain de la supériorité de son espèce, se comportant comme le roi du monde en pays conquis. Il a bien essayé de résister, mais son matos de fortune ne faisait pas le poids face à la puissance de feu de l’envahisseur. Le temps de décocher une flèche et il s’était déjà pris trois balles dans le buffet. La lutte était inégale et s’est achevée comme elle devait s’achever : par la victoire écrasante du pot de fer contre le pot de terre, sa prise de contrôle totale du territoire et sa relégation aux archives de la vieille nation indienne, réduite à survivre sur les lopins de terre indignes généreusement alloués par le gouvernement fédéral des Etats Unis d’Amérique. Le Visage Pâle est arrivé («~et je vis un cheval fauve, et celui qui était monté dessus avait nom la Mort, et l’Enfer suivait après lui~», Apocalypse 6:8) et il a dit : vous êtes ici chez moi, faites vos valises et foutez le camp. Fini les chants et les danses au clair de lune au son du tambour, les discussions interminables au coin du feu et la chasse au bison dans les vertes prairies du Wyoming et de l’Oklahoma. On est venu ici pour s’établir, entre gens civilisés, on n’a pas besoin de sauvages pour tenir la chandelle. On va faire des trous partout pour trouver de l’or et du pétrole. Ouais, avec des grosses machines qui fument et font du bruit. Grâce à ça on va devenir très riche et on va transformer vos terrains vagues en cités radieuses et tentaculaires. On va aussi construire des lignes de chemin de fer pour transporter nos marchandises d’un bout à l’autre du pays, des grosses bagnoles pour mettre plein d’essence dedans et une route pour aller de Chicago à Santa Monica. On va vous la mettre bien profond et vous ne pourrez rien faire pour nous en empêcher. Car la terre n’appartient à personne, même si les membres d’une même famille y sont enterrés depuis des générations, et comme elle n’appartient à personne, elle appartient à tout le monde, et donc à moi en particulier. Tu ne croyais tout de même pas que l’état de grâce allait durer éternellement, que t’allais continuer à te la couler douce sous ton tipi, caracoler dans les Rocheuses et fumer le calumet de la paix pendant que nous, fleurons de l’espèce humaine et garants de la civilisation, on allait rester sans rien dire coincé sur un rocher de l’autre côté de l’Atlantique ? Non, mon gars, on n’a pas fait tout ce chemin pour enfiler des perles et chanter des cantiques. Donc, je te le dis tout net, ce qui va se passer maintenant, c’est qu’on va te parquer dans des endroits pourris où même les rats et les cafards ne voudraient loger pour rien au monde. Mais comme on n’est pas des monstres, on va te filer du tord-boyau à volonté pour que tu puisses passer tes journées à picoler pour oublier que tu t’es fait entuber dans les grandes largeurs. Des questions ? Quoi ? Si nous aussi on picole ? Bien sûr, qu’on picole, mais nous on a de bonnes raisons pour le faire. Autre chose ? Non ? Dans ce cas ramassez vos cliques, vos claques, et foutez-moi le camp d’ici ! L’aigle s’est envolé, le coyote s’est tu, le chien de prairie a regagné sa tanière, le loup ne chante plus sous la lune (non, il n’a plus envie depuis que les Indiens sont partis), mais soyez au moins certains d’une chose, farouches guerriers des sauvages plaines de l’Ouest, c’est que vos noms resteront à jamais gravés dans nos mémoires, la mienne et celle de tous ceux qui croient encore en la justice et la liberté (rires) : Cheyennes, Sioux, Pawnees, Kiowas, Crows, Shoshones, Arapahos, Comanches, Cherokees, Apaches, Têtes Plates, Nez Percés, Pieds Noirs, Culs d’Oursins et Sacs à Puces. Vos âmes pures chevauchent désormais dans les grandes plaines de l’au-delà aux côtés de Wakan Tanka, Pah, Shakouroun, du corbeau, du vieil homme et du grand lièvre. Amen, tchuss, Allahu akbar et gode save the Gouine !

Spinell (je rappelle à toutes fins utiles qu’on est en train de parler de Maniac, le film de Lustig) n’avait pas fait les années d’études nécessaires à l’obtention d’un diplôme d’historien en bonnet difforme (ce qui n’est pas mon cas non plus, je vous rassure tout de suite, pas plus que celui de Lustig ou Caroline Munro, l’actrice principale du film, et je ne parle même pas de Gail Lawrence, Kelly Piper et Sharon Mitchell), mais il avait vu suffisamment de westerns pour savoir à quoi s’en tenir au sujet des Amérindiens. Et il adorait ça, les Amérindiens. Lui-même, d’ailleurs, avec sa sale gueule de psychopathe au look ringard et aux yeux exorbités, aurait très bien pu tourner dans un western horrifique du genre The Wind, Dead in Tombstone ou The Burrowers, ou encore, dans un style plus classique, l’intemporel Jesse James contre Frankenstein. Que celles et ceux qui ne l’ont pas vu au moins une bonne douzaine de fois soient immédiatement traduits en justice et bannis à tout jamais des salles de cinéma de France et de Navarre ! En l’an de grâce 1966, Jesse James contre Frankenstein et Billy the Kid contre Dracula sortent coup sur coup. L’émotion est grande (nulle), la foule (ne) se presse (pas), la critique se déchaîne (l’ignore totalement). On doit cette prouesse technique à William Beaudine, génie méconnu dont personne n’a jamais entendu parler et tout le monde se contrefout éperdument. Baudine vient du cinéma muet, où il avait l’habitude d’enchaîner les films par treize à la douzaine. Je considère pour ma part que la réputation de films comme La Parade du rire, Le Retour de Philo Vance et l’inénarrable Gorille de Brooklyn mériterait d’être revue à la hausse, n’en déplaise aux amateurs des frères Dardenne, Xavier Dolan et Bruno Dumont. On me murmure dans l’oreillette qu’il avait prévu d’enchaîner avec Doc Holliday contre la Momie et Calamity Jane contre Jack l’Eventreur, films qu’il n’a malheureusement pas eu le temps de tourner pour des raisons de santé. Il va de soi que je me ferais un plaisir de le faire moi-même si j’avais un peu de temps et d’argent devant moi. J’appellerais l’agent de Scarlett Johansson et je lui dirais : j’ai un rôle pour Scarlett. Lui : Ah bon ? Quelle heureuse nouvelle ! Moi : Oui, le vais tourner Calamity Jane contre Jack l’Eventreur avec mon téléphone portable et j’ai pensé à elle pour le rôle principal. Je vais aussi écrire la musique du film et jouer le rôle de Jack l’Eventreur. Lui : Pas con. Moi : Et tu sais quoi ? Lui : Non. Moi : J’ai pensé à Scarlett pour le rôle de Calamity Jane. Lui : Excellente idée, je vais voir si elle est disponible. Je te rappelle dans une heure. Moi : Okay, pas de problème. Il faut savoir que lui et moi on est comme cul et chemise, même si c’est généralement plutôt moi qui fait le cul et lui la chemise. Lui, une heure plus tard comme convenu : Désolé, vieux, mais ça ne va pas être possible. Moi : Non ? Lui : Si. Elle doit tourner dans un remake de L’Exorciste signé Mike Flanagan. Moi : Elle doit jouer quoi ? Lui : On ne sait encore pas trop. Peut-être Regan MacNeil, devenue une séduisante jeune femme, qui doit à nouveau faire face à des démons bien décidés à lui faire dire des horreurs et tourner la tête à 360 degrés. Ou alors Linda, la fille que le père Karras et Sharon Spencer, la secrétaire de Chris MacNeil, ont eu en secret, et qui se transforme en succube sous l’influence de Lamashtu, une divinité summérienne qui n’est autre que la propre femme de Pazuzu. Moi, déçu : Bon, tant pis. Je vais voir si Leven Rambin est disponible. Lui : La fille du Dr Sloan dans Grey’s Anatomy ? Moi : Oui, je l’ai trouvée très bonne dans Mank de David Fincher. Lui : Elle a joué dedans ? Moi : Oui, on l’aperçoit dans une scène ou deux. Elle crève l’écran, je trouve. Lui : Mmmmouais, je suis pas sûr. Prends plutôt Amanda Seyfried, à ce moment-là. Ou Sydney Sweeney (NDLR : recordwoman du cri le plus long et l’infanticide le plus spectaculaire dans Immaculate de Michael Mohan). Moi : OK, merci, c’est pas grave. De toute façon, je ne pense pas que Scarlett aurait accepté de jouer gratos dans mon film. Lui : Je ne pense pas non plus. Mais on ne sait jamais, Scarlett est une jeune femme tout à fait imprévisible.

Bon sang, je n’en reviens pas qu’on ait pu passer en une fraction de seconde des Sioux à Sydney Sweeney, laquelle, soit dit en passant, pourrait se révéler tout à fait attractive dans un western gore où elle porterait des jeans American Eagle et jouerait la fille d’un pionnier enlevée par des Indiens cannibales. D’ailleurs, je suis à peu près certain que si Franck Zito avait croisé Sydney Sweeney dans les rues de New York, il aurait aussitôt fait une fixette sur elle et rêvé jour et nuit de lui offrir une petite coupe. Et quand je parle de petite coupe, il ne s’agit bien évidemment pas d’une petite coupe de Champagne, millésimé ou non, mais d’une coupe de cheveux. D’après Hérodote, les Scythes, farouches guerriers venus des steppes pontiques d’Anatolie, avaient pour habitude de boire le sang de leurs victimes. Grand bien leur fasse, il ne faisait pas toujours très chaud dans le secteur, les combats étaient souvent longs et éprouvants, on peut comprendre qu’ils aient eu besoin de se détendre un peu après l’effort. C’est le genre de situation où on est bien content de boire un petit verre de vin chaud, par exemple. Comme ils n’en avaient pas sous la main, une bonne rasade de sang frais faisait l’affaire. Ils auraient pu s’en tenir là, mais non. Après avoir étanché leur soif, ils tranchaient la tête de leur ennemi pour la ramener au chef qui les félicitait en leur donnant de grandes tapes dans le dos. Les rires et les chants fusaient autour du feu qui crépitait gaiment sous la voûte étoilée, dans une saine ambiance de camaraderie où tous les excès étaient autorisés. Ensuite, les femmes et les enfants étaient violés par les participants qui, il faut bien le dire, puaient méchamment de la gueule à force de boire du sang et s’empiffrer de viande crue. En guise de souvenir, il était d’usage de conserver le cuir chevelu de la victime. Une fois débarrassée de ses impuretés, la chose était utilisée en tant que serviette pour la table, ce qui ne manquait quand même pas d’une certaine classe. Et quand on était un valeureux guerrier et qu’on en avait accumulé plein au cours de ses nombreuses campagnes, il était de bon ton de les coudre ensemble pour s’en fabriquer des vêtements. Un petit gilet bien seyant, par exemple, ou une paire de moufles. Ed Gein n’aurait pas fait mieux, sauf que lui préférait déterrer des cadavres plutôt que risquer de prendre un mauvais coup sur un champ de bataille. Zito, pour sa part, se contentait plus modestement de scalper des filles pour habiller le crâne de ses mannequins, leur donner un peu de cette humanité qui leur faisait cruellement défaut.

Et enfin, last but not least comme disent nos amis anglo-saxons, incontestables spécialistes du genre, l’inoubliable Silence des agneaux de Jonathan Demme (d’après le bouquin de Thomas Harris, journaliste et écrivain de seconde zone qui a eu la riche idée de transformer un psychiatre réputé, brillant, cultivé, mélomane et gastronome, en sociopathe manipulateur, sadique et cannibale), dans lequel un certain Jame Gumb, alias Buffalo Bill, jeune homme perturbé par une identité sexuelle mal définie, lui aussi adepte des travaux de couture, enlève des femmes pour leur voler leur peau. Il utilise notamment, pour les faire monter dans son van, le coup dit «~du bras dans le plâtre~», un stratagème emprunté à Ted Bundy. Ce dernier, élevé par les parents de sa mère qui lui ont longtemps fait croire qu’ils étaient se parents et qu’elle était sa sœur, n’a jamais réussi à s’intégrer parfaitement au monde des gens normaux. D’autant moins, si l’on en croit certaines des rumeurs sournoises qui couraient dans le voisinage, que son grand-père était son père, autrement dit qu’il était le fruit pourri d’une relation incestueuse entre sa mère et son grand-père. Le coup dit «~du bras dans le plâtre~» est librement inspiré des pratiques de Ted. Ce dernier, en effet, n’avait pas de van, mais une Coccinelle jaune. Il n’avait pas non plus le bras dans le plâtre, technique un peu complexe à mettre en œuvre, mais en écharpe. Ce qui est certain, en tout cas, c’est qu’il se pointait sur les campus et se servait de cette prétendue infirmité pour s’attirer les bonnes grâces de ses victimes. Il leur demandait de l’aider à charger des objets lourds et encombrants dans le coffre de sa voiture, et dès qu’elles étaient affairées à se rendre utiles il en profitait pour les agresser lâchement par derrière. Teddy était un petit malin qui avait un gros problème avec les femmes. Il avait aussi toutes sortes de vilains penchants qui le tenaient éloigné d’une vie épanouie en société. C’est dommage, car en dépit d’un monosourcil pas très gracieux dont il n’a semble-t-il jamais jugé utile de se départir, Bundy appartenait plutôt à la catégorie des beaux gosses charismatiques qui n’ont qu’à claquer des doigts pour que les plus belles femmes leur tombent dans les bras, la tête en arrière, les yeux révulsés, le souffle court et la bouche entrouverte. Rien à voir avec la plupart de nos tueurs en série hexagonaux, qui sont vieux, moches, puent du bec et n’ont finalement pas d’autre choix que s’en prendre à des enfants en bas âge ou des handicapés. Un Emile Louis, par exemple, aurait eu beau se pointer sur des campus avec le bras en écharpe dans une Coccinelle jaune, je doute fort qu’une étudiante, même grosse et moche, ait consenti à lever le petit doigt pour l’aider à charger des objets lourds et encombrants dans le coffre de sa voiture. Son physique ingrat et son QI de mouche à merde l’ont contraint à s’attaquer à des proies sans défense, de pauvres créatures qu’il reluquait dans le rétro de son bus avant de les agresser sexuellement et les laisser pour mortes dans des endroits déserts. De la même façon, un Michel Fourniret, malin comme un renard mais vieux et surtout terriblement moche, aussi appétissant qu’un rat d’égout tombé d’une benne à ordures, devait recourir aux services de sa compagne demeurée pour embarquer des gamines dans sa camionnette et les conduire à leur mort. Un peu à la façon d’un Albert Fish, alias le Vampire de Brooklyn, l’Ogre de Wysteria, the Grey Man ou encore le Moon Maniac, cet espèce de taré made in USA, oiseau de malheur à tête de charognard qui aimait torturer les enfants et s’enfoncer des aiguilles dans le cul. Entendez par là que les Etats Unis ont eux aussi eu leur lot de types moches et déjantés dont la seule apparence aurait suffi à vous faire sauter à pieds joints dans le premier vaisseau en partance pour l’espace. On ne peut pas éternellement se flageller, même s’il est clair que les tueurs yankees ont globalement plus de classe que les nôtres. Un Ed Kemper, for exemple, deux mètres pour cent cinquante kilos de barbaque et un QI de surdoué, donne forcément à réfléchir, surtout quand on sait qu’il s’est payé le luxe, après avoir assassiné ses grands-parents, massacré un certain nombre d’étudiantes et planté des fléchettes dans la tête coupée de sa mère, de se rendre gentiment à la police. On se dit qu’il ne faut finalement pas grand-chose pour que l’esprit humain se mette à dysfonctionner dans les grandes largeurs. Parce que si tel n’est pas le cas, si dysfonctionnement il n’y a pas, alors on est en droit de se poser de sérieuses questions sur l’avenir de notre espèce.

\textsc{Greg} : On se croirait dans Psychose.

\textsc{Moi} : En plus exotique.

Greg, étreignant nerveusement la crosse du Bersa Thunder 380 CC dissimulé sous sa veste : J’ai peur.

\textsc{Moi} : Ne t’en fais pas, je m’occupe de tout.

\textsc{Lui} : C’est bien ce qui me fait peur.

Comme à peu près tout ce qui se trouvait dans l’hôtel, à part les animaux empaillés, Greg et moi, le type de la Réception appartenait à la catégorie de ce qu’on appelle pudiquement les «~personnes de couleur~». Greg, qui était en train de glisser lentement dans l’alcoolisme dit «~mondain~», n’était plus très loin d’appartenir à la catégorie très convoitée des personnes de couleur rouge, comme les Amérindiens, bien sûr (encore que cela ne concerne originellement qu’une très petite partie d’entre eux qui avaient pour habitude de s’enduire la peau d’ocre rouge), mais aussi la grande communauté des victimes de coups de soleil (qui ne constituent pas une ethnie à proprement parler), et surtout la seule et unique personne de Donald Trump (le fameux «~rouge Trump~» qu’on retrouve non seulement sur le visage de l’intéressé, mais aussi les «~power ties~», casquettes et autres produits dérivés en vente libre sur le Trump Store).

Sur le revers de sa veste, le Réceptionniste portait un badge sur lequel était inscrit «~Dumo~». J’en ai déduit qu’il s’agissait de son prénom, lequel, à une lettre près (un B en l’occurrence), aurait été celui du sympathique éléphanteau des studios Disney, animal avec lequel, mis à part un certain embonpoint, il ne partageait pas de ressemblance directe. En effet, il avait des oreilles minuscules, à peine plus développées que des branchies, et un nez court et épaté, comme si on avait tapé dessus pendant des heures et des heures avec un rouleau à pâtisserie, un maillet ou tout autre objet contondant, en bois de préférence, autrement dit un organe aux antipodes de cette chose longue et majestueuse qu’on appelle une trompe.

Personne n’a jamais vu un œuf avec des cheveux (sauf en Chine, bien sûr, terre de tous les mystères y compris les plus absurdes et incongrus, je pense notamment à cette manie qu’ils ont de bouffer des œufs bouillis des heures durant dans de la pisse de collégien, pratique tout de même un peu étrange dont on peine à entrevoir clairement la finalité, sauf en Chine, disais-je, où un villageois des environs de Quanzhou, dans le sud-est du pays, a découvert un œuf couvert de poil dans le ventre d’une truie, curieux objet dont la valeur marchande, s’il s’agit bien de la concrétion attendue, pourrait avoisiner le million de dollars), mais on s’attend à en trouver, ne serait-ce quelques uns, sur le crâne d’un être humain. Même celui d’un Chinois chauve, un bonze, par exemple, ou un vieil herboriste du Sichuan, sur lequel on finit toujours par en dénicher un qui traîne au détour d’un pli ou une oreille. Et pourtant, aussi incroyable que cela puisse paraître, je vous fiche mon billet de longues heures de recherche, à la loupe ou au microscope, n’auraient pas révélé la présence du moindre élément de ce type sur le crâne de Dumo, entièrement revêtu d’une substance laquée plus proche de la boule de billard que du cuir chevelu.

Par ailleurs, je ne sais pas si vous avez déjà essayé de pianoter sur le clavier d’un ordinateur avec des boudins créoles à la place des doigts (ou des saucisses de Toulouse), mais même si vous ne l’avez jamais fait, je suis certain que vous n’aurez aucun mal à vous représenter la difficulté de la tâche. Et compatir du même coup, en bon chrétien, bouddhiste ou musulman que vous êtes (je milite en effet, à défaut de l’abandon pur et simple de toute espèce de religion, hormis peut-être le vaudou, la santeria et accessoirement le Temple Satanique de Greaves \& Jarry à Salem, pour une approche plus œcuménique, mystique, néo-romantique et libérale de la Foi, enfin délivrée de ses chaînes et autres clous christiques plantés dans les chairs tuméfiées de l’Espérance et la Rédemption), au calvaire de Dumo, qui se trouvait exactement dans la très pénible situation que je viens d’essayer, au travers de métaphores charcutières à mon sens assez pertinentes, de décrire aussi fidèlement que possible. Car en effet, le pauvre vieux s’échinait à pianoter sur un clavier dont les touches étaient dix fois trop petites pour ses extrémités digitales. Forcément, la virtuosité s’en trouvait grandement altérée, sa bonne humeur également, et l’ampleur de la tâche mobilisait l’entièreté de ses facultés intellectuelles. Son état de concentration, extrême, n’ouvrait aucune brèche sur le monde extérieur. On aurait pu forcer l’entrée de l’hôtel au bulldozer sans qu’il s’en aperçoive, et un troupeau de buffles lancés à pleine vitesse aurait pu faire de même sans obtenir davantage de résultat. Des gouttes de sueur perlaient sur son front, avant de descendre le long de ses joues, tels des petits animaux rampants laissant une trace humide dans leur sillage, et s’écraser lourdement en contrebas, sur le clavier de l’ordinateur, ce qui obligeait Dumo à l’essuyer sans arrêt avec des mouchoirs en papier, les mêmes (enfin d’autres, sortis des mêmes paquets) dont il se servait pour s’éponger le visage aussi souvent que possible avant de les jeter sans précaution dans la corbeille sise à proximité (il ne prenait pas le temps de viser, autant dire que les trois quarts atterrissaient à côté). De la même façon que certains se rongent les ongles, manipulent un objet quelconque, remuent frénétiquement telle ou telle partie de leur corps (le plus souvent la jambe, comme s’ils étaient pressés de s’enfuir), ou se triturent fébrilement une mèche de cheveux (toujours la même, comme si elle était rattachée à certaines terminaisons nerveuses déterminantes pour le succès de l’opération), lui se caressait machinalement le haut du crâne avec le plat de la main. Ce faisant, il se retrouvait avec une main pleine de sueur qu’il devait à son tour essuyer (avec un de ces mouchoirs en papier dont il faisait un usage compulsif, comme je l’ai indiqué, mais aussi très souvent avec certaines parties parmi les plus accessibles de ses propres vêtements, avec les conséquences désastreuses que l’on imagine en termes d’apparence et de propreté), avant de la tremper à nouveau dans la sueur de son crâne, l’essuyer à nouveau, et ainsi de suite, devenant ainsi la proie du jeu de dupe dont il était le principal artisan.

N’entrevoyant aucune issue favorable à cette effroyable tragédie, je me suis vu, l’espace d’un court instant, dégainer Manu et tirer une balle dans la tête de ce pauvre Dumo.

Et me suis aussitôt ravisé, bien entendu, conscient que le remède aurait quand même été quelque peu disproportionné. J’ajoute que des témoins se trouvaient sur les lieux, assez peu nombreux, certes, mais non moins vigilants, qui se seraient fait une joie de témoigner en ma défaveur. Quand on est, comme moi, quelqu’un dont l’essentiel de l’activité consiste à envoyer ses concitoyens croupir derrière les barreaux d’une cellule, il ne fait pas bon se retrouver dans la même situation. Tous n’attendent qu’une seule chose, en dehors de la libération conditionnelle ou la réduction de peine : vous voir débarquer pour vous mettre en pièces. Le monde carcéral est une poubelle dans laquelle on entasse les rebuts de la société, ses déjections, les corps étrangers qu’on extirpe de son épiderme. On aimerait tirer la chasse une bonne fois pour toutes, mais l’éthique commande de les maintenir en vie, leur offrir le gîte et le couvert avant de les relâcher dans la nature avec l’espoir que ce séjour à l’ombre leur aura rafraîchi les idées. Mais si vous placez des voleurs, des violeurs et des assassins dans le même périmètre, qu’est-ce qui va se passer à votre avis ? Eh bien soit ils s’entretuent, ce que tout le monde souhaite intérieurement sans oser publiquement l’avouer, soit ils se serrent les coudes entre confrères et ressortent de là gonflés à bloc comme jamais, bien décidés à unir leurs forces pour prendre leur revanche sur la société, lui faire payer au centuple le préjudice subi. En enfer, les enfants de chœur virent leur cuti. Rien de tel, pour former des criminels endurcis, que de leur tanner le cuir en prison. Le raisin de la haine fermente pour engendrer le nectar du mal. L’ennui, c’est qu’on ne sait pas quoi en faire, et que les entretenir ad vitam æternam finit par coûter cher à la société. Imaginez un instant que tout délinquant, aussi jeune et minime soit-il, soit éliminé physiquement au moindre faux pas, réduit en cendres et rayé des cadres de la civilisation, en partant du principe que les chances qu’il s’amende sont nulles, et que même si par miracle il y parvenait, le jeu n’en vaut de toute façon pas la chandelle. Autant former tout de suite des gens instruits et compétents plutôt que se casser le cul (et la tirelire) à essayer de rattraper de justesse des repris de justice. Pourquoi, en gros, ne pas travailler à l’américaine ? On sait que la police ne peut pas être partout, en permanence à tous les coins de rues, l’agent du maintien de l’ordre n’a pas quatre bras et encore moins le don d’ubiquité, alors pourquoi ne pas permettre aux honnêtes gens de faire eux-mêmes le sale boulot. Vous travaillez d’arrache-pied pour gagner honnêtement votre vie, vous revenez tranquillement du cinéma ou du restaurant après une dure journée de labeur, comme les parents de Bruce Wayne, et vous vous faites sauvagement agresser par un junkie nauséabond qui en veut à votre portefeuille et aux bijoux de votre femme. Bienvenue à Gotham City. Vous attendez quoi, qu’il vous assassine froidement dans une ruelle sombre et humide ? Non, bien sûr : vous sortez votre flingue (vous ne sortez jamais sans lui, il est votre ami pour la vie)et faites sauter le caisson du rat d’égout sans autre forme de procès (je sais bien qu’il faut que les juges et les avocats gagnent leur croûte, mais ce serait bien qu’ils ne le fassent pas au détriment de notre santé). Et hop, vous faites d’une paire deux couilles : non seulement vous sauvez votre portefeuille et accessoirement votre femme et (surtout) ses bijoux (qui vous ont coûté un bras, parce que vous, contrairement à certains qui ne s’embarrassent pas de principes, vous payez vos dettes rubis sur l’ongle), mais en plus vous participez gracieusement au ramassage des ordures ménagères, lesquelles, on le sait, sont de plus en plus envahissantes dans nos cités laissées à l’abandon. D’accord, vous êtes une grosse merde et votre conscience vous gratte un peu le cul pendant un jour ou deux, mais au moins vous assumez vos responsabilités civiques et méritez pleinement votre qualificatif de citoyen modèle (et pouvez toujours aller vous confesser au curé de la paroisse qui se fera un plaisir de vous donner l’absolution). Fini de faire le tri, on ratisse large et on fait le ménage à la louche, tout doit disparaître. Même chose pour le petit voyou qui vole une barre de céréales chez l’épicier du coin : négatif, votre Horreur, je plaide coupable, mon client est une ordure et on ne va pas s’amuser à lui taper gentiment sur les doigts alors qu’on sait pertinemment qu’il va recommencer à la première occasion. On le tue tout de suite, on gagne du temps, de l’argent, et tout le monde est content (sauf lui et sa famille, bien sûr, mais ça on s’est fout, c’est de leur faute, ils n’avaient qu’à mieux l’éduquer, et d’ailleurs on va éradiquer le nid dans la foulée). Je ne sais pas, moi. J’essaie juste de trouver des solutions pour éviter que nos femmes et nos enfants se fassent tripoter dans les transports en commun, nos petits vieux dépouiller par des individus peu scrupuleux (alors que les croquemorts s’en chargent très bien en toute légalité). Oui, d’accord, pour le petit voyou qui a volé une barre de céréales chez l’épicier du coin, on n’est peut-être pas obligé de le tuer tout de suite. On peut commencer par lui couper la main qui a commis le larcin, histoire de lui faire comprendre que ce n’est pas parce qu’on n’a pas d’argent pour s’acheter à manger qu’il faut se croire autorisé à dérober le bien d’autrui. Dans ce cas-là, on ne mange pas, voilà tout, ou alors, si on a vraiment très faim, on se trouve un travail honnête pour subvenir à ses besoins. Ce n’est pas le travail qui manque, vous savez. Bon, il est certain que compte tenu de son niveau d’études, le petit voyou en question ne va pas pouvoir prétendre à un salaire mirobolant. Mais il gagnera tout de même suffisamment pour s’acheter honnêtement une jolie petite pomme, et il la mangera avec d’autant plus de satisfaction qu’elle aura de la valeur pour lui. Si elle représente la moitié de son salaire, par exemple, vous pouvez être certain qu’il la mangera avec énormément de satisfaction, en savourant chaque bouchée jusqu’à l’extase. Alors que le riche, lui, le pauvre, ne prend plus aucun plaisir à manger une pomme, ou alors seulement si elle recouverte d’une feuille d’or et si on a pris soin de remplacer les pépins par des diamants. Comme le pauvre, le riche ne prend plaisir à manger une chose que si elle représente une partie non négligeable de son salaire, ce qui signifie qu’il n’est pas évident de se nourrir quand on a la malchance de gagner cent ou cent cinquante mille euros par mois. Même la truffe blanche ou le caviar à cinq mille balles le kilo, il faut déjà en avaler pas mal avant de commencer à ressentir un semblant de bien-être, une vague sensation de plaisir. D’autre part, le riche ne prend plaisir à être riche que s’il en permanence le référent de la misère sous les yeux, de même que le pauvre ne prend plaisir à l’être que s’il a en permanence le référent de l’abondance et la richesse sous les yeux. Et c’est exactement ce qu’on s’efforce de lui fournir, avec la presse people, les jeux débiles où des animateurs blindés de fric se foutent ouvertement de sa gueule, les centres commerciaux rutilants, les offres soi-disant spéciales et les supermarchés qui créent l’illusion de l’abondance en débordant de produits identiques dont seule la présentation diffère. Il se repaît des faits et gestes des riches et des puissants, se saigne aux quatre veines pour assister à des matchs de foot où les joueurs gagnent trois ou quatre mille fois le SMIC, et dans le même temps se plaint que la vie est trop chère et fonce sur les flics qui tentent de le contrôler alors qu’il roule bourré à cent à l’heure dans les rues de la ville. Il estime sans doute, en compensation de la vie de merde qu’il a courageusement accepté d’assumer, avoir droit à une certaine tolérance de la part de Nation reconnaissante. Le problème, c’est que la Nation ne lui reconnaît que le droit de se taire, estimant pour sa part que lui accorder le droit de vivre est déjà un privilège inestimable (mais un mal nécessaire à la survie du régime). Il se comporte d’autant plus mal qu’il n’a pas grand-chose à perdre, à part sa vie, à laquelle il ne tient pas plus que ça, ou sa liberté, qui ne lui sert pas à grand-chose s’il ne peut pas rouler bourré, insulter les gens et dégrader le bien d’autrui. On pourrait croire, quand il s’émerveille devant une voiture de luxe qu’il ne pourra jamais s’offrir, une star qu’il ne pourra jamais approcher, une actrice somptueuse qu’il ne pourra jamais soumettre à ses exigences sexuelles (et heureusement pour elle, la pauvre), que c’est de la convoitise. Mais non, ça le rend vraiment heureux et fier d’être pauvre, d’appartenir à la noble corporation des traîne-savates alcooliques et édentés (les fameux «~sans-dents~» du président Hollande, politicien médiocre mais humoriste réputé), et il ne changerait de catégorie sociale pour rien au monde. Si on lui donnait dix millions de dollars en petites coupures usagées dans un sac poubelle, il ne les prendrait pas. Enfin si, il les prendrait, mais il ne saurait pas quoi en faire, le pauvre. Autant donner de la confiture à un cochon. Il continuerait à boire de la piquette, mais il en boirait dix fois plus compte tenu de son budget illimité, et son espérance de vie, déjà aussi mince qu’une feuille de papier à cigarette (cigarettes qu’il fume par paquets entiers, de même qu’il s’adonne sans retenue à tous les vices à sa disposition), s’en trouverait réduite d’autant. Il essayerait bien de le placer à droite à gauche, sur les conseils avisés de quelque escroc en col blanc, de s’acheter une grosse voiture, une grosse bagnole, une grosse maison avec une grosse femme et des gros enfants, mais il se retrouverait vite entouré d’une nuée de parasites, contraint de se séparer de tous ses «~amis~» qui ne seraient en réalité qu’une bande de profiteurs de la pire espèce, totalement dénués de scrupules, et ne pourrait espérer aucun salut de la part des riches, lesquels continueraient éternellement à le considérer comme un corps étranger et opposer une fin de non-recevoir à ses tentatives de rapprochement. Sa vie, qui était déjà un pur calvaire, tournerait au cauchemar, car il se verrait contraint, après que sa femme ait demandé le divorce et tenté de le soulager de la moitié de sa fortune, de terminer ses jours dans la solitude la plus effroyable, passer ses journées à picoler des grands crus au bord de sa piscine et rouler sans but sur les hauteurs de Cannes au volant de sa Maserati flambant neuve. Quelle horreur ! La raison pour laquelle les riches, les vrais, les de pères en fils depuis des générations, n’ont que des amis riches, qui ne sont pas des amis, du reste, car il est bien entendu que le riche n’a pas d’ami, mais seulement des amis riches, autrement dit des amis qui n’en sont pas, au mieux des compagnons d’infortune et de perdition, hermétiques à la compassion mais rompus à toutes les subtilités du secret bancaire et la fiscalité paradisiaque, c’est précisément que les riches, qui le sont déjà par définition, et ce depuis longtemps, le plus longtemps possible de façon à ne laisser planer aucun doute sur le sujet, n’ont à priori aucune raison d’abuser d’eux. Le riche et le pauvre doivent être maintenus à bonne distance l’un de l’autre, ni trop proche ni trop éloignée, de sorte qu’ils puissent continuer à s’observer dans l’irrespect mutuel et l’aversion réciproque. C’est sur cet équilibre instable que repose l’avenir de la civilisation occidentale, dont on peut craindre, effectivement, qu’elle ne s’écroule à tout moment. Et j’ai envie de dire à mes amis développés, tant sur le plan de la morphologie, la technologie et la culture, très supérieurs à la moyenne dans leur approche de l’existence, mélomanes d’exception et lecteurs assidus des plus grands philosophes, qui, soudain pris de court par la déferlante qui menace de les engloutir, cherchent désespérément mon regard dans les ténèbres pour se raccrocher à quelque chose de profondément humain, sensible et empathique : oui, mes frères, c’est mal barré, mais tout n’est pas perdu. Grâce à l’oncle Sam, l’Armée rouge et le Soviet suprême, et même sans l’appui du gnome coréen à tête de poupon maléfique qui nous déteste cordialement, j’affirme sans une once de tremblement dans la voix que nous disposons de suffisamment d’ogives nucléaires pour maintenir à distance les hordes de pauvres et autres déshérités qui se pressent à nos portes. Disparaissez Marcheurs blancs, émissaires de la Mort, et laissez-nous jouir en toute liberté de nos Livrets A, LEP et PEL. Foutez-nous la paix et laissez-nous toucher en paix nos allocations familiales, revenus de solidarité active et autres primes de Noël. Non, vous ne toucherez pas à nos comptes en Suisse, pas plus qu’à nos épouses et encore moins nos filles, et n’espérez pas tremper un jour vos fesses ramollies dans nos piscines, vos mouillettes impies dans de jaune de nos œufs, ni vos lèvres perfides dans le champagne de nos coupes.

J’ai chargé Greg de faire le guet.

Comment j’ai fait ça ?

Rien de plus simple : je l’ai pris solennellement à part, dans un renfoncement de la cage d’escalier digne d’un palais des mille et une nuits, et, après lui avoir fait prêter allégeance à la cause et jurer fidélité à ma personne, je lui ai confié la lourde responsabilité de surveiller nos arrières, autrement dit me signaler immédiatement tout fait ou geste qui lui semblerait un tant soit peu suspect.

Quel grand moment d’émotion ! Je le vois encore fondre en larmes, tel un gros bébé au visage anguleux et la ceinture abdominale légèrement relâchée, l’exercice n’étant pas son fort, et s’écrouler comme une merde à mes pieds, bouleversé par l’immense honneur que je lui accordais. Je ne voudrais pas entrer dans les détails, mais il avait toujours souffert d’un manque de reconnaissance total de la part de son père, qui n’avait cessé de le dénigrer et lui faire sentir que, quoi qu’il fasse, il ne parviendrait jamais à se hisser à la hauteur de ses exigences. C’est dur pour fils de se faire traiter comme une sous-merde par son propre géniteur, celui auquel on pense devoir la vie (alors qu’il n’y est pour rien, en fait), c’est pas terrible pour la confiance, et il faut souvent des années de psychanalyse pour s’en remettre, et encore jamais complètement. Ce type était clairement une ordure qui n’aurait jamais dû avoir d’enfant. Seulement voilà, la nature se fiche que les ordures, assassins, nazis, trafiquants de drogue, politiciens corrompus et autres, se reproduisent au même titre que les honnêtes gens. Vous pouvez être le type le plus dégénéré qui soit, vous finirez toujours par trouver une fille aussi déglinguée que vous qui sera ravie d’offrir son ventre à vos ébats. La nature ne fait pas de différence entre ses enfants, tous ont les mêmes chances de se reproduire, et elle fait en sorte que chacun trouve chaussure à son pied, même s’il pue des pieds et s’il s’agit d’une vieille godasse trouée sortie d’une poubelle. Elle a donné à tous le pouvoir de niquer, le mode d’emploi et les outils pour le faire. Et elle a fait en sorte, histoire d’être bien sûre de ne pas rater son coup, que tous ne pensent qu’à ça vingt-quatre heures sur vingt-quatre, sans distinction de revenus, quotient intellectuel, familial ou autre. Et sans se balader avec la trique au vent vingt-quatre heures sur vingt-quatre, ce qui pourrait se révéler rapidement pénible et handicapant. D’où le coup de l’engin à géométrie variable, facile à ranger discrètement dans un fond de culotte pour ne pas donner l’alerte en permanence, créer un état de panique générale impossible à endiguer. La femme, par contre, avec ses seins non rétractables, dispose d’un handicap certain en la matière. Facile à repérer, il lui est d’autant plus difficile d’échapper à la concupiscence de ses concitoyens. On peut dire que la nature lui a joué un tour pendable. C’est la fable du pot de miel et du saumon. Imaginez qu’on vous dise : Tu vois cette forêt ? Elle est infestée d’ours bruns, des gros qui n’ont qu’une seule idée en tête : bouffer. Et maintenant, tu vois cette table ? Dessus, il y a un pot de miel et un saumon. Tu choisis l’un ou l’autre, ce que tu préfères, tu le poses en équilibre sur ta tête, et tu vas te balader au milieu des ours. Il est clair que vous avez toutes les chances de vous faire arracher la tête à tous les coins d’arbres. Depuis, la vie de la femme est un véritable cauchemar, un parcours du combattant qui force l’admiration. À moins de s’emmailloter sous trente ou quarante couches de fringues, de dissimuler son visage et l’abondante chevelure soyeuse et parfumée dont elle est naturellement pourvue, elle est condamnée à vivre dans un monde où ses chances de survie, sexuellement parlant, flirtent avec le zéro absolu. Les Arabes, qui l’obligent à se couvrir de la tête aux pieds avant de mettre le nez dehors, ne laissant la place qu’à une ouverture grillagée ou une fente minuscule pour lui permettre de respirer et voir sans être vue, sont en fait d’ardents protecteurs de la femme contre leur propre tyrannie. Non seulement ils reconnaissent implicitement qu’ils sont tous des obsédés sexuels incapables de contrôler leurs pulsions, ce qui est déjà une belle preuve de courage et d’abnégation, de recul sur soi (même s’il ne s’accompagne pas nécessairement d’un sens de l’humour à toute épreuve), mais ils lui indiquent (de façon assez autoritaire, il est vrai) le moyen de se prémunir contre leurs ardeurs, sachant qu’eux-mêmes, à moins de se crever les yeux, sont incapables d’avoir une approche saine et dépassionnée de la plastique féminine. Un banc de morues ne peut pas évoluer sans protection au milieu des récifs. Les requins sont partout, affamés (ou pas, du reste, car le requin a toujours faim, qu’il ait déjà mangé ou non), et la moindre écaille qui scintille à l’horizon les plonge aussitôt dans un état de frénésie proche de la démence. Même s’il n’y a pas énormément de requins dans le golfe Persique, et si la morue n’est pas le plat favori des Iraniens, nos amis Arabes, nobles descendants du royaume de Saba et des Omeyyades, ont compris la nécessité de protéger la femme de la violence de leur désir. Pas question pour eux d’envoyer des filles en hot-pants de cheerleaders se trémousser au milieu d’une caravane de bédouins qui viennent de traverser le désert à dos de chameau, des gens dont les burnes, sous l’effet de la chaleur et des chocs répétés, ont atteint un tel degré de dilatation qu’elles sont susceptibles d’exploser à tout moment, comme les graines du cornichon d’âne ou de l’herbe à Robert qu’on effleure par inadvertance. Ce n’est que dans la plus stricte intimité, à l’abri des regards concupiscents de la meute en chaleur, que la femme peut se dévoiler. L’efficacité est indéniable, mais on peut critiquer la méthode, qui est aux antipodes de notre façon de procéder. En effet, nous avons décidé de traiter le mal par le mal. Puisque nous sommes tous des gros porcs lubriques incapables de contrôler nos pulsions, autant y aller à fond et affronter nos démons à bras-le-corps. Le danger, en dissimulant la femme, c’est que cette occultation ne fasse qu’exacerber les passions et entraîner l’imaginaire vers des horizons aussi dangereux qu’insoupçonnés. Si vous sentez qu’on vous cache quelque chose, ou cherche à vous le cacher, votre envie de savoir est d’autant plus vive, intense, et peut rapidement virer à l’obsession. Vous imaginez des choses qui ne sont pas conformes à la réalité et courez droit à la déception, la frustration. Sauf, bien sûr, si vous vous gardez de toute extrapolation intempestive et conservez intact votre sens du merveilleux. Mais cela n’est pas donné à tout le monde. Pragmatiques, nous avons décidé qu’il valait mieux jouer la carte de la transparence : aux premiers rayons de soleil (oui, il ne fait pas toujours une chaleur à crever dans nos contrées) sur fond de ciel bleu dégagé et d’oiseaux qui chantent dans les arbres à la végétation naissante, la femme, à l’instar de l’homme, pourra elle aussi se trimballer à moitié nue dans les rues de la cité, confiante dans le fait que nous saurons garder nos distances et éviter regards et réflexions salaces concernant tout ou partie de son anatomie. Ne sommes-nous pas des gens civilisés, qui n’avons nul besoin de recourir à de grossiers stratagèmes pour conserver élégance et dignité ? Après tout, les Africains arrivent très bien, dans leurs contrées lointaines, à vivre en permanence entourés de femmes largement dévêtues sans développer aucun ressentiment particulier. Tout se passe très bien pour eux, ils s’accouplent quand bon leur semble, de la façon qui leur plaît, et personne ne trouve rien à redire à la situation. C’est bien la preuve, si on n’est pas complètement con, qu’on n’est pas obligé de couvrir sa femme de la tête aux pieds pour éviter les problèmes. De toute façon, quoi qu’on fasse et qui qu’on fesse, il y aura toujours des emmerdeurs pour s’affranchir des règles et traverser en dehors des clous.

J’ai relevé Greg, l’ai serré fort dans mes bras, tel un père qui verrait son enfant pour la dernière fois, ou une maman ours qui serrerait dans ses bras son bébé ours avant de le laisser partir seul dans les profondeurs de la forêt infestée de méchants chasseurs alcooliques et réactionnaires, puis me suis dirigé d’un pas lourd mais décidé vers la Réception.

Nous étions, je le rappelle, en territoire ennemi. Le coup pouvait venir de n’importe où, n’importe quand, et se présenter sous n’importe quelle forme, y compris la plus innocente, comme ces enfants qui surgissent dans le djebel et s’avancent vers vous avec une bombe à la ceinture, le sourire aux lèvres et les bras chargés de dattes. Je te raconte pas le méchoui, sidi. Alors oui, je sais ce que vous allez me dire : pas la peine faire d’en faire des caisses, de sombrer dans le délire paranoïaque des références au djihad et à l’apocalypse selon Saint-Maclou ! On se calme et on boit frais à Saint-Tropez, comme disait le regretté Max Pécas, petit producteur de navets connu pour la qualité de ses bulbes (La Baie du désir, Je suis une nymphomane, Marche pas sur mes lacets, Mieux vaut être riche et bien portant que fauché et mal foutu, etc). Vous êtes juste deux touristes mal réveillés dans le hall d’un hôtel de luxe, en plein jour, armés jusqu’aux dents qui plus est, on ne voit donc pas très bien ce qui pourrait vous arriver, même s’il est clair que vous n’êtes pas à proprement parler les bienvenus dans le contexte, compte tenu de la connerie qui vous anime et la prétention abyssale qui vous caractérise, sans parler de la couleur de peau blafarde et grassouillette d’asticot avec laquelle vous avez eu le toupet de venir au monde. Certes, ce n’est pas de gaité de cœur qu’on vous voit fouler le plancher ancestral de cet édifice prestigieux, mais ce n’est pas comme si vous étiez deux chérubins prépubères égarés dans un congrès de pédophiles, ou, pire encore, deux Juifs handicapés, homosexuels et communistes ayant atterri, suite à une cascade d’événements tous plus rocambolesques les uns que les autres, en plein milieu d’une réunion de nostalgiques du Troisième Reich.

C’est donc plein d’espoir et la poitrine gonflée d’orgueil que Greg, rasséréné par les propos lénifiants (j’aimerais, au risque d’altérer la fluidité de la narration, avoir une pensée émue pour le jeune adulte dont le niveau de vocabulaire~-- et de culture générale~-- n’excède pas celui d’un enfant de cinq ans d’il y a cinquante ans et qui, s’accrochant à chaque phrase tel un naufragé à une planche de bois vermoulu, essaie malgré tout courageusement de lire ce livre, au demeurant moins hermétique que Finnegans Wake, Le Roi pâle ou La Maison des feuilles, et lui dire que non, tout n’est pas perdu, à condition bien sûr qu’il cesse immédiatement d’ingurgiter des torrents de soupe indigeste avec des gros morceaux de caca qui pue à l’intérieur sur X, TikTok, WhatsApp, Instagram et les autres, sans quoi il parviendra à un niveau~-- level, je traduis en anglais par charité chrétienne, pour l’aider à ne pas décrocher totalement~-- de décérébration si stratosphérique que même ses propres enfants, qu’il aura bien évidemment achetés en solde sur Internet et renvoyés plusieurs fois pour vice de forme, au risque de se retrouver avec des articles ne correspondant plus du tout à ses attentes, inexistantes de toute façon, ne le reconnaîtront plus) que je venais (je vous avais prévenu qu’on allait perdre en lisibilité) de lui déverser à flux tendu dans le creux de l’oreille, s’est élancé, tel un jeune faon qui s’éveille dans la forêt enchantée de ses ancêtres, bercé de mille senteurs et sonorités à la fois étranges et familières qui l’émerveillent autant qu’elles l’interloquent, sur le sentier lumineux (rien à voir avec le parti du camarade Gonzalo, je vous rassure tout de suite) de l’épanouissement personnel, respirant à pleins poumons l’air nouveau de la confiance retrouvée.

Après quoi je me suis présenté à l’accueil, arborant le sourire satisfait du voyageur qui vient d’effectuer une confortable traversée de l’Atlantique en jet privé, entouré d’un essaim d’hôtesses vibrionnantes laissant flotter dans leur sillage un subtil bouquet d’exhalaisons paradisiaques, et j’ai attendu que Dumo, toujours très affairé à son travail, daigne s’intéresser à moi.

Ce qu’il n’a pas fait, manquant à tous ses devoirs avec une impudence désarmante de naturel.

Mais aussi terriblement irritante.

Du coup, ma pression artérielle est montée en flèche.

Et il m’arrive, quand ma pression artérielle monte en flèche, de faire des choses que je ne fais pas en temps normal, ou nettement

moins, comme par exemple tatouer mes initiales au fer rouge sur la poitrine des gens, leur arracher les yeux et les remplacer par des balles de ping pong, leur couper le nez, la langue et les oreilles avec des ciseaux rouillés, ou encore leur ouvrir la boîte crânienne à l’emporte-pièce pour leur siroter la cervelle à la paille.

Voilà comment je me suis retrouvé dans un état d’exaspération que je ne me souvenais plus avoir atteint depuis le jour où Zarina, après avoir englouti un nombre indéterminé de Girofliers du Clair de Lune (cocktail dévastateur à base d’amaretto, grappa, sirop de framboise et clou de girofle), m’avait sauvagement agressé sous le prétexte fallacieux que j’aurais, je dis bien «~aurais~», car je ne me souviens absolument pas de l’avoir fait, approché sa sœur d’un peu trop près. Accusation profondément injuste, perfide et mensongère, dont je peine aujourd’hui encore, longtemps après les faits, à me relever (mais ça va, hein, j’ai suivi une thérapie assez musclée à base d’électrochocs et produits stupéfiants qui me permet enfin d’entrevoir le bout du tunnel, même si je passe encore par des phases de mélancolie semblables à d’épaisses nappes de brume qui colle à la peau et s’insinue au plus profond de votre être). Car tenez-vous bien, à en croire les dires de l’écervelée, nos lèvres (les miennes et celles de Tosca, sa sœur jumelle, en tout point semblables aux siennes, hormis peut-être une imperceptible inflexion à la commissure, dessinant, en de très rares occasions, une microscopique et fugace fossette en sortie de joue) se seraient pratiquement abouchées au cours d’un échange passionné concernant les relations pour le moins sulfureuses du roi Zog d’Albanie avec une certaine Tatiana Visirova, subtil mélange d’épices chinoises, roumaines et russo-polonaises savamment assemblées en une seule et même petite personne passablement dévergondée, assez peu douée pour les études, mais suffisamment à l’aise pour se produire dans le plus simple appareil sur la scène des Folies-Bergère et déclencher des tonnerres d’applaudissements à chacune de ses apparitions.

J’ai pris sur moi pour ne pas sortir Manu et exploser le crâne bosselé du crétin des Alpes qui me faisait face.

Vers la fin du XVIIIe siècle, à l’aube des sports d’hiver et la révolution technologique, les premiers touristes parviennent au sommet des monts alpins et découvrent avec stupeur qu’ils ne sont pas seuls. D’étranges créatures, simiesques, contrefaites, et affligées pour la plupart de goitres spectaculaires, vivent dans ces contrées reculées. Il devient rapidement évident que ces créatures, en dépit de leurs malformations et leur intelligence limitée, ne représentent aucun danger particulier. De retour en ville, ils écument les salons pour faire part de leur découverte, ne lésinant pas sur les détails les plus atroces, n’hésitant pas à forcer le trait pour les besoins de la narration. Le bourgeois frissonne, les jeunes filles se pâment et tombent inanimées dans les bras des beaux parleurs, leurs lèvres entrouvertes exhalant le souffle rauque des désirs inassouvis. En secret, elles rêvent d’étreintes avec ces monstres innommables qu’on imagine membrés comme des taureaux, portant jusqu’aux genoux de lourdes paires de burnes couvertes d’une épaisse fourrure. Qui sont ces humanoïdes ? Sont-ils les descendants des Nains d’Erebor, dont ils ont la taille réduite, les traits grossiers, la silhouette massive et la force de cheval ? Sont-ils consanguins, cannibales, sodomites ? Quelles sont leurs idoles ? S’adonnent-ils à des rites païens placés sous le signe de la nécrophilie et la zoophilie ? On envisage un temps de les exterminer, ou les réduire à l’esclavage, les faire travailler à des tâches ingrates comme des bêtes de somme, puis on choisit de les parquer dans des sanatoriums pour les étudier à loisir et tenter de percer le mystère de leur existence et leur laideur extrême. L’armée envisage un temps de transformer les mâles en super-combattants pour les envoyer sur le front en première ligne en cas de conflit. Leur aspect menaçant devrait suffire à pousser l’ennemi à rebrousser chemin sans demander son reste. Quelques années plus tard, la science livre son verdict : non, il ne s’agit pas d’une espèce d’hommes semi-préhistoriques ayant miraculeusement échappé à la civilisation, et encore moins de créatures fantastiques sorties du ventre de villageoises engrossées par des trolls, des elfes ou des lutins, mais de pauvres hères souffrant de la thyroïde en raison du manque d’iode. Peu festif mais vrai. Grâce aux progrès de la médecine, le crétin des Alpes a aujourd’hui disparu, même s’il y a toujours autant de crétins dans les Alpes que partout ailleurs.


\textsc{Moi} : Pourquoi j’ai l’impression que vous vous foutez de ma gueule ?

Lui, tout sourire : Je ne me permettrais pas, monsieur le commissaire.

Une des choses qui m’auraient fait le plus plaisir, à ce moment-là, aurait été d’être pilote de chasse, que des touristes français (venus massacrer des lions, des éléphants et des rhinos) soient sauvagement assassinés et décapités par des bushmen dans le désert du Kalahari, et que la guerre soit déclarée à la Namibie par un président de la République aux abois, en totale perdition, disposant d’une cote de popularité frisant le ridicule, suspecté de corruption passive, fraude fiscale, trafic de drogue, proxénétisme aggravé, prise illégale d’intérêts, détournement de fonds publics et association de malfaiteurs, prêt à tout, y compris rayer un pays de la carte, pour redorer son blason aux frais de la princesse. J’aurais quitté les lieux sans dire un mot, rejoint ma base, serais monté dans mon Rafale et revenu larguer tout mon stock de bombes et de missiles sur le Caribbean Hôtel, ne laissant à sa place qu’un cratère fumant jonché de gravats et de cadavres entremêlés.

Greg, que je n’avais pas entendu arriver, s’est pointé derrière moi et m’a donné quelques coups de coude dans le dos pour me signaler sa présence et son intention de s’entretenir avec moi.

\textsc{Moi} : Quoi, qu’est-ce qui se passe ?

Il a alors attiré mon attention, de façon non verbale, en le pointant du doigt, sur un fait qui venait de se produire et pouvait se révéler d’une importance capitale pour la suite de notre enquête : une porte d’ascenseur venait de s’ouvrir, non loin d’une girafe en train de brouter les feuilles en plastique d’un arbre factice, et de cette ouverture venait de surgir ce qui ressemblait comme deux gouttes d’eau à une femme de chambre poussant un chariot adapté à l’exercice de ses fonctions.

Croiser une femme de chambre est chose assez courante dans un hôtel, je vous le concède, mais ce qui la rendait particulièrement intéressante dans le cas présent, c’était que la femme de chambre en question ressemblait elle-même comme deux gouttes d’eau à Repentance Whittingham, alias Atiena, la Gardienne de mes deux.

Repentance semblait parfaitement détendue, comme une fille à qui tout sourit dans la vie et qui n’a à priori aucune raison de s’inquiéter pour son avenir.

Je dis bien «~à priori~», car toute personne qui a quelque chose à se reprocher et croise mon chemin a du souci à se faire pour son avenir. D’ailleurs, toute personne qui croise mon chemin, même si elle n’a rien à se reprocher, a du souci à se faire pour son avenir. Je peux très bien, sur un coup de tête, parce que je me suis levé du pied gauche ou de la couille droite, parce qu’il tombe sur le monde un fin crachin qui m’indispose, parce que ma femme me trompe avec mon chien, parce que j’ai écumé sans succès toute ma garde-robe à la recherche d’une paire de chaussettes qui ne soit pas trouée au niveau du gros orteil, pour tout un tas de raisons passant par tous les stades de l’insignifiance, décider de lui loger une balle de crâne, avant d’exécuter tous les témoins de la scène (ce qui a chaque fois constitue une nouvelle scène qui peut avoir de nouveaux témoins, et ainsi de suite, de sorte qu’il faut parfois éliminer tout un quartier, voire une ville entière et une bonne partie de la campagne environnante, pour se débarrasser enfin du dernier témoin, ou, si je suis dans un bon jour, ce qui n’arrive pratiquement jamais, lui laisser la vie après lui avoir coupé la langue et crevé les yeux et les tympans pour m’assurer de son innocuité, ce qui représente un surcroît de travail non négligeable pour finalement pas grand-chose, sachant que la plupart des gens préfèrent largement être morts plutôt que vivre dans les conditions épouvantables que je viens de décrire, sans voir ni entendre quoi que ce soit, ni être en capacité de prononcer le moindre mot, même s’il n’y a pas grand-chose à dire quand on en est réduit à un tel degré d’infirmité), regagner tranquillement mon domicile avec mes fringues couvertes de sang, me faire couler un bain, mettre un CD de Funki Porcini (The Mulberry Files, par exemple, ou At The Edge Of The World), me glisser dans l’eau tel un reptile sournois, une larve machiavélique, un cafard dans le plus simple appareil, et me laisser glisser sans résistance dans la tiédeur parfumée de l’oubli en sirotant un grand verre de cognac.

\textsc{Greg} : Tu vois ce que je vois ?

\textsc{Moi} : Oui, c’est Repentance Whittingham.

\textsc{Lui} : Qui ça ?

\textsc{Moi} : Atiena, la Gardienne de l’ennui. Son vrai nom, c’est Repentance Whittingham.

\textsc{Lui} : Comment tu sais ça ?

\textsc{Moi} : C’est Dumo qui me l’a dit.

\textsc{Lui} : Qui ça ?

\textsc{Moi} : Dumo, le réceptionniste.

\textsc{Lui} : On fait quoi ?

\textsc{Moi} : On va juste lui poser quelques questions, gentiment, poliment, comme des gens bien éduqués, et on verra bien ce qu’il en ressort.

\textsc{Lui} : Et s’il n’en ressort rien ?

\textsc{Moi} : On surveille l’entrée, on attend qu’elle sorte, on l’embarque et on l’emmène au labo pour la cuisiner à l’abri des regards indiscrets.

Greg, outré : Mais c’est parfaitement illégal.

\textsc{Moi} : En effet.

\textsc{Lui} : C’est contraire aux droits de l’homme, la femme, et tout ce qui s’ensuit. C’est très grave.

\textsc{Moi} : Si on s’en tient au code pénal, il s’agit effectivement d’un enlèvement passible d’une lourde peine de prison, surtout s’il s’accompagne d’actes de torture et de barbarie.

\textsc{Greg} : Désolé, mais je ne mange pas de ce pain-là. Ma réputation et mon honneur sont en jeu, sans parler de ma licence professionnelle.

\textsc{Moi} : Ne t’en fais pas, j’assume l’entière responsabilité de la mission. En cas de pépin, il te suffira de nier avoir eu connaissance de mes agissements, j’abonderai dans ton sens et tu t’en sortiras sans une égratignure. Mais trêve de bavardage. Cet endroit est une véritable jungle, nous ne sommes pas les bienvenus, et le personnel fera tout ce qui est en son pouvoir pour nous mettre des bâtons dans les roues, voire nous éliminer et faire disparaître nos corps dans les profondeurs de l’hôtel. Il faut agir vite, pendant que la cible est encore en vue, sinon on va perdre sa trace au détour d’un couloir. Elle a l’avantage du terrain, qu’elle arpente depuis de longues années et dont elle connaît chaque recoin par cœur.

\textsc{Greg} : On n’en sait rien.

\textsc{Moi} : Quoi ?

\textsc{Lui} : Si elle travaille ici depuis de longues années.

\textsc{Moi} : Non, mais on peut le supposer. On doit le faire, même, car il ne faut jamais sous-estimer l’adversaire. Allons-y !

Au moment où je prononçais ces mots, le regard de Repentance Whittingham a croisé le mien.

Il ne lui a pas fallu plus d’une demi-seconde pour me reconnaître et comprendre que ma présence en ces lieux n’était pas forcément la meilleure nouvelle de l’année.

Elle a lâché son chariot et s’est précipitée vers l’entrée avec une telle célérité (je ne saurais la quantifier au juste, mais je pense qu’on ne devait pas être loin de celle de la sonde Parker Solar Probe soumise à l’influence gravitationnelle de Vénus) que Greg et moi en sommes restés comme deux ronds de flan, bras ballants, yeux écarquillés et bouche grande ouverte, l’air si profondément idiot qu’on n’a pas été étonnés de voir tout le personnel présent éclater de rire et se foutre ouvertement de notre gueule.

Le temps de reprendre nos esprits, ravaler notre rancœur et prendre sur nous pour ne pas dégommer tout le monde à coups de flingues, et on s’est lancés à la poursuite de la fugitive.

Repentance Whittingham s’est engouffrée dans une Cooper S garée un peu plus loin, le genre de petite bombe qui permet de se faire la malle en vitesse quand on n’a pas la conscience tranquille, horriblement envie de pisser, ou simplement de se faire plaisir sur le bitume quand on est amateur de sensations fortes. Vous en connaissez beaucoup, vous, des femmes de chambre qui roulent en Cooper S ? Moi pas. On est déjà sur de la femme de chambre de qualité supérieure, haut de gamme, pur porc, élevée au grain sous la mer (d’où ces petits arômes iodés fleurant bon la moule à marée basse qui font la joie des gastronomes en culotte courte, en slip, ou même à poil pour les plus intrépides), ou alors une femme de chambre qui, en plus de passer une partie de son temps à faire des lits et récurer des chiottes, se livre à d’autres activités pour arrondir ses fins de mois (ce qui n’est pas une fin en soi, ni en soie, les humoristes apprécieront). À moins, bien sûr, qu’il ne s’agisse d’une femme qui a ou vient d’hériter d’une somme d’argent assez conséquente, ou de gagner au loto, et continue néanmoins à exercer sa profession parce qu’il s’agit pour elle non pas de quelque chose d’essentiellement alimentaire, avant tout destiné à la nourrir elle et le gamin trisomique qu’elle élève seule depuis le décès par arme à feu de son conjoint alcoolique et violent, mais d’une authentique passion qu’elle n’entend pas abandonner, même pour tout l’or du monde.

Repentance Whittingham a fait rugir le moteur de son petit bolide, pas très content d’être réveillé en sursaut (le genre de moteur qui n’est pas du matin, préfère largement s’exprimer à la tombée de la nuit sur les grands boulevards ou les petites routes de campagne), puis s’est enfuie en laissant dans son sillage des traces de gomme brûlée additionnées d’un épais nuage de gaz d’échappement, riches, comme chacun le sait, en monoxyde de carbone, dioxyde d’azote et autres hydrocarbures aromatiques polycycliques très mauvais pour la santé. Mais la petite peste se foutait royalement que les gens s’intoxiquent en respirant ses déjections. Tout ce qui l’intéressait, c’était de rouler comme une dingue dans les rues de la ville, sans se soucier un instant des dommages collatéraux, vies brisées et familles en pleurs que sa conduite irresponsable pouvait occasionner.

Si vous voulez mon avis, Repentance Whittingham n’était pas encore prête à se repentir, implorer la clémence des instances supérieures de l’univers, demander pardon à qui que ce soit, ni même éprouver le moindre soupçon d’un commencement de début de vague regret pour toutes les mauvaises actions dont elle s’était rendue coupable au cours de son existence entièrement dévolue à la domination d’autrui et sa soumission sans réserve à ses exigences les plus dégradantes, injustes, obscènes et farfelues.

De mon côté, après m’être quelque peu pris les pieds dans le tapis et avoir bien failli m’empaler sur la corne du buffle qui trônait à l’entrée, je déployais à présent l’ensemble de mes facultés motrices et capacités énergétiques pour rejoindre mon véhicule au plus vite.

J’étais plus ou moins dans les temps, jusqu’au moment où Greg a mis le pied sur une crotte de chien et ripé en beauté avant de s’étaler de tout son long dans le caniveau. Cette chute s’est accompagnée d’une bordée d’injures et de grossièretés que je n’aurai pas la faiblesse de reproduire en ces lignes, mais sachez qu’elles s’adressaient non seulement à la gent canine, ce ramassis de bâtards attardés tout juste bons à se renifler le cul et gueuler sans raison, mais surtout leurs soi-disant propriétaires, maîtres ou appelez ça comme vous voudrez, sombres crétins et abrutis de première classe qui considèrent l’espace public comme des chiottes à clébard. D’après Greg, on devait non seulement leur infliger une forte amende en cas de non-ramassage de crotte, mais aussi les condamner à ingérer l’objet du délit en mastiquant longuement chaque bouchée.

N’ayant rien d’autre sous la main pour nettoyer la semelle de sa chaussure, Greg, toujours en jurant comme un charretier, s’est mis à la racler furieusement contre le bord du trottoir, maudissant la négligence des usagers et l’incapacité des élus locaux à assurer la propreté de leur territoire. Dans ces conditions, l’interdiction massive et définitive de tout canidé dans l’espace public lui apparaissait comme la seule mesure raisonnable à prendre, le chien devant être réservé à un usage purement domestique, comme les fonctions de gardiennage ou d’ami de substitution pour les gens qui vivent dans la solitude la plus extrême, ou encore la pratique de la chasse en milieu forestier, endroit où ces sales bêtes peuvent pisser et chier à tout bout de champ sans que cela prête à conséquence, au même titre que les renards dont ils sont les proches cousins. Le chat, par contre, qui jamais ne s’abaisserait à chier devant tout le monde et encore moins laisser sa crotte à l’abandon, avait toute sa place dans la cité, d’autant que sa présence dissuadait les rats et autres rongeurs malintentionnés de s’installer ouvertement dans nos murs.

Cette opération de nettoyage nous a pris quelques précieuses secondes supplémentaires, de sorte que Repentance Whittingham, alias la Gardienne de la Nuit et accessoirement la femme de chambre la plus rapide du monde, avait déjà quelques longueurs d’avance sur nous quand nous nous sommes enfin lancés à sa poursuite.

Mais, grâce aux deux cent cinquante bourrins survitaminés attelés à ma charrette, il ne m’a pas fallu longtemps pour la rattraper. Le seul détail véritablement incommodant dans cette affaire, c’était que l’habitacle de la Kangoo, en dépit des efforts désespérés de Greg pour s’en défaire, empestait la merde de chien.

Me voyant débouler dans son rétro, elle a accéléré à son tour, prenant tous les risques pour me distancer, et je dois bien admettre qu’elle était loin d’être maladroite avec un volant entre les mains. Si un jour on la virait de son emploi de femme de chambre, elle pourrait toujours se recycler dans la course automobile. Cela dit, on n’était pas sur la Whaanga Coast ou au Panzerplatte. En ville, les limitations de vitesse sont réduites au plus bas et les panneaux de signalisation nombreux pour les faire respecter. Cela dit, faire partie de la noble famille des Représentants de l’Ordre présente tout de même quelques avantages, comme par exemple celui de disposer d’un gyrophare et une sirène pour s’affranchir des règles de circulation en vigueur. L’usager doit être averti quand des cinglés roulent à tombeau ouvert dans les rues de la cité. Il doit savoir que les forces de l’ordre (au même titre que celles du feu et de la santé), toujours sur la brèche pour garantir sa sécurité au péril de la leur, sont prioritaires quand des actions de ce type sont engagées sur la chaussée. Même si mon véhicule de service, qui se trouvait aussi être mon véhicule personnel (un ludospace très agréable en conduite normale et pratique pour transporter des objets lourds et encombrants), n’était pas une réplique exacte de la Pursuit Special de Mad Max, elle n’en constituait pas moins une arme de destruction massive une fois lancée à pleine vitesse dans les rues de la ville. J’ai donc mis en branle le système d’avertissement sonore et lumineux dont j’étais le dépositaire.

Au volant de sa Cooper S, totalement insensible aux appels à la raison que je multipliais à son égard, Repentance Whittingham filait comme une flèche et je lui suçais la roue (à défaut d’autre chose) au plus près, ne laissant entre nos deux véhicules que l’épaisseur d’une feuille de papier à cigarette. Dans les films d’action, les gens se collent au cul, roulent portière contre portière, et n’hésitent pas à froisser de la tôle pour arriver à leurs fins. Ils s’en foutent, c’est la production qui paye. Dans le cas présent, je le répète, il s’agissait d’un véhicule que j’avais payé de ma poche, rubis sur l’ongle, le prix d’une bouchée de très bon pain, je vous le concède, puisque je l’avais acheté d’occasion, mais dans lequel j’avais par la suite investi de confortables sommes d’argent pour le transformer en authentique machine de guerre habilement dissimulée sous les dehors inoffensifs d’un utilitaire sans autre prétention que celle de l’être, utile. On sait par expérience que sapiens sapiens (environ deux millions de glandes sudoripares, vitesse de pointe aux alentours de 45 km/h, inventeur de la bombe atomique, du petit salé aux lentilles et du saut à l’élastique) est très souvent au moins aussi attaché à sa voiture qu’à sa femme et ses enfants. Il la dorlote, la bichonne, la caresse amoureusement, se mire avec délectation dans sa carrosserie rutilante, s’enfonce jusqu’aux ouïes dans ses sièges en cuir pleine fleur pour savourer le chant rauque et mélodieux de son six cylindres sublimé par une ligne d’échappement optimisée, lui récure les jantes à la brosse à dents et polit les chromes à la peau de chamois. Il ne fait aucun doute, si la chose était matériellement possible, qu’il n’hésiterait pas à s’accoupler avec elle dans les positions les plus torrides du kamasutra automobile. On imagine alors facilement la suite, aussi terrifiante que fascinante : pour une raison quelconque, échappant totalement à l’entendement des apprentis sorciers du transhumanisme, la bagnole tombe enceinte. Après quelques semaines d’une grossesse tumultueuse, elle donne naissance à une sorte de monstre de Frankenstein supersonique, à mi-chemin entre l’homme et la Formule 1. À l’instar du singe et du nain (je ne fais bien entendu aucune analogie qualitative entre l’un et l’autre), la chose est dotée de membres inférieurs nettement plus courts que la moyenne, avec des cuisses très puissantes. Capable de se mouvoir efficacement aussi bien à la verticale qu’à l’horizontale, c’est toutefois dans cette dernière configuration qu’elle produit les accélérations les plus fulgurantes et pulvérise haut la main les records de vitesse des animaux les plus rapides de la planète. Et je parle ici uniquement des mammifères terrestres, comme le guépard, le lièvre et l’antilope, et de la poiscaille, comme l’espadon, le marlin et le thon banane. J’exclus volontairement les volatiles, comme le faucon, l’aigle et le martinet, et plus encore les insectes, minuscules créatures sex pedibus aux performances exceptionnelles, sachant que la vorace cicindèle, par exemple, qui ne dépasse pas les deux centimètres de long, flirte allègrement avec les sept cents kilomètres/heure, à tel point que ses propres yeux n’arrivent plus à suivre et qu’elle peine à distinguer ses proies dans le feu de l’action. Mais s’il fallait à tout prix trouver un alter ego vaguement humain à cette abomination mutante, ultime rejeton des passions déviantes de la sexualité automobile considérée comme un des beaux-arts, c’est clairement du côté des speedsters Jay Garrick, Wally West, Barry Allen, Pietro Maximoff (Vif-Argent), Danica Williams et Hunter Zolomon (alias Zoom ou Reverse-Flash) qu’il conviendrait de se tourner.

Bref, on s’en fout.

Ce qui est certain, c’est que personne, à commencer par moi, n’aurait pu se douter de ce qui allait se passer dans un très proche avenir.

Traquée, comme on le répète toutes les trente secondes dans ces pathétiques émissions de télé pour handicapés mentaux et baltringues en surcharge pondérale spécialisées dans les faits divers sanglants, cold cases et autres affaires judiciaires retentissantes dont tout le monde se contrefout éperdument, Repentance Whittingham a pris tous les risques pour tenter de nous semer. Elle aurait pu, pour rester dans la ligne cinématographique adoptée précédemment, être interprétée par la très contagieuse Antonia Thomas, père anglais, mère jamaïcaine, mélange aux yeux verts hautement détonant, susceptible de déclencher des tempêtes de niveau 5 dans les parties basses de la sphère anatomique : vents violents, slips arrachés et projetés à des lieues à la ronde, couilles tuméfiées, décharges à répétition, danger de priapisme et crise cardiaque. Pour celles et ceux qui ne seraient pas au courant, je me dois de préciser qu’Antonia, depuis quelques années maintenant, exerce la noble profession d’actrice au pays de Très Sa Gracieuse Majesté la Reine de Mes Deux. Hélas, son talent n’est pas reconnu à sa juste valeur par ces cons de rosbifs qui n’y entendent rien à l’art, la bouffe et la beauté féminine, rien à rien en général, sans quoi ils ne passeraient pas leur temps à se prosterner comme des carpettes décérébrées devant le Royal Vampire qui leur suce la moelle pour maintenir ses privilèges et son train de vie pharaonique. La vérité, c’est que ces insulaires, comme tous les gens de leur sorte, sont des asociaux de première classe, arrimés telles des patelles paranoïdes à leur foutu rocher. L’ineptie atavique et l’aveuglement quasi systémique de ses compatriotes, alcooliques pour la plupart il faut bien le dire, ont conduit à cette absurdité stratosphérique qui ne cesse de susciter mon exaspération (et celle, je veux le croire, de toutes les âmes sensibles qui ont encore le sens du beau et vouent, à travers ses productions les plus inimitables et abouties, un culte irréductible à la nature : si on veut la voir (Antonia) en activité et se prendre à rêver de serrer son petit corps frémissant entre ses bras pantelants, lui bouffer le nez à pleine bouche et éventuellement se livrer sur elle à des activités que la morale réprouve (je m’en excuse d’avance auprès d’elle et tous les membres de sa famille, mais je ne fais que traduire le sentiment général des heureux élus qui ont eu la chance de croiser sa route, hommes, femmes et animaux confondus, y compris insectes et batraciens, heureuses la mouche qui pète et la grenouille qui coasse sur son épaule dénudée), on doit malheureusement se satisfaire de somnoler avec une demi-molle devant des teen dramas et autres séries télé aussi diversement mémorables que Misfits et Lovesick, sans oublier Good Doctor aux côtés de Freddie Highmore, alors frais émoulu de la série Bates Motel. À noter que ce même Freddie Highmore, à l’instar d’une Olivia Cooke (délicieuse) ou un Nestor Carbonell (ténébreux à souhait avec ses sourcils épais et son regard de braise), qui lui donnent la réplique dans ladite série (Bates Motel, très réussie au demeurant), peine lui aussi à s’extirper des griffes du petit écran. Seule Vera Farmiga (qui joue le rôle de Norma Louise Bates, la mère de Freddie Highmore dans Bates Motel), née de parents ukrainiens, ancienne de la St. John the Baptist Ukrainian Catholic School de Syracuse, dans l’État de New York, et actrice pour laquelle j’éprouve une appétence aussi bizarrement obsessionnelle que dépourvue de tout caractère sexuel (même si je la trouve excessivement désirable, paradoxe que je peine à expliquer, je l’avoue, et ne tiens même pas spécialement à le faire, tant je préfère que le mystère reste entier, mais que j’aurais néanmoins, s’il fallait absolument tenter de fournir un élément de réponse, tendance à mettre sur le compte de l’aura de maternité quasi virginale qui émane de son adorable personne), s’en sort avec les honneurs. Croyez-moi ou non, mais c’est un bien triste monde que celui dans lequel la plupart d’entre nous survivent avec l’énergie du désespoir, et heureusement qu’il existe des gens comme Antonia Thomas et surtout Vera Farmiga pour badigeonner d’un peu de baume nos cœurs meurtris par les assauts répétés de l’existence.

Donc, comme je vous le disais, l’horreur est montée d’un cran.

Poussée dans ses derniers retranchements, telle une mygale acculée dans le fond de sa tanière, Repentance Whittingham s’est décidée à jouer le tout pour le tout. Refus de priorité, dépassement par la droite, emprunt de voie non autorisée, non respect des règles les plus élémentaires de bonne conduite et du bien circuler ensemble, elle a en quelques minutes pulvérisé tous les records détenus par les pires chauffards de la planète, prouesse d’autant plus remarquable qu’elle n’était à priori pas sous l’emprise de l’alcool ou d’une quelconque drogue, de synthèse ou autre. Franchement, si je n’avais pas eu entre les mains le volant sport de ma Kangoo Interceptor Turbo +, capable de franchir le mur du son par simple pression du gros orteil sur la pédale d’accélérateur, je crois que j’aurais été incapable de suivre le mouvement et obligé de renoncer, la mort dans l’âme et la larme à l’œil, à mettre un terme à la folle cavale de la Gardienne de la Nuit. Mais c’est mal me connaître que de croire que j’allais renoncer aussi facilement. J’avais en effet, outre une formation de cavalier de la Garde républicaine et d’enquêteur subaquatique (niveau bac ou RNCP 4), choses qui, quand on aime l’eau et les chevaux, peuvent toujours servir en cas de crise, suivi les cours de pilotage VRI du commandant Valentin Deschanel, spécialisé dans les go fast et les interventions en zone urbaine à forte densité, star incontesté de la discipline, hélas tragiquement décédé quelques mois auparavant dans un accident de la circulation, et ce alors qu’il rentrait tranquillement chez après avoir fait ses courses à la supérette du coin. Il faut dire qu’il était en T-MAX 530 (quarante cinq chevaux à six mille cinq cent tours/minute, le scooter préféré des dealers) et que le choc avait été d’une violence inouïe (et non pas inuite comme le prétendent certains imbéciles patentés dont on se demande encore comment ils ont réussi à ne pas faire virer de l’école et passer leur bac avec succès, le concept, je le rappelle, n’ayant aucun rapport particulier avec le Groenland et sa population autochtone, pas plus qu’avec la baleine, l’ours polaire et le caribou, même si l’ours polaire, plus encore que le caribou ou le pacifique cétacé, est capable de faire preuve de violence quand il se sent menacé, à noter la rime plutôt riche avec cétacé), j’en veux pour preuve ses mandarines, yaourts aux fruits (il adorait les yaourts aux fruits, personne n’est parfait) et blancs de poireau éparpillés dans tout le périmètre. Une vraie boucherie, et pour la perte sinon d’un ami à proprement parler, même si nous entretenions toujours d’excellentes relations, au moins d’un mentor qui m’avait enseigné les joies de la glisse et du talon-pointe.

Tandis que nous nous trouvions sur une artère passablement fréquentée, à slalomer dangereusement entre les usagers qui nous maudissaient au passage, Repentance Whittingham a tenté une manœuvre particulièrement osée, sinon suicidaire, qui consiste à doubler à l’aveugle dans un virage en priant le ciel que personne n’ait la mauvaise idée de se pointer en face. Ce cas de figure, systématiquement représenté dans les courses-poursuites au cinéma, donne toujours lieu, même si on sait que ça va passer de justesse, à des moments de franche rigolade, surtout quand le pauvre type qui arrivait paisiblement en sens inverse finit les quatre roues en l’air dans le décor, les cheveux en bataille et les fringues en lambeaux, et contemple, dépité, sa belle voiture toute neuve bonne pour la casse. C’est le genre de sitation cruelle et parfaitement injuste qui, reconnaissons-le à notre corps défendant, trouve bien souvent un écho favorable dans le public, preuve que l’être humain est encore loin d’en avoir fini avec ses vieux démons.

Sauf que dans la vraie vie, sans caméra ni perchman, les choses ne se passent pas toujours comme on voudrait. Le scénario n’est pas écrit, on improvise en permanence, et il n’est pas possible de retourner quarante fois la scène pour obtenir satisfaction. La première est la bonne, il faut être au top tout de suite et ne surtout pas compter sur le montage ou les effets spéciaux pour arrondir les angles.

Et ce qui devait arriver arriva : au moment où Repentance effectuait sur les chapeaux de roues son dépassement non autorisé pour cause d’absence totale de visibilité dans un virage particulièrement dangereux, un Sprinter (utilitaire léger de chez Mercedes, ndlr) arrivait en sens inverse. À son volant, se trouvait un triste sire qui lui-même roulait à tombeau ouvert parce qu’il était en retard à son boulot, peintre en bâtiment en l’occurrence. Non content de baigner dans son jus comme une grosse merde fraîchement pondue, il écoutait du Spear of Longinus à fond les ballons, fumait clope sur clope et braillait comme un veau dans l’habitacle enfumé de sa poubelle.

On ne va pas se mentir, se voiler la fesse, tenter de minimiser pieusement les faits : le choc a été d’une violence impitoyable, provoquant un de ces vacarmes épouvantables qui évoquent irrésistiblement un tremblement de terre, une explosion due au gaz ou une attaque de missiles russes, et la projection dans l’atmosphère d’une pluie de débris automobiles comparables aux scories d’une éruption volcanique de grande ampleur.

Disons-le tout net, la Mini ne faisait pas le poids face au Sprinter. Elle a été littéralement pulvérisée sous l’impact. Tout est allé tellement vite que le Sprinter (dont le conducteur, qui avait pris une cuite retentissante la veille, n’était sans doute pas au mieux de sa forme) n’a même pas esquissé la moindre tentative de dégagement. De mon côté, tandis que Greg tétanisé par la peur n’avait même plus la force de hurler, se contentant d’ouvrir un bec de cent pieds de long d’où aucun son ne sortait, j’ai sauté à pieds joints sur le frein et réussi de justesse, grâce à mon sens aigu du pilotage et ma parfaite connaissance de l’arme de destruction massive qui me tenait lieu de véhicule, à éviter le massacre. Après avoir embouti la Mini de plein fouet, le conducteur a cédé à la panique, et le Sprinter a continué sa route en zigzaguant dangereusement avant d’aller s’écraser à grand bruit contre un platane.

Je suis sorti de la voiture (le premier, Greg avait besoin d’un peu de temps pour se remettre de ses émotions) et me suis précipité au chevet de Repentance Whittingham. Enchevêtrée dans un amas de tôle indescriptible, j’ai constaté qu’elle donnait encore quelques vagues signes de vie. Naturellement, j’ai aussitôt appelé les secours. Même si j’avais quelques raisons de lui en vouloir, notamment celle de m’avoir fait risquer ma vie dans cette course-poursuite endiablée, l’idée de voir une aussi belle chose disparaître à tout jamais de la surface de la Terre me semblait intellectuellement irrecevable. Et oui, pas la peine de hurler, je sais très bien que les femmes, et les êtres vivants en général, ne sont pas des choses, même si on pourrait très bien partager le monde entre choses vivantes ou non, le fait d’être en vie se signalant essentiellement par sa nature dégénérative et éphémère. La vie, en sursis permanent dans le couloir de la mort, est une situation des plus inconfortables. L’être humain, conscient de cette précarité, a tôt fait de sombrer dans le doute et la paranoïa. Il tente, par tous les moyens, de prolonger son existence. Mais pourquoi s’acharner à vivre dans un monde aussi hostile, avec une sentence de mort épinglée au milieu du front ? Le danger vient de partout, y compris de l’intérieur, et peut surgir à tout moment, y compris celui où on s’y attend le moins. Malheur à l’inconscient, galvanisé par la jeunesse ou les stupéfiants, et souvent les deux, qui tente de tromper la mort, car elle finira un jour ou l’autre par l’emporter. La chose vivante, condamnée à disparaître, l’est aussi à se reproduire pour se survivre à elle-même. Sans cette fonction essentielle, toute vie est impossible. Elle peut aussi se dupliquer par ses propres moyens, sans l’aide d’un tiers, mais elle ne peut échapper au processus. La chose, par contre, la vraie, parfaite en soi, se suffit à elle-même et n’a nul besoin de ce stratagème pour perdurer. Son existence, sinon définitive, est au moins certaine et indéniable. Elle n’a nul besoin de régénérescence, renaissance, reset ou mise à jour, nul besoin d’évoluer, de s’adapter à son environnement. Elle est intemporelle, exempte de tout questionnement, toute remise en question. C’est ici, sans doute, que s’opère le distinguo entre la chose naturelle et l’objet fabriqué, artificiel. L’objet, en effet, est soumis à un objectif qui le dépérennise, le voue, au même titre que l’être vivant, à l’obsolescence et la disparition, l’obligeant à se transformer sans cesse pour subsister. C’est ainsi que l’objet, la machine, par exemple, transcende sa nature inanimée pour se rapprocher artificiellement du principe vital. À travers l’homme, l’objet se déplace, pense, vit et meurt. Il est, en quelque sorte, l’ultime avatar de son désir d’affranchissement des lois de l’existence. Depuis des millénaires qu’il se torture inutilement les méninges, l’homme n’aspire qu’à une seule chose : devenir une machine, un automate capable, sans le moindre effort, la moindre prise de tête, crise de conscience ou autre, le moindre doute sur la nature de ses actes, d’accomplir des miracles et de battre tous les records, y compris de longévité, de réduire définitivement au silence et l’impuissance sa vieille ennemie la Mort. L’acte sexuel, par exemple, se doit d’être entièrement dévolu au plaisir, et non plus à cette fonction dégradante et avilissante qu’est la reproduction. Ainsi, ce piège odieux tendu par la Nature pour nous contraindre à signer notre arrêt de mort, entériner l’acte de décès de nos années de jeunesse et d’insouciance, se transforme en gag de l’arroseur arrosé. Débarrassé de ces oripeaux d’un autre âge, le plaisir sexuel peut enfin s’exercer sans limite ni contrainte, faire feu de tout bois. Jouissons sans entrave, morale ou autre, et laissons les basses-œuvres de la chose reproductive à celles et ceux qui n’ont d’autre moyen de subsistance que de fabriquer des enfants à la chaîne. Voilà un monde plus juste, où chacun se spécialise dans son domaine de compétence et vit pleinement sa vie sans remords ni regret. Dans ce monde plus juste, les nantis, dont la recherche du plaisir est le principal, sinon seul et unique domaine de compétence, viennent au secours de ceux qui ont voué leur existence à la douleur. Voir les autres comme des objets, des instruments qu’on peut manipuler à loisir pour satisfaire ses exigences, se satisfaire, est en droite ligne du chemin d’excellence que nous nous sommes tracé. Notre objectif, je le rappelle, est de devenir des machines de guerre, utraperformantes, conçues pour dominer et conquérir le monde. Dans un premier temps, bien sûr, avant de s’attaquer au reste de l’univers, et j’en profite au passage pour rendre grâce à Elon Musk (visionnaire sud-africain de race blanche dont le caractère ultra-reproducteur, à priori paradoxal, s’explique uniquement par son appétence pour les femmes jeunes et jolies bien entendu, mais surtout son narcissisme exacerbé et la volonté de se dupliquer à l’infini) d’avoir eu la présence d’esprit de se ménager (à lui et quelques fidèles) des bases arrières dans l’espace afin de ne pas être pris au dépourvu le moment venu, sachant que les gens, hélas bien trop rares, qui voient un peu plus loin que le bout de leur nez, commencent à se trouver singulièrement à l’étroit sur ce lopin de terre étriqué qu’est la planète bleue. Toutes les créatures faibles et geignardes qui auront l’impudence de se mettre en travers de notre marche triomphale seront impitoyablement exterminées.

Et voilà pourquoi cette chère Repentance Whittingham, qui, j’ose le dire, était l’irréfutable incarnation de l’éternel féminin en son sens le plus mythologique (autant dire carrément mytho, archétypal, archi-typique, désespérément romantique, goethien, gothique, et, en un mot comme en cent, au moins en ce qui me concerne car tous les égouts sont dans la nature, assez éloigné de la blonde platine à la Ginger Rogers, d’autant que j’ai une sainte horreur de la comédie musicale et que Fred Astaire m’a toujours fait penser à un sosie virevoltant de Stan Laurel) du terme, n’avait à mon sens nul besoin de se livrer à cette malheureuse tentative de dépassement, au sens propre comme au figuré, laquelle ne pouvait, en définitive, que la confronter brutalement à sa propre finitude.

Quand il a été constaté qu’elle était encore en vie, même si celle-ci ne tenait plus qu’à un fil, Greg et moi, tandis que les badauds (à commencer par le conducteur de la vénérable Ford Fiesta que la Mini avait tenté de doubler) commençaient à affluer de toute part, nous sommes dirigés vers la carcasse du Sprinter.

Le baiser passionné qu’il avait échangé avec le tronc du platane lui avait explosé le moteur et fait voler le parebrise en éclats.

À l’intérieur, se trouvait une vieille connaissance que Greg, pour avoir suffisamment enquêté sur l’affaire Tiago Alvarez (Sally Robinson le harcelait encore quasi quotidiennement pour le pousser à franchir le Rubicon en éliminant l’assassin présumé de son bien-aimé) a parfaitement identifiée dès le premier coup d’œil. Aussi improbable que cela puisse paraître, il s’agissait ni plus ni moins que de l’ignoble néonazi psychopathe, pervers et pyromane Noé Desmarais, fondateur, avec les sieurs Aymeric Jégou et Milo Monteil, de la sympathique petite association de malfaiteurs connue sous le nom de Disciples de la Colère. Personnellement, je préférais la dénomination de Disciples de la Connerie, qu’ils avaient poussé à un rare degré de perfection.

Petite piqûre de rappel pour la route : un beau jour, un certain Léopold Chiasson de Bellisle, comte de son état, s’éprend de son garde-chasse, un jeune dieu répondant au nom de Robert Pleimelding. Dès qu’il voit son petit cul apparaître au coin d’un bois (quand il est à quatre pattes en train de ramasser des châtaignes, par exemple, ou de renifler une truffe avec son flair de lévrier), le comte est tellement excité qu’il serait prêt à enfiler cul sec le premier sanglier qui lui passe à portée d’entrejambe. Car le problème, voyez-vous, c’est que Robert est marié, et que rien ne prouve qu’il ait envie de se faire ramoner la tuyauterie par son employeur (même si à l’époque, je dis ça je dis rien, les gens n’étaient peut-être pas aussi tatillons sur les conditions de travail et les droits du citoyen, autrement dit rechignaient moins à payer de leur personne pour se donner les moyens de réussir dans la vie). Fou de désir, vous l’aurez compris, Bellisle lui sort le grand jeu : havane, cognac centenaire, grosse bûche dans la cheminée du salon, préludes de Chopin avec légers craquements d’époque par un Arthur Rubinstein au sommet de son art, peignoir en soie savamment entrouvert pour laisser deviner une anatomie aussi avantageuse que remarquablement bien conservée, interminable discussion sur la question de savoir lequel du 12 ou du 20 est le meilleur calibre pour la chasse à la bécasse, sachant que le plomb de 8 bourre grasse reste sans doute le meilleur compromis en toute circonstance, etc, etc. Petit hommage en passant à la vieille sorcière de la Madrague récemment disparue, ex-héroïne de La Vérité (vaudeville judiciaire et musical tragique signé Henri-Georges Clouzot), même si, qu’on le veuille ou non, ses amitiés politiques douteuses avec la fille cadette du caliborgnon de la Trinité-sur-Mer risque de ternir durablement l’image de passionaria de la cause animale qu’elle entendait laisser. Bellisle, disposant de moyens quasi illimités pour parvenir à ses fins, ne tarde pas, en dépit de la résistance héroïque qu’elle lui oppose, à réduire sa proie à sa merci. La romance peut commencer, et se prolonger ainsi jusqu’à la mort du comte, qui laisse alors assez d’argent à Robert et sa famille pour vivre confortablement en chantant les louanges et bénissant chaque jour le nom de leur généreux donateur. Dans la foulée, l’aristocrate lui octroie également les quelques cinquante hectares de forêt (et pas de la friche dégueulasse, hein, du terrain vague hérissé d’arbres faméliques tout juste bon à stocker des déchets nucléaires, non, de la bonne vieille forêt bien épaisse regorgeant du plus fin gibier et des meilleurs champignons comestibles poussant comme une bénédiction divine au pied des plus beaux fûts), de forêt, disais-je, que Robert, même s’il n’a plus le pas aussi souple qu’auparavant, continue à arpenter inlassablement pendant ses vieux jours, la pipe au bec, comme il l’a toujours fait et rêve de le faire encore longtemps après sa mort dans les forêts enchantées du paradis. C’est là, au cours d’une de ses balades avec sa fidèle Greta (un drahthaar, croisement de griffon kortals et de braque allemand à poil court, excellent chien d’arrêt qui, le cas échéant, n’hésitera pas un instant à se jeter à l’eau pour repêcher une gélinotte criblée de plombs), son juxtaposé Chapuis Progress Grand Luxe calibre 12/70 (crosse anglaise en noyer premier choix, plaque de couche en bois de rose et sujets animaliers gravés en taille douce sur les contre-platines et le dessous de bascule, une arme de collection qu’il n’aurait jamais eu les moyens de s’offrir sans les largesses de Chiasson) et sa gibecière à rabat en cuir pleine fleur Lazzaro Bernardini (encore du cousu main qui cadre assez peu avec le statut de garde-chasse à la petite semaine de l’intéressé), qu’il tombe sur une vision d’horreur qui restera à tout jamais gravée dans sa mémoire : un corps (humain, le corps), qu’une quelconque entité malfaisante a manifestement tenté de détruire intégralement par le feu.

S’il était envisageable que le sujet, pour des raisons diverses (démence, fanatisme religieux ou mystique particulière, satanisme, pratique du vaudou, sur fond d’addiction à l’alcool ou toute autre drogue dure sous quelque forme que ce soit), ait tenté de mettre fin à ses jours de cette façon que je qualifierai de pour le moins médiévale, il l’était nettement moins qu’il se soit fait désintégrer par des extraterrestres en train de reconnaître les lieux dans la perspective d’une prochaine invasion. Tout bien considéré, le plus crédible restait que le pauvre avait été la victime d’actes de torture et de barbarie de la part d’un ou plusieurs individus qu’il restait à identifier et punir à la hauteur de leurs agissements.

Dépêché sur les lieux en compagnie de Zaahid Shirani, légiste d’origine hindoue avec lequel j’avais, dans un premier temps, noué des relations purement amicales, avant que celles-ci ne se transforment en relations quasi familiales (le ténébreux personnage avait fort habilement mis le grappin sur Tosca, la sœur jumelle de ma dulcinée), j’avais d’abord pensé à un règlement de comptes entre truands, la technique dite «~du barbecue~» étant souvent utilisée par ce genre de clientèle pour tenter de faire entrave à l’action de la justice (infraction, je le rappelle en passant au cas où certains d’entre vous auraient dans l’idée de céder à la tentation, passible au bas mot de trois ans d’emprisonnement et de 75 000 euros d’amende).

C’était compter sans la perspicacité de Shirani, limier diabolique capable de débusquer sans effort la plus microscopique aiguille dans la plus monumentale botte de foin. L’animal avait réussi, Dieu sait comment, à dresser le profil génétique de la victime et le comparer à celui de Tiago Alvarez, un ambulancier gay qui s’était récemment évaporé dans la nature, disparition jugée plus qu’inquiétante sur laquelle, par le plus grand des hasards, enquêtait Grégoire Lussier, un autre mien ami, pour le compte d’une (un) certaine Sally Robinson, sosie de Danny DeVito en jupon. Grâce à Cerqueira, gorille portugais au cerveau de moule qui tentait tant bien que mal de dissimuler son homosexualité à son entourage (je l’avais surpris en flagrant délit de racolage sur la voie publique, et menacé de le foutre en taule et ébruiter son petit secret s’il ne se montrait pas coopératif), à commencer par les membres de sa fine équipe de sympathisants d’extrême-droite, le lien entre Alvarez et Desmarais avait pu être établi. D’après le grand singe en question, c’était Noé Desmarais, suprémaciste blanc, homophobe et pyromane, qui avait carbonisé Alvarez au lance-flammes, en joyeuse compagnie de deux autres ordures de son espèce, Aymeric Jégou et Milo Monteil.

J’ai brandi ma plaque et hurlé à la cantonade : POLICE !!! CIRCULEZ, Y A RIEN À VOIR !!!

On le sait, les gens adorent les trucs scabreux. Ils vous diront que non, mais le fait est que s’il se passe quelque chose d’horrible quelque part, ils ont toutes les peines du monde à ne se précipiter pour jeter un œil. Ils sont comme ces charognards qui reniflent l’odeur de la viande froide et rappliquent ventre à terre pour prendre part au festin. Avant ils se contentaient de regarder, maintenant ils filment, ce qui leur permet d’une part de revoir la scène encore et encore, d’autre part d’en faire profiter celles et ceux qui n’auraient pas eu la chance d’y assister. Avant, ils n’avaient que leur parole à opposer aux sceptiques et aux jaloux, maintenant ils ont les images pour preuve de leur bonne foi. Ces images, d’ailleurs, peuvent leur assurer une petite rente s’ils sont les seuls à les posséder. Les chaînes d’info en continu se feront une joie de les récupérer pour les diffuser en boucle auprès du grand public. Naturellement, plus c’est ignoble et horrifique, et plus le rapport est important. Tout est fait, en ce bas monde, pour flatter les plus bas instincts de la communauté. C’est ainsi que les réseaux sociaux, ramassis de crétins décérébrés et de sociopathes à la petite semaine, peuvent se développer à la vitesse d’une portée de cafards dans un placard rempli de victuailles. Les vautours parlent aux vautours, partagent l’info en temps réel, vingt-quatre heures sur vingt-quatre. Les hyènes ricanent, les requins de la finance entrent dans la danse et se remplissent la panse. Honni soit qui mal y pense, et longue vie en passant à la reine Victoria, au prince de Galles, à Ed Wood (le comte d’Halifax, pas l’auteur de La Fiancée du monstre et Necromania, qualifié par les frères Medved de «~plus mauvais réalisateur de tous les temps~», ce qui est plutôt un compliment venant de réacs dans leur genre), au gentilhomme huissier de (pas à) la verge noire, au duc de Kent et à Naruhito, empereur du Japon, à sa gracieuse épouse Masako Owada, et bien sûr à leur fille Aiko, princesse de Toshi et Grand-cordon de l’ordre de la Couronne précieuse, au même titre que la reine d’Espagne et la princesse Basma de Jordanie, également (pour celles et ceux que ça intéressent, même si je me doute bien qu’ils ne sont légion) Grand-Croix de l’ordre royal de l’Étoile polaire de Suède. Les gens que je viens de citer, citoyens haut-placés de l’univers, quasi divinités vénérées par des peuples tout entiers (lesquels, il faut bien le reconnaître, ne barbotent pas toujours dans l’opulence et ont par conséquent d’autant plus de mérite à admirer des gens qui se goinfrent sur leur dos), bénéficient d’une protection toute particulière pour garantir leur intimité. Mais vous, qui n’avez pas de particule et encore moins de sang royal qui circule dans vos veines, sachez que quel que soit l’endroit où vous vous trouviez (j’allais dire cachiez, sans doute ce que vous auriez de mieux à faire), il y aura toujours quelqu’un pour vous filmer à votre insu. Non pas que votre vie présente un quelconque intérêt, mais le voyeurisme est aujourd’hui parvenu à un tel degré d’omniprésence qu’il est devenu impossible de s’y soustraire. Fini le bon temps où on pouvait agresser une petite vieille en toute sécurité, tabasser un étranger ou harceler sexuellement sa secrétaire sans que la moitié de la planète soit au courant dans les secondes qui suivent. Aujourd’hui, outre les caméras de surveillance qui fleurissent à tous les coins de rues, il faut compter sur les particulier qui vivent en permanence l’œil rivé à l’écran de leur smartphone. L’être humain a toujours été sujet au voyeurisme, c’est vrai, mais ce qui était jadis honteux peut aujourd’hui s’afficher au grand jour en toute légalité. De la même façon, l’exhibitionnisme n’est plus réservé à une certaine catégorie de personnel, comme les politiciens, les artistes ou les gens qui se baladent à poil sous des imperméables même quand il ne pleut pas, de préférence à la sortie des écoles. Non, vous pouvez maintenant être une parfaite nullité totalement dépourvue de charisme et réussir à capter l’attention de millions de followers encore plus nuls que vous. On dit follower parce que suiveur ou suiveuse c’est pas terrible, sinon franchement péjoratif, au même titre que disciple, qui fait un peu gourou d’une secte de demeurés, ou encore abonné, qui fait référence à la téléphonie des années 70 et cadre assez mal avec l’idée de modernité véhiculée par les nouvelles technologies.

Avant, si vous étiez un gros pervers de voyeur, vous aviez vu Psychose trente ou quarante fois, et, sans aller jusqu’à empailler des oiseaux, appliquiez la méthode dite «~Norman Bates~», c’est-à-dire que aviez acheté une perceuse et fait des trous un peu partout dans votre baraque pour vous rincer l’œil. Mais comme vous n’aviez pas de motel perdu au fin fond de la cambrousse, et qu’en plus vous aviez hérité d’une tête qui n’inspirait pas vraiment confiance, le plus dur était de trouver des femmes qui, pour une raison ou pour autre, acceptent de franchir le seuil de votre porte. Par chance, vous aviez un peu de famille, des sœurs, des tantes, des nièces, et même un fils qui s’était dégoté une femme somptueuse qui lui avait donné de nombreux et beaux enfants, dont trois filles pas piquées des hannetons, et qui, malgré le fait que vous sembliez bizarre çà tout le monde (le petit dernier, auquel vous fichiez une trouille bleue, ne se résignait à vous embrasser que contraint et forcé par ses parents, et encore n’aurait-il pas fait pire grimace si on l’avait forcé à rouler une galoche à une vieille méduse échouée sur la plage des Sablettes à La Seyne-sur-Mer), venait vous rendre visite de temps à autre. Vous pouviez alors, l’œil rivé à l’un ou l’autre des nombreux trous qui garnissaient votre intérieur, celui de la salle de bain ou des toilettes notamment, satisfaire pleinement vos répugnantes ambitions. Certes, tout cela demandait une préparation minutieuse et un art consommé de la dissimulation, car si jamais quelqu’un avait découvert votre petit manège, vous pouviez définitivement dire adieu au peu de vie sociale qu’il vous restait, et accessoirement vous préparer à aller finir vos jours en prison, endroit où on pratique encore la séparation des sexes et où on ne dispose de toute façon d’aucun outil pour faire des trous dans les murs. Aujourd’hui, tout se passe comme si un petit malin, un visionnaire et bienfaiteur de l’humanité, conscient de la détresse qui accablait ses semblables et bien déterminé à leur venir en aide, avait trouvé le moyen de faire des trous partout dans le monde pour que tout le monde puisse se rincer l’œil sans limite d’âge ni de distance. Des tas d’exhibitionnistes frustrés, contraint de vivre dans la honte et le déni, ont enfin pu afficher leur différence au grand jour. Tu veux voir mon zizi (pensée émue pour Francky Vincent, inoubliable auteur de La Braguette d’or, Alice ça glisse et La Chatte de la voisine), ma petite chérie ? Non ? Pas de problème, je te le montre quand même. Oui, je sais, tu n’as que dix ans, mais il n’est jamais trop tôt pour s’instruire. Je vais te montrer à quoi ça sert, ce qui se passe quand on le secoue, et tu me remercieras plus tard. Voilà quand même une approche autrement constructive et conviviale du bien vivre ensemble (endroit coloré et agréable, salles bien conçues, belle initiative de la municipalité), qui devrait, sauf imprévu, largement contribuer à apaiser les tensions qui divisent les peuples et dressent inutilement les gens les uns contre les autres. Grâce à Internet, chacun peut enfin vivre sa passion sans se soucier du qu’en-dira-t-on. Ces échanges entre gens de bonne compagnie, hors de tout jugement, sont le lubrifiant qui permet à la machine sociale de tourner sans accroc. Désormais, si un accident se produit quelque part et que ne pouvez y assister, d’abord parce que personne n’a jugé utile de vous prévenir, ce qui est déjà assez grave, et ensuite parce qu’on ne peut pas être partout en même temps, soyez rassuré : même s’il se produit en haute mer, au sommet de l’Himalaya ou aux abords de la planète Mars, les rescapés auront pris soin de filmer la scène sous toutes les coutures. Vous assisterez, comme si vous y étiez, à la lente agonie de celles et ceux qui n’auront pas eu la chance de mourir sur le coup (que vous aurez vu mourir aussi, ne vous inquiétez pas). S’il s’agit d’un naufrage, vous verrez les moins chanceux se faire dévorer par les requins, et le reporter improvisé, animé par une foi intense et la volonté farouche de servir l’info à tout prix, préférera mille fois se faire dévorer lui-même plutôt que de renoncer à sa mission. Et s’il a encore la force, dans un ultime sursaut d’orgueil, de filmer sa fuite effrénée dans les eaux glacées de l’océan, vous verrez, en un sublime ralenti magnifié par les filtres idoines, les terribles prédateurs se rapprocher inexorablement de leur proie et la mettre en pièces jusqu’à ce que soit tranchée, en un ultime gros plan spectaculaire, la main qui tenait courageusement l’appareil hors des flots écumeux et rougis par le sang. Si on est au sommet de l’Himalaya, ce qui peut arriver aussi, quelque part entre le Népal et le Tibet, et que les survivants, pris dans une crevasse et gravement blessés dans leur chute, n’ont pas d’autre choix que de s’entredévorer pour survivre, il y en aura toujours pour filmer pendant que les autres passent à table. Je pense aussi à la Grande Guerre (celle de 14-18 pour les ignares), époque où la photographie n’en était encore qu’à ses premiers balbutiements en noir et blanc. On dispose bien de quelques images rudimentaires, d’une netteté approximative, mais quelle perte pour l’humanité de ne pas avoir pu voir et savoir ce qui se passait réellement dans les tranchées. Les derniers poilus sont morts, et la plupart yoyotaient sérieusement dans les dernières années de leur existence, chose bien sûr tout-à-fait normale dont il ne saurait ne leur être fait grief, surtout quand on a vécu les horreurs de la guerre. Difficile néanmoins, dans ces conditions, de se fier à leur témoignage, même si on ne rendra jamais assez hommage au courage et la ténacité dont ils ont fait preuve pour empêcher le pays de tomber aux mains de ceux qui n’étaient pas encore des enfoirés de nazis mais n’allaient pas tarder à le devenir. Si Manfred von Richthofen, par exemple, alias le Baron Rouge, avait pu filmer tout ce qui s’est passé à bord de son Fokker, je pense que ces images feraient aujourd’hui encore la une du box-office. Pour ce qui est de la planète Mars, avec les transmissions satellites actuelles, on pourrait vivre le massacre en direct sans aucune difficulté, une bière à la main, confortablement installé dans le canapé de son salon, entouré de poupées sexuelles hyperréalistes en provenance de Chine ou du Japon. Une certaine idée du bonheur, un peu déviante, peut-être, mais tellement contemporaine. Il faut vivre avec son temps, nom de Dieu, et le temps est aux pédophiles qui montrent leur bite sur les réseaux sociaux, aux profanateurs de sépultures qui vont pisser sur la tombe de Badinter, aux dirigeants irrédentistes et mégalos qui font main basse sur les richesses du monde en agitant l’épouvantail d’une troisième guerre mondiale, aux tueries de masse dans les écoles et les universités, aux ballots de coke qui s’échouent sur les plages de la Côte d’Opale, aux riches de plus en plus riches, aux pauvres de plus en plus pauvres, aux cons de plus en plus cons, aux morts de plus en plus morts, aux gamines qu’on shoote de force pour les prostituer dans des chambres d’hôtel sordides, aux réfugiés qui se noient dans l’océan (ils n’ont même pas la chance de tomber sur un ballot de coke qui leur permettrait de s’offrir des vêtements secs et des vrais faux papiers), aux gens de plus en plus gros à force de bouffer de la merde (qui est en vente libre, contrairement au crack, aux sels de bain, au fentanyl et à la kétamine, très nocifs je le rappelle, alors que l’ayahusaca et les champignons magiques, produits naturels s’il en est, sont toujours là pour vous procurer voyage et dépaysement à petit prix), aux gens qui s’exhibent à moitié à poil dans des escape games débiles et autres programmes affligeants (liste non-exhaustive pour prendre pleinement la mesure de l’ampleur des dégâts : Bâtard Academy, Qui veut épouser mon fils ? Personne merci, L’Amour est dans ton cul, Norbert crétin d’office, Mariés au premier coup de bite, La Villa des culs brisés, Danse avec les porcs, The Voice : La Poisse, L’île de la perdition, Le Meilleur bon à rien, Crétin Express, Astikoh-Lanta, Top Larbin, Cauchemar dans les latrines avec le trois-quarts centre étoilé de la cuisine française, etc, etc, etc) suivis par des millions de téléspectateurs qui sont à priori des gens normaux comme vous et moi (j’espère bien que non, sinon tout est foutu), j’en passe et des meilleures, des vertes, des pires et des pas mûres.

Donc, comme je le disais avant d’expliquer les raisons de cet emportement, j’ai brandi ma plaque et hurlé à la cantonade : POLICE !!! CIRCULEZ, Y A RIEN À VOIR !!!

Et j’ai ajouté, ce qui n’était pas absolument nécessaire techniquement mais très satisfaisant au niveau du bien-être et l’épanouissement personnel : FOUTEZ LE CAMP, BANDE DE CHAROGNARDS !!!!!!

La plupart de ces enfoirés ont foutu le camp, conformément à mes instructions on ne peut plus claires, mais certains, plus tenaces et affamés que d’autres, ont continué à rôder dans les parages, se planquant derrière les arbres et les tas d’ordures pour s’adonner à leur vice. Je précise que, suite à une grève prolongée des agents de propreté urbaine, plus communément appelés éboueurs, les poubelles vomissaient leurs entrailles et l’espace public s’était transformé en décharge à ciel ouvert, avec diverses conséquences, dont au moins deux passablement préoccupantes sur le plan sanitaire et social : d’une part attirer toute une faune de bestioles peu regardantes sur l’état de fraîcheur de leur alimentation, d’autre part empuantir l’atmosphère de façon significative. Même les oiseaux, qui d’ordinaire se plaisaient à faire des vocalises dans la ramure environnante, avaient déserté la place.

Les pompiers, les flics, les secours sont arrivés, toutes sirènes hurlantes.

Desmarais, assis au volant de sa bagnole comme si de rien n’était, à ceci près qu’il était tout de même légèrement disloqué de toute part, ne présentait aucune blessure apparente. L’idée qu’il ait pu survivre à l’accident était pour moi d’une extrême contrariété. J’étais, vous le savez, bien décidé à le buter. J’avais beau tourner et retourner la question dans tous les sens, je ne voyais aucune circonstance atténuante à accorder à cette ordure. Même s’il avait eu une enfance horriblement malheureuse, si son père l’avait élevé dans le culte du Troisième Reich et obligé à apprendre Mein Kampf par cœur dès son plus jeune âge, il devait disparaître sans laisser de traces. Et comme il y avait de fortes chances que sa femme partage ses opinions, qui n’étaient d’ailleurs pas des opinions mais seulement une expression idéologisée de la haine, la bêtise et la vulgarité, il aurait été prudent de l’exterminer dans la foulée. Même chose pour ses gosses, auxquels qu’il avait vraisemblablement inculqué, comme son père l’avait fait avec lui, les valeurs de l’idéologie nazie, valeurs qu’eux-mêmes se feraient un devoir de transmettre à leur progéniture. Mais peut-être les avait-il découvertes lui-même, comme un grand, après une quelconque tragédie qui l’avait laissé profondément meurtri et gorgé d’amertume dans un monde cruel où il n’avait plus sa place. Imaginez par exemple que sa mère, après avoir pris conscience (un peu tard, malheureusement) que son mari n’était qu’une brute, un alcoolique et un crétin antisémite, raciste et xénophobe, ait cédé aux tentations de l’adultère avec un employé du gaz venu relever les compteurs (et les jupes par la même occasion). C’est déjà très embêtant, car chacun sait que la brute raciste et xénophobe est très à cheval sur les valeurs morales et la fidélité conjugale (surtout en ce qui concerne sa femme), mais si l’employé du gaz en question cumule avec une rare insolence toutes les tares les plus rédhibitoires aux yeux de ladite brute, on coure droit à la catastrophe. Même s’il avait été le parfait aryen dans toute sa splendeur, la chose aurait été difficile à avaler. Mais imaginez, ne serait-ce qu’un instant, qu’il soit juif, arabe, communiste, franc-maçon, bisexuel, et, pour couronner le tout, affligé de quelque handicap mineur ou discrète malformation congénitale (un sexe anormalement développé, par exemple, et capable de soutenir une érection des heures durant sans montrer le moindre signe de faiblesse). Le mari trompé, quand il découvre le pot aux roses (en l’occurrence des roses fanées dont les tiges putréfiées baignent dans l’eau croupie), entre dans une fureur noire. Après être allé se recueillir une dernière fois sur la tombe de Rudolf Hess à Wunsiedel, en Bavière, il endosse son plus bel uniforme de la SS, vérifie que le chargeur de son semi-automatique Luger P08 est plein à craquer, puis se dirige d’un pas ferme vers la chambre d’hôtel où il sait que les amants ont l’habitude de se retrouver pour donner libre cours à leur frénésie sexuelle. Là, il les trouve au lit, en train de forniquer comme des bêtes, et la vision de cette espèce d’animal velu en train de prendre sa femme en levrette le pousse à commettre l’irréparable. Il l’abat d’une balle en pleine tête, chose qui, et c’est pour dire à quel point cet employé du gaz est anormalement constitué, n’entrave en rien sa vigueur sexuelle. Même mort, la tête à moitié arrachée, il continue à limer furieusement la malheureuse couverte de sang et de cervelle qui hurle tant et plus. Désireux de mettre un terme à ses souffrances, notre homme l’abat à son tour d’une balle dans la tête. Mais le monstre, loin de s’arrêter à ce genre de contretemps, continue à besogner le cadavre comme si de rien n’était. Le malheureux, qui n’est alors pas loin de perdre la raison, vide son chargeur sur le zombie qui est toujours en train de s’acharner sur la dépouille de sa volage moitié. Voyant que tout cela est sans effet, et comprenant que le monstre, qui le dévisage en se léchant de façon obscène les babines ensanglantées, et au sujet duquel j’ai indiqué précédemment qu’il était à voile et à vapeur, ne va pas tarder à s’en prendre sexuellement à lui, il recharge son arme et se fait sauter la cervelle. Et, détail sordide que je ne pas encore eu le courage de vous révéler, Desmarais père ne s’est pas rendu seul à cette expédition punitive qui vient de tourner au cauchemar. Non, il a emmené avec lui son fils Noé, treize ans moment des faits, afin qu’il soit témoin de l’indignité de sa mère. Le petit Noé, donc, a assisté à tout. Butalement plongé jusqu’aux ouïes dans la triste réalité d’une existence qu’il n’avait pas choisie, il a vu sa mère adorée mourir des mains de son père, puis son père se donner la mort pour échapper aux assauts de l’amant de sa femme transformé en zombie. Lui-même, Noé, n’a dû son salut qu’à une fuite effrénée dans les couloirs de l’hôtel, avant que le zombie ne soit finalement carbonisé au lance-flammes par le GIGN. On peut comprendre, et je suis tout disposé à le faire, que de tels événements aient influencé durablement le cours de son existence et précipité sa chute. On peut comprendre que sa raison en ait été ébranlée, et qu’il ait développé un certain penchant pour les arts du feu et la pyrotechnie. Néanmoins, quels que soient le trauma originel, l’éducation pourrie et l’état de dégradation mentale auquel il était parvenu, on ne pouvait continuer à le laisser sévir impunément. Face à un détraqué de ce calibre, on ne pouvait en aucun cas se permettre de miser sur la clémence ou les bons sentiments. On aurait pu le mettre en prison, bien sûr, et il aurait passé le restant de ses jours à ruminer sa vengeance. Mais à quoi bon, puisqu’il était irrécupérable, et que même s’il l’avait été, ça aurait de toute façon été un boulet accroché au pied de la société. Même s’il avait cessé de foutre le feu aux gens, il aurait continué à répandre ses idées malsaines et participer à la déliquescence ambiante. On aurait aussi pu le foutre en HP et le bourrer de médocs jusqu’à sa mort. Dans un cas comme dans l’autre, à part faire bosser les toubibs et les gardiens, le retour sur investissement aurait été nul. En humaniste convaincu, je suis et ai toujours été un farouche adversaire de la peine de mort, en ce sens que la mort n’est pas une peine mais une nécessité, au mieux un incident de parcours. C’est sur la mort des uns, le terreau de leur cadavre, que fleurit la vie des autres. Et je suis d’avis que la société ne doit en aucun cas se salir les mains en effectuant les sales besognes nécessaires à sa survie. C’est à certains citoyens, plus dévoués que d’autres à la cause commune, de s’y coller, et la justice n’a rien à voir là-dedans. On peut m’objecter que ce n’est pas à moi de décider qui doit ou non cesser de respirer. Ce serait vrai si je décidais de quoi que ce soit, mais ce n’est pas le cas. Je ne décide de rien du tout, pas plus que le marteau ne décide de taper sur la tête du clou. Je fais ce que j’ai à faire en toute décontraction, en toute humilité. J’essaie de le faire bien, en bon artisan respectueux de la tradition et de ses outils, à l’image de mon père qui, après l’armée, avait œuvré un temps comme tueur à gage pour un caïd de la drogue, avant de partir planter ses choux dans des contrées plus vertes. Même s’il était au service de l’une d’entre elles, et non des moindres, mon père n’exécutait que des ordures, raison pour laquelle j’ai toujours considéré qu’il exerçait une profession de salubrité publique. D’autant qu’il travaillait proprement, sans excès de zèle, ce qui explique sans doute que personne ne soit jamais revenu de l’au-delà pour se plaindre de ses agissements. Comme lui, à ceci près que je fais dans le bénévolat et n’attends aucune espèce de reconnaissance de la part de qui que ce soit, je supprime certaines personnes peu fréquentables parce qu’il faut bien que quelqu’un le fasse, et que cette mission m’a été dévolue par des forces supérieures sur lesquelles je n’ai aucune autorité. Et non, je ne suis pas ce genre de cinglé qui entend des voix lui murmurer dans le creux de l’oreille qu’il doit prendre les armes et endosser sa plus belle, blanche et étincelante armure pour aller défendre la veuve et l’orphelin. Personne ne me parle, ne me dit quoi faire. J’agis en toute simplicité, aussi naturellement que l’agneau tète sa mère avant que le loup ne la croque, en toute simplicité lui aussi.

Cela dit, pour en revenir à la situation présente, je n’avais rien contre le fait que le destin fasse le boulot à ma place. Après tout, je n’étais pas responsable de l’existence de cette tête de nœud, et n’avais donc aucune raison particulière de m’infliger la corvée de le faire disparaître.

Comme beaucoup d’urgentistes souffrant de handicap visuel, le docteur Sébastien Charrier portait des lunettes Lazarus Pierson en titane avec verres antibuée, antireflet, antitout, charnières flexibles et plaquettes nasales antidérapantes, spécialement conçues pour procurer le maximum de confort aux membres du corps médical. L’urgentiste de terrain, confronté à des conditions de travail souvent difficiles, se doit d’être parfaitement équipé pour donner le meilleur de lui-même. Charrier était le genre d’homme qui s’entretient physiquement et intellectuellement pour être toujours au top des ses capacités. Il était marié et père de trois enfants. Sa femme le trompait avec son gynécologue, le docteur Rémi Durand, qui était aussi le meilleur ami de son mari, mais il s’agit là d’un cas de figure tellement banal que de nos jours plus personne n’y prête attention. D’autant qu’on ne pouvait pas dire à proprement parler qu’elle le trompait avec le docteur Rémi Durand, puisque le docteur Charrier était parfaitement au courant de ses agissements. Non seulement il était au courant, mais lui-même couchait avec la femme du docteur Durand, son meilleur ami, lequel meilleur ami était bien entendu lui aussi parfaitement au courant des agissements de son épouse. Charrier était couvert de poil, comme un singe, et sa libido était semblable à une bête sauvage qu’il avait toutes les peines du monde à tenir en laisse. Déjà, quand il était petit, ses parents faisaient venir à la maison des amis avec lesquels ils organisaient des orgies sexuelles où tout le monde était convié, y compris les enfants. Il ne s’agissait pas d’inceste à proprement parler, mais le fait est que sa mère et la plupart de ses frères et ses sœurs l’avaient sucé à de nombreuses reprises. Même son grand-père, qui ne crachait pas sur les petits garçons à l’occasion, lui avait plus d’une fois sucé la bite, tout comme lui-même, dans un souci d’équité et de retour d’ascenseur, avait plus d’une fois sucé la bite et avalé la semence grand-paternelle. C’est dans ce contexte qu’il avait décidé, comme son père, lui-même chirurgien de renom, et son grand-père avant lui, généraliste unanimement apprécié pour ne pas dire vénéré, de consacrer sa vie à sauver celle des autres. Mais quand on passe le plus clair de son temps à côtoyer la mort, il faut bien se détendre un peu en rentrant chez soi. Tout le monde l’avait bien compris, raison pour laquelle des partouzes étaient régulièrement organisées chez les Charrier ou les Durand, tous deux possesseurs de vastes maisons de maîtres dans la campagne environnante. Les Charrier, par exemple, étaient les heureux propriétaires d’un château du XVIIIe (quinze pièces, parc paysager de deux hectares, piscine et dépendances), dans le Loiret. Naturellement, quand on le voyait débarquer comme ça sur les lieux d’une catastrophe, on ne pouvait pas se douter que le docteur Sébastien Charrier était obsédé par les jeunes et jolies infirmières qui gravitaient autour de lui. Tellement qu’il lui arrivait de leur proposer, pour arrondir leurs fins de mois (tout le monde sait que le métier d’infirmière, qui exige non seulement des compétences médicales, mais aussi beaucoup de patience et d’empathie face à des gens en situation de stress et de faiblesse extrême, est très insuffisamment rémunéré et ne suscite plus guère de vocations), de venir passer le week-end avec lui et ses amis à la campagne. Tout ce qu’elles avaient à faire était de venir dans leur tenue de travail, autrement dit en petite tenue sous leur blouse (en coton de préférence, dessous sexy appréciés), et de s’occuper gentiment des pensionnaires. Il leur était également demandé de ne faire aucune différence entre homme et femme, de quelque âge, condition sociale (elles n’avaient pas de souci à se faire, il n’y aurait que des gens bien élevés et très à l’aise financièrement) ou origine ethnique que ce soit, les séminaires en question étant placés sous le signe de la confraternité intergénérationnelle, culturelle et sexuelle entre les peuples. Toute pratique, tant qu’elle ne portait pas (gravement, on ne pouvait jamais totalement exclure tel ou tel suçon ou légère trace de morsure, lesquels, de toute façon, feraient l’objet d’un dédommagement adapté au préjudice) atteinte à l’intégrité physique de la personne, devait être acceptée et satisfaite dans la joie et la bonne humeur la plus exubérante. Si elles respectaient à la lettre ces quelques directives, leur train de vie pourrait connaître une embellie spectaculaire. À elles le coupé sport, les robes de princesses, les bijoux chatoyants, les restaurants étoilés et les voyages en première classe aux Seychelles et à Bora Bora. La plus stricte discrétion était bien entendu de mise, sachant que la plupart des gens sont bien trop étroitement cadenassés dans un carcan de valeurs morales d’un autre âge pour admettre que certains aient impunément accès à des plaisirs auxquels ils n’auront jamais droit.

Jusqu’au jour où l’une des infirmières en question, une certaine Alena Benesch (blonde, le teint pâle, avec des grands yeux noisette et un visage d’ange tombé du ciel), est allée porter plainte au commissariat le plus proche.

Et devinez qui se trouvait dans ce commissariat le plus proche ?

Votre serviteur.

Et c’est à lui qu’on a refilé la patate chaude.

Vêtu de mon plus chouette costume, mon plus charmant sourire aux lèvres, accompagné d’une poignée de flics en uniforme pour renforcer l’aspect solennel de la démarche, je suis allé sonner à la porte de Charrier, dans les beaux quartiers, là où toutes les baraques, ceintes de hauts murs hérissés de tessons de bouteilles ou de clôtures électriques, sont équipées d’alarme dernier cri qui retentissent au moindre frémissement, déclenchant aussitôt le bouclage de la zone et le parachutage sur site des meilleurs éléments du RAID et du GIGN. Des maîtres-chiens lourdement armés patrouillent dans les rues, et des snipers surveillent H24 les alentours du haut de miradors placés à tous les coins de rues.

Il convient également, si l’on ne souhaite pas finir ses jours en fauteuil roulant, de se méfier des pièges à loup dans les parcs et mines antipersonnel dans les allées de jardin.

Eh bien croyez-le ou non, j’ai été reçu comme un cheveu sur la soupe.

C’est tout juste si on ne pas claqué la porte au nez.

En fait, ça n’aurait pas été pire si j’avais été un huissier de justice ou un témoin de Jéhovah .

Je parle d’un témoin de Jéhovah mâle, bien sûr, avec les oreilles décollées et un physique d’expert-compatible, ou une bonne femme de cinquante piges mal fagotée avec des dents jaunes et un fort strabisme divergent.

Parce que je vous fiche mon billet que si j’avais été un témoin de Jéhovah femelle avec la plastique de Scarlett Johansson, Jenna Ortega, Rachel McAdams ou Gal Gadot, le tout emballé dans une blouse d’infirmière ultra sexy avec les boutons prêts à exploser sous la pression des formes généreuses qu’elle s’efforçait désespérément de contenir, on m’aurait déroulé le tapis rouge, offert des fleurs, un verre de Champomy, et aussitôt proposé un emploi à plein temps comme garde d’enfants à domicile, gouvernante, femme de ménage, jardinière de légumes ou n’importe quoi d’autre pour me garder à portée de braguette et tenter de m’entraîner corps (surtout) et âme dans la spirale du vice.

Poussé dans ses retranchements, Charrier a admis qu’il lui arrivait de recevoir des filles dans son château du XVIIIe, mais que celles-ci étaient majeures, traitées avec les mêmes égards que les autres invités, sans que ne soit jamais exigée aucune contrepartie de leur part.

À l’époque j’étais comme toi, lecteur, jeune et fougueux inspecteur, épris de justice et de liberté, révolté à l’idée que les puissants abusent de leurs prérogatives en toute impunité. Je rêvais, à cheval sur mon blanc destrier, engoncé dans ma plus resplendissante armure et coiffé de mon plus bel heaume, de voler au secours de la veuve et l’orphelin en proie aux persécutions de nobles dévoyés, d’usuriers pervers et de crapules sans foi ni loi. D’étranges rumeurs circulaient au sujet du château de la Frétoise, près de Montargis, entre Préfontaines et Corquilleroy.

Quelques années auparavant, une fille avait disparu sans laisser de traces dans le secteur. Son vélo, ainsi qu’une de ses chaussures et une pince à cheveux, avaient été retrouvés sur la route de Nargis, au lieu-dit du Bois au Notaire. La pauvre enfant venait tout juste de fêter ses dix-sept ans. Les recherches n’ont rien donné, mais les allées et venues nocturnes du côté de la Frétoise avaient éveillé les soupçons de plusieurs personnes du voisinage, même si l’endroit est particulièrement isolé. Le propriétaire des lieux, un certain Sébastien Charrier, avait été entendu. Il avait déclaré n’être au courant de rien, mais s’était montré des plus désagréables, traitant les enquêteurs avec dédain, n’hésitant pas à leur faire sentir qu’ils n’étaient que des moins-que-rien indignes de sa compagnie, des pauvres types auxquels il ne se résignait à adresser la parole que contraint et forcé. Le même sort m’a été réservé lorsque je l’ai entendu à mon tour, suite au dépôt de plainte de l’infirmière qui affirmait avoir été violentée durant son séjour au château. Un soir, par exemple, après un repas bien arrosé, on l’avait trainée de force dans le salon et obligée à se dévêtir entièrement devant les invités, tous complètement bourrés et incapables de la moindre retenue. Elle se souvient que certains, à commencer par le docteur Charrier lui-même, avaient le sexe à l’air et se masturbaient outrageusement devant elle. C’est ce même docteur Charrier qui lui avait ensuite copieusement aspergé les parties génitales avec de la crème chantilly, avant de faire venir Hermann, son dogue allemand, pour qu’il nettoie la zone à grands coups de langue sous les éclats de rire et les encouragements obscènes de l’assistance. Ames sensibles s’abstenir. Je vous laisse néanmoins imaginer, si toutefois vous en avez le courage, la terreur de cette jeune femme, livrée à la frénésie d’un monstre de près de quatre-vingt-dix kilos qui aurait très bien pu ne pas se satisfaire de cet amuse-gueule et décidé de passer sans plus tarder au plat de résistance. Quelle horreur ! Si vous ajoutez à cela le caractère extrêmement humiliant et dégradant de la scène, vous transpirez à grosses gouttes et prenez aussitôt la mesure du traumatisme subi.

Naturellement, cette enflure de Charrier a nié les faits avec la dernière énergie, affirmant que la fille mentait pour se faire du fric sur son dos. Selon lui, ce tissu de conneries grosses comme un troupeau d’éléphants risquait de jeter un voile crasseux sur la blancheur immaculée de sa réputation. En conséquence, il a menacé de porter plainte en retour pour diffamation. Son avocat était le genre de Pavarotti du barreau capable de faire passer un curé pédophile pour un honnête serviteur de Dieu. Il a également claironné qu’il allait s’empresser de solliciter ses relations dans les plus hautes sphères de l’État, afin que celles et ceux qui avaient prêté une oreille complaisante à ces élucubrations soient sanctionnés à la hauteur de leurs agissements. Peu de temps après, j’ai été convoqué chez le dirlo qui m’a gentiment expliqué que Charrier était un type comme ça, un héros des temps moderne qui ne comptait pas ses heures pour sauver la vie des gens. Charrier n’était pas le Gilles de Rais de Montargis, le Barbe Bleue de la Frétoise, l’Ogre de Préfontaines qui nourrit ses molosses avec de la chair humaine, et il fallait séance tenante arrêter de l’emmerder avec cette histoire de dogue allemand bouffeur de chatte à la chantilly. Tout cela ne tiendrait pas une seconde devant les tribunaux. D’autant que l’infirmière en question, même si elle en avait toutes les apparences, était loin d’être une princesse de conte de fée, aussi pure et innocente que la rosée du matin ou la fleur des champs fraîchement éclose. Ses états de service faisaient mention d’un certain nombre de délits qui cadraient assez mal avec le numéro de novice de couvent des Ursulines qu’elle avait tenté de nous faire avaler : excès de vitesse, conduite en état d’ivresse et usage de stupéfiants.

Résultat des courses : Alena Benesch a retiré sa plainte et Charrier s’en est sorti sans une égratignure, lavé de tout soupçon, blanc comme la neige qui recouvrait jadis les vastes plaines de notre enfance.

Naturellement, Benesch a été gentiment remerciée par la Direction et priée d’aller exercer ses talents de garde-malade dans un autre établissement, si possible dans une autre galaxie, à des années-lumière de la planète Terre. Charrier, grand seigneur, lui a versé un petit pécule pour l’aider à tenir le coup en attendant de trouver un nouveau job. Pas rancunier pour un sou, il lui a glissé dans le creux de l’oreille qu’il y aurait toujours un bol de soupe pour elle à la Frétoise. Et pas seulement de la soupe : il avait fait le plein de chantilly, et Hermann, au bord de la dépression depuis son départ précipité, la réclamait avec insistance.

Tu te demandes peut-être, ami lecteur dont je connais la sagacité et la soif de transparence, comment je suis au courant de tous ces détails concernant la vie privée de Charrier, notamment son enfance dévoyée au sein d’une famille sexuellement dysfonctionnelle ?

La réponse est simple : quand j’enquête, j’enquête, ce qui signifie que je passe au crible tous les éléments ayant trait à l’enquête en question, y compris les plus insignifiants. J’effectue des recoupements, fouille dans les tiroirs, les archives, les boîtes à chaussures, les armoires à linge et les cartons à chapeau, épluche les livrets de comptes, les carnets de liaison, les journaux intimes, visite les jardins secrets, déterre les cadavres et plonge tête baissée dans les profondeurs existentielles des protagonistes de l’affaire. Ce travail de fourmi, long et ingrat, me permet de débusquer sans coup férir l’ivraie qui se dissimule au sein du bon grain. Ainsi, Charrier avait des frères et sœurs avec lesquels il n’était peut-être pas forcément dans les meilleurs termes du monde, au point qu’il ne parlait quasiment plus à certains d’entre eux. C’est vers eux qu’il fallait se tourner pour obtenir de précieuses informations sur un frère qui ne leur inspirait plus que mépris et répulsion. Il devenait alors possible de reconstituer pas à pas le parcours d’un Charrier, comprendre comment avait pu s’échafauder dans son cerveau malade cette passion dévorante pour les dogues allemands, les infirmières et la chantilly.

Voilà comment, ami lecteur, j’ai pu apprendre toutes ces choses passionnantes sur la vie du docteur Sébastien Charrier, et te donner ainsi l’impression que j’étais dans le secret des dieux, alors que tout cela n’était finalement rien d’autre que le fruit d’un travail lent et minutieux. Cela dit, oui, je peux aussi être un tout petit peu dans le secret des dieux, car, sauf le respect que je te dois, je ne suis pas non plus obligé de tout te dire. C’est encore moi le seul maître à bord de ce rafiot littéraire qui vogue sans relâche sur les flots écumeux de la syntaxe, la rhétorique, la synecdoque, l’oxymore, la litote, la métaphore, l’antonomase, l’allégorie, le truisme et le paradoxe. Donc oui, je ne suis pas obligé de tout de dire, et peux choisir à mon gré le moment de te dévoiler ou pas tel ou tel aspect de l’histoire qui nous occupe.

Charrier m’a immédiatement reconnu.

Il a sorti de sa poche une seringue remplie d’un liquide verdâtre qu’il a essayé de me planter dans le bras.

Je ne sais toujours pas quel était ce liquide verdâtre de merde, de la pisse de rat, du jus de cadavre ou autre chose, mais quelque chose me dit que je serais mort dans d’atroces souffrances s’il avait réussi à me l’injecter.

J’ai esquivé le coup, une courte lutte s’est engagée, à l’issue de laquelle il s’est retrouvé au sol.

J’ai tenté de le ramener à de meilleurs sentiments avec un bon coup de latte dans les parties, mais il a réussi à m’attraper le pied, le tordre et me faire perdre l’équilibre.

Je me suis retrouvé au sol à mon tour, la cheville endolorie, pendant que Charrier se relevait d’un bond avec l’élégance d’un sportif de haut niveau.

Greg s’est jeté sur lui pour l’étrangler, mais Charrier l’a vu arriver et reçu avec une série de coups qui l’ont laissé sans voix, notamment le crochet au foie et l’uppercut à la mâchoire, tous deux assénés avec une précision d’autant plus redoutable que l’anatomie n’avait aucun secret pour lui.

Greg, faisant sienne la devise de l’inspecteur Harry Callahan dans Magnum Force, «~l’homme sage est celui qui connaît ses limites~», n’a pas jugé utile de se révéler après avoir mordu la poussière.

Dans une autre vie, Charrier avait été champion de France universitaire de boxe anglaise. Il avait perdu en vitesse, son jeu de jambes n’était plus ce qu’il était et ses réflexes s’étaient quelque peu émoussés, mais il lui restait encore largement de quoi faire illusion sur un ring.

Pas besoin d’être diplômé de l’ESSEC et encore moins d’être le principal actionnaire de la Royal Caribbean Cruises Ltd. pour comprendre que la situation était en train de se barrer méchamment en couille.

Il m’a fallu développer des trésors de résistance à la douleur et de volonté farouche de survivre dans ce monde cruel qui est le nôtre pour réussir enfin à retrouver cette position qui, au même titre que le cheval, la femme et le dromadaire, compte au rang des plus belles conquêtes de l’homme, je veux bien sûr parler de la bipédie. Grâce à elle, nous avons tout le loisir de conserver la pleine et entière jouissance de nos membres supérieurs, nos mains en particulier, ce qui nous a permis d’accomplir des miracles qui seraient à tout jamais restés hors de portée si nous avions été contraints de nous déplacer à quatre pattes.

C’est alors que j’ai vu Charrier se diriger à grands pas vers moi, les poings serrés et les traits horriblement déformés par la haine, l’envie de me détruire entièrement, me mouliner, me torréfier, me hacher menu, me réduire en cendre, m’éradiquer définitivement de la surface de la Terre (510 millions de kilomètres carrés tout de même, dont 70\% de flotte, rivières, lacs et profondeurs océaniques peuplés de créatures aussi étranges que primitives).

La situation était d’autant plus préoccupante qu’il avait un scalpel à la main, instrument dont les qualités de tranchant ne sont plus à démontrer.

Greg au sol, et apparemment bien décidé à y rester, je ne pouvais plus compter que sur moi.

Et Manu, bien sûr, fidèle serviteur de la Loi qui ne m’avait jamais trahi, ne s’était jamais enrayé, n’avait jamais connu la moindre avarie en plus de vingt ans de bons et loyaux services, vingt longues années d’épreuves traversées côte à côte, la main (ou la crosse, si vous préférez) dans la main, le doigt sur la détente.

Il y a des moments, dans l’existence, où l’heure n’est plus aux conciliabules, atermoiements et autres vaines tergiversations.

Quand l’ennemi fond sur vous, l’écume aux lèvres, et que vous êtes clairement en infériorité numérique, le mieux est encore de faire feu sans se poser de question.

C’est ce que j’ai fait, à deux reprises.

Le premier projectile a raté sa cible, à savoir la tête de Charrier, mais le second a fait mouche.

Son crâne s’est ouvert comme un œuf à la coque, et sa cervelle a été projetée dans les airs, tel un drôle d’objet volant mal identifié.

Elle a effectué quelques tours sur elle-même, avant d’atterrir sur l’épaule d’un bonze qui circulait en trottinette électrique sur le trottoir d’en face. Soyons clair : j’ai eu beau poser la question à tous les témoins qui avaient assisté à la scène, et dieu sait qu’il y en avait un paquet, aucun n’a été en mesure de m’expliquer ce que ce foutu bonze faisait là, en toute illégalité qui plus est, ce qui n’est à priori pas dans le style des bonzes, toujours respectueux des lois, adeptes de la discrétion et désireux de se fondre dans la foule (même si l’espèce de soutane orange dans laquelle ils se trimballent n’est sans doute pas le meilleur moyen d’y parvenir), sachant qu’on n’avait pas vu de bonze dans le secteur depuis au moins trois ou quatre siècles, en admettant qu’on en ait jamais vu un, raison pour laquelle personne ne s’attendait à en voir un, et encore moins au guidon d’une putain de trottinette électrique, le bonze n’étant généralement pas pressé et préférant faire usage de ses pieds, à l’ancienne, pour se transporter d’un point à un autre.

Le bonze a tourné la tête, vu la cervelle sur son épaule, poussé un cri (oui, les bonzes aussi poussent des cris, peut-être pas autant que les gens normaux, les anachorètes ou les membres des autres congrégations, mais ils en poussent aussi), tenté de s’en débarrasser, perdu le contrôle de sa trottinette, réussi de justesse à éviter un bac à fleurs, avant d’aller s’écraser sur une borne anti-stationnement, effectuer un vol plané d’anthologie et se retrouver les quatre fers en l’air au beau milieu de la chaussée. Tandis qu’il tentait de se relever, un bus est arrivé à pleine vitesse et lui a roulé sur la tête, laquelle a explosé comme une vieille citrouille pourrie en répendant son contenu sur le bitume.

Coupez !

En fait non, les choses ne sont exactement passées de cette manière.

Vous le savez comme moi, la réalité est une tambouille désespérément fade. Si on veut lui donner un semblant de saveur, il ne faut pas lésiner sur les épices. C’est d’ailleurs ce que la plupart des gens, dès qu’ils ont un moment de libre, s’emploient à faire.

Le mensonge, par exemple, ou le fait de travestir plus ou moins subtilement la vérité, sont des pratiques courantes en la matière.

Quand Roger Borniche (ex-comique troupier reconverti en flic à la Sûreté puis écrivain à succès) raconte comment il a serré Émile Buisson, alias l’ignoble Fatalitas, ennemi public numéro 1, dans la petite auberge de Normandie où ce dernier était tranquillement en train de déjeuner (pâté de campagne, avec salade et cornichons, tripes au calva, camembert et tarte aux fraises, il s’agit d’un scoop mondial puisque le contenu de ce déjeuner n’avait encore jamais été dévoilé, on se demande d’ailleurs bien pourquoi quand on sait à quel point les gens sont friands de ce genre de détails), il enjolive copieusement la scène, allant jusqu’à faire croire que c’est sa propre femme, Martine, qui a passé les menottes à Buisson (alors, je le rappelle, que deux autres flics, dont c’est le métier de menotter les gens, étaient présents dans la salle). Sacré Roger ! Non, en réalité, Martine n’était là que pour endormir la méfiance de Buisson, individu extrêmement dangereux en permanence sur le qui-vive, et en aucun cas risquer de prendre un mauvais coup en procédant elle-même à son arrestation.

Moi-même, en l’occurrence, qui n’ai rien à envier à Roger Borniche, à ceci près (manquerait plus que ça !) que je n’ai jamais été comique troupier, chansonnier ou agent de sécurité dans un grand magasin (il n’y a pas de sot métier, je vous l’accorde, mais ça montre bien à quel point l’approche du maintien de l’ordre était différente en ce temps-là, même si aujourd’hui encore il n’est pas rare que d’anciens acteurs de seconde zone accèdent aux plus hautes fonctions de l’État), ne rechigne pas à mettre un peu de piment dans le ragoût fadasse de l’existence, quitte à rétablir, une fois la supercherie découverte, l’exacte vérité des faits.

Dans le cas présent, je suppose que la présence d’un bonze en trottinette a dû sembler bizarre aux plus méfiants d’entre vous, d’autant qu’il n’y a aucun monastère dans les environs, et que même s’il y en avait un, il n’est pas du tout certain que les moines auraient l’autorisation d’utiliser un tel moyen de locomotion.

Je vous rassure tout de suite : il n’y en avait pas.

Pas à ma connaissance, en tout cas.

Par contre, le docteur Sébastien Charrier, lui, était bien là.

Je lui ai dit : Docteur Charrier ! Si je m’attendais à vous trouver ici !

\textsc{Charrier} : On se connaît ?

\textsc{Moi} : Vous ne vous rappelez pas ?

\textsc{Charrier} : Me rappeler de quoi ? On s’est déjà vu quelque part ?

\textsc{Moi} : L’affaire Alena Benesch, ça vous dit quelque chose ?

\textsc{Charrier} : Attendez voir… Mais oui, bien sûr, cette petite pute qui avait essayé de me faire chanter en prétendant que je lui avais fait bouffer la chatte par Hermann, mon dogue allemand !

\textsc{Moi} : Oui. C’est moi qui étais chargé de l’enquête. Il va bien, au fait ?

\textsc{Lui} : Qui ? Hermann ?

\textsc{Moi} : Oui.

\textsc{Lui} : La pauvre bête est morte de chagrin il y a quelques années de cela. Je crois qu’elle s’était beaucoup attachée à cette petite. Le coup de foudre existe aussi chez les animaux, vous savez.

\textsc{Moi} : J’ignorais.

\textsc{Lui} : Oui, je sais, on pense toujours que ce ne sont que des brutes épaisses incapables de sentiment. Eh bien il n’en est rien, ils sont beaucoup plus sensibles qu’on ne le pense.

\textsc{Moi} : Je suis vraiment désolé.

\textsc{Lui} : C’est gentil à vous. Je l’ai enterré au fond du jardin et vais quotidiennement me recueillir sur sa tombe.

\textsc{Moi} : C’est l’avantage d’avoir une grande propriété.

\textsc{Lui} : Vous connaissez la Frétoise ?

\textsc{Moi} : J’y suis allé une fois ou deux. Très bel endroit, à la fois authentique et élégant.

Lui, manifestement ému : Oui, un joyau historique rénové avec passion au cœur d’un environnement préservé. Nous sommes actuellement en train d’aménager le colombier pour y faire une chambre d’ami. C’est une bonne idée, vous ne trouvez pas ?

\textsc{Moi} : Excellente

\textsc{Lui} : Il faudra venir dîner un de ces soirs. Vous êtes marié ?

\textsc{Moi} : Non, pas encore.

\textsc{Lui} : Une fiancée, alors. Ravissante, je suppose. Il faudra penser à nous l’amener.

\textsc{Moi} : Je n’y manquerai pas.

\textsc{Lui} : Quoiqu’il en soit, pour en revenir à cette petite garce d’Alena Benesch, je continue de penser qu’elle n’a eu que ce qu’elle méritait. Il m’est arrivé de penser que j’avais peut-être été un peu trop dur avec elle. Je ne suis pas un mauvais homme, vous savez, et toujours de mon mieux pour réparer mes torts. Si torts il y a, bien entendu. La vérité, c’est que je ne me sens coupable de rien en ce qui la concerne. Toujours est-il que l’autre jour, croyez-le ou non, elle est venue sonner à ma porte en disant qu’elle était dans une misère noire et avait besoin d’un petit coup de main.

\textsc{Moi} : Et qu’est-ce que vous avez fait ?

\textsc{Lui} : Ça reste entre nous, mais ma chère épouse, qui est une vraie salope soit dit en passant, adore me regarder baiser avec une autre femme. On a tous nos petites manies, n’est-ce pas. Alena Benesch a pris un peu de poids, c’est vrai, mais elle est encore tout à fait comestible. Je lui ai proposé de la sodomiser sous les yeux de ma femme, si elle n’avait rien contre le fait de se faire défoncer le cul par un ancien interne des hôpitaux de Paris. En échange, je pourrais essayer de faire jouer mes relations pour qu’elle retrouve un poste dans une clinique privée. J’ai un ami qui adore les femmes plutôt bien en chair, un violeur notoire qui a pour habitude d’abuser de ses patientes quand elles sont dans les vapes. Je ne sais que ça ne se fait pas, mais c’est un ami et je ne me vois pas le balancer aux flics. D’autant que la plupart d’entre elles ne se souviennent de rien à leur réveil. Je vous choque ?

\textsc{Moi} : Un peu, oui. Et celles qui se souviennent, je peux savoir ce que vous en faites ?

\textsc{Lui} : On les envoie chez le psy.

\textsc{Moi} : Un ami à vous, je suppose ?

\textsc{Lui} : Evidemment. Comme elles n’ont que de très vagues souvenirs sur lesquels elles sont incapables de mettre un nom ou un visage, le psy leur explique qu’elles sont en pleine bouffée délirante, sans doute liée à l’un ou l’autre de ces traumas d’enfance mal gérés, ou pas gérés du tout parce que totalement passés sous les radars, qui refont surface après des années d’enfouissement, comme des saletés de zombies qui refoulent du bec et tentent de vous bouffer tout cru. Mais dieu merci, on n’est plus au Moyen Âge. De nos jours, on peut être cinglé sans se retrouver en train de griller sur un bûcher. Les chercheurs bossent comme des dingues, pour des salaires de misère, et nous disposons de molécules de plus en plus sophistiquées pour remettre un peu d’ordre dans les cerveaux détraqués.

\textsc{Moi} : N’empêche qu’il abuse d’elles.

\textsc{Lui} : Oui, on peut dire ça. Mais à ce moment-là, on peut aussi dire qu’il les viole avec ses doigts en procédant aux examens d’usage.

\textsc{Moi} : Oui, enfin, ce n’est pas tout à fait la même chose. Là, les viole carrément avec sa bite.

\textsc{Lui} : Je sais, c’est moche. Très moche, même, mais il prétend avoir un meilleur diagnostic avec sa verge qu’avec les autres outils dont il dispose, trop grossiers à son goût. Je sais que c’est faux, qu’il ment, qu’il se ment à lui-même, mais il n’est pire sourd que celui qui ne veut rien entendre. J’ai beau essayer de lui ouvrir les yeux, il s’obstine dans le déni. Je lui dis : Roman (il s’appelle Roman), mon ami, je t’en supplie, va voir un psy. Un jour ou l’autre, une patiente va se réveiller pendant que tu es en train de l’ausculter avec ta bite, en tout bien tout honneur, et il va en résulter un de ces putains de scandales qui éclaboussent la profession toute entière. Pense à tes collègues, tous ces gens qui se battent becs et ongles pour que les gens se bourrent de médocs jusqu’à cent ans. Il me répond : oui, je ferais bien quelques séances d’hypnose, mais j’ai peur de me faire violer pendant mon sommeil. Je ne fais aucune confiance à tous ces enfoirés de psys ! Vous le voyez, on n’en sort pas. À propos, vous êtes toujours dans la police ?

\textsc{Moi} : Oui, plus ou moins.

\textsc{Lui} : Dans ce cas, je compte sur votre discrétion. Vous savez ce que c’est : les gens sont méchants, ils n’aiment pas les riches. Dès que vous gagnez un peu plus de pognon qu’eux, ils font tout ce qui est en leur pouvoir pour vous mettre des bâtons dans les roues. Heureusement qu’ils n’en ont aucun, sinon celui de descendre dans la rue pour agiter des banderoles et se gargariser de slogans anticapitalistes, sans quoi ils seraient pires que tous ces dictateurs qui dirigent le monde d’une main de fer. Je peux savoir ce qui s’est passé, ici ?

\textsc{Moi} : Accident de la circulation.

\textsc{Lui} : Belle boucherie !

\textsc{Moi} : Oui. Vous pensez qu’elle s’en sortira ?

\textsc{Lui} : Contusions multiples, hémorragie interne au niveau de la cavité abdominale, possible trauma crânien, j’en passe et des meilleurs. Elle est dans la coma. Nul ne peut dire quand elle en sortira, si elle en sort, et encore moins dans quel état. Vous lui vouliez quoi, à cette petite ?

\textsc{Moi} : L’interroger. J’ai de bonne raison de penser qu’elle est impliquée dans la disparition d’un collègue de travail, qui se trouve aussi être un de mes plus proches amis.

\textsc{Lui} : J’ai moi-même perdu un excellent ami.

\textsc{Moi} : Vraiment ?

\textsc{Lui} : Oui, très récemment. Il s’appelait Rémi Durand et était gynécologue.

\textsc{Moi} : Ce nom me dit quelque chose.

\textsc{Lui} : Bien évidemment, que ça vous dit quelque chose. Si vous avez enquêté sur la Frétoise, vous n’ignorez pas que Rémi y passait quasiment tous ses week-ends, plus une bonne partie des vacances scolaires et ses congés de maternité.

\textsc{Moi} : Je vois. Qu’est-ce qui s’est passé, au juste ?

\textsc{Lui} : On l’a retrouvé pendu dans son garage.

\textsc{Moi} : Pendu ???!!!!!!!!!!

\textsc{Lui} : Dans son garage, oui.

\textsc{Moi} : D’après mon expérience personnelle, qui est tout de même loin d’être négligeable, il est assez rare que les gens se pendent dans leur garage, ou alors seulement s’ils sont victimes de harcèlement sur les réseaux sociaux ou viennent d’apprendre fortuitement qu’ils sont atteints d’une maladie grave qui ne leur laisse que quelques heures à vivre. Ils ont généralement bien trop de respect pour leur voiture pour lui imposer une telle humiliation.

\textsc{Lui} : N’est-ce pas. Au lieu de ça, Rémi était en pleine forme et venait tout juste de s’offrir une Porsche Cayman GT4 RS dont il était extrêmement fier.

\textsc{Moi} : C’est troublant, en effet.

\textsc{Lui} : Sincèrement, mon ami, vous pensez vraiment qu’un type qui a une Porsche Cayman GT4 RS dans son garage a la moindre envie de mettre fin à ses jours ?

\textsc{Moi} : Quelle couleur, la Cayman GT4 RS ?

\textsc{Lui} : Jaune, avec pack Clubsport, réservoir de 90 litres et trousse de premiers secours en alcantara !

\textsc{Moi} : Une merveille.

\textsc{Lui} : Absolue ! Vous êtes comme moi, n’est-ce pas ?

\textsc{Moi} : Comment ça ?

\textsc{Lui} : Vous ne croyez pas un instant à la thèse du suicide.

\textsc{Moi} : Le suicide d’un gynéco bien dans sa peau qui vient de s’offrir une Porsche Cayman GT4 RS jaune avec pack Clubsport, réservoir de 90 litres et trousse de premiers secours ?

\textsc{Lui} : Oui, et toutes les options disponibles, l’intérieur full cuir bien évidemment, mais aussi les rétros extérieurs à capteurs de pluie, la reconnaissance des panneaux de signalisation, les tapis de sol en peau de fesse, les jantes en or massif, le système audio Bose et les ceintures de sécurité couleur chair !

\textsc{Moi} : Terrifiant ! J’ai beau tourner et retourner le problème dans tous les sens, je ne vois pas comment un type qui a tout ça dans son garage pourrait avoir la moindre envie de se suicider.

\textsc{Lui} : Et imaginez que ce même type soit marié à une jeune femme somptueuse dont il pourrait être le père ! Vous croyez vraiment qu’un tel homme aurait envie de mettre fin à ses jours ?

\textsc{Moi} : En aucun cas. Vous avez contacté le procureur ?

\textsc{Lui} : Bien évidemment, vous me prenez pour qui ! J’ai exigé qu’une expertise médico-légale soit diligentée dans les plus brefs délais, et j’ai demandé à y assister personnellement.

Moi, sortant un Hemingway Short Story de ma poche : Vous avez raison, on n’est jamais mieux servi que par soi-même. Vous fumez ?

\textsc{Lui} : Vous avez enquêté sur moi, non ?

\textsc{Moi} : Un peu, oui. Enquête de routine, vous savez ce que c’est.

\textsc{Lui} : Dans ce cas, vous devez savoir que je ne crache pas sur un petit cigare de temps à autre.

\textsc{Moi} : Et c’est tout à votre honneur. Tu en veux un aussi, Greg ?

\textsc{Greg} : Je ne voudrais surtout pas déranger.

\textsc{Moi} : Mais pas du tout, voyons, qu’est-ce que tu vas imaginer. C’est juste que comme je sais que tu ne fumes pas, ou quasiment pas, je me suis dit qu’il n’était peut-être pas complètement indispensable de te proposer un cigare.

\textsc{Greg} : Je ne fume pas en temps normal, c’est tout à fait vrai. Mais aujourd’hui, les circonstances sont suffisamment exceptionnelles pour que je fasse une exception à la règle.

\textsc{Moi} : C’est la raison pour laquelle, en dépit des éléments dont je viens de te faire part, que je me suis permis de te demander si tu voulais toi aussi un cigare.

\textsc{Greg} : Dans ce cas, je te répondrai ceci : oui, Djef, même s’il est vrai que je ne fume pas ou quasiment pas, c’est avec le plus grand plaisir que je vais accepter le cigare que tu m’offres si gentiment.

\textsc{Moi} : Bien. On en était où, nous ?

\textsc{Charrier} : Vous veniez de me proposer un cigare.

\textsc{Moi} : Que vous aviez accepté, c’est bien ça ?

\textsc{Lui} : C’est bien ça.

\textsc{Moi} : Et avant ?

\textsc{Lui} : Avant, on était en train de parler de la mort suspecte de mon ami Rémi Durand, gynécologue ayant pignon sur rue qui n’avait aucune raison de mettre fin à ses jours, et ce d’autant moins qu’il était marié à très jolie fille beaucoup plus jeune que lui et venait tout juste de s’offrir une Porsche Cayman GT4 RS jaune avec pack Clubsport, réservoir de 90 litres et trousse de premiers secours en alcantara, le nec plus ultra en matière de chic automobile.

\textsc{Moi} : Vous pensez qu’on l’a tué ?

\textsc{Lui} : Je ne vois pas d’autre explication.

\textsc{Moi} : Il s’agit peut-être d’un accident.

\textsc{Lui} : Vous plaisantez ?

\textsc{Moi} : On ne sait jamais. Imaginez un type qui décide d’aller dans son garage pour bricoler un truc au plafond, accrocher une corde, par exemple. Il monte sur un tabouret, et, pour une raison ou pour une autre, perd l’équilibre et se retrouve avec la corde enroulée autour du cou. Il se débat, tente désespérément de se raccrocher au tabouret. Mais il donne un coup de pied dans le tabouret, le tabouret tombe, et notre homme se retrouve pris au piège.

\textsc{Lui} : Ridicule !

\textsc{Moi} : Ou alors, il a peut-être eu, pendant un bref instant, l’idée de mettre fin à ses jours. On croit connaître ses amis, et on découvre parfois qu’ils nous ont caché des choses pendant des années. François Vérove, alias «~le Grêlé~», a vécu pendant trente-cinq sans attirer les soupçons. Ancien gendarme, bon père de famille, qu’est-ce que vous croyez qu’il s’est passé quand ses proches ont appris qu’il s’agissait en fait d’un tueur en série pédophile de la pire espèce, responsable d’au moins une bonne demi-douzaine de meurtres, et sans doute beaucoup plus, la liste exacte de ses victimes n’ayant jamais pu être établie avec certitude ?

\textsc{Lui} : Il sont tombés des nues, je suppose.

\textsc{Moi} : Peut-être que votre ami Rémi vous cachait des choses, lui aussi. Vous seriez surpris d’apprendre le nombre de gens qui ont une double vie.

\textsc{Lui} : Vous insinuez que Rémi était un tueur en série pédophile ?

\textsc{Moi} : Pas le moins du monde. Enfin, on ne sait jamais. Vous m’avez dit qu’il avait une femme beaucoup plus jeune que lui. Peut-être qu’elle le trompait et qu’il ne l’a pas supporté.

\textsc{Lui} : Elle ne l’a jamais trompé. Elle couchait avec d’autres hommes, c’est vrai, mais Rémi était parfaitement au courant et couchait lui aussi avec d’autres femmes, en toute transparence.

\textsc{Moi} : La vôtre, par exemple.

\textsc{Lui} : Par exemple. Mais j’ai souvent couché avec la sienne. J’ai toujours considéré comme une chose parfaitement normale que mes amis couchent avec ma femme, et m’autorisent à faire de même avec la leur. Je ne sais pas si vous êtes marié, lieutenant…

\textsc{Moi} : Commandant.

\textsc{Lui} : … commandant, mais si vous l’êtes vous devez savoir à quel point il est rébarbatif de coucher toujours avec la même femme, quel soit l’amour qu’on lui porte. L’amour et le sexe sont deux choses totalement différentes, qui ne devraient rien avoir à faire ensemble. Le désir est une chose, l’amour en est une autre, et la confusion qui règne entre les deux est la source de nombreux problèmes. Vous pouvez vivre avec quelqu’un toute votre vie, et continuer à l’aimer, mais certainement pas à le désirer comme au premier jour. Si certains y arrivent, tant mieux pour eux, mais moi ce n’est pas mon cas. Je ne vois pas au nom de quoi je devrais m’interdire d’être attiré par d’autres femmes et de coucher avec elles si le désir est réciproque. Quel spectacle pathétique de voir tous ces vieux types, mariés depuis des siècles à une femme qui ne ressemble physiquement plus à rien, baver comme des malades sur le cul des gamines qui passent à leur portée. Mais vous savez ce qui me fait le plus marrer ?

\textsc{Moi} : Non, dites-moi.

\textsc{Lui} : C’est de voir les jeunes mariés errer dans les rayons des supermarchés.

\textsc{Moi} : Mais encore ?

\textsc{Lui} : Madame marche en tête, et monsieur suit, l’air grognon et la mine déconfite, poussant un caddie rempli jusqu’aux ouïes de couches-culottes, lait en poudre et toute une ribambelle de produits ultra transformés bourrés d’huile de palme hydrogénée, amidon modifié, agents de texture, isolats de protéines et autres perturbateurs endocriniens diversement cancérigènes. Il a pourtant toutes les raisons d’être heureux, le jeune père de famille : il vient d’acheter un pavillon avec un petit lopin de terre pour faire pousser deux patates et trois petits pois, de changer de bagnole et d’accéder aux joies de la paternité. Seulement voilà, madame vient de prendre vingt kilos en neuf mois et il sait pertinemment qu’elle ne réussira jamais à s’en débarrasser. D’autant qu’elle ne fera pas le moindre effort pour ça, pour la bonne et simple raison qu’elle veut un autre enfant et part du principe que c’est pas la peine de suer sang et eau pour perdre vingt kilos si c’est pour en reprendre trente dans la foulée. Et elle se dit aussi que si son mari l’aime, il aura toujours autant envie d’elle même si elle ressemble à un éléphant de mer. Du coup, elle va garder ses vingt kilos et en reprendre une bonne vingtaine de plus pendant sa prochaine grossesse, perdant définitivement toute chance de retrouver un jour sa taille d’antan. Résultat des courses : ils vont commencer à s’engueuler, divorcer dans un an, deux ou trois si tout va bien, et madame va aller s’inscrire à la salle de sport du coin dans l’espoir de trouver un nouveau pigeon pour lui témoigner un peu d’affection. Vous ne trouvez pas que ça fait froid dans le dos ?

\textsc{Moi} : Vu comme ça, ce n’est effectivement pas très engageant.

\textsc{Lui} : Franchement répugnant, vous voulez dire !

\textsc{Moi} : J’aimerais qu’on en revienne à Rémi. Vous n’avez rien remarqué de bizarre pendant les jours ou les semaines qui ont précédé son décès ?

\textsc{Lui} : Non, rien du tout. Je vous le répète, Rémi allait parfaitement bien et n’avait aucune raison de mettre fin à ses jours. Il est évident que quelqu’un l’a tué en essayant de faire passer le crime pour un accident, et je ne doute pas que l’analyse médico-légale le confirmera.

\textsc{Moi} : Vous lui connaissez des ennemis ?

\textsc{Lui} : Les riches ont des tas d’ennemis.

\textsc{Moi} : Vous, peut-être ?

\textsc{Lui} : Moi ? Vous êtes fou !

\textsc{Moi} : Vous m’avez dit qu’il couchait avec votre femme. On a vu des gens en tuer d’autres pour moins que ça.

\textsc{Lui} : Je vous ai dit aussi que je m’en foutais, et que je couchais aussi avec la sienne. Il nous arrivait aussi de coucher tous ensemble, si vous voulez tout savoir.

\textsc{Moi} : Drôles de pratiques.

\textsc{Lui} : Je ne vous demande pas de vous joindre à nous.

\textsc{Moi} : Je peux vous poser une question ?

\textsc{Lui} : Si vous y tenez. Au fait, je ne sais pas si je vous l’ai dit, mais ce cigare est excellent. Il vient de Cuba, je suppose.

\textsc{Moi} : Non, de République dominicaine. Vous avez entendu parler d’Arturo Fuente ?

\textsc{Lui} : Non.

\textsc{Moi} : C’est un de ces émigrés espagnols qui ont fui Cuba pendant la guerre hispano-américaine. Il a atterri à Tampa, en Floride, et s’est lancé dans la fabrication de cigares avec des feuilles en provenance de Cuba. Une production d’abord confidentielle, limitée à quelques milliers de cigares par an roulés dans le salon et la cuisine par les membres de la famille. Ensuite, quand ça commençait à plutôt bien marcher, il y a eu le Che, Castro et la révolution cubaine, avec pour conséquence la rupture des relations diplomatiques entre Cuba et les USA. Du coup, la manne cubaine s’est tarie. Fuente a donc commencé à se fournir au Mexique et à Porto Rico, avec un succès mitigé, avant de changer son fusil d’épaule et aller s’installer au Nicaragua, nouvel Eldorado du cigare et dictature bananière sous contrôle américain. Mais à la fin des années 70, la révolution sandiniste a éclaté, les Somoza ont été foutus à la porte, et la fabrique Fuente, emblème d’une époque révolue, a été entièrement détruite par les flammes. Nouvel exil à Santiago, en République dominicaine, avec pour tout bagage un solide savoir-faire et une volonté farouche de tout casser. Aujourd’hui, associée à la famille Newman, Fuente produit des dizaines de millions de cigares par an, dont quelques uns parmi les plus réputés et onéreux de la planète. Celui que vous êtes en train de fumer, par exemple, le Short Story de la série Hemingway, est un vibrant hommage à l’écrivain qui a passé une bonne partie de sa vie à Cuba et appréciait tout particulièrement les Cohiba, une des plus prestigieuses marques de cigares.

\textsc{Lui} : Les meilleurs, à ce qu’il paraît.

\textsc{Moi} : Les plus chers, en tout cas. Oui, c’est ce que disent les snobs qui fument pour se donner un genre et n’y connaissent rien. C’était sans doute vrai avant l’embargo, et jusqu’à la fin des années 70 ou 80, mais depuis le cigare a fait son chemin un peu partout dans le monde et l’hégémonie cubaine n’est plus d’actualité, notamment en ce qui concerne le rapport qualité-prix. En cause le développement des marchés internationaux comme l’Inde, la Chine et le Moyen-Orient. Le pays n’arrive plus à suivre, avec pour conséquences une tendance à la surproduction, une baisse notoire de la qualité de fabrication et une hausse constante des prix. Les cigares autrefois abordables ont pulvérisé toutes les limites de la décence tarifaire. Le SIGLO VI de Cohiba, par exemple, grand classique s’il en est, se négocie aujourd’hui aux alentours de cent-dix euros pièce, et certaines séries spéciales montent à trois ou quatre cent. D’autre part, même s’il représente toujours dans l’imaginaire collectif la référence absolue en matière de cigare, force est de constater que le havane peine à se renouveler, innover tant sur la plan de la forme que du fond, tandis que les autres rivalisent de créativité pour exciter les papilles du consommateur.

\textsc{Lui} : Si vous le dites.

\textsc{Moi} : Je l’affirme haut et fort et ne cesserai de le clamer jusqu’à mon dernier souffle, n’en déplaise aux crétins prétentieux qui ne jurent que par le havane !

\textsc{Lui} : Grand bien vous fasse.

\textsc{Moi} : Maintenant, si vous le voulez bien, j’aimerais vous poser une petite question. Rien de personnel, rassurez-vous.

\textsc{Lui} : De quoi s’agit-il ?

\textsc{Moi} : De l’individu assis au volant de cette voiture.

Je parlais de Noé Desmarais, qui n’avait pas bougé un cil depuis le début de la conversation.

\textsc{Lui} : Vous voulez savoir s’il est mort, c’est ça ?

\textsc{Moi} : J’aimerais bien, oui.

\textsc{Lui} : C’est un ami à vous ?

\textsc{Moi} : Pas exactement, mais c’est une histoire un peu longue à raconter. Tout ce que je peux vous dire, c’est qu’il arrivait en face quand la Mini a essayé de doubler la Fiesta. Et ça a fait un grand BOUM !

Lui, agitant le truc qui pendait à son cou : Vous savez ce que c’est ?

\textsc{Moi} : Oui, un stéthoscope.

\textsc{Lui} : Mais pas n’importe lequel. C’est un Redmann, la Rolls du stéthoscope. Avec ça, vous pouvez entendre respirer un moucheron et battre le cœur d’un ver de terre, qui en possède cinq soit dit en passant.

\textsc{Moi} : Et alors ?

\textsc{Lui} : Alors cet homme est mort, il n’y a aucun doute là-dessus.

\textsc{Moi} : Vous en êtes sûr ?

\textsc{Lui} : Sûr et certain. Vous n’oseriez tout de même pas mettre en doute mes compétences ?

\textsc{Moi} : Loin de moi cette idée absurde.

\textsc{Lui} : Dans ce cas, vous ne m’en voudrez pas de prendre congé. J’ai encore des tas de vie à sauver qui m’attendent.

FIN

Provisoire, bien entendu.

Il est toujours extrêmement douloureux de mettre un point final à un récit qui a occupé de longs mois de votre existence, pompé une bonne partie de votre énergie et mobilisé toutes vos facultés créatrices. C’est comme dire au revoir à un vieil ami, lui serrer une dernière fois la main sans savoir si on le reverra un jour. On la garde longtemps dans le creux de la sienne, comme un petit animal blessé, on se refuse obstinément à la lâcher. Et puis on rentre chez soi, triste, et on avale une belle assiette de rognons de veau à la crème pour se donner une contenance, tenter d’oublier que toutes les choses ont une fin, les meilleures comme les pires, ce qui est une bonne chose pour les pires mais moins bonne pour les bonnes.

Sous la pression de mes fans, qui commencent à trouver le temps long (vous m’avez manqué, vous aussi), et surtout de mon éditeur qui a grand besoin de renflouer les caisses de sa modeste entreprise (les temps sont durs pour tout le monde, et ce serait pour lui un crève-cœur de devoir vendre son Ferretti Custom Line 97 ou hypothéquer sa villa de Saint-Raphaël pour sauver les meubles), je me vois dans l’obligation de remettre à plus tard un certain nombre des affaires en cours.

Je pense notamment à Jaya, la fille adorée de mon ami Zaahid Shirani, tombée entre les griffes d’un certain Simon Keskula, guide spirituel et maître incontesté d’un secte post-apocalyptique connue sous le nom d’Alliance de la Révélation. J’ai promis à Zaahid de tout mettre en œuvre pour que Jaya rentre au bercail et que le monstre qui la tenait sous sa dépendance soit définitivement mis hors d’état de nuire. Et comme je suis un homme de parole, je vais faire ce que j’ai dit. Et surtout, je ne manquerai pas de vous narrer par le détail comment, par quel stratagème machiavélique, ruse subtile et technique d’infiltration digne des meilleurs services de renseignement, et accessoirement usage immodéré de la force, sinon la violence la plus aveugle et éthiquement condamnable, je serai parvenu à mes fins.

En attendant, je sais qu’une double question vous ronge la cervelle aussi sûrement qu’un rat affamé s’attaque à un morceau de gruyère ou un vieux quignon de pain : Repentance Whittingham, alias la Gardienne de la Nuit ou la femme de ménage la plus rapide du monde, est-elle sortie du coma, et quid de Titus Beaugendre, porté disparu après une rencontre tant fortuite que suspecte avec la demoiselle en question ?

Eh bien… on n’en sait trop rien, à vrai dire, mais je ne manquerai pas de vous le faire savoir si j’apprends quelque chose à ce sujet. Tout ce que je sais pour l’instant, et c’est assez maigre je vous le concède, c’est qu’elle a disparu du jour au lendemain de sa chambre d’hôpital. Et comme je doute fort qu’elle ait été capable de le faire par ses propres moyens, l’action d’un tiers n’est pas à exclure.

Pour ce qui est de Titus, je vous propose un petit flashback juste avant le mot FIN, au moment où le docteur Charrier nous a annoncé que lui et son Redmann, la Rolls du stéthoscope, étaient formels sur le fait que Noé Desmarais ne ferait plus jamais joujou avec les allumettes, ni d’ailleurs avec quoi que ce soit d’autre, l’envie de faire joujou avec quelque objet ou organe que ce soit lui étant définitement passée. Desmarais refroidi, la joyeuse petite bande de néonazis des Disciples de la Colère était en partie démantelée. Ne restait plus qu’à exterminer les sieurs Monteil et Jégou, individus peu recommandables auxquels j’entendais bien réserver un traitement à la hauteur de leurs exploits. Pour ce faire, je m’étais dit que ça pourrait être sympa de transformer ma salle de bain en chambre à gaz. On enlevait Monteil et Jégou, on les obligeait à revêtir un pyjama rayé, on les affamait pendant quelques semaines, puis, quand ils commençaient à sentir si mauvais que même les mouches à merde s’enfuyaient à tire-d’aile à leur approche, on leur offrait une petite douche gratuite au Zyklon B, le célèbre insecticide à base de cyanure de la Deutsche Gesellschaft fur Schadlingsbekampfung (il faut reconnaître que les Allemands ont un certain talent pour créer des mots de quinze kilomètres de long totalement imprononçables pour toute personne non germanique, à tel point que je me demande si ce n’est pas une des raisons principales de leur manque de popularité et relatif isolement sur la scène internationale, outre le fait qu’ils construisent des voitures rapides très appréciées des trafiquants de drogue, boivent beaucoup de bière et sont nuls en cuisine). Après quoi on en faisait des steaks hachés, merguez, saucisses et andouillettes, et on organisait une grande fiesta dans le quartier avec barbecue à gogo jusqu’à épuisement des stocks. On joignait l’utile à l’agréable, et je doute fort que la police s’amuserait à aller fourrer son nez dans les cuvettes de chiottes du voisinage. Je ne connais pas les effets du piment et des épices sur les composés organiques, mais je suppose qu’il est tout à fait possible de rechercher des traces d’ADN dans une merguez ou une chipolata. Si on le faisait plus souvent, quelque chose me dit qu’on pourrait avoir des surprises de taille.


\textsc{Greg} : Je vais appeler Sally Robinson. Elle sera sûrement ravie d’apprendre que Desmarais n’est plus de ce monde.

\textsc{Moi} : Et moi je vais appeler Bérénice pour lui dire qu’on est toujours sans nouvelles de Titus.

\textsc{Greg} : C’est quand même bizarre, cette histoire.

\textsc{Moi} : On nage en pleine absurdité.

\textsc{Greg} : Je me pose des questions.

\textsc{Moi} : À propos de quoi ?

\textsc{Lui} : Titus. Il aime sa femme, ses gosses, je ne comprends pas pourquoi il a suivi cette fille sans discuter.

\textsc{Moi} : C’est toujours pareil : on pense connaître les gens, et on découvre des zones d’ombre qu’on ne soupçonnait pas.

\textsc{Greg} : Tu penses qu’il est encore en vie ?

\textsc{Moi} : Je n’en ai aucune idée. Peut-être qu’il est enchaîné quelque part, en train de se faire violer et torturer par une bande de suprémacistes blancs homosexuels.

\textsc{Greg} : Ou alors il a décidé que sa vie ne correspondait plus à ses rêves et il a foutu le camp sans laisser d’adresse.

\textsc{Moi} : Il m’a souvent parlé de son envie de retourner en Sierra Leone pour renouer avec ses origines et retrouver la trace de ses ancêtres. Je crois savoir que l’idée ne plaisait pas plus que ça à Bérénice.

\textsc{Greg} : Il t’avait semblé bizarre, ces derniers temps ?

\textsc{Moi} : Pas plus que d’habitude. Il a toujours été un peu mystique. Il y a quelque temps, il s’est découvert une passion pour Edward Tylor\nf{Edward Burnett Tylor (1832--1917), anthropologue britannique considéré comme l’un des fondateurs de l’anthropologie culturelle. Auteur de \textit{La Civilisation primitive} (1871), il fut le premier à définir scientifiquement le concept de culture et développa une théorie de l’animisme comme forme originelle de la religion. Il fut nommé premier professeur d’anthropologie à Oxford en 1896. \source{fr.wikipedia.org/wiki/Edward\_Tylor}}, un anthropologue british du XIX\textsuperscript{e} qui s’intéressait à l’animisme, au totémisme et à la cosmogonie. Titus potassait aussi des trucs sur la culture yoruba et les babalaos.

\textsc{Greg} : Les quoi ?

\textsc{Moi} : Les babalaos. Ce sont des espèces de sorciers qui pratiquent la divination avec des noix de palmier.

Pendant ce temps, les pompiers avaient découpé la Mini pour extraire Repentance Whittingham de son cercueil de tôle.

La femme de ménage la plus rapide du monde avait ensuite été transportée toutes sirènes hurlantes vers l’hôpital le plus proche.

\textsc{Greg} : Et Bérénice ?

\textsc{Moi} : Quoi, Bérénice ?

\textsc{Greg} : Tu lui as parlé, récemment ?

\textsc{Moi} : Bien sûr, que je lui ai parlé récemment ! Elle m’appelle toutes les dix secondes pour savoir ce qui est arrivé à son cher et tendre. D’ailleurs c’est bizarre, ça fait un moment qu’elle ne m’a pas rappelé.

\textsc{Greg} : Elle n’a rien remarqué de particulier, elle non plus ?

\textsc{Moi} : Elle, non. Mais moi, oui.

\textsc{Greg} : Comment ça ?

\textsc{Moi} : J’ai remarqué quelque chose de particulier. Pas à propos de Titus, mais de Bérénice, justement.

\textsc{Greg} : Ah bon ?

\textsc{Moi} : Oui. Depuis quelque temps, Bérénice a un comportement inhabituel, que je ne lui connaissais pas auparavant.

\textsc{Greg} : Tu peux être plus précis.

\textsc{Moi} : C’est assez indéfinissable, en fait. C’est quelque chose dans son attitude générale, sa façon de s’habiller, de parler, et même de se déplacer.

\textsc{Lui} : Elle ne se déplace plus comme avant ?

\textsc{Moi} : Elle se déplace bizarrement, avec une démarche plus féline que d’habitude.

\textsc{Greg} : Plus féline ?

\textsc{Moi} : Oui, plus féline. Je te l’ai dit, c’est assez indéfinissable. Cela dit, en la voyant, je me suis dit que quelque avait changé en elle, comme si elle n’était plus tout à fait la même.

\textsc{Greg} : Des soucis personnels, peut-être.

\textsc{Moi} : Même son odeur n’est plus la même.

\textsc{Greg} : Elle a peut-être tout simplement changé de parfum.

\textsc{Moi} : Ne sois pas idiot. Si tu veux vraiment que je te dise ce que je pense…

\textsc{Greg} : Oui, j’aimerais bien.

\textsc{Moi} : … je vais te le dire, mais arrête de me couper sans arrêt ! Si tu veux vraiment que je te dise ce que je pense, je ne serais pas étonné qu’elle voie quelqu’un d’autre.

\textsc{Greg} : Non ???!!!

\textsc{Moi} : Si.

\textsc{Greg} : Mais alors, ça change tout !

\textsc{Moi} : Comment ça ?

\textsc{Greg} : Si elle voit quelqu’un d’autre, il se peut que Titus soit au courant, ce qui expliquerait son comportement bizarre de ces derniers temps.

\textsc{Moi} : Titus n’était pas bizarre ces derniers temps. Il n’est devenu bizarre qu’hier soir, quand cette fille a débarqué de nulle part. Dès qu’elle est entrée dans son champ de vision, c’était comme si son esprit avait quitté son corps pour aller gambader tel un guépard dans les vastes plaines du Kalahari, royaume de ceux qui suivent l’éclair et ramassent par terre. Tu te rappelles quand elle lui a dit «~toi venir avec moi dans chambre~», l’étrange lueur qu’il y avait dans ses yeux quand elle prononcé ces mots ?

\textsc{Greg} : Les yeux de qui ?

\textsc{Moi} : Ceux de Titus, bien sûr ! Tu le fais exprès, ou quoi ?

\textsc{Lui} : Excuse-moi, mais j’ai un peu de mal à te suivre dans tes divagations. Et non, je n’ai pas vraiment fait attention à la lueur qu’il y avait dans les yeux de Titus quand elle a prononcé ces mots. Je devais être occupé ailleurs. Cela dit, si Bérénice a effectivement un amant comme tu le prétends, ça pourrait expliquer bien des choses.

\textsc{Moi} : Je ne prétends rien du tout. Je dis juste qu’il n’est pas totalement impossible, compte tenu des éléments dont je dispose, que Bérénice ait un amant.

\textsc{Greg} : Quels éléments ?

\textsc{Moi} : Des éléments…

\textsc{Greg} : Tu veux dire des éléments vraiment… vraiment…

\textsc{Moi} : Compromettants, oui. Mais ce serait peut-être à toi de m’en dire un peu plus sur le sujet…

\textsc{Greg}, feignant la surprise : À moi ?

\textsc{Moi} : Oui, à toi. Je crois savoir que Bérénice est une excellente danseuse de tango…

\textsc{Greg} : C’est possible, en effet.

\textsc{Moi} : Ne fais pas comme si tu ne le savais pas.

\textsc{Greg} : J’en ai vaguement entendu parler.

\textsc{Moi} : Un peu plus que vaguement, je pense. Je crois savoir aussi que tu t’es toi-même découvert une passion pour Carlos Gardel\nf{Carlos Gardel (1890--1935), chanteur et acteur franco-argentin, figure légendaire du tango. Surnommé «~El Zorzal Criollo~» (le merle créole), il enregistra plus de 900 chansons et contribua à diffuser le tango dans le monde entier. Il mourut dans un accident d’avion à Medellín à l’apogée de sa gloire. \source{fr.wikipedia.org/wiki/Carlos\_Gardel}}, Rodolfo Biagi\nf{Rodolfo Biagi (1906--1969), pianiste et chef d’orchestre argentin de tango, surnommé «~Manos Brujas~» (les mains ensorcelées). Son style, caractérisé par un rythme syncopé et énergique, fit de son orchestre l’un des plus populaires de l’âge d’or du tango dans les années 1930--40. \source{fr.wikipedia.org/wiki/Rodolfo\_Biagi}}, Francisco Canaro\nf{Francisco Canaro (1888--1964), violoniste, compositeur et chef d’orchestre argentin d’origine uruguayenne, l’un des maîtres du tango. Il fonda son premier orchestre en 1916 et composa plus de 5 000 œuvres, dont de nombreux classiques du genre. Il fut également promoteur du tango en Europe, notamment à Paris dans les années 1920. \source{fr.wikipedia.org/wiki/Francisco\_Canaro}}, El Pibe\nf{Juan Carlos Cobián (1896--1953), dit «~El Pibe~» (le gamin), pianiste et compositeur argentin de tango. Auteur de classiques tels que \textit{Los Mareados} et \textit{Shusheta}, il fut l’un des novateurs du genre et influença de nombreux musiciens de sa génération. \source{fr.wikipedia.org/wiki/Juan\_Carlos\_Cobi\%C3\%A1n}} et Anibal «~El Gordo~» Troilo\nf{Aníbal Troilo (1914--1975), surnommé «~El Gordo~» ou «~Pichuco~», bandonéoniste et chef d’orchestre argentin, l’une des figures les plus aimées du tango. Son orchestre fut au cœur de l’âge d’or du genre dans les années 1940. Compositeur de classiques comme \textit{Sur} et \textit{La última curda}, il est considéré comme le maître incontesté du bandoneón. \source{fr.wikipedia.org/wiki/An\%C3\%ADbal\_Troilo}}, pour n’en citer que quelques-uns.

\textsc{Greg}, de plus en plus mal à l’aise : One day I was at home, et non pas reincarnated as the 7th Prince, en train de bouffer des raviolis en boîte, quand j’ai entendu une chanson de La Tana, Susana Rinaldi\nf{Susana Rinaldi (née en 1935), chanteuse et actrice argentine surnommée «~La Tana~». Considérée comme l’une des plus grandes interprètes de tango, elle se distingue par sa voix grave et son phrasé dramatique. Elle a connu une carrière internationale, notamment en Europe, et s’engage également en politique en faveur des droits de l’homme. \source{fr.wikipedia.org/wiki/Susana\_Rinaldi}} de son vrai nom, à la radio. Stupeur et tremblements, un déclic s’est produit dans ma tête. J’ai aussitôt balancé mes raviolis à la poubelle et décidé de changer de vie. Ma passion pour le tango était née.

\textsc{Moi} : Et tu t’es acheté une paire de Danilo Pinto à mille cinq cent balles !

\textsc{Greg} : Quand on aime on ne compte pas.

\textsc{Moi} : Des Danilo Pinto dont tu ne te sépares quasiment plus. La preuve, tu les as aux pieds ce matin, alors qu’on n’était pas franchement partis pour danser le tango.

\textsc{Greg} : Elles sont très confortables.

\textsc{Moi} : Je n’en doute pas. Mais ça n’explique pas pourquoi tu t’es inscrit dans la même école de danse que Bérénice…

\textsc{Greg} : Je me trompe ou tu es en train d’insinuer des choses pas très catholiques à mon sujet ?

\textsc{Moi} : Je n’insinue pas, j’affirme. J’affirme, mon cher Grégoire, que Bérénice et toi dans comme des Argentins jusque tard dans la nuit au Palazzo Cristal Club de la rue Féret. Des Argentins désargentés, peut-être, mais des Argentins tout de même.

\textsc{Greg} : C’est faux !

\textsc{Moi}, d’un calme olympien : Non, c’est authentiquement vrai.

\textsc{Lui} : Comment tu peux savoir ça ?

\textsc{Moi} : Je le sais, c’est tout.

\textsc{Lui} : C’est Bérénice qui te l’a dit ?

\textsc{Moi} : Ah ah ! Donc tu admets que Bérénice et toi vous trémoussez comme des dingues au Palazzo ?

\textsc{Lui} : On danse, c’est tout.

\textsc{Moi} : Oui, avec la femme de Titus. Titus à qui tu t’es bien gardé de dire que tu dansais le tango avec sa femme. Reconnais que tes méthodes sont particulièrement sournoises.

\textsc{Greg} : Je suppose qu’il est au courant.

\textsc{Moi} : Non, il ne l’est pas. Enfin, pas officiellement. Et ce n’est pas tout.

Greg a ouvert la bouche. J’ai senti qu’il voulait dire quelque chose, plaider sa cause, mais il s’est aussitôt ravisé, comprenant qu’il avait le droit de garder le silence et que tout ce qu’il dirait pourrait être retenu contre lui.

J’ai donc repris le fil de mon réquisitoire : On vous a vu sortir main dans la main et vous roulez des pelles sur le trottoir !

Greg a bruyamment ravalé la salive qui affluait dans le fond de sa gorge : Pardon ?

L’accusation était ignoble, je l’admets, mais hélas non dépourvue de fondement.

\textsc{Moi} : Pelles, galoches, palots, patins, appelle comme tu veux, toujours est-il qu’on vous vu vous boulotter les amygdales !

\textsc{Lui}, feignant assez maladroitement la stupéfaction (même un acteur de sitcom ou téléfilm français aurait largement fait mieux) : Hein ? Quoi ? Comment ? On ? Qui ça, on ?

Comme un bon journaliste ne cite jamais ses sources, un bon flic ne révèle jamais le nom de ses tontons.

Mais à vous, à toi, lecteur (je pense qu’on en est arrivé à un stade de notre relation où on peut raisonnablement envisager de se tutoyer), je peux bien le dire.

Je te connais suffisamment pour savoir, si je ne le fais pas, que tu vas psychoter, te creuser les méninges et douter de tout, y compris de Dieu, l’abbé Pierre, Donald Trump et Vladimir Poutine, t’interroger sur le sens de la vie, les dangers de la 5G, la réalité augmentée, du deep learning et de l’IA générative, jusqu’à en perdre le boire et le manger, le sommeil aussi, passer tes nuits les yeux grand ouverts à fixer le plafond, le cœur battant, à suer sang et eau dans le fond du lit conjugal de moins en moins conjugué, insensible aux appels désespérés de ta femme pour que tu t’intéresses enfin un peu à elle, ses désirs, ses rêves, son corps à l’abandon. Je sais que tu vas errer sans but dans les rues de la ville, telle une bête malade, chercher refuge dans des lieux de perdition, boire plus que de raison, et que tes amis, lassés de tes absences et ton humeur chagrine, vont te quitter un par un. Puis ce sera au tour de ta moitié, devenue au fil du temps ton tiers, ton quart, ton huitième, une fraction de plus en plus insignifiante de toi-même, de faire ses valises. Ayant épuisé les trésors de patience dont elle disposait, elle retournera chez sa mère, laquelle en sera enchantée pour au moins trois raisons : 1. elle vit seule avec sa chatte et se fait chier comme un rat mort ; 2. elle a toujours été extrêmement possessive et aurait voulu garder sa fille pour elle toute seule ; 3. elle t’a toujours considéré comme une erreur de la nature et un bon à rien. Mais ta femme ne partira pas seule : elle emportera avec elle la seule chose qui ne retenait encore accroché du bout des doigts à l’existence, je veux bien sûr parler de tes trois enfants. Quand tu rentrais à la maison, après ce qui aurait dû être une dure journée de labeur, ils te donnaient du «~monsieur~», non parce que vous êtes de la haute et avez conservé les usages de la vieille noblesse française (vous êtes des roturiers de la pire espèce, sans une once de sang royal dans les veines), mais parce qu’ils ne te reconnaissaient tout simplement plus. Et quand je dis «~aurait dû être~» une dure journée de labeur, c’est parce que tu as perdu ton boulot, bien sûr, et passes tes journées assis sur un banc à ruminer tes vilaines et vaines pensées, et envisager le moyen le plus expéditif de mettre un terme à tes souffrances. Tel Jean-Claude Romand\nf{Jean-Claude Romand (né en 1954), imposteur et assassin français. Pendant dix-huit ans, il fit croire à son entourage qu'il était médecin chercheur à l'OMS à Genève, tout en ne travaillant pas et en dilapidant l'épargne de sa famille. Découvert sur le point d'être démasqué, il assassina sa femme, ses deux enfants et ses parents en janvier 1993, avant de mettre le feu à sa maison. Condamné à la réclusion criminelle à perpétuité, son histoire inspira notamment le roman de Emmanuel Carrère \textit{L'Adversaire} (2000). \source{fr.wikipedia.org/wiki/Jean-Claude\_Romand}}, tu iras toquer à la porte des moines de la congrégation de Solesmes, à Notre-Dame de Fontgombault, mais ceux-ci, persuadés d’avoir affaire à un suppositoire de Satan, refuseront de t’accueillir. Par charité chrétienne, sur l’insistance du père-abbé qui est un homme profondément bon et ouvert d’esprit, cette bande de clowns tonsurés te proposera toutefois une place de jardinier au monastère Saint-Joseph de Séguéya, en Guinée-Conakry, poste que tu auras la fermeté de refuser d’une part parce que tu n’as pas la main verte, d’autre part parce que tu as toujours eu une peur bleue des Noirs, lesquels restent pour toi des créatures étranges aux yeux globuleux et à la bouche remplie de dents. C’est ainsi, seul sur le canapé du salon, la libido en berne devant les chaînes d’info en continu qui achèveront de te liquéfier le cerveau, un vieil exemplaire du \textit{Désespéré} de Léon Bloy\nf{Léon Bloy (1846--1917), écrivain et pamphlétaire français catholique, connu pour son style virulent et mystique. \textit{Le Désespéré} (1886), son premier roman fortement autobiographique, dépeint la lutte d'un homme profondément religieux contre la médiocrité bourgeoise et la lâcheté spirituelle. Il exerça une influence notable sur des écrivains comme Georges Bernanos et Jacques Maritain. \source{fr.wikipedia.org/wiki/L\%C3\%A9on\_Bloy}} à portée de main, entouré de bouteilles vides, de toiles d’araignées et de déchets alimentaires en voie de putréfaction, tu dépériras lentement dans l’indifférence générale, après que ton père t’aura déshérité et ta mère se sera désagrégée de chagrin comme une vieille chaussette pourrie. Jusqu’au jour où tes voisins (avec lesquels tu n’as jamais entretenu la moindre relation), incommodés par l’odeur, appelleront la police qui se fera une joie de venir défoncer ta porte et découvrir avec horreur ton cadavre grouillant d’asticots.

Aussi, parce que je te suis reconnaissant des efforts que tu as consentis pour arriver jusqu’ici, je vais te le dire.

En fait, aussi étrange et stupéfiant que cela puisse paraître, je tenais cette information capitale de la bouche même de Sam Girard, cuisinier hors pair et ancien des forces spéciales de l’armée de terre, qui se trouvait résider non loin du Palazzo Cristal Club. C’est là, depuis le bar où il avait coutume de se désaltérer et refaire le monde avec d’autres habitués, assez rugueux pour la plupart, qu’il les avait distinctement aperçus en train de se livrer à des activités qui avait plus à voir avec les bordels de Buenos Aires qu’avec le Rio de la Plata et ses danses locales. Prudent, et habitué au secret comme la plupart des militaires, il ne s’en était ouvert qu’à moi. Pourquoi moi ? Eh bien mais tout simplement parce que je suis une personne de confiance, comme vous n’aurez sans doute pas manqué de le remarquer. Si vous êtes dans la détresse, affective ou autre, je saurai trouver les mots justes pour remettre sur les rails la locomotive cabossée de votre existence. Je dis ça, mais je n’ignore pas que c’est dans la nature des gens de raconter leur vie au premier venu. Ils ne vous voient pas comme un être humain, mais comme une benne à ordures dans laquelle déverser leurs immondices. Ce sont les mêmes qui abandonnent leurs papiers gras sur la chaussée et leurs mégots de clopes dans la forêt, au risque de foutre le feu. Peu importe l’environnement, l’important est de se débarrasser à bon compte de ses déchets. Et si vous avez la faiblesse de prêter ne serait-ce qu’une vague oreille aux divagations de l’un d’entre eux, vous pouvez être certain qu’ils ne seront pas long à faire la queue devant chez vous pour faire leurs besoins sur votre paillasson. Pour ce qui est des soi-disant secrets qu’ils sont censés garder, dites-vous bien que c’est un fardeau dont ils se débarrasseront à la première occasion. Ils vous refilent la patate chaude et reprennent leur petite vie tranquille comme si de rien n’était, la conscience aussi légère et vierge qu’au jour de leurs premiers vagissements. Libre à vous de la refiler à quelqu’un d’autre, qui à son tour la refilera à quelqu’un d’autre, et ainsi de suite jusqu’à ce que tout le monde soit au courant. Au courant de quoi, c’est une autre question. Car il va de soi, au cours du périple mouvementé qui a été le sien, que le récit n’a plus grand-chose à voir avec ce qu’il était à l’origine.

Dans le cas présent, la patate chaude en question était une première main, passée directement du producteur au consommateur, de la terre à l’assiette. Rien à voir avec ces cochonneries bourrées d’additifs tous plus toxiques les uns que les autres. Sam l’avait récoltée à la source, et me l’avait livrée telle qu’elle, sans la nettoyer, lui faire subir le moindre traitement qui aurait pu altérer sa texture et sa saveur. Elle avait de la mâche, du goût, et le mieux, pour les préserver, était de la cuisiner le moins possible. Certains disent qu’on peut sublimer le produit, le rendre encore meilleur qu’il n’est en réalité, mais si une chose est parfaite en soi, ce que les mêmes ne cessent de répéter avec les yeux révulsés et des trémolos dans la voix, alors je ne vois pas l’intérêt d’en rajouter, sinon celui de s’en attribuer indûment les mérites grâce à un tour de passe-passe discutable. Par exemple, le type qui essaie de vous vendre un jambon prétendument incomparable, va d’abord vous beurrer la tartine d’une épaisse couche de superlatifs concernant son produit, sans la moindre pudeur. Il pourrait se contenter de vous en couper une tranche et attendre patiemment votre réaction. Mais non, il prépare le terrain, vous assure que vous allez participez à une expérience unique dont vous sortirez à jamais transformé, vivre un moment de grâce absolue, toucher du doigt les arcanes de la mystique plotinienne et approcher au plus près des plus grands mystères qui agitent l’humanité depuis la nuit des temps. Vous êtes l’élu, celui qui a été choisi pour assister à la Révélation. À tel point, même si vous le pensez profondément, que vous hésiterez à lui dire que son soi-disant merveilleux jambon de pays n’est rien d’autre qu’une merde infâme indigne du plus vil clébard. Pire encore, vous serez dans l’obligation d’en faire l’acquisition alors que vous n’en avez pas la moindre envie. Et comme vous aurez été bien reçu (c’est tout juste si on ne sera pas allé vous chercher au milieu de la rue pour vous traîner dans la boutique), qu’on vous aura déroulé le tapis rouge, traité comme une personnalité de premier plan, vous culpabiliserez d’autant plus de ne pas souscrire à l’offre proposée. Celle-ci, du reste, sera largement prohibitive, pour ne pas dire totalement déconnectée de la réalité, mais c’est un abus dont vous ne pourrez hélas faire mention sans passer aussitôt pour un effroyable radin, un pique-assiette qui profite de la générosité de ses hôtes pour s’empiffrer à bon compte. Ainsi, vous aurez bel et bien assisté à la Révélation de votre propre connerie, la facilité avec laquelle il est possible de vous la fourrer bien profond. Voilà pourquoi, lorsque vous êtes en affaires, et c’est un petit conseil que vous donne comme ça en passant, partez toujours du principe que votre partenaire est une ordure qui ne pense qu’à vous arnaquer, fera tout pour endormir votre vigilance et se barrer avec la caisse à la première occasion. Ne faites confiance à personne, à commencer vous-même, car vous êtes et serez toujours votre pire ennemi, d’autant plus dangereux qu’il est enclin à tout vous pardonner, y compris vos plus monumentales erreurs.

Donc, pour en revenir à notre affaire, je ne pouvais pas répondre à la question de Greg (Hein ? Quoi ? Comment ? On ? Qui ça, on ?) autre chose que : Désolé, mais je ne peux pas te le dire.

\textsc{Greg} : Je ne sais pas qui t’a raconté ça, mais je peux t’assurer qu’il n’y a rien entre Bérénice et moi. Crois-le ou non, mais c’est tout à fait par hasard qu’on s’est retrouvé dans ce club de danse.

\textsc{Moi} : Dans ce cas, je trouve étrange que tu ne m’en aies jamais rien dit.

\textsc{Lui} : L’occasion ne s’est pas encore présentée, voilà tout. C’est tout récent, je te le rappelle.

\textsc{Moi} : Récent ou pas, je pense que si tu avais rencontré Bérénice dans un club de danse, tu n’aurais pas manqué de m’en faire part. À moins, bien sûr, que tu aies une bonne raison de ne pas le faire.

\textsc{Greg}, aussi convaincant qu’un politicien qui affirme n’avoir jamais eu connaissance des agissements de Jeffrey Epstein\nf{Jeffrey Epstein (1953--2019), financier américain reconnu coupable de trafic sexuel de mineures. Il fut arrêté en 2019 et retrouvé mort dans sa cellule dans des circonstances controversées. Ses réseaux impliquant de nombreuses personnalités politiques et économiques mondiales alimentèrent de multiples théories et enquêtes judiciaires. \source{fr.wikipedia.org/wiki/Jeffrey\_Epstein}} alors que son nom revient plus de cinq mille fois dans les dossiers afférents, et n’être allé à Little Saint James\nf{Little Saint James est une île privée des Îles Vierges américaines, surnommée «~l’île de Pédophile~», que Jeffrey Epstein acquit en 1998. Elle servit de cadre à de nombreuses rencontres avec des personnalités de premier plan et à des actes de trafic sexuel. Après l’arrestation d’Epstein, l’île fut saisie par les autorités américaines. \source{fr.wikipedia.org/wiki/Little\_Saint\_James}} qu’une ou deux fois, et encore uniquement pour jouer au tennis, prendre des bains de mer et pêcher le marlin : Je suis vraiment très triste.

\textsc{Moi} : Ah bon ?

\textsc{Lui} : Ben oui, que tu puisses penser une chose pareille de moi.

\textsc{Moi} : Pourquoi tu ne dis pas tout simplement la vérité ?

\textsc{Lui} : Quelle vérité ?

\textsc{Moi} : Que tu te tapes la femme de Titus. En plus de Lou et peut-être une demi-douzaine d’autres, va savoir. À partir de là, on peut tout imaginer.

\textsc{Lui} : Comme quoi, par exemple ?

\textsc{Moi} : Eh bien, par exemple, que Bérénice et toi avez imaginé un plan machiavélique pour vous débarrasser du mari gênant. Vous louez les services de cette prétendue femme de ménage et Gardienne de mes couilles, qui est en réalité une redoutable tueuse à gage, et vous faites croire à tout le monde qu’elle et Titus sont partis couler des jours heureux quelque part sous les Tropiques.

\textsc{Greg}, avec pour la première depuis longtemps une vague touche de sincérité dans le regard : C’est du grand n’importe quoi !

Bon, je reconnais qu’il m’arrive parfois d’avoir des idées farfelues, sinon franchement débiles, et celle-ci figurait en bonne place dans le haut du panier. Si je n’avais pas cette profonde connaissance de moi-même, fruit d’une longue et attentive fréquentation de mon humble personne, j’en viendrais presque à douter de l’état de ma santé mentale. Mais s’il faut douter à tout prix, j’aime autant douter de celle des autres. Greg est un ami et le restera, même s’il couchait avec sa propre mère, ce qui ne risque pas d’arriver puisqu’elle est morte, paix à son âme. Mais peut-être l’ont-ils fait, se sont-ils vautrés tels des hippopotames en rut dans la boue de l’inceste, des sangsues boursouflées de désir dans la vase abjecte de la consanguinité. Si tel est le cas, je n’en ai jamais rien su et préfère n’en jamais rien savoir. Cela dit, Greg m’avait souvent donné cette bizarre impression de traverser l’existence sur la pointe des pieds, en catimini. On ne le sentait pas réellement investi dans les affaires du monde. Je n’irais pas jusqu’à dire qu’il se foutait royalement de tout, ce serait sans doute exagéré, mais il y avait tout de même chez lui une forme de dilettantisme qui flirtait parfois assez dangereusement avec l’absentéisme pur et simple. Je pense que ça ne lui aurait fait ni chaud ni froid si on lui avait annoncé qu’il allait mourir demain. Ses traits ne seraient pas décomposés, ni sa gorge nouée. Contrairement à ces condamnés à mort qui s’accrochent désespérément à la vie et pleurent toutes les larmes de leur corps au moment fatidique, abandonnant au pied du gibet les derniers oripeaux de leur dignité, c’est le cœur léger qu’il aurait offert son cou à la corde du bourreau. Un jour, on lui avait fait cadeau de la vie. C’était, comme à nous tous, son premier cadeau d’anniversaire. Sauf que pour lui, le cadeau en question ressemblait davantage au jackpot d’une loterie infernale dont il aurait été non pas le grand gagnant, l’heureux élu, mais le grand perdant, la principale victime. Quand certains passent le restant de leurs jours à les remercier pour leur avoir donné la vie, permis de gambader sur terre tels des faons émerveillés, d’autres vouent leurs géniteurs aux gémonies et n’aspirent qu’à leur faire payer le prix de cette trahison. Quand on aime vraiment quelqu’un, on ne lui donne jamais le jour. Certes, on n’aura jamais le plaisir de faire sa connaissance, contempler sur ses traits la marque de sa glorieuse lignée, mais au moins on n’aura pas son sang sur les mains. Tel un Marc-Antoine Jacquinot, avec lequel il partageait de nombreux traits de caractère, Greg n’avait jamais jugé utile de faire valoir ses droits à la paternité. Tous deux étaient, quoi qu’on en pense, en parfaite adéquation avec cette tendance de la jeunesse actuelle qui consiste à ne plus faire d’enfant, au point que d’aucuns commencent à s’en inquiéter sérieusement. C’est chronophage, ça coûte bonbon et ça ne rapporte le plus souvent que des emmerdements. D’autre part, avec les familles sans cesse recomposées, faites de bric et de broc, la notion de descendance perd de sa pertinence. D’autant qu’il ne sera bientôt plus nécessaire de payer de sa personne, la reproduction pouvant être assurée in vitro à partir d’une sélection aléatoire d’échantillons prélevés sur l’ensemble de la population, avec tous les avantages d’un brassage ethnique intempestif. Et sans doute même qu’un jour, ovule et spermatozoïde pourront être fabriqués de toutes pièces en laboratoire, ce qui permettra non seulement de s’affranchir du concept de race, problématique s’il en est, mais aussi d’exercer un contrôle total sur la démographie, tant sur la plan de la quantité que la qualité. Il est loin le temps où on accouchait dans la douleur, et où la question se posait souvent de savoir si on devait sauver la vie de la mère ou celle de l’enfant. Ce choix, il faut bien le reconnaître, était rarement favorable à la parturiente, surtout si l’enfant à naître était un mâle (pour un bien, en quelque sorte). Enfin, le monde fait aujourd’hui figure d’un tel bourbier que beaucoup rechignent à y plonger d’innocentes créatures qui n’ont rien demandé à personne. À quoi bon les envoyer au casse-pipe, les condamner à livrer une guerre perdue d’avance. Et surtout, à quoi bon rajouter du malheur au malheur quand il y a déjà de par le monde des tas d’orphelins qui crèvent la gueule ouverte dans le caniveau, sont livrés pieds et poings liés à une horde de prédateurs qui abusent d’eux en toute impunité. Bien sûr, la plupart d’entre eux sont des boules de haine et de ressentiment difficiles à gérer, mais l’heure est venue de se détacher de ces vieux principes éculés de filiation et de maternité. La reproduction doit être assurée de façon mécanique et désintéressée. Après tout, faire des enfants n’est pas un investissement à long terme, une assurance-vie pour de futurs vieux qui n’ont pas envie de crever seuls comme des chiens dans un pavillon de banlieue décrépi. C’est une nécessité biologique, si on veut que l’espèce perdure, qui doit être appréhendée de façon scientifique et non plus sentimentale et affective, avec la niaiserie résignée de ces jeunes parents qui endurent H24 les réflexions idiotes de leur entourage radotant. Notre existence n’est que le fruit du hasard, et nos enfants ne nous appartiennent pas davantage que nous ne nous appartenons à nous-mêmes. Ton corps t’appartient… Mon cul, oui ! C’est pas comme si tu étais allé l’acheter au supermarché du coin, et tu auras beau l’entretenir aussi bien sinon mieux que ta baraque ou ta bagnole, ça ne changera rien à l’affaire. Si tu peux toujours revendre ou refiler ta baraque et ta bagnole à tes enfants, personne ne voudra de ton corps. Ta carcasse sera dévolue aux vautours et tu te feras ratisser jusqu’à l’os, à l’exception bien sûr de tes dents en or, prothèses de hanche et autres babioles imputrescibles qui seront récupérées par la communauté. Car n’oublie jamais ceci, ma frère, mon sœur ou qui que tu sois : ton corps est la propriété de Dame Nature qui te l’a généreusement (plus ou moins, certains auraient des raisons de se plaindre) prêté pour te servir d’enveloppe physique pendant ton séjour sur terre. Il faut le restituer le jour de ta mort, sensiblement dans l’état où tu l’as trouvé le jour de ta naissance, c'est-à-dire peu fonctionnel et totalement dépendant des autres pour assurer sa survie. Sauf que cette fois, les autres n’ont pas la moindre envie de te donner la becquée, te porter sur leurs épaules et te torcher le cul. Ils n’ont plus envie de te courir après quand tu t’enfuis à quatre pattes à travers la maison en ricanant comme un dingue et lâchant des caisses à tout bout de champ. Ils sont fatigués de faire tes courses, hurler parce que ton sonotone déconne, te ramasser parce que tu n’arrêtes pas de tomber (ça te rapproche de la tombe), aller te récupérer au commissariat parce qu’on t’a encore chopé en train de te balader à poil dans la rue, taguer des obscénités sur les murs et pisser sur les devantures de magasins. Ils n’ont qu’une hâte, c’est que tu casses ta pipe et arrêtes de faire chier le monde avec tes jérémiades. Même tes gosses, que tu appelles madame ou monsieur parce que tu ne les reconnais plus, commencent à trouver le temps long. Eux et leurs propres enfants auraient mieux à faire que de passer leurs dimanches à l’EHPAD, mais comme ils savent que tu es capable de les déshériter sur un simple coup de tête, ils continuent à venir pour ne pas risquer de passer à côté du pactole. Mais toi tu t’en fous, parce que tu as cinq ans dans ta tête, et que tu sais que ton grand âge te permet de faire à peu près tout et n’importe quoi sans que personne n’ose lever le petit doigt. Alors tu vas voir le caïd du coin, met de la schnouff dans ta chicorée et t’achètes un flingue pour braquer une banque. Quand les flics arrivent, tu défourailles à tout-va, mais personne ne riposte parce que ça ferait désordre de buter un vieux qui n’a plus toute sa tête. On attend que le chargeur soit vide, puis on vient te chercher pour t’escorter gentiment jusqu’à l’asile le plus proche, asile dont tu ne tarderas d’ailleurs pas à t’évader pour mettre à nouveau la ville à feu et à sang, passer en justice pour dégradation de bien public, et faire marrer tout le tribunal en te foutant de la gueule des juges et désavouant publiquement ton avocat pour cause d’incompétence chronique et inculture caractérisée.

Bref, les jeux sont faits à notre insu, et nous ne sommes que des billes qui virevoltent sur la table de la roulette en espérant tomber sur le bon numéro. Beaucoup, aujourd’hui, semblent assez peu motivés pour participer à ce qui leur apparaît de plus en plus comme un jeu de dupe, une vaste supercherie. La reproduction ne fait plus recette, la contraception bat son plein. Chez les femmes principalement, qui sont en première ligne. Si elles éprouvaient jadis une certaine fierté à tripler de volume en neuf mois et se traîner comme des vaches à lait en priant le ciel que tout se passe bien au moment de l’accouchement, elles ne semblent plus aujourd’hui très réceptives à la démarche. Physiquement parlant, elles préfèrent le style Coca-Cola au style Orangina, la taille de guêpe à l’embonpoint gravidique. Dans le monde moderne, biberon, couche, siège-bébé, poussette et dépression ne font plus rêver personne, les gens ayant autre chose à foutre que passer leur temps à pouponner. La grossesse, vécue comme un parcours du combattant digne des Forces Spéciales dans l’enfer djihadiste du sanctuaire de Tofagala, est une chose beaucoup trop sérieuse pour être confiée à des gens dont ce n’est pas le métier. Seules des femmes surentraînées, au mental d’acier, peuvent se permettre d’affronter cette épreuve sans s’exposer aux ravages du stress post-traumatique. Donner la vie, au même titre que la prendre, exige des compétences qui ne sont pas à la portée du premier venu. Face aux difficultés croissantes de l’existence, la progéniture elle-même semble accessoire, un luxe dispensable réservé à une élite qui a les moyens de ne pas s’impliquer dans la vie familiale et surtout faire élever ses enfants par d’autres, des professionnels spécialement formés pour gérer les parcours scolaires, les conflits de cour de récréation, les chagrins d’amour et les crises d’adolescence.

Voilà pourquoi, pour en revenir à Greg, même s’il brûlait d’une quelconque flamme pour Bérénice, ce qui était fort probable à moins que Sam n’ait été abusé par ses sens (le fait est qu’il avait tendance à picoler, en plus de certaines mauvaises habitudes toxicologiques contractées au cours de ses années de baroud au sein de la Légion), j’avais tout lieu de penser que ce modeste incendie ne survivrait pas à la première averse.

Un flic en uniforme s’est pointé pour me faire part d’une nouvelle de la toute première importance, susceptible d’éclairer l’affaire en cours d’un jour radicalement nouveau. Il avait mis la main, en jetant négligemment un œil (réflexe professionnel) dans le coffre de la Mini, sur une importante quantité de ce qu’il est convenu d’appeler un produit stupéfiant. En l’occurrence, il s’agissait de toute évidence de cet ester méthylique de benzoylecgonine plus connu sous le nom de chlorhydrate de cocaïne, ou cocaïne tout court, alcaloïde tropanique très recherché pour ses propriétés psychoactives. À vue de groin, il y en avait une bonne dizaine de kilos, soit une valeur marchande avoisinant les sept cent mille euros. Il va sans dire que la présence de ce chargement pour le moins compromettant expliquait à lui seul la conduite (dangereuse) de la très belle et sulfureuse Repentance Whittingham, la femme de ménage la plus rapide du monde qui ne se contentait apparemment pas de faire succinctement le ménage dans un hôtel de luxe pour personnes de couleur, hôtel de luxe qui d’ailleurs, au vu des récents événements, n’était peut-être pas seulement, pour reprendre mot pour mot l’excellente définition du Larousse, le respectable «~établissement commercial mettant à la disposition d’une clientèle itinérante des chambres meublées pour un prix journalier~» (et excessivement élevé, ajouterai-je) qu’il prétendait.

Voilà, c’est tout pour l’instant.

Tout ce que je puis ajouter, à l’heure où j’écris ces lignes, c’est qu’on est toujours sans nouvelles de Titus Beaugendre.
%%% 
%%% Est-il besoin de préciser que je m’associe pleinement à sa femme et ses enfants, dont je partage l’affliction, pour dénoncer l’indolence, sinon l’inaction des pouvoirs publics, et exiger que toute la lumière soit faite au plus vite sur %%% cette affaire.
%%% 
%%% Naturellement, vous serez les premiers informés dès que j’en saurai un peu plus à ce sujet.
%%% 
%%% En attendant, je vous souhaite bon vent (dans tous les sens du terme, y compris bien sûr l’expulsion plus ou moins sonore et trébuchante de ces gaz intestinaux qu’il serait dangereux de conserver indéfiniment à l’intérieur de soi).
%%% 
%%% \textsc{PS} : Si je n’aime pas les fins, toujours décevantes et redondantes, inutiles, je n’aime pas non plus les titres, ennuyeux, racoleurs et commerciaux, profondément réducteurs et indigents.
%%% 
%%% Si vous lisez MADAME BOVARY sur une couverture, par exemple, vous viennent aussitôt en tête des adjectifs peu flatteurs comme bovin ou bavarois, lesquels, à moins d’être féru de produits laitiers ou de culture germanique, ne donnent %%% guère envie d’ouvrir le livre.
%%% 
%%% De la même façon, si vous lisez DRACULA, vous viennent aussitôt en tête des images à caractère sexuel pour le moins déplacées, même si, je vous l’accorde, les canines du vampire qui pénètrent dans la chair tendre d’une gorge féminine ne %%% sont pas totalement exemptes de sensualité.
%%% 
%%% J’avais, dans un premier temps, pensé appeler ce livre L’ÉCUME DES ABAT-JOUR, en hommage à Boris Vian\nf{Boris Vian (1920--1959), écrivain, poète, musicien et ingénieur français, figure majeure de Saint-Germain-des-Prés. Auteur %%% notamment de \textit{L’Écume des jours} (1947), il signa également sous le pseudonyme Vernon Sullivan des romans noirs sulfureux. Son titre \textit{L’Écume des jours} inspira ici le calembour \textit{L’Écume des abat-jour}. \source{fr.%%% wikipedia.org/wiki/Boris\_Vian}}, auteur que je connais de nom et de réputation, mais dont je confesse, à ma grande honte, n’avoir jamais lu le moindre livre. Je ne pouvais donc pas, ne serait-ce que par éthique littéraire, retenir ce %%% titre, assez rigolo par ailleurs.
%%% 
%%% J’ai ensuite pensé l’appeler LES GLAPISSEMENTS DE L’ENNUI, en référence aux CROASSEMENTS DE LA NUIT (titre français assez disgracieux de STILL LIFE WITH CROWS) de Douglas Preston\nf{Douglas Preston (né en 1956) et Lincoln Child (né en %%% 1957) sont deux romanciers américains qui collaborent depuis \textit{La Relique} (1995). Spécialisés dans les thrillers mêlant policier, fantastique et horreur, ils ont créé le personnage de l’agent du FBI Aloysius Pendergast. \textit%%% {Still Life with Crows} (2003), traduit en français sous le titre \textit{Croassements de la nuit}, est le quatrième roman de cette série. \source{fr.wikipedia.org/wiki/Douglas\_Preston\_et\_Lincoln\_Child}} \& Lincoln Child, deux %%% auteurs de romans policiers teintés de fantastique que j’apprécie particulièrement, même si leur prose à quatre mains est loin d’être un modèle d’intelligence et de créativité, tant sur la forme que le fond (il est bon, parfois, de %%% s’autoriser une certaine dose de médiocrité). J’ai rapidement laissé tomber l’idée, divertissante, certes, mais sans grand intérêt.
%%% 
%%% J’ai alors pensé, de façon plus impersonnelle, l’intituler LIVRE UN ou MON PREMIER ROMAN, mais la connotation enfantine m’est apparue par trop manifeste.
%%% 
%%% \enlargethispage{\baselineskip}%
%%% Si j’ai finalement, en désespoir de cause, résolu de tisser la métaphore alimentaire, c’est parce que manger (et boire) reste une de mes activités favorites. C’est aussi, accessoirement, la garantie de ne plus avoir à se creuser le chou %%% pour trouver un titre, la gastronomie internationale ne manquant pas de recettes aussi populaires que savoureuses. J’aurais aussi bien pu appeler ce bouquin CASHER BLUES, en référence subtile et odorante au CASHEL BLUE, ce fromage %%% irlandais à pâte persillée, ou JAMBALAYA, best-seller de la cuisine cajun, TOUOP-TOUOP KELONG, recette camerounaise à base de banane plantain et poisson fumé, ou encore MOROS Y CRISTIANOS, du nom de ce délicieux plat cubain à base de %%% riz blanc et haricots noirs, sans que personne n’y trouve rien à redire, ni trahir aucunement la nature conviviale et épicée du présent ouvrage.
%%% 
%%% 
%%% 


% ═══ PAGE BLANCHE FINALE (p. 466, verso) ══════════════════════════════════════
\clearpage
\hbox{}\thispagestyle{empty}

\end{document}

\end{tabularx}
}

\end{document}
