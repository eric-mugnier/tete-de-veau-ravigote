\documentclass[tikz, border=0pt]{standalone}

% ─── PACKAGES ─────────────────────────────────────────────────────────────────
\usepackage[utf8]{inputenc}
\usepackage[T1]{fontenc}
\usepackage{ebgaramond}
\usepackage[french]{babel}
\usepackage{xcolor}
\usepackage{tikz}
\usetikzlibrary{calc, shapes.geometric}
\usepackage{microtype}
\usepackage{graphicx}

% ─── COULEURS MINUIT ──────────────────────────────────────────────────────────
\definecolor{minuitblue}{RGB}{28,72,172}   % bleu Minuit
\definecolor{blanc}{RGB}{255,255,255}

% ─── DOCUMENT ─────────────────────────────────────────────────────────────────
\begin{document}

% ─── CANVAS TIKZ (unité = 1mm, origine bas-gauche) ───────────────────────────
% Zones :
%   Fond perdu gauche   :  0 –   3 mm
%   4e de couverture    :  3 – 143 mm  (140 mm)
%   Dos                 : 143 – 161 mm ( 18 mm)
%   1ère de couverture  : 161 – 301 mm (140 mm)
%   Fond perdu droit    : 301 – 304 mm
%   Fond perdu bas/haut :  0 –   3 mm / 208 – 211 mm

\begin{tikzpicture}[x=1mm, y=1mm, font=\normalfont]

  % ── FOND BLANC GLOBAL ──────────────────────────────────────────────────────
  \fill[blanc] (0,0) rectangle (304,211);

  % ─────────────────────────────────────────────────────────────────────────
  % ══ 1ÈRE DE COUVERTURE (161 – 301 mm) ═══════════════════════════════════
  % ─────────────────────────────────────────────────────────────────────────

  % ── Cadre rectangle unique (style Minuit) ──────────────────────────────
  \draw[black, line width=0.6pt]
    (166, 8) rectangle (298, 203);

  % ── Auteur (noir, petite police, centré, haut) ─────────────────────────
  \node[anchor=north] at (231, 196)
    {\fontsize{9.5}{11}\selectfont ÉRIC MUGNIER};

  % ── Titre (bleu, gras, allcaps, grand) ─────────────────────────────────
  \node[
    anchor=north,
    text width=118mm,
    align=center,
    text=minuitblue
  ] at (231, 181)
  {\fontsize{22}{26}\selectfont\bfseries
    TÊTE DE VEAU RAVIGOTE};

  % ── Mention "roman" ────────────────────────────────────────────────────
  \node[anchor=north] at (231, 152)
    {\fontsize{9}{11}\selectfont\itshape roman};

  % ── Logo CT (entre "roman" et la photo) ────────────────────────────────
  \node[anchor=center] at (231, 122)
    {\includegraphics[width=16mm]{images/ct_watermark.png}};

  % ── Bandeau photo (pleine largeur, jusqu'en bas) ──────────────────────
  \begin{scope}
    \clip (161, 0) rectangle (304, 105);
    \node[anchor=north west, inner sep=0] at (161, 105)
      {\includegraphics[width=143mm]{images/Mugnier2.jpg}};
  \end{scope}



  % ─────────────────────────────────────────────────────────────────────────
  % ══ DOS (143 – 161 mm) ═══════════════════════════════════════════════════
  % ─────────────────────────────────────────────────────────────────────────

  % Filets latéraux du dos
  \draw[black!25, line width=0.2pt] (143,0) -- (143,211);
  \draw[black!25, line width=0.2pt] (161,0) -- (161,211);

  % Titre vertical (bleu, de bas en haut)
  \node[rotate=90, anchor=center, text=minuitblue] at (152, 112)
    {\fontsize{8.5}{10}\selectfont\bfseries TÊTE DE VEAU RAVIGOTE};

  % Auteur en haut
  \node[anchor=north] at (152, 199)
    {\fontsize{6.5}{8}\selectfont ÉRIC MUGNIER};

  % ── Logo CT (dos, bas, discret) ───────────────────────────────────────────
  \node[anchor=south] at (152, 6)
    {\includegraphics[width=6mm]{images/ct_watermark.png}};


  % ─────────────────────────────────────────────────────────────────────────
  % ══ 4ÈME DE COUVERTURE (3 – 143 mm) ═════════════════════════════════════
  % ─────────────────────────────────────────────────────────────────────────

  % Cadre rectangle (comme la 1ère de couverture)
  \draw[black, line width=0.6pt]
    (8, 8) rectangle (138, 203);

  % ── Résumé
  \node[
    anchor=north west,
    text width=112mm,
    align=justify,
    inner sep=0pt
  ] at (14, 194)
  {%
    \fontsize{9}{13.5}\selectfont
    Quand le père Vidal disparaît dans des circonstances troubles et que des
    colis macabres commencent à circuler dans la ville, le commandant Beauvais
    se lance dans une enquête labyrinthique. Mais le polar n'est ici qu'un fil
    rouge dérisoire : ce qui compte, ce sont les digressions – sur les chats,
    les dinosaures, l'Église, les pavillons de banlieue, la décadence de
    l'humanité.

    \medskip
    Roman-fleuve picaresque dans la lignée de Cervantes et Rabelais,
    \textit{Tête de veau ravigote} mêle la rage sadienne au pessimisme
    schopenhauerien. Éric Mugnier y déploie une prose incandescente, truculente
    et désespérée, qui rappelle autant \textit{Là-bas} de Huysmans que les
    pamphlets du XVIIIe~siècle. Inachevable, et délibérément inachevé, ce
    premier roman ignore la résolution et préfère la dérive : une logorrhée
    vengeresse et jubilatoire.%
  };

  % Filet séparateur
  \draw[black!35, line width=0.3pt] (14, 41) -- (132, 41);

  % ── Bio
  \node[
    anchor=north west,
    text width=112mm,
    align=justify,
    inner sep=0pt
  ] at (14, 38)
  {%
    \fontsize{8}{11}\selectfont
    Éric Mugnier est né en 1960. Il vit à Chaumont-en-Bassigny. Peintre,
    compositeur et parolier, il se consacre depuis quarante ans à la
    création. \textit{Tête de veau ravigote} est son premier roman.%
  };

\end{tikzpicture}

\end{document}
